\documentclass[a4j]{jsarticle}
\usepackage{amsmath,amsthm,amssymb,bm,color,mathrsfs,url}
\usepackage{epic,eepic,here}
\usepackage[dvipdfm]{graphicx}
\usepackage[hypertex]{hyperref}
\title{
Dynamic Structure Factor
}
\author{Masaru Okada (岡田 大)}
\date{\today}
\begin{document}
\maketitle

These are notes on the dynamic structure factor. The discussion is set in the context of neutron scattering.

\subsection*{On the Interaction Between Neutrons and the Lattice}

Consider a neutron with momentum $\vec{p}$ being scattered by a crystal, emerging with momentum $\vec{p'}$.
Before the scattering, the lattice is assumed to be in an eigenstate of the crystal Hamiltonian with energy $E_{\rm i}$, and after scattering, it is in an eigenstate with energy $E_{\rm f}$.
The states of the combined neutron-lattice system are, respectively:

\hspace{5mm}
Wave function and eigenenergy before scattering:
\begin{eqnarray}
	\Psi_{\rm i}
	&=&
	\psi_{\vec{p}}(\vec{r}) \Phi_{\rm i}
	\\[3mm]
	\varepsilon_{\rm i}
	&=&
	E_{\rm i} + \dfrac{p^{2}}{2 M_{n}}
\end{eqnarray}

\hspace{5mm}
Wave function and eigenenergy after scattering:
\begin{eqnarray}
	\Psi_{\rm f}
	&=&
	\psi_{\vec{p}'}(\vec{r}) \Phi_{\rm f}
	\\[3mm]
	\varepsilon_{\rm f}
	&=&
	E_{\rm f} + \dfrac{p'^{2}}{2 M_{n}}
\end{eqnarray}
Here, $M_{n}$ is the neutron mass, and
$V$ is the volume of the system, used as a normalization constant.
The function $\psi_{\vec{p}}(\vec{r})$ is a plane wave
$
	\psi_{\vec{p}}(\vec{r}) = \dfrac{1}{\sqrt{V}} e^{i \vec{p} \cdot \vec{r}}
$.
The energy gain and momentum change of the lattice are:
\begin{eqnarray}
	\hbar \omega
	&=&
	\dfrac{p'^{2}}{2 M_{n}}
	-
	\dfrac{p^{2}}{2 M_{n}}
	\\[2mm]
	\hbar \vec{q}
	&=&
	\vec{p'}
	-
	\vec{p}
\end{eqnarray}
Letting $\vec{R}$ be a lattice point and $\vec{r}(\vec{R})$ be the position of the ion belonging to $\vec{R}$, which fluctuates due to heat etc., the neutron-lattice interaction is:
\begin{eqnarray}
	U(\vec{r})
	&=&
	\sum_{\vec{R}}
	u \big[ \vec{r} - \vec{r}(\vec{R}) \big]
	\ = \
	\dfrac{1}{V}
	\sum_{\vec{k},\vec{R}}
	u_{\vec{k}} e^{i \vec{k} \cdot [ \vec{r} - \vec{r}(\vec{R}) ]}
\end{eqnarray}

The range of the interaction $u \big[ \vec{r} - \vec{r}(\vec{R}) \big]$ is
at most the size of the nucleus, i.e., about $10^{-13}$ cm.
Its Fourier component $u_{\vec{k}}$ is thought to vary on the scale of $k \sim 10^{13}$ cm$^{-1}$.
The important scale for the phonon spectrum is a wave vector of about $10^{8}$ cm$^{-1}$,
and compared to this, the variation of $k$ is sufficiently slow (by a factor of $10^5$).
Therefore, the Fourier component of the interaction $u_{\vec{k}}$ can be regarded as independent of the wave vector,
and we will write this constant as $u_{0}$.

Suppose the total scattering cross-section for a single lattice point is given by $4 \pi a^{2}$.
Using the averaged length $a$ that characterizes this scattering,
\begin{eqnarray}
	u_{0}
	&=&
	4 \pi a^{2} \times \dfrac{\hbar^{2}}{2M_{n}} \dfrac{1}{a}
\end{eqnarray}
we have,
\begin{eqnarray}
	U(\vec{r})
	&=&
	\dfrac{ 2 \pi \hbar^{2} a }{ M_{n} V }
	\sum_{ \vec{k} , \vec{R} }
	e^{i \vec{k} \cdot [ \vec{r} - \vec{r}(\vec{R}) ] }
\end{eqnarray}
Note that performing the $\vec{k}$ integration yields,
\begin{eqnarray}
	U(\vec{r})
	&=&
	\dfrac{ 2 \pi \hbar^{2} a }{ M_{n} }
	\sum_{ \vec{R} }
	\int \dfrac{d^{3} \vec{k}}{(2 \pi)^{3}}
	e^{i \vec{k} \cdot [ \vec{r} - \vec{r}(\vec{R}) ] }
	\nonumber \\[3mm] &=&
	\dfrac{ 2 \pi \hbar^{2} a }{ M_{n} }
	\sum_{ \vec{R} }
	\delta [ \vec{r} - \vec{r}(\vec{R}) ]
\end{eqnarray}
This naturally shows that the assumptions we have made are,
in fact, equivalent to assuming that the interaction is a point-contact type that acts only on the lattice points.

\subsection*{Fermi's Golden Rule and the Scattering Cross-Section}

The probability $P$ per unit time of scattering from $\vec{p}$ to $\vec{p'}$ can be calculated
to the lowest order of perturbation theory using the formula known as Fermi's Golden Rule.
Writing the inner product of functions $f=f(x)$ and $g=g(x)$ as
\begin{eqnarray}
	\int dx \big[ f(x) \big]^{*} g(x)
	&=&
	\big( f , g \big)
\end{eqnarray}

we can calculate $P$ as:

\begin{eqnarray}
	P
	&=&
	\dfrac{2 \pi}{\hbar}
	\sum_{\rm f}
	\delta( \varepsilon_{\rm f} - \varepsilon_{\rm i} )
	\Big| \big( \Psi_{\rm f} , U \Psi_{\rm i} \big) \Big|^{2}
	\nonumber \\[2mm] &=&
	\dfrac{2 \pi}{\hbar}
	\sum_{\rm f}
	\delta( E_{\rm f} - E_{\rm i} + \hbar \omega )
	\bigg|
	\dfrac{1}{V} \int d^{3} \vec{r} \
	e^{i \vec{q} \cdot \vec{r}}
	\big( \Phi_{\rm f} , U(\vec{r}) \Phi_{\rm i} \big)
	\bigg|^{2}
	\nonumber \\[2mm] &=&
	\dfrac{2 \pi}{\hbar}
	\dfrac{1}{V^{2}}
	\Big( \dfrac{2 \pi \hbar^{2} a}{M_{n} V} \Big)^{2}
	\sum_{\rm f}
	\delta( E_{\rm f} - E_{\rm i} + \hbar \omega )
	\bigg|
	\sum_{ \vec{k} , \vec{R} }
	\int d^{3} \vec{r} \
	e^{i (\vec{k} + \vec{q} ) \cdot \vec{r}}
	\big( \Phi_{\rm f} ,
	e^{ - i \vec{k} \cdot \vec{r}(\vec{R}) }
	\Phi_{\rm i} \big)
	\bigg|^{2}
	\nonumber \\[2mm] &=&
	\dfrac{ ( 2 \pi \hbar )^{3} }{ V^{2} ( M_{n} V )^{2} } a^{2}
	\sum_{\rm f}
	\delta( E_{\rm f} - E_{\rm i} + \hbar \omega )
	\bigg|
	\sum_{ \vec{k},\vec{R} }
	( 2 \pi \hbar )^{3}
	\delta (\vec{k} + \vec{q} )
	\big( \Phi_{\rm f} ,
	e^{ - i \vec{k} \cdot \vec{r}(\vec{R}) }
	\Phi_{\rm i} \big)
	\bigg|^{2}
	\nonumber \\[2mm] &=&
	\dfrac{ ( 2 \pi \hbar )^{3} }{ V^{2} ( M_{n} V )^{2} } a^{2}
	\sum_{\rm f}
	\delta( E_{\rm f} - E_{\rm i} + \hbar \omega )
	\bigg|
	\sum_{ \vec{R} }
	( 2 \pi \hbar )^{3}
	\dfrac{V}{( 2 \pi \hbar )^{3}}
	\int d^{3} \vec{k} \
	\delta (\vec{k} + \vec{q} )
	\big( \Phi_{\rm f} ,
	e^{ - i \vec{k} \cdot \vec{r}(\vec{R}) }
	\Phi_{\rm i} \big)
	\bigg|^{2}
	\nonumber \\[2mm] &=&
	\dfrac{ ( 2 \pi \hbar )^{3} }{ ( M_{n} V )^{2} }
	a^{2}
	\sum_{\rm f}
	\delta( E_{\rm f} - E_{\rm i} + \hbar \omega )
	\bigg|
	\sum_{\vec{R}}
	\big( \Phi_{\rm f} , e^{i \vec{q} \cdot \vec{r}(\vec{R})} \Phi_{\rm i} \big)
	\bigg|^{2}
\end{eqnarray}

The scattering probability $P$ is related to the measurable quantity, the differential scattering cross-section
$\dfrac{d^{3} \sigma}{d^{2} \Omega d E}$.
The incident neutron flux is

\begin{eqnarray}
	j
	&=&
	\dfrac{p}{M_{n}} \big| \psi_{\vec{p}} \big|^{2}
	\ = \
	\dfrac{1}{V}
	\dfrac{p}{M_{n}}
\end{eqnarray}

From the conservation of flux,

'(Integral of differential scattering cross-section over all solid angles and energies) = (Sum of probability $P$ over all states)'

Therefore, the following holds:
\begin{eqnarray}
	\int
	j
	\dfrac{ d^{3} \sigma }{ d^{2} \Omega d E }
	d^{2} \Omega d E
	&=&
	\int P \dfrac{V}{ ( 2 \pi \hbar)^{3} }
	\ d^{3} \vec{p'}
\end{eqnarray}
The left-hand side is,
\begin{eqnarray}
	\int
	j
	\dfrac{ d^{3} \sigma }{ d^{2} \Omega d E }
	\ d^{2} \Omega d E
	&=&
	\int
	\dfrac{1}{V}
	\dfrac{p}{M_{n}}
	\dfrac{ d^{3} \sigma }{ d^{2} \Omega d E }
	\ d^{2} \Omega d E
\end{eqnarray}
On the other hand, the right-hand side is
\begin{eqnarray}
	\int P \dfrac{V}{ ( 2 \pi \hbar)^{3} }
	\ d^{3} \vec{p'}
	&=&
	\int P \dfrac{V}{ ( 2 \pi \hbar)^{3} }
	p'^{2}
	\ d p' d^{2} \Omega
	\nonumber \\[3mm] &=&
	\int P \dfrac{V}{ ( 2 \pi \hbar)^{3} }
	M_{n} p'
	\ d E d^{2} \Omega
\end{eqnarray}
Comparing both sides,
\begin{eqnarray}
	\dfrac{1}{V}
	\dfrac{p}{M_{n}}
	\dfrac{ d^{3} \sigma }{ d^{2} \Omega d E }
	&=&
	P \dfrac{V}{ ( 2 \pi \hbar)^{3} }
	M_{n} p'
\end{eqnarray}
That is,
\begin{eqnarray}
	\dfrac{ d^{3} \sigma }{ d^{2} \Omega d E }
	&=&
	\dfrac{p'}{p}
	\dfrac{(M_{n} V)^{2}}{ ( 2 \pi \hbar)^{3} }
	P
	\nonumber \\[2mm] &=&
	\dfrac{p'}{p}
	\dfrac{(M_{n} V)^{2}}{ ( 2 \pi \hbar)^{3} }
	\cdot
	\dfrac{ ( 2 \pi \hbar )^{3} }{ ( M_{n} V )^{2} }
	a^{2}
	\sum_{\rm f}
	\delta( E_{\rm f} - E_{\rm i} + \hbar \omega )
	\bigg|
	\sum_{\vec{R}}
	\big( \Phi_{\rm f} , e^{i \vec{q} \cdot \vec{r}(\vec{R})} \Phi_{\rm i} \big)
	\bigg|^{2}
	\nonumber \\[2mm] &=&
	\dfrac{p'}{p}
	\dfrac{ N a^{2} }{ \hbar }
	S_{\rm i}(\vec{q},\omega)
\end{eqnarray}
This $S_{\rm i}(\vec{q},\omega)$ is defined as follows.
\begin{eqnarray}
	S_{\rm i}(\vec{q},\omega)
	&=&
	\dfrac{1}{N}
	\sum_{\rm f}
	\delta \Big( \dfrac{E_{\rm f} - E_{\rm i}}{\hbar} + \omega \Big)
	\bigg|
	\sum_{\vec{R}}
	\big( \Phi_{\rm f} , e^{i \vec{q} \cdot \vec{r}(\vec{R})} \Phi_{\rm i} \big)
	\bigg|^{2}
\end{eqnarray}
Here, $N$ is the number of lattice sites in the system.
Now, the following identity holds for the Heisenberg operator $A(t)=e^{i H t/ \hbar} A e^{-i H t/ \hbar}$:

\begin{eqnarray}
	\big( \Phi_{\rm f} , A(t) \Phi_{\rm i} \big)
	&=&
	\big( \Phi_{\rm f} , e^{i H t/ \hbar} A e^{-i H t/ \hbar} \Phi_{\rm i} \big)
	\nonumber \\[2mm] &=&
	\big( \Phi_{\rm f} , e^{i E_{\rm f} t/ \hbar} A e^{-i E_{\rm i} t/ \hbar} \Phi_{\rm i} \big)
	\nonumber \\[2mm] &=&
	e^{i ( E_{\rm f} - E_{\rm i} ) t/ \hbar} \big( \Phi_{\rm f} , A \Phi_{\rm i} \big)
\end{eqnarray}

By expanding the delta function (Fourier transform) and using this,


\begin{eqnarray}
	S_{\rm i}(\vec{q},\omega)
	&=&
	\dfrac{1}{N}
	\int \dfrac{ dt }{ 2 \pi }
	e^{ i \omega t }
	\sum_{\rm f}
	e^{ i ( E_{\rm f} - E_{\rm i} ) t / \hbar}
	\bigg|
	\sum_{\vec{R}}
	\big( \Phi_{\rm f} , e^{i \vec{q} \cdot \vec{r}(\vec{R})} \Phi_{\rm i} \big)
	\bigg|^{2}
	\nonumber \\[2mm] &=&
	\dfrac{1}{N}
	\int \dfrac{ dt }{ 2 \pi }
	e^{ i \omega t }
	\sum_{\rm f}
	e^{ i ( E_{\rm f} - E_{\rm i} ) t / \hbar}
	\sum_{\vec{R}}
	\big( \Phi_{\rm f} , e^{i \vec{q} \cdot \vec{r}(\vec{R})} \Phi_{\rm i} \big)
	\sum_{\vec{R}'}
	\big( \Phi_{\rm i} , e^{ - i \vec{q} \cdot \vec{r}(\vec{R}')} \Phi_{\rm f} \big)
	\nonumber \\[2mm] &=&
	\dfrac{1}{N}
	\int \dfrac{ dt }{ 2 \pi }
	e^{ i \omega t }
	\sum_{\rm f}
	\sum_{\vec{R},\vec{R}'}
	\big( \Phi_{\rm i} , e^{ - i \vec{q} \cdot \vec{r}(\vec{R}')} \Phi_{\rm f} \big)
	\big( \Phi_{\rm f} , e^{i \vec{q} \cdot \vec{r}(\vec{R},t)} \Phi_{\rm i} \big)
\end{eqnarray}
Furthermore, since $\Phi_{\rm f}$ forms a complete set, for operators $A$ and $B$,

\begin{eqnarray}
	\sum_{\rm f}
	\big( \Phi_{\rm i} , A \Phi_{\rm f} \big)
	\big( \Phi_{\rm f} , B \Phi_{\rm i} \big)
	&=&
	\big( \Phi_{\rm i} , AB \Phi_{\rm i} \big)
\end{eqnarray}
also holds.
Letting $\delta \vec{R} \ (= \vec{r}(\vec{R}) - \vec{R} )$ be the displacement of the ion from the lattice point due to thermal motion,
\begin{eqnarray}
	S_{\rm i}(\vec{q},\omega)
	&=&
	\dfrac{1}{N}
	\int \dfrac{ dt }{ 2 \pi }
	e^{ i \omega t }
	\sum_{\rm f}
	\sum_{\vec{R},\vec{R}'}
	\big( \Phi_{\rm i} , e^{ - i \vec{q} \cdot [ \delta \vec{R}' + \vec{R}' ]  } \Phi_{\rm f} \big)
	\big( \Phi_{\rm f} , e^{i \vec{q} \cdot [ \delta \vec{R}(t) + \vec{R} ] } \Phi_{\rm i} \big)
	\nonumber \\[2mm] &=&
	\dfrac{1}{N}
	\int \dfrac{ dt }{ 2 \pi }
	e^{ i \omega t }
	\sum_{\rm f}
	\sum_{\vec{R},\vec{R}'}
	e^{ i \vec{q} \cdot (\vec{R} - \vec{R}')}
	\big( \Phi_{\rm i} , e^{ - i \vec{q} \cdot \delta \vec{R}' } \Phi_{\rm f} \big)
	\big( \Phi_{\rm f} , e^{i \vec{q} \cdot \delta \vec{R}(t) } \Phi_{\rm i} \big)
	\nonumber \\[2mm] &=&
	\dfrac{1}{N}
	\int \dfrac{ dt }{ 2 \pi }
	e^{ i \omega t }
	\sum_{\vec{R},\vec{R}'}
	e^{ i \vec{q} \cdot (\vec{R} - \vec{R}')}
	\big( \Phi_{\rm i} , e^{ - i \vec{q} \cdot \delta \vec{R}' } e^{i \vec{q} \cdot \delta \vec{R}(t) } \Phi_{\rm i} \big)
\end{eqnarray}

In general, the crystal in the initial state is in thermal equilibrium.
To find the scattering cross-section, we should take a statistical average over all initial states.
If we write the statistical average of an operator $A$ as

\begin{eqnarray}
	\big\langle A \big\rangle
	&=&
	\sum_{\rm i}
	\dfrac{ e^{ - E_{\rm i} / k_{\rm B} T } \big( \Phi_{\rm i} , A \Phi_{\rm i} \big) }{e^{ - E_{\rm i} / k_{\rm B} T }}
\end{eqnarray}
the statistical average of $S_{\rm i}(\vec{q},\omega)$ is:

\begin{eqnarray}
	S(\vec{q},\omega)
	&=&
	\dfrac{1}{N}
	\int \dfrac{ dt }{ 2 \pi }
	e^{ i \omega t }
	\sum_{\vec{R},\vec{R}'}
	e^{ i \vec{q} \cdot (\vec{R} - \vec{R}')}
	\big\langle e^{ - i \vec{q} \cdot \delta \vec{R}' } e^{i \vec{q} \cdot \delta \vec{R}(t) } \big\rangle
\end{eqnarray}
This quantity is called the \textbf{dynamic structure factor},
and it is related to the differential scattering cross-section as follows.

\begin{eqnarray}
	\dfrac{ d^{3} \sigma }{ d^{2} \Omega d E }
	&=&
	\dfrac{p'}{p}
	\dfrac{ N a^{2} }{ \hbar }
	S(\vec{q},\omega)
\end{eqnarray}
The dynamic structure factor $S(\vec{q},\omega)$ depends only on the structure of the scatterer.
\subsection*{About the Dynamic Structure Factor}

Let's consider the term inside $S(\vec{q},\omega)$,
$\big\langle e^{ - i \vec{q} \cdot \delta \vec{R}' } e^{i \vec{q} \cdot \delta \vec{R}(t) } \big\rangle$.
For linear operators $A$ and $B$, the relation

\begin{eqnarray}
	\big\langle e^{ A } e^{ B } \big\rangle
	&=&
	{\rm exp} \Big( \dfrac{1}{2} \big\langle A^{2} + 2 AB + B^{2} \big\rangle \Big)
\end{eqnarray}
holds (this is the Gaussian approximation), so

\begin{eqnarray}
	\Big\langle e^{ - i \vec{q} \cdot \delta \vec{R}' } e^{i \vec{q} \cdot \delta \vec{R}(t) } \Big\rangle
	&=&
	{\rm exp} \Big\langle - \dfrac{1}{2} \Big[ \vec{q} \cdot \delta \vec{R}' \Big]^{2} \Big\rangle
	\ \ \!
	{\rm exp} \Big\langle \Big[ \vec{q} \cdot \delta \vec{R}' \Big] \Big[ \vec{q} \cdot \delta \vec{R}(t) \Big] \Big\rangle
	\ \ \!
	{\rm exp} \Big\langle - \dfrac{1}{2} \Big[ \vec{q} \cdot \delta \vec{R}(t) \Big]^{2} \Big\rangle
	\nonumber \\
\end{eqnarray}
Since observable physical quantities depend only on relative coordinates and relative time,

\begin{eqnarray}
	{\rm exp} \Big\langle \Big[ \vec{q} \cdot \delta \vec{R}' \Big]^{2} \Big\rangle
	\ = \
	{\rm exp} \Big\langle \Big[ \vec{q} \cdot \delta \vec{R}(t) \Big]^{2} \Big\rangle
	\ = \
	2W
	\ = \
	{\rm const.}
\end{eqnarray}
(This $e^{-2W}$ is the Debye-Waller factor).
Furthermore, by shifting the position coordinates,
letting $ \delta \vec{R}' \to \delta \vec{R}_{0}$,
and rewriting
$\delta \vec{R} \to \delta \vec{R} + \delta \vec{R}_{0} - \delta \vec{R}'$
as
$\delta \vec{\tilde{R}}$,
\begin{eqnarray}
	{\rm exp} \Big\langle \Big[ \vec{q} \cdot \delta \vec{R}' \Big] \Big[ \vec{q} \cdot \delta \vec{R}(t) \Big] \Big\rangle
	&=&
	{\rm exp} \Big\langle \Big[ \vec{q} \cdot \delta \vec{R}_{0} \Big] \Big[ \vec{q} \cdot \delta \vec{\tilde{R}}(t) \Big] \Big\rangle
\end{eqnarray}
From these steps,

\begin{eqnarray}
	S(\vec{q},\omega)
	&=&
	e^{-2W}
	\int \dfrac{ dt }{ 2 \pi }
	e^{ i \omega t }
	\sum_{ \vec{\tilde{R}} }
	e^{ i \vec{q} \cdot \vec{\tilde{R}} }
		{\rm exp}
	\Big\langle \big[ \vec{q} \cdot \delta \vec{R}_{0} \big] \big[ \vec{q} \cdot \delta \vec{\tilde{R}}(t) \big] \Big\rangle
	\nonumber \\[2mm] &=&
	e^{-2W}
	\int \dfrac{ dt }{ 2 \pi }
	e^{ i \omega t }
	\sum_{ \vec{R} }
	e^{ i \vec{q} \cdot \vec{R} }
		{\rm exp}
	\Big\langle \big[ \vec{q} \cdot \delta \vec{R}_{0} \big] \big[ \vec{q} \cdot \delta \vec{R}(t) \big] \Big\rangle
\end{eqnarray}
(The transformation to the second line is just relabeling the summation variable from $\vec{\tilde{R}}$ to $\vec{R}$ for clarity.)

\begin{eqnarray}
	{\rm exp}
	\Big\langle \big[ \vec{q} \cdot \delta \vec{R}_{0} \big] \big[ \vec{q} \cdot \delta \vec{R}(t) \big] \Big\rangle
	&=&
	\sum_{m=0}^{\infty}
	\dfrac{1}{m!}
	\Big( \Big\langle \big[ \vec{q} \cdot \delta \vec{R}_{0} \big] \big[ \vec{q} \cdot \delta \vec{R}(t) \big] \Big\rangle \Big)^{m}
\end{eqnarray}
can be expanded. The $m$-th term here represents the contribution of $m$ phonons; for example, $m=0$ is zero-phonon scattering (elastic scattering), and $m=1$ is called one-phonon scattering.

\end{document}
\documentclass[uplatex,a4j,12pt,dvipdfmx]{article}
\usepackage[english]{babel}
\usepackage[letterpaper,top=2cm,bottom=2cm,left=3cm,right=3cm,marginparwidth=1.75cm]{geometry}
\usepackage{amsmath, amssymb}
\usepackage{graphicx}
\usepackage[colorlinks=true, allcolors=blue]{hyperref}
\usepackage{tikz-cd}


\title{
\hspace{2.8em} Bogoliubov Transformation \newline as a Natural Transformation
}

\author{
Masaru Okada
}

\begin{document}
\maketitle

\begin{abstract}
    The Bogoliubov transformation is a linear transformation that preserves specific algebraic structures in quantum field theory. It possesses a deep mathematical structure that goes beyond a simple matrix diagonalization operation. This article explains how this transformation is connected to the concepts of natural transformations in category theory and Lie algebra cohomology, from a mathematically rigorous perspective.
\end{abstract}


\section{Lie Algebras and C*-algebras}

To gain a deeper understanding of the algebraic structure of the Bogoliubov transformation, it is essential to know the relationship between two important mathematical objects: Lie algebras and C*-algebras. These concepts are deeply linked to the mathematical description of physical quantities in quantum mechanics.

\subsection{Definition of a Lie Algebra}

A Lie algebra is a vector space with an operation called the commutator product. This is a mathematical abstraction of the commutation relations of operators in quantum mechanics.

\begin{itemize}
    \item \textbf{Definition of a Lie Algebra}:
    A vector space $\mathfrak{g}$ over a field $k$ is called a Lie algebra $(\mathfrak{g}, [\cdot, \cdot])$ if it is equipped with a bilinear map $[\cdot, \cdot]: \mathfrak{g} \times \mathfrak{g} \to \mathfrak{g}$ that satisfies the following properties:
    \begin{enumerate}
        \item \textbf{Alternating Property}: For any $X \in \mathfrak{g}$, $[X, X] = 0$.
        \item \textbf{Jacobi Identity}: For any $X, Y, Z \in \mathfrak{g}$,
        \[
            [X, [Y, Z]] + [Y, [Z, X]] + [Z, [X, Y]] = 0
        \]
        holds.
    \end{enumerate}
    \item \textbf{Connection to Physics}:
    In quantum mechanics, physical quantities (observables) are represented by linear operators, and their commutation relation $[A, B] = AB - BA$ determines their physical properties (e.g., the uncertainty principle). This commutation relation satisfies the axioms of a Lie bracket. Therefore, the set of physical quantities forms a Lie algebra with the commutator product as the Lie bracket.
\end{itemize}

\subsection{Definition of a C*-algebra}

A C*-algebra is an algebra over complex numbers that combines analytical structures (a norm and completeness) with algebraic structures (multiplication, addition, and an adjoint operation).

\begin{itemize}
    \item \textbf{Definition of a C*-algebra}:
    A complex algebra $A$ is called a C*-algebra if it satisfies the following properties:
    \begin{enumerate}
        \item $A$ is a complex vector space, and multiplication and addition satisfy the associative and distributive laws.
        \item A norm $\|\cdot\|$ is defined on $A$, and $A$ is a Banach space (a complete normed space) with respect to this norm.
        \item An involution $*: A \to A$ is defined, satisfying the following properties:
        \begin{itemize}
            \item $(x+y)^* = x^*+y^*$
            \item $(\alpha x)^* = \bar{\alpha} x^*$
            \item $(xy)^* = y^*x^*$
            \item $(x^*)^* = x$
        \end{itemize}
        \item The C*-identity: For any $x \in A$,
        \[
            \|x^*x\| = \|x\|^2
        \]
        holds.
    \end{enumerate}
    \item \textbf{Connection to Physics}:
    In quantum mechanics, the set of all bounded linear operators on a Hilbert space forms a C*-algebra. This C*-algebra provides an algebraic framework that encompasses all observable quantities of a quantum system. Self-adjoint elements ($A=A^*$) correspond to physical observables.
\end{itemize}

\subsection{Relationship between Lie Algebras and C*-algebras}

Lie algebras and C*-algebras play complementary roles in the description of quantum mechanics.

\begin{itemize}
    \item \textbf{Lie Algebra}: Captures the \textbf{local properties} of continuous symmetries (e.g., rotations, translations) of a physical system through a linear structure called commutation relations.
    \item \textbf{C*-algebra}: Describes the entire set of observable quantities of a physical system in a \textbf{global framework} with analytical and algebraic structures.
\end{itemize}
The transition from a Lie algebra to a C*-algebra occurs through the representation theory of groups. By differentiating a representation of a Lie group (a concrete realization of the physical system's symmetry), we obtain a representation of the corresponding Lie algebra. The C*-algebra is then generated from the set of operators spanned by this Lie algebra representation. This relationship shows that elements of a Lie algebra, such as the Hamiltonian and momentum operators in quantum mechanics, are embedded in a broader algebraic structure, the C*-algebra.


\section{Bogoliubov Transformation as a Natural Transformation}

The Bogoliubov transformation is a crucial operation in BCS theory of superconductivity and Bose-Einstein condensation, which defines new quasiparticles from the original particle creation and annihilation operators. The "naturalness" of this transformation can be rigorously formulated as a \textbf{natural transformation} in category theory.

\subsection{Setting up the Functors}

First, we define an appropriate category $\mathcal{C}$ and functors $F, G: \mathcal{C} \to \mathcal{D}$ on it to describe the Bogoliubov transformation.

\begin{itemize}
    \item \textbf{Category $\mathcal{C}$}: The objects are the momenta $k \in \mathbb{R}^3$ that describe the state of the physical system, and the morphisms are continuous maps $f: k \to k'$ between momenta.
    \item \textbf{Functor $F$ (Original Particle System)}: This maps each momentum $k \in \mathcal{C}$ to the complex vector space $V_{k} = \mathrm{span}_\mathbb{C}\{a_{k}, a_{k}^{\dagger}\}$ spanned by the pair of creation and annihilation operators $\{a_{k}, a_{k}^{\dagger}\}$ corresponding to that momentum. A morphism $f: k \to k'$ induces a linear transformation $F(f): V_{k} \to V_{k'}$.
    \item \textbf{Functor $G$ (Quasiparticle System)}: This maps each momentum $k \in \mathcal{C}$ to the vector space $W_{k} = \mathrm{span}_\mathbb{C}\{\alpha_{k}, \alpha_{k}^{\dagger}\}$ spanned by the quasiparticle operator pair $\{\alpha_{k}, \alpha_{k}^{\dagger}\}$ defined by the Bogoliubov transformation. A morphism $f: k \to k'$ induces a linear transformation $G(f): W_{k} \to W_{k'}$.
\end{itemize}

\subsection{Formulation as a Natural Transformation}

The Bogoliubov transformation is defined as a \textbf{natural transformation} $\text{Bog}: F \Rightarrow G$ from functor $F$ to functor $G$. This transformation is given as a family of linear isomorphisms $\{\text{Bog}_{k}\}_{k \in \mathcal{C}}$, one for each momentum $k$, where $\text{Bog}_{k}: F(k) \to G(k)$.

By the definition of a natural transformation, for any morphism $f: k \to k'$, the following diagram must be commutative:

\[
	\begin{tikzcd}
		F(k) \arrow[r, "F(f)"] \arrow[d, "\text{Bog}_{k}"'] & F(k') \arrow[d, "\text{Bog}_{k'}"] \\
		G(k) \arrow[r, "G(f)"'] & G(k')
	\end{tikzcd}
\]

This commutativity means that the Bogoliubov transformation is consistent with coordinate transformations (momentum transformations) of the physical system. In other words, the operation of transforming the original particles and then applying the Bogoliubov transformation to get quasiparticles is equivalent to first applying the Bogoliubov transformation to get quasiparticles and then transforming those quasiparticles.

\subsection{Proof of Commutativity}

Let's show this diagram is commutative by component calculation.
The Bogoliubov transformation is defined using coefficients $u_{k}, v_{k}$ as follows:
\[
    \begin{cases}
        \alpha_{k} = u_{k} a_{k} - v_{k} a_{-k}^{\dagger} \\
        \alpha_{k}^{\dagger} = u_{k} a_{k}^{\dagger} - v_{k} a_{-k}
    \end{cases}
\]
A momentum transformation $f: k \to k'$ induces an operator-level map $a_{k} \mapsto a_{k'}$. Similarly, $a_{-k} \mapsto a_{-k'}$ also holds.

\paragraph{Calculation of the left side: $\text{Bog}_{k'} \circ F(f)$}
Acting on the operator $a_{k}$:
\[
    (\text{Bog}_{k'} \circ F(f))(a_{k}) = \text{Bog}_{k'}(a_{k'}) = u_{k'} \alpha_{k'} - v_{k'} \alpha_{-k'}^{\dagger}
\]

\paragraph{Calculation of the right side: $G(f) \circ \text{Bog}_{k}$}
Acting on the operator $a_{k}$:
\[
    (G(f) \circ \text{Bog}_{k})(a_{k}) = G(f)(\text{Bog}_{k}(a_{k})) = G(f)(u_{k} \alpha_{k} - v_{k} \alpha_{-k}^{\dagger})
\]
By the linearity of $G(f)$:
\[
    = u_{k} G(f)(\alpha_{k}) - v_{k} G(f)(\alpha_{-k}^{\dagger})
\]
Assuming $G(f)(\alpha_{k}) = \alpha_{k'}$, we get:
\[
    = u_{k} \alpha_{k'} - v_{k} \alpha_{-k'}^{\dagger}
\]

In a system with physical symmetries (e.g., translational symmetry), the Bogoliubov transformation coefficients $u_{k}, v_{k}$ depend only on the magnitude of the momentum, not its direction. Therefore, $u_{k'} = u_{k}$ and $v_{k'} = v_{k}$ hold. Under this condition, the left and right sides are equal.

\section{Bogoliubov Transformation and Lie Algebra Cohomology}

The Bogoliubov transformation is not just a linear transformation; it \textbf{preserves the algebraic structure of commutation relations}. This property can be understood more deeply using the concept of Lie algebra cohomology.

\subsection{Lie Algebra and the Coboundary Operator}

\paragraph{Lie Algebra $\mathfrak{g}$}:
The boson operators $\{a_{k}, a_{k}^{\dagger}\}$ generate the \textbf{Heisenberg Lie algebra} with the commutator product $[A, B] = AB-BA$ as the Lie bracket. The space of matrices $\mathrm{End}(V_{k})$ that represent physical quantities like the Hamiltonian, when added to this algebra, also forms a Lie algebra.

\paragraph{Coboundary Operator $d_{\text{H}}$}:
For a certain physical quantity $H$ (the Hamiltonian), we define the \textbf{coboundary operator} $d_{\text{H}}: \mathfrak{g} \to \mathfrak{g}$ as a commutator product:
\[
    d_{\text{H}}(X) = [H, X]
\]
This operator has the role of generating non-conserved operators (those that do not commute with $H$) from commutative quantities.

\subsection{Cohomological Interpretation of Off-diagonal Terms}

The Hamiltonian of a superconductor $H_{\text{0}}$ is expressed as the sum of a diagonal term $H_{\text{diag}}$ and an off-diagonal term $H_{\text{off-diag}}$.
\[
    H_{\text{0}} = H_{\text{diag}} + H_{\text{off-diag}}
\]
Here, the Bogoliubov transformation is the operation that eliminates this off-diagonal term. This operation suggests that the off-diagonal term is a \textbf{coboundary}.

\textbf{Assertion}: The off-diagonal term $H_{\text{off-diag}}$ can be expressed as a coboundary of the diagonalized Hamiltonian $H_{\text{diag}}$, using some operator $B \in \mathfrak{g}$.
\[
    H_{\text{off-diag}} = d_{H_{\text{diag}}}(B) = [H_{\text{diag}}, B]
\]
This $B$ corresponds to the operator that generates the Bogoliubov transformation. This relation indicates that through the Bogoliubov transformation, the off-diagonal terms are identified as "quantities to be removed" (coboundaries).

\subsection{Physical Equivalence and Cohomology Groups}

The Hamiltonians $H_{\text{0}}$ and $H_{\text{diag}}$ connected by the Bogoliubov transformation describe the same physical system. This equivalence can be interpreted from the perspective of cohomology as follows.

\begin{itemize}
    \item Cohomology Group $H^1$: The first cohomology group $H^1(\mathfrak{g}, \mathfrak{g})$ on a Lie algebra $\mathfrak{g}$ is the quotient space of the space of cocycles (conserved quantities) by the space of coboundaries.
    \[
    H^1(\mathfrak{g}) = \frac{\ker d_{\text{H}}}{\mathrm{im} d_{\text{H}}} = \frac{\{X \mid [H, X] = 0\}}{\{[H, Y] \mid Y \in \mathfrak{g}\}}
    \]
    \item Physical Significance: The Bogoliubov transformation transforms a Hamiltonian into a "cohomologically equivalent" description through its commutator product. This suggests that the off-diagonal terms, similar to gauge transformations, are "apparent quantities" that depend on the form of the description, while the essential physics of the system (the cohomology class) remains invariant.
\end{itemize}

\section{Universality and Physical Significance}

The Bogoliubov transformation is the \textbf{most natural and universal method} for defining quasiparticles in the theory of superconductivity. This universality has physical significance in the following ways:

\begin{itemize}
    \item \textbf{Concept of Quasiparticles}: Quasiparticles are a \textbf{universal "mathematical tool"} for describing the essential physics of an interacting many-body system as simple particles.
    \item \textbf{Gauge Invariance}: The Bogoliubov transformation is consistent with physical symmetries (e.g., momentum conservation), and through it, essential physical quantities of the system, such as its energy eigenvalues, are kept invariant.
\end{itemize}
The Bogoliubov transformation is not just a mathematical operation; it is a powerful tool for understanding complex phenomena in quantum many-body systems more deeply through systematic mathematical concepts like \textbf{natural transformations in category theory} and \textbf{Lie algebra cohomology}.
\end{document}
\documentclass[uplatex,a4j,12pt,dvipdfmx]{jsarticle}
\usepackage{amsmath,amsthm,amssymb,bm,color,enumitem,mathrsfs,url,epic,eepic,ascmac,ulem,here,ascmac}
\usepackage[letterpaper,top=2cm,bottom=2cm,left=3cm,right=3cm,marginparwidth=1.75cm]{geometry}
\usepackage[english]{babel}
\usepackage[dvipdfm]{graphicx}
\usepackage[hypertex]{hyperref}
\usepackage{tikz-cd}
\title{
関数繰り込み群を利用したガウス積分
}
\author{岡田 大(Okada Masaru)}

\date{\today}

\begin{document}

\maketitle

\begin{abstract}
	関数繰り込み群(fRG)を利用して、厳密解が分かっているガウス積分を解いてみるという、
	fRGの入門的な計算例を示したノート。
	このノートの目的は、fRGが高エネルギーから低エネルギーへと揺らぎを順に取り込んでいく様子を、厳密解が分かっているガウス積分で実演することである。
\end{abstract}

\tableofcontents
\section{このノートの流れ}

\begin{enumerate}
	\item 問題設定: $Z = \int dx e^{-S(x)}$ ($S(x) = \frac{1}{2}m^2 x^2$)を計算したい。
	\item fRGの導入: レギュレーター $R_k$ を導入し、高エネルギースケール $\Lambda$ から $k=0$ まで、有効作用 $\Gamma_k$ を流す(フローさせる)。
	\item Wetterich Equation: その流れを記述するのが Wetterich Equation。
	\item 計算: 方程式を解き、$k=0$ での有効作用 $\Gamma_{k=0}$ を求める。
	\item 結論: $\Gamma_{k=0}$ から $Z = e^{-\Gamma_{k=0}(\phi=0)}$ を計算すると、ガウス積分の正しい答え $\frac{\sqrt{2\pi}}{m}$ が得られる。
\end{enumerate}

\section{Wetterch Equation}

\[
	\partial_{k} \Gamma_{k}
	=
	\frac{1}{2} {\rm STr} \left[ (\Gamma^{(2)}_{k} + R_{k})^{-1} (\partial_{k} R_{k}) \right]
\]
ここで STr はスーパートレースを表す。

これがfRGの核心となるフロー方程式になる。

$\Gamma_{k}$ はスケール $k$ における有効平均作用を表す。
これには $k$ 以上の高エネルギースケールの揺らぎが取り込まれ、
$k$ 未満の揺らぎはまだ取り込まれていない状態の作用である。

$R_{k}$ はレギュレーターと呼ばれる関数であり、
$k$ よりも低いエネルギーのゆらぎを抑制する天井のような設定になっている。
$k = \Lambda$ から $k=0$ へと変化させることは、
この上限を徐々に開放し、
ゆらぎを取り込んでいく操作になる。

$\Gamma^{(2)}_{k} = \frac{\partial^{2} \Gamma_{k}}{\partial \phi^{2}}$
はヘシアンに相当する量で、
場$ \phi = \langle x \rangle$に対する2階微分。
\section{レギュレーター $R_{k}$ とフロー方程式の具体化}

\subsection{Gaussian Integral}

以下のガウス積分は初等的に求められる。
\[
	Z =
	\int^{\infty}_{- \infty} dx e^{ -\frac{1}{2} m^{2} x^{2} }
	=
	\frac{ \sqrt{2 \pi} }{ m }
\]

この積分結果をfRGの枠組みを使って再現してみるというのがこのノートの目的である。

${}$

分配関数と作用の関係は以下である。
\[
	Z = e^{-\Gamma_{k=0}(\phi=0)}
\]
fRGのフレームワークを用いて $\Gamma_{k} (\phi)$ の関数形を導出してから
$\Gamma_{k=0}(\phi=0)$ と置くことで分配関数を求める。

\subsection{Effective Actionの仮定}

この問題は相互作用の無い理論なので、
$\Gamma_{k}$ の場 $\phi$ に関する部分は $k$ に依存せず、
古典作用 $S(\phi)=\frac{1}{2} m^{2} \phi^{2}$ と同じ形であると仮定する。
そこで
\[
	\Gamma_{k} (\phi) = \frac{1}{2} m^{2} \phi^{2} + C_{k}
\]
と置く。

ここで $C_{k}$ は $\phi$ に依らない定数項である。
ここに場のゆらぎの寄与が全て蓄積される。

この仮定から
\[
	\Gamma^{(2)}_{k}
	=
	\frac{\partial^{2} \Gamma_{k}}{\partial \phi^{2}}
	=
	m^{2}
	=
	\mathrm{const.}
\]
となる。
\subsection{レギュレーター $R_{k}$}

レギュレーター $R_{k}$ として
\[
	\begin{array}{rcl}
		R_{k}              & = & k^{2} \\
		\partial_{k} R_{k} & = & 2 k
	\end{array}
\]
を選択する。

\[
	Z_{k} =
	\int^{\infty}_{- \infty} dx e^{ -\frac{1}{2} m^{2} x^{2} - \frac{1}{2} R_{k} x^{2} }
\]

大きな $R_{k}$ の極限ではclassicalになる。

一方で、$R_{k} \to 0$ の極限で求めたい分配関数 $Z_{k=0}$ になる。
\section{Wetterich Equation}

\subsection{Wetterich Equationへ代入}

この問題は0次元であるので、Wetterich Equation の STr は不要で、単純なスカラーの式になる。
\[
	\partial_k \Gamma_k = \frac{1}{2} ( \Gamma_k^{(2)} + R_k )^{-1} (\partial_k R_k)
	= \frac{1}{2} ( m^2 + k^2 )^{-1} (2k)
	= \frac{k}{m^2+k^2}
\]

この式は$\Gamma_k$ のうち、
$\phi$ によらない定数項 $C_k$ だけが $k$ と共に流れていくことを示している。

\subsection{フローの積分}

$\partial_k \Gamma_k$ が求まったので、これを積分することで作用 $\Gamma_{k}$ が求まる。

\[
	\begin{array}{rcl}
		\Gamma_{k}(\phi) & = & \displaystyle \Gamma_{\Lambda} (\phi) - \int^{\Lambda}_{k} dk' \partial_{k'} \Gamma_{k'}(\phi)                 \\
		                 & = & \displaystyle \Gamma_{\Lambda} (\phi) - \frac{1}{2} \ln \left( \frac{m^{2} + \Lambda^{2}}{m^{2}+k^{2}} \right)
	\end{array}
\]

ここで、$\Lambda$ は非常に大きい値(UVカットオフ)である。

\subsubsection{初期条件: $k=\Lambda$}

$k=\Lambda$ ではレギュレーター $R_{\Lambda} = \Lambda^{2}$ によって揺らぎが抑制されている。

$\Gamma_{\Lambda} (\phi)$ は古典極限になっている。

このときの分配関数 $Z_{\Lambda}$ は以下のように計算できる。
\[
	Z_{\Lambda} =
	\int^{\infty}_{- \infty} dx e^{ -\frac{1}{2} m^{2} x^{2} - \frac{1}{2} R_{\Lambda} x^{2} }
	=
	\int^{\infty}_{- \infty} dx e^{ -\frac{1}{2} (m^{2} + \Lambda^2) x^{2} }
	=
	\sqrt{ \frac{2 \pi}{m^{2} + \Lambda^2} }
\]
$\Gamma_\Lambda$ はこの $Z_\Lambda$ と $S(\phi)$ から求まる。

\[
	\Gamma_\Lambda(\phi) = S(\phi) - \ln Z_\Lambda = \frac{1}{2}m^2 \phi^2 - \ln \left( \sqrt{\frac{2\pi}{m^2+\Lambda^2}} \right)
\]

\[
	\Gamma_\Lambda(\phi) = \frac{1}{2}m^2 \phi^2 + \frac{1}{2} \ln \left( \frac{m^2+\Lambda^2}{2\pi} \right)
\]

\subsubsection{作用 $\Gamma_{k}(\phi)$ の関数形}

積分結果に、この $\Gamma_\Lambda(\phi)$ を代入する。

\[
	\Gamma_k(\phi) = \left[ \frac{1}{2}m^2 \phi^2 + \frac{1}{2} \ln \left( \frac{m^2+\Lambda^2}{2\pi} \right) \right] - \frac{1}{2} \ln \left( \frac{m^{2} + \Lambda^{2}}{m^{2}+k^{2}} \right)
\]

$\ln$ の項が打ち消し合い、
\[
	\Gamma_k(\phi) = \frac{1}{2}m^2 \phi^2 + \frac{1}{2} \ln \left( \frac{m^2+k^2}{2\pi} \right)
\]
となる。
\subsubsection{求めたかった $k=0$ における作用 $\Gamma_{0}(\phi)$ の関数形}

$k=0$ では、
レギュレーター $R_0 = 0$ となり、
全てのゆらぎを取り込んだ、
元の理論の真の有効作用 $\Gamma_{k=0}$ が得られる。
\[
	\Gamma_{k=0}(\phi) = \frac{1}{2}m^2 \phi^2 + \frac{1}{2} \ln \left( \frac{m^2}{2\pi} \right)
\]

\subsubsection{ガウス積分の値}

求めたい $Z$ は、$Z = e^{-\Gamma_{k=0}(\phi=0)}$ で与えられる。
\[
	\Gamma_{k=0}(0) = \frac{1}{2} \ln \left( \frac{m^2}{2\pi} \right)
\]
\[
	Z = e^{-\Gamma_{k=0}(0)} = \exp \left[ -\frac{1}{2} \ln \left( \frac{m^2}{2\pi} \right) \right]
	= \exp \left[ \ln \left( \left(\frac{m^2}{2\pi}\right)^{-1/2} \right) \right]
	= \left( \frac{m^2}{2\pi} \right)^{-1/2}
	= \sqrt{\frac{2\pi}{m^2}}
	= \frac{\sqrt{2\pi}}{m}
\]
\section{まとめ}

このノートでは、関数繰り込み群 (fRG) という高度な理論的手法を、
単純なガウス積分(相互作用のない理論)に対して用いた。

fRGのフローを通して、厳密かつ系統的に「ゆらぎ」の効果
(この場合は $\phi$ によらない定数項、すなわち分配関数そのもの)
を取り込んでいくかを示した。

\end{document}
\documentclass[uplatex,a4j,12pt,dvipdfmx]{jsarticle}
\usepackage{amsmath,amsthm,amssymb,bm,color,enumitem,mathrsfs,url,epic,eepic,ascmac,ulem,here,ascmac}
\usepackage[letterpaper,top=2cm,bottom=2cm,left=3cm,right=3cm,marginparwidth=1.75cm]{geometry}
\usepackage[english]{babel}
\usepackage[dvipdfm]{graphicx}
\usepackage[hypertex]{hyperref}
\usepackage{tikz-cd}
\title{
Gaussian Integral using the Functional Renormalization Group
}
\author{Masaru Okada}

\date{\today}

\begin{document}

\maketitle

\begin{abstract}
	A note presenting an introductory computational example of the fRG:
	using the functional renormalization group (fRG) to solve a Gaussian integral,
	for which an exact solution is known.
	The purpose of this note is to demonstrate how the fRG sequentially integrates out
	fluctuations from high to low energies, using a Gaussian integral as an
	exactly solvable example.
\end{abstract}

\tableofcontents
\section{Flow of this note}

\begin{enumerate}
	\item Problem Setup: We want to calculate $Z = \int dx e^{-S(x)}$ (where $S(x) = \frac{1}{2}m^2 x^2$).
	\item Introduction of fRG: Introduce a regulator $R_k$ and 'flow' the effective action $\Gamma_k$ from a high energy scale $\Lambda$ down to $k=0$.
	\item Wetterich Equation: This flow is described by the Wetterich Equation.
	\item Calculation: Solve the equation to find the effective action at $k=0$, $\Gamma_{k=0}$.
	\item Conclusion: Calculating $Z = e^{-\Gamma_{k=0}(\phi=0)}$ from $\Gamma_{k=0}$ yields the correct answer for the Gaussian integral, $\frac{\sqrt{2\pi}}{m}$.
\end{enumerate}

\section{Wetterch Equation}

\[
	\partial_{k} \Gamma_{k}
	=
	\frac{1}{2} {\rm STr} \left[ (\Gamma^{(2)}_{k} + R_{k})^{-1} (\partial_{k} R_{k}) \right]
\]
Here, STr denotes the supertrace.

This is the flow equation that lies at the heart of the fRG.

$\Gamma_{k}$ represents the effective average action at scale $k$.
This is an action in which fluctuations from high energy scales above $k$ have been integrated out,
while fluctuations below $k$ have not yet been included.

$R_{k}$ is a function called the regulator,
which acts as a 'cap' to suppress fluctuations with energies lower than $k$.
Changing $k$ from $\Lambda$ down to $0$ corresponds to the operation
of gradually removing this cap and incorporating the fluctuations.

$\Gamma^{(2)}_{k} = \frac{\partial^{2} \Gamma_{k}}{\partial \phi^{2}}$
is the quantity corresponding to the Hessian,
the second derivative with respect to the field $\phi = \langle x \rangle$.

\section{Specifying the Regulator $R_{k}$ and the Flow Equation}

\subsection{Gaussian Integral}

The following Gaussian integral can be obtained elementarily.
\[
	Z =
	\int^{\infty}_{- \infty} dx e^{ -\frac{1}{2} m^{2} x^{2} }
	=
	\frac{ \sqrt{2 \pi} }{ m }
\]

The objective of this note is to reproduce this integral's result using the fRG framework.

${}$

The relationship between the partition function and the action is as follows:
\[
	Z = e^{-\Gamma_{k=0}(\phi=0)}
\]
We will find the partition function by first deriving the functional form of $\Gamma_{k} (\phi)$
using the fRG framework, and then evaluating it at $\Gamma_{k=0}(\phi=0)$.

\subsection{Assumption for the Effective Action}

Since this problem is a non-interacting theory,
we assume that the part of $\Gamma_k$ dependent on the field $\phi$ does not depend on $k$
and retains the same form as the classical action $S(\phi)=\frac{1}{2} m^{2} \phi^{2}$.
Therefore, we set
\[
	\Gamma_{k} (\phi) = \frac{1}{2} m^{2} \phi^{2} + C_{k}
\]

Here, $C_k$ is a constant term independent of $\phi$.
All contributions from field fluctuations are accumulated here.

From this assumption, it follows that
\[
	\Gamma^{(2)}_{k}
	=
	\frac{\partial^{2} \Gamma_{k}}{\partial \phi^{2}}
	=
	m^{2}
	=
	\mathrm{const.}
\]

\subsection{The Regulator $R_{k}$}

We choose the following for the regulator $R_k$:
\[
	\begin{array}{rcl}
		R_{k}              & = & k^{2} \\
		\partial_{k} R_{k} & = & 2 k
	\end{array}
\]

\[
	Z_{k} =
	\int^{\infty}_{- \infty} dx e^{ -\frac{1}{2} m^{2} x^{2} - \frac{1}{2} R_{k} x^{2} }
\]

In the limit of large $R_{k}$, the system becomes classical.

Conversely, in the limit $R_{k} \to 0$, we recover the desired partition function $Z_{k=0}$.

\section{Wetterich Equation}

\subsection{Substitution into the Wetterich Equation}

Since this problem is 0-dimensional, the STr in the Wetterich Equation is unnecessary,
and it becomes a simple scalar equation.
\[
	\partial_k \Gamma_k = \frac{1}{2} ( \Gamma_k^{(2)} + R_k )^{-1} (\partial_k R_k)
	= \frac{1}{2} ( m^2 + k^2 )^{-1} (2k)
	= \frac{k}{m^2+k^2}
\]

This equation shows that
only the $\phi$-independent constant term $C_k$ within $\Gamma_k$ 'flows' with $k$.

\subsection{Integration of the Flow}

Now that $\partial_k \Gamma_k$ has been found, we can integrate it to find the action $\Gamma_{k}$.

\[
	\begin{array}{rcl}
		\Gamma_{k}(\phi) & = & \displaystyle \Gamma_{\Lambda} (\phi) - \int^{\Lambda}_{k} dk' \partial_{k'} \Gamma_{k'}(\phi)                 \\
		                 & = & \displaystyle \Gamma_{\Lambda} (\phi) - \frac{1}{2} \ln \left( \frac{m^{2} + \Lambda^{2}}{m^{2}+k^{2}} \right)
	\end{array}
\]

Here, $\Lambda$ is a very large value (the UV cutoff).

\subsubsection{Initial Condition: $k=\Lambda$}

At $k=\Lambda$, fluctuations are suppressed by the regulator $R_{\Lambda} = \Lambda^{2}$.

$\Gamma_{\Lambda} (\phi)$ is in the classical limit.

The partition function at this scale, $Z_{\Lambda}$, can be calculated as follows:
\[
	Z_{\Lambda} =
	\int^{\infty}_{- \infty} dx e^{ -\frac{1}{2} m^{2} x^{2} - \frac{1}{2} R_{\Lambda} x^{2} }
	=
	\int^{\infty}_{- \infty} dx e^{ -\frac{1}{2} (m^{2} + \Lambda^2) x^{2} }
	=
	\sqrt{ \frac{2 \pi}{m^{2} + \Lambda^2} }
\]
$\Gamma_\Lambda$ is determined from this $Z_\Lambda$ and $S(\phi)$.

\[
	\Gamma_\Lambda(\phi) = S(\phi) - \ln Z_\Lambda = \frac{1}{2}m^2 \phi^2 - \ln \left( \sqrt{\frac{2\pi}{m^2+\Lambda^2}} \right)
\]

\[
	\Gamma_\Lambda(\phi) = \frac{1}{2}m^2 \phi^2 + \frac{1}{2} \ln \left( \frac{m^2+\Lambda^2}{2\pi} \right)
\]

\subsubsection{Functional Form of the Action $\Gamma_{k}(\phi)$}

We substitute this $\Gamma_\Lambda(\phi)$ into the integral result.

\[
	\Gamma_k(\phi) = \left[ \frac{1}{2}m^2 \phi^2 + \frac{1}{2} \ln \left( \frac{m^2+\Lambda^2}{2\pi} \right) \right] - \frac{1}{2} \ln \left( \frac{m^{2} + \Lambda^{2}}{m^{2}+k^{2}} \right)
\]

The $\ln$ terms cancel out, leaving:
\[
	\Gamma_k(\phi) = \frac{1}{2}m^2 \phi^2 + \frac{1}{2} \ln \left( \frac{m^2+k^2}{2\pi} \right)
\]

\subsubsection{Functional Form of the Desired Action at $k=0$, $\Gamma_{0}(\phi)$}

At $k=0$, the regulator becomes $R_0 = 0$,
and we obtain the true effective action $\Gamma_{k=0}$ of the original theory,
which incorporates all fluctuations.
\[
	\Gamma_{k=0}(\phi) = \frac{1}{2}m^2 \phi^2 + \frac{1}{2} \ln \left( \frac{m^2}{2\pi} \right)
\]

\subsubsection{Value of the Gaussian Integral}

The desired $Z$ is given by $Z = e^{-\Gamma_{k=0}(\phi=0)}$.
\[
	\Gamma_{k=0}(0) = \frac{1}{2} \ln \left( \frac{m^2}{2\pi} \right)
\]
\[
	Z = e^{-\Gamma_{k=0}(0)} = \exp \left[ -\frac{1}{2} \ln \left( \frac{m^2}{2\pi} \right) \right]
	= \exp \left[ \ln \left( \left(\frac{m^2}{2\pi}\right)^{-1/2} \right) \right]
	= \left( \frac{m^2}{2\pi} \right)^{-1/2}
	= \sqrt{\frac{2\pi}{m^2}}
	= \frac{\sqrt{2\pi}}{m}
\]
\section{Conclusion}

In this note, the advanced theoretical method of the functional renormalization group (fRG)
was applied to a simple Gaussian integral (a non-interacting theory).

It has been shown how, through the fRG flow, the effect of fluctuations
(in this case, the $\phi$-independent constant term, i.e., the partition function itself)
is rigorously and systematically incorporated.

\end{document}
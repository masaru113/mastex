\documentclass[uplatex,a4j,12pt,dvipdfmx]{jsarticle}
\usepackage[english]{babel}
\usepackage[letterpaper,top=2cm,bottom=2cm,left=3cm,right=3cm,marginparwidth=1.75cm]{geometry}
\usepackage{amsmath,amsthm,amssymb,bm,color,mathrsfs,url}
\usepackage{epic,eepic,here}
\usepackage[dvipdfm]{graphicx}
\usepackage[hypertex]{hyperref}
\title{Migdal-Eliashberg Theory in Nonequilibrium}
\author{Masaru Okada}
\date{\today}
\def\sla#1{\rlap/#1}
\begin{document}
\maketitle

\begin{abstract}
	Notes on Migdal-Eliashberg theory in nonequilibrium.
	The superconducting gap equation in the Migdal-Eliashberg model extended to nonequilibrium.
\end{abstract}

We start from Matsubara space.
Setting the Matsubara frequency for fermions as $\varepsilon_{n} = 2 \pi i T (n + 1/2)$, the Dyson equation for the Gorkov Green's function is

\begin{eqnarray}
	\sum_{\varepsilon_{m}}
	\int \dfrac{ d^{3} \vec{k}_{1} }{ (2 \pi)^{3} }
	\Big[ \check{G}^{-1}_{\varepsilon_{n}} (\vec{p} - \vec{k}_{1} , \omega_{1} ) - \check{\mathit{\Sigma}}_{\varepsilon_{n},\varepsilon_{m}}(\vec{p},\vec{k}_{1}) \Big]
	\check{G}_{\varepsilon_{m},\varepsilon_{\vec{k}}} ( \vec{p} - \vec{k}_{1} , \vec{p} - \vec{k} )
	&=&
	\check{1} (2 \pi)^{3} \delta(\vec{k}) \delta_{\varepsilon_{n} - \varepsilon_{\vec{k}}}
	\label{eqn:dysonone}
\end{eqnarray}

A function with two energy variables as a subscript is a two-particle function.
$\check{G}^{-1}_{\varepsilon}$ has the following structure:

\begin{eqnarray}
	\check{G}^{-1}_{\varepsilon}(\vec{p}-\vec{k}_{1},\omega_{1})
	&=&
	\left(
	\begin{array}{cc}
		\xi_{\vec{p}} - \varepsilon & 0
		\\
		0                           & \xi_{\vec{p}} + \varepsilon
	\end{array}
	\right)
	(2 \pi)^{3}
	\delta(\vec{k}_{1})
	\delta_{\omega_{1}}
	+
	\check{H}_{\omega_{1}}
\end{eqnarray}

We set $\omega_{1} = \varepsilon_{n} - \varepsilon_{m}$.
The time-dependent external field $\check{H}_{\omega_{1}}$ is currently assumed to be:

\begin{eqnarray}
	\check{H}_{\omega_{1}}
	&=&
	\left(
	\begin{array}{cc}
			- \dfrac{e}{c} \vec{v}_{\rm F} \vec{A}_{\omega_{1}} (\vec{k}_{1}) + e \phi_{\omega_{1}}(\vec{k}_{1}) & - \Delta_{\omega_{1}} (\vec{k}_{1})
			\\
			\Delta^{*}_{\omega_{1}} (\vec{k})                                                                    & \dfrac{e}{c} \vec{v}_{\rm F} \vec{A}_{\omega_{1}} (\vec{k}_{1}) + e \phi_{\omega_{1}} (\vec{k}_{1})
		\end{array}
	\right)
\end{eqnarray}

The N-th order electron Green's function $\check{G}^{(N)}$ expanded up to N-th order in $\check{H}_{\omega_{j}}$ is:

\begin{eqnarray}
	\check{G}^{(N)}_{\varepsilon_{n},\varepsilon_{n} - \omega}
	&=&
	(-1)^{N}
	G^{(0)}_{\varepsilon_{n}}
	\check{H}_{\omega_{1}}
	G^{(0)}_{\varepsilon_{n} - \omega_{1}}
	\check{H}_{\omega_{2}}
	G^{(0)}_{\varepsilon_{n} - \omega_{1} - \omega_{2} }
	\ \cdots \
	\check{H}_{\omega_{N}}
	G^{(0)}_{\varepsilon_{n} - \omega }
\end{eqnarray}

Here, we set $\omega = \omega_{1} + \omega_{2} + \cdots + \omega_{N}$. The free electron Green's function $G^{(0)}$ is:

\begin{eqnarray}
	G^{(0)}_{\varepsilon_{n}}
	&=&
	\dfrac{1}{ \xi_{\vec{p}} - \varepsilon_{n} }
\end{eqnarray}

(This has the opposite sign to AGD, etc.).
Using the residue theorem, the following equality holds for a complex number $z$:

\begin{eqnarray}
	T \sum_{n} \check{G}^{(N)}_{\varepsilon_{n},\varepsilon_{n} - \omega}
	&=&
	\oint \!\! \dfrac{dz}{4 \pi i}
	\check{G}^{(N)}_{z,z - \omega}
	{\rm tanh} \dfrac{z}{2 T}
\end{eqnarray}

The integration path is a circle of infinite radius centered at the origin, in the clockwise direction.
When the interaction Hamiltonian does not have time dependence, we only needed to consider the paths around the singularities originating from ${\rm tanh}(z/2 T)$, i.e., $z=2 \pi i T (n + 1/2)$.
However, in this case, the N-th order electron Green's function also has singular lines originating from $\check{H}_{\omega_{j}}$:

\begin{eqnarray}
	{\rm Im}
	\Big( z - \sum^{l}_{j = 0} \omega_{j} \Big)
	&=&
	0
	\label{eqn:tokuisen}
\end{eqnarray}

Therefore, we split the path into $j$ branches, running from ${\rm Re}(z) = - \infty$ to $\infty$ along the upper side $(+i0)$ of each singular line, and from ${\rm Re}(z) = \infty$ to $-\infty$ along the lower side $(-i0)$.
(Here, we set $\omega_{0}=0$ for simplicity of notation.)
In this way, the integral can be expressed using the complex variable $\varepsilon$:

\begin{eqnarray}
	\oint \!\! \dfrac{dz}{4 \pi i}
	\check{G}^{(N)}_{z,z - \omega}
	{\rm tanh} \dfrac{z}{2 T}
	&=&
	\int \! \! \dfrac{d \varepsilon}{4 \pi i}
	\Big(
	\delta_{0} \check{G}^{(N)}
	+
	\delta_{1} \check{G}^{(N)}
	+
	\cdots
	+
	\delta_{N} \check{G}^{(N)}
	\Big)
	{\rm tanh} \dfrac{\varepsilon}{2 T}
\end{eqnarray}

Here, $\delta_{l} \check{G}^{(N)}$ represents the jump in the Green's function at the l-th singular line (Eq. (\ref{eqn:tokuisen})), and is specifically written as:

\begin{eqnarray}
	\delta_{l} \check{G}^{(N)}
	&=&
	\Big[ \check{G}^{(N)}_{z,z - \omega} \Big]_{z = z_{l} + 0}
	-
	\Big[ \check{G}^{(N)}_{z,z - \omega} \Big]_{z = z_{l} - 0}
	\hspace{10mm}
	\left( {\rm where} \ \ \ z_{l} = \varepsilon + i \ \! {\rm Im} \sum^{l}_{j=0} \omega_{j} \right)
\end{eqnarray}

This is where the analyticity differs from that of the equilibrium Green's function, as we consider the jumps on the singular lines originating from the time-dependent external field.






We perform analytical continuation.
This jump, for example in the case of $l=1$, is:

\begin{eqnarray}
	\delta_{1} \check{G}^{(N)}
	&=&
	(-1)^{N}
	G^{(0)}_{\varepsilon + \omega_{1} }
	\check{H}_{\omega_{1}}
	\Big( G^{(0)}_{\varepsilon + i 0 } - G^{(0)}_{\varepsilon - i 0 } \Big)
	\check{H}_{\omega_{2}}
	G^{(0)}_{\varepsilon - \omega_{2} }
	\ \cdots \
	\check{H}_{\omega_{N}}
	G^{(0)}_{\varepsilon + \omega_{1} - \omega }
\end{eqnarray}

However, as is conventional, by using the retarded (advanced) Green's function $G^{(0) R(A)}$ where the poles are infinitesimally shifted up (down) along the imaginary axis, we perform the analytical continuation as:

\begin{eqnarray}
	\delta_{1} \check{G}^{(N)}
	&=&
	(-1)^{N}
	G^{(0)R}_{\varepsilon + \omega_{1} }
	\check{H}_{\omega_{1}}
	\Big( G^{(0)R}_{\varepsilon} - G^{(0)A}_{\varepsilon} \Big)
	\check{H}_{\omega_{2}}
	G^{(0)A}_{\varepsilon - \omega_{2} }
	\ \cdots \
	\check{H}_{\omega_{N}}
	G^{(0)A}_{\varepsilon + \omega_{1} - \omega }
\end{eqnarray}

Shifting the integration dummy variable $\varepsilon + \omega \to \varepsilon$ and explicitly writing the variables gives:

\begin{eqnarray}
	\delta_{1} \check{G}^{(N)}_{\varepsilon , \varepsilon - \omega}
	&=&
	(-1)^{N}
	G^{(0)R}_{\varepsilon }
	\check{H}_{\omega_{1}}
	\Big( G^{(0)R}_{\varepsilon - \omega_{1}} - G^{(0)A}_{\varepsilon - \omega_{1}} \Big)
	\check{H}_{\omega_{2}}
	G^{(0)A}_{\varepsilon - \omega_{1} - \omega_{2} }
	\ \cdots \
	\check{H}_{\omega_{N}}
	G^{(0)A}_{\varepsilon - \omega }
\end{eqnarray}

The integral can then be written as:

\begin{eqnarray}
	\oint \!\! \dfrac{dz}{4 \pi i}
	\check{G}^{(N)}_{z,z - \omega}
	{\rm tanh} \dfrac{z}{2 T}
	&=&
	\int^{\infty}_{-\infty} \! \dfrac{d \varepsilon}{4 \pi i}
	\Big[
	{\rm tanh} \Big( \dfrac{\varepsilon}{2 T} \Big)
	\delta_{0} \check{G}^{(N)}_{\varepsilon , \varepsilon - \omega}
	+
	{\rm tanh} \Big( \dfrac{\varepsilon - \omega_{1}}{2 T} \Big)
	\delta_{1} \check{G}^{(N)}_{\varepsilon , \varepsilon - \omega}
	\nonumber \\ &&
	+
	{\rm tanh} \Big( \dfrac{\varepsilon - \omega_{1} - \omega_{2}}{2 T} \Big)
	\delta_{1} \check{G}^{(N)}_{\varepsilon , \varepsilon - \omega}
	+
	\cdots
	+
	{\rm tanh} \Big( \dfrac{\varepsilon - \omega}{2 T} \Big)
	\delta_{N} \check{G}^{(N)}_{\varepsilon , \varepsilon - \omega}
	\Big]
	\nonumber \\[3mm] &=&
	\sum_{l=0}^{N}
	\int^{\infty}_{-\infty} \! \dfrac{d \varepsilon}{4 \pi i}
	{\rm tanh} \Big( \dfrac{\varepsilon - \sum_{j=0}^{l} \omega_{j} }{2 T} \Big)
	\delta_{l} \check{G}^{(N)}_{\varepsilon , \varepsilon - \omega}
\end{eqnarray}

We now denote the integrand as:

\begin{eqnarray}
	\check{G}^{(N)K}_{\varepsilon , \varepsilon - \omega}
	&=&
	\sum_{l=0}^{N}
	{\rm tanh} \Big( \dfrac{\varepsilon - \sum_{j=0}^{l} \omega_{j} }{2 T} \Big)
	\delta_{l} \check{G}^{(N)}_{\varepsilon , \varepsilon - \omega}
\end{eqnarray}

Writing the unperturbed distribution function (Fermi distribution function) as $f^{(0)}(\varepsilon) = {\rm tanh}\dfrac{\varepsilon}{2 T}$,
(although at this point it's not yet obvious that $\check{G}^{K}$ coincides with the conventional Keldysh Green's function),
the fully perturbed Keldysh Green's function $\check{G}^{K} = \displaystyle \lim_{N \to \infty} \check{G}^{(N)K}$ can be written as:

\begin{eqnarray}
	\check{G}^{K}_{\varepsilon , \varepsilon - \omega}
	&=&
	\check{G}^{R}_{\varepsilon , \varepsilon - \omega}
	f^{(0)}(\varepsilon - \omega)
	-
	f^{(0)}(\varepsilon)
	\check{G}^{A}_{\varepsilon , \varepsilon - \omega}
	+
	\check{G}^{(a)}_{\varepsilon , \varepsilon - \omega}
\end{eqnarray}

The remainder function $\check{G}^{(a)}$ is:

\begin{eqnarray}
	\check{G}^{(N)(a)}_{\varepsilon , \varepsilon - \omega}
	&=&
	(-1)^{N-1}
	G^{(0)R}_{\varepsilon}
	\check{h}_{\varepsilon , \varepsilon - \omega_{1}}
	G^{(0)A}_{\varepsilon - \omega_{1}}
	\ \cdots \
	\check{H}_{\omega_{N}}
	G^{(0)A}_{\varepsilon - \omega}
	\ + \cdots
	\nonumber \\ && +
	(-1)^{N-1}
	G^{(0)R}_{\varepsilon}
	\check{H}_{\omega_{1}}
	\ \cdots \
	G^{(0)R}_{\varepsilon - ( \omega - \omega_{N} )}
	\check{h}_{\varepsilon - ( \omega - \omega_{N} ) , \varepsilon - \omega}
	G^{(0)A}_{\varepsilon - \omega}
\end{eqnarray}

And specifically for an equilibrium system (free system), it can be simply written as:

\begin{eqnarray}
	\check{G}^{(a)}_{\varepsilon , \varepsilon - \omega} ( \vec{p} , \vec{p} - \vec{k})
	&=&
	\int \!\! \dfrac{d \varepsilon'  d \omega' }{ (2 \pi)^{2} } \dfrac{d^{3} \vec{k'}  d^{3} \vec{k''} }{ (2 \pi)^{6} }
	G^{R}_{\varepsilon , \varepsilon'} (\vec{p} , \vec{p} - \vec{k'})
	\check{h}_{\varepsilon' , \varepsilon' - \omega'} (\vec{k''})
	G^{A}_{\varepsilon' - \omega' , \varepsilon - \omega} (\vec{p} - \vec{k'} - \vec{k''} , \vec{p} - \vec{k} )
\end{eqnarray}

where we abbreviated the product of the external field and the distribution function as $\check{h}_{\varepsilon , \varepsilon - \omega}$:

\begin{eqnarray}
	\check{h}_{\varepsilon , \varepsilon - \omega}
	&=&
	- \check{H}_{\omega} \Big( f^{(0)}(\varepsilon) - f^{(0)}(\varepsilon - \omega) \Big)
\end{eqnarray}










${}$











We now develop the discussion for the case where Migdal's theorem holds.
In that case, we can assume that the fully perturbed phonon Green's function is equal to the free Green's function.
We assume that the self-energy $\check{\mathit{\Sigma}}_{\varepsilon_{n},\varepsilon_{m}}$ in Eq. (\ref{eqn:dysonone}) is due only to the electron-phonon interaction.
If we write the phonon Green's function as $D_{\varepsilon}$:

\begin{eqnarray}
	D_{\varepsilon' - \varepsilon} ( \vec{p} {\ \!}' - \vec{p} {\ \!} )
	&=&
	\dfrac{ \omega^{2}_{ \vec{p}' - \vec{p} } }{ \omega^{2}_{ \vec{p}' - \vec{p} } - ( \varepsilon' - \varepsilon )^{2} }
\end{eqnarray}

Here, we set $\varepsilon' - \varepsilon=2 \pi i m$.
Under Migdal's theorem, the phonon self-energy $\check{\mathit{\Sigma}}^{(\rm ph)}_{\varepsilon , \varepsilon - \omega}$ is:

\begin{eqnarray}
	\check{\mathit{\Sigma}}^{(\rm ph)}_{\varepsilon , \varepsilon - \omega} (\vec{p} , \vec{p} - \vec{k})
	&=&
	T
	\sum_{\varepsilon'}
	\int \!\! \dfrac{ d^{3} \vec{p {} \ \! '} }{ (2 \pi)^{3} }
	g^{2}
	D_{\varepsilon' - \varepsilon} ( \vec{p} {\ \!}' - \vec{p} {\ \!} )
	\check{G}_{\varepsilon' , \varepsilon' - \omega} (\vec{p {} '} , \vec{p {} '} - \vec{k})
\end{eqnarray}

Up to the N-th order of the external field,

\begin{eqnarray}
	\check{\mathit{\Sigma}}^{({\rm ph})(N)}_{\varepsilon , \varepsilon - \omega}
	&=&
	g^{2} \oint \!\! \dfrac{d z}{4 \pi i}
	\Big( D_{z - \varepsilon} \check{G}_{z,z-\varepsilon}^{(N)} \Big)
	{\rm tanh} \Big( \dfrac{z}{2T} \Big)
\end{eqnarray}

For simplicity, the $d^{3} \vec{p}$ integral is not written explicitly.
Let's look at the analyticity that appears in the integrand.
The singularities (lines) are, firstly, the electron Green's function having ${\rm Im}(z - \omega_{j})=0$, and the phonon Green's function having ${\rm Im}(z - \varepsilon_{n})=0$.
Now, $\omega_{j} = 2 \pi i T k_{j}$ and $\varepsilon_{n} = 2 \pi i T (n + 1/2)$ are both imaginary, but using a real number $\varepsilon'$, we substitute variables as $z= \varepsilon' + i \ \! {\rm Im} (\varepsilon_{n})$ for $\check{G}$ and $z= \varepsilon' + i \ \! {\rm Im} (\omega_{j})$ for $D$.

\begin{eqnarray}
	\check{\mathit{\Sigma}}^{(\rm ph)}_{\varepsilon , \varepsilon - \omega}
	&=&
	g^{2} \int \!\! \dfrac{d \varepsilon'}{4 \pi i}
	\Big[
		{\rm coth} \Big( \dfrac{\varepsilon'}{2T} \Big)
		\big( D^{R}_{\varepsilon'} - D^{A}_{\varepsilon'} \big)
		\check{G}_{ \varepsilon' + \varepsilon , \varepsilon' + \varepsilon - \omega }^{(N)} \Big]
	\nonumber \\ && +
	g^{2} \int \!\! \dfrac{d \varepsilon'}{4 \pi i}
	{\rm tanh} \Big( \dfrac{\varepsilon'}{2T} \Big)
	\Big[
		D_{\varepsilon' - \varepsilon} \delta_{0} \check{G}^{(N)}_{\varepsilon' , \varepsilon' - \omega}
		+
		D_{\varepsilon' - (\varepsilon - \omega_{1})} \delta_{1} \check{G}^{(N)}_{\varepsilon' , \varepsilon' - \omega}
		+ \cdots +
		D_{\varepsilon' - (\varepsilon - \omega)} \delta_{N} \check{G}^{(N)}_{\varepsilon' , \varepsilon' - \omega}
		\Big]
	\label{eqn:anaricgma}
\end{eqnarray}

The first term on the left-hand side arose from changing the integration path to be parallel to the real axis around the singularities originating from the $D$ lines, and the second term from the singularities originating from $\check{G}_{z,z-\varepsilon}^{(N)}$ (i.e., from the "external field" $\check{H}_{\omega_{j}}$).
We consider analytical continuation.
Looking at the integrand of Eq. (\ref{eqn:anaricgma}), it can be confirmed that the singularities of $\check{\mathit{\Sigma}}^{(\rm ph)}_{\varepsilon , \varepsilon - \omega}$ as a function of $\varepsilon$ coincide with the singularities of $\check{G}_{z,z-\varepsilon}^{(N)}$.
Therefore, the self-energy can also be analytically continued, just like the total Green's function, for all orders of interaction.
By performing variable transformations like $\displaystyle \varepsilon' \to \varepsilon' - \sum_{l=0}^{j} \omega_{l}$ for the singularities (lines) of each $\check{G}^{(N)}$ and $\displaystyle \varepsilon' \to \varepsilon' - \varepsilon$ for the singularities (lines) of $D$, we obtain the analytically continued phonon self-energy.

\begin{eqnarray}
	\check{\mathit{\Sigma}}^{{(\rm ph)} {{R}\atop{A}} }_{\varepsilon , \varepsilon - \omega} (\vec{p} , \vec{p} - \vec{k})
	&=&
	g^{2} \int \!\! \dfrac{d \varepsilon'}{4 \pi i} \dfrac{d^{3} \vec{p'}}{ (2 \pi)^{3}}
	\Big[
		{\rm coth} \Big( \dfrac{ \varepsilon' - \varepsilon }{2T} \Big)
		\big( D^{R}_{\varepsilon'} - D^{A}_{\varepsilon'} \big)
		\check{G}^{ {{R}\atop{A}} }_{ \varepsilon' , \varepsilon'- \omega }
		+
		D^{ {{A}\atop{R}} }_{\varepsilon' - \varepsilon} \check{G}^{K}_{ \varepsilon' , \varepsilon'- \omega }
		\Big]
\end{eqnarray}

The Keldysh component is:

\begin{eqnarray}
	\check{\mathit{\Sigma}}^{{(\rm ph)} K }_{\varepsilon , \varepsilon - \omega} (\vec{p} , \vec{p} - \vec{k})
	&=&
	g^{2} \int \!\! \dfrac{d \varepsilon'}{4 \pi i} \dfrac{d^{3} \vec{p'}}{ (2 \pi)^{3}}
	\big( D^{R}_{\varepsilon'} - D^{A}_{\varepsilon'} \big)
	\Big[
		{\rm coth} \Big( \dfrac{ \varepsilon' - \varepsilon }{2T} \Big)
		\check{G}^{K}_{ \varepsilon' , \varepsilon'- \omega }
		-
		\big( \check{G}^{R}_{ \varepsilon' , \varepsilon'- \omega } - \check{G}^{A}_{ \varepsilon' , \varepsilon'- \omega } \big)
		\Big]
\end{eqnarray}

And by including the Keldysh components in Nambu space, the size of the Green's function matrix becomes $4 \times 4$.

\begin{eqnarray}
	\breve{G}
	&=&
	\left(
	\begin{array}{cc}
			\check{G}^{R} & \check{G}^{K}
			\\
			0             & \check{G}^{A}
		\end{array}
	\right)
	\ \ = \ \
	\left(
	\begin{array}{cccc}
			G^{R}           & F^{R}               & G^{K}           & F^{K}
			\\
			- F^{\dagger R} & \check{\bar{G}}^{R} & - F^{\dagger K} & \check{\bar{G}}^{K}
			\\
			0               & 0                   & G^{A}           & F^{A}
			\\
			0               & 0                   & - F^{\dagger A} & \check{\bar{G}}^{A}
		\end{array}
	\right)
\end{eqnarray}

The gap equation of the Migdal-Eliashberg model (the nonlinear Eliashberg equation under nonequilibrium conditions) becomes:

\begin{eqnarray}
	\Delta_{\varepsilon , \varepsilon - \omega}^{(*)} (\vec{p} , \vec{p} - \vec{k})
	&=&
	\dfrac{g^{2}}{2}
	\int \!\! \dfrac{d \varepsilon'}{4 \pi i} \dfrac{d^{3} \vec{p'}}{ (2 \pi)^{3}}
	\big( D^{R}_{\varepsilon' - \varepsilon} + D^{A}_{\varepsilon' - \varepsilon} \big)
	F^{(\dagger)K}_{\varepsilon' , \varepsilon' - \omega}
\end{eqnarray}

Since $F^{K}$ becomes nearly zero and does not remain finite for the region where $\varepsilon \gg \Delta$, the case where $\varepsilon \ll \omega_{\rm D}$ with respect to the Debye frequency $\omega_{\rm D}$ becomes important.
Furthermore, in this case, $(D^{R} + D^{A})/2 \simeq 1$, and the dependence on $\varepsilon$ and $\omega$ becomes minimal. Then, the linearized Eliashberg equation:

\begin{eqnarray}
	\Delta_{\omega}^{} (\vec{k})
	&=&
	g^{2}
	\int \!\! \dfrac{d \varepsilon}{4 \pi i} \dfrac{d^{3} \vec{p'}}{ (2 \pi)^{3}}
	F^{K}_{\varepsilon , \varepsilon - \omega}
\end{eqnarray}

can also be reduced to an equation similar to the one obtained by assuming an s-wave in a BCS-like manner (i.e., without considering the electron-phonon interaction).


\end{document}
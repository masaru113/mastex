\documentclass[uplatex,a4j,12pt,dvipdfmx]{jsarticle}
\usepackage{amsmath,amsthm,amssymb,bm,color,enumitem,mathrsfs,url,epic,eepic,ascmac,ulem,here,ascmac}
\usepackage[letterpaper,top=2cm,bottom=2cm,left=3cm,right=3cm,marginparwidth=1.75cm]{geometry}
\usepackage[english]{babel}
\usepackage[dvipdfm]{graphicx}
\usepackage[hypertex]{hyperref}
\title{
Dirty limit. Quasiclassical Green function $\theta$-parameterization
}
\author{Masaru Okada}

\date{\today}

\begin{document}

\maketitle

The Usadel equation in zero magnetic field is
\begin{eqnarray}
	i D \vec{\nabla} ( \check{g} \vec{\nabla} \check{g} )
	+ \check{H}_{0} \check{g} - \check{g} \check{H}_{0}
	&=&
	0
	,
\end{eqnarray}
where the quasiclassical Nambu-Green function $\check{g}$ and the non-perturbative Hamiltonian $\check{H}_{0}$ are respectively
\begin{eqnarray}
	\check{g}
	\ = \
	\left(
	\begin{array}{cc}
			g             & f  \\[2mm]
			- f^{\dagger} & -g
		\end{array}
	\right)
	,
	\hspace{10mm}
	\check{H}_{0}
	\ = \
	\left(
	\begin{array}{cc}
			- i \omega_{n} & - \Delta     \\[2mm]
			\Delta^{*}     & i \omega_{n}
		\end{array}
	\right)
	.
\end{eqnarray}
$g$ ($f$) is the (anomalous) Green function.
$\Delta$ is a constant superconducting gap.

Especially, in homogeneous state, Green functions can be written as
\begin{eqnarray}
	g
	\ = \
	- \dfrac{\omega_{n}}{\sqrt{\omega_{n}^{2} + |\Delta|^{2} }}
	,
	\hspace{10mm}
	f
	\ = \
	\dfrac{\Delta}{i \sqrt{\omega_{n}^{2} + |\Delta|^{2} }}
	.
\end{eqnarray}

Matsubara frequency $\omega_{n}$ is able to be extended (retarded) analytical continuation:
$i \omega_{n} \to E + i \eta$

where superconducting excitation energy $E = \sqrt{\varepsilon^{2} + |\Delta|^{2}} $ is real and $\eta$ is infinitesimal positive number.

$\check{g}$ satisfy the condition of 2-dimensional rotation matrix that $ \ {\rm Tr} \check{g} = 0 \ $ and $ \ {\rm det} \check{g} = 1 \ $.

Therefore $\check{g}$ can be parameterized by $\theta(x)$:
\begin{eqnarray}
	\check{g}(x)
	\ = \
	\left(
	\begin{array}{cc}
			{\rm cos} [\theta(x)]              & {\rm sin} [\theta(x)] e^{i \chi} \\[2mm]
			{\rm sin} [\theta(x)] e^{- i \chi} & -{\rm cos} [\theta(x)]
		\end{array}
	\right)
	.
\end{eqnarray}
In zero-field, $\chi$ is constant.
It can be put $\chi=0$.

The (2,1) element of the Usadel equation in the nomal metal has the form

\begin{eqnarray}
	-i D \dfrac{\partial}{\partial x}
	\Big\{ {\rm cos} [ \theta(x) ] \dfrac{\partial}{\partial x} \big( - {\rm sin} [ \theta(x)] \big)
	+ {\rm sin} [ \theta(x) ] \dfrac{\partial}{\partial x} {\rm cos} [ \theta(x)] \Big\}
	&=&
	2 i \omega_{n} {\rm sin} [ \theta(x) ]
	.
\end{eqnarray}

Further, there are useful relations

\begin{eqnarray}
	\dfrac{\partial}{\partial x} {\rm sin} [ \theta(x)]
	\ = \
	{\rm cos} [ \theta(x)] \dfrac{\partial \theta(x) }{\partial x}
	,
	\hspace{10mm}
	\dfrac{\partial}{\partial x} {\rm cos} [ \theta(x)]
	\ = \
	- {\rm sin} [ \theta(x)] \dfrac{\partial \theta(x) }{\partial x}
	.
\end{eqnarray}

After analytical continuation, the obtained form becomes much easier

\begin{eqnarray}
	D \dfrac{\partial^{2} \theta(x)}{\partial x^{2}}
	+
	2 i E {\rm sin} [ \theta(x) ]
	&=&
	0
	.
\end{eqnarray}

\end{document}
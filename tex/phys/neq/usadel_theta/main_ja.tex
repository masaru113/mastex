\documentclass[uplatex,a4j,12pt,dvipdfmx]{jsarticle}
\usepackage{amsmath,amsthm,amssymb,bm,color,enumitem,mathrsfs,url,epic,eepic,ascmac,ulem,here,ascmac}
\usepackage[letterpaper,top=2cm,bottom=2cm,left=3cm,right=3cm,marginparwidth=1.75cm]{geometry}
\usepackage[english]{babel}
\usepackage[dvipdfm]{graphicx}
\usepackage[hypertex]{hyperref}
\title{
Dirty limit. Quasiclassical Green function $\theta$-parameterization
}
\author{岡田 大(Okada Masaru)}

\date{\today}

\begin{document}

\maketitle

磁場が入っていないUsadel方程式は次式で与えられる。
\begin{eqnarray}
	i D \vec{\nabla} ( \check{g} \vec{\nabla} \check{g} )
	+ \check{H}_{0} \check{g} - \check{g} \check{H}_{0}
	&=&
	0
	,
\end{eqnarray}
ここで、準古典南部‐グリーン関数 $\check{g}$ と非摂動ハミルトニアン $\check{H}_{0}$ はそれぞれ
\begin{eqnarray}
	\check{g}
	\ = \
	\left(
	\begin{array}{cc}
			g             & f  \\[2mm]
			- f^{\dagger} & -g
		\end{array}
	\right)
	,
	\hspace{10mm}
	\check{H}_{0}
	\ = \
	\left(
	\begin{array}{cc}
			- i \omega_{n} & - \Delta     \\[2mm]
			\Delta^{*}     & i \omega_{n}
		\end{array}
	\right)
	.
\end{eqnarray}
$g$ ($f$) は (異常) グリーン関数、
$\Delta$ は超伝導ギャップで定数である。

特に、一様な状態では、グリーン関数は次のように書ける。
\begin{eqnarray}
	g
	\ = \
	- \dfrac{\omega_{n}}{\sqrt{\omega_{n}^{2} + |\Delta|^{2} }}
	,
	\hspace{10mm}
	f
	\ = \
	\dfrac{\Delta}{i \sqrt{\omega_{n}^{2} + |\Delta|^{2} }}
	.
\end{eqnarray}

松原周波数 $\omega_{n}$ は、(遅延) 解析接続によって次のように拡張できる:
$i \omega_{n} \to E + i \eta$

ここで、超伝導励起エネルギー $E = \sqrt{\varepsilon^{2} + |\Delta|^{2}} $ は実数であり、$\eta$ は無限小の正の数である。

$\check{g}$ は、 $ \ {\rm Tr} \check{g} = 0 \ $ かつ $ \ {\rm det} \check{g} = 1 \ $ という、2次元回転行列の条件を満たす。

したがって、$\check{g}$ は $\theta(x)$ を用いて次のようにパラメータ表示できる:
\begin{eqnarray}
	\check{g}(x)
	\ = \
	\left(
	\begin{array}{cc}
			{\rm cos} [\theta(x)]              & {\rm sin} [\theta(x)] e^{i \chi} \\[2mm]
			{\rm sin} [\theta(x)] e^{- i \chi} & -{\rm cos} [\theta(x)]
		\end{array}
	\right)
	.
\end{eqnarray}
ゼロ磁場では位相 $\chi$ は定数であり、
$\chi=0$ と取っても良い。

常伝導金属中でのUsadel方程式の (2,1) 成分は、以下の形をとる。

\begin{eqnarray}
	-i D \dfrac{\partial}{\partial x}
	\Big\{ {\rm cos} [ \theta(x) ] \dfrac{\partial}{\partial x} \big( - {\rm sin} [ \theta(x)] \big)
	+ {\rm sin} [ \theta(x) ] \dfrac{\partial}{\partial x} {\rm cos} [ \theta(x)] \Big\}
	&=&
	2 i \omega_{n} {\rm sin} [ \theta(x) ]
	.
\end{eqnarray}

さらに、以下の関係式を用いることで(これは合成関数を微分しただけ)。

\begin{eqnarray}
	\dfrac{\partial}{\partial x} {\rm sin} [ \theta(x)]
	\ = \
	{\rm cos} [ \theta(x)] \dfrac{\partial \theta(x) }{\partial x}
	,
	\hspace{10mm}
	\dfrac{\partial}{\partial x} {\rm cos} [ \theta(x)]
	\ = \
	- {\rm sin} [ \theta(x)] \dfrac{\partial \theta(x) }{\partial x}
	.
\end{eqnarray}

解析接続を行うと、得られる式はより簡単な形になる。

\begin{eqnarray}
	D \dfrac{\partial^{2} \theta(x)}{\partial x^{2}}
	+
	2 i E {\rm sin} [ \theta(x) ]
	&=&
	0
	.
\end{eqnarray}

\end{document}
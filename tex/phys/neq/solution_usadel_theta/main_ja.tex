\documentclass[uplatex,a4j,12pt,dvipdfmx]{jsarticle}
\usepackage{amsmath,amsthm,amssymb,bm,color,enumitem,mathrsfs,url,epic,eepic,ascmac,ulem,here,ascmac}
\usepackage[letterpaper,top=2cm,bottom=2cm,left=3cm,right=3cm,marginparwidth=1.75cm]{geometry}
\usepackage[english]{babel}
\usepackage[dvipdfm]{graphicx}
\usepackage[hypertex]{hyperref}
\title{金属・超伝導界面における \\ $\theta$ パラメータ表示されたUsadel方程式の解}

\author{
岡田 大 (Okada Masaru)
}

\begin{document}

\maketitle

\begin{abstract}
	超伝導・金属の界面における状態密度を調べる。
	空間座標 $x<0$ の領域で超伝導、$x>0$ の領域で金属であるような界面について考察する。
\end{abstract}

\section{$\theta$ パラメータ表示}

一様な状態において、
南部空間における準古典グリーン関数は
\begin{eqnarray}
	\check{g}_{\omega_{n}}
	&=&
	\left(
	\begin{array}{cc}
			g_{\omega_{n}}            & f_{\omega_{n}}
			\\[2mm]
			-f^{\dagger}_{\omega_{n}} & - g_{\omega_{n}}
		\end{array}
	\right)
	\nonumber \\[2mm] &=&
	\dfrac{1}{ \sqrt{ \omega_{n}^{2} + | \Delta |^{2} } }
	\left(
	\begin{array}{cc}
			\omega_{n}   & - i \Delta
			\\[2mm]
			i \Delta^{*} & - \omega_{n}
		\end{array}
	\right)
\end{eqnarray}
ここで $g_{\omega_{n}}$ を ${\rm cos} \theta$ で置き換えると、
$\theta = \theta(x) = {\rm arctan} \dfrac{\Delta(x)}{\omega_{n}}$
となり、
\begin{eqnarray}
	\check{g}
	&=&
	\left(
	\begin{array}{cc}
			{\rm cos} \theta        & - i \ \! {\rm sin} \theta
			\\[2mm]
			i \ \! {\rm sin} \theta & - {\rm cos} \theta
		\end{array}
	\right)
	\nonumber \\[3mm] &=&
	\check{\tau}_{3} {\rm cos} \theta
	+
	\check{\tau}_{2} {\rm sin} \theta
\end{eqnarray}
ただし $\check{\tau}$ はパウリ行列であり、
$\Delta$ は実数とした。
このとき、Usadel方程式は以下のようになる。
\begin{eqnarray}
	D
	\dfrac{\partial^{2} \theta}{\partial x^{2}}
	&=&
	2 \omega_{n} {\rm sin} \theta
\end{eqnarray}

\section{無限遠での境界条件}

Usadel方程式は次のように書き換えることができる。
\begin{eqnarray}
	\dfrac{\partial}{\partial x}
	\Big[
		\dfrac{D}{2}
		\Big( \dfrac{\partial \theta}{\partial x} \Big)^{2}
		+
		2 \omega_{n}
		{\rm cos} \theta
		\Big]
	&=&
	0
\end{eqnarray}
簡単のため、定数 $A$ を導入すると、
\begin{eqnarray}
	\dfrac{\partial \theta}{\partial x}
	&=&
	\pm
	\sqrt{
		\dfrac{2A}{D}
		-
		\dfrac{4 \omega_{n}}{D}
		{\rm cos} \theta
	}
\end{eqnarray}
\subsection{超伝導体極限}

$x \to - \infty$ の極限において、
この領域はバルクの超伝導状態であり、
一様な状態とみなすことができる。
したがって、境界条件は
\begin{eqnarray}
	\left\{
	\begin{array}{rcl}
		\dfrac{\partial \theta}{\partial x} \Big|_{x \to - \infty}
		 & = & 0
		\\[4mm]
		\displaystyle \lim_{x \to - \infty} \theta
		 & = &
		{\rm arctan} \dfrac{\Delta_{0}}{ \omega_{n} }
		\ \ = \ \
		\Theta_{\omega_{n}}
	\end{array}
	\right.
\end{eqnarray}
となる。ここで $\Theta_{\omega_{n}}$ を定義した。
これらの条件から、 $A$ の値が
$\displaystyle A=\dfrac{2 \omega_{n}^{2}}{ \sqrt{ \omega_{n}^{2} + \Delta_{0}^{2} } }$
と決まる。
表現を簡単にするため、
もう一度、定数
$B=\dfrac{2 \omega_{n}}{A} = \dfrac{ \sqrt{ \omega_{n}^{2} + \Delta_{0}^{2} }}{\omega_{n}}$
を再定義してみよう。
\begin{eqnarray}
	\dfrac{\partial \theta}{\partial x}
	&=&
	\pm
	\sqrt{ \dfrac{4 \omega_{n}}{DB} }
	\sqrt{ 1 - B {\rm cos} \theta }
	\ \ = \
	\pm
	\sqrt{ \dfrac{4 \omega_{n}}{DB} }
	\sqrt{ 1 - B (1 - 2 {\rm sin}^{2} \dfrac{\theta}{2} ) }
	\\[2mm]
	2
	\dfrac{\partial \theta}{\partial x}
	&=&
	\pm
	\sqrt{ \dfrac{4 \omega_{n}}{DB} }
	\sqrt{ 1- B }
	\sqrt{ 1 + \dfrac{2B}{1-B} {\rm sin}^{2} \theta }
\end{eqnarray}
これを明示的に書くと、
\begin{eqnarray}
	\dfrac{\partial \theta}{\partial x}
	&=&
	\pm
	\sqrt{ \dfrac{ \omega_{n} \big( \omega_{n} - \sqrt{\omega_{n}^{2} + \Delta_{0}^{2}} \big) }{ D \sqrt{\omega_{n}^{2} + \Delta_{0}^{2}} } }
	\sqrt{ 1 - \dfrac{ 2 \sqrt{\omega_{n}^{2} + \Delta_{0}^{2}} }{ \sqrt{\omega_{n}^{2} + \Delta_{0}^{2}} - \omega_{n} } {\rm sin}^{2} \theta }
\end{eqnarray}
したがって、この方程式は容易に解くことができ、
その解は積分定数 $C_{S}$ と
第一種不完全楕円積分
$\displaystyle F(\theta,k) = \int^{\theta}_{0} \dfrac{d \theta'}{ \sqrt{1 - k^{2} {\rm sin}^{2} \theta'}}$
を用いて次のように表される。
\begin{eqnarray}
	F \Big(\theta,\sqrt{ \dfrac{2}{ 1 - {\rm cos} \Theta_{\omega_{n}} } } \Big)
	&=&
	\pm
	\dfrac{x}{2}
	\sqrt{ \dfrac{ \omega_{n} }{ D } }
	\sqrt{ {\rm cos} \Theta_{\omega_{n}} - 1 }
	+
	C_{S}
	\hspace{10mm}
	(x<0)
\end{eqnarray}

\subsection{常伝導金属極限}

反対に、
$x \to + \infty$ の極限は、
常伝導金属 ($\Delta=0$) の領域に対応する。
\begin{eqnarray}
	\left\{
	\begin{array}{rcl}
		\dfrac{\partial \theta}{\partial x} \Big|_{x \to \infty}
		 & = & 0
		\\[4mm]
		\displaystyle \lim_{x \to \infty} \theta
		 & = &
		0
	\end{array}
	\right.
\end{eqnarray}
このとき、パラメータ $A$ は $A=2 \omega_{n}$ と決まる。
\begin{eqnarray}
	\dfrac{\partial \theta}{\partial x}
	&=&
	\pm
	2 \sqrt{ \dfrac{2 \omega_{n}}{D} } \Big| {\rm sin} \dfrac{\theta}{2} \Big|
\end{eqnarray}
これは初等的な計算の範囲で解くことができる。
\begin{eqnarray}
	{\rm sgn} \Big( {\rm sin} \dfrac{\theta}{2} \Big)
	\
	{\rm ln} \Big( {\rm tan} \dfrac{\theta}{4} \Big)
	&=&
	\pm
	2 x \sqrt{ \dfrac{2 \omega_{n}}{D} }
	+
	C_{N}
	\hspace{10mm}
	(x>0)
\end{eqnarray}
ここで、複素関数 ${\rm sgn}(z)$ は ${\rm sgn}(z) = \dfrac{z}{|z|}$ と定義され、
$C_{N}$ は未定の定数である。
\section{超伝導体・常伝導金属界面の条件}

超伝導体側 ($x<0$) と常伝導金属側 ($x>0$) から、
それぞれの $x$ 微分は
\begin{eqnarray}
	\left\{
	\begin{array}{clllc}
		\dfrac{\partial \theta_{S}}{\partial x}
		 & =                        &
		\pm 2
		\sqrt{
			\dfrac{\omega_{n}}{D_{S}}
		}
		\sqrt{
		{\rm cos} \Theta_{\omega_{n}}
		-
		{\rm cos} \theta_{S}
		}
		\hspace{10mm}
		 & (x<0 \ : \ {\rm super})  &
		\\[8mm]
		\dfrac{\partial \theta_{N}}{\partial x}
		 & =                        &
		\pm
		2 \sqrt{ \dfrac{2 \omega_{n}}{D_{N}} } \Big| {\rm sin} \dfrac{\theta_{N}}{2} \Big|
		\hspace{10mm}
		 & (x>0 \ : \ {\rm nomal} ) &
	\end{array}
	\right.
\end{eqnarray}
となる。
$x=0$ の界面において、
関数 $\theta$ は以下の境界条件を満たす。
\begin{eqnarray}
	\left\{
	\begin{array}{clc}
		\gamma_{0} \xi_{N}
		\dfrac{\partial \theta_{N}}{\partial x}
		\Big|_{x \to +0}
		 & = &
		\displaystyle
		\lim_{x \to \pm0}
		{\rm sin}(\theta_{S} - \theta_{N})
		\\[6mm]
		\gamma_{1} \xi_{N}
		\dfrac{\partial \theta_{N}}{\partial x}
		\Big|_{x \to +0}
		 & = &
		\xi_{S}
		\dfrac{\partial \theta_{S}}{\partial x}
		\Big|_{x \to -0}
	\end{array}
	\right.
\end{eqnarray}
$\gamma_{0,1}$ は近接効果のパラメータ、
$ \xi_{N,S} = \sqrt{\dfrac{D_{N,S}}{2 \pi \Delta_{0}}} $
である。
未知数である
$ \ \theta_{S0}=\displaystyle \lim_{x \to -0} \theta_{S} \ $
と
$ \ \theta_{N0}=\displaystyle \lim_{x \to +0} \theta_{N} \ $
は、以下の連立方程式を満たす。
\begin{eqnarray}
	\left\{
	\begin{array}{lcl}
		\gamma_{0}
		\sqrt{ \dfrac{\omega_{n}}{\Delta_{0}} } {\rm sin} \dfrac{\theta_{N0}}{2}
		 & = &
		{\rm sin}(\theta_{S0} - \theta_{N0})
		\\[6mm]
		\gamma_{1}
		{\rm sin} \dfrac{\theta_{N0}}{2}
		 & = &
		\sqrt{
		{\rm cos} \Theta_{\omega_{n}}
		-
		{\rm cos} \theta_{S0}
		}
	\end{array}
	\right.
\end{eqnarray}
ここで
$C_{N}$ は複素数であるため、位相因子
${\rm sgn} \Big( {\rm sin} \dfrac{\theta_{N0}}{2} \Big)$
は $C_{N}$ に繰り込まれ、
また $\gamma_{0,1}$ も 1 のオーダーの因子として繰り込まれている。
$ \gamma_{B} = \dfrac{R_{B}}{\rho_{N} \xi_{N}} $
,
$ \gamma = \dfrac{\rho_{S} \xi_{S}}{\rho_{N} \xi_{N}} $
\section{数値計算}

まず、
ソルバーに複素数 $\gamma_{0,1}$ を与え、方程式を解く。
$\theta_{N0}$ は、以下の超越方程式を解くことによって得られるはずである。
\begin{eqnarray}
	\theta_{N0}
	&=&
	{\rm arccos}
	\Big( {\rm cos} \Theta_{\omega_{n}} - \gamma_{1}^{2} {\rm sin}^{2} \dfrac{\theta_{N0}}{2} \Big)
	-
	{\rm arcsin}
	\Big( \gamma_{0} \sqrt{ \dfrac{\omega_{n}}{\Delta_{0}} } {\rm sin} \dfrac{\theta_{N0}}{2} \Big)
	\label{eqn:theta_N0}
\end{eqnarray}
この超越方程式が解けたなら、
次に $\theta_{S0}$ を解く。これは以下のように決定される。
\begin{eqnarray}
	\theta_{S0}
	&=&
	{\rm arccos}
	\Big( {\rm cos} \Theta_{\omega_{n}} - \gamma_{1}^{2} {\rm sin}^{2} \dfrac{\theta_{N0}}{2} \Big)
\end{eqnarray}
$C_{N}$ と $C_{S}$ も同時に定まる。
\begin{eqnarray}
	\left\{
	\begin{array}{lcl}
		C_{N}
		 & = &
		{\rm ln} \Big( {\rm tan} \dfrac{\theta_{N0}}{4} \Big)
		\\[6mm]
		C_{S}
		 & = &
		F \Big(\theta_{S0} , \sqrt{ \dfrac{2}{ 1 - {\rm cos} \Theta_{\omega_{n}} } } \Big)
		\ \ = \ \
		\displaystyle
		\theta_{S0}
		\int^{1}_{0} \dfrac{d x}{ \sqrt{1 - \dfrac{2 {\rm sin}^{2}(\theta_{S0} x) }{ 1 - {\rm cos} \Theta_{\omega_{n}} } }}
	\end{array}
	\right.
\end{eqnarray}
楕円積分の逆関数は、
ヤコビ (Jacobi) の楕円関数を用いて表すことができる。
ヤコビの振幅関数 $\ {\rm am}(z,m) \ $ は、次のように定義される。
\begin{eqnarray}
	F({\rm am}(x,k^{2}),k)
	&=&
	x
\end{eqnarray}
言い換えれば、次のように書ける。
\begin{eqnarray}
	x
	&=&
	\int^{{\rm am}(x,m)}_{0} \dfrac{d \theta}{ \sqrt{1 - m \ {\rm sin}^{2}x}}
\end{eqnarray}
そして、ヤコビの楕円関数 sn, cn, dn は、それぞれ次のように定義される。
\begin{eqnarray}
	\left\{
	\begin{array}{rcl}
		{\rm sn} (x,m)
		 & = &
		{\rm sin} [{\rm am}(x,m)]
		\\[4mm]
		{\rm cn} (x,m)
		 & = &
		{\rm cos} [{\rm am}(x,m)]
		\\[4mm]
		{\rm dn} (x,m)
		 & = &
		\sqrt{1 - m \ {\rm sin}^{2} [{\rm am}(x,m)]}
	\end{array}
	\right.
\end{eqnarray}
sn, cn, dn はランベルト (Lambert) 級数展開の形で書き直すことができる。

解を陽に書き下すために、
フェルミエネルギーから測った状態密度 (DOS) を
ヤコビの楕円関数を用いて表現することができる。
\begin{eqnarray}
	&&
	- {\rm Im}[{\rm cos}\theta_{\omega_{n}}(x)]
	\nonumber \\[3mm]
	&=&
	\left\{
	\begin{array}{ll}
		-{\rm Im} \Big[
			{\rm cn}
			\Big(
			\dfrac{x}{2}
			\sqrt{ \dfrac{ \omega_{n} }{ D } }
			\sqrt{ {\rm cos} \Theta_{\omega_{n}} - 1 }
			+
			C_{S}
			\ , \
			\dfrac{2}{ 1 - {\rm cos} \Theta_{\omega_{n}} }
			\Big)
			\Big]
		 &
		\hspace{10mm}
		(x<0 \ : \ {\rm super})
		\\[7mm]
		-{\rm Im} \Big(
		{\rm cos}
		\Big\{
		4
		{\rm arctan}
		\Big[
			{\rm exp}
			\Big(
			2x
			\sqrt{ \dfrac{2 \omega_{n} }{ D } }
			+
			C_{N}
			\Big)
			\Big]
		\Big\}
		\Big)
		 &
		\hspace{10mm}
		(x>0 \ : \ {\rm nomal})
	\end{array}
	\right.
\end{eqnarray}
特に、 $i \omega_{n} \to E + i 0^{+}$ という解析接続を行った後では、
ZEDOS ( $E=0$ での状態密度) を描画することができる。
\begin{eqnarray}
	- {\rm Im}[{\rm cos}\theta(x,E=0)]
	&=&
	\left\{
	\begin{array}{ll}
		-{\rm Im} \Big[
			{\rm cn}
			\Big(
			i x 0^{+}
			+
			C_{S}
			\ , \
			2
			\Big)
			\Big]
		 &
		\hspace{10mm}
		(x<0 \ : \ {\rm super})
		\\[7mm]
		-{\rm Im} \Big(
		{\rm cos}
		\Big\{
		4
		{\rm arctan}
		\Big[
			{\rm exp}
			\Big(
			x
			0^{+}
			+
			C_{N}
			\Big)
			\Big]
		\Big\}
		\Big)
		 &
		\hspace{10mm}
		(x>0 \ : \ {\rm nomal} )
	\end{array}
	\right.
	\hspace{10mm}
\end{eqnarray}

\section{ルンゲ=クッタを利用して常微分方程式のまま解く}

\subsection{問題設定}

解析接続されたUsadel方程式は次のようになる。

\begin{eqnarray}
	\pi
	\theta''(x,E)
	+
	i E {\rm sin} \theta(x,E)
	&=&
	0
\end{eqnarray}

ここでプライム($'$)は$\partial/\partial x$であり、
変数はそれぞれ規格化された距離とエネルギー$
	x/ \xi \to x$、$ E/\Delta_{0} \to E $
である。
ここで、
$\theta(x,E) = \theta_{r}(x,E) +i \theta_{i}(x,E)$
として、
複素関数をそのまま扱うことを避けて、実部と虚部に分ける。
このとき方程式は2つに分離される。

\begin{eqnarray}
	\left\{
	\begin{array}{rcl}
		\pi \theta_{r}''(x,E)
		-
		E {\rm cos} \theta_{r}(x,E) \ \! {\rm sinh} \theta_{i}(x,E)
		 & = &
		0
		\\[5mm]
		\pi \theta_{i}''(x,E)
		+
		E {\rm sin} \theta_{r}(x,E) \ \! {\rm cosh} \theta_{i}(x,E)
		 & = &
		0
	\end{array}
	\right.
\end{eqnarray}

この連立方程式は、物理的な考察
から漸近条件が次のように課される。

\begin{eqnarray}
	\left\{
	\begin{array}{rclll}
		\theta_{r}(\infty,E)
		                                     & = &
		0
		, \hspace{5mm} \theta_{i}(\infty,E)  &
		=                                    &
		0
		\\[4mm]
		\theta_{r}(-\infty,E)
		                                     & = &
		0
		, \hspace{5mm} \theta_{i}(-\infty,E) &
		=
		                                     &
		{\rm arctanh} \dfrac{1}{ E }
	\end{array}
	\right.
\end{eqnarray}

(超伝導体と常伝導体の)界面
($x=\pm0$)では境界条件として次が成り立つ。

\begin{eqnarray}
	&&
	\left\{
	\begin{array}{clc}
		\gamma
		\theta_{r}'(+0,E)
		 & = &
		\theta_{r}'(-0,E)
		\\[6mm]
		\gamma
		\theta_{i}'(+0,E)
		 & = &
		\theta_{i}'(-0,E)
	\end{array}
	\right.
	,
	\\[4mm]
	&&
	\left\{
	\begin{array}{clc}
		\gamma_{B}
		\theta_{r}'(+0,E)
		 & = &
		{\rm sin} \Big[ \theta_{r}(-0,E) - \theta_{r}(+0,E) \Big]
		{\rm cosh} \Big[ \theta_{i}(-0,E) - \theta_{i}(+0,E) \Big]
		\\[6mm]
		\gamma_{B}
		\theta_{i}'(+0,E)
		 & = &
		{\rm cos} \Big[ \theta_{r}(-0,E) - \theta_{r}(+0,E) \Big]
		{\rm sinh} \Big[ \theta_{i}(-0,E) - \theta_{i}(+0,E) \Big]
	\end{array}
	\right.
\end{eqnarray}

ここで$\gamma$,$\gamma_{B}$は近接効果を特徴付けるパラメータであり、問題に合わせて手で与える定数である。

%次のページでこの問題を解く方法の例を挙げる。
\subsection{戦略}

以下では簡単のためにエネルギーを固定して、関数の中の$E$を表示しない。

まず、古典的な4次のRunge-Kutta(RK4)を界面から十分離れた位置
\footnote{
	当然、$L \to \infty$で厳密であるが、
	RK4で逐次積分していくことを考えるので、
	途中に不安定な点が存在すると途端に解は信頼性を失ってしまう。
	今回、解くべき方程式は非線形な連立微分方程式であり、
	途中で不安定になることは簡単に予想できる。
	従って、$x=-L$から界面$x=-0$へたどり着くまで不安定な点が出ない範囲の大きさの$L$を用意する必要があり、
	この方程式を解く人が自分でチューニングして十分大きくてかつ不安定にならない$L$を決めるしかない。

}
$x=-L$
から界面に向かって走らせる。
ここで漸近条件から0階導関数の初期値は決定しているので、
1階導関数の値$\theta'(-L)$さえ決まれば、
$x=-0$までの関数の値は逐次決定される。
$x=-0$から$x=+0$へ接続するには界面における境界条件を解く必要がある。
$x=-0$における値が既知であるとして、
$\theta_{r}(+0)$に関する次のような1元の方程式へ還元される。
簡単のために
$t=\theta_{r}(-0) - \theta_{r}(+0)$と置換すると、

\begin{eqnarray}
	{\rm tan}
	(t)
	&=&
	a
	\
	{\rm tanh}
	\left\{
	{\rm arccosh}
	\left[
		\dfrac{ b }{ {\rm sin} (t) }
		\right]
	\right\}
\end{eqnarray}

ここで表記の簡単のために
それぞれ
$
	a
	\ = \
	\dfrac{ \theta_{r}'(-0) }{ \theta_{i}'(-0) }
$
、
$
	b
	\ = \
	\dfrac{ \gamma_{B} }{ \gamma }
	\theta_{i}'(-0)
$
と置いた。
この方程式の解は次で与えられる。

\begin{eqnarray}
	\theta_{r}(+0)
	&=&
	\theta_{r}(-0)
	+
	{\rm arcsin}
	\sqrt{
		\dfrac{ 1 }{ 2 }
		\left(
		\dfrac{ b^{2} }{ a^{2} }
		-
		\dfrac{ \sqrt{ a^{4} b^{4} - 2 a^{4} b^{2} + a^{4} + 2 a^{2} b^{4} + 2 a^{2} b^{2} + b^{4} } }{ a^{2} }
		+
		b^{2} + 1
		\right)
	}
	\nonumber \\
\end{eqnarray}

この解より、境界条件からすぐに得られる次の関係式
\footnote{
	逆双曲線関数は、実関数の範囲内で全く同値な式
	%
	%
	%
	%
	

	\begin{eqnarray}
		{\rm arcsinh}(y)
		&=&
		{\rm ln} \left( y + \sqrt{ 1 + y^{2} } \right)
		\\
		{\rm arctanh}\dfrac{1}{E}
		\ = \
		{\rm arccoth}(E)
		&=&
		\dfrac{1}{2}
		{\rm ln} \dfrac{1+E}{1-E}
	\end{eqnarray}
	

	%
	%
	%
	%
	を数値計算では用いる。
}

\begin{eqnarray}
	\theta_{i}(+0)
	&=&
	\theta_{i}(-0)
	-
	{\rm arcsinh}
	\left\{
	\dfrac{ \gamma_{B} }{ \gamma }
	\dfrac{ \theta_{i}'(-0) }{ {\rm cos} \left[ \theta_{r}(-0) - \theta_{r}(+0) \right] }
	\right\}
	,
\end{eqnarray}

から$\theta_{i}(+0)$も同時に求めることができる。

さらに境界条件から$\theta'(+0)$の値も分かるので、
$x=+0$での関数の値は全て得られる。
この値を初期値として
RK4を走らせることで、
$x>0$の場合も全く同様の流れで
$x=+L$の値まで求まる。

${}$

未知の定数$c_{1}$、$c_{2}$を用いて整理する。
初期値$\theta'(-L)=c_{1}+i c_{2}$を手で与えてRK4で$x=-0$まで走らせる。
境界条件が陽に解けたので、$x=-0$での値をそのまま代入するだけで$x=+0$での値が求まる。
$x=+0$での値を初期値として、
ここからRK4で出発して$\theta(L)$を得る。
この$\theta(L)$が最小
\footnote{厳密には$\ \displaystyle \lim_{L \to \infty} f(c_{1},c_{2})=0$である。}
になるような
$(c_{1},c_{2})$を探す。

このときの$\theta(+L)$は、$x=-L$における初期値$(c_{1},c_{2})$が決まれば定まるものとして、それを関数$f$とみなせば

\begin{eqnarray}
	\big| \theta_{r}(L) + i \theta_{i}(L) \big|^{2}
	&=&
	f(c_{1},c_{2})
\end{eqnarray}

の最小化問題である。
\subsection{手順}

$\hspace{-5mm} \cdot$
まずパラメータ$\gamma$、$\gamma_{B}$を入力する。

$\hspace{-5mm} \cdot$
初期値$(c_{1},c_{2})$を振って、$x=-L$から界面$x=-0$に向かってRK4を解く。

$\hspace{-5mm} \cdot$
境界条件を満たすように
$x=+0$での値を求める。

$\hspace{-5mm} \cdot$
$x=+0$での値を初期値として$x=L$に向かってRK4を回す。

$\hspace{-5mm} \cdot$
得られた$\theta(L)$から$\big| \theta(L) \big|^{2}=f(c_{1},c_{2})$
を各$(c_{1},c_{2})$の場合で計算する。

$\hspace{-5mm} \cdot$
この$f(c_{1},c_{2})$が最小になるような初期値$(c_{1},c_{2})$を用いて計算されたものが解である。

$\hspace{-5mm} \cdot$
これを各$E$について繰り返す。

\end{document}
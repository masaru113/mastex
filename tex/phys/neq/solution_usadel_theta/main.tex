\documentclass[uplatex,a4j,12pt,dvipdfmx]{jsarticle}
\usepackage{amsmath,amsthm,amssymb,bm,color,enumitem,mathrsfs,url,epic,eepic,ascmac,ulem,here,ascmac}
\usepackage[letterpaper,top=2cm,bottom=2cm,left=3cm,right=3cm,marginparwidth=1.75cm]{geometry}
\usepackage[english]{babel}
\usepackage[dvipdfm]{graphicx}
\usepackage[hypertex]{hyperref}
\title{Solution of the $\theta$-parametrized Usadel Equation \\ at a Metal-Superconductor Interface}

\author{
Masaru Okada
}

\begin{document}

\maketitle

\begin{abstract}
	The density of states at a superconductor-metal interface is investigated.
	We consider an interface where the region $x<0$ is superconducting and the region $x>0$ is metallic.
\end{abstract}

\section{$\theta$-parametrization}

In a homogeneous state,
the quasi-classical Green's function in Nambu space is
\begin{eqnarray}
	\check{g}_{\omega_{n}}
	&=&
	\left(
	\begin{array}{cc}
			g_{\omega_{n}}            & f_{\omega_{n}}
			\\[2mm]
			-f^{\dagger}_{\omega_{n}} & - g_{\omega_{n}}
		\end{array}
	\right)
	\nonumber \\[2mm] &=&
	\dfrac{1}{ \sqrt{ \omega_{n}^{2} + | \Delta |^{2} } }
	\left(
	\begin{array}{cc}
			\omega_{n}   & - i \Delta
			\\[2mm]
			i \Delta^{*} & - \omega_{n}
		\end{array}
	\right)
\end{eqnarray}
If $g_{\omega_{n}}$ is replaced by ${\rm cos} \theta$,
then $\theta = \theta(x) = {\rm arctan} \dfrac{\Delta(x)}{\omega_{n}}$,
and thus
\begin{eqnarray}
	\check{g}
	&=&
	\left(
	\begin{array}{cc}
			{\rm cos} \theta        & - i \ \! {\rm sin} \theta
			\\[2mm]
			i \ \! {\rm sin} \theta & - {\rm cos} \theta
		\end{array}
	\right)
	\nonumber \\[3mm] &=&
	\check{\tau}_{3} {\rm cos} \theta
	+
	\check{\tau}_{2} {\rm sin} \theta
\end{eqnarray}
where $\check{\tau}$ are the Pauli matrices
and $\Delta$ is taken to be real.
The Usadel equation then becomes
\begin{eqnarray}
	D
	\dfrac{\partial^{2} \theta}{\partial x^{2}}
	&=&
	2 \omega_{n} {\rm sin} \theta
\end{eqnarray}

\section{Boundary condition at infinity}

The Usadel equation can be rewritten as
\begin{eqnarray}
	\dfrac{\partial}{\partial x}
	\Big[
		\dfrac{D}{2}
		\Big( \dfrac{\partial \theta}{\partial x} \Big)^{2}
		+
		2 \omega_{n}
		{\rm cos} \theta
		\Big]
	&=&
	0
\end{eqnarray}
For simplicity, introducing a constant $A$ yields
\begin{eqnarray}
	\dfrac{\partial \theta}{\partial x}
	&=&
	\pm
	\sqrt{
		\dfrac{2A}{D}
		-
		\dfrac{4 \omega_{n}}{D}
		{\rm cos} \theta
	}
\end{eqnarray}
\subsection{Superconductor limit.}

In the limit $x \to - \infty$,
this domain is the bulk superconductor,
which can be regarded as a homogeneous state.
The boundary conditions are therefore
\begin{eqnarray}
	\left\{
	\begin{array}{rcl}
		\dfrac{\partial \theta}{\partial x} \Big|_{x \to - \infty}
		 & = & 0
		\\[4mm]
		\displaystyle \lim_{x \to - \infty} \theta
		 & = &
		{\rm arctan} \dfrac{\Delta_{0}}{ \omega_{n} }
		\ \ = \ \
		\Theta_{\omega_{n}}
	\end{array}
	\right.
\end{eqnarray}
where $\Theta_{\omega_{n}}$ has now been defined.
These conditions determine the value
$\displaystyle A=\dfrac{2 \omega_{n}^{2}}{ \sqrt{ \omega_{n}^{2} + \Delta_{0}^{2} } }$.
To simplify the expression,
let us once again re-define a constant
$B=\dfrac{2 \omega_{n}}{A} = \dfrac{ \sqrt{ \omega_{n}^{2} + \Delta_{0}^{2} }}{\omega_{n}}$.
\begin{eqnarray}
	\dfrac{\partial \theta}{\partial x}
	&=&
	\pm
	\sqrt{ \dfrac{4 \omega_{n}}{DB} }
	\sqrt{ 1 - B {\rm cos} \theta }
	\ \ = \
	\pm
	\sqrt{ \dfrac{4 \omega_{n}}{DB} }
	\sqrt{ 1 - B (1 - 2 {\rm sin}^{2} \dfrac{\theta}{2} ) }
	\\[2mm]
	2
	\dfrac{\partial \theta}{\partial x}
	&=&
	\pm
	\sqrt{ \dfrac{4 \omega_{n}}{DB} }
	\sqrt{ 1- B }
	\sqrt{ 1 + \dfrac{2B}{1-B} {\rm sin}^{2} \theta }
\end{eqnarray}
Explicitly, we have
\begin{eqnarray}
	\dfrac{\partial \theta}{\partial x}
	&=&
	\pm
	\sqrt{ \dfrac{ \omega_{n} \big( \omega_{n} - \sqrt{\omega_{n}^{2} + \Delta_{0}^{2}} \big) }{ D \sqrt{\omega_{n}^{2} + \Delta_{0}^{2}} } }
	\sqrt{ 1 - \dfrac{ 2 \sqrt{\omega_{n}^{2} + \Delta_{0}^{2}} }{ \sqrt{\omega_{n}^{2} + \Delta_{0}^{2}} - \omega_{n} } {\rm sin}^{2} \theta }
\end{eqnarray}
Therefore, it can be solved easily
and the solution can be represented using
an integration constant $C_{S}$ and
the incomplete elliptic integral of the first kind
$\displaystyle F(\theta,k) = \int^{\theta}_{0} \dfrac{d \theta'}{ \sqrt{1 - k^{2} {\rm sin}^{2} \theta'}}$ :
\begin{eqnarray}
	F \Big(\theta,\sqrt{ \dfrac{2}{ 1 - {\rm cos} \Theta_{\omega_{n}} } } \Big)
	&=&
	\pm
	\dfrac{x}{2}
	\sqrt{ \dfrac{ \omega_{n} }{ D } }
	\sqrt{ {\rm cos} \Theta_{\omega_{n}} - 1 }
	+
	C_{S}
	\hspace{10mm}
	(x<0)
\end{eqnarray}

\subsection{Normal metal limit.}

On the other hand,
the limit $x \to + \infty$ corresponds to
the normal metal ($\Delta=0$) region.
\begin{eqnarray}
	\left\{
	\begin{array}{rcl}
		\dfrac{\partial \theta}{\partial x} \Big|_{x \to \infty}
		 & = & 0
		\\[4mm]
		\displaystyle \lim_{x \to \infty} \theta
		 & = &
		0
	\end{array}
	\right.
\end{eqnarray}
In this case, the parameter $A$ is determined to be $A=2 \omega_{n}$.
\begin{eqnarray}
	\dfrac{\partial \theta}{\partial x}
	&=&
	\pm
	2 \sqrt{ \dfrac{2 \omega_{n}}{D} } \Big| {\rm sin} \dfrac{\theta}{2} \Big|
\end{eqnarray}
This can be solved within the regime of elementary calculation.
\begin{eqnarray}
	{\rm sgn} \Big( {\rm sin} \dfrac{\theta}{2} \Big)
	\
	{\rm ln} \Big( {\rm tan} \dfrac{\theta}{4} \Big)
	&=&
	\pm
	2 x \sqrt{ \dfrac{2 \omega_{n}}{D} }
	+
	C_{N}
	\hspace{10mm}
	(x>0)
\end{eqnarray}
Here, the complex function ${\rm sgn}(z)$ is defined as ${\rm sgn}(z) = \dfrac{z}{|z|}$
and $C_{N}$ is an unknown constant.
\section{Interface condition between superconductor and normal metal}

From the superconductor side ($x<0$) and the normal metal side ($x>0$),
the respective $x$-derivatives are
\begin{eqnarray}
	\left\{
	\begin{array}{clllc}
		\dfrac{\partial \theta_{S}}{\partial x}
		 & =                         &
		\pm 2
		\sqrt{
			\dfrac{\omega_{n}}{D_{S}}
		}
		\sqrt{
		{\rm cos} \Theta_{\omega_{n}}
		-
		{\rm cos} \theta_{S}
		}
		\hspace{10mm}
		 & (x<0 \ : \ {\rm super})   &
		\\[8mm]
		\dfrac{\partial \theta_{N}}{\partial x}
		 & =                         &
		\pm
		2 \sqrt{ \dfrac{2 \omega_{n}}{D_{N}} } \Big| {\rm sin} \dfrac{\theta_{N}}{2} \Big|
		\hspace{10mm}
		 & (x>0 \ : \ {\rm normal} ) &
	\end{array}
	\right.
\end{eqnarray}
At the interface at $x=0$,
the function $\theta$ satisfies the following boundary conditions:
\begin{eqnarray}
	\left\{
	\begin{array}{clc}
		\gamma_{0} \xi_{N}
		\dfrac{\partial \theta_{N}}{\partial x}
		\Big|_{x \to +0}
		 & = &
		\displaystyle
		\lim_{x \to \pm0}
		{\rm sin}(\theta_{S} - \theta_{N})
		\\[6mm]
		\gamma_{1} \xi_{N}
		\dfrac{\partial \theta_{N}}{\partial x}
		\Big|_{x \to +0}
		 & = &
		\xi_{S}
		\dfrac{\partial \theta_{S}}{\partial x}
		\Big|_{x \to -0}
	\end{array}
	\right.
\end{eqnarray}
$\gamma_{0,1}$ are the proximity effect parameters,
and
$ \xi_{N,S} = \sqrt{\dfrac{D_{N,S}}{2 \pi \Delta_{0}}} $
.
The unknown quantities
$ \ \theta_{S0}=\displaystyle \lim_{x \to -0} \theta_{S} \ $
and
$ \ \theta_{N0}=\displaystyle \lim_{x \to +0} \theta_{N} \ $
satisfy
the simultaneous equations
\begin{eqnarray}
	\left\{
	\begin{array}{lcl}
		\gamma_{0}
		\sqrt{ \dfrac{\omega_{n}}{\Delta_{0}} } {\rm sin} \dfrac{\theta_{N0}}{2}
		 & = &
		{\rm sin}(\theta_{S0} - \theta_{N0})
		\\[6mm]
		\gamma_{1}
		{\rm sin} \dfrac{\theta_{N0}}{2}
		 & = &
		\sqrt{
		{\rm cos} \Theta_{\omega_{n}}
		-
		{\rm cos} \theta_{S0}
		}
	\end{array}
	\right.
\end{eqnarray}
where,
since $C_{N}$ is complex, the phase factor
${\rm sgn} \Big( {\rm sin} \dfrac{\theta_{N0}}{2} \Big)$
is renormalized into $C_{N}$,
and
$\gamma_{0,1}$ are also renormalized by factors of order unity.
$ \gamma_{B} = \dfrac{R_{B}}{\rho_{N} \xi_{N}} $
,
$ \gamma = \dfrac{\rho_{S} \xi_{S}}{\rho_{N} \xi_{N}} $
\section{Numerical calculation}

First,
the solver is given the complex numbers $\gamma_{0,1}$ for the equation.
$\theta_{N0}$ should be obtainable by solving this transcendental equation:
\begin{eqnarray}
	\theta_{N0}
	&=&
	{\rm arccos}
	\Big( {\rm cos} \Theta_{\omega_{n}} - \gamma_{1}^{2} {\rm sin}^{2} \dfrac{\theta_{N0}}{2} \Big)
	-
	{\rm arcsin}
	\Big( \gamma_{0} \sqrt{ \dfrac{\omega_{n}}{\Delta_{0}} } {\rm sin} \dfrac{\theta_{N0}}{2} \Big)
	\label{eqn:theta_N0}
\end{eqnarray}
If the transcendental equation can be solved,
next $\theta_{S0}$ should be solved, which is determined as follows:
\begin{eqnarray}
	\theta_{S0}
	&=&
	{\rm arccos}
	\Big( {\rm cos} \Theta_{\omega_{n}} - \gamma_{1}^{2} {\rm sin}^{2} \dfrac{\theta_{N0}}{2} \Big)
\end{eqnarray}
$C_{N}$ and $C_{S}$ will also be fixed at the same time.
\begin{eqnarray}
	\left\{
	\begin{array}{lcl}
		C_{N}
		 & = &
		{\rm ln} \Big( {\rm tan} \dfrac{\theta_{N0}}{4} \Big)
		\\[6mm]
		C_{S}
		 & = &
		F \Big(\theta_{S0} , \sqrt{ \dfrac{2}{ 1 - {\rm cos} \Theta_{\omega_{n}} } } \Big)
		\ \ = \ \
		\displaystyle
		\theta_{S0}
		\int^{1}_{0} \dfrac{d x}{ \sqrt{1 - \dfrac{2 {\rm sin}^{2}(\theta_{S0} x) }{ 1 - {\rm cos} \Theta_{\omega_{n}} } }}
	\end{array}
	\right.
\end{eqnarray}
The inverse function of the elliptic integral can be represented
using Jacobi's elliptic functions.
Jacobi's amplitude function $\ {\rm am}(z,m) \ $ is defined as
\begin{eqnarray}
	F({\rm am}(x,k^{2}),k)
	&=&
	x
\end{eqnarray}
In other words, it can be written as
\begin{eqnarray}
	x
	&=&
	\int^{{\rm am}(x,m)}_{0} \dfrac{d \theta}{ \sqrt{1 - m \ {\rm sin}^{2}x}}
\end{eqnarray}
Then Jacobi's elliptic functions sn, cn, and dn are defined respectively as,
\begin{eqnarray}
	\left\{
	\begin{array}{rcl}
		{\rm sn} (x,m)
		 & = &
		{\rm sin} [{\rm am}(x,m)]
		\\[4mm]
		{\rm cn} (x,m)
		 & = &
		{\rm cos} [{\rm am}(x,m)]
		\\[4mm]
		{\rm dn} (x,m)
		 & = &
		\sqrt{1 - m \ {\rm sin}^{2} [{\rm am}(x,m)]}
	\end{array}
	\right.
\end{eqnarray}
sn, cn, and dn can be rewritten in Lambert series expansion forms.

To be written explicitly,
it is possible to use Jacobi's elliptic functions to express
the density of states (DOS) measured from the Fermi energy.
\begin{eqnarray}
	&&
	- {\rm Im}[{\rm cos}\theta_{\omega_{n}}(x)]
	\nonumber \\[3mm]
	&=&
	\left\{
	\begin{array}{ll}
		-{\rm Im} \Big[
			{\rm cn}
			\Big(
			\dfrac{x}{2}
			\sqrt{ \dfrac{ \omega_{n} }{ D } }
			\sqrt{ {\rm cos} \Theta_{\omega_{n}} - 1 }
			+
			C_{S}
			\ , \
			\dfrac{2}{ 1 - {\rm cos} \Theta_{\omega_{n}} }
			\Big)
			\Big]
		 &
		\hspace{10mm}
		(x<0 \ : \ {\rm super})
		\\[7mm]
		-{\rm Im} \Big(
		{\rm cos}
		\Big\{
		4
		{\rm arctan}
		\Big[
			{\rm exp}
			\Big(
			2x
			\sqrt{ \dfrac{2 \omega_{n} }{ D } }
			+
			C_{N}
			\Big)
			\Big]
		\Big\}
		\Big)
		 &
		\hspace{10mm}
		(x>0 \ : \ {\rm normal})
	\end{array}
	\right.
\end{eqnarray}
Especially, after analytical continuation $i \omega_{n} \to E + i 0^{+}$,
the ZEDOS (DOS at $E=0$) can be depicted
\begin{eqnarray}
	- {\rm Im}[{\rm cos}\theta(x,E=0)]
	&=&
	\left\{
	\begin{array}{ll}
		-{\rm Im} \Big[
			{\rm cn}
			\Big(
			i x 0^{+}
			+
			C_{S}
			\ , \
			2
			\Big)
			\Big]
		 &
		\hspace{10mm}
		(x<0 \ : \ {\rm super})
		\\[7mm]
		-{\rm Im} \Big(
		{\rm cos}
		\Big\{
		4
		{\rm arctan}
		\Big[
			{\rm exp}
			\Big(
			x
			0^{+}
			+
			C_{N}
			\Big)
			\Big]
		\Big\}
		\Big)
		 &
		\hspace{10mm}
		(x>0 \ : \ {\rm normal} )
	\end{array}
	\right.
	\hspace{10mm}
\end{eqnarray}

\section{Solving the ODE directly using the Runge-Kutta method}

\subsection{Problem setup}

The analytically continued Usadel equation is as follows:

\begin{eqnarray}
	\pi
	\theta''(x,E)
	+
	i E {\rm sin} \theta(x,E)
	&=&
	0
\end{eqnarray}

Here, the prime ($'$) denotes $\partial/\partial x$,
and the variables are the normalized distance and energy,
$x/ \xi \to x$ and $ E/\Delta_{0} \to E $, respectively.
By setting
$\theta(x,E) = \theta_{r}(x,E) +i \theta_{i}(x,E)$,
we avoid handling the complex function directly and separate it into real and imaginary parts.
The equation then separates into two:

\begin{eqnarray}
	\left\{
	\begin{array}{rcl}
		\pi \theta_{r}''(x,E)
		-
		E {\rm cos} \theta_{r}(x,E) \ \! {\rm sinh} \theta_{i}(x,E)
		 & = &
		0
		\\[5mm]
		\pi \theta_{i}''(x,E)
		+
		E {\rm sin} \theta_{r}(x,E) \ \! {\rm cosh} \theta_{i}(x,E)
		 & = &
		0
	\end{array}
	\right.
\end{eqnarray}

From physical considerations, the following asymptotic conditions
are imposed on this system of equations:

\begin{eqnarray}
	\left\{
	\begin{array}{rclll}
		\theta_{r}(\infty,E)
		                                     & = &
		0
		, \hspace{5mm} \theta_{i}(\infty,E)  &
		=                                    &
		0
		\\[4mm]
		\theta_{r}(-\infty,E)
		                                     & = &
		0
		, \hspace{5mm} \theta_{i}(-\infty,E) &
		=
		                                     &
		{\rm arctanh} \dfrac{1}{ E }
	\end{array}
	\right.
\end{eqnarray}

At the interface (between the superconductor and normal metal)
($x=\pm0$), the following boundary conditions hold:

\begin{eqnarray}
	&&
	\left\{
	\begin{array}{clc}
		\gamma
		\theta_{r}'(+0,E)
		 & = &
		\theta_{r}'(-0,E)
		\\[6mm]
		\gamma
		\theta_{i}'(+0,E)
		 & = &
		\theta_{i}'(-0,E)
	\end{array}
	\right.
	,
	\\[4mm]
	&&
	\left\{
	\begin{array}{clc}
		\gamma_{B}
		\theta_{r}'(+0,E)
		 & = &
		{\rm sin} \Big[ \theta_{r}(-0,E) - \theta_{r}(+0,E) \Big]
		{\rm cosh} \Big[ \theta_{i}(-0,E) - \theta_{i}(+0,E) \Big]
		\\[6mm]
		\gamma_{B}
		\theta_{i}'(+0,E)
		 & = &
		{\rm cos} \Big[ \theta_{r}(-0,E) - \theta_{r}(+0,E) \Big]
		{\rm sinh} \Big[ \theta_{i}(-0,E) - \theta_{i}(+0,E) \Big]
	\end{array}
	\right.
\end{eqnarray}

Here, $\gamma$ and $\gamma_{B}$ are parameters characterizing the proximity effect,
which are constants manually supplied for the problem.


\subsection{Strategy}

For simplicity in the following discussion, the energy $E$ is fixed
and will be omitted from the function arguments.

First, the classical 4th-order Runge-Kutta (RK4) method is run
from a position sufficiently far from the interface,
\footnote{
	Naturally, the solution is exact as $L \to \infty$.
	However, since we are integrating sequentially using RK4,
	the solution will immediately lose reliability if an unstable point exists in the path.
	The equation to be solved here is a system of nonlinear differential equations,
	and instability can be easily anticipated.
	Therefore, it is necessary to prepare an $L$ large enough to be in the asymptotic region, yet not so large that it
	introduces instability before reaching the interface $x=-0$ from $x=-L$.
	The person solving this equation must manually tune $L$ to find a value
	that is sufficiently large but does not cause instability.
}
$x=-L$,
towards the interface.
The initial value of the 0th-order derivative (the function value) is determined by the asymptotic conditions.
Thus, if the 1st-order derivative $\theta'(-L)$ is determined,
the function's values up to $x=-0$ are sequentially determined.
To connect from $x=-0$ to $x=+0$, the boundary conditions at the interface must be solved.
Assuming the values at $x=-0$ are known,
the problem reduces to a single-variable equation for $\theta_{r}(+0)$.
For simplicity, substituting
$t=\theta_{r}(-0) - \theta_{r}(+0)$ gives:

\begin{eqnarray}
	{\rm tan}
	(t)
	&=&
	a
	\
	{\rm tanh}
	\left\{
	{\rm arccosh}
	\left[
		\dfrac{ b }{ {\rm sin} (t) }
		\right]
	\right\}
\end{eqnarray}

Here, for notational simplicity, we have set
$
	a
	\ = \
	\dfrac{ \theta_{r}'(-0) }{ \theta_{i}'(-0) }
$
and
$
	b
	\ = \
	\dfrac{ \gamma_{B} }{ \gamma }
	\theta_{i}'(-0)
$.
The solution to this equation is given by:

\begin{eqnarray}
	\theta_{r}(+0)
	&=&
	\theta_{r}(-0)
	+
	{\rm arcsin}
	\sqrt{
		\dfrac{ 1 }{ 2 }
		\left(
		\dfrac{ b^{2} }{ a^{2} }
		-
		\dfrac{ \sqrt{ a^{4} b^{4} - 2 a^{4} b^{2} + a^{4} + 2 a^{2} b^{4} + 2 a^{2} b^{2} + b^{4} } }{ a^{2} }
		+
		b^{2} + 1
		\right)
	}
	\nonumber \\
\end{eqnarray}

From this solution, $\theta_{i}(+0)$ can also be determined simultaneously
from the following relation, which is immediately obtained from the boundary conditions
\footnote{
	For numerical calculations, the following equivalent expressions for inverse hyperbolic functions
	within the real domain are used:



	\begin{eqnarray}
		{\rm arcsinh}(y)
		&=&
		{\rm ln} \left( y + \sqrt{ 1 + y^{2} } \right)
		\\
		{\rm arctanh}\dfrac{1}{E}
		\ = \
		{\rm arccoth}(E)
		&=&
		\dfrac{1}{2}
		{\rm ln} \dfrac{1+E}{1-E}
	\end{eqnarray}


}
:

\begin{eqnarray}
	\theta_{i}(+0)
	&=&
	\theta_{i}(-0)
	-
	{\rm arcsinh}
	\left\{
	\dfrac{ \gamma_{B} }{ \gamma }
	\dfrac{ \theta_{i}'(-0) }{ {\rm cos} \left[ \theta_{r}(-0) - \theta_{r}(+0) \right] }
	\right\}
	,
\end{eqnarray}

Furthermore, the value of $\theta'(+0)$ is also known from the boundary conditions,
meaning all function values at $x=+0$ are obtained.
By running RK4 using these values as the initial conditions,
the values for $x>0$ can be found up to $x=+L$
in exactly the same manner.

${}$

We can summarize this using unknown constants $c_{1}$ and $c_{2}$.
Provide an initial value $\theta'(-L)=c_{1}+i c_{2}$ manually and run RK4 up to $x=-0$.
Since the boundary conditions have been solved explicitly,
the values at $x=+0$ are found simply by substituting the values from $x=-0$.
Using the values at $x=+0$ as initial conditions,
run RK4 from here to obtain $\theta(L)$.
We then search for the $(c_{1},c_{2})$ that minimizes
this $\theta(L)$
\footnote{Strictly, $\ \displaystyle \lim_{L \to \infty} f(c_{1},c_{2})=0$.}.

This $\theta(+L)$ is determined by the initial values $(c_{1},c_{2})$ at $x=-L$.
Viewing this as a function $f$,

\begin{eqnarray}
	\big| \theta_{r}(L) + i \theta_{i}(L) \big|^{2}
	&=&
	f(c_{1},c_{2})
\end{eqnarray}

it becomes a minimization problem.
\subsection{Procedure}

$\hspace{-5mm} \cdot$
First, input the parameters $\gamma$ and $\gamma_{B}$.

$\hspace{-5mm} \cdot$
%Inside the E loop:
Sweep the initial values $(c_{1},c_{2})$ and solve the RK4
from $x=-L$ toward the interface $x=-0$.

$\hspace{-5mm} \cdot$
Determine the values at $x=+0$ that satisfy the boundary conditions.

$\hspace{-5mm} \cdot$
Using the values at $x=+0$ as initial conditions, run RK4 toward $x=L$.

$\hspace{-5mm} \cdot$
Calculate $f(c_{1},c_{2}) = \big| \theta(L) \big|^{2}$
from the resulting $\theta(L)$ for each pair of $(c_{1},c_{2})$.

$\hspace{-5mm} \cdot$
The solution is the one calculated using the initial values $(c_{1},c_{2})$
that minimize this $f(c_{1},c_{2})$.

$\hspace{-5mm} \cdot$
Repeat this process for each $E$.

\end{document}
\documentclass[uplatex,a4j,12pt,dvipdfmx]{jsarticle}
\usepackage[english]{babel}
\usepackage[letterpaper,top=2cm,bottom=2cm,left=3cm,right=3cm,marginparwidth=1.75cm]{geometry}
\usepackage{amsmath,amsthm,amssymb,bm,color,mathrsfs,url}
\usepackage{epic,eepic,here}
\usepackage[dvipdfm]{graphicx}
\usepackage[hypertex]{hyperref}
\title{
Keldysh Green関数
}
\author{岡田 大 (Okada Masaru)}
\date{\today}
\def\sla#1{\rlap/#1}
\begin{document}
\maketitle

\begin{abstract}
	非平衡な系を扱うときに有用なKeldysh Green関数のメモ。
\end{abstract}

\ \\

Keldysh Green関数を次で定義する。

\begin{eqnarray}
	G_{\alpha \beta}(1,2)
	&=&
	i \Big\langle \hat{T}_{c} \Big[ \tilde{\psi}_{\alpha}(\vec{r}_{1},t_{1}) \tilde{\psi}_{\beta}^{\dagger}(\vec{r}_{2},t_{2}) \Big] \Big\rangle_{\rm st}
\end{eqnarray}

(符号はKopninに従った。AGDとは逆になっている。)
ここで用いた記法をそれぞれ説明する。
まず統計平均$\langle \cdots \rangle_{\rm st}$は次のように定義されている。

\begin{eqnarray}
	{\rm Tr}
	\left[ {\rm exp} \left( \dfrac{\Omega + \mu \hat{N} - \hat{\mathcal{H}}(t_{0}) }{T} \right)
		\Big( \tilde{\psi}_{\alpha}(1) \tilde{\psi}_{\beta}^{\dagger}(2) \Big)
		\right]
	&=&
	\Big\langle \tilde{\psi}_{\alpha}(1) \tilde{\psi}_{\beta}^{\dagger}(2) \Big\rangle_{\rm st}
\end{eqnarray}

変数は、$1=(\vec{r}_{1},t_{1})$のようなshort hand notationを使った。また、$\tilde{\psi}_{\alpha}(1)$はHeisenberg演算子であり、

\begin{eqnarray}
	\tilde{\psi}_{\alpha}(\vec{r},t)
	&=&
	\hat{S}^{-1}(t,t_{0})
	\tilde{\psi}_{\alpha}(\vec{r},t_{0})
	\hat{S}(t,t_{0})
\end{eqnarray}

と定義される。S行列は

\begin{eqnarray}
	\hat{S}(t,t_{0})
	&=&
	\hat{T}_{t}
	{\rm exp} \left[ - i \int^{t}_{t_{0}} ( \hat{\mathcal{H}} - \mu \hat{N} ) dt' \right]
\end{eqnarray}

時刻$t=t_{0}$のとき温度は$T$であるとする。
非平衡の相互作用が印加される時刻は$t_{0} = - \infty$であり、
$t_{0}={\rm max} \{ t_{1},t_{2}\}$まで相互作用は印加され続ける。
このときの時間軸を正の向きに走る経路を$c_{1}$とする。
逆に$t_{0}={\rm max} \{ t_{1},t_{2}\}$から$t_{0} = - \infty$まで逆向きに走る経路を$c_{2}$
とし、全ての経路を$c=c_{1}+c_{2}$と定義する。
(これはKeldysh経路と呼ばれるが、最初に考案したのはJ. Schwingerである。)
さらに全Keldysh経路$c$に沿って定義される不等号$>_{c}$、$<_{c}$を導入し、
Keldysh経路上における時間順序積$\hat{T}_{c}$を

\begin{eqnarray}
	\hat{T}_{c}
	\Big[ \tilde{\psi}_{\alpha}(1) \tilde{\psi}_{\beta}^{\dagger}(2) \Big]
	&=&
	\left\{
	\begin{array}{ll}
		\tilde{\psi}_{\alpha}(1) \tilde{\psi}_{\beta}^{\dagger}(2),     & (t_{1} >_{c} t_{2})
		\\[3mm]
		\mp \tilde{\psi}_{\beta}^{\dagger}(2) \tilde{\psi}_{\alpha}(1), & (t_{1} <_{c} t_{2})
	\end{array}
	\right.
\end{eqnarray}

と定義する。複合は上がフェルミオン、下がボゾンである。

新しく次の2つのGreen関数も定義する。

\begin{eqnarray}
	G^{>}_{\alpha \beta}(1,2)
	&=&
	i \Big\langle \tilde{\psi}_{\alpha}(1) \tilde{\psi}_{\beta}^{\dagger}(2) \Big\rangle_{\rm st}
	\\[3mm]
	G^{<}_{\alpha \beta}(1,2)
	&=&
	\mp i \Big\langle \tilde{\psi}_{\beta}^{\dagger}(2) \tilde{\psi}_{\alpha}(1) \Big\rangle_{\rm st}
\end{eqnarray}

これを用いると、

\begin{eqnarray}
	G_{\alpha \beta}(1,2)
	&=&
	\left\{
	\begin{array}{ll}
		G^{>}_{\alpha \beta}(1,2), & (t_{1} >_{c} t_{2})
		\\[3mm]
		G^{<}_{\alpha \beta}(1,2), & (t_{1} <_{c} t_{2})
	\end{array}
	\right.
\end{eqnarray}

と書ける。これらの関数を行列要素に持つ関数を次のように定義する。

\begin{eqnarray}
	\check{\underbar{G}}
	&=&
	\left(
	\begin{array}{ll}
			G^{11} & G^{12}
			\\
			G^{21} & G^{22}
		\end{array}
	\right)
\end{eqnarray}

この4成分の関数が成す空間はKeldysh空間と呼ばれる。
成分はそれぞれ

\begin{eqnarray}
	\left\{
	\begin{array}{llc}
		G^{11}(1,2) & = & i \Big\langle \hat{T}_{t} \Big[ \tilde{\psi}_{\alpha}(1) \tilde{\psi}_{\beta}^{\dagger}(2) \Big] \Big\rangle_{\rm st}
		\\
		G^{12}(1,2) & = & G^{<}_{\alpha \beta}(1,2)
		\\
		G^{21}(1,2) & = & G^{>}_{\alpha \beta}(1,2)
		\\
		G^{22}(1,2) & = & i \Big\langle \hat{\bar{T}}_{t} \Big[ \tilde{\psi}_{\alpha}(1) \tilde{\psi}_{\beta}^{\dagger}(2) \Big] \Big\rangle_{\rm st}
	\end{array}
	\right.
\end{eqnarray}

(2,2)成分の$\hat{\bar{T}}_{t}$は反時間順序積であり、

\begin{eqnarray}
	\hat{\bar{T}}_{t}
	\Big[ \tilde{\psi}_{\alpha}(1) \tilde{\psi}_{\beta}^{\dagger}(2) \Big]
	&=&
	\left\{
	\begin{array}{ll}
		\tilde{\psi}_{\alpha}(1) \tilde{\psi}_{\beta}^{\dagger}(2),     & (t_{1} < t_{2})
		\\[3mm]
		\mp \tilde{\psi}_{\beta}^{\dagger}(2) \tilde{\psi}_{\alpha}(1), & (t_{1} > t_{2})
	\end{array}
	\right.
\end{eqnarray}

と定義される。

通常のGreen関数法と同様にretarded Green関数とadvanced Green関数をそれぞれ次のように定義する。

\begin{eqnarray}
	G^{R}_{\alpha \beta}(1,2)
	&=&
	i \theta( t_{1} - t_{2} ) \Big\langle \tilde{\psi}_{\alpha}(1) \tilde{\psi}_{\beta}^{\dagger}(2) \pm \tilde{\psi}_{\beta}^{\dagger}(2) \tilde{\psi}_{\alpha}(1) \Big\rangle_{\rm st}
	\\[3mm]
	G^{A}_{\alpha \beta}(1,2)
	&=&
	- i \theta( t_{2} - t_{1} ) \Big\langle \tilde{\psi}_{\alpha}(1) \tilde{\psi}_{\beta}^{\dagger}(2) \pm \tilde{\psi}_{\beta}^{\dagger}(2) \tilde{\psi}_{\alpha}(1) \Big\rangle_{\rm st}
\end{eqnarray}

これらに加えてKeldysh Green関数

\begin{eqnarray}
	G^{K}_{\alpha \beta}(1,2)
	&=&
	G^{<}_{\alpha \beta}(1,2)
	+
	G^{>}_{\alpha \beta}(1,2)
	\label{eqn:keldyshg}
\end{eqnarray}

も定義する。これら3つの関数はさっきのKeldysh空間の行列$\check{\underbar{G}}$の成分を用いて、

\begin{eqnarray}
	G^{R}
	&=&
	G_{11} - G_{12}
	\ \ = \ \
	G_{21} - G_{22}
	\\[1mm]
	G^{A}
	&=&
	G_{11} - G_{21}
	\ \ = \ \
	G_{12} - G_{22}
	\\[1mm]
	G^{K}
	&=&
	G_{12} + G_{21}
	\ \ = \ \
	G_{11} + G_{22}
\end{eqnarray}

と表すこともできる。従って、次のようにKeldysh変換(Keldysh回転)と呼ばれる操作を行うと、

\begin{eqnarray}
	\check{G}
	&=&
	\check{L}
	\check{\tau}_{3}
	\check{\underbar{G}}
	\check{L}^{\rm T}
	\ \ = \ \
	\left(
	\begin{array}{ll}
			G^{R} & G^{K}
			\\
			0     & G^{A}
		\end{array}
	\right)
\end{eqnarray}

のような表式が得られる。ただし、

\begin{eqnarray}
	\check{L}
	&=&
	\dfrac{1}{\sqrt{2}}(\check{1} - i \check{\tau}_{2})
\end{eqnarray}

と置いた。
$G^{ij}$(ここで$i,j=1,2$)はそれぞれ従属しているが、
これら$G^{R}$、$G^{A}$、$G^{K}$はそれぞれ互いに独立である。
非平衡Green関数を扱う場合、この行列$\check{G}$の基底の取り方が便利であり、標準的である。


\ \\

Keldysh Green関数を用いて相互作用が印加された場合のフェルミオンの粒子数$N$を数える。

\begin{eqnarray}
	N
	&=&
	\sum_{\alpha} \Big\langle \tilde{\psi}_{\alpha}^{\dagger}(1) \tilde{\psi}_{\alpha}(1) \Big\rangle_{\rm st}
	\nonumber \\[3mm] &=&
	i \sum_{\alpha} G^{<}_{\alpha \alpha}(1,1)
\end{eqnarray}

これを次の恒等式

\begin{eqnarray}
	\lim_{t_{1} \to t_{2}} \Big[ G^{>}_{\alpha \beta}(1,2) - G^{<}_{\alpha \beta}(1,2) \Big]
	&=&
	i \delta(\vec{r}_{1} - \vec{r}_{2}) \delta_{\alpha \beta}
\end{eqnarray}

と式(\ref{eqn:keldyshg})を用いて、

\begin{eqnarray}
	\lim_{t_{1} \to t_{2}}
	G^{<}_{\alpha \beta}(1,2)
	&=&
	\dfrac{1}{2}
	\lim_{t_{1} \to t_{2}}
	G^{K}_{\alpha \beta}(1,2)
	-
	i \delta(\vec{r}_{1} - \vec{r}_{2}) \delta_{\alpha \beta}
\end{eqnarray}

であることを用いると、
相互作用が入っていない場合の粒子数を$N_{0}$として$N=N_{0}+\delta N$と書くと、

\begin{eqnarray}
	N
	&=&
	N_{0} +
	\dfrac{i}{2}
	\lim_{(\vec{r}_{1} ,t_{1}) \to (\vec{r}_{2},t_{2})}
	\sum_{\alpha}
	\delta G^{K}_{\alpha \alpha}(1,2)
\end{eqnarray}

と書ける。ここで$\delta G^{K}$は同時刻におけるGreen関数の跳びを表し、

\begin{eqnarray}
	\delta G^{K}
	&=&
	G^{R} - G^{A}
	\ \ = \ \
	G^{>} - G^{<}
\end{eqnarray}

と置き換えることができる。また、

\begin{eqnarray}
	G^{K} (1,2)
	&=&
	\int \!\! \dfrac{d \varepsilon \ d \omega}{(2 \pi)^{2}} \dfrac{d^{3} \vec{p} \ d^{3} \vec{k}}{(2 \pi)^{6}}
	e^{- i \varepsilon (t_{1} - t_{2})}
	e^{- i \omega (t_{1} + t_{2})}
	e^{i \vec{p} (\vec{r}_{1} - \vec{r}_{2})}
	e^{i \vec{k} (\vec{r}_{1} + \vec{r}_{2})/2}
	\ \
	G^{K}_{\varepsilon_{+},\varepsilon_{-}} (\vec{p}_{+},\vec{p}_{-})
\end{eqnarray}

このようにフーリエ変換すると、

\begin{eqnarray}
	N(\omega,\vec{k})
	&=&
	N_{0} -
	\dfrac{i}{2}
	\sum_{\alpha}
	\int \!\! \dfrac{d \varepsilon}{2 \pi} \dfrac{d^{3} \vec{k}}{(2 \pi)^{3}}
	G^{K}_{\varepsilon_{+},\varepsilon_{-}} (\vec{p}_{+},\vec{p}_{-})
\end{eqnarray}

と表すこともできる。ここで表記の簡単のために$\vec{p}_{\pm}=\vec{p} \pm \dfrac{\vec{k}}{2}$、$\varepsilon_{\pm}=\varepsilon \pm \dfrac{\omega}{2}$と書いた。



従来のGreen関数法と同様にDyson方程式も構成することができる。
無摂動のGreen関数を$G^{(0)}_{\varepsilon}$と書いて、

\begin{eqnarray}
	G^{(0)}_{\varepsilon}
	&=&
	\dfrac{1}{\xi_{\vec{p}} - \varepsilon}
\end{eqnarray}

と定義する。$\xi_{\vec{p}}$は1粒子のバンド分散を表す。
外部ポテンシャル$\check{U}(\vec{r},t)$が印加されたとき、
$\check{U}$の一次までで、

\begin{eqnarray}
	{{G^{ik}}^{(1)}}(1,1')
	&=&
	- \int \!\! d^{3} \vec{r}_{2} \int^{\infty}_{-\infty} \!\!\! dt_{2} \
	{{G^{ij}}^{(0)}} (1,2) U^{jl}(2) {G^{lk}}^{(0)}(2,1')
\end{eqnarray}
\footnote{ただし、この時間はKeldysh経路ではなく、通常と同様に負から正に向かう時間発展をする。}

$i,j,k=1,2$を表す。ここで外部ポテンシャルは、Pauli行列$\check{\tau}_{i}$を用いて


\begin{eqnarray}
	\check{U}
	&=&
	U \check{\tau}_{3}
	\ \ = \ \
	\left(
	\begin{array}{ll}
			U & 0
			\\
			0 & -U
		\end{array}
	\right)
\end{eqnarray}

のような構造を持っているものと仮定した。
行列成分の添字と積分記号を抜いて、

\begin{eqnarray}
	\check{\underbar{G}}^{(1)}
	&=&
	-
	\check{\underbar{G}}^{(0)}
	\check{U}
	\check{\underbar{G}}^{(0)}
\end{eqnarray}

と書いて、2次、3次、$\cdots$、と続けていくとall orderで

\begin{eqnarray}
	\check{\underbar{G}}
	&=&
	\check{\underbar{G}}^{(0)}
	-
	\check{\underbar{G}}^{(0)}
	\check{U}
	\check{\underbar{G}}
\end{eqnarray}

と書けることが分かる。Keldysh回転を施すと、

\begin{eqnarray}
	\check{G}
	&=&
	\check{G}^{(0)}
	-
	\check{G}^{(0)}
	(U \check{1})
	\check{G}
\end{eqnarray}

相互作用の無い場合のGreen関数の逆演算子

\begin{eqnarray}
	{{\check{G}}^{(0)}{}}^{-1}
	&=&
	- i \partial_{t} + \xi_{\vec{p}}
\end{eqnarray}

を用いて、

\begin{eqnarray}
	({{\check{G}}^{(0)}{}}^{-1} + U ) \check{G}
	&=&
	\check{1}
\end{eqnarray}

と、従来と同様に書ける。
以上は、例えば不純物散乱などに応用できる枠組みであるが、
フォノンとの相互作用 and/or 電子間相互作用が印加された場合であっても同様に、
自己エネルギーもKeldysh形式で導入することができて次のように書き直せる。
(フォノンの場合を別のノートにまとめる。)
[Rammer and Smith(1986)]

\begin{eqnarray}
	({{\check{G}}^{(0)}{}}^{-1} + \check{\mathit{\Sigma}} ) \check{G}
	&=&
	\check{1}
	,
	\hspace{10mm}
	\check{\mathit{\Sigma}}
	\ \ = \ \
	\left(
	\begin{array}{ll}
			\mathit{\Sigma}^{R} & \mathit{\Sigma}^{K}
			\\
			0                   & \mathit{\Sigma}^{A}
		\end{array}
	\right)
\end{eqnarray}


付録として、無摂動のGreen関数の表現をそれぞれ示す。(符号の定義はAGDと逆になっていることに注意する。)

\begin{eqnarray}
	&&
	\left(
	\begin{array}{ll}
		{G^{11}{}}^{(0)}_{\varepsilon}(\vec{p}) & {G^{12}{}}^{(0)}_{\varepsilon}(\vec{p})
		\\
		{G^{21}{}}^{(0)}_{\varepsilon}(\vec{p}) & {G^{22}{}}^{(0)}_{\varepsilon}(\vec{p})
	\end{array}\right)
	\\ = &&
	\left(
	\begin{array}{cc}
		\dfrac{1}{\xi_{\vec{p}} - ( \varepsilon + i0 ) } \mp 2 \pi i n_{\vec{p}} \delta( \xi_{\vec{p}} - \varepsilon )
		 &
		\pm 2 \pi i n_{\vec{p}} \delta( \xi_{\vec{p}} - \varepsilon )
		\\
		- 2 \pi i ( 1 \mp n_{\vec{p}} ) \delta( \xi_{\vec{p}} - \varepsilon )
		 &
		- \dfrac{1}{\xi_{\vec{p}} - ( \varepsilon - i0 ) } \mp 2 \pi i n_{\vec{p}} \delta( \xi_{\vec{p}} - \varepsilon )
	\end{array}\right)
\end{eqnarray}
\begin{eqnarray}
	\left(
	\begin{array}{ll}
		{G^{R}{}}^{(0)}_{\varepsilon}(\vec{p}) & {G^{K}{}}^{(0)}_{\varepsilon}(\vec{p})
		\\
		{G^{A}{}}^{(0)}_{\varepsilon}(\vec{p}) & 0
	\end{array}\right)
	&=&
	\left(
	\begin{array}{cc}
		\dfrac{1}{\xi_{\vec{p}} - ( \varepsilon + i0 ) }
		 &
		- 2 \pi i n_{\vec{p}} \delta( \xi_{\vec{p}} - \varepsilon )
		\\
		\dfrac{1}{\xi_{\vec{p}} - ( \varepsilon - i0 )}
		 &
		0
	\end{array}\right)
\end{eqnarray}



また、BCS超伝導状態にも拡張することができる。Green逆演算子を次のように定義すれば良い。

\begin{eqnarray}
	\check{G}^{-1}_{\varepsilon}(\vec{p}-\vec{k}_{1},\varepsilon_{1})
	&=&
	\left(
	\begin{array}{ll}
		\xi_{\vec{p}} - \varepsilon & 0
		\\
		0                           & \xi_{\vec{p}} + \varepsilon
	\end{array}
	\right)
	(2 \pi)^{4}
	\delta(\varepsilon_{1})
	\delta(\vec{k}_{1})
	+
	\check{H}_{\varepsilon_{1}}
	\\[3mm]
	\check{H}_{\varepsilon_{1}}
	&=&
	\left(
	\begin{array}{ll}
			- \dfrac{e}{c} \vec{v}_{\rm F} \vec{A}(\vec{k}) + e \phi & - \Delta(\vec{k})
			\\
			\Delta^{*}(\vec{k})                                      & \dfrac{e}{c} \vec{v}_{\rm F} \vec{A}(\vec{k}) + e \phi
		\end{array}
	\right)
\end{eqnarray}


\end{document}

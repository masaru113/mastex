\documentclass[uplatex,a4j,12pt,dvipdfmx]{jsarticle}
\usepackage[english]{babel}
\usepackage[letterpaper,top=2cm,bottom=2cm,left=3cm,right=3cm,marginparwidth=1.75cm]{geometry}
\usepackage{amsmath,amsthm,amssymb,bm,color,mathrsfs,url}
\usepackage{epic,eepic,here}
\usepackage[dvipdfm]{graphicx}
\usepackage[hypertex]{hyperref}
\title{非平衡におけるMigdal-Eliashberg理論}
\author{Masaru Okada}
\date{\today}
\def\sla#1{\rlap/#1}
\begin{document}
\maketitle

\begin{abstract}
	非平衡におけるMigdal-Eliashberg理論についてのメモ。
	非平衡に拡張されたMigdal-Eliashbergモデルにおける超伝導ギャップ方程式を導出する。
\end{abstract}

松原空間から出発する。
フェルミオンの松原周波数を$\varepsilon_{n} = 2 \pi i T (n + 1/2)$と置いて、
Gorkov Green関数に対するDyson方程式は

\begin{eqnarray}
	\sum_{\varepsilon_{m}}
	\int \dfrac{ d^{3} \vec{k}_{1} }{ (2 \pi)^{3} }
	\Big[ \check{G}^{-1}_{\varepsilon_{n}} (\vec{p} - \vec{k}_{1} , \omega_{1} ) - \check{\mathit{\Sigma}}_{\varepsilon_{n},\varepsilon_{m}}(\vec{p},\vec{k}_{1}) \Big]
	\check{G}_{\varepsilon_{m},\varepsilon_{\vec{k}}} ( \vec{p} - \vec{k}_{1} , \vec{p} - \vec{k} )
	&=&
	\check{1} (2 \pi)^{3} \delta(\vec{k}) \delta_{\varepsilon_{n} - \varepsilon_{\vec{k}}}
	\label{eqn:dysonone}
\end{eqnarray}

下付き添字にエネルギーの変数を2つ持つ関数は2体の関数である。
$\check{G}^{-1}_{\varepsilon}$は次のような構造を持つ。

\begin{eqnarray}
	\check{G}^{-1}_{\varepsilon}(\vec{p}-\vec{k}_{1},\omega_{1})
	&=&
	\left(
	\begin{array}{cc}
		\xi_{\vec{p}} - \varepsilon & 0
		\\
		0                           & \xi_{\vec{p}} + \varepsilon
	\end{array}
	\right)
	(2 \pi)^{3}
	\delta(\vec{k}_{1})
	\delta_{\omega_{1}}
	+
	\check{H}_{\omega_{1}}
\end{eqnarray}

$\omega_{1} = \varepsilon_{n} - \varepsilon_{m}$と置いた。
時間に依存する外場$\check{H}_{\omega_{1}}$は今は次のようなものを仮定している。

\begin{eqnarray}
	\check{H}_{\omega_{1}}
	&=&
	\left(
	\begin{array}{cc}
			- \dfrac{e}{c} \vec{v}_{\rm F} \vec{A}_{\omega_{1}} (\vec{k}_{1}) + e \phi_{\omega_{1}}(\vec{k}_{1}) & - \Delta_{\omega_{1}} (\vec{k}_{1})
			\\
			\Delta^{*}_{\omega_{1}} (\vec{k})                                                                    & \dfrac{e}{c} \vec{v}_{\rm F} \vec{A}_{\omega_{1}} (\vec{k}_{1}) + e \phi_{\omega_{1}} (\vec{k}_{1})
		\end{array}
	\right)
\end{eqnarray}

$\check{H}_{\omega_{j}}$の$N$次までで展開された電子のGreen関数$\check{G}^{(N)}$は

\begin{eqnarray}
	\check{G}^{(N)}_{\varepsilon_{n},\varepsilon_{n} - \omega}
	&=&
	(-1)^{N}
	G^{(0)}_{\varepsilon_{n}}
	\check{H}_{\omega_{1}}
	G^{(0)}_{\varepsilon_{n} - \omega_{1}}
	\check{H}_{\omega_{2}}
	G^{(0)}_{\varepsilon_{n} - \omega_{1} - \omega_{2} }
	\ \cdots \
	\check{H}_{\omega_{N}}
	G^{(0)}_{\varepsilon_{n} - \omega }
\end{eqnarray}

ここで$\omega = \omega_{1} + \omega_{2} + \cdots + \omega_{N}$と置いた。自由な電子のGreen関数$G^{(0)}$は

\begin{eqnarray}
	G^{(0)}_{\varepsilon_{n}}
	&=&
	\dfrac{1}{ \xi_{\vec{p}} - \varepsilon_{n} }
\end{eqnarray}

である(AGD等と符号が逆)。
留数定理を用いると、複素数$z$に対して次の等式が成り立つ。

\begin{eqnarray}
	T \sum_{n} \check{G}^{(N)}_{\varepsilon_{n},\varepsilon_{n} - \omega}
	&=&
	\oint \!\! \dfrac{dz}{4 \pi i}
	\check{G}^{(N)}_{z,z - \omega}
	{\rm tanh} \dfrac{z}{2 T}
\end{eqnarray}

積分経路は時計回りに原点中心の半径無限大の円である。
相互作用のハミルトニアンに時間依存性が無い場合は、
積分経路は${\rm tanh}(z/2 T)$由来の特異点、
すなわち$z=2 \pi i T (n + 1/2)$の周りを取るもののみを考えればよかった。
しかし、今の場合、$N$次の電子のGreen関数には$\check{H}_{\omega_{j}}$由来の特異線

\begin{eqnarray}
	{\rm Im}
	\Big( z - \sum^{l}_{j = 0} \omega_{j} \Big)
	&=&
	0
	\label{eqn:tokuisen}
\end{eqnarray}

もあるので、その各々の特異線の上側$(+i0)$に沿って右向きに${\rm Re}(z)$を$- \infty$から$\infty$
まで、下側$(-i0)$にそって左向きに${\rm Re}(z)$を$\infty$から$-\infty$まで走らせるように経路を$j$個に分岐させる。
(ここで表記の簡単のため$\omega_{0}=0$とした。)
このようにして積分は複素数の変数$\varepsilon$を用いて、

\begin{eqnarray}
	\oint \!\! \dfrac{dz}{4 \pi i}
	\check{G}^{(N)}_{z,z - \omega}
	{\rm tanh} \dfrac{z}{2 T}
	&=&
	\int \! \! \dfrac{d \varepsilon}{4 \pi i}
	\Big(
	\delta_{0} \check{G}^{(N)}
	+
	\delta_{1} \check{G}^{(N)}
	+
	\cdots
	+
	\delta_{N} \check{G}^{(N)}
	\Big)
	{\rm tanh} \dfrac{\varepsilon}{2 T}
\end{eqnarray}

と表すことが出来る。ここで$\delta_{l} \check{G}^{(N)}$は$l$番目の特異線(式(\ref{eqn:tokuisen}))におけるGreen関数の跳びを表し、具体的には

\begin{eqnarray}
	\delta_{l} \check{G}^{(N)}
	&=&
	\Big[ \check{G}^{(N)}_{z,z - \omega} \Big]_{z = z_{l} + 0}
	-
	\Big[ \check{G}^{(N)}_{z,z - \omega} \Big]_{z = z_{l} - 0}
	\hspace{10mm}
	\left( {\rm where} \ \ \ z_{l} = \varepsilon + i \ \! {\rm Im} \sum^{l}_{j=0} \omega_{j} \right)
\end{eqnarray}

と書ける。
このように時間依存する外場に由来する特異線上の跳びを考える点で平衡状態のGreen関数と解析性が異なる。






解析接続を施す。
この跳びは、例えば$l=1$の場合、

\begin{eqnarray}
	\delta_{1} \check{G}^{(N)}
	&=&
	(-1)^{N}
	G^{(0)}_{\varepsilon + \omega_{1} }
	\check{H}_{\omega_{1}}
	\Big( G^{(0)}_{\varepsilon + i 0 } - G^{(0)}_{\varepsilon - i 0 } \Big)
	\check{H}_{\omega_{2}}
	G^{(0)}_{\varepsilon - \omega_{2} }
	\ \cdots \
	\check{H}_{\omega_{N}}
	G^{(0)}_{\varepsilon + \omega_{1} - \omega }
\end{eqnarray}

であるが、従来通り極を虚軸方向へ無限小だけ上(下)にずらしたretarded (advanced) Green関数$G^{(0) R(A)}$を用いて、

\begin{eqnarray}
	\delta_{1} \check{G}^{(N)}
	&=&
	(-1)^{N}
	G^{(0)R}_{\varepsilon + \omega_{1} }
	\check{H}_{\omega_{1}}
	\Big( G^{(0)R}_{\varepsilon} - G^{(0)A}_{\varepsilon} \Big)
	\check{H}_{\omega_{2}}
	G^{(0)A}_{\varepsilon - \omega_{2} }
	\ \cdots \
	\check{H}_{\omega_{N}}
	G^{(0)A}_{\varepsilon + \omega_{1} - \omega }
\end{eqnarray}

と解析接続を行う。
積分のダミー変数を$\varepsilon + \omega \to \varepsilon$と移し替えて

\begin{eqnarray}
	\delta_{1} \check{G}^{(N)}_{\varepsilon , \varepsilon - \omega}
	&=&
	(-1)^{N}
	G^{(0)R}_{\varepsilon }
	\check{H}_{\omega_{1}}
	\Big( G^{(0)R}_{\varepsilon - \omega_{1}} - G^{(0)A}_{\varepsilon - \omega_{1}} \Big)
	\check{H}_{\omega_{2}}
	G^{(0)A}_{\varepsilon - \omega_{1} - \omega_{2} }
	\ \cdots \
	\check{H}_{\omega_{N}}
	G^{(0)A}_{\varepsilon - \omega }
\end{eqnarray}

と変数を露わに書くことにすると、

\begin{eqnarray}
	\oint \!\! \dfrac{dz}{4 \pi i}
	\check{G}^{(N)}_{z,z - \omega}
	{\rm tanh} \dfrac{z}{2 T}
	&=&
	\int^{\infty}_{-\infty} \! \dfrac{d \varepsilon}{4 \pi i}
	\Big[
	{\rm tanh} \Big( \dfrac{\varepsilon}{2 T} \Big)
	\delta_{0} \check{G}^{(N)}_{\varepsilon , \varepsilon - \omega}
	+
	{\rm tanh} \Big( \dfrac{\varepsilon - \omega_{1}}{2 T} \Big)
	\delta_{1} \check{G}^{(N)}_{\varepsilon , \varepsilon - \omega}
	\nonumber \\ &&
	+
	{\rm tanh} \Big( \dfrac{\varepsilon - \omega_{1} - \omega_{2}}{2 T} \Big)
	\delta_{1} \check{G}^{(N)}_{\varepsilon , \varepsilon - \omega}
	+
	\cdots
	+
	{\rm tanh} \Big( \dfrac{\varepsilon - \omega}{2 T} \Big)
	\delta_{N} \check{G}^{(N)}_{\varepsilon , \varepsilon - \omega}
	\Big]
	\nonumber \\[3mm] &=&
	\sum_{l=0}^{N}
	\int^{\infty}_{-\infty} \! \dfrac{d \varepsilon}{4 \pi i}
	{\rm tanh} \Big( \dfrac{\varepsilon - \sum_{j=0}^{l} \omega_{j} }{2 T} \Big)
	\delta_{l} \check{G}^{(N)}_{\varepsilon , \varepsilon - \omega}
\end{eqnarray}

被積分関数を次のように表記することにする。

\begin{eqnarray}
	\check{G}^{(N)K}_{\varepsilon , \varepsilon - \omega}
	&=&
	\sum_{l=0}^{N}
	{\rm tanh} \Big( \dfrac{\varepsilon - \sum_{j=0}^{l} \omega_{j} }{2 T} \Big)
	\delta_{l} \check{G}^{(N)}_{\varepsilon , \varepsilon - \omega}
\end{eqnarray}

無摂動の分布関数(Fermi分布関数)を$f^{(0)}(\varepsilon) = {\rm tanh}\dfrac{\varepsilon}{2 T}$と書くと、
(この時点ではまだ$\check{G}^{K}$が従来の意味のKeldysh Green関数に一致することは自明ではないが、)
フルに摂動が加わったKeldysh Green関数$\check{G}^{K} = \displaystyle \lim_{N \to \infty} \check{G}^{(N)K}$は次のように
置くことができる。

\begin{eqnarray}
	\check{G}^{K}_{\varepsilon , \varepsilon - \omega}
	&=&
	\check{G}^{R}_{\varepsilon , \varepsilon - \omega}
	f^{(0)}(\varepsilon - \omega)
	-
	f^{(0)}(\varepsilon)
	\check{G}^{A}_{\varepsilon , \varepsilon - \omega}
	+
	\check{G}^{(a)}_{\varepsilon , \varepsilon - \omega}
\end{eqnarray}

おつりの関数$\check{G}^{(a)}$は

\begin{eqnarray}
	\check{G}^{(N)(a)}_{\varepsilon , \varepsilon - \omega}
	&=&
	(-1)^{N-1}
	G^{(0)R}_{\varepsilon}
	\check{h}_{\varepsilon , \varepsilon - \omega_{1}}
	G^{(0)A}_{\varepsilon - \omega_{1}}
	\ \cdots \
	\check{H}_{\omega_{N}}
	G^{(0)A}_{\varepsilon - \omega}
	\ + \cdots
	\nonumber \\ && +
	(-1)^{N-1}
	G^{(0)R}_{\varepsilon}
	\check{H}_{\omega_{1}}
	\ \cdots \
	G^{(0)R}_{\varepsilon - ( \omega - \omega_{N} )}
	\check{h}_{\varepsilon - ( \omega - \omega_{N} ) , \varepsilon - \omega}
	G^{(0)A}_{\varepsilon - \omega}
\end{eqnarray}

であり、特に平衡系(自由な系)では単純に

\begin{eqnarray}
	\check{G}^{(a)}_{\varepsilon , \varepsilon - \omega} ( \vec{p} , \vec{p} - \vec{k})
	&=&
	\int \!\! \dfrac{d \varepsilon'  d \omega' }{ (2 \pi)^{2} } \dfrac{d^{3} \vec{k'}  d^{3} \vec{k''} }{ (2 \pi)^{6} }
	G^{R}_{\varepsilon , \varepsilon'} (\vec{p} , \vec{p} - \vec{k'})
	\check{h}_{\varepsilon' , \varepsilon' - \omega'} (\vec{k''})
	G^{A}_{\varepsilon' - \omega' , \varepsilon - \omega} (\vec{p} - \vec{k'} - \vec{k''} , \vec{p} - \vec{k} )
\end{eqnarray}

と書ける。ただし、外場と分布関数の積$\check{h}_{\varepsilon , \varepsilon - \omega}$を

\begin{eqnarray}
	\check{h}_{\varepsilon , \varepsilon - \omega}
	&=&
	- \check{H}_{\omega} \Big( f^{(0)}(\varepsilon) - f^{(0)}(\varepsilon - \omega) \Big)
\end{eqnarray}

と略記した。










${}$











Migdalの定理が成立する場合の議論を以下で展開する。
その場合、摂動がフルに入っているフォノンのGreen関数は自由なGreen関数と等しいとすることが出来る。
式(\ref{eqn:dysonone})の自己エネルギー
$\check{\mathit{\Sigma}}_{\varepsilon_{n},\varepsilon_{m}}$
は電子-フォノンの相互作用に依るもののみと仮定する。
フォノンのGreen関数を$D_{\varepsilon}$と書くと、

\begin{eqnarray}
	D_{\varepsilon' - \varepsilon} ( \vec{p} {\ \!}' - \vec{p} {\ \!} )
	&=&
	\dfrac{ \omega^{2}_{ \vec{p}' - \vec{p} } }{ \omega^{2}_{ \vec{p}' - \vec{p} } - ( \varepsilon' - \varepsilon )^{2} }
\end{eqnarray}

ここで$\varepsilon' - \varepsilon=2 \pi i m$と置いた。
Migdalの定理の下、
フォノンの自己エネルギー$\check{\mathit{\Sigma}}^{(\rm ph)}_{\varepsilon , \varepsilon - \omega}$は、

\begin{eqnarray}
	\check{\mathit{\Sigma}}^{(\rm ph)}_{\varepsilon , \varepsilon - \omega} (\vec{p} , \vec{p} - \vec{k})
	&=&
	T
	\sum_{\varepsilon'}
	\int \!\! \dfrac{ d^{3} \vec{p {} \ \! '} }{ (2 \pi)^{3} }
	g^{2}
	D_{\varepsilon' - \varepsilon} ( \vec{p} {\ \!}' - \vec{p} {\ \!} )
	\check{G}_{\varepsilon' , \varepsilon' - \omega} (\vec{p {} '} , \vec{p {} '} - \vec{k})
\end{eqnarray}

である。
外場の$N$次までで、


\begin{eqnarray}
	\check{\mathit{\Sigma}}^{({\rm ph})(N)}_{\varepsilon , \varepsilon - \omega}
	&=&
	g^{2} \oint \!\! \dfrac{d z}{4 \pi i}
	\Big( D_{z - \varepsilon} \check{G}_{z,z-\varepsilon}^{(N)} \Big)
	{\rm tanh} \Big( \dfrac{z}{2T} \Big)
\end{eqnarray}

簡単のために$d^{3} \vec{p}$積分は露わに書いていない。
被積分関数に表れる解析性を見る。
特異点(線)は、まず電子のGreen関数が${\rm Im}(z - \omega_{j})=0$を持っており、
フォノンのGreen関数が${\rm Im}(z - \varepsilon_{n})=0$を持っている。
今、$\omega_{j} = 2 \pi i T k_{j}$、$\varepsilon_{n} = 2 \pi i T (n + 1/2)$はそれぞれ虚数であるが、
実数$\varepsilon'$を用いて、
$\check{G}$に対しては$z= \varepsilon' + i \ \! {\rm Im} (\varepsilon_{n})$、
$D$に対しては$z= \varepsilon' + i \ \! {\rm Im} (\omega_{j})$のように変数を置換する。

\begin{eqnarray}
	\check{\mathit{\Sigma}}^{(\rm ph)}_{\varepsilon , \varepsilon - \omega}
	&=&
	g^{2} \int \!\! \dfrac{d \varepsilon'}{4 \pi i}
	\Big[
		{\rm coth} \Big( \dfrac{\varepsilon'}{2T} \Big)
		\big( D^{R}_{\varepsilon'} - D^{A}_{\varepsilon'} \big)
		\check{G}_{ \varepsilon' + \varepsilon , \varepsilon' + \varepsilon - \omega }^{(N)} \Big]
	\nonumber \\ && +
	g^{2} \int \!\! \dfrac{d \varepsilon'}{4 \pi i}
	{\rm tanh} \Big( \dfrac{\varepsilon'}{2T} \Big)
	\Big[
		D_{\varepsilon' - \varepsilon} \delta_{0} \check{G}^{(N)}_{\varepsilon' , \varepsilon' - \omega}
		+
		D_{\varepsilon' - (\varepsilon - \omega_{1})} \delta_{1} \check{G}^{(N)}_{\varepsilon' , \varepsilon' - \omega}
		+ \cdots +
		D_{\varepsilon' - (\varepsilon - \omega)} \delta_{N} \check{G}^{(N)}_{\varepsilon' , \varepsilon' - \omega}
		\Big]
	\label{eqn:anaricgma}
\end{eqnarray}

左辺第一項は$D$線に由来する特異点の周りで、
第二項は$\check{G}_{z,z-\varepsilon}^{(N)}$に由来する(すなわち"外場"$\check{H}_{\omega_{j}}$に由来する)特異点の周りで、
それぞれ実軸と平行になるよう積分経路を変更して発生した項である。
解析接続することを考える。
式(\ref{eqn:anaricgma})の被積分関数を見ると、$\varepsilon$の関数として、
$\check{\mathit{\Sigma}}^{(\rm ph)}_{\varepsilon , \varepsilon - \omega}$の特異点は
$\check{G}_{z,z-\varepsilon}^{(N)}$の特異点と一致することが確認できる。
従って、相互作用の全ての次数でトータルのGreen関数と同様に自己エネルギーも解析接続することが出来る。
各$\check{G}^{(N)}$の特異点(線)に対しては$\displaystyle \varepsilon' \to \varepsilon' - \sum_{l=0}^{j} \omega_{l}$のように、
$D$線の特異点(線)に対しては$\displaystyle \varepsilon' \to \varepsilon' - \varepsilon$
と変数変換すると、解析接続されたフォノンの自己エネルギーが得られる。

\begin{eqnarray}
	\check{\mathit{\Sigma}}^{{(\rm ph)} {{R}\atop{A}} }_{\varepsilon , \varepsilon - \omega} (\vec{p} , \vec{p} - \vec{k})
	&=&
	g^{2} \int \!\! \dfrac{d \varepsilon'}{4 \pi i} \dfrac{d^{3} \vec{p'}}{ (2 \pi)^{3}}
	\Big[
		{\rm coth} \Big( \dfrac{ \varepsilon' - \varepsilon }{2T} \Big)
		\big( D^{R}_{\varepsilon'} - D^{A}_{\varepsilon'} \big)
		\check{G}^{ {{R}\atop{A}} }_{ \varepsilon' , \varepsilon'- \omega }
		+
		D^{ {{A}\atop{R}} }_{\varepsilon' - \varepsilon} \check{G}^{K}_{ \varepsilon' , \varepsilon'- \omega }
		\Big]
\end{eqnarray}

Keldysh成分は、

\begin{eqnarray}
	\check{\mathit{\Sigma}}^{{(\rm ph)} K }_{\varepsilon , \varepsilon - \omega} (\vec{p} , \vec{p} - \vec{k})
	&=&
	g^{2} \int \!\! \dfrac{d \varepsilon'}{4 \pi i} \dfrac{d^{3} \vec{p'}}{ (2 \pi)^{3}}
	\big( D^{R}_{\varepsilon'} - D^{A}_{\varepsilon'} \big)
	\Big[
		{\rm coth} \Big( \dfrac{ \varepsilon' - \varepsilon }{2T} \Big)
		\check{G}^{K}_{ \varepsilon' , \varepsilon'- \omega }
		-
		\big( \check{G}^{R}_{ \varepsilon' , \varepsilon'- \omega } - \check{G}^{A}_{ \varepsilon' , \varepsilon'- \omega } \big)
		\Big]
\end{eqnarray}

であり、南部空間にKeldysh成分も含ませることでGreen関数の行列のサイズは$4 \times 4$になる。

\begin{eqnarray}
	\breve{G}
	&=&
	\left(
	\begin{array}{cc}
			\check{G}^{R} & \check{G}^{K}
			\\
			0             & \check{G}^{A}
		\end{array}
	\right)
	\ \ = \ \
	\left(
	\begin{array}{cccc}
			G^{R}           & F^{R}               & G^{K}           & F^{K}
			\\
			- F^{\dagger R} & \check{\bar{G}}^{R} & - F^{\dagger K} & \check{\bar{G}}^{K}
			\\
			0               & 0                   & G^{A}           & F^{A}
			\\
			0               & 0                   & - F^{\dagger A} & \check{\bar{G}}^{A}
		\end{array}
	\right)
\end{eqnarray}

Migdal-Eliashbergモデルのギャップ方程式(非平衡下における非線形Eliashberg方程式)は次のようになる。

\begin{eqnarray}
	\Delta_{\varepsilon , \varepsilon - \omega}^{(*)} (\vec{p} , \vec{p} - \vec{k})
	&=&
	\dfrac{g^{2}}{2}
	\int \!\! \dfrac{d \varepsilon'}{4 \pi i} \dfrac{d^{3} \vec{p'}}{ (2 \pi)^{3}}
	\big( D^{R}_{\varepsilon' - \varepsilon} + D^{A}_{\varepsilon' - \varepsilon} \big)
	F^{(\dagger)K}_{\varepsilon' , \varepsilon' - \omega}
\end{eqnarray}

$F^{K}$は$\varepsilon \gg \Delta$の領域に対してほとんど有限の値を残さず0になるので、
Debye周波数$\omega_{\rm D}$に対して$\varepsilon \ll \omega_{\rm D}$の場合が重要になる。
さらにこの場合、$(D^{R} + D^{A})/2 \simeq 1$となり、
$\varepsilon$と$\omega$には殆ど依存しなくなる。このとき線形化されたEliashberg方程式

\begin{eqnarray}
	\Delta_{\omega}^{} (\vec{k})
	&=&
	g^{2}
	\int \!\! \dfrac{d \varepsilon}{4 \pi i} \dfrac{d^{3} \vec{p'}}{ (2 \pi)^{3}}
	F^{K}_{\varepsilon , \varepsilon - \omega}
\end{eqnarray}

となり、BCS的に(つまり電子-フォノンによる相互作用を考えずに)$s$波を仮定した場合と同様の方程式に還元することもできる。


\end{document}

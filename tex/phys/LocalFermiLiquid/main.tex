\documentclass[a4j]{jarticle}
\usepackage[english]{babel}
\usepackage{amsmath,amsthm,amssymb,bm,color,mathrsfs}
\usepackage{epic,eepic,here}
\usepackage[dvipdfm]{graphicx}
\title{\ \\[-35mm] 2-body mechanics in the Local Fermi Liquid
}
\author{ Masaru Okada}
\date{\today}
\begin{document}
\allowdisplaybreaks
\maketitle

\ \\[-20mm]

\section*{Motivation}


We will derive some physical quantities like $\chi_{s}$ (\textbf{spin susceptibility}), $\chi_{c}$ (\textbf{charge susceptibility}), and $R_{W}$ (\textbf{Wilson ratio}).
Understanding these quantities helps us characterize strongly correlated electron systems and understand the behavior of $f$-electron systems.


\section*{Duality between the local Fermi Liquid and the Anderson model}
First, we will discuss the simplest Anderson Hamiltonian.
\begin{align}
	\hspace{17.3mm}
	\mathcal{H}
	=
	\sum_{\bm{k}\sigma}
	\varepsilon_{\bm{k}}
	c^{\dagger}_{\bm{k} \sigma}
	c_{\bm{k} \sigma}
	+
	\sum_{\sigma}
	\varepsilon_{d}
	n_{d \sigma}
	+
	U
	n_{d \uparrow}
	n_{d \downarrow}
	+
	\frac{1}{\sqrt{N_{\rm A}}}
	\sum_{\bm{k}\sigma}
	(
	V_{\bm{k}}
	c_{\bm{k}\sigma}^{\dagger}
	d_{\sigma}
	+
	{\rm h.c.}
	)
\end{align}
Here, $c_{\bm{k}\sigma}$ is the annihilation operator for a quasiparticle in a conduction band with momentum $\bm{k}$ and spin $\sigma(=\uparrow,\downarrow)$, $\varepsilon_{\bm{k}}$ is the band dispersion of the conduction electron, $\varepsilon_{d}$ is the $d$-level energy, $d_{\sigma}$ is the annihilation operator for a $d$-electron with spin $\sigma$, $n_{d \sigma}=d_{\sigma}^{\dagger} d_{\sigma}$ is the number of $d$-electrons with spin $\sigma$, $U$ is the strength of the Coulomb interaction between $d$-electrons, and $V_{\bm{k}}$ is the strength of the interaction between the $d$-electron and the conduction electron.

The term "$d$-electron" is used for historical reasons.
We can replace "$d$-electron" with "$f$-electron" and the physics remains the same.

Note that this model neglects the interaction between conduction electrons.
Furthermore, we consider only a single $d$-band.
In reality, to understand Fe-compounds, for example, we would need to consider five $d$-bands.
\footnote{
	Since Ce and Yb compounds have 4$f$-electrons, this model alone is insufficient for understanding them.
	However, the meaning of spin susceptibility, charge susceptibility, and Wilson ratio can still be interpreted within the same framework.
}

${}$

The Fermi liquid (FL) model
\footnote{
	An interacting itinerant electron system is well described by a non-interacting system (the Fermi gas model) with renormalized weights when the system is at low temperatures and the quasiparticles are near the Fermi surface.
	Such a system is called a Fermi liquid, by analogy to the Fermi gas.
}
is a standard model for metals.
To use the information of the FL state, we assume that the quasiparticles are near the Fermi surface.

We now treat the $U$ term as a perturbation.

\begin{align}
	\mathcal{H}_{0}
	 & =
	\sum_{\bm{k}\sigma}
	\varepsilon_{\bm{k}}
	c^{\dagger}_{\bm{k} \sigma}
	c_{\bm{k} \sigma}
	+
	\sum_{\sigma}
	\varepsilon_{d}
	n_{d \sigma}
	+
	\frac{1}{\sqrt{N_{\rm A}}}
	\sum_{\bm{k}\sigma}
	(
	V_{\bm{k}}
	c_{\bm{k}\sigma}^{\dagger}
	d_{\sigma}
	+
	{\rm h.c.}
	)
	\hspace{3mm}
	,
	\hspace{5mm}
	\mathcal{H}
	-
	\mathcal{H}_{0}
	=
	U
	n_{d \uparrow}
	n_{d \downarrow}
\end{align}

Our first goal is to diagonalize the non-perturbative term $\mathcal{H}_{0}$.

\begin{align}
	\hspace{10.2mm}
	\mathcal{H}_{0}
	=
	\sum_{\bm{k}\sigma}
	\varepsilon_{\bm{k}}
	c^{\dagger}_{\bm{k} \sigma}
	c_{\bm{k} \sigma}
	+
	\sum_{\sigma}
	\varepsilon_{d}
	n_{d \sigma}
	+
	\frac{1}{\sqrt{N_{\rm A}}}
	\sum_{\bm{k}\sigma}
	(
	V_{\bm{k}}
	c_{\bm{k}\sigma}^{\dagger}
	d_{\sigma}
	+
	{\rm h.c.}
	)
	\
	\to
	\hspace{5mm}
	\mathcal{H}_{0}
	=
	\sum_{n \sigma}
	\varepsilon_{n}
	c_{n \sigma}^{\dagger}
	c_{n \sigma}
	\hspace{10.2mm}
\end{align}

To perform the diagonalization, we create linear combinations with coefficients $( n | \bm{k} )$ and $( n | d )$.

\begin{align}
	\hspace{49mm}
	c^{\dagger}_{n \sigma}
	=
	\sum_{\bm{k}}
	( n | \bm{k} )
	c^{\dagger}_{\bm{k} \sigma}
	+
	( n | d )
	d^{\dagger}_{\sigma}
	\hspace{49mm}
\end{align}
\\[-12mm]
\begin{align}
	\hspace{47.4mm}
	c_{n \sigma}
	=
	\sum_{\bm{k}}
	( n | \bm{k} )^{*}
	c_{\bm{k} \sigma}
	+
	( n | d )^{*}
	d_{\sigma}
	\hspace{47.2mm}
\end{align}
These coefficients, $( n | \bm{k} )$, $( n | d )$, and $\varepsilon_{n}$, are determined by the commutation relation $\displaystyle \big[ \mathcal{H}_{0} , c^{\dagger}_{n \sigma} \big] = \varepsilon_{n} c^{\dagger}_{n \sigma} $.
We can use the fact that $
	\big[
		c^{(\dagger)}_{\alpha},d^{(\dagger)}_{\alpha}
		\big]
	=
	0$.
\begin{align}
	\begin{split}
		\bigg[
			\sum_{\bm{k}\sigma}
		 & \varepsilon_{\bm{k}}
			c^{\dagger}_{\bm{k} \sigma}
			c_{\bm{k} \sigma}
			+
			\sum_{\sigma}
			\varepsilon_{d}
			n_{d \sigma}
			+
			\frac{1}{\sqrt{N_{\rm A}}}
			\sum_{\bm{k}\sigma}
			(
			V_{\bm{k}}
			c_{\bm{k}\sigma}^{\dagger}
			d_{\sigma}
			+
			{\rm h.c.}
			)
			\
			,
		\\
		 & \sum_{\bm{k}'}
			( n | \bm{k}' )
			c^{\dagger}_{\bm{k}' \sigma'}
			+
			( n | d )
			d^{\dagger}_{\sigma'}
			\bigg]
	\end{split}
	\nonumber \\[4mm] &=
	\sum_{\bm{k} \bm{k}' \sigma}
	\varepsilon_{\bm{k}}
	( n | \bm{k}' )
	\Big[
		c^{\dagger}_{\bm{k} \sigma}
		c_{\bm{k} \sigma}
		\
		,
		\
		c^{\dagger}_{\bm{k}' \sigma'}
		\Big]
	+
	\frac{1}{\sqrt{N_{\rm A}}}
	\sum_{\bm{k} \bm{k}' \sigma}
	V_{\bm{k}}
	(n | \bm{k})
	\Big[
		d_{\sigma}^{\dagger}
		c_{\bm{k}\sigma}
		\
		,
		\
		c^{\dagger}_{\bm{k}' \sigma'}
		\Big]
	\nonumber \\ &
	+
	\sum_{\bm{k}' \sigma}
	\varepsilon_{d}
	( n | d )
	\Big[
		d^{\dagger}_{\sigma}
		d_{\sigma}
		\
		,
		\
		d^{\dagger}_{\sigma'}
		\Big]
	+
	\frac{1}{\sqrt{N_{\rm A}}}
	\sum_{\bm{k} \sigma}
	V_{\bm{k}} (n | d)
	\Big[
		c_{\bm{k}\sigma}^{\dagger}
		d_{\sigma}
		\
		,
		\
		d^{\dagger}_{\sigma'}
		\Big]
	\nonumber \\[4mm] &=
	\sum_{\bm{k} \bm{k}' \sigma}
	\varepsilon_{\bm{k}}
	( n | \bm{k}' )
	c^{\dagger}_{\bm{k} \sigma}
	\delta_{\bm{k}, \bm{k}'}
	\delta_{\sigma, \sigma'}
	+
	\frac{1}{\sqrt{N_{\rm A}}}
	\sum_{\bm{k} \bm{k}' \sigma}
	V_{\bm{k}} (n | \bm{k})
	d_{\sigma}^{\dagger}
	\delta_{\bm{k}, \bm{k}'}
	\delta_{\sigma, \sigma'}
	\nonumber \\ &
	+
	\sum_{\sigma}
	\varepsilon_{d}
	( n | d )
	d^{\dagger}_{\sigma}
	\delta_{\sigma, \sigma'}
	+
	\frac{1}{\sqrt{N_{\rm A}}}
	\sum_{\bm{k} \sigma}
	V_{\bm{k}} (n | d)
	c_{\bm{k}\sigma}^{\dagger}
	\delta_{\sigma, \sigma'}
	\nonumber \\[4mm] &=
	\sum_{\bm{k}}
	\bigg[
		\varepsilon_{\bm{k}}
		( n | \bm{k} )
		+
		\frac{1}{\sqrt{N_{\rm A}}}
		V_{\bm{k}} (n | d)
		\bigg]
	c^{\dagger}_{\bm{k} \sigma}
	+
	\bigg[
		\varepsilon_{d}
		( n | d )
		+
		\frac{1}{\sqrt{N_{\rm A}}}
		\sum_{\bm{k}}
		V_{\bm{k}} (n | \bm{k})
		\bigg]
	d^{\dagger}_{\sigma}
\end{align}

On the other hand, the left-hand side of the commutation relation $( \displaystyle \big[ \mathcal{H}_{0} , c^{\dagger}_{n \sigma} \big] = \varepsilon_{n} c^{\dagger}_{n \sigma} )$ is,

\begin{align}
	\varepsilon_{n}
	c_{n \sigma}^{\dagger}
	 & =
	\sum_{\bm{k}}
	\varepsilon_{n}
	( n | \bm{k} )
	c^{\dagger}_{\bm{k} \sigma}
	+
	\varepsilon_{n}
	( n | d )
	d^{\dagger}_{\sigma}
\end{align}

We can obtain the relations by comparing the coefficients.

\begin{align}
	\hspace{44.3mm}
	\varepsilon_{n}
	( n | \bm{k} )
	=
	\varepsilon_{\bm{k}}
	( n | \bm{k} )
	+
	\frac{1}{\sqrt{N_{\rm A}}}
	V_{\bm{k}} (n | d)
	\hspace{44.3mm}
\end{align}
\\[-10mm]
\begin{align}
	\hspace{41.5mm}
	\varepsilon_{n}
	( n | d )
	=
	\varepsilon_{d}
	( n | d )
	+
	\frac{1}{\sqrt{N_{\rm A}}}
	\sum_{\bm{k}}
	V_{\bm{k}} (n | \bm{k})
	\hspace{41.5mm}
\end{align}

Conversely, if we act the operator $\displaystyle \sum_{n} ( n | \bm{k} )^{*}$ on both sides,

\begin{align}
	c^{\dagger}_{n \sigma}
	 & =
	\sum_{\bm{k}'}
	( n | \bm{k}' )
	c^{\dagger}_{\bm{k}' \sigma}
	+
	( n | d )
	d^{\dagger}_{\sigma}
	\nonumber \\
	\hspace{26mm}
	\sum_{n} ( n | \bm{k} )^{*}
	c^{\dagger}_{n \sigma}
	 & =
	\sum_{\bm{k}'}
	\sum_{n}
	( n | \bm{k}' )
	( n | \bm{k} )^{*}
	c^{\dagger}_{\bm{k}' \sigma}
	+
	\sum_{n} ( n | \bm{k} )^{*}
	( n | d )
	d^{\dagger}_{\sigma}
	\nonumber \\ &=
	\sum_{\bm{k}'}
	\sum_{n}
	( n | \bm{k}' )
	( \bm{k} | n )
	c^{\dagger}_{\bm{k} \sigma}
	+
	\sum_{n}
	( \bm{k} | n )
	( n | d )
	d^{\dagger}_{\sigma}
	\nonumber \\ &=
	c^{\dagger}_{\bm{k} \sigma}
	\nonumber
	\hspace{80mm}
	(4.62)
\end{align}

If there is an orthogonality $\displaystyle
	(\bm{k}'|\bm{k})=\delta_{\bm{k}' \bm{k}}
$, we can obtain the operator relation.

In the same way, by taking the operator $\displaystyle \sum_{n} ( n | d )^{*}$ on both sides,

\begin{align}
	\hspace{26mm}
	\sum_{n} ( n | d )^{*}
	c^{\dagger}_{n \sigma}
	 & =
	\sum_{\bm{k}}
	\sum_{n}
	( n | \bm{k} )
	( n | d )^{*}
	c^{\dagger}_{\bm{k}' \sigma}
	+
	\sum_{n} ( n | d )^{*}
	( n | d )
	d^{\dagger}_{\sigma}
	\nonumber \\ &=
	\sum_{\bm{k}}
	\sum_{n}
	( d | n )
	( n | \bm{k} )
	c^{\dagger}_{\bm{k} \sigma}
	+
	d^{\dagger}_{\sigma}
	\sum_{n}
	\big|
	( n | d )
	\big|^{2}
	\nonumber \\ &=
	d^{\dagger}_{\sigma}
	\nonumber
	\hspace{82mm}
	(4.63)
\end{align}

is obtained.
Here, we have assumed that $\displaystyle
	\sum_{n}
	\big|
	( n | d )
	\big|^{2}
	=1
$.


The interaction term of the Hamiltonian used in the discussion of Fermi liquids was as follows.
\begin{align}
	\hspace{33.5mm}
	\mathcal{H}'_{\rm FL}
	=
	\frac{U}{N_{\rm A}}
	\sum_{\bm{k}_{1} \bm{k}_{2} \bm{k}_{3} \bm{k}_{4}}
	c_{\bm{k}_{1} \uparrow}^{\dagger}
	c_{\bm{k}_{2} \downarrow}^{\dagger}
	c_{\bm{k}_{3} \downarrow}
	c_{\bm{k}_{4} \uparrow}
	\delta_{\bm{k}_{1}+\bm{k}_{2} , \bm{k}_{3}+\bm{k}_{4}}
	\nonumber
	\hspace{33.5mm}
	(3.2')
\end{align}
Indeed, substituting (4.63) gives
\begin{align}
	&
	\mathcal{H} - \mathcal{H}_{0}
	=
	U
	d_{\uparrow}^{\dagger}
	d_{\uparrow}
	d_{\downarrow}^{\dagger}
	d_{\downarrow}
	\nonumber \\
	&=
	U
	\sum_{n_{1},n_{2},n_{3},n_{4}}
	(n_{1} | d)^{*}
	(n_{2} | d)
	(n_{3} | d)^{*}
	(n_{4} | d)
	c_{n_{1} \uparrow}^{\dagger}
	c_{n_{2} \uparrow}
	c_{n_{3} \downarrow}^{\dagger}
	c_{n_{4} \downarrow}
\end{align}
By analogy with equation (3.2), by expressing it in terms of a change of basis in this way, it appears as if it can be treated as a Fermi liquid.

${}$

The original spirit of Nozi$\acute{\rm e}$res's local Fermi liquid was to discover that the most important physical quantity in the problem of conduction electron scattering by an impurity potential is the phase shift $\eta$ of the conduction electrons near the Fermi surface.
All information about the scattering problem is contained in $\eta$.
Based on Landau's Fermi liquid theory, the state of the $s$-$d$ system can be described by the distribution function $n_{\sigma}$ of quasiparticles with spin $\sigma$.
The phase shift of the scattered quasiparticles then depends on the quasiparticle energy $\varepsilon_{\sigma}$ and the distribution.
Specifically, it can be written as follows:
\begin{align}
	\eta_{\sigma}(\varepsilon_{\sigma} , n_{\sigma'})
\end{align}
We expand this phase shift in terms of the deviation from the ground state distribution,
\begin{align}
	\delta n_{\sigma} = n_{\sigma} - n_{\sigma}^{(0)}
\end{align}
and take it to the first order:
\begin{align}
	\eta_{\sigma}(\varepsilon)
	=
	\eta_{\sigma}^{(0)}(\varepsilon)
	+
	\sum_{\sigma',\varepsilon'}
	\phi_{\sigma,\sigma'}(\varepsilon,\varepsilon')
	\delta n_{\sigma}(\varepsilon)
\end{align}
Furthermore, we expand $\eta_{\sigma}^{(0)}(\varepsilon)$ and the expansion coefficients $\phi_{\sigma,\sigma'}(\varepsilon,\varepsilon')$ in terms of the energy $\varepsilon$ measured from the chemical potential as follows:
\begin{align}
	\eta_{\sigma}^{(0)}(\varepsilon)
	 & =
	\eta_{\sigma}^{(0)}
	+
	\alpha_{1} \varepsilon
	+
	\alpha_{2} \varepsilon^{2}
	+
	\cdots
	\\
	\phi_{\sigma,\sigma'}(\varepsilon,\varepsilon')
	 & =
	\phi_{\sigma,\sigma'}^{(0)}
	+
	\beta_{1}
	\phi_{\sigma,\sigma'}(\varepsilon,\varepsilon')
	+
	\cdots
\end{align}
If we restrict the problem to low temperatures and low magnetic fields, we only need to take the first order with respect to $\varepsilon$, $T$, and $H$.
This is similar to the spirit of Fermi liquid theory discussed earlier.

${}$

With this motivation, this notebook will examine the density and spin responses within a range where the system can be treated as a Fermi liquid (i.e., at low temperatures and near the Fermi surface).
These are two-body correlation quantities and are related to the interaction function $f_{\sigma\sigma'}(\bm{k},\bm{k}')$ discussed in the chapter on Fermi liquids.
(
When considering perturbation theory, this corresponds to finding the vertex in the language of Green's functions.
In contrast, the part of the discussion on the $T$-matrix corresponds to discussing the one-body Green's function, where the deviation from the bare one-body Green's function was discussed through the self-energy.
Incidentally, an exact solution for the Anderson model using the Bethe ansatz is also known, and its solution can be expanded in a power series of $U$.
This supports the validity of discussing the model with a $U$ expansion, similar to the treatment of Fermi liquids in the previous chapter.
)

\section*{Green function}

The analysis using thermal Green's functions is superior in the following ways.

${}$

$\hspace{10mm} \textcircled{\footnotesize 1} $
It is convenient for perturbation calculations because we can use the technique of Feynman diagrams.

$\hspace{10mm} \textcircled{\footnotesize 2} $
By performing analytical continuation of the Matsubara frequency, we can obtain the retarded Green's function,

$\hspace{12mm}$
from which we can calculate dynamic physical quantities.

$\hspace{10mm} \textcircled{\footnotesize 3} $
It can also be combined with the path integral method.

${}$

Since it is convenient to use thermal Green's functions for this discussion as well, we will set them up here.

The $d$-electron Green's function\footnote{The word "thermal" for Green's functions will be omitted below.} is defined as follows.
\begin{align}
	G_{d \sigma}(\tau)
	=
	-
	\big\langle
	{\rm \hat{T}}_{\tau}
	\big[
		d_{\sigma}(\tau)
		d_{\sigma}^{\dagger}(0)
		\big]
	\big\rangle
\end{align}
The variable of the creation and annihilation operators is imaginary time $\tau=it$.
The operators are given time dependence using the Heisenberg representation.
\begin{align}
	d_{\sigma}(\tau)
	=
	e^{\mathcal{H}\tau}
	d_{\sigma}
	e^{-\mathcal{H}\tau}
\end{align}
This $\mathcal{H}$ is the Anderson Hamiltonian (including all perturbative terms).
$
	{\rm \hat{T}}_{\tau}
	\big[
		\cdots
		\big]
$
is the time-ordering operator, which arranges the Heisenberg operators in the brackets $[\cdots]$ from right to left in order of increasing time.
Since they are fermions, we must pay attention to the sign that appears when they are reordered.
The interaction-included average $\langle \cdots \rangle$ is taken as follows.
\begin{align}
	\langle \cdots \rangle
	=
	\frac{
		{\rm Tr} \big[ e^{- \beta \mathcal{H}} \cdots \big]
	}{
		{\rm Tr} \big[ e^{- \beta \mathcal{H}} \big]
	}
	\label{eqn:12ave}
\end{align}
$\beta=(k_{\rm B}T)^{-1}$ is the inverse temperature.

We can change the variable from imaginary time to energy via a Fourier transform.
\begin{align}
	G_{d \sigma}(i \varepsilon_{n})
	=
	\int^{\beta}_{0}
	d\tau
	e^{i \varepsilon_{n} \tau}
	G_{d \sigma}(\tau)
\end{align}
Here, $n$ is an integer, and $\varepsilon_{n}=(2n+1)\pi/\beta$ is the energy, which takes on discrete values and is called the fermionic Matsubara frequency.

The Green's function for conduction electrons is defined with a similar setup.
\begin{align}
	G_{\bm{k}\bm{k}' \sigma}(\tau)
	 & =
	-
	\big\langle
	{\rm \hat{T}}_{\tau}
	\big[
		c_{\bm{k} \sigma}(\tau)
		c_{\bm{k}' \sigma}^{\dagger}(0)
		\big]
	\big\rangle
	\\
	G_{\bm{k}\bm{k}' \sigma}(i \varepsilon_{n})
	 & =
	\int^{\beta}_{0}
	d\tau
	e^{i \varepsilon_{n} \tau}
	G_{\bm{k}\bm{k}' \sigma}(\tau)
\end{align}
Each Green's function can be distinguished by its index.

The equation of motion that $G_{\bm{k}\bm{k}' \sigma}(\tau)$ obeys can be expressed as,
\begin{align}
	G_{\bm{k}\bm{k}' \sigma}(\tau)
	=
	\frac{\delta_{\bm{k},\bm{k}'}}{
		i \varepsilon_{n} - \varepsilon_{\bm{k}}
	}
	+
	\frac{1}{
		i \varepsilon_{n} - \varepsilon_{\bm{k}}
	}
	T_{\sigma}(i \varepsilon_{n})
	\frac{1}{
		i \varepsilon_{n} - \varepsilon_{\bm{k}'}
	}
\end{align}
.
The first term represents the motion of an electron without an impurity (the bare Green's function), and the second term represents scattering by an impurity.
The $T$-matrix that describes this scattering is
\begin{align}
	T_{\sigma}(i \varepsilon_{n})
	=
	|V|^{2}
	G_{d \sigma}(i \varepsilon_{n})
\end{align}
and the energy-represented $G_{d \sigma}$ is
\begin{align}
	G_{d \sigma}(i \varepsilon_{n})
	=
	\frac{1}{
		\displaystyle
		i \varepsilon_{n}
		-
		\varepsilon_{d}
		-
		|V|^{2}
		\displaystyle
		\sum_{\bm{k}}
		\frac{1}{
			i \varepsilon_{n} - \varepsilon_{\bm{k}}
		}
		-
		\Sigma_{d\sigma}(i \varepsilon_{n})
	}
\end{align}
\footnote{
	This is consistent with the $T$-matrix defined in (4.65) of "Solid-State Electron Theory," but with $N_{\rm A}=1$.
}
.
The effects of the interaction are all included in the self-energy part of the $d$-electron, $\Sigma_{d\sigma}(i \varepsilon_{n})$.

Assuming that the conduction electron band is wide and that $\varepsilon_{n}$ is small compared to the bandwidth, the term in the denominator of $G_{d \sigma}(i \varepsilon_{n})$ that contains the sum over wave numbers (the wave number integral term of the free Green's function) can be approximated by the Cauchy principal value integral
\footnote{
	The Cauchy principal value integral for the following complex function is very useful and often appears.
	\begin{align}
		\lim_{\mu \to 0}
		\mathcal{P}
		\int
		\frac{dx}{x \pm i \mu}
		=
		\mathcal{P}
		\int
		\frac{dx}{x}
		\mp i \pi
		\delta(x)
	\end{align}
}
:
\begin{align}
	|V|^{2}
	\sum_{\bm{k}}
	\frac{1}{
		i \varepsilon_{n} - \varepsilon_{\bm{k}}
	}
	 & \simeq
	-
	i
	\pi
	|V|^{2}
	N(\varepsilon_{\rm F})
	{\rm sgn}(i \varepsilon_{n})
	\nonumber \\
	 & =
	-i \Delta {\rm sgn}(i \varepsilon_{n})
\end{align}
In this case, the $d$-electron Green's function simplifies to
\footnote{
	The sign function ${\rm sgn}$ that appears here is defined as
	\begin{align}
		{\rm sgn}(x)
		\ \!
		{=^{{}^{\hspace{-3mm} \rm def }}}
		\frac{x}{|x|}
	\end{align}
	.
	We deliberately write the argument as $i \varepsilon_{n}$ to make it clear that the variable is not a continuous variable but a discrete Matsubara frequency, and this sign function is a real function, not a complex one.
}
.
\begin{align}
	G_{d \sigma}(i \varepsilon_{n})
	=
	\frac{1}{
		\displaystyle
		i \varepsilon_{n}
		-
		\varepsilon_{d}
		+
		i \Delta {\rm sgn}(i \varepsilon_{n})
		-
		\Sigma_{d\sigma}(i \varepsilon_{n})
	}
\end{align}

\section*{Hartree-Fock approx.}

We will discuss an approach for when the perturbation $U$ is added, within the framework of the Hartree-Fock (HF) approximation.
(We will only introduce the results as this is a digression from the main topic.)

In the identity
\begin{align}
	U
	n_{d \uparrow}
	n_{d \downarrow}
	 & =
	U
	\langle n_{d \uparrow} \rangle_{{}_{\rm HF}}
	n_{d \downarrow}
	+
	U
	\langle n_{d \downarrow} \rangle_{{}_{\rm HF}}
	n_{d \uparrow}
	-
	U
	\langle n_{d \uparrow} \rangle_{{}_{\rm HF}}
	\langle n_{d \downarrow} \rangle_{{}_{\rm HF}}
	\nonumber \\ &+
	U
	\Big(
	n_{d \uparrow}
	-
	\langle n_{d \uparrow} \rangle_{{}_{\rm HF}}
	\Big)
	\Big(
	n_{d \downarrow}
	-
	\langle n_{d \downarrow} \rangle_{{}_{\rm HF}}
	\Big)
	\label{eqn:HF17}
\end{align}
, the HF approximation neglects only the last term on the right-hand side, which corresponds to fluctuations.

Adding the first and second terms on the right-hand side is equivalent to replacing $\varepsilon_{d}$ with $\varepsilon_{d} + U \langle n_{d \uparrow} \rangle_{{}_{\rm HF}}$ and $\varepsilon_{d} + U \langle n_{d \downarrow} \rangle_{{}_{\rm HF}}$, respectively, in the original Hamiltonian (4.1).

The third term on the right-hand side is an offset term that does not change the physical quantities when taking the expectation value.

The coefficients $\langle n_{d \uparrow} \rangle_{{}_{\rm HF}}$ and $\langle n_{d \downarrow} \rangle_{{}_{\rm HF}}$ are determined by self-consistently solving the following coupled equations.
\begin{align}
	\langle n_{d \sigma} \rangle_{{}_{\rm HF}}
	 & =
	\int d \varepsilon
	f(\varepsilon) N_{d \sigma}(\varepsilon)
	\label{eqn:motonon}
	\\
	N_{d \sigma}(\varepsilon)
	 & =
	-
	\frac{1}{\pi}
	{\rm Im}G_{d \sigma}(i \varepsilon_{n} \to \varepsilon + i0)
\end{align}

Specifically, at absolute zero,
\begin{align}
	\langle n_{d \sigma} \rangle_{{}_{\rm HF}}
	 & =
	\int d \varepsilon
	\ \!
	\theta(- \varepsilon) N_{d \sigma}(\varepsilon)
	\nonumber \\ &=
	\frac{1}{\pi}
	\int^{0}_{-\infty}
	\!\!
	d \varepsilon
	\frac{
		\Delta
	}{
		\big(
		\varepsilon
		-
		\varepsilon_{d}
		-
		U
		\langle n_{d, -\sigma} \rangle_{{}_{\rm HF}}
		\big)^{2}
		+
		\Delta^{2}
	}
	\nonumber \\ &=
	\frac{1}{2}
	-
	\frac{1}{\pi}
	{\rm arctan}
	\frac{
		\varepsilon_{d}
		+
		U
		\langle n_{d, -\sigma} \rangle_{{}_{\rm HF}}
	}{
		\Delta
	}
	\label{eqn:hfnumber}
\end{align}
and an analytical solution can be found.
Depending on the magnitude of $U$ relative to the threshold $\dfrac{1}{N_{d}(\varepsilon_{\rm F})}$, there are two types of solutions.

For $U < \dfrac{1}{N_{d}(\varepsilon_{\rm F})}$,
$
	\langle n_{d \sigma} \rangle_{{}_{\rm HF}}
	=
	\langle n_{d, -\sigma} \rangle_{{}_{\rm HF}}
$
is the solution, which means that the non-magnetic state is stable.

For $U > \dfrac{1}{N_{d}(\varepsilon_{\rm F})}$,
$
	\langle n_{d \sigma} \rangle_{{}_{\rm HF}}
	\neq
	\langle n_{d, -\sigma} \rangle_{{}_{\rm HF}}
$
is the solution, which means a magnetic state is the solution.

\ \\[2mm]

The HF approximation is valid when the fluctuations neglected in equation (\ref{eqn:HF17}) are small.
However, these fluctuation effects are crucial when calculating quantities such as the impurity susceptibility $\chi_{ds}$, as we will see later.
In the HF approximation
(or more precisely, the Random Phase Approximation (RPA)
\footnote{
	When attempting to solve for such two-body correlations, the framework corresponding to the HF approximation is called the RPA.
}
), the impurity susceptibility $\chi_{ds}$ is a series that uses the dimensionless interaction $u=\dfrac{U}{\pi \Delta}$,
\begin{align}
	\chi_{ds}
	\sim
	1
	+
	u
	+
	u^{2}
	+
	u^{3}
	+
	\cdots
	=
	\frac{1}{1-u}
\end{align}
and diverges at $u=1$.
However, the coefficients of higher-order terms in $U$ for the correct $\chi_{ds}$ are smaller than those calculated by the mean field.
This means that within the mean-field approximation, which neglects these fluctuations, $\chi_{ds}$ is overestimated, and when fluctuations are included, the higher-order contributions cancel out, and the series gradually converges.
This will be demonstrated in the next section.

\section*{Ground state}

This is a slight digression from the content of the seminar, so I will only present the results.

We pointed out that the HF approximation, which treats $U$ as a perturbation, leads to a geometric series-like divergence.
I will now introduce the danger of the HF approximation by discussing the exact solution for the ground state, which includes $U=\infty$.

Let's consider a situation where there is, on average, one electron in the $d$ orbital, i.e., $
	\langle n_{d \sigma} \rangle
	=
	\langle n_{d, -\sigma} \rangle
	=
	\dfrac{1}{2}
$.

Yamada has shown
\footnote{
	K. Yamada: Prog. Theor. Phys. {\bf 53}, 970 (1975)
}
that the difference between the dimensionless ground state energy $\varepsilon_{g}(u=0)$ of the Anderson Hamiltonian with $U=0$ divided by $\pi \Delta$ and the dimensionless ground state energy $\varepsilon_{g}(u)$ with an interaction $U$ is
\begin{align}
	\varepsilon_{g}(u)-
	\varepsilon_{g}(u=0)
	=
	\frac{1}{4}
	u
	-
	\bigg[
		\frac{1}{4}
		-
		\frac{7}{4 \pi^{2}}
		\zeta(3)
		\bigg]
	u^{2}
	+
	0.000795u^{4}
	+
	\mathcal{O}(u^{6})
	\label{eqn:yamadakousaku}
\end{align}
.

In fact, an exact solution for the Anderson model, treating $U$ as a perturbation, has been obtained based on the Bethe ansatz, and Ueda and Apel's calculations
\footnote{
	K. Ueda and W. Apel: J. Phys. C{\bf 16}, L849 (1983)
}
have already shown that
\begin{align}
	\hspace{-5mm}
	\varepsilon_{g}(u)-
	\varepsilon_{g}(u=0)
	 & =
	-
	\frac{\Delta}{2}
	u
	+
	4 \Delta
	u
	\sum_{M=1}^{\infty}
	C_{2M}
	u^{2M}
	+
	\frac{\Delta}{\pi}
	\Big[
	{\rm log}(1 + u^{2})
	-
	2 u \ \!
	{\rm arctan}
	(u)
	\Big]
	\\
	C_{2M}
	 & =
	\sum_{n=0}^{M}
	(-1)^{2M+n}
	\frac{(4M-2n)!}{(2n)!(2M-2n)!}
	\pi^{-2M+2n-2}
	\Big( 1-2^{-2M+2n-2} \Big)
	\zeta(2M-2n+2)
	\hspace{15mm}
\end{align}
.
In particular, the coefficient of the fourth-order term in $u$ is $
	\displaystyle \frac{\pi^{2}}{96}
	-
	\displaystyle \frac{21}{8}
	\zeta(3)
	+
	\displaystyle \frac{30}{\pi^{2}}
	\displaystyle \frac{31}{32}
	\zeta(5)
$, which is consistent with equation (\ref{eqn:yamadakousaku}).
In the case of half-filling we are considering, due to electron-hole symmetry, only even-order terms in $u$ remain, except for the first-order term.

As an example of the two-body correlation discussed in the previous section, let's look at the coefficients of the quantity below, which is obtained from $E_{g}$:
\begin{align}
	\langle n_{d \uparrow} n_{d \downarrow} \rangle
	 & =
	\frac{\partial E_{g}}{\partial U}
	=
	\frac{1}{4}
	-
	\bigg[
		\frac{1}{2}
		-
		\frac{7}{2 \pi^{2}}
		\zeta(3)
		\bigg]
	u
	+
	0.0032 u^{3}
	+
	\mathcal{O}(u^{5})
	\label{eqn:4_4_21}
\end{align}
.
Perturbation theory cannot find the coefficients of higher-order terms in $U$, but physically, the coefficients should approach 0 as $U \to \infty$.
In fact, equation (\ref{eqn:4_4_21}) decreases monotonically from its maximum value of $\dfrac{1}{4}$
\footnote{
	In a very rough one-body approximation, if we separate the product of operators, we get $
		\langle n_{d \uparrow} n_{d \downarrow} \rangle
		\to
		\langle n_{d \uparrow} \rangle
		\langle n_{d \downarrow} \rangle
		=
		\dfrac{1}{4}
	$
}.
This mathematical decrease represents the physical effect of electron correlation, and it is thought that higher-order terms in $U$ are canceled out by effects such as electron-hole polarization fluctuations (e.g., Thomas-Fermi screening), resulting in a decrease.

${}$

In this section, I wanted to discuss the dangers of the mean-field approximation, so I have portrayed the HF approximation and RPA as villains.
Since it would be cruel to bully them too much, I will conclude with a defense.

After all, for any many-body system, the most fundamental approach to try first is the mean-field approximation.
It is computationally much lighter than second-order perturbation theory or RPA and may even be solvable by hand.
(The possibility of finding an analytical solution is higher than with any other approach.)

There is still work to be done with the mean-field approximation.
For example, Mr. Tomura, a senior in Professor Ueda's lab, calculated a Hubbard model with 1/5 depletion within the mean-field framework and discovered a Dirac cone in the band dispersion
\footnote{
	Masaki Tomura, Master's thesis (University of Tokyo, 2013 [scheduled for submission])
}.

\section*{Impurity susceptibility}

We will add the following term as a perturbation to the Anderson Hamiltonian
\footnote{
	Note that $n_{d \sigma}=d_{\sigma}^{\dagger}
		d_{\sigma}$
	is an operator, while
	$n_{\sigma}$
	is a c-number.
	The symbols are similar, so be careful not to confuse them.
}
.
\begin{align}
	\hspace{55.5mm}
	\mathcal{H}'= - \sum_{\sigma}
	h_{\sigma}
	d_{\sigma}^{\dagger}
	d_{\sigma}
	\nonumber
	\hspace{55.5mm}
	(4.73)
\end{align}
$h_{\sigma}$ represents a spin-dependent perturbation, such as the Zeeman energy
\begin{align}
	h_{\sigma}
	=
	\frac{\sigma}{2}
	g
	\mu_{{}_{\rm B}}
	H
\end{align}
.

The change in the number of electrons, expanded as a power series in the linear response regime with respect to the perturbation $h_{\sigma}$, is
\begin{align}
	\bigg(
	n_{\sigma}
	+
	\frac{\partial n_{\sigma}}{\partial h_{\sigma'}}
	\bigg|_{h_{\sigma'} \to 0}
	h_{\sigma'}
	+
	\mathcal{O}({h_{\sigma}}^{2})
	\bigg)
	-
	n_{\sigma}
	 & \simeq
	\frac{\partial n_{\sigma}}{\partial h_{\sigma'}}
	\bigg|_{h_{\sigma'} \to 0}
	h_{\sigma'}
\end{align}
.
We consider the change in the number of electrons to the lowest order, which is the first order.
Using the definition of the average, equation (\ref{eqn:12ave}),
\begin{align}
	\hspace{35.5mm}
	\frac{\partial n_{\sigma}}{\partial h_{\sigma'}}
	\bigg|_{h_{\sigma'} \to 0}
	=
	\frac{\partial}{\partial h_{\sigma'}}
	\frac{
		{\rm Tr} \big[ e^{- \beta (\mathcal{H}+\mathcal{H}') } n_{d \sigma} \big]
	}{
		{\rm Tr} \big[ e^{- (\mathcal{H}+\mathcal{H}') } \big]
	}
	\bigg|_{h_{\sigma'} \to 0}
	\nonumber
	\hspace{35.5mm}
	(4.74)
\end{align}
Let's examine the right-hand side of (4.74).
The density matrix $e^{- \beta (\mathcal{H}+\mathcal{H}')}$ can be expanded using the $S$-matrix as follows:
\begin{align}
	\hspace{22.4mm}
	e^{- \beta (\mathcal{H}+\mathcal{H}') }
	 & =
	e^{- \beta \mathcal{H} }
	+
	e^{- \beta \mathcal{H} }
	\ \!
	\hat{\rm T}_{\tau}
	{\rm exp}
	\bigg[
		-
		\int^{\beta}_{0}d \tau
		\mathcal{H}'(\tau)
		d \tau
		\bigg]
	\nonumber \\
	 & =
	e^{- \beta \mathcal{H} }
	+
	e^{- \beta \mathcal{H} }
	\int^{\beta}_{0}d \tau
	e^{\tau \mathcal{H}}
	( - \mathcal{H}' )
	e^{- \tau \mathcal{H}}
	+
	\mathcal{O}({h_{\sigma}}^{2})
	\nonumber
	\hspace{22.4mm}
	(4.75)
\end{align}
Using this, and the linearity of the trace,
\begin{align}
	&
	{\rm Tr}
	\bigg[
		\Big\{
		e^{- \beta \mathcal{H} }
		+
		e^{- \beta \mathcal{H} }
		\int^{\beta}_{0}d \tau
		e^{\tau \mathcal{H}}
		( - \mathcal{H}' )
		e^{- \tau \mathcal{H}}
		\Big\}
		n_{d \sigma}
		\bigg]
	\nonumber \\
	 & =
	{\rm Tr}
	\Big[
		e^{- \beta \mathcal{H} }
		n_{d \sigma}
		\Big]
	+
	\sum_{\sigma'}
	{\rm Tr}
	\bigg[
		e^{- \beta \mathcal{H} }
		\int^{\beta}_{0}d \tau
		e^{\tau \mathcal{H}}
		h_{\sigma'}
		n_{d \sigma'}
		e^{- \tau \mathcal{H}}
		n_{d \sigma}
		\bigg]
	\nonumber
	\\
	 & =
	{\rm Tr}
	\Big[
		e^{- \beta \mathcal{H} }
		n_{d \sigma}
		\Big]
	+
	\sum_{\sigma'}
	{\rm Tr}
	\bigg[
		e^{- \beta \mathcal{H} }
		\int^{\beta}_{0}d \tau
		h_{\sigma'}
		n_{d \sigma'}(\tau)
		n_{d \sigma}(0)
		\bigg]
\end{align}
so
\begin{align}
	\frac{
		{\rm Tr} \big[ e^{- \beta (\mathcal{H}+\mathcal{H}') } n_{d \sigma} \big]
	}{
		{\rm Tr} \big[ e^{- (\mathcal{H}+\mathcal{H}') } \big]
	}
	 & =
	\dfrac{
		{\rm Tr}
		\Big[
			e^{- \beta \mathcal{H} }
			n_{d \sigma}
			\Big]
		+
		\displaystyle \sum_{\sigma'}
		{\rm Tr}
		\bigg[
			e^{- \beta \mathcal{H} }
			\int^{\beta}_{0}d \tau
			h_{\sigma'}
			n_{d \sigma'}(\tau)
			n_{d \sigma}(0)
			\bigg]
	}{
		{\rm Tr}
		\Big[
			e^{- \beta \mathcal{H} }
			\Big]
	}
	\nonumber \\ &=
	\langle n_{d \sigma} \rangle
	+
	\sum_{\sigma'}
	\int^{\beta}_{0}d \tau
	\langle
	h_{\sigma'}
	n_{d \sigma'}(\tau)
	n_{d \sigma}(0)
	\rangle
\end{align}
When we differentiate with respect to $h_{\sigma}$, the first term is a constant and disappears.
The second term that remains is:
\begin{align}
	\frac{\partial}{\partial h_{\sigma'}}
	\sum_{\sigma''}
	\int^{\beta}_{0}d \tau
	\langle
	h_{\sigma''}
	n_{d \sigma''}(\tau)
	n_{d \sigma}(0)
	\rangle
	 & =
	\int^{\beta}_{0}d \tau
	\langle
	n_{d \sigma'}(\tau)
	n_{d \sigma}(0)
	\rangle
\end{align}
We found that the susceptibility can be calculated from the density-density response.

\section*{Generalized Friedel sum rule}

Let's look at the left-hand side of (4.74).
The local change in the total number of electrons due to the impurity is the change in the conduction electrons.
The number of electrons can be calculated by integrating the density of states
\footnote{
	The reason for this rather obscure notation is that we want to perform analytical continuation $i \varepsilon_{n} \to \omega + i0$ partway through.
	Therefore, $i \varepsilon_{n}$ cannot be taken outside of the $\omega$-integral.
}
.
\begin{align}
	n_{\sigma}
	 & =
	-
	\frac{1}{\pi}
	{\rm Im}
	\bigg[
		\int^{\infty}_{-\infty}
		d \omega
		f(\omega)
		G_{d \sigma}(i \varepsilon_{n})
		\bigg]
	-
	\frac{1}{\pi}
	{\rm Im}
	\sum_{\bm{k} \bm{k}'}
	\bigg[
		\int^{\infty}_{-\infty}
		d \omega
		f(\omega)
		G_{\bm{k} \bm{k}' \sigma}(i \varepsilon_{n})
		\bigg]
	\nonumber \\[3mm] &=
	-
	\frac{1}{\pi}
	{\rm Im}
	\bigg[
		\int^{\infty}_{-\infty}
		d \omega
		f(\omega)
		G_{d \sigma}(i \varepsilon_{n})
		\bigg]
	\nonumber \\ &
	-
	\frac{1}{\pi}
	{\rm Im}
	\sum_{\bm{k}}
	\bigg[
		\int^{\infty}_{-\infty}
		d \omega
		f(\omega)
		\bigg\{
		\frac{1}{
			i \varepsilon_{n} - \varepsilon_{\bm{k}}
		}
		+
		\frac{1}{
			i \varepsilon_{n} - \varepsilon_{\bm{k}}
		}
		T_{\sigma}(i \varepsilon_{n})
		\frac{1}{
			i \varepsilon_{n} - \varepsilon_{\bm{k}}
		}
		\bigg\}
		\bigg]
	\nonumber \\[3mm] &=
	\int^{\infty}_{-\infty}
	d \omega
	f(\omega)
	\bigg(- \frac{1}{\pi} \bigg)
	{\rm Im}
	\bigg[
		\frac{\partial}{\partial i \varepsilon_{n}}
		{\rm log}
		\bigg\{
		- \frac{1}{G_{d \sigma}(i \varepsilon_{n})}
		\bigg\}
		+
		G_{d \sigma}(i \varepsilon_{n})
		\frac{\partial}{\partial i \varepsilon_{n}}
		\Sigma_{d \sigma}(i \varepsilon_{n})
		\bigg]
\end{align}
The second term inside the imaginary part of the last right-hand side can be dropped due to the identity obtained by Luttinger
\footnote{
	J. M. Luttinger: Phys. Rev. {\bf 119}, 1153 (1960) ;
	J. M. Luttinger and J. C. Ward: Phys. Rev. {\bf 118}, 1417 (1960)
	\hspace{3mm}
	This identity was used to prove Luttinger's theorem (the volume enclosed by the Fermi surface does not change due to interaction).
}
:
\begin{align}
	\lim_{T \to 0}
	\int^{\infty}_{-\infty}
	d \omega
	f(\omega)
	\bigg(- \frac{1}{\pi} \bigg)
	{\rm Im}
	\bigg[
		G_{d \sigma}(i \varepsilon_{n})
		\frac{\partial}{\partial i \varepsilon_{n}}
		\Sigma_{d \sigma}(i \varepsilon_{n})
		\bigg]
	 & =
	0
\end{align}
To use this, we will assume absolute zero for the rest of this section.
Applying analytical continuation $i \varepsilon_{n} \to \omega + i0$, we get
\begin{align}
	n_{\sigma}
	 & =
	\int^{\infty}_{-\infty}
	d \omega
	\theta(-\omega)
	\bigg(- \frac{1}{\pi} \bigg)
	{\rm Im}
	\bigg[
		\frac{\partial}{\partial i \varepsilon_{n}}
		{\rm log}
		\bigg\{
		- \frac{1}{G_{d \sigma}(i \varepsilon_{n})}
		\bigg\}
		\bigg]
	\nonumber \\[3mm] &=
	- \frac{1}{\pi}
	{\rm Im}
	\bigg[
		{\rm log}
		\bigg\{
		\varepsilon_{d}
		+
		\Sigma_{d\sigma}(+i 0)
		-
		i \Delta
		\bigg\}
		\bigg]
\end{align}
.
Assuming that this logarithmic function returns the principal value ${\rm log} z = {\rm log} |z| + i \ \! {\rm arg} z$, we get
\begin{align}
	n_{\sigma}
	 & =
	- \frac{1}{\pi}
	{\rm arg}
	\bigg\{
	\varepsilon_{d}
	+
	\Sigma_{d\sigma}(+i 0)
	-
	i \Delta
	\bigg\}
	\nonumber \\ &=
	\frac{1}{\pi}
	{\rm arctan}
	\frac{\varepsilon_{d}
		+
		\Sigma_{d\sigma}(+i 0)
	}{
		\Delta
	}
\end{align}

${}$

The Green's function $G_{d \sigma}(i \varepsilon_{n})$ contains information about the correlation acting between $d_{\sigma}(\tau)$ and $d^{\dagger}_{\sigma}(0)$.
When we find the phase shift $\eta_{\sigma}$ on the Fermi surface $(\omega=0)$ in this correlation, we get
\begin{align}
	\eta_{\sigma}
	 & =
	{\rm arg}
	\Big[ G_{d \sigma}(i \varepsilon_{n}) \Big]
	\ = \
	{\rm arctan}
	\frac{
		{\rm Re} \big\{ G_{d \sigma}(i \varepsilon_{n}) \big\}
	}{
		{\rm Im} \big\{ G_{d \sigma}(i \varepsilon_{n}) \big\}
	}
	\nonumber \\ &=
	{\rm arctan}
	\frac{
		\varepsilon_{d} + \Sigma_{d \sigma} (+i 0)
	}{
		\Delta
	}
	\nonumber \\ &=
	\pi n_{\sigma}
\end{align}
which shows that the phase shift due to the interaction and the change in the number of electrons due to the interaction correspond one-to-one
\footnote{
	This is an extension of the Friedel sum rule from Kirii-kun's section to include interactions.
}.
This is the origin of what Nozi$\acute{\rm e}$res called the local Fermi liquid.
Similar to Landau's Fermi liquid theory, finding the phase shift $\eta_{\sigma}$ leads to finding all physical quantities with interactions (under the approximations included in Fermi liquid theory).

\section*{Impurity susceptibility \ 2}

Applying the perturbation from equation (4.73) shifts the impurity level from $\varepsilon_{d}$ to $\varepsilon_{d} - h_{\sigma}$.
Since the Friedel sum rule holds even when interactions are applied,
\begin{align}
	\frac{\partial n_{\sigma}}{\partial h_{\sigma'}}
	\bigg|_{h_{\sigma'} \to 0}
	 & =
	- \frac{1}{\pi}
	{\rm Im}
	\frac{\partial }{\partial h_{\sigma'}}
	{\rm log}
	\Big\{
	\varepsilon_{d}
	-
	h_{\sigma}
	+
	\Sigma_{d\sigma}(+i 0)
	-
	i \Delta
	\Big\}
	\bigg|_{h_{\sigma'} \to 0}
	\nonumber \\[3mm] &=
	\frac{1}{\pi}
	{\rm Im}
	\frac{
		\delta_{\sigma,\sigma'}
		-
		\dfrac{\partial \Sigma_{d \sigma}(+i0)}{\partial h_{\sigma'}}
		\bigg|_{h_{\sigma'} \to 0}
	}
	{
		\varepsilon_{d}
		+
		\Sigma_{d\sigma}(+i 0)
		-
		i \Delta
	}
	\nonumber \\[3mm] &=
	\frac{1}{\pi}
	\frac{
		\Delta
	}
	{
		\{
		\varepsilon_{d}
		+
		\Sigma_{d\sigma}(+i 0)
		\}^{2}
		+
		\Delta^{2}
	}
	\bigg[
		\delta_{\sigma,\sigma'}
		-
		\dfrac{\partial \Sigma_{d \sigma}(+i0)}{\partial h_{\sigma'}}
		\bigg|_{h_{\sigma'} \to 0}
		\bigg]
\end{align}
The coefficient in the parentheses represents the density of states at the Fermi energy with the full interaction included.
\begin{align}
	\frac{1}{\pi}
	\frac{
		\Delta
	}
	{
		\{
		\varepsilon_{d}
		+
		\Sigma_{d\sigma}(+i 0)
		\}^{2}
		+
		\Delta^{2}
	}
	\ = \
	-
	\frac{1}{\pi}
	{\rm Im}G_{d \sigma}(\omega=0)
	\ = \
	N_{d}(\varepsilon_{\rm F})
\end{align}
From the previously derived equation
\begin{align}
	\hspace{39.4mm}
	\frac{
		\partial
		n_{\sigma}
	}{
		\partial
		h_{\sigma'}
	}
	\bigg|_{h_{\sigma'} \to 0}
	 & =
	\int^{\beta}_{0}d \tau
	\langle
	\tilde{n}_{d \sigma'}(\tau)
	\tilde{n}_{d \sigma}(0)
	\rangle
	\nonumber
	\hspace{39.4mm}
	(4.76)
\end{align}
, we can say that
\begin{align}
	\int^{\beta}_{0}d \tau
	\langle
	\tilde{n}_{d \sigma'}(\tau)
	\tilde{n}_{d \sigma}(0)
	\rangle
	 & =
	N_{d}(\varepsilon_{\rm F})
	\bigg[
		\delta_{\sigma,\sigma'}
		-
		\dfrac{\partial \Sigma_{d \sigma}(+i0)}{\partial h_{\sigma'}}
		\bigg|_{h_{\sigma'} \to 0}
		\bigg]
\end{align}
.
(Here, we have set $ \tilde{n}_{d \sigma} = n_{d \sigma} - \langle n_{d \sigma} \rangle $.)

${}$

Now, we substitute $
	\langle S \rangle
	=
	\dfrac{1}{2}
	\big(
	\tilde{n}_{d \uparrow} - \tilde{n}_{d \downarrow}
	\big)
$
and $
	\langle N \rangle
	=
	\dfrac{1}{2}
	\big(
	\tilde{n}_{d \uparrow} + \tilde{n}_{d \downarrow}
	\big)
$
into $n_{\sigma}$ in equation (4.76).
The impurity spin susceptibility $\chi_{ds}$ and charge susceptibility $\chi_{dc}$ are defined as follows:
\begin{align}
	\frac{
		\partial
		\langle S \rangle
	}{
		\partial
		h_{\sigma'}
	}
	\bigg|_{h_{\sigma'} \to 0}
	 & =
	\chi_{ds}
	\\
	\frac{
		\partial
		\langle N \rangle
	}{
		\partial
		h_{\sigma'}
	}
	\bigg|_{h_{\sigma'} \to 0}
	 & =
	\chi_{dc}
\end{align}
$\chi_{ds}$ represents how much the difference between the number of spin-up and spin-down electrons changes when a magnetic field $h_{\sigma}$ is applied, while $\chi_{dc}$ shows how much the total number of electrons changes (since charge corresponds to the number of electrons).
Therefore,
\begin{align}
	\hspace{30.7mm}
	\chi_{ds}
	 & =
	\int^{\beta}_{0}d \tau
	\Big\langle
	\dfrac{1}{2}
	\big(
	\tilde{n}_{d \uparrow} - \tilde{n}_{d \downarrow}
	\big)(\tau)
	\dfrac{1}{2}
	\big(
	\tilde{n}_{d \uparrow} - \tilde{n}_{d \downarrow}
	\big)(0)
	\Big\rangle
	\nonumber
	\hspace{30.7mm}
	(4.77)
\end{align}
\\[-12mm]
\begin{align}
	\hspace{30.7mm}
	\chi_{dc}
	 & =
	\int^{\beta}_{0}d \tau
	\Big\langle
	\dfrac{1}{2}
	\big(
	\tilde{n}_{d \uparrow} + \tilde{n}_{d \downarrow}
	\big)(\tau)
	\dfrac{1}{2}
	\big(
	\tilde{n}_{d \uparrow} + \tilde{n}_{d \downarrow}
	\big)(0)
	\Big\rangle
	\nonumber
	\hspace{30.7mm}
	(4.78)
\end{align}
Similarly, the longitudinal susceptibility $\chi_{\uparrow \uparrow}$ and transverse susceptibility $\chi_{\uparrow \downarrow}$ are defined as follows:
\begin{align}
	\hspace{36mm}
	\chi_{\uparrow \uparrow}
	 & =
	\int^{\beta}_{0}d \tau
	\langle
	\tilde{n}_{d \uparrow}(\tau)
	\tilde{n}_{d \uparrow}(0)
	\rangle
	\ = \
	\chi_{ds} + \chi_{dc}
	\nonumber
	\hspace{36mm}
	(4.79)
\end{align}
\\[-12mm]
\begin{align}
	\hspace{34.7mm}
	\chi_{\uparrow \downarrow}
	 & =
	\int^{\beta}_{0}d \tau
	\langle
	\tilde{n}_{d \uparrow}(\tau)
	\tilde{n}_{d \downarrow}(0)
	\rangle
	\ = \
	- \chi_{ds} + \chi_{dc}
	\nonumber
	\hspace{34.7mm}
	(4.80)
\end{align}

From equation (44), the longitudinal susceptibility $\chi_{\uparrow \uparrow}$ and transverse susceptibility $\chi_{\uparrow \downarrow}$ can be expressed using the self-energy:
\begin{align}
	\chi_{\uparrow \uparrow}
	 & =
	\int^{\beta}_{0}d \tau
	\langle
	\tilde{n}_{d \uparrow}(\tau)
	\tilde{n}_{d \uparrow}(0)
	\rangle
	\ = \
	N_{d}(\varepsilon_{\rm F})
	\bigg[
		1
		-
		\frac{\partial \Sigma_{d \uparrow} (+i0) }{\partial h_{\uparrow}}
		\bigg|_{h_{\uparrow} \to 0}
		\bigg]
	\\
	\chi_{\uparrow \downarrow}
	 & =
	\int^{\beta}_{0}d \tau
	\langle
	\tilde{n}_{d \uparrow}(\tau)
	\tilde{n}_{d \downarrow}(0)
	\rangle
	\ = \
	-
	N_{d}(\varepsilon_{\rm F})
	\frac{\partial \Sigma_{d \uparrow} (+i0) }{\partial h_{\downarrow}}
	\bigg|_{h_{\downarrow} \to 0}
\end{align}
By solving these simultaneous equations in reverse, we get
\begin{align}
	\chi_{ds}
	 & =
	\frac{1}{2}
	\big(
	\chi_{\uparrow \uparrow}
	-
	\chi_{\uparrow \downarrow}
	\big)
	\ = \
	\frac{1}{2}
	N_{d}(\varepsilon_{\rm F})
	\bigg[
		1
		-
		\frac{\partial \big\{ \Sigma_{d \uparrow} (+i0) - \Sigma_{d \downarrow} (+i0) \big\} }{\partial h_{\uparrow}}
		\bigg|_{h_{\uparrow} \to 0}
		\bigg]
	\\
	\chi_{dc}
	 & =
	\frac{1}{2}
	\big(
	\chi_{\uparrow \uparrow}
	+
	\chi_{\uparrow \downarrow}
	\big)
	\ = \
	\frac{1}{2}
	N_{d}(\varepsilon_{\rm F})
	\bigg[
		1
		-
		\frac{\partial\big\{ \Sigma_{d \uparrow} (+i0) + \Sigma_{d \downarrow} (+i0) \big\} }{\partial h_{\uparrow}}
		\bigg|_{h_{\uparrow} \to 0}
		\bigg]
\end{align}
.

\section*{Wilson ratio}

The contribution of the impurity to the specific heat $C$
\footnote{We have omitted the symbol $\Delta$ for "contribution." This means $C \to \Delta C$, $\Omega \to \Delta \Omega$.}
can be found from the contribution to the thermodynamic potential $\Omega$ using
\begin{align}
	C
	 & =
	-T
	\frac{\partial^{2} \Omega}{\partial T^{2}}
\end{align}
.
The ratio of the first-order coefficient (Sommerfeld coefficient) $\gamma$ from the low-temperature expansion of the specific heat contribution at $T=0$,
\begin{align}
	\gamma
	 & =
	\frac{C}{T}
	\
	+
	\mathcal{O}(T)
\end{align}
and the charge susceptibility $\chi_{dc}$ is called the Wilson ratio,
\begin{align}
	R_{\rm W}
	 & =
	\frac{\chi_{ds}/\chi_{ds}^{(0)}}{\gamma/\gamma^{(0)}}
\end{align}
. The superscript "($^{(0)}$)" denotes the contribution from only the conduction electrons, i.e., $U=0$.

To find $R_{\rm W}$, we need to find $\gamma$.
If we perform a perturbation expansion of the thermodynamic potential (or a calculation using vertex functions and the Ward identity), the Sommerfeld coefficient can be expressed using the self-energy as follows:
\footnote{
	Refer to Hiroyuki Shiba, "Physics of Electron Correlation";
	Kousaku Yamada, "Electron Correlation";
	A. C. Hewson, "The Kondo Problem to Heavy Fermions"; \\
	A. A. Abrikosov, L. P. Gorkov and I. E. Dzyaloshinski, "Methods of Quantum Field Theory in Statistical Physics," etc.
}
\begin{align}
	\gamma
	 & =
	\frac{2 \pi^{2}}{3}
	k_{\rm B}^{2}
	N_{d}(\varepsilon_{\rm F})
	\bigg[
		1
		-
		\frac{\partial \Sigma_{d \sigma} (i \varepsilon_{n}) }{\partial i \varepsilon_{n}}
		\bigg|_{\omega \to 0}
		\bigg]
	\label{eqn:54num}
\end{align}
Furthermore, it can be shown through the Ward identity that
\begin{align}
	\frac{\partial \Sigma_{d \sigma} (i \varepsilon_{n}) }{\partial i \varepsilon_{n}}
	\bigg|_{\omega \to 0}
	 & =
	\dfrac{\partial \Sigma_{d \sigma}(+i0)}{\partial h_{\sigma}}
	\bigg|_{h_{\sigma} \to 0}
	\label{eqn:55num}
\end{align}
. $21$
From these relations, equations (\ref{eqn:54num}), (\ref{eqn:55num}), and (4.76), we can see that
\begin{align}
	\bigg(
	\dfrac{2 \pi^{2}}{3}
	k_{\rm B}^{2}
	\bigg)^{-1}
	\gamma
	 & =
	\int^{\beta}_{0}d \tau
	\langle
	\tilde{n}_{d \uparrow}(\tau)
	\tilde{n}_{d \uparrow}(0)
	\rangle
	\ = \
	\chi_{\uparrow \uparrow}
\end{align}
.

Based on the above, we can rewrite the Wilson ratio as follows:
\begin{align}
	R_{\rm W}
	 & =
	\frac{
		1
		-
		\dfrac{\partial\big\{ \Sigma_{d \uparrow} (i \varepsilon_{n}) - \Sigma_{d \downarrow} (i \varepsilon_{n}) \big\} }{\partial i \varepsilon_{n} }
		\bigg|_{\omega \to 0}
	}{
		1
		-
		\dfrac{\partial \Sigma_{d \sigma} (i \varepsilon_{n}) }{\partial i \varepsilon_{n}}
		\bigg|_{\omega \to 0}
	}
	\ = \
	\frac{\chi_{ds} }{\dfrac{1}{2} ( \chi_{dc} + \chi_{ds} ) }
\end{align}
.

For $U=0$, the self-energy part becomes 0, so $R_{\rm W}=1$.

Also, in the limit of $U \to \infty$, charge fluctuations are completely suppressed, so $\chi_{dc}=0$
\footnote{
	On the other hand, in the thermodynamic limit of $U \to \infty$, it seems that $\chi_{ds} \to \infty$.
}.
Therefore, we can expect that in this limit, $R_{\rm W} \to 2$.
In fact, Wilson calculated $R_{\rm W}$ using a method called Numerical Renormalization Group (NRG) and confirmed that $R_{\rm W} \to 2$ in the limit of $U \to \infty$
\footnote{
	A well-known paper is K. G. Wilson: Rev. Mod. Phys. {\bf 47}, 773 (1975).
	Wilson was awarded the Nobel Prize in Physics in 1982 for this achievement.
}.

The NRG method can cover a wide range from strong coupling to weak coupling, including areas that could not be handled by the $J$ and $U$ expansions we discussed earlier.

Furthermore, the first person to find an exact solution for the Anderson model for all orders in $U$ based on the Bethe ansatz was Wiegmann
\footnote{
	A. M. Tsvelik and O. B. Wiegmann: Adv. Phys. {\bf 32}, 453 (1983).
}.
However, this solution was limited to half-filling, and a more general exact solution was later found by Kawakami and Okiji
\footnote{
	N. Kawakami and A. Okiji: Phys. Lett. {\bf 86A}, 483 (1981) ;
	N. Kawakami and A. Okiji: J. Appl. Phys. {\bf 55}, 1931 (1984).
}.
It is said that this solution suddenly came to Professor Kawakami when he was in his first or second year of his master's program.


\end{document}
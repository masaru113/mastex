\documentclass[uplatex]{jsarticle}
\usepackage[english]{babel}
\usepackage[letterpaper,top=2cm,bottom=2cm,left=3cm,right=3cm,marginparwidth=1.75cm]{geometry}
\usepackage{amsmath, amssymb}
\usepackage[dvipdfmx]{graphicx}

\title{
Chern-Simons形式の導出のメモ
}

\author{
mastex
}

\begin{document}
\maketitle

\begin{abstract}
	忘備録。Chern-Simons 3-形式
	$\omega_{3} = {\rm tr} \left( {AdA + \dfrac{2}{3}A^{3}} \right) $
	を空で導出できるようにメモ。
\end{abstract}

\section{$p$-形式の共変微分}


外微分$d$,接続$A$,
共変微分$D=d+A$,
曲率$F=D^{2} = dA+A^{2}$
のように書く。

$p$-形式$C$、
適当な微分形式$\phi$に対して、

$$
	d(C\phi)
	=(dC)\phi + (-1)^{p}Cd\phi
$$

ここで接続$A$を入れて外微分を共変微分に$d \to D$と置き換えると、

$$
	D(C\phi)
	=(DC)\phi + (-1)^{p}CD\phi
$$

移行して整理すると、

$$
	(DC)\phi
	= D(C\phi) - (-1)^{p}CD\phi
$$

$$
	= (dC) \phi + (AC) \phi - (-1)^{p}CA\phi
$$

$$
	= \big( dC + [A,C] \big) \phi
$$

ここで、
$$
	[A,C]
	=
	AC - (-1)^{p}CA
$$
と置いた。

よって、$p$-形式$C$に共変微分$D$を作用させたものは
$$
	DC = dC + [A,C]
$$
となる。


\section{パラメータ付き接続の導入}

ここで、$s \in \mathbb{R}$ として、 $A_{s} = sA$ とする。

対応して、
$$
	\left\{
	\begin{matrix}
		 & \hspace{-57pt} D_{s} = d + sA           \\
		 & F_{s} = (D_{s})^{2} = sdA + s^{2} A^{2}
	\end{matrix}
	\right.
$$

このとき、
$$
	\left\{
	\begin{matrix}
		 & \dfrac{dF_{s}}{ds} = dA + 2sA^{2} = D_{s} A \\
		 & \hspace{-70pt} D_{s} F_{s} = 0
	\end{matrix}
	\right.
$$
なども得られる。


\section{Chern-Simons $2n-1$ 形式の導出}

Chern-Simons $2n-1$ 形式 $\omega_{2n-1}$ の定義は
$$
	d \omega_{2n-1}
	=
	{\rm tr} F^{n}
$$

ここで
$$
	{\rm tr} F^{n}
	=
	\int^{1}_{0} ds \dfrac{d}{ds} {\rm tr} F^{n}_{s}
$$
と書けることを用いる。

右辺の積分の中身は
$$
	\dfrac{d}{ds} {\rm tr} F^{n}_{s}
		=
		{\rm tr}
	\dfrac{d F^{n}_{s}}{ds} n F^{n-1}_{s}
$$
$$
	=
	n
	\ {\rm tr}
	(D_{s} A) F^{n-1}_{s}
$$
$$
	=
	n
	\ {\rm tr}
	D_{s} (A F^{n-1}_{s})
$$

ここで、
$$
	D_{s} F^{n-1}_{s}
	=
	0
$$
を用いた。

さらに
$$
	D_{s} C = dC + [A_{s},C]
$$
を用いると、
$$
	n
	\ {\rm tr}
	D_{s} (A F^{n-1}_{s})
	=
	n
	\ {\rm tr}
	D_{s} (A F^{n-1}_{s})
	+
	[D_{s}, (A F^{n-1}_{s})]
$$
第2項はゼロなので、結局
$$
	\dfrac{d}{ds} \ {\rm tr} F^{n}_{s}
	=
	d
	\Big( n \ {\rm tr} (A F^{n-1}_{s}) \Big)
$$

Chern-Simons形式の定義に戻ると
$$
	d \omega_{2n-1}
	=
	d
	\Big( \int^{1}_{0} ds \ n \ {\rm tr} (A F^{n-1}_{s}) \Big)
$$


$$
	\omega_{2n-1} = \int_0^1 ds \, n \, \text{tr} (A F_s^{n-1}) \quad \text{+ 完全形式}
$$
と書ける。

$n=2$
のとき、
$$
	\omega_{3}
	=
	\int^{1}_{0} ds \ 2 \ {\rm tr} A (sdA + s^{2} A^{2})
	=
	{\rm tr} \left( AdA + \dfrac{2}{3} A^{3} \right)
$$
\end{document}
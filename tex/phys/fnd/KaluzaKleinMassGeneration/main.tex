\documentclass{article}
\usepackage[english]{babel}
\usepackage[letterpaper,top=2cm,bottom=2cm,left=3cm,right=3cm,marginparwidth=1.75cm]{geometry}
\usepackage{amsmath, amssymb}
\usepackage{graphicx}
\usepackage[colorlinks=true, allcolors=blue]{hyperref}

\title{
Massless Real Scalar Field in 5D Minkowski Space \newline Gaining Mass Upon Compactification of One Spatial Dimension
}

\author{
Masaru Okada
}

\begin{document}
\maketitle

\begin{abstract}
	%UZUZだけに、空間が渦を巻いた理論を考えてみましょう。
	These are notes from a chat I had with a friend.

	I'll make a note of how a massless real scalar field acquires mass when one spatial dimension is compactified in 5D Minkowski spacetime.
\end{abstract}

\section{Set Up}

Let $\mu,\nu = 0,1,2,3$ and $M,N=0,1,2,3,4$.
We define the metric $\eta_{\mu \nu}$ for 4D Minkowski space $M_{4}$ and the metric $\eta_{MN}$ for 5D Minkowski space with the following signatures:

$$
	\eta_{\mu \nu}
	=
	\begin{pmatrix}
		-1 &   &   &   \\
		   & 1 &   &   \\
		   &   & 1 &   \\
		   &   &   & 1 \\
	\end{pmatrix}
	\ , \ \
	\eta_{MN}
	=
	\begin{pmatrix}
		-1 &   &   &   &   \\
		   & 1 &   &   &   \\
		   &   & 1 &   &   \\
		   &   &   & 1 &   \\
		   &   &   &   & 1 \\
	\end{pmatrix}
$$

We compactify the 5th component (the 4th spatial dimension) into a circle with radius $a$.
That is, we parameterize the 5th component, which has become a circle, with an angle $\theta$.
Then the coordinates become:
$$
	(x^0, x^1, x^2, x^3, x^4)
	\to
	(x^0, x^1, x^2, x^3, \theta)
$$

The infinitesimal element of the 5th component is $dx^{4} = a d \theta$, so
the line element $ds$ of this space $M_{4} \times S^{1}$ is
$$
	ds^{2}
	=
	\eta_{\mu \nu}
	dx^{\mu} dx^{\nu}
	+
	a^{2} d \theta^{2}
$$
The metric $g_{MN}$ of this space is
$$
	g_{MN}
	=
	\begin{pmatrix}
		-1 &   &   &   &       \\
		   & 1 &   &   &       \\
		   &   & 1 &   &       \\
		   &   &   & 1 &       \\
		   &   &   &   & a^{2} \\
	\end{pmatrix}
	\ , \ \
	g^{MN}
	=
	\begin{pmatrix}
		-1 &   &   &   &                  \\
		   & 1 &   &   &                  \\
		   &   & 1 &   &                  \\
		   &   &   & 1 &                  \\
		   &   &   &   & \dfrac{1}{a^{2}} \\
	\end{pmatrix}
$$

\section{
  How does a Massless Scalar Field $\phi$ Change When Compactified from $M_{5} \to M_{4} \times S^{1}$?
 }

For simplicity, we'll write the spacetime coordinates as $x$ in 4D and $X$ in 5D.

The action for a massless real scalar field $\phi(x)$ on $M_{4}$ is as follows:

$$
	S
	=
	\int d^{4} x
	\left(
	- \dfrac{1}{2}
	\partial_{\mu} \phi(x)
	\partial^{\nu} \phi(x)
	\right)
$$

For the Lagrangian to be invariant in the general case of $g_{\mu \nu}$, it is necessary to write it as:

$$
	S
	=
	\int d^{4} x
	\sqrt{ - g }
	\left(
	- \dfrac{1}{2}
	\partial_{\mu} \phi(x)
	\partial^{\nu} \phi(x)
	\right)
	\ \ \ \
	{ \rm where }
	\ \ \ \
	g = \det(g_{\mu \nu})
$$
We will now consider the action by extending this to the 5D case:
$$
	S
	=
	\int d^{5} X
	\sqrt{ - g }
	\left(
	- \dfrac{1}{2}
	g^{MN}
	\partial_{M} \phi(X)
	\partial_{N} \phi(X)
	\right)
	\ \ \ \
	{ \rm where }
	\ \ \ \
	g = \det(g_{MN})
$$

The 5th component of the scalar field is compactified, and since its period is $2 \pi$, we can perform a Fourier series expansion.

$$
	\phi(X)
	=
	\phi(x,\theta)
$$
$$
	=
	\dfrac{1}{
		\sqrt{ 2 \pi}
	}
	\sum_{n}
	\phi_{n}(x)
	e^{i n \theta}
$$

Substituting this into the action:

$$
	S
	=
	- \dfrac{1}{2}
	\int d^{5} X
	\sqrt{ - \det(g_{MN}) }
	g^{MN}
	\partial_{M}
	\left(
	\dfrac{1}{
		\sqrt{ 2 \pi}
	}
	\sum_{m}
	\phi_{m}(x)
	e^{i m \theta}
	\right)
	\partial_{N}
	\left(
	\dfrac{1}{
		\sqrt{ 2 \pi}
	}
	\sum_{n}
	\phi_{n}(x)
	e^{i n \theta}
	\right)
$$

Here,
$$
	\det(g_{MN})
	=
	\begin{vmatrix}
		-1 &   &   &   &       \\
		   & 1 &   &   &       \\
		   &   & 1 &   &       \\
		   &   &   & 1 &       \\
		   &   &   &   & a^{2} \\
	\end{vmatrix}
	=
	- a^{2}
$$
Furthermore, when we separate the calculation into 4D Minkowski space and $S^{1}$ components, we get:

$$
	=
	- \dfrac{1}{2}
	a
	\int d^{4} x d \theta
	\left\{
	g_{\mu \nu}
	\partial_{\mu}
	\left(
	\dfrac{1}{
		\sqrt{ 2 \pi}
	}
	\sum_{m}
	\phi_{m}(x)
	e^{i m \theta}
	\right)
	\partial_{\nu}
	\phi(x)
	\left(
	\dfrac{1}{
		\sqrt{ 2 \pi}
	}
	\sum_{n}
	\phi_{n}(x)
	e^{i n \theta}
	\right)
	\right.
$$
$$
	\left.
	+
	g^{\theta \theta}
	\partial_{\theta}
	\left(
	\dfrac{1}{
		\sqrt{ 2 \pi}
	}
	\sum_{m}
	\phi_{m}(x)
	e^{i m \theta}
	\right)
	\partial_{\theta}
	\left(
	\dfrac{1}{
		\sqrt{ 2 \pi}
	}
	\sum_{n}
	\phi_{n}(x)
	e^{i n \theta}
	\right)
	\right\}
$$

\subsection{Calculation of the First Term of the Action}

The first term is:
$$
	- \dfrac{1}{2}
	a
	\int d^{4} x d \theta
	\sum_{n,m}
	\dfrac{1}{2 \pi}
	e^{i (n+m) \theta}
	\eta^{\mu \nu}
	\partial_{\mu}
	\phi_{m}(x)
	\partial_{\nu}
	\phi_{n}(x)
$$

Now, let's evaluate the $\theta$ integral.
$$
	\int d \theta
	e^{i (n+m) \theta}
	=
	2 \pi
	\delta_{n,-m}
$$
So we can also perform the summation over $m$ to get:
$$
	- \dfrac{1}{2}
	a
	\sum_{n}
	\int d^{4} x
	\eta^{\mu \nu}
	\partial_{\mu}
	\phi_{m}(x)
	\partial_{\nu}
	\phi_{-n}(x)
$$

Since $\phi$ is a real scalar field,
$$
	\phi(x) = \phi^{*}(x)
$$
Therefore,
$$
	\sum_{n}
	\dfrac{1}{\sqrt{ 2 \pi}}
	\phi_{n}(x)
	e^{i n \theta}
	=
	\sum_{n}
	\dfrac{1}{\sqrt{ 2 \pi}}
	\phi_{n}^{*}(x)
	e^{- i n \theta}
$$
$$
	=
	\sum_{n}
	\dfrac{1}{\sqrt{ 2 \pi}}
	\phi_{-n}^{*}(x)
	e^{i n \theta}
$$
In the last equality, we substituted $n \to -n$.

The set $\left\{ e^{i n \theta} \right\}_{n=0,\pm 1, \pm 2 ,\cdots}$ is an orthonormal basis, so:

$$
	\phi_{n}(x) = \phi^{*}_{-n}(x)
$$
In particular, when $n=0$:
$$
	\phi_{0}(x) = \phi^{*}_{0}(x)
$$
so
$
	\phi_{0}(x)
	\in \mathbb{R}
$

Thus, the first term of the action is:
$$
	- \dfrac{1}{2}
	a
	\sum_{n \in \mathbb{Z}}
	\int d^{4} x
	\partial_{\mu}
	\phi_{m}(x)
	\partial^{\nu}
	\phi_{n}^{*}(x)
$$

$$
	=
	-a
	\sum_{n=1}^{\infty}
	\int d^{4} x
	\partial_{\mu}
	\phi_{m}(x)
	\partial^{\nu}
	\phi_{n}^{*}(x)
	- \dfrac{1}{2}
	a
	\int d^{4} x
	\partial_{\mu}
	\phi_{0}(x)
	\partial^{\nu}
	\phi_{0}^{*}(x)
$$

\subsection{Calculation of the Second Term of the Action}

Let's proceed with the calculation of the second term of the action.

Here, $ g^{\theta \theta} = \dfrac{1}{a^{2}}$, and the $\theta$ derivative is:
$$
	\partial_{\theta}
	\left(
	\dfrac{1}{\sqrt{ 2 \pi}}
	\sum_{m}
	\phi_{m}(x)
	e^{i m \theta}
	\right)
	=
	\dfrac{1}{\sqrt{ 2 \pi}}
	\sum_{m}
	\phi_{m}(x)
	(im)
	e^{i m \theta}
$$
So,
$$
	- \dfrac{1}{2}
	a
	\int d^{4} x d \theta
	g^{\theta \theta}
	\partial_{\theta}
	\left(
	\dfrac{1}{\sqrt{ 2 \pi}}
	\sum_{m}
	\phi_{m}(x)
	e^{i m \theta}
	\right)
	\partial_{\theta}
	\left(
	\dfrac{1}{\sqrt{ 2 \pi}}
	\sum_{n}
	\phi_{n}(x)
	e^{i n \theta}
	\right)
$$

$$
	=
	- \dfrac{1}{2}
	a
	\int d^{4} x
	\dfrac{1}{a^{2}}
	\dfrac{1}{2 \pi}
	\sum_{m,n}
	\left\{
	\int d \theta
	e^{i (n+m) \theta}
	\right\}
	( - n m )
	\phi_{n}(x)
	\phi_{m}(x)
$$

$$
	=
	- \dfrac{1}{2}
	a
	\int d^{4} x
	\dfrac{1}{a^{2}}
	\dfrac{1}{2 \pi}
	\sum_{m,n}
	\left\{
	2 \pi
	\delta_{n,-m}
	\right\}
	( - n m )
	\phi_{n}(x)
	\phi_{m}(x)
$$

Performing the summation over $m$, we get:
$$
	=
	- \dfrac{1}{2}
	a
	\int d^{4} x
	\dfrac{1}{a^{2}}
	\sum_{n}
	n^{2}
	\phi_{n}(x)
	\phi_{n}^{*}(x)
$$

$$
	=
	- \dfrac{1}{2}
	a
	\sum_{n=1}^{\infty}
	\int d^{4} x
	\dfrac{ n^{2} }{ a^{2} }
	\phi_{n}(x)
	\phi_{n}^{*}(x)
$$

From the above, the action for a massless real scalar field on $M_{4} \times S^{1}$ is:
$$
	S
	=
	-a
	\sum_{n=1}^{\infty}
	\int d^{4} x
	\partial_{\mu}
	\phi_{m}(x)
	\partial^{\nu}
	\phi_{n}^{*}(x)
	-
	\dfrac{1}{2}
	a
	\int d^{4} x
	\partial_{\mu}
	\phi_{0}(x)
	\partial^{\nu}
	\phi_{0}^{*}(x)
	- \dfrac{1}{2}
	a
	\sum_{n=1}^{\infty}
	\int d^{4} x
	\dfrac{ n^{2} }{ a^{2} }
	\phi_{n}(x)
	\phi_{n}^{*}(x)
$$

$$
	=
	a
	\int d^{4} x
	\left\{
	-
	\dfrac{1}{2}
	\partial_{\mu}
	\phi_{0}(x)
	\partial^{\nu}
	\phi_{0}^{*}(x)
	-
	\sum_{n=1}^{\infty}
	\left[
		\partial_{\mu}
		\phi_{m}(x)
		\partial^{\nu}
		\phi_{n}^{*}(x)
		+
		\dfrac{ n^{2} }{ a^{2} }
		\phi_{n}(x)
		\phi_{n}^{*}(x)
		\right]
	\right\}
$$

\section{Interpretation}

Let's write the integrand of the 4D coordinate $x$ in the action as $L_{4}$:

$$
	L_{4}
	=
	-
	\dfrac{1}{2}
	\partial_{\mu}
	\phi_{0}(x)
	\partial^{\nu}
	\phi_{0}^{*}(x)
	-
	\sum_{n=1}^{\infty}
	\left[
		\partial_{\mu}
		\phi_{m}(x)
		\partial^{\nu}
		\phi_{n}^{*}(x)
		+
		\dfrac{ n^{2} }{ a^{2} }
		\phi_{n}(x)
		\phi_{n}^{*}(x)
		\right]
$$

$$
	=
	L_{\rm free}
	+
	L_{\rm mass}
$$

The first term,
$$
	L_{\rm free}
	=
	-
	\dfrac{1}{2}
	\partial_{\mu}
	\phi_{0}(x)
	\partial^{\nu}
	\phi_{0}^{*}(x)
$$
is a massless real scalar field on 4D Minkowski space, and the second term,
$$
	L_{\rm mass}
	=
	-
	\sum_{n=1}^{\infty}
	\left[
		\partial_{\mu}
		\phi_{m}(x)
		\partial^{\nu}
		\phi_{n}^{*}(x)
		+
		\dfrac{ n^{2} }{ a^{2} }
		\phi_{n}(x)
		\phi_{n}^{*}(x)
		\right]
$$
is the sum of complex scalar fields $\phi_{n}(x) \ ( n=1,2,3, \cdots )$ with an effective mass of $M_{n} = \left| \dfrac{n}{a} \right|$.

In the end, a massless real scalar field on the compactified space $M_{4} \times S^{1}$ can be seen, from a perspective that ignores the $S^{1}$, as a superposition on $M_{4}$ of:
\begin{equation*}
	\begin{cases}
		\phi_{0}(x) & \ \ \ \cdots \ \ \text{a massless real scalar field}                                  \\
		\phi_{n}(x) & \ \ \ \cdots \ \ \text{a complex scalar field with mass } \left| \dfrac{n}{a} \right|
	\end{cases}
\end{equation*}

In conclusion, it has been shown that when one spatial dimension of a 5D spacetime is compactified, a massless real scalar field from the original theory gives rise to complex scalar fields that have mass.

If the radius $a$ of the compactified dimension is very small,
the mass $M_{n} = \left| \dfrac{n}{a} \right|$ becomes very large,
so the complex scalar fields $\phi_{n}(x)$ do not affect physical observations,
and only the massless real scalar field $\phi_{0}(x)$ is observed.

It is known that 5D general relativity, when viewed as a theory on 4D spacetime with a compactified $S^{1}$ like this, gives rise to 4D general relativity and 4D electromagnetism.

It was thought that by extending this to higher dimensions, it might be possible to unify
\begin{equation*}
	\begin{cases}
		\cdot \ \ \text{gravity}          \\
		\cdot \ \ \text{electromagnetism} \\
		\cdot \ \ \text{the weak force}   \\
		\cdot \ \ \text{the strong force}
	\end{cases}
\end{equation*}
This idea led to attempts at supergravity theory, but the theory was found to contradict observations.
Currently, string theory is considered the most promising candidate for a unified theory of the four forces.

It is known that supergravity theory is the low-energy limit of superstring theory.

\end{document}
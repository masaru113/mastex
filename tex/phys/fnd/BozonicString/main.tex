\documentclass[uplatex]{jsarticle}
\usepackage[english]{babel}
\usepackage[letterpaper,top=2cm,bottom=2cm,left=3cm,right=3cm,marginparwidth=1.75cm]{geometry}
\usepackage{amsmath, amssymb}
\usepackage[dvipdfmx]{graphicx}
\usepackage{here}

\title{
Bosonic Strings Short Introduction
}

\author{
Masaru Okada
}

\begin{document}
\maketitle

\begin{abstract}
	A summary note for myself.

	We consider the theory of bosonic strings (a theory without supersymmetry).

	We will confirm that mass energy arises from the vibrational energy of the string, and that the spacetime dimension becomes 26.

	We will confirm that gauge theory can be derived from Open Strings and general relativity can be derived from Closed Strings.

	We will also confirm that spin arises from the orbital angular momentum originating from the string's vibrations.

\end{abstract}

\section{D-Dimensional Minkowski Spacetime}

Let's consider a $D$-dimensional Minkowski spacetime as
$\mathbb{R}^{1,D-1}$,
and its coordinates as
$\{ X^{\mu} \}_{\mu=0,1,\cdots,D-1}$.

The Minkowski metric is defined as follows:

$$
	ds^{2} = - \eta_{\mu \nu} dX^{\mu} dX^{\nu}
$$


where
$\eta_{\mu \nu} = {\rm diag} (-1, 1,1,\cdots,1)$.

\section{The Motion of a String}

A string is a 1-dimensional object that moves in
$\mathbb{R}^{1,D-1}$.

The trajectory of the string is called the worldsheet, and
$\mathbb{R}^{1,D-1}$
is called the target spacetime.

If the worldsheet parameters are
$(\tau,\sigma)$,
the position of the string in the target space can be written as
$X^{\mu}=X^{\mu}(\tau, \sigma)$.

The canonical conjugate momentum
$P^{\mu}=P^{\mu}(\tau, \sigma)$
is
$$
	P^{\mu}
	=
	\frac{1}{2 \pi \alpha'}
	\partial^{\tau} X^{\mu}
$$
and the momentum of the string's center of mass,
$p^{\mu}$,
is
$$
	p^{\mu}(\tau)
	=
	\int^{l}_{0}
	d \sigma
	P^{\mu}(\tau, \sigma)
$$
where
$l$
is the length of the string, and
$\alpha'=l^{2}$
is the tension, and these are parameters.

We set up a coordinate system where the momentum of the string's center of mass is
$p^{\mu} = (p^{0} , p^{1}, 0 , 0 \cdots, 0)$.

\section{Light-Cone Coordinates}

Since the string doesn't vibrate in the
$X^{0}, X^{1}$
directions, we use light-cone coordinates in the target space.

$$
	X^{\pm} = \frac{1}{\sqrt{2}} (X^{0} \pm X^{1})
$$

In light-cone coordinates, the inner product becomes:

$$
	A_{\mu} B^{\mu}
	= - A^{+} B^{-} - A^{-} B^{+} + \sum_{i=2}^{D-1} A^{i} B^{i}
$$

Furthermore, we set the worldsheet time coordinate as
$\tau=X^{+}$.

\section{String Equation of Motion}

The string satisfies the wave equation.

$$
	\left(
	\frac{\partial^{2}}{\partial \tau^{2}}
	-
	\frac{\partial^{2}}{\partial \sigma^{2}}
	\right)
	X^{\mu}
	=0
$$


Note that the string is not yet quantized at this point.

Also, while the string wave equation formally resembles the massless Klein-Gordon equation,
as we will see later, the string is not necessarily massless.
Rather, the mass of the string arises from its vibrations.

\section{Solution for the Open String}

By imposing the free endpoint condition
$$
	\partial_{\sigma} X^{\mu} |_{\sigma=0,\pi} = 0
$$
the solution becomes as follows:

$$
	X^{i}(\tau, \sigma)
	=
	x^{i} + \frac{p^{i}}{\pi p^{+}} \tau
	+
	i (2 \alpha')^{1/2} \sum_{n \in \mathbb{Z} \backslash \{ 0 \} }
	\frac{1}{n}
	\alpha^{i}_{n} e^{- i n \tau}
	\cos n \sigma
$$

Here, we've set $x^{i}$ and $p^{i}$ to be the position and momentum of the string's center of mass.

Recalling high school physics, the motion of a particle with velocity
$v$
is
$x = x_{0} + vt$,
which corresponds to the first and second terms of the Open String solution.
The third term, the oscillatory part, is the characteristic feature originating from the string.

\section{Solution for the Closed String}

By imposing the smooth connection conditions at the ends
$$
	X^{\mu}(\tau, 0)
	=
	X^{\mu}(\tau, 2 \pi)
$$
$$
	\partial_{\sigma} X^{\mu}(\tau, 0)
	=
	\partial_{\sigma} X^{\mu}(\tau, 2 \pi)
$$

the solution becomes as follows:

$$
	X^{i}(\tau, \sigma)
	=
	x^{i} + \frac{p^{i}}{\pi p^{+}} \tau
	+
	i \left( \frac{\alpha'}{2} \right)^{1/2} \sum_{n \in \mathbb{Z} \backslash \{ 0 \} }
	\frac{1}{n}
	\left(
	\alpha^{i}_{n} e^{- 2 i n (\tau + \sigma) }
	+
	\tilde{\alpha}^{i}_{n} e^{- 2 i n (\tau - \sigma) }
	\right)
$$

Unlike the Open String case, the third term, which originates from the string, is separated into clockwise and counter-clockwise parts.
The Fourier coefficients use different variables for clockwise and counter-clockwise motion.

\section{Canonical Quantization of the String}

Up to this point, we've been considering the classical string.
From here on, we will consider the quantum string.

Recalling the canonical commutation relation in quantum mechanics,

$$
	[x^{i} , p^{j}] = \sqrt{-1} \delta^{ij}
$$

In a similar way, we have
$$
	[X^{i}(\tau, \sigma) , P^{j}(\tau, \sigma')] = \sqrt{-1} \delta^{ij} \delta(\sigma - \sigma')
$$

Here we set
$\hbar=1$.
We perform quantization by imposing these canonical commutation relations.

Upon Fourier transforming, the following relations are derived among the coefficients:
$$
	[\alpha^{i}_{n} , \alpha^{j}_{m}] = n \delta^{ij} \delta_{n+m, 0}
$$
$$
	[ \tilde{\alpha}^{i}_{n} , \tilde{\alpha}^{j}_{m}] = n \delta^{ij} \delta_{n+m, 0}
$$

Since these form a Heisenberg algebra, they appear to be creation and annihilation operators for string vibrations in the physical picture.

\begin{table}[H]
	\centering
	\begin{tabular}{|c|c|c|}
		\hline
		Open String   & $\alpha^{i}_{n(<0)}$         & Creation operator for $|n|$-th vibration in the $X^{i}$ direction     \\ \hline
		Open String   & $\alpha^{i}_{n(>0)}$         & Annihilation operator for $n$-th vibration in the $X^{i}$ direction       \\ \hline
		Closed String & $\alpha^{i}_{n(<0)}$         & Creation operator for clockwise $|n|$-th vibration in the $X^{i}$ direction \\ \hline
		Closed String & $\alpha^{i}_{n(>0)}$         & Annihilation operator for clockwise $n$-th vibration in the $X^{i}$ direction   \\ \hline
		Closed String & $\tilde{\alpha}^{i}_{n(<0)}$ & Creation operator for counter-clockwise $|n|$-th vibration in the $X^{i}$ direction \\ \hline
		Closed String & $\tilde{\alpha}^{i}_{n(>0)}$ & Annihilation operator for counter-clockwise $n$-th vibration in the $X^{i}$ direction   \\ \hline
	\end{tabular}
\end{table}

\section{On-Shell Condition}

Now, we have set up a coordinate system where the string's center of mass momentum is
$p^{\mu} = (p^{0} , p^{1}, 0 , 0 \cdots, 0)$.
Therefore, if the mass of the string is
$m$,
then
$$
	m^{2}
	=
	- p^{\mu} p_{\mu}
	=
	2 p^{+} p^{-}
$$

Recalling that
$$
	p^{\mu}
	=
	\int^{l}_{0} d \sigma
	P^{\mu}(\tau, \sigma)
$$
it is clear that
$m^{2}$
is determined by the string's vibrations.
The more vigorously the string vibrates, the heavier it becomes.

Expanding the canonical momentum density,
$$
	m^{2}
	=
	\frac{1}{2}
	\sum^{D-1}_{i=2}
	\int^{l}_{0} d \sigma
	\left(
	\partial_{\tau} X^{i} \partial_{\tau} X^{i}
	+
	\partial_{\sigma} X^{i} \partial_{\sigma} X^{i}
	\right)
$$

Recalling that the Hamiltonian of a harmonic oscillator can be written as
$
	\frac{p^{2}}{2m}
	+
	\frac{1}{2} m \omega^{2} x^{2}
$,
the first and second terms correspond to the kinetic and potential terms, respectively.

\section{Squared Mass of the Open String}

Continuing the calculation for the Open String, we get
$$
	m^{2}
	=
	\frac{1}{\alpha'}
	\sum^{\infty}_{n=1}
	\sum^{D-1}_{i=2}
	n
	\left(
	N_{i,n} + \frac{1}{2}
	\right)
$$

$N_{i,n}=\frac{1}{n} \alpha^{i}_{-n} \alpha^{i}_{n}$
is the number operator.
The second term in the parenthesis,
$\frac{1}{2}$,
represents the zero-point vibration.

An apparent divergence appears here, but we
\footnote{The discussion to converge this divergence seems very complicated.}
substitute
$$
	\sum^{\infty}_{n=1} n
	=
	\sum^{\infty}_{n=1}
	\frac{1}{n^{-1}}
	=
	\zeta(-1)
$$
and replace it with the special value of the zeta function:
$$
	\zeta(-1)
	=
	- \frac{1}{12}
$$
From the above procedure, the squared mass of the Open string is
$$
	m^{2}
	=
	\frac{1}{\alpha'}
	\left(
	\sum^{\infty}_{n=1}
	\sum^{D-1}_{i=2}
	n N_{i,n}
	\right)
	- \frac{D-2}{24 \alpha'}
$$

\section{Ground State of the Open String}

The ground state of the Open String,
$| 0 \rangle$,
is defined for all
$n>0$
as follows:
$$
	\alpha^{i}_{n}
	| 0 \rangle
	= 0
$$

In this case, the squared mass is determined only by the zero-point vibration.

$$
	m^{2}
	= - \frac{D-2}{24 \alpha'}
$$

If the dimension is
$D>2$,
the squared mass becomes negative, and a tachyon with an imaginary mass appears.
This is a flaw in the theory and one of the reasons why the theory of bosonic strings is considered a toy model that does not describe reality.

\section{First Excited State of the Open String and Dimensional Constraint}

The first excited state of the Open String is the state where only a single vibration is active:
$$
	\alpha^{i}_{-1}
	| 0 \rangle
	\ , \ \ i = 2,3,\cdots, D-1
$$
and there are
$D-2$
such states (there is
$SO(D-2)$
symmetry).

In this case, the squared mass is
$$
	m^{2}
	=
	\frac{1}{\alpha'}
	\left(
	1 - \frac{D-2}{24}
	\right)
	=
	\frac{26 - D}{24 \alpha'}
$$

In the discussion so far, Lorentz symmetry appears to be broken during quantization, but for this symmetry to hold, this mass must be zero (Wigner's classification).

In this case, the allowed dimension of the target space is
$$
	D=26
$$
.

(Although not written in this note, in superstring theory with supersymmetry, only
$D=10$
is allowed.)

\section{Squared Mass of the Closed String}

The squared mass of the Closed string is
$$
	m^{2}
	=
	\frac{2}{\alpha'}
	\left(
	\sum^{\infty}_{n=1}
		(
		\alpha^{i}_{-n} \alpha^{i}_{n}
		+
		\tilde{\alpha}^{i}_{n} \tilde{\alpha}^{i}_{n}
		)
	-
	\frac{D-2}{24}
	-
	\frac{D-2}{24}
	\right)
$$

Due to translational symmetry, the constraint condition
$$
	\sum^{\infty}_{n=1}
	\alpha^{i}_{-n} \alpha^{i}_{n}
	=
	\sum^{\infty}_{n=1}
	\tilde{\alpha}^{i}_{-n} \tilde{\alpha}^{i}_{n}
$$
is imposed.

\section{Ground State of the Closed String}

The ground state of the Closed String,
$| 0 \rangle$,
is defined for all
$n>0$
by the following equation:
$$
	\alpha^{i}_{n}
	| 0 \rangle
	=
	\tilde{\alpha}^{i}_{n}
	| 0 \rangle
	=
	0
$$

In this case, the mass is
$$
	m^{2}
	= - \frac{D-2}{12 \alpha'}
$$
and as with the Open String, a tachyon appears again.

\section{First Excited State of the Closed String and Dimensional Constraint}

The first excited state for the Closed case uses both the clockwise and counter-clockwise creation operators:
$$
	\alpha^{i}_{-1}
	\tilde{\alpha}^{j}_{-1}
	| 0 \rangle
	\ , \ \ i, j = 2,3,\cdots, D-1
$$

In this case, the squared mass is
$$
	m^{2}
	=
	\frac{2}{\alpha'}
	\left(
	1 - \frac{D-2}{24}
	\right)
	=
	\frac{26 - D}{12 \alpha'}
$$

Following a similar discussion to the Open String, a dimensional constraint on the target spacetime of
$$
	D=26
$$
is also obtained for the Closed case.

\section{Correspondence to the Standard Model and General Relativity}

The first excited state of the Open String,
$
	\alpha^{i}_{-1}
	| 0 \rangle
$,
has a single dimensional index and is a vector. This corresponds to the
gauge field
$A_{\mu}$.


On the other hand, the first excited state of the Closed String,
$
	\alpha^{i}_{-1}
	\tilde{\alpha}^{j}_{-1}
	| 0 \rangle
$,
has two dimensional indices and is a rank-2 tensor.
A rank-2 tensor can be irreducibly decomposed into a scalar, a rank-2 antisymmetric tensor, and a rank-2 symmetric tensor.
These correspond to the scalar field $\phi$, the field curvature (antisymmetric tensor) $F_{\mu \nu}$, and the gravitational field (symmetric tensor) $G_{\mu \nu}$ respectively.

This is the reason why it is said that gauge theory and gravity theory emerge from string theory.

\section{Spin}

In string theory, spin is naturally derived by evaluating the orbital angular momentum of the string.

Let's consider the Open String.
The orbital angular momentum density of the string is
$$
	L^{ij}(\tau, \sigma)
	=
	X^{i} P^{j}
	-
	P^{i} X^{j}
$$
and the total angular momentum of the string is,
$$
	l^{ij}
	=
	\int^{\pi}_{0} d \sigma
	L^{ij}
$$
Calculating this, we get,
$$
	l^{ij}
	=
	x^{i} p^{j}
	-
	p^{i} x^{j}
	+
	S^{ij}
$$
Here we've set
$$
	S^{ij}
	=
	-i \sum_{n=1}^{\infty} \frac{1}{n}
	(
	\alpha^{i}_{-n} \alpha^{j}_{n}
	-
	\alpha^{j}_{-n} \alpha^{i}_{n}
	)
$$
.

The first and second terms of
$l^{ij}$
are quantities originating from the center of mass position and momentum, while the third term
$S^{ij}$
is the orbital angular momentum originating from the string's vibrations, which does not depend on the center of mass position and momentum.
$S^{ij}$
satisfies the definition of the spin operator.

\section{Particularly Important Omissions in This Note}

Discussion of the Nambu-Goto action and the Polyakov action.

Discussion that in addition to light-cone quantization, there are also canonical quantization and BRST quantization which preserve covariance.

Discussion of zeta function regularization.


\begin{thebibliography}{9}

	\bibitem{Rudin:Zwiebach}
	Barton Zwiebach.
	\newblock A First Course in String Theory 2nd Edition.

	\bibitem{Rudin:Ilinski}
	David McMahon.
	\newblock String Theory Demystified.

\end{thebibliography}

\end{document}
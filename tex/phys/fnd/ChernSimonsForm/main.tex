\documentclass[uplatex]{jsarticle}
\usepackage[english]{babel}
\usepackage[letterpaper,top=2cm,bottom=2cm,left=3cm,right=3cm,marginparwidth=1.75cm]{geometry}
\usepackage{amsmath, amssymb}
\usepackage[dvipdfmx]{graphicx}

\title{
\textbf{Notes on the Derivation of the Chern-Simons Form}
}

\author{
Masaru Okada
}

\begin{document}
\maketitle

\begin{abstract}
	A memorandum. My goal is to be able to derive the Chern-Simons 3-form
	$\omega_{3} = {\rm tr} \left( {AdA + \dfrac{2}{3}A^{3}} \right) $
	from scratch.
\end{abstract}

\section{\textbf{Covariant Derivative of a $p$-form}}


We write the exterior derivative as $d$, the connection as $A$,
the covariant derivative as $D=d+A$,
and the curvature as $F=D^{2} = dA+A^{2}$.

For a $p$-form $C$ and an appropriate differential form $\phi$, we have:

$$
	d(C\phi)
	=(dC)\phi + (-1)^{p}Cd\phi
$$

Now, if we introduce the connection $A$ and replace the exterior derivative with the covariant derivative, $d \to D$, we get:

$$
	D(C\phi)
	=(DC)\phi + (-1)^{p}CD\phi
$$

Rearranging the terms, we get:

$$
	(DC)\phi
	= D(C\phi) - (-1)^{p}CD\phi
$$

$$
	= (dC) \phi + (AC) \phi - (-1)^{p}CA\phi
$$

$$
	= \big( dC + [A,C] \big) \phi
$$

where we have defined
$$
	[A,C]
	=
	AC - (-1)^{p}CA
$$
Therefore, the covariant derivative $D$ acting on a $p$-form $C$ is
$$
	DC = dC + [A,C]
$$
This is the result.


\section{\textbf{Introduction of a Parameterized Connection}}

Here, we introduce a parameter $s \in \mathbb{R}$ and set $A_{s} = sA$.

Correspondingly, we have:
$$
	\left\{
	\begin{array}{rcl}
		 D_{s} &=& d + sA\\
		 F_{s} = (D_{s})^{2} &=& sdA + s^{2} A^{2}
	\end{array}
	\right.
$$

In this case, we also get
$$
	\left\{
	\begin{matrix}
		 & \dfrac{dF_{s}}{ds} = dA + 2sA^{2} = D_{s} A \\
		 & \hspace{-70pt} D_{s} F_{s} = 0
	\end{matrix}
	\right.
$$
and so on.


\section{\textbf{Derivation of the Chern-Simons $2n-1$ Form}}

The Chern-Simons $2n-1$ form $\omega_{2n-1}$ is defined by
$$
	d \omega_{2n-1}
	=
	{\rm tr} F^{n}
$$

Here we use the fact that
$$
	{\rm tr} F^{n}
	=
	\int^{1}_{0} ds \dfrac{d}{ds} {\rm tr} F^{n}_{s}
$$
The integrand of the right-hand side is
$$
	\dfrac{d}{ds} {\rm tr} F^{n}_{s}
		=
		{\rm tr}
	\dfrac{d F^{n}_{s}}{ds} n F^{n-1}_{s}
$$
$$
	=
	n
	\ {\rm tr}
	(D_{s} A) F^{n-1}_{s}
$$
$$
	=
	n
	\ {\rm tr}
	D_{s} (A F^{n-1}_{s})
$$

Here we used
$$
	D_{s} F^{n-1}_{s}
	=
	0
$$
Furthermore, using
$$
	D_{s} C = dC + [A_{s},C]
$$
we get
$$
	n
	\ {\rm tr}
	D_{s} (A F^{n-1}_{s})
	=
	n
	\ {\rm tr}
	D_{s} (A F^{n-1}_{s})
	+
	[D_{s}, (A F^{n-1}_{s})]
$$
The second term is zero, so we finally get
$$
	\dfrac{d}{ds} \ {\rm tr} F^{n}_{s}
	=
	d
	\Big( n \ {\rm tr} (A F^{n-1}_{s}) \Big)
$$

Returning to the definition of the Chern-Simons form,
$$
	d \omega_{2n-1}
	=
	d
	\Big( \int^{1}_{0} ds \ n \ {\rm tr} (A F^{n-1}_{s}) \Big)
$$


$$
	\omega_{2n-1} = \int_0^1 ds \, n \, \text{tr} (A F_s^{n-1}) \quad \text{+ exact form}
$$
This can be written.

When $n=2$,
$$
	\omega_{3}
	=
	\int^{1}_{0} ds \ 2 \ {\rm tr} A (sdA + s^{2} A^{2})
	=
	{\rm tr} \left( AdA + \dfrac{2}{3} A^{3} \right)
$$
\end{document}
\documentclass[a4j]{jsarticle}
\usepackage{amsmath,amsthm,amssymb,bm,color,mathrsfs,url}
\usepackage{epic,eepic,here}
\usepackage[dvipdfm]{graphicx}
\usepackage[hypertex]{hyperref}
\title{
\hspace{3em} 異方的超伝導に拡張されたGor'kov方程式の解と \newline 電子Raman応答関数
}
\author{岡田 大 (Okada Masaru)}
\date{\today}
\begin{document}
\maketitle

\ \\[-15mm]

BCS平均場ハミルトニアン$\mathcal{H}$から出発する。
\begin{align}
	\mathcal{H}
	 & =
	\mathcal{H}_{0}
	\ + \
	\mathcal{H}_{\rm BCS}
	\ \ ,
	\nonumber \\[2mm]
	\mathcal{H}_{0}
	 & =
	\sum_{\bm{k},s}
	\varepsilon(\bm{k})
	c_{\bm{k} s}^{\dagger}
	c_{\bm{k} s}
	\ \ ,
	\nonumber \\[2mm]
	\mathcal{H}_{\rm BCS}
	 & =
	\dfrac{1}{2}
	\sum_{\bm{k},s_{1},s_{2}}
	\Big[
		\Delta_{ s_{1} s_{2} }( \bm{k} )
		c_{\bm{k} s_{1}}^{\dagger}
		c_{-\bm{k} s_{2}}^{\dagger}
		\ - \
		\Delta_{ s_{1} s_{2} }^{*}( - \bm{k} )
		c_{-\bm{k} s_{1}}
		c_{\bm{k} s_{2}}
		\Big]
	\ \ ,
\end{align}

有限温度のGreen関数を松原形式で次のように導入する。
\begin{align}
	G_{ss'}(\bm{k} , \bm{k}' ; \tau)
	 & =
	-
	\langle T_{\tau} \{ c_{\bm{k}s}(\tau) c_{\bm{k}'s'}^{\dagger}(0) \} \rangle
	\ \ ,
	\\[3mm]
	F_{ss'}(\bm{k} , \bm{k}' ; \tau)
	 & =
	\langle T_{\tau} \{ c_{\bm{k}s}(\tau) c_{\bm{k}'s'}(0) \} \rangle
	\ \ , \ \
	F_{ss'}^{\dagger}(\bm{k} , \bm{k}' ; \tau)
	 & =
	\langle T_{\tau} \{ c_{\bm{k}' s' }^{\dagger}(\tau) c_{\bm{k} s }^{\dagger}(0) \} \rangle
	\ \ .
\end{align}
ここで慣例に従って$F_{ss'}^{\dagger}$と書いたが、
$F_{ss'}$のエルミート共役を取っているわけではない。
これらのGreen関数は$c-$数である。

座標変数$\tau \to i \omega_{n}$へのFourier変換は次で定義されている。
\begin{align}
	G_{ss'}(\bm{k} , \bm{k}' ; \tau)
	 & =
	\dfrac{1}{\beta} \sum_{n}
	G_{ss'}(\bm{k} , \bm{k}' ; i \omega_{n} )
	e^{- i \omega_{n} \tau }
	\ \ , \nonumber \\[2mm]
	F_{ss'}^{(\dagger)}(\bm{k} , \bm{k}' ; \tau)
	 & =
	\dfrac{1}{\beta} \sum_{n}
	F_{ss'}^{(\dagger)}(\bm{k} , \bm{k}' ; i \omega_{n} )
	e^{- i \omega_{n} \tau }
	\ \ .
\end{align}
ここで、$\omega_{n}=(2n+1)\pi k_{B} T$、($n \in \mathbb{Z}$)はfermionicな松原周波数である。
系が均一な場合、これらのGreen関数の運動量変数は$G$関数の場合$\bm{k}=\bm{k}'$、$F^{(\dagger)}$関数の場合$\bm{k}=-\bm{k}'$となり、1つの運動量の値で指定できる。
\begin{align}
	G_{ss'}(\bm{k} , i \omega_{n} )
	 & =
	-
	\int_{0}^{\beta}
	d \tau
	\langle T_{\tau} \{ c_{\bm{k}s}(\tau) c_{\bm{k} s'}^{\dagger}(0) \} \rangle
	e^{i \omega_{n} \tau}
	\ \ ,
	\\[3mm]
	F_{ss'}(\bm{k} , i \omega_{n} )
	 & =
	\int_{0}^{\beta}
	d \tau
	\langle T_{\tau} \{ c_{ \bm{k} s }(\tau) c_{-\bm{k} s' }(0) \} \rangle
	e^{i \omega_{n} \tau}
	\ \ , \ \
	F_{ss'}^{\dagger}(\bm{k} , i \omega_{n} )
	 & =
	\int_{0}^{\beta}
	d \tau
	\langle T_{\tau} \{ c_{ - \bm{k} s' }^{\dagger}(\tau) c_{ \bm{k} s }^{\dagger}(0) \} \rangle
	e^{i \omega_{n} \tau}
	\ \ .
\end{align}
以下では均一な系に関してのみ考える。

Green関数のスペクトル表現を求める為に演算子の時間発展を求める。
Heisenbergの運動方程式を用いる。
\begin{align}
	\partial_{\tau} c_{\bm{k}s} (\tau)
	 & =
	\left[ \mathcal{H} \ , \ c_{\bm{k}s}(\tau) \right]
	\ = \
	e^{\mathcal{H} \tau} \left[ \mathcal{H} \ , \ c_{\bm{k}s} \right] e^{- \mathcal{H} \tau}
\end{align}
この算数には交換子と反交換子の関係
\begin{align}
	[AB \ , \ C]
	 & =
	A \{ B \ , \ C \} - \{ A \ , \ C \} B
	\ \ ,
\end{align}
も有用である。
$\mathcal{H}_{0}$に関して、
\begin{align}
	[ \mathcal{H}_{0} \ , \ c_{\bm{k}s} ]
	 & =
	\sum_{\bm{k}',s'}
	\varepsilon(\bm{k}')
	\left[
		c_{\bm{k}' s'}^{\dagger}
		c_{\bm{k}' s'}
		\ , \
		c_{\bm{k}s}
		\right]
	\nonumber \\[2mm] &=
	-
	\sum_{\bm{k}',s'}
	\varepsilon(\bm{k}')
	\left\{
	c_{\bm{k}' s'}^{\dagger}
	\ , \
	c_{\bm{k}s}
	\right\}
	c_{\bm{k}' s'}
	\nonumber \\[2mm] &=
	-
	\varepsilon( \bm{k} )
	c_{\bm{k} s}
	\ \ .
\end{align}
であり、続いて$\mathcal{H}_{\rm BCS}$に関して、
\begin{align}
	[ \mathcal{H}_{\rm BCS} \ , \ c_{\bm{k}s} ]
	 & =
	\left[
		\dfrac{1}{2}
		\sum_{\bm{k}',s_{1},s_{2}}
		\Big(
		\Delta_{ s_{1} s_{2} }( \bm{k}' )
		c_{\bm{k}' s_{1}}^{\dagger}
		c_{-\bm{k}' s_{2}}^{\dagger}
		\ - \
		\Delta_{ s_{1} s_{2} }^{*}( - \bm{k}' )
		c_{-\bm{k}' s_{1}}
		c_{\bm{k}' s_{2}}
		\Big)
		\ , \
		c_{\bm{k}s}
		\right]
	\nonumber \\[2mm] &=
	\dfrac{1}{2}
	\sum_{\bm{k}',s_{1},s_{2}}
	\Delta_{ s_{1} s_{2} }( \bm{k}' )
	\left[
		c_{\bm{k}' s_{1}}^{\dagger}
		c_{-\bm{k}' s_{2}}^{\dagger}
		\ , \
		c_{\bm{k}s}
		\right]
	\nonumber \\[2mm] &=
	\dfrac{1}{2}
	\sum_{\bm{k}',s_{1},s_{2}}
	\Delta_{ s_{1} s_{2} }( \bm{k}' )
	\Big(
	c_{\bm{k}' s_{1}}^{\dagger}
	\left\{
	c_{-\bm{k}' s_{2}}^{\dagger}
	\ , \
	c_{\bm{k}s}
	\right\}
	-
	\left\{
	c_{\bm{k}' s_{1}}^{\dagger}
	\ , \
	c_{\bm{k}s}
	\right\}
	c_{-\bm{k}' s_{2}}^{\dagger}
	\Big)
	\nonumber \\[2mm] &=
	\dfrac{1}{2}
	\sum_{\bm{k}',s_{1},s_{2}}
	\Delta_{ s_{1} s_{2} }( \bm{k}' )
	\Big(
	c_{\bm{k}' s_{1}}^{\dagger}
	\delta_{\bm{k},-\bm{k}'}
	\delta_{s,s_{2}}
	\ - \
	c_{-\bm{k}' s_{2}}^{\dagger}
	\delta_{\bm{k},\bm{k}'}
	\delta_{s,s_{1}}
	\Big)
	\ \ .
\end{align}
添字を注意して入れ替えると右辺の二項はそれぞれ等しいことが分かる。
すなわち、以下のように二項目に関してのみ添字の入れ替えの操作を注意深く行うと良い。
\begin{align}
	[ \mathcal{H}_{\rm BCS} \ , \ c_{\bm{k}s} ]
	 & =
	\dfrac{1}{2}
	\sum_{\bm{k}',s_{1},s_{2}}
	\Delta_{ s_{1} s_{2} }( \bm{k}' )
	c_{\bm{k}' s_{1}}^{\dagger}
	\delta_{\bm{k},-\bm{k}'}
	\delta_{s,s_{2}}
	\ - \
	\dfrac{1}{2}
	\sum_{\bm{k}',s_{1},s_{2}}
	\Delta_{ s_{1} s_{2} }( - \bm{k}' )
	c_{\bm{k}' s_{2}}^{\dagger}
	\delta_{\bm{k},-\bm{k}'}
	\delta_{s,s_{1}}
	\nonumber \\[2mm] &=
	\dfrac{1}{2}
	\sum_{\bm{k}',s_{1},s_{2}}
	\Delta_{ s_{1} s_{2} }( \bm{k}' )
	c_{\bm{k}' s_{1}}^{\dagger}
	\delta_{\bm{k},-\bm{k}'}
	\delta_{s,s_{2}}
	\ + \
	\dfrac{1}{2}
	\sum_{\bm{k}',s_{1},s_{2}}
	\Delta_{ s_{2} s_{1} }( \bm{k}' )
	c_{\bm{k}' s_{2}}^{\dagger}
	\delta_{\bm{k},-\bm{k}'}
	\delta_{s,s_{1}}
	\nonumber \\[2mm] &=
	\dfrac{1}{2}
	\sum_{\bm{k}',s_{1},s_{2}}
	\Delta_{ s_{1} s_{2} }( \bm{k}' )
	c_{\bm{k}' s_{1}}^{\dagger}
	\delta_{\bm{k},-\bm{k}'}
	\delta_{s,s_{2}}
	\ + \
	\dfrac{1}{2}
	\sum_{\bm{k}',s_{1},s_{2}}
	\Delta_{ s_{1} s_{2} }( \bm{k}' )
	c_{\bm{k}' s_{1}}^{\dagger}
	\delta_{\bm{k},-\bm{k}'}
	\delta_{s,s_{2}}
	\nonumber \\[2mm] &=
	\sum_{ s' }
	\Delta_{ s' s }( - \bm{k} )
	c_{ - \bm{k} s'}^{\dagger}
	\nonumber \\[2mm] &=
	-
	\sum_{ s' }
	\Delta_{ s s' }( \bm{k} )
	c_{ - \bm{k} s'}^{\dagger}
	\ \ .
\end{align}
以上の二項より演算子の時間発展は、
\begin{align}
	\partial_{\tau}
	c_{\bm{k}s} (\tau)
	 & =
	-
	\varepsilon( \bm{k} )
	c_{\bm{k} s} (\tau)
	\ - \
	\sum_{ s' }
	\Delta_{ s s' }( \bm{k} )
	c_{ - \bm{k} s'}^{\dagger} ( \tau )
	\ \ .
	\label{eqn:enzansihattenntime}
\end{align}
この関係を用いてGreen関数の運動方程式を導く。
Green関数の時間発展は、
\begin{align}
	\partial_{\tau} G_{ss'}(\bm{k} , \tau)
	 & =
	\partial_{\tau}
	\left(
	-
	\theta(\tau)
	\langle c_{\bm{k}s}(\tau) c_{\bm{k} s'}^{\dagger} \rangle
	\ + \
	\theta(- \tau)
	\langle c_{\bm{k} s'}^{\dagger} c_{\bm{k}s}(\tau) \rangle
	\right)
	\nonumber \\[4mm] &=
	-
	[\partial_{\tau} \theta(\tau) ]
	\langle c_{\bm{k}s}(\tau) c_{\bm{k} s'}^{\dagger} \rangle
	\ - \
	\theta(\tau)
	\langle
	[
		\partial_{\tau}
		c_{\bm{k}s}(\tau)
	]
	c_{\bm{k} s'}^{\dagger} \rangle
	\nonumber \\[2mm] &
	+
	[\partial_{\tau} \theta(- \tau) ]
	\langle c_{\bm{k} s'}^{\dagger} c_{\bm{k}s}(\tau) \rangle
	\ + \
	\theta(- \tau)
	\langle c_{\bm{k} s'}^{\dagger}
		[
			\partial_{\tau}
			c_{\bm{k}s}(\tau)
		]
	\rangle
	\nonumber \\[4mm] &=
	-
	\delta(\tau)
	\langle c_{\bm{k}s}(\tau) c_{\bm{k} s'}^{\dagger} \rangle
	\ + \
	\theta(\tau)
	\varepsilon( \bm{k} )
	\langle
	c_{\bm{k} s} (\tau)
	c_{\bm{k} s'}^{\dagger}
	\rangle
	\ + \
	\theta(\tau)
	\sum_{ s'' }
	\Delta_{ s s'' }( \bm{k} )
	\langle
	c_{ - \bm{k} s''}^{\dagger}(\tau)
	c_{\bm{k} s'}^{\dagger}
	\rangle
	\nonumber \\[2mm] &
	-
	\delta ( \tau )
	\langle c_{\bm{k} s'}^{\dagger} c_{\bm{k}s}(\tau) \rangle
	\ - \
	\theta(- \tau)
	\varepsilon( \bm{k} )
	\langle
	c_{\bm{k} s'}^{\dagger}
	c_{\bm{k} s} (\tau)
	\rangle
	\ - \
	\theta(-\tau)
	\sum_{ s'' }
	\Delta_{ s s'' }( \bm{k} )
	\langle
	c_{\bm{k} s'}^{\dagger}
	c_{ - \bm{k} s''}^{\dagger}(\tau)
	\rangle
	\nonumber \\[2mm] &=
	\varepsilon( \bm{k} )
	G_{ss'}(\bm{k} , \tau)
	\ + \
	\sum_{s''}
	\Delta_{ s s'' }( \bm{k} )
	F_{ s' s'' }^{\dagger} ( \bm{k} , \tau)
\end{align}
デルタ関数に比例する項は
$$
	\delta ( \tau )
	\langle c_{\bm{k} s'}^{\dagger} c_{\bm{k}s}(\tau) \rangle
	\ = \
	\delta ( 0 )
	\langle c_{\bm{k} s'}^{\dagger} c_{\bm{k}s} \rangle
	\ = \
	-
	\delta ( 0 )
	\langle c_{\bm{k}s} c_{\bm{k} s'}^{\dagger} \rangle
	\ = \
	-
	\delta ( \tau )
	\langle c_{\bm{k}s}(\tau) c_{\bm{k} s'}^{\dagger} \rangle
$$
の機構から相殺されている。
両辺の変数が$\tau \to i \omega_{n}$となるようにFourier変換すると、
\begin{align}
	\int^{\beta}_{0} d \tau
	\partial_{\tau}
	G_{ss'}(\bm{k} , \tau)
	e^{i \omega_{n} \tau}
	 & =
	\int^{\beta}_{0} d \tau
	\varepsilon( \bm{k} )
	G_{ss'}(\bm{k} , \tau)
	e^{i \omega_{n} \tau}
	\ + \
	\int^{\beta}_{0} d \tau
	\sum_{s''}
	\Delta_{ s'' s }( \bm{k} )
	F_{ s' s'' }^{\dagger} ( \bm{k} , \tau)
	e^{i \omega_{n} \tau}
	\nonumber \\[2mm]
	\longleftrightarrow \ \ \
	i \omega_{n}
	G_{ss'}(\bm{k} , i \omega_{n} )
	 & =
	\varepsilon( \bm{k} )
	G_{ss'}(\bm{k} , i \omega_{n} )
	\ + \
	\sum_{s''}
	\Delta_{ s s'' }( \bm{k} )
	F_{ s' s'' }^{\dagger} ( \bm{k} , i \omega_{n} )
\end{align}
この方程式は$G_{ss'}(\bm{k} , i \omega_{n} )$に関して閉じておらず、
$F_{ss'}^{\dagger}(\bm{k} , i \omega_{n} )$に関しても同様に運動方程式を立てて、
それぞれ連立して解く必要がある。
式\ref{eqn:enzansihattenntime}のエルミート共役を返して、
\begin{align}
	\partial_{\tau}
	c_{\bm{k}s}^{\dagger} (\tau)
	 & =
	-
	\varepsilon( \bm{k} )
	c_{\bm{k} s}^{\dagger} (\tau)
	\ - \
	\sum_{ s' }
	\Delta_{ s s' }^{*} ( \bm{k} )
	c_{ - \bm{k} s'} ( \tau )
	\ \ ,
\end{align}
の関係式を用いる。
\begin{align}
	\partial_{\tau}
	F_{ss'}^{\dagger} (\bm{k} , \tau)
	 & =
	\partial_{\tau}
	\left(
	\theta(\tau)
	\langle c_{-\bm{k}s'}^{\dagger} (\tau) c_{\bm{k}s}^{\dagger} \rangle
	\ - \
	\theta(- \tau)
	\langle c_{\bm{k} s}^{\dagger} c_{- \bm{k} s'}(\tau) \rangle
	\right)
	\nonumber \\[4mm] &=
	[ \partial_{\tau} \theta(\tau) ]
	\langle c_{-\bm{k}s'}^{\dagger} (\tau) c_{\bm{k}s}^{\dagger} \rangle
	\ + \
	\theta(\tau)
	\langle [ \partial_{\tau} c_{-\bm{k}s'}^{\dagger} (\tau) ] c_{\bm{k}s}^{\dagger} \rangle
	\nonumber \\[2mm] &
	-
	[ \partial_{\tau} \theta(- \tau) ]
	\langle c_{\bm{k} s}^{\dagger} c_{- \bm{k} s'}^{\dagger} (\tau) \rangle
	\ - \
	\theta(- \tau)
	\langle c_{\bm{k} s}^{\dagger} [ \partial_{\tau} c_{- \bm{k} s'}^{\dagger} (\tau) ] \rangle
	\nonumber \\[4mm] &=
	\delta(\tau)
	\langle c_{-\bm{k}s'}^{\dagger} (\tau) c_{\bm{k}s}^{\dagger} \rangle
	\ - \
	\varepsilon( \bm{k} )
	\theta(\tau)
	\langle
	c_{-\bm{k} s'}^{\dagger} (\tau)
	c_{\bm{k}s}^{\dagger}
	\rangle
	\ - \
	\sum_{ s'' }
	\Delta_{ s' s'' }^{*} ( - \bm{k} )
	\theta(\tau)
	\langle
	c_{ \bm{k} s''} ( \tau )
	c_{\bm{k}s}^{\dagger}
	\rangle
	\nonumber \\[2mm] &
	+
	\delta(\tau)
	\langle c_{\bm{k} s}^{\dagger} c_{- \bm{k} s'}^{\dagger} (\tau) \rangle
	\ + \
	\theta(- \tau)
	\varepsilon( \bm{k} )
	\langle
	c_{\bm{k} s}^{\dagger}
	c_{- \bm{k} s'}^{\dagger} (\tau)
	\rangle
	\ + \
	\sum_{ s'' }
	\Delta_{ s' s'' }^{*} ( - \bm{k} )
	\theta(- \tau)
	\langle
	c_{\bm{k} s}^{\dagger}
	c_{ \bm{k} s''} ( \tau )
	\rangle
	\nonumber \\[2mm] &=
	-
	\varepsilon( \bm{k} )
	F_{ss'}^{\dagger} (\bm{k} , \tau)
	\ - \
	\sum_{ s'' }
	\Delta_{ s' s'' }^{*} ( - \bm{k} )
	G_{s''s} (\bm{k} , \tau)
	\ \ ,
	\\[5mm]
	\longleftrightarrow \ \ \
	i \omega_{n}
	F_{ss'}^{\dagger} (\bm{k} , i \omega_{n} )
	 & =
	-
	\varepsilon( \bm{k} )
	F_{ss'}^{\dagger} (\bm{k} , i \omega_{n})
	\ - \
	\sum_{ s'' }
	\Delta_{ s' s'' }^{*} ( - \bm{k} )
	G_{s''s} (\bm{k} , i \omega_{n} )
	\ \ .
\end{align}
以上から、
それぞれのGreen関数のスペクトル表示を得るために解くべき連立方程式は次のようになる。
\begin{align}
	G_{ss'}(\bm{k} , i \omega_{n} )
	 & =
	\sum_{s''}
	\dfrac{
		\Delta_{ s s'' }( \bm{k} )
	}{
		i \omega_{n} - \varepsilon( \bm{k} )
	}
	F_{ s' s'' }^{\dagger} ( \bm{k} , i \omega_{n} )
	\ \ ,
	\\[4mm]
	F_{ss'}^{\dagger} (\bm{k} , i \omega_{n} )
	 & =
	-
	\sum_{ s'' }
	\dfrac{
		\Delta_{ s' s'' }^{*} ( - \bm{k} )
	}
	{
		i \omega_{n} + \varepsilon( \bm{k} )
	}
	G_{s''s} (\bm{k} , i \omega_{n} )
	\ \ .
\end{align}
対ポテンシャルの表現行列がユニタリーな場合のGor'kov方程式の解は、
素励起のエネルギースペクトルを
\begin{align}
	E_{\bm{k}}
	 & =
	\sqrt{
		\varepsilon^{2}(\bm{k}) + | \bm{d}(\bm{k}) |^{2}
	}
	\ \ ,
\end{align}
と書いて、
\begin{align}
	\hat{G}(\bm{k} , i \omega_{n})
	 & =
	-
	\dfrac{ i \omega_{n} + \varepsilon(\bm{k}) }
	{ \omega_{n}^{2} + E_{\bm{k}}^{2} }
	\hat{\sigma}_{0}
	\ \ ,
	\\[2mm]
	\hat{F}(\bm{k},i \omega_{n})
	 & =
	\dfrac{ i \bm{d}(\bm{k}) \cdot \hat{\bm{\sigma}} \hat{\sigma}_{y} }
	{ \omega_{n}^{2} + E_{\bm{k}}^{2} }
	\ = \
	\dfrac{ \hat{\Delta}(\bm{k}) }
	{ \omega_{n}^{2} + E_{\bm{k}}^{2} }
	\ \ .
\end{align}
と表現できる。
これらを用いて電子ラマンの応答関数を計算する。
Bosonicな松原周波数$\nu_{n}=2m\pi k_{B} T$、($m \in \mathbb{Z}$)を用いて、
\begin{align}
	\chi_{\tilde{\rho} \tilde{\rho}}(\bm{q},i \nu_{m})
	 & =
	-
	\int^{\beta}_{0} d \tau
	\langle T_{\tau}[ \tilde{\rho}_{\bm{q}}^{\dagger} (\tau) \tilde{\rho}_{\bm{q}} ] \rangle
	e^{i \nu_{m} \tau}
	\nonumber \\[2mm]
	 & =
	-
	\int^{\beta}_{0} d \tau
	\sum_{\bm{k}_{1} , \bm{k}_{2} , s_{1} ,s_{2} }
	\gamma_{\bm{k}_{1}}
	\gamma_{\bm{k}_{2}}
	\langle T_{\tau} [
			c_{ \bm{k}_{1}+\bm{q} , s_{1} }^{\dagger} (\tau)
			c_{ \bm{k}_{1} , s_{1} } (\tau)
			c_{ \bm{k}_{2}-\bm{q} , s_{2} }^{\dagger}
			c_{ \bm{k}_{2} , s_{2} }
		] \rangle
	e^{i \nu_{m} \tau}
	\nonumber \\[2mm]
	 & =
	-
	\int^{\beta}_{0} d \tau
	\sum_{\bm{k}_{1} , \bm{k}_{2} , s_{1} ,s_{2} }
	\gamma_{\bm{k}_{1}}
	\gamma_{\bm{k}_{2}}
	\Big\{
	\langle T_{\tau} [
			c_{ \bm{k}_{1}+\bm{q} , s_{1} }^{\dagger} (\tau)
			c_{ \bm{k}_{1} , s_{1} } (\tau)
		] \rangle
	\langle T_{\tau} [
			c_{ \bm{k}_{2}-\bm{q} , s_{2} }^{\dagger}
			c_{ \bm{k}_{2} , s_{2} }
		] \rangle
	\nonumber \\[2mm] &
	\hspace{25mm} -
	\langle T_{\tau} [
			c_{ \bm{k}_{1}+\bm{q} , s_{1} }^{\dagger} (\tau)
			c_{ \bm{k}_{2}-\bm{q} , s_{2} }^{\dagger}
		] \rangle
	\langle T_{\tau} [
			c_{ \bm{k}_{1} , s_{1} } (\tau)
			c_{ \bm{k}_{2} , s_{2} }
		] \rangle
	\nonumber \\[2mm] &
	\hspace{25mm} +
	\langle T_{\tau} [
			c_{ \bm{k}_{1}+\bm{q} , s_{1} }^{\dagger} (\tau)
			c_{ \bm{k}_{2} , s_{2} }
		] \rangle
	\langle T_{\tau} [
			c_{ \bm{k}_{1} , s_{1} } (\tau)
			c_{ \bm{k}_{2}-\bm{q} , s_{2} }^{\dagger}
		] \rangle
	\Big\}
	e^{i \nu_{m} \tau}
	\ \ .
\end{align}
最後の右辺一行目の括弧の中は同時刻のGreen関数であり、
定数となるので落とす。
符号も整理して、
\begin{align}
	\chi_{\tilde{\rho} \tilde{\rho}}(\bm{q},i \nu_{n})
	 & =
	\int^{\beta}_{0} d \tau
	\sum_{\bm{k}_{1} , \bm{k}_{2} , s_{1} ,s_{2} }
	\gamma_{\bm{k}_{1}}
	\gamma_{\bm{k}_{2}}
	\Big\{
	\langle T_{\tau} [
			c_{ \bm{k}_{1}+\bm{q} , s_{1} }^{\dagger} (\tau)
			c_{ \bm{k}_{2}-\bm{q} , s_{2} }^{\dagger}
		] \rangle
	\langle T_{\tau} [
			c_{ \bm{k}_{1} , s_{1} } (\tau)
			c_{ \bm{k}_{2} , s_{2} }
		] \rangle
	e^{i \nu_{m} \tau}
	\nonumber \\[2mm] &
	\hspace{25mm} -
	\langle T_{\tau} [
			c_{ \bm{k}_{1}+\bm{q} , s_{1} }^{\dagger} (\tau)
			c_{ \bm{k}_{2} , s_{2} }
		] \rangle
	\langle T_{\tau} [
			c_{ \bm{k}_{1} , s_{1} } (\tau)
			c_{ \bm{k}_{2}-\bm{q} , s_{2} }^{\dagger}
		] \rangle
	\Big\}
	e^{i \nu_{m} \tau}
	\ \ .
	\label{eqn:timeintegralbefore141125}
\end{align}

スピン$s_{1},s_{2}$に関する和を展開する。
括弧の中の第一項は、
\begin{align}
	 &
	\sum_{\bm{k}_{1} , \bm{k}_{2} , s_{1} ,s_{2} }
	\gamma_{\bm{k}_{1}}
	\gamma_{\bm{k}_{2}}
	\langle T_{\tau} [
			c_{ \bm{k}_{1}+\bm{q} , s_{1} }^{\dagger} (\tau)
			c_{ \bm{k}_{2}-\bm{q} , s_{2} }^{\dagger}
		] \rangle
	\langle T_{\tau} [
			c_{ \bm{k}_{1} , s_{1} } (\tau)
			c_{ \bm{k}_{2} , s_{2} }
		] \rangle
	\\ &=
	\sum_{\bm{k}_{1} , \bm{k}_{2} }
	\gamma_{\bm{k}_{1}}
	\gamma_{\bm{k}_{2}}
	\Big\{
	\langle T_{\tau} [
			c_{ \bm{k}_{1}+\bm{q} , \uparrow }^{\dagger} (\tau)
			c_{ \bm{k}_{2}-\bm{q} , \uparrow }^{\dagger}
		] \rangle
	\langle T_{\tau} [
			c_{ \bm{k}_{1} , \uparrow } (\tau)
			c_{ \bm{k}_{2} , \uparrow }
		] \rangle
	\nonumber \\ &
	+
	\langle T_{\tau} [
			c_{ \bm{k}_{1}+\bm{q} , \uparrow }^{\dagger} (\tau)
			c_{ \bm{k}_{2}-\bm{q} , \downarrow }^{\dagger}
		] \rangle
	\langle T_{\tau} [
			c_{ \bm{k}_{1} , \uparrow } (\tau)
			c_{ \bm{k}_{2} , \downarrow }
		] \rangle
	\nonumber \\[2mm] &
	+
	\langle T_{\tau} [
			c_{ \bm{k}_{1}+\bm{q} , \downarrow }^{\dagger} (\tau)
			c_{ \bm{k}_{2}-\bm{q} , \uparrow }^{\dagger}
		] \rangle
	\langle T_{\tau} [
			c_{ \bm{k}_{1} , \downarrow } (\tau)
			c_{ \bm{k}_{2} , \uparrow }
		] \rangle
	\nonumber \\[2mm] &
	+
	\langle T_{\tau} [
			c_{ \bm{k}_{1}+\bm{q} , \downarrow }^{\dagger} (\tau)
			c_{ \bm{k}_{2}-\bm{q} , \downarrow }^{\dagger}
		] \rangle
	\langle T_{\tau} [
			c_{ \bm{k}_{1} , \downarrow } (\tau)
			c_{ \bm{k}_{2} , \downarrow }
		] \rangle
	\Big\}
	\ \ .
\end{align}
運動量に関する和は$\bm{k}_{2}=-\bm{k}_{1}$の場合のみ残り、
$\bm{k}_{2}$に対する和を実行すると、
\begin{align}
	 &
	\sum_{\bm{k}_{1} , \bm{k}_{2} , s_{1} ,s_{2} }
	\gamma_{\bm{k}_{1}}
	\gamma_{\bm{k}_{2}}
	\langle T_{\tau} [
			c_{ \bm{k}_{1}+\bm{q} , s_{1} }^{\dagger} (\tau)
			c_{ \bm{k}_{2}-\bm{q} , s_{2} }^{\dagger}
		] \rangle
	\langle T_{\tau} [
			c_{ \bm{k}_{1} , s_{1} } (\tau)
			c_{ \bm{k}_{2} , s_{2} }
		] \rangle
	\\ &=
	\sum_{ \bm{k} }
	\gamma_{\bm{k}}
	\gamma_{-\bm{k}}
	\Big\{
	\langle T_{\tau} [
			c_{ - \bm{k} + \bm{q} , \uparrow }^{\dagger} (\tau)
			c_{ \bm{k} - \bm{q} , \uparrow }^{\dagger}
		] \rangle
	\langle T_{\tau} [
			c_{ - \bm{k} , \uparrow } (\tau)
			c_{ \bm{k} , \uparrow }
		] \rangle
	\nonumber \\ &
	+
	\langle T_{\tau} [
			c_{ - \bm{k} + \bm{q} , \uparrow }^{\dagger} (\tau)
			c_{ \bm{k} - \bm{q} , \downarrow }^{\dagger}
		] \rangle
	\langle T_{\tau} [
			c_{ - \bm{k} , \uparrow } (\tau)
			c_{ \bm{k} , \downarrow }
		] \rangle
	\nonumber \\[2mm] &
	+
	\langle T_{\tau} [
			c_{ - \bm{k} + \bm{q} , \downarrow }^{\dagger} (\tau)
			c_{ \bm{k} - \bm{q} , \uparrow }^{\dagger}
		] \rangle
	\langle T_{\tau} [
			c_{ - \bm{k} , \downarrow } (\tau)
			c_{ \bm{k} , \uparrow }
		] \rangle
	\nonumber \\[2mm] &
	+
	\langle T_{\tau} [
			c_{ - \bm{k} + \bm{q} , \downarrow }^{\dagger} (\tau)
			c_{ \bm{k} - \bm{q} , \downarrow }^{\dagger}
		] \rangle
	\langle T_{\tau} [
			c_{ - \bm{k} , \downarrow } (\tau)
			c_{ \bm{k} , \downarrow }
		] \rangle
	\Big\}
	\nonumber \\[2mm]
	 & =
	\sum_{ \bm{k} }
	\gamma_{\bm{k}}
	\gamma_{-\bm{k}}
	\Big\{
	F_{ \uparrow \uparrow }^{\dagger}( \bm{k} - \bm{q} , \tau )
	F_{ \uparrow \uparrow } ( - \bm{k} , \tau )
	\nonumber \\ &
	+
	F_{ \downarrow \uparrow }^{\dagger}( \bm{k} - \bm{q} , \tau )
	F_{ \uparrow \downarrow } ( - \bm{k} , \tau )
	\nonumber \\[2mm] &
	+
	F_{ \uparrow \downarrow }^{\dagger}( \bm{k} - \bm{q} , \tau )
	F_{ \downarrow \uparrow } ( - \bm{k} , \tau )
	\nonumber \\[2mm] &
	+
	F_{ \downarrow \downarrow }^{\dagger}( \bm{k} - \bm{q} , \tau )
	F_{ \downarrow \downarrow } ( - \bm{k} , \tau )
	\Big\}
	\ \ .
\end{align}

次に式(\ref{eqn:timeintegralbefore141125})の右辺第二項も同様に変形する。
スピンの和を取った後、運動量$\bm{k}_{2}$の和は$\bm{k}_{2} = \bm{k}_{1} + \bm{q}$の場合のみ残る。
\begin{align}
	 &
	\sum_{\bm{k}_{1} , \bm{k}_{2} , s_{1} ,s_{2} }
	\gamma_{\bm{k}_{1}}
	\gamma_{\bm{k}_{2}}
	\langle T_{\tau} [
			c_{ \bm{k}_{1}+\bm{q} , s_{1} }^{\dagger} (\tau)
			c_{ \bm{k}_{2} , s_{2} }
		] \rangle
	\langle T_{\tau} [
			c_{ \bm{k}_{1} , s_{1} } (\tau)
			c_{ \bm{k}_{2}-\bm{q} , s_{2} }^{\dagger}
		] \rangle
	\\ &=
	\sum_{\bm{k}_{1} , \bm{k}_{2} }
	\gamma_{\bm{k}_{1}}
	\gamma_{\bm{k}_{2}}
	\Big\{
	\langle T_{\tau} [
			c_{ \bm{k}_{1}+\bm{q} , \uparrow }^{\dagger} (\tau)
			c_{ \bm{k}_{2} , \uparrow }
		] \rangle
	\langle T_{\tau} [
			c_{ \bm{k}_{1} , \uparrow } (\tau)
			c_{ \bm{k}_{2}-\bm{q} , \uparrow }^{\dagger}
		] \rangle
	\nonumber \\[2mm] &
	+
	\langle T_{\tau} [
			c_{ \bm{k}_{1}+\bm{q} , \uparrow }^{\dagger} (\tau)
			c_{ \bm{k}_{2} , \downarrow }
		] \rangle
	\langle T_{\tau} [
			c_{ \bm{k}_{1} , \uparrow } (\tau)
			c_{ \bm{k}_{2}-\bm{q} , \downarrow }^{\dagger}
		] \rangle
	\nonumber \\[2mm] &
	+
	\langle T_{\tau} [
			c_{ \bm{k}_{1}+\bm{q} , \downarrow }^{\dagger} (\tau)
			c_{ \bm{k}_{2} , \uparrow }
		] \rangle
	\langle T_{\tau} [
			c_{ \bm{k}_{1} , \downarrow } (\tau)
			c_{ \bm{k}_{2}-\bm{q} , \uparrow }^{\dagger}
		] \rangle
	\nonumber \\[2mm] &
	+
	\langle T_{\tau} [
			c_{ \bm{k}_{1}+\bm{q} , \downarrow }^{\dagger} (\tau)
			c_{ \bm{k}_{2} , \downarrow }
		] \rangle
	\langle T_{\tau} [
			c_{ \bm{k}_{1} , \downarrow } (\tau)
			c_{ \bm{k}_{2}-\bm{q} , \downarrow }^{\dagger}
		] \rangle
	\Big\}
	\nonumber \\[3mm]
	 & =
	\sum_{ \bm{k} }
	\gamma_{\bm{k}}
	\gamma_{\bm{k}+\bm{q}}
	\Big\{
	\langle T_{\tau} [
			c_{ \bm{k}+\bm{q} , \uparrow }^{\dagger} (\tau)
			c_{ \bm{k}+\bm{q} , \uparrow }
		] \rangle
	\langle T_{\tau} [
			c_{ \bm{k} , \uparrow } (\tau)
			c_{ \bm{k} , \uparrow }^{\dagger}
		] \rangle
	\nonumber \\[2mm] &
	+
	\langle T_{\tau} [
			c_{ \bm{k} , \uparrow }^{\dagger} (\tau)
			c_{ \bm{k} , \downarrow }
		] \rangle
	\langle T_{\tau} [
			c_{ \bm{k} , \uparrow } (\tau)
			c_{ \bm{k} , \downarrow }^{\dagger}
		] \rangle
	\nonumber \\[2mm] &
	+
	\langle T_{\tau} [
			c_{ \bm{k}+\bm{q} , \downarrow }^{\dagger} (\tau)
			c_{ \bm{k}+\bm{q} , \uparrow }
		] \rangle
	\langle T_{\tau} [
			c_{ \bm{k} , \downarrow } (\tau)
			c_{ \bm{k} , \uparrow }^{\dagger}
		] \rangle
	\nonumber \\[2mm] &
	+
	\langle T_{\tau} [
			c_{ \bm{k}+\bm{q} , \downarrow }^{\dagger} (\tau)
			c_{ \bm{k}+\bm{q} , \downarrow }
		] \rangle
	\langle T_{\tau} [
			c_{ \bm{k} , \downarrow } (\tau)
			c_{ \bm{k} , \downarrow }^{\dagger}
		] \rangle
	\Big\}
	\nonumber \\[3mm]
	 & =
	-
	\sum_{ \bm{k} }
	\gamma_{\bm{k}}
	\gamma_{\bm{k}+\bm{q}}
	\Big\{
	G_{\uparrow \uparrow}( \bm{k}+\bm{q} , \tau)
	G_{\uparrow \uparrow}( \bm{k} , - \tau)
	\nonumber \\[-1mm] &
	+
	G_{\downarrow \downarrow}( \bm{k}+\bm{q} , \tau)
	G_{\downarrow \downarrow}( \bm{k} , - \tau)
	\Big\}
	\ \ .
\end{align}
ここで
ユニタリー条件
$G_{\uparrow \downarrow} = G_{\downarrow \uparrow} = 0$
を用いた。

まとめると、
\begin{align}
	 &
	\chi_{\tilde{\rho} \tilde{\rho}}(\bm{q},i \nu_{m})
	\ = \
	-
	\int^{\beta}_{0} d \tau
	\langle T_{\tau}[ \tilde{\rho}_{\bm{q}}^{\dagger} (\tau) \tilde{\rho}_{\bm{q}} ] \rangle
	e^{i \nu_{m} \tau}
	\nonumber \\[2mm]
	 & =
	\sum_{ \bm{k} }
	\int^{\beta}_{0} d \tau
	\Big\{
	\gamma_{\bm{k}}
	\gamma_{-\bm{k}}
	\Big[
		F_{ \uparrow \uparrow }^{\dagger}( \bm{k} - \bm{q} , \tau )
		F_{ \uparrow \uparrow } ( - \bm{k} , \tau )
		\ + \
		F_{ \downarrow \uparrow }^{\dagger}( \bm{k} - \bm{q} , \tau )
		F_{ \uparrow \downarrow } ( - \bm{k} , \tau )
	\nonumber \\[-1mm] & \hspace{29mm} +
	F_{ \uparrow \downarrow }^{\dagger}( \bm{k} - \bm{q} , \tau )
	F_{ \downarrow \uparrow } ( - \bm{k} , \tau )
	\ + \
	F_{ \downarrow \downarrow }^{\dagger}( \bm{k} - \bm{q} , \tau )
	F_{ \downarrow \downarrow } ( - \bm{k} , \tau )
	\Big]
	\nonumber \\[1mm] & \hspace{15mm} -
	\gamma_{\bm{k}}
	\gamma_{\bm{k}+\bm{q}}
	\Big[
		G_{\uparrow \uparrow}( \bm{k}+\bm{q} , \tau)
		G_{\uparrow \uparrow}( \bm{k} , - \tau)
		\ + \
		G_{\downarrow \downarrow}( \bm{k}+\bm{q} , \tau)
		G_{\downarrow \downarrow}( \bm{k} , - \tau)
		\Big]
	\Big\}
	e^{i \nu_{m} \tau}
\end{align}
続いて時間に関してFourier変換を行う。
異常Green関数に関して、
\begin{align}
	\int^{\beta}_{0} d \tau
	F_{ s' s }^{\dagger}( \bm{k} - \bm{q} , \tau )
	F_{ s s' } ( - \bm{k} , \tau )
	e^{i \nu_{m} \tau}
	 & =
	\dfrac{1}{\beta^{2}}
	\int^{\beta}_{0} d \tau
	\sum_{n_{1},n_{2}}
	F_{ s' s }^{\dagger}( \bm{k} - \bm{q} , i \omega_{n_{1}} )
	F_{ s s' } ( - \bm{k} , i \omega_{n_{2}} )
	e^{i ( \nu_{m} - \omega_{n_{1}} - \omega_{n_{2}} ) \tau }
	\nonumber \\[2mm] &=
	\dfrac{ 1 }{\beta^{2}} \sum_{n_{1},n_{2}}
	F_{ s' s }^{\dagger}( \bm{k} - \bm{q} , i \omega_{n_{1}} )
	F_{ s s' } ( - \bm{k} , i \omega_{n_{2}} )
	\ \ \beta
	\delta_{ \omega_{n_{1}} , \nu_{m} - \omega_{n_{2}} }
	\nonumber \\[2mm] &=
	\dfrac{ 1 }{\beta} \sum_{n}
	F_{ s' s }^{\dagger}( \bm{k} - \bm{q} , i \nu_{m} - i \omega_{n} )
	F_{ s s' } ( - \bm{k} , i \omega_{n} )
\end{align}
同様にして、
\begin{align}
	\int^{\beta}_{0} d \tau
	G_{ s s } ( \bm{k} + \bm{q} , \tau )
	G_{ s s } ( \bm{k} , - \tau )
	e^{i \nu_{m} \tau}
	 & =
	\dfrac{1}{\beta^{2}}
	\int^{\beta}_{0} d \tau
	\sum_{n_{1},n_{2}}
	G_{ s s } ( \bm{k} + \bm{q} , i \omega_{n_{1}} )
	G_{ s s } ( \bm{k} , i \omega_{n_{2}} )
	e^{i ( \nu_{m} - \omega_{n_{1}} + \omega_{n_{2}} ) \tau }
	\nonumber \\[2mm] &=
	\dfrac{ 1 }{\beta^{2}} \sum_{n_{1},n_{2}}
	G_{ s s } ( \bm{k} + \bm{q} , i \omega_{n_{1}} )
	G_{ s s } ( \bm{k} , i \omega_{n_{2}} )
	\ \ \beta
	\delta_{ \omega_{n_{1}} , \nu_{m} + \omega_{n_{2}} }
	\nonumber \\[2mm] &=
	\dfrac{ 1 }{\beta} \sum_{n}
	G_{ s s } ( \bm{k} + \bm{q} , i \nu_{m} + i \omega_{n} )
	G_{ s s } ( \bm{k} , i \omega_{n} )
\end{align}
以上から応答関数は次のようになる。
\begin{align}
	 &
	\chi_{\tilde{\rho} \tilde{\rho}}(\bm{q},i \nu_{m})
	\ = \
	-
	\int^{\beta}_{0} d \tau
	\langle T_{\tau}[ \tilde{\rho}_{\bm{q}}^{\dagger} (\tau) \tilde{\rho}_{\bm{q}} ] \rangle
	e^{i \nu_{m} \tau}
	\nonumber \\[2mm]
	 & =
	\dfrac{1}{\beta}
	\sum_{n}
	\sum_{ \bm{k} }
	\sum_{s s'}
	\Big[
		\gamma_{\bm{k}}
		\gamma_{-\bm{k}}
		F_{ s' s }^{\dagger}( \bm{k} - \bm{q} , i \nu_{m} - i \omega_{n} )
		F_{ s s' } ( - \bm{k} , i \omega_{n} )
	\nonumber \\ & \hspace{15mm} -
		\gamma_{\bm{k}}
		\gamma_{\bm{k}+\bm{q}}
		G_{ s s } ( \bm{k} + \bm{q} , i \nu_{m} + i \omega_{n} )
		G_{ s' s' } ( \bm{k} , i \omega_{n} )
		\Big]
\end{align}

$\bm{q} \to \bm{0}$の極限を考える。
これ以降はこの場合のみ考えるので、
$\chi_{\tilde{\rho} \tilde{\rho}}(\bm{0},i \nu_{m}) = \chi_{\tilde{\rho} \tilde{\rho}}(i \nu_{m})$
と略記する。
\begin{align}
	\chi_{\tilde{\rho} \tilde{\rho}}(i \nu_{m})
	 & =
	\dfrac{1}{\beta}
	\sum_{n}
	\sum_{ \bm{k} }
	\sum_{s s'}
	\Big[
		\gamma_{\bm{k}}
		\gamma_{-\bm{k}}
		F_{ s' s }^{\dagger}( \bm{k} , i \nu_{m} - i \omega_{n} )
		F_{ s s' } ( - \bm{k} , i \omega_{n} )
	\nonumber \\[2mm] & \hspace{15mm} -
	\gamma_{\bm{k}}^{2}
	G_{ s s } ( \bm{k} , i \nu_{m} + i \omega_{n} )
	G_{ s' s' } ( \bm{k} , i \omega_{n} )
	\Big]
	\nonumber \\[2mm]
	 & =
	\dfrac{1}{\beta}
	\sum_{n}
	\sum_{ \bm{k} }
	\left[
		\gamma_{\bm{k}}
		\gamma_{-\bm{k}}
		\sum_{s s'}
		\dfrac{ \Delta_{s' s}(\bm{k}) }{ ( \nu_{m} - \omega_{n} )^{2} + E_{\bm{k}}^{2} }
		\
		\dfrac{ \Delta_{s s'}(-\bm{k}) }{ \omega_{n}^{2} + E_{-\bm{k}}^{2} }
		\right.
	\nonumber \\ & \hspace{15mm} \left.
		-
		\gamma_{\bm{k}}^{2}
		\dfrac{ i \omega_{n} + i \nu_{m} + \varepsilon(\bm{k}) }{ ( \nu_{m} + \omega_{n} )^{2} + E_{\bm{k}}^{2} }
		\
		\dfrac{ i \omega_{n} + \varepsilon(\bm{k}) }{ \omega_{n}^{2} + E_{\bm{k}}^{2} }
		\right]
\end{align}

\end{document}
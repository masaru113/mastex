\documentclass[uplatex,a4j,12pt,dvipdfmx]{jsarticle}
\usepackage{amsmath,amsthm,amssymb,bm,color,enumitem,mathrsfs,url,epic,eepic,ascmac,ulem,here,ascmac}
\usepackage[letterpaper,top=2cm,bottom=2cm,left=3cm,right=3cm,marginparwidth=1.75cm]{geometry}
\usepackage[english]{babel}
\usepackage[dvipdfm]{graphicx}
\usepackage[hypertex]{hyperref}
\usepackage{tikz-cd}
\title{
スピン空間における異方的BCS-南部グリーン関数
}
\author{Masaru Okada}

\date{\today}

\begin{document}

\maketitle

\begin{abstract}
	このノートでは、スピン一重項および三重項を扱える南部 $\otimes$ スピン空間における異方的BCS平均場ハミルトニアンを導入し、ボゴリューボフ変換を用いて対角化することでその固有値と変換行列を導出する。
	さらに、運動方程式を用いて南部 $\otimes$ スピン空間におけるグリーン関数(通常および異常成分)を導出し、その具体的な表式を与える。
\end{abstract}

\tableofcontents

\section{(南部 $\!\! \otimes \!\!$ スピン)空間の平均場ハミルトニアン}

\subsection{スピン一重項(従来のBCS)の場合}

まずはハミルトニアンから始めよう。

\begin{eqnarray}
	H
	&=&
	H_{0}
	+
	H_{\rm MF}
\end{eqnarray}
%
ここで、

\begin{eqnarray}
	H_{0}
	&=&
	\sum_{\vec{k},\sigma}
	\xi_{\vec{k}}
	c^{\dagger}_{\vec{k} \sigma}
	c_{\vec{k} \sigma}
	\ \ = \ \
	\sum_{\vec{k}}
	\left(
	c_{\vec{k} \uparrow}^{\dagger} \ \ , \ \
	c_{-\vec{k} \downarrow}
	\right)
	\!\!
	\left(
	\begin{array}{cc}
			\xi_{\vec{k}} & 0
			\\[3mm]
			0             & -\xi_{\vec{k}}
		\end{array}
	\right)
	\!\!\!
	\left(
	\begin{array}{c}
			c_{\vec{k} \uparrow} \\[3mm]
			c_{-\vec{k} \downarrow}^{\dagger}
		\end{array}
	\right)
	,
	\\[4mm]
	H_{\rm MF}
	&=&
	\Delta^{*}
	\sum_{\vec{k}}
	\big(
	c_{-\vec{k} \downarrow}
	c_{\vec{k} \uparrow}
	-
	c_{-\vec{k} \uparrow}
	c_{\vec{k} \downarrow}
	\big)
	+
	\Delta
	\sum_{\vec{k}}
	\big(
	c^{\dagger}_{\vec{k} \uparrow}
	c^{\dagger}_{-\vec{k} \downarrow}
	-
	c^{\dagger}_{-\vec{k} \uparrow}
	c^{\dagger}_{\vec{k} \downarrow}
	\big)
	\\ &=&
	\Delta
	\sum_{\vec{k}, \sigma}
	c^{\dagger}_{\vec{k} \sigma}
	c^{\dagger}_{-\vec{k} \bar{\sigma}}
	\ \ +
	{\rm H.c.}
	\\ &=&
	\sum_{\vec{k}}
	\left(
	c_{\vec{k} \uparrow}^{\dagger} \ \ , \ \
	c_{-\vec{k} \downarrow}
	\right)
	\!\!
	\left(
	\begin{array}{cc}
			0          & \Delta \\[3mm]
			\Delta^{*} & 0
		\end{array}
	\right)
	\!\!\!
	\left(
	\begin{array}{c}
			c_{\vec{k} \uparrow} \\[3mm]
			c_{-\vec{k} \downarrow}^{\dagger}
		\end{array}
	\right)
	,
\end{eqnarray}
%
すると、

\begin{eqnarray}
	\vec{c}^{\ \dagger}_{\vec{k}}
	&=&
	\left(
	c_{\vec{k} \uparrow}^{\dagger} \ \ , \ \
	c_{-\vec{k} \downarrow}
	\right)
	\ , \hspace{10mm}
	\vec{c}_{\vec{k}}
	\ \ = \ \
	\left(
	\begin{array}{c}
			c_{\vec{k} \uparrow} \\[3mm]
			c_{-\vec{k} \downarrow}^{\dagger}
		\end{array}
	\right)
	,
\end{eqnarray}
%
は2成分(南部)スピノルと呼ばれる。
また、異常期待値 $\Delta$ は次のように定義される。

\begin{eqnarray}
	\Delta
	&=&
	\sum_{\vec{k}}
	\left\langle
	c_{\vec{k} \uparrow}
	c_{-\vec{k} \downarrow}
	\right\rangle
\end{eqnarray}
%
これで、このハミルトニアン $H$ はスピノル $\vec{c}_{\vec{k}}^{\ (\dagger)}$ と $2 \times 2$ 行列 $\hat{H}$ を使って表すことができる。

\begin{eqnarray}
	H
	&=&
	\sum_{\vec{k}}
	\vec{c}^{\ \dagger}_{\vec{k}}
	\!\!
	\left(
	\begin{array}{cc}
			\xi_{\vec{k}} & \Delta         \\[3mm]
			\Delta^{*}    & -\xi_{\vec{k}}
		\end{array}
	\right)
	\!\!
	\vec{c}_{\vec{k}}
	\ \ = \ \
	\sum_{\vec{k}}
	\vec{c}^{\ \dagger}_{\vec{k}}
	\
	\hat{H}
	\
	\vec{c}_{\vec{k}}
	.
\end{eqnarray}
%

$\hat{H}$ は任意の実数パラメータ $\lambda$ を用いて容易に対角化できる。

\begin{eqnarray}
	{\rm det} ( \hat{H} - \lambda \ \! \hat{1}_{2 \times 2})
	&=&
	0
	\\[2mm]
	\longrightarrow
	\ \ \
	\lambda
	&=&
	\pm
	\sqrt{ \xi_{\vec{k}}^{2} + | \Delta_{\vec{k}} |^{2} }
	\ = \
	\pm E_{\vec{k}}
	.
\end{eqnarray}
%
これで固有値 $E_{\vec{k}}$ が定義された。

対角化された基底 $\vec{a}$ を得るために、ボゴリューボフ変換の行列 $\hat{U}$ を用いる。

\begin{eqnarray}
	\vec{c}_{\vec{k}}
	&=&
	\hat{U}
	\vec{a}_{\vec{k}}
	\ = \
	\left(
	\begin{array}{cc}
			u_{\vec{k}}                 & - v_{\vec{k}} e^{i \varphi}
			\\[3mm]
			v_{\vec{k}} e^{- i \varphi} & u_{\vec{k}}
		\end{array}
	\right)
	\!\!\!
	\left(
	\begin{array}{c}
			a_{\vec{k} \uparrow} \\[3mm]
			a_{-\vec{k} \downarrow}^{\dagger}
		\end{array}
	\right)
	\\[2mm]
	\hspace{5mm}
	H
	&=&
	\sum_{\vec{k}}
	\vec{c}^{\ \dagger}_{\vec{k}}
	\
	\hat{H}
	\
	\vec{c}_{\vec{k}}
	\ = \
	\sum_{\vec{k}}
	\vec{a}^{\ \dagger}_{\vec{k}}
	\
	\hat{U}^{\dagger}
	\
	\hat{H}
	\
	\hat{U}
	\
	\vec{a}_{\vec{k}}
	\ = \
	\sum_{\vec{k}}
	\vec{a}^{\ \dagger}_{\vec{k}}
	\left(
	\begin{array}{cc}
			E_{\vec{k}} & 0
			\\[3mm]
			0           & - E_{\vec{k}}
		\end{array}
	\right)
	\vec{a}_{\vec{k}}
	.
\end{eqnarray}
%
特に知る必要があるのはユニタリーな場合($\hat{U}^{\dagger} = \hat{U}^{-1}$ となり、整数 $n$ に対して $\varphi=2 \pi n$ とおける場合)だけ。

\begin{eqnarray}
	\hat{1}_{2 \times 2}
	&=&
	\hat{U}^{\dagger}
	\hat{U}
	\ = \
	\left(
	\begin{array}{cc}
			u_{\vec{k}}                   & v_{\vec{k}} e^{i \varphi}
			\\[3mm]
			- v_{\vec{k}} e^{- i \varphi} & u_{\vec{k}}
		\end{array}
	\right)
	\!\!\!
	\left(
	\begin{array}{cc}
			u_{\vec{k}}                 & - v_{\vec{k}} e^{i \varphi}
			\\[3mm]
			v_{\vec{k}} e^{- i \varphi} & u_{\vec{k}}
		\end{array}
	\right)
	\ = \
	\left(
	\begin{array}{cc}
			u_{\vec{k}}^{2} + v_{\vec{k}}^{2} & 0
			\\[3mm]
			0                                 & u_{\vec{k}}^{2} + v_{\vec{k}}^{2}
		\end{array}
	\right)
	.
\end{eqnarray}
%
$u_{\vec{k}}^{2} + v_{\vec{k}}^{2} = 1$ という条件が現れた。
行列 $\hat{U}$ の成分に関する連立方程式を解くこともできる。

\begin{eqnarray}
	\hat{H}
	\
	\hat{U}
	&=&
	\hat{U}
	\left(
	\begin{array}{cc}
			E_{\vec{k}} & 0
			\\[3mm]
			0           & - E_{\vec{k}}
		\end{array}
	\right)
	\hspace{5mm}
	\longleftarrow
	\hspace{5mm}
	\hat{U}^{\dagger}
	\
	\hat{H}
	\
	\hat{U}
	\ = \
	\left(
	\begin{array}{cc}
			E_{\vec{k}} & 0
			\\[3mm]
			0           & - E_{\vec{k}}
		\end{array}
	\right)
	\\[2mm]
	\left(
	\begin{array}{cc}
			u_{\vec{k}} \xi_{\vec{k}} + v_{\vec{k}} \Delta     & u_{\vec{k}} \Delta - v_{\vec{k}} \xi_{\vec{k}}
			\\[3mm]
			\Delta^{*} u_{\vec{k}} - v_{\vec{k}} \xi_{\vec{k}} & - v_{\vec{k}} \Delta^{*} - u_{\vec{k}} \xi_{\vec{k}}
		\end{array}
	\right)
	&=&
	E_{\vec{k}}
	\left(
	\begin{array}{cc}
			u_{\vec{k}} & v_{\vec{k}}
			\\[3mm]
			v_{\vec{k}} & - u_{\vec{k}}
		\end{array}
	\right)
	\\[3mm]
	\longrightarrow
	\hspace{5mm}
	u_{\vec{k}}^{2}
	&=&
	\dfrac{1}{2} \left( 1 + \dfrac{\xi_{\vec{k}}}{E_{\vec{k}}} \right)
	, \hspace{5mm}
	v_{\vec{k}}^{2}
	\ = \
	\dfrac{1}{2} \left( 1 - \dfrac{\xi_{\vec{k}}}{E_{\vec{k}}} \right)
\end{eqnarray}
%
これで、スピン一重項の場合について、固有値 $E_{\vec{k}}$ とボゴリューボフ変換行列の成分 $u_{\vec{k}}$, $v_{\vec{k}}$ が、既知の値 $\xi_{\vec{k}}$, $\Delta$ で書き表された。



\subsection{一般化されたスピンの場合}

次に、スピン一重項の場合だけでなく、任意のスピン三重項の場合についても物理量を計算できるように、2成分の南部スピノル $\vec{c}_{\vec{k}}$ を4成分のものに一般化する。

\begin{eqnarray}
	H
	\ = \
	\dfrac{1}{2}
	\sum_{\vec{k}}
	\vec{c}^{\ \dagger}_{\vec{k}}
	\hat{H}
	\vec{c}_{\vec{k}}
	&=&
	\dfrac{1}{2}
	\sum_{\vec{k}}
	\big( c^{\dagger}_{\vec{k} \uparrow} \ , \ c^{\dagger}_{\vec{k} \downarrow} \ , \ c_{-\vec{k} \uparrow} \ , \ c_{-\vec{k} \downarrow}  \big)
	\!\!\!
	\left(
	\begin{array}{cc}
			\xi_{\vec{k}} \hat{1}_{2 \times 2} & \hat{\Delta}_{\vec{k}}               \\[3mm]
			\hat{\Delta}^{*}_{\vec{k}}         & - \xi_{\vec{k}} \hat{1}_{2 \times 2}
		\end{array}
	\right)
	\!\!\!
	\left(
	\begin{array}{c}
			c_{\vec{k} \uparrow}            \\[2mm]
			c_{\vec{k} \downarrow}          \\[2mm]
			c^{\dagger}_{-\vec{k} \uparrow} \\[2mm]
			c^{\dagger}_{-\vec{k} \downarrow}
		\end{array}
	\right)
\end{eqnarray}
%
異常期待値から構成される行列 $\hat{\Delta}_{\vec{k}}$ の定義は、

\begin{eqnarray}
	(\hat{\Delta}_{\vec{k}})_{\sigma\sigma'}
	&=&
	-V \sum_{\vec{k}'} \vec{k} \cdot \vec{k'}
	\big\langle c_{\vec{k'} \sigma} c_{-\vec{k}' \sigma'} \big\rangle
	.
\end{eqnarray}
%
クーパー対の全角運動量に垂直なベクトル $\vec{d}$ の成分を、行列 $\hat{\Delta}_{\vec{k}}$ の基底として選ぶことができる。

\begin{eqnarray}
	\hat{\Delta}_{\vec{k}}
	&=&
	\left(
	\begin{array}{cc}
			- d^{x}_{\vec{k}} + i d^{y}_{\vec{k}} & d^{z}_{\vec{k}}                     \\[2mm]
			d^{z}_{\vec{k}}                       & d^{x}_{\vec{k}} + i d^{y}_{\vec{k}}
		\end{array}
	\right)
	.
\end{eqnarray}
%
行列 $\hat{\Delta}_{\vec{k}}$ がユニタリーであるとき、

\begin{eqnarray}
	\hat{\Delta}_{\vec{k}} \hat{\Delta}^{\dagger}_{\vec{k}}
	&=&
	\left(
	\begin{array}{cc}
			{d^{x}_{\vec{k}}}^{2} + {d^{y}_{\vec{k}}}^{2} + {d^{z}_{\vec{k}}}^{2} & 0                                                                     \\[3mm]
			0                                                                     & {d^{x}_{\vec{k}}}^{2} + {d^{y}_{\vec{k}}}^{2} + {d^{z}_{\vec{k}}}^{2}
		\end{array}
	\right)
	\ \ \propto \ \
	\hat{1}_{2 \times 2}
\end{eqnarray}
%
すなわち、

\begin{eqnarray}
	{d^{x}_{\vec{k}}}^{2} + {d^{y}_{\vec{k}}}^{2} + {d^{z}_{\vec{k}}}^{2}
	\ = \
	\dfrac{1}{2}
	{\rm Tr} \big[ \hat{\Delta}_{\vec{k}} \hat{\Delta}^{\dagger}_{\vec{k}} \big]
	, \hspace{5mm}
	\hat{\Delta}_{\vec{k}} \hat{\Delta}^{\dagger}_{\vec{k}}
	\ = \
	\dfrac{1}{2}
	{\rm Tr} \big[ \hat{\Delta}_{\vec{k}} \hat{\Delta}^{\dagger}_{\vec{k}} \big]
	\ \
	\hat{1}_{2 \times 2}
	.
\end{eqnarray}
%
すると、$4 \times 4$ のハミルトニアンは具体的に次のように書ける。

\begin{eqnarray}
	\hat{H}
	&=&
	\left(
	\begin{array}{cccc}
			\xi_{\vec{k}}                         & 0                                   & - d^{x}_{\vec{k}} + i d^{y}_{\vec{k}} & d^{z}_{\vec{k}}                     \\[2mm]
			0                                     & \xi_{\vec{k}}                       & d^{z}_{\vec{k}}                       & d^{x}_{\vec{k}} + i d^{y}_{\vec{k}} \\[2mm]
			- d^{x}_{\vec{k}} - i d^{y}_{\vec{k}} & d^{z}_{\vec{k}}                     & - \xi_{\vec{k}}                       & 0                                   \\[2mm]
			d^{z}_{\vec{k}}                       & d^{x}_{\vec{k}} - i d^{y}_{\vec{k}} & 0                                     & - \xi_{\vec{k}}
		\end{array}
	\right)
	.
\end{eqnarray}
%
固有値方程式 ${\rm det}(\hat{H} - \lambda \hat{1}_{4 \times 4})$ は、いくつかの面倒なプロセスを経れば解くことができる。

\begin{eqnarray}
	0
	&=&
	\left|
	\begin{array}{cccc}
		{\xi_{\vec{k}}} - \lambda                & 0                                      & - {d^{x}_{\vec{k}}} + i{d^{y}_{\vec{k}}} & {d^{z}_{\vec{k}}}                      \\[2mm]
		0                                        & {\xi_{\vec{k}}} - \lambda              & {d^{z}_{\vec{k}}}                        & {d^{x}_{\vec{k}}} + i{d^{y}_{\vec{k}}} \\[2mm]
		- {d^{x}_{\vec{k}}} - i{d^{y}_{\vec{k}}} & {d^{z}_{\vec{k}}}                      & - {\xi_{\vec{k}}} -\lambda               & 0                                      \\[2mm]
		{d^{z}_{\vec{k}}}                        & {d^{x}_{\vec{k}}} - i{d^{y}_{\vec{k}}} & 0                                        & - {\xi_{\vec{k}}} -\lambda
	\end{array}
	\right|
	\nonumber \\[3mm] &=&
	({\xi_{\vec{k}}} - \lambda)
	\left|
	\begin{array}{ccc}
		{\xi_{\vec{k}}} - \lambda              & {d^{z}_{\vec{k}}}          & {d^{x}_{\vec{k}}} + i{d^{y}_{\vec{k}}} \\[2mm]
		{d^{z}_{\vec{k}}}                      & - {\xi_{\vec{k}}} -\lambda & 0                                      \\[2mm]
		{d^{x}_{\vec{k}}} - i{d^{y}_{\vec{k}}} & 0                          & - {\xi_{\vec{k}}} -\lambda
	\end{array}
	\right|
	\nonumber \\[3mm] && +
	(- {d^{x}_{\vec{k}}} - i{d^{y}_{\vec{k}}})
	\left|
	\begin{array}{ccc}
		0                                      & - {d^{x}_{\vec{k}}} + i{d^{y}_{\vec{k}}} & {d^{z}_{\vec{k}}}                      \\[2mm]
		{\xi_{\vec{k}}} - \lambda              & {d^{z}_{\vec{k}}}                        & {d^{x}_{\vec{k}}} + i{d^{y}_{\vec{k}}} \\[2mm]
		{d^{x}_{\vec{k}}} - i{d^{y}_{\vec{k}}} & 0                                        & - {\xi_{\vec{k}}} -\lambda
	\end{array}
	\right|
	\nonumber \\[3mm] && -
	{d^{z}_{\vec{k}}}
	\left|
	\begin{array}{ccc}
		0                         & - {d^{x}_{\vec{k}}} + i{d^{y}_{\vec{k}}} & {d^{z}_{\vec{k}}}                      \\[2mm]
		{\xi_{\vec{k}}} - \lambda & {d^{z}_{\vec{k}}}                        & {d^{x}_{\vec{k}}} + i{d^{y}_{\vec{k}}} \\[2mm]
		{d^{z}_{\vec{k}}}         & - {\xi_{\vec{k}}} -\lambda               & 0
	\end{array}
	\right|
	.
\end{eqnarray}
%
右辺の各項は、以下のように展開できる。

\begin{eqnarray}
	({\xi_{\vec{k}}} - \lambda)
	\left|
	\begin{array}{ccc}
		{\xi_{\vec{k}}} - \lambda              & {d^{z}_{\vec{k}}}          & {d^{x}_{\vec{k}}} + i{d^{y}_{\vec{k}}} \\[2mm]
		{d^{z}_{\vec{k}}}                      & - {\xi_{\vec{k}}} -\lambda & 0                                      \\[2mm]
		{d^{x}_{\vec{k}}} - i{d^{y}_{\vec{k}}} & 0                          & - {\xi_{\vec{k}}} -\lambda
	\end{array}
	\right|
	&=&
	( {\xi_{\vec{k}}}^{2} - \lambda^{2} )
	( {\xi_{\vec{k}}}^{2} - \lambda^{2} + {d^{x}_{\vec{k}}}^{2} +{d^{y}_{\vec{k}}}^{2} + {d^{z}_{\vec{k}}}^{2} )
	\\[3mm]
	(- {d^{x}_{\vec{k}}} - i{d^{y}_{\vec{k}}})
	\left|
	\begin{array}{ccc}
		0                                      & - {d^{x}_{\vec{k}}} + i{d^{y}_{\vec{k}}} & {d^{z}_{\vec{k}}}                      \\[2mm]
		{\xi_{\vec{k}}} - \lambda              & {d^{z}_{\vec{k}}}                        & {d^{x}_{\vec{k}}} + i{d^{y}_{\vec{k}}} \\[2mm]
		{d^{x}_{\vec{k}}} - i{d^{y}_{\vec{k}}} & 0                                        & - {\xi_{\vec{k}}} -\lambda
	\end{array}
	\right|
	&=&
	({d^{x}_{\vec{k}}}^{2} +{d^{y}_{\vec{k}}}^{2})
	( {\xi_{\vec{k}}}^{2} - \lambda^{2} + {d^{x}_{\vec{k}}}^{2} +{d^{y}_{\vec{k}}}^{2} + {d^{z}_{\vec{k}}}^{2} )
	\\[3mm]
	-
	{d^{z}_{\vec{k}}}
	\left|
	\begin{array}{ccc}
		0                         & - {d^{x}_{\vec{k}}} + i{d^{y}_{\vec{k}}} & {d^{z}_{\vec{k}}}      \\[2mm]
		{\xi_{\vec{k}}} - \lambda & {d^{z}_{\vec{k}}}                        & {d^{x}_{\vec{k}}} + iy \\[2mm]
		z                         & - {\xi_{\vec{k}}} -\lambda               & 0
	\end{array}
	\right|
	&=&
	{d^{z}_{\vec{k}}}^{2}
	( {\xi_{\vec{k}}}^{2} - \lambda^{2} + {d^{x}_{\vec{k}}}^{2} +{d^{y}_{\vec{k}}}^{2} + {d^{z}_{\vec{k}}}^{2} )
\end{eqnarray}
%
全ての項をまとめると、次のようになる。

\begin{eqnarray}
	( {\xi_{\vec{k}}}^{2} - \lambda^{2} + {d^{x}_{\vec{k}}}^{2} +{d^{y}_{\vec{k}}}^{2} + {d^{z}_{\vec{k}}}^{2} )^{2}
	&=&
	0
	\\[3mm]
	\to \hspace{2mm}
	\lambda
	&=&
	\pm \sqrt{ \xi_{\vec{k}}^{2} + {d^{x}_{\vec{k}}}^{2} + {d^{y}_{\vec{k}}}^{2} + {d^{z}_{\vec{k}}}^{2} }
	=
	\pm \sqrt{ \xi_{\vec{k}}^{2} + \dfrac{1}{2} {\rm Tr} \big[ \hat{\Delta}_{\vec{k}} \hat{\Delta}^{\dagger}_{\vec{k}} \big] }
	=
	\pm E_{\vec{k}}
\end{eqnarray}
%
最終的に、$4 \times 4$ 行列 $\hat{H}$ に対する固有値 $E_{\vec{k}}$ を得ることができた。

対角化された基底 $\vec{a}$(これは2成分ベクトルではなく4成分ベクトルである)をどのように表現するかを知るため、$2 \times 2$ のブロック行列 $\hat{u}_{\vec{k}}$ と $\hat{v}_{\vec{k}}$ を用いて、$4 \times 4$ のボゴリューボフ変換行列 $\hat{U}$ を定義する。

\begin{eqnarray}
	\vec{c}_{\vec{k}}
	&=&
	\left(
	\begin{array}{cc}
			\hat{u}_{\vec{k}}       & - \hat{v}_{\vec{k}} \\[3mm]
			\hat{v}_{- \vec{k}}^{*} & \hat{u}_{- \vec{k}}
		\end{array}
	\right)
	\!\!\!
	\left(
	\begin{array}{c}
			a_{\vec{k} \uparrow}            \\[2mm]
			a_{\vec{k} \downarrow}          \\[2mm]
			a^{\dagger}_{-\vec{k} \uparrow} \\[2mm]
			a^{\dagger}_{-\vec{k} \downarrow}
		\end{array}
	\right)
	\ \ = \ \
	\hat{U}
	\vec{a}_{\vec{k}}
\end{eqnarray}
%
\ \\[-10mm]

\begin{eqnarray}
	H
	&=&
	\dfrac{1}{2}
	\sum_{\vec{k}}
	\vec{c}^{\ \dagger}_{\vec{k}}
	\
	\hat{H}
	\
	\vec{c}_{\vec{k}}
	\ = \
	\dfrac{1}{2}
	\sum_{\vec{k}}
	\vec{a}^{\ \dagger}_{\vec{k}}
	\
	\hat{U}^{\dagger}
	\
	\hat{H}
	\
	\hat{U}
	\
	\vec{a}_{\vec{k}}
	\ = \
	\dfrac{1}{2}
	\sum_{\vec{k}}
	\vec{a}^{\ \dagger}_{\vec{k}}
	\left(
	\begin{array}{cccc}
			E_{\vec{k}} & 0           & 0             & 0
			\\[3mm]
			0           & E_{\vec{k}} & 0             & 0
			\\[3mm]
			0           & 0           & - E_{\vec{k}} & 0
			\\[3mm]
			0           & 0           & 0             & - E_{\vec{k}}
		\end{array}
	\right)
	\vec{a}_{\vec{k}}
	.
\end{eqnarray}
%
この時点では、$2 \times 2$ のブロック行列 $\hat{u}_{\vec{k}}$ と $\hat{v}_{\vec{k}}$ は未知である。

\begin{eqnarray}
	\left(
	\begin{array}{cc}
		\hat{u}_{\vec{k}}       & - \hat{v}_{\vec{k}} \\[2mm]
		\hat{v}_{- \vec{k}}^{*} & \hat{u}_{-\vec{k}}
	\end{array}
	\right)^{\dagger}
	\!\!\!
	\left(
	\begin{array}{cc}
			\xi_{\vec{k}} \hat{1}_{2 \times 2} & \hat{\Delta}_{\vec{k}}               \\[2mm]
			\hat{\Delta}^{*}_{\vec{k}}         & - \xi_{\vec{k}} \hat{1}_{2 \times 2}
		\end{array}
	\right)
	\!\!\!
	\left(
	\begin{array}{cc}
			\hat{u}_{\vec{k}}       & - \hat{v}_{\vec{k}} \\[2mm]
			\hat{v}_{- \vec{k}}^{*} & \hat{u}_{-\vec{k}}
		\end{array}
	\right)
	&=&
	\left(
	\begin{array}{cc}
			E_{\vec{k}} \hat{1}_{2 \times 2} & 0                                  \\[2mm]
			0                                & - E_{\vec{k}} \hat{1}_{2 \times 2}
		\end{array}
	\right)
	\\[2mm]
	\left(
	\begin{array}{cc}
			\xi_{\vec{k}} \hat{1}_{2 \times 2} & \hat{\Delta}_{\vec{k}}               \\[2mm]
			\hat{\Delta}^{*}_{\vec{k}}         & - \xi_{\vec{k}} \hat{1}_{2 \times 2}
		\end{array}
	\right)
	\!\!\!
	\left(
	\begin{array}{cc}
			\hat{u}_{\vec{k}}       & - \hat{v}_{\vec{k}} \\[2mm]
			\hat{v}_{- \vec{k}}^{*} & \hat{u}_{-\vec{k}}
		\end{array}
	\right)
	&=&
	\left(
	\begin{array}{cc}
			\hat{u}_{\vec{k}}       & - \hat{v}_{\vec{k}} \\[2mm]
			\hat{v}_{- \vec{k}}^{*} & \hat{u}_{-\vec{k}}
		\end{array}
	\right)
	\!\!\!
	\left(
	\begin{array}{cc}
			E_{\vec{k}} \hat{1}_{2 \times 2} & 0                                  \\[2mm]
			0                                & - E_{\vec{k}} \hat{1}_{2 \times 2}
		\end{array}
	\right)
	\hspace{10mm}
\end{eqnarray}
%
この手順が許されるのは、$\hat{U}$ がユニタリー($\hat{U}^{\dagger} = \hat{U}^{-1}$)だからである。
ただちに、行列 $\hat{u}_{\vec{k}}$ と $\hat{v}_{\vec{k}}$ に対するいくつかの拘束条件が得られる。

\begin{eqnarray}
	\left\{
	\begin{array}{rccl}
		\hat{u}_{\vec{k}}
		 & =                                      &
		\dfrac{ \hat{\Delta}_{\vec{k}} }{ E_{\vec{k}} - \xi_{\vec{k}} }
		\hat{v}_{-\vec{k}}^{*}
		 & \hspace{10mm} \cdots \ \ (1,1,{\rm A})
		\\[5mm]
		\hat{v}_{\vec{k}}
		 & =                                      &
		\dfrac{ \hat{\Delta}_{\vec{k}} }{ E_{\vec{k}} + \xi_{\vec{k}} } \hat{u}_{-\vec{k}}
		 & \hspace{10mm} \cdots \ \ (1,2,{\rm A})
		\\[5mm]
		\hat{v}_{-\vec{k}}^{*}
		 & =                                      &
		\dfrac{ \hat{\Delta}^{*}_{\vec{k}} }{ E_{\vec{k}} + \xi_{\vec{k}} } \hat{u}_{\vec{k}}
		 & \hspace{10mm} \cdots \ \ (2,1,{\rm A})
		\\[5mm]
		\hat{u}_{-\vec{k}}
		 & =                                      &
		\dfrac{ \hat{\Delta}^{*}_{\vec{k}} }{ E_{\vec{k}} - \xi_{\vec{k}} } \hat{v}_{\vec{k}}
		 & \hspace{10mm} \cdots \ \ (2,2,{\rm A})
	\end{array}
	\right.
\end{eqnarray}
%
%%%%
%%%%
\if0
	%%%%
	%%%
	%
	\begin{eqnarray}
		\left(
		\begin{array}{cc}
				\hat{u}_{-\vec{k}}        & \hat{v}_{\vec{k}} \\[2mm]
				- \hat{v}_{- \vec{k}}^{*} & \hat{u}_{\vec{k}}
			\end{array}
		\right)
		\!\!\!
		\left(
		\begin{array}{cc}
				\hat{u}_{\vec{k}}       & - \hat{v}_{\vec{k}} \\[2mm]
				\hat{v}_{- \vec{k}}^{*} & \hat{u}_{-\vec{k}}
			\end{array}
		\right)
		\ \ = \ \
		\hat{1}_{4 \times 4}
		.
	\end{eqnarray}
	%
	Each components have these relations as follows,

	\begin{eqnarray}
		\left\{
		\begin{array}{rccl}
			\hat{u}_{-\vec{k}} \hat{v}_{\vec{k}}
			 & =                                      &
			\hat{v}_{\vec{k}} \hat{u}_{-\vec{k}}
			 & \hspace{10mm} \cdots \ \ (1,2,{\rm B})
			\\[2mm]
			\hat{v}_{-\vec{k}}^{*} \hat{u}_{\vec{k}}
			 & =                                      &
			\hat{u}_{\vec{k}} \hat{v}_{-\vec{k}}^{*}
			 & \hspace{10mm} \cdots \ \ (2,1,{\rm B})
		\end{array}
		\right.
		,
		\hspace{10mm}
		\left\{
		\begin{array}{rccl}
			\hat{u}_{-\vec{k}} \hat{u}_{\vec{k}} + \hat{v}_{\vec{k}} \hat{v}_{-\vec{k}}^{*}
			 & =                                      &
			\hat{1}_{2 \times 2}
			 & \hspace{10mm} \cdots \ \ (1,1,{\rm B})
			\\[2mm]
			- \hat{v}_{-\vec{k}}^{*} \hat{v}_{\vec{k}} + \hat{u}_{\vec{k}} \hat{u}_{-\vec{k}}
			 & =                                      &
			\hat{1}_{2 \times 2}
			 & \hspace{10mm} \cdots \ \ (2,2,{\rm B})
		\end{array}
		\right.
	\end{eqnarray}
	%
	First, we use the (1,2,B)-component,

	\begin{eqnarray}
		\begin{array}{rcrcl}
			 &   &
			\hat{u}_{-\vec{k}} \hat{v}_{\vec{k}}
			 & = &
			\hat{v}_{\vec{k}} \hat{u}_{-\vec{k}}
			\nonumber \\[4mm] &\longleftrightarrow& \hspace{8mm}
			\Big( - \dfrac{ \hat{\Delta}^{\dagger}_{\vec{k}} }{ E_{\vec{k}} - \xi_{\vec{k}} } \hat{v}_{\vec{k}} \Big)
			\hat{v}_{\vec{k}}
			 & = &
			\hat{v}_{\vec{k}}
			\Big( - \dfrac{ \hat{\Delta}^{\dagger}_{\vec{k}} }{ E_{\vec{k}} - \xi_{\vec{k}} } \hat{v}_{\vec{k}} \Big)
			\nonumber \\[4mm] &\longleftrightarrow& \hspace{8mm}
			\hat{\Delta}^{\dagger}_{\vec{k}} \hat{v}_{\vec{k}}
			 & = &
			\hat{v}_{\vec{k}} \hat{\Delta}^{\dagger}_{\vec{k}}
		\end{array}
		\\[-5mm]
	\end{eqnarray}
	%
	Second, (2,1,B)-component is a commutation relation,

	\begin{eqnarray}
		\begin{array}{rcrcl}
			 &   &
			\hat{v}_{-\vec{k}}^{*} \hat{u}_{\vec{k}}
			 & = &
			\hat{u}_{\vec{k}} \hat{v}_{-\vec{k}}^{*}
			\nonumber \\[4mm] &\longleftrightarrow& \hspace{8mm}
			\Big( \dfrac{ \hat{\Delta}^{\dagger}_{\vec{k}} }{ E_{\vec{k}} + \xi_{\vec{k}} } \hat{u}_{\vec{k}} \Big)
			\hat{u}_{\vec{k}}
			 & = &
			\hat{u}_{\vec{k}}
			\Big( \dfrac{ \hat{\Delta}^{\dagger}_{\vec{k}} }{ E_{\vec{k}} + \xi_{\vec{k}} } \hat{u}_{\vec{k}} \Big)
			\nonumber \\[4mm] &\longleftrightarrow& \hspace{8mm}
			\hat{\Delta}^{\dagger}_{\vec{k}} \hat{u}_{\vec{k}}
			 & = &
			\hat{u}_{\vec{k}} \hat{\Delta}^{\dagger}_{\vec{k}}
		\end{array}
		\\[-5mm]
	\end{eqnarray}
	%
	Therefore we can find the relations

	\begin{eqnarray}
		\hat{u}_{\vec{k}}
		\ \ = \ \
		\dfrac{ \hat{\Delta}_{\vec{k}} \hat{u}_{\vec{k}} \hat{\Delta}^{\dagger}_{\vec{k}} }{ \dfrac{1}{2} {\rm Tr} \big[ \hat{\Delta}_{\vec{k}} \hat{\Delta}^{\dagger}_{\vec{k}} \big] }
		\ \ = \ \
		\dfrac{ \hat{\Delta}^{\dagger}_{\vec{k}} \hat{u}_{\vec{k}} \hat{\Delta}_{\vec{k}} }{ \dfrac{1}{2} {\rm Tr} \big[ \hat{\Delta}_{\vec{k}} \hat{\Delta}^{\dagger}_{\vec{k}} \big] }
		\\[5mm]
		\hat{v}_{\vec{k}}
		\ \ = \ \
		\dfrac{ \hat{\Delta}_{\vec{k}} \hat{v}_{\vec{k}} \hat{\Delta}^{\dagger}_{\vec{k}} }{ \dfrac{1}{2} {\rm Tr} \big[ \hat{\Delta}_{\vec{k}} \hat{\Delta}^{\dagger}_{\vec{k}} \big] }
		\ \ = \ \
		\dfrac{ \hat{\Delta}^{\dagger}_{\vec{k}} \hat{v}_{\vec{k}} \hat{\Delta}_{\vec{k}} }{ \dfrac{1}{2} {\rm Tr} \big[ \hat{\Delta}_{\vec{k}} \hat{\Delta}^{\dagger}_{\vec{k}} \big] }
	\end{eqnarray}
	%If $\hat{u}_{\vec{k}}$、$\hat{v}_{\vec{k}}$ are invertible,
	we can see

	\begin{eqnarray}
		\begin{array}{rcrcl}
			 &   &
			\hat{u}_{-\vec{k}} \hat{u}_{\vec{k}} - \hat{v}_{\vec{k}} \hat{v}_{-\vec{k}}^{*}
			 & = &
			\hat{1}_{2 \times 2}
			\nonumber \\[4mm] &\longleftrightarrow& \hspace{8mm}
			\Big( - \dfrac{ \hat{\Delta}^{\dagger}_{\vec{k}} }{ E_{\vec{k}} - \xi_{\vec{k}} } \hat{v}_{\vec{k}} \Big)
			\hat{u}_{\vec{k}}
			-
			\hat{v}_{\vec{k}}
			\Big( \dfrac{ \hat{\Delta}^{\dagger}_{\vec{k}} }{ E_{\vec{k}} + \xi_{\vec{k}} } \hat{u}_{\vec{k}} \Big)
			 & = &
			\hat{1}_{2 \times 2}
			\nonumber \\[4mm] &\longleftrightarrow& \hspace{8mm}
			\Big(
			- \dfrac{ \hat{\Delta}^{\dagger}_{\vec{k}} }{ E_{\vec{k}} - \xi_{\vec{k}} } \hat{v}_{\vec{k}}
			-
			\hat{v}_{\vec{k}}
			\dfrac{ \hat{\Delta}^{\dagger}_{\vec{k}} }{ E_{\vec{k}} + \xi_{\vec{k}} }
			\Big)
			\hat{u}_{\vec{k}}
			 & = &
			\hat{1}_{2 \times 2}
			\nonumber \\[4mm] &\longleftrightarrow& \hspace{8mm}
			{\hat{u}_{\vec{k}}}^{-1}
			 & = &
			- \dfrac{ \hat{\Delta}^{\dagger}_{\vec{k}} }{ E_{\vec{k}} - \xi_{\vec{k}} } \hat{v}_{\vec{k}}
			-
			\hat{v}_{\vec{k}}
			\dfrac{ \hat{\Delta}^{\dagger}_{\vec{k}} }{ E_{\vec{k}} + \xi_{\vec{k}} }
			\nonumber \\[4mm] &\longleftrightarrow& \hspace{8mm}
			 & = &
			- \hat{v}_{\vec{k}} \hat{\Delta}^{\dagger}_{\vec{k}}
			\Big(
			\dfrac{ 1 }{ E_{\vec{k}} - \xi_{\vec{k}} }
			+
			\dfrac{ 1 }{ E_{\vec{k}} + \xi_{\vec{k}} }
			\Big)
		\end{array}
	\end{eqnarray}
	%
	From (2,2,B)-component, ${\hat{v}_{\vec{k}}}^{-1}$ is also able to be

	\begin{eqnarray}
		\begin{array}{rcrcl}
			 &   &
			- \hat{v}_{-\vec{k}}^{*} \hat{v}_{\vec{k}} + \hat{u}_{\vec{k}} \hat{u}_{-\vec{k}}
			 & = &
			\hat{1}_{2 \times 2}
			\nonumber \\[4mm] &\longleftrightarrow& \hspace{8mm}
			- \Big( \dfrac{ \hat{\Delta}^{\dagger}_{\vec{k}} }{ E_{\vec{k}} + \xi_{\vec{k}} } \hat{u}_{\vec{k}} \Big)
			\hat{v}_{\vec{k}}
			+
			\hat{u}_{\vec{k}}
			\Big( - \dfrac{ \hat{\Delta}^{\dagger}_{\vec{k}} }{ E_{\vec{k}} - \xi_{\vec{k}} } \hat{v}_{\vec{k}} \Big)
			 & = &
			\hat{1}_{2 \times 2}
			\nonumber \\[4mm] &\longleftrightarrow& \hspace{8mm}
			\Big( - \dfrac{ \hat{\Delta}^{\dagger}_{\vec{k}} }{ E_{\vec{k}} + \xi_{\vec{k}} } \hat{u}_{\vec{k}}
			-
			\hat{u}_{\vec{k}}
			\dfrac{ \hat{\Delta}^{\dagger}_{\vec{k}} }{ E_{\vec{k}} - \xi_{\vec{k}} } \Big)
			\hat{v}_{\vec{k}}
			 & = &
			\hat{1}_{2 \times 2}
			\nonumber \\[4mm] &\longleftrightarrow& \hspace{8mm}
			{\hat{v}_{\vec{k}}}^{-1}
			 & = &
			- \dfrac{ \hat{\Delta}^{\dagger}_{\vec{k}} }{ E_{\vec{k}} + \xi_{\vec{k}} } \hat{u}_{\vec{k}}
			-
			\hat{u}_{\vec{k}}
			\dfrac{ \hat{\Delta}^{\dagger}_{\vec{k}} }{ E_{\vec{k}} - \xi_{\vec{k}} }
			\nonumber \\[7mm] &&
			 & = &
			- \hat{u}_{\vec{k}} \hat{\Delta}^{\dagger}_{\vec{k}}
			\Big(
			\dfrac{ 1 }{ E_{\vec{k}} + \xi_{\vec{k}} }
			+
			\dfrac{ 1 }{ E_{\vec{k}} - \xi_{\vec{k}} }
			\Big)
		\end{array}
		\\[-7mm]
	\end{eqnarray}
	%
	It's put it all together, we can find the relation

	\begin{eqnarray}
		\left\{
		\begin{array}{rcl}
			{\hat{u}_{\vec{k}}}^{-1}
			 & = &
			- \hat{v}_{\vec{k}} \hat{\Delta}^{\dagger}_{\vec{k}}
			\dfrac{ 2 E_{\vec{k}} }{ \dfrac{1}{2} {\rm Tr} \big[ \hat{\Delta}_{\vec{k}} \hat{\Delta}^{\dagger}_{\vec{k}} \big] }
			\\[7mm]
			{\hat{v}_{\vec{k}}}^{-1}
			 & = &
			- \hat{u}_{\vec{k}} \hat{\Delta}^{\dagger}_{\vec{k}}
			\dfrac{ 2 E_{\vec{k}} }{ \dfrac{1}{2} {\rm Tr} \big[ \hat{\Delta}_{\vec{k}} \hat{\Delta}^{\dagger}_{\vec{k}} \big] }
		\end{array}
		\right.
	\end{eqnarray}
	%
	%%%%
	%%%%
\fi
%%%%
%%%%
これらの関係は、もし $\hat{u}_{\vec{k}}$(あるいは $\hat{v}_{\vec{k}}$)が単位行列 $\hat{1}_{2 \times 2}$ に比例するならば、
$\hat{v}_{\vec{k}}$(あるいは $\hat{u}_{\vec{k}}$)は $\hat{\Delta}_{\vec{k}}$ に比例することを示している。
ここでは、$\hat{u}_{\vec{k}} \propto \hat{1}_{2 \times 2}$ という条件を選ぶことにしよう。

\begin{eqnarray}
	\hat{u}_{\vec{k}}
	\ \ = \ \
	\dfrac{ \hat{1}_{2 \times 2} }{ f(\vec{k}) }
	\ ,
	\hspace{10mm}
	\hat{u}_{\vec{k}}^{-1}
	\ \ = \ \
	f({\vec{k}}) \hat{1}_{2 \times 2}
\end{eqnarray}
%
このとき、$f(\vec{k})$ は未知の関数である。
いかにして関数 $f(\vec{k})$ の表現を得るか、という問題に行き着いた。

\begin{eqnarray}
	\left\{
	\begin{array}{rccll}
		\hat{v}_{\vec{k}}
		 & =                   &
		\dfrac{ \hat{\Delta}_{\vec{k}} }{ E_{\vec{k}} + \xi_{\vec{k}} } \hat{u}_{-\vec{k}}
		 &
		= \ \
		\dfrac{ \hat{\Delta}_{\vec{k}} }{ E_{\vec{k}} + \xi_{\vec{k}} } \cdot \dfrac{1}{f(-\vec{k})}
		\\[5mm]
		\hat{u}_{-\vec{k}}
		 & =                   &
		\dfrac{ \hat{\Delta}^{\dagger}_{\vec{k}} }{ E_{\vec{k}} - \xi_{\vec{k}} } \hat{v}_{\vec{k}}
		 &
		= \ \
		\dfrac{ \dfrac{1}{2} {\rm Tr} \big[ \hat{\Delta}_{\vec{k}} \hat{\Delta}^{\dagger}_{\vec{k}} \big] }{ E^{2}_{\vec{k}} - \xi^{2}_{\vec{k}} } \cdot \dfrac{1}{f(-\vec{k})}
		\hat{1}_{2 \times 2}
		\ \ = \ \
		\dfrac{ \hat{1}_{2 \times 2} }{ f(-\vec{k}) }
		 & \ \ \ \cdots \ (自明)
		\\[5mm]
		\hat{v}_{-\vec{k}}^{*}
		 & =                   &
		\dfrac{ \hat{\Delta}^{\dagger}_{\vec{k}} }{ E_{\vec{k}} + \xi_{\vec{k}} } \hat{u}_{\vec{k}}
		 &
		= \ \
		\dfrac{ \hat{\Delta}^{\dagger}_{\vec{k}} }{ E_{\vec{k}} + \xi_{\vec{k}} } \cdot \dfrac{1}{f(\vec{k})}
		\\[5mm]
		\hat{u}_{\vec{k}}
		 & =                   &
		\dfrac{ \hat{\Delta}_{\vec{k}} }{ E_{\vec{k}} - \xi_{\vec{k}} }
		\hat{v}_{-\vec{k}}^{*}
		 &
		= \ \
		\dfrac{ \dfrac{1}{2} {\rm Tr} \big[ \hat{\Delta}_{\vec{k}} \hat{\Delta}^{\dagger}_{\vec{k}} \big] }{ E^{2}_{\vec{k}} - \xi^{2}_{\vec{k}} } \cdot \dfrac{1}{f(\vec{k})}
		\hat{1}_{2 \times 2}
		\ \ = \ \
		\dfrac{ \hat{1}_{2 \times 2} }{ f(\vec{k}) }
		 & \ \ \ \cdots \ (自明)
	\end{array}
	\right.
\end{eqnarray}
%
これらの拘束条件に加えて、$\hat{U}$ に対するユニタリー条件は次のように記述できる。

\begin{eqnarray}
	1
	\ = \
	\left|
	{\rm det}
	\left(
	\begin{array}{cc}
			\hat{u}_{\vec{k}}       & - \hat{v}_{\vec{k}} \\[2mm]
			\hat{v}_{- \vec{k}}^{*} & \hat{u}_{-\vec{k}}
		\end{array}
	\right)
	\right|
	&=&
	\left|
	{\rm det} \Big( \hat{u}_{\vec{k}} \Big)
	\
	{\rm det} \Big[ \hat{u}_{- \vec{k}} - \hat{v}_{- \vec{k}}^{*} \hat{u}_{\vec{k}}^{-1} ( - \hat{v}_{\vec{k}} ) \Big]
	\right|
	\nonumber \\[2mm]
	\longleftrightarrow
	\hspace{10mm}
	\Big|
	{\rm det} \Big( \hat{u}_{\vec{k}} \Big) {\rm det} \Big( \hat{u}_{- \vec{k}} \Big)
	+
	{\rm det} \Big( \hat{v}_{\vec{k}} \Big) {\rm det} \Big( \hat{v}_{- \vec{k}}^{*} \Big)
	\Big|
	&=&
	1
	,
\end{eqnarray}
%
これは次の関係を導く。

\begin{eqnarray}
	1
	&=&
	\Big|
	{\rm det} \Big( \hat{u}_{\vec{k}} \Big) {\rm det} \Big( \hat{u}_{- \vec{k}} \Big)
	+
	{\rm det} \Big( \hat{v}_{\vec{k}} \Big) {\rm det} \Big( \hat{v}_{- \vec{k}}^{*} \Big)
	\Big|
	\nonumber \\[2mm] \longleftrightarrow \hspace{9mm}
	\Big| f(\vec{k})f(-\vec{k}) \Big|
	&=&
	1
	+
	\dfrac{
		\dfrac{1}{2} {\rm Tr} \big[ \hat{\Delta}_{\vec{k}} \hat{\Delta}^{\dagger}_{\vec{k}} \big]
	}
	{
		\Big( E_{\vec{k}} + \xi_{\vec{k}} \Big)^{2}
	}
\end{eqnarray}
%
したがって、$\hat{u}_{\vec{k}} = \hat{u}_{ - \vec{k} }$ のとき、次を得る。

\begin{eqnarray}
	\left(
	\begin{array}{cc}
			\hat{u}_{\vec{k}}       & - \hat{v}_{\vec{k}} \\[2mm]
			\hat{v}_{- \vec{k}}^{*} & \hat{u}_{-\vec{k}}
		\end{array}
	\right)
	&=&
	\dfrac{ 1 }{ \sqrt{ \big( E_{\vec{k}} + \xi_{\vec{k}} \big)^{2} + \dfrac{1}{2} {\rm Tr} \big[ \hat{\Delta}_{\vec{k}} \hat{\Delta}^{\dagger}_{\vec{k}} \big] } }
	\left(
	\begin{array}{cc}
			( E_{\vec{k}} + \xi_{\vec{k}} ) \hat{1}_{2 \times 2} & - \hat{\Delta}_{\vec{k}}                             \\[2mm]
			\hat{\Delta}_{\vec{k}}^{\dagger}                     & ( E_{\vec{k}} + \xi_{\vec{k}} ) \hat{1}_{2 \times 2}
		\end{array}
	\right)
\end{eqnarray}
%
これがこの小節の目標(原文: gorl)である。

この結果は、三重項の場合から一重項の場合に帰着できるだろうか?
秩序パラメータの (2,1) 成分 $(\hat{\Delta}_{\vec{k}})_{\uparrow \downarrow} = d_{z} = \Delta$ を考えてみよう。

\begin{eqnarray}
	{(\hat{v}_{\vec{k}})_{\uparrow \downarrow}}^{2}
	&=&
	\dfrac{ \Delta^{2} }{ \big( E_{\vec{k}} + \xi_{\vec{k}} \big)^{2} + E_{\vec{k}}^{2} - \xi_{\vec{k}}^{2} }
	\nonumber \\[2mm] &=&
	\dfrac{1}{2}
	\left( 1 - \dfrac{\xi_{\vec{k}}}{E_{\vec{k}}} \right)
\end{eqnarray}
%
この値は、前の小節で現れた一重項の場合の $v_{\vec{k}}$ と一致している。

\section{(南部 $\!\! \otimes \!\!$ スピン)空間のグリーン関数}

本節では、異方的(スピン依存)BCSモデルに対するグリーン関数を導出する。
まず、運動方程式から出発する。
\subsection{スピン一重項(従来のBCS)の場合}

南部空間におけるグリーン関数($2 \times 2$ 行列)を、今、次のように定義する。

\begin{eqnarray}
	i \hat{G}(k)
	&=&
	\int \! dx
	\left\langle \hat{\rm T} \left[ \vec{c}_{\vec{k}}(x) \vec{c}_{\vec{k}}^{\ \dagger} \right] \right\rangle
	e^{ i k \cdot x }
	\ \ = \ \
	\int \! dx
	\left\langle \hat{\rm T} \left[
		\left(
		\!\!
		\begin{array}{c}
				c_{\vec{k} \uparrow}(x)
				\\[3mm]
				c_{-\vec{k} \downarrow}^{\dagger}(x)
			\end{array}
		\!\!
		\right)
		\left( c_{\vec{k} \uparrow}^{\dagger} \ \ , \ \ c_{-\vec{k} \downarrow} \right)
		\right]
	\right\rangle
	e^{ i k \cdot x }
	\nonumber \\[3mm] &=&
	\int \! dx
	\left\langle \hat{\rm T} \left[
		\left(
		\begin{array}{cc}
			c_{\vec{k}\uparrow}(x) c_{\vec{k}\uparrow}^{\dagger}              & c_{\vec{k}\uparrow}(x) c_{-\vec{k}\downarrow}
			\\[3mm]
			c_{-\vec{k}\downarrow}^{\dagger}(x) c_{\vec{k}\uparrow}^{\dagger} & c_{-\vec{k}\downarrow}^{\dagger}(x) c_{-\vec{k}\downarrow}
		\end{array}
		\right)
		\right] \right\rangle
	e^{ i k \cdot x }
	\ = \
	i
	\int \! dx
	\left(
	\begin{array}{cc}
			G(x)       & F(x)
			\\[3mm]
			\bar{F}(x) & \bar{G}(x)
		\end{array}
	\right)
	e^{ i k \cdot x }
\end{eqnarray}
%
ここで $\hat{\rm T} \left[ \cdots \right]$ は時間順序積演算子である。
$x^{\mu}=(t,\vec{r})$ および $k^{\mu}=(\omega,\vec{k})$
( i.e., $k \cdot x = g_{\mu \nu} k^{\mu} x^{\nu} = \omega t- \vec{k} \cdot \vec{r}$ )
は、簡略記法で書かれた4元運動量ベクトルである。
これらの関数の遅延(retarded)部分も、次のように定義される。

\begin{eqnarray}
	i
	\hat{G}^{R}(k)
	&=&
	\int \! dx
	\left\langle
	\left(
	\begin{array}{cc}
			\left\{ c_{\vec{k}\uparrow}(x), c_{\vec{k}\uparrow}^{\dagger} \right\}              & \left\{ c_{\vec{k}\uparrow}(x), c_{-\vec{k}\downarrow} \right\}
			\\[3mm]
			\left\{ c_{-\vec{k}\downarrow}^{\dagger}(x), c_{\vec{k}\uparrow}^{\dagger} \right\} & \left\{ c_{-\vec{k}\downarrow}^{\dagger}(x), c_{-\vec{k}\downarrow} \right\}
		\end{array}
	\right)
	\right\rangle
	e^{ i k \cdot x }
	\ = \
	i
	\int \! dx
	\left(
	\begin{array}{cc}
			G^{R}(x)       & F^{R}(x)
			\\[3mm]
			\bar{F}^{R}(x) & \bar{G}^{R}(x)
		\end{array}
	\right)
	e^{ i k \cdot x }
\end{eqnarray}
%

対角化された基底(準粒子) $a_{\vec{k} \sigma}^{(\dagger)}$
(これはスピン添字 $\sigma=\uparrow,\downarrow$ によってベクトルとみなせる)に対する運動方程式は以下で得られる:

\begin{eqnarray}
	i \dfrac{d a_{\vec{k}\uparrow}(t)}{dt}
	&=&
	\left[ a_{\vec{k}\uparrow} , H \right]
	\nonumber \\[2mm] &=&
	\sum_{\vec{q}}
	\left[ a_{\vec{k}\uparrow} , \vec{c}_{\vec{q}}^{\ \dagger} \hat{H} \vec{c}_{\vec{q}} \right]
	\ \ = \ \
	\sum_{\vec{q}}
	\left[ a_{\vec{k}\uparrow} \ \ , \ \ \left( c_{\vec{q} \uparrow}^{\dagger} \ \ , \ \ c_{-\vec{q} \downarrow} \right)
		\!\!\!
		\left(
		\begin{array}{cc}
				\xi_{\vec{q}} & \Delta
				\\[3mm]
				\Delta^{*}    & - \xi_{\vec{q}}
			\end{array}
		\right)
		\!\!\!
		\left(
		\!\!
		\begin{array}{c}
				c_{\vec{q} \uparrow}
				\\[3mm]
				c_{-\vec{q} \downarrow}^{\dagger}
			\end{array}
		\!\!
		\right)
		\right]
	\nonumber \\[2mm] &=&
	\sum_{\vec{q}}
	\left[ a_{\vec{k}\uparrow} , \vec{a}_{\vec{q}}^{\ \dagger} \hat{U}^{\dagger} \hat{H} \hat{U} \vec{a}_{\vec{q}} \right]
	\ \ = \ \
	\sum_{\vec{q}}
	\left[ a_{\vec{k}\uparrow} \ \ , \ \ \left( a_{\vec{q} \uparrow}^{\dagger} \ \ , \ \ a_{-\vec{q} \downarrow} \right)
		\!\!\!
		\left(
		\begin{array}{cc}
				E_{\vec{q}} & 0
				\\[3mm]
				0           & - E_{\vec{q}}
			\end{array}
		\right)
		\!\!\!
		\left(
		\!\!
		\begin{array}{c}
				a_{\vec{q} \uparrow}
				\\[3mm]
				a_{-\vec{q} \downarrow}^{\dagger}
			\end{array}
		\!\!
		\right)
		\right]
	\nonumber \\[2mm] &=&
	\sum_{\vec{q}}
	E_{\vec{q}}
	\left[
		a_{\vec{k}\uparrow}
		\ \ , \ \
		a_{\vec{q} \uparrow}^{\dagger} a_{\vec{q} \uparrow} - a_{-\vec{q} \downarrow} a_{-\vec{q} \downarrow}^{\dagger}
		\right]
	\nonumber \\[2mm] &=&
	\sum_{\vec{q}}
	E_{\vec{q}}
	\left(
	\left\{ a_{\vec{k}\uparrow} \ , \ a_{\vec{q} \uparrow}^{\dagger} \right\} a_{\vec{q} \uparrow}
	-
	a_{\vec{q} \uparrow}^{\dagger} \left\{ a_{\vec{k}\uparrow} \ , \ a_{\vec{q} \uparrow} \right\}
	-
	\left\{ a_{\vec{k}\uparrow} \ , \ a_{-\vec{q} \downarrow} \right\} a_{-\vec{q} \downarrow}^{\dagger}
	+
	a_{-\vec{q} \downarrow}  \left\{ a_{\vec{k}\uparrow} \ , \ a_{-\vec{q} \downarrow}^{\dagger} \right\}
	\right)
	\nonumber \\[2mm] &=&
	E_{\vec{k}}
	a_{\vec{k} \uparrow}
\end{eqnarray}
%
これを積分すると、

\begin{eqnarray}
	a_{\vec{k} \uparrow}(t)
	&=&
	e^{ - i E_{\vec{k}} t }
	a_{\vec{k} \uparrow}(0)
\end{eqnarray}
%
同様にして、$a_{-\vec{k} \downarrow}^{\dagger}$ に対する方程式を得る。

\begin{eqnarray}
	i \dfrac{d a_{-\vec{k} \downarrow}^{\dagger} }{dt}
	&=&
	\left[ a_{-\vec{k} \downarrow}^{\dagger} , H \right]
	\ = \
	-
	E_{\vec{k}}
	a_{-\vec{k} \downarrow}^{\dagger}
\end{eqnarray}
%
そして、

\begin{eqnarray}
	a_{-\vec{k} \downarrow}^{\dagger}(t)
	&=&
	e^{ i E_{\vec{k}} t }
	a_{-\vec{k} \downarrow}^{\dagger}(0)
\end{eqnarray}
%

スピノル $\vec{c}$ と $\vec{a}$ の間の関係は以下で与えられる。

\begin{eqnarray}
	c_{\vec{k} \uparrow}(t)
	&=&
	u_{\vec{k}} a_{\vec{k} \uparrow}(t)
	-
	v_{\vec{k}} a_{- \vec{k} \downarrow}^{\dagger}(t)
	\nonumber \\[2mm]
	&=&
	u_{\vec{k}} e^{ - i E_{\vec{k}} t }
	a_{\vec{k} \uparrow}
	-
	v_{\vec{k}} e^{ i E_{\vec{k}} t }
	a_{-\vec{k} \downarrow}^{\dagger}
	\\[5mm]
	c_{- \vec{k} \downarrow}^{\dagger}(t)
	&=&
	u_{\vec{k}} a_{- \vec{k} \downarrow}^{\dagger}(t)
	+
	v_{\vec{k}} a_{\vec{k} \uparrow}(t)
	\nonumber \\[2mm]
	&=&
	u_{\vec{k}} e^{ i E_{\vec{k}} t }
	a_{-\vec{k} \downarrow}^{\dagger}
	+
	v_{\vec{k}} e^{ - i E_{\vec{k}} t }
	a_{\vec{k} \uparrow}
\end{eqnarray}
%

(1,1) 成分の遅延グリーン関数は、以下のように明らかになる。

\begin{eqnarray}
	&&
	i
	G_{\vec{k}}^{R}(t)
	\ \ = \ \
	\theta(t)
	\left\langle \left\{
	c_{\vec{k} \uparrow}(t), c_{\vec{k} \uparrow}^{\dagger}
	\right\} \right\rangle
	\ \ = \ \
	\theta(t)
	\Big[
		\left\langle
		c_{\vec{k} \uparrow}(t) c_{\vec{k} \uparrow}^{\dagger}
		\right\rangle
		+
		\left\langle
		c_{\vec{k} \uparrow}^{\dagger} c_{\vec{k} \uparrow}(t)
		\right\rangle
		\Big]
	\nonumber \\[4mm] &=&
	\theta(t)
	\left\langle
	\left(
	u_{\vec{k}} e^{ - i E_{\vec{k}} t }
	a_{\vec{k} \uparrow}
	-
	v_{\vec{k}} e^{ i E_{\vec{k}} t }
	a_{-\vec{k} \downarrow}^{\dagger}
	\right)
	\!\!
	\left(
	u_{\vec{k}} a_{\vec{k} \uparrow}^{\dagger}
	-
	v_{\vec{k}} a_{- \vec{k} \downarrow}
	\right)
	\right\rangle
	\nonumber
	\\[2mm]
	&&
	\hspace{3mm}
	+
	\theta(t)
	\left\langle
	\left(
	u_{\vec{k}} a_{\vec{k} \uparrow}^{\dagger}
	-
	v_{\vec{k}} a_{- \vec{k} \downarrow}
	\right)
	\!\!
	\left(
	u_{\vec{k}} e^{ - i E_{\vec{k}} t }
	a_{\vec{k} \uparrow}
	-
	v_{\vec{k}} e^{ i E_{\vec{k}} t }
	a_{-\vec{k} \downarrow}^{\dagger}
	\right)
	\right\rangle
	\nonumber \\[4mm]
	&=&
	\theta(t)
	\left\langle
	u_{\vec{k}}^{2}
	e^{ - i E_{\vec{k}} t }
	a_{\vec{k} \uparrow}
	a_{\vec{k} \uparrow}^{\dagger}
	-
	u_{\vec{k}}
	v_{\vec{k}}
	e^{ - i E_{\vec{k}} t }
	a_{\vec{k} \uparrow}
	a_{-\vec{k} \downarrow}
	-
	u_{\vec{k}}
	v_{\vec{k}}
	e^{ i E_{\vec{k}} t }
	a_{-\vec{k} \downarrow}^{\dagger}
	a_{\vec{k} \uparrow}^{\dagger}
	+
	v_{\vec{k}}^{2}
	e^{ i E_{\vec{k}} t }
	a_{-\vec{k} \downarrow}^{\dagger}
	a_{-\vec{k} \downarrow}
	\right\rangle
	\nonumber
	\\[2mm]
	&&
	\hspace{3mm}
	+
	\theta(t)
	\left\langle
	u_{\vec{k}}^{2}
	e^{ - i E_{\vec{k}} t }
	a_{\vec{k} \uparrow}^{\dagger}
	a_{\vec{k} \uparrow}
	-
	u_{\vec{k}}
	v_{\vec{k}}
	e^{ i E_{\vec{k}} t }
	a_{\vec{k} \uparrow}^{\dagger}
	a_{-\vec{k} \downarrow}^{\dagger}
	-
	u_{\vec{k}}
	v_{\vec{k}}
	e^{ - i E_{\vec{k}} t }
	a_{-\vec{k} \downarrow}
	a_{\vec{k} \uparrow}
	+
	v_{\vec{k}}^{2}
	e^{ i E_{\vec{k}} t }
	a_{-\vec{k} \downarrow}
	a_{-\vec{k} \downarrow}^{\dagger}
	\right\rangle
	\nonumber \\[4mm]
	&=&
	\theta(t)
	\left(
	u_{\vec{k}}^{2}
	e^{ - i E_{\vec{k}} t }
	\left\langle
	a_{\vec{k} \uparrow}
	a_{\vec{k} \uparrow}^{\dagger}
	\right\rangle
	+
	v_{\vec{k}}^{2}
	e^{ i E_{\vec{k}} t }
	\left\langle
	a_{-\vec{k} \downarrow}^{\dagger}
	a_{-\vec{k} \downarrow}
	\right\rangle
	+
	u_{\vec{k}}^{2}
	e^{ - i E_{\vec{k}} t }
	\left\langle
	a_{\vec{k} \uparrow}^{\dagger}
	a_{\vec{k} \uparrow}
	\right\rangle
	+
	v_{\vec{k}}^{2}
	e^{ i E_{\vec{k}} t }
	\left\langle
	a_{-\vec{k} \downarrow}
	a_{-\vec{k} \downarrow}^{\dagger}
	\right\rangle
	\right)
	\nonumber \\[4mm]
	&=&
	\theta(t)
	\nonumber \\ && \times
	\left[
		u_{\vec{k}}^{2}
		e^{ - i E_{\vec{k}} t }
		\dfrac{1}{2}
		\Big( 1 + {\rm tanh}\dfrac{\beta E_{\vec{k}}}{2} \Big)
		+
		v_{\vec{k}}^{2}
		e^{ i E_{\vec{k}} t }
		\dfrac{1}{2}
		\Big( 1 - {\rm tanh}\dfrac{\beta E_{\vec{k}}}{2} \Big)
		\right.
		\nonumber \\ && +
		\left.
		u_{\vec{k}}^{2}
		e^{ - i E_{\vec{k}} t }
		\dfrac{1}{2}
		\Big( 1 - {\rm tanh}\dfrac{\beta E_{\vec{k}}}{2} \Big)
		+
		v_{\vec{k}}^{2}
		e^{ i E_{\vec{k}} t }
		\dfrac{1}{2}
		\Big( 1 + {\rm tanh}\dfrac{\beta E_{\vec{k}}}{2} \Big)
		\right]
	\nonumber \\[4mm]
	&=&
	\theta(t)
	\left(
	u_{\vec{k}}^{2}
	e^{ - i E_{\vec{k}} t }
	+
	v_{\vec{k}}^{2}
	e^{ i E_{\vec{k}} t }
	\right)
\end{eqnarray}
%
あるいは、フーリエ空間では、

\begin{eqnarray}
	&&
	G^{R}(k)
	\ = \
	\int \! dt \
	G^{R}_{\vec{k}}(t)
	e^{i \omega t}
	\nonumber \\ &=&
	-
	i
	\lim_{\eta \to +0}
	\int^{\infty}_{0} \!\! dt \
	\left(
	u_{\vec{k}}^{2}
	e^{ - i E_{\vec{k}} t }
	+
	v_{\vec{k}}^{2}
	e^{ i E_{\vec{k}} t }
	\right)
	e^{i \omega t - \eta t}
	\nonumber \\ &=&
	\lim_{\eta \to +0}
	\left(
	\dfrac{u_{\vec{k}}^{2}}{ \omega - E_{\vec{k}} + i \eta }
	+
	\dfrac{v_{\vec{k}}^{2}}{ \omega + E_{\vec{k}} + i \eta }
	\right)
	.
\end{eqnarray}
%
虚数部は状態密度に関連している。

\begin{eqnarray}
	&&
	- \dfrac{ 1 }{ \pi }
	\sum_{\vec{k}}
	{\rm Im} G^{R}(k)
	\nonumber \\ &=&
	-
	\dfrac{1}{\pi}
	\lim_{\eta \to +0}
	\sum_{\vec{k}}
	\left(
	{\rm Im}
	\dfrac{u_{\vec{k}}^{2}}{ \omega - E_{\vec{k}} + i \eta }
	+
	{\rm Im}
	\dfrac{v_{\vec{k}}^{2}}{ \omega + E_{\vec{k}} + i \eta }
	\right)
	\nonumber \\ &=&
	\sum_{\vec{k}}
	\left[
		u_{\vec{k}}^{2}
		\delta( \omega - E_{\vec{k}} )
		+
		v_{\vec{k}}^{2}
		\delta( \omega + E_{\vec{k}} )
		\right]
\end{eqnarray}
%
遅延部分の $(1,2)$ 成分 $F^{R}(k)$ も同様に与えられる。

\begin{eqnarray}
	&&
	i
	F_{\vec{k}}^{R}(t)
	\nonumber \\[4mm] &=&
	\theta(t)
	\left\langle \left\{
	c_{\vec{k} \uparrow}(t), c_{-\vec{k} \downarrow}
	\right\} \right\rangle
	\nonumber \\[4mm] &=&
	\theta(t)
	\Big[
		\left\langle
		c_{\vec{k} \uparrow}(t) c_{-\vec{k} \downarrow}
		\right\rangle
		+
		\left\langle
		c_{-\vec{k} \downarrow} c_{\vec{k} \uparrow}(t)
		\right\rangle
		\Big]
	\nonumber \\[4mm] &=&
	\theta(t)
	\left\langle
	\left(
	u_{\vec{k}} e^{ - i E_{\vec{k}} t }
	a_{\vec{k} \uparrow}
	-
	v_{\vec{k}} e^{ i E_{\vec{k}} t }
	a_{-\vec{k} \downarrow}^{\dagger}
	\right)
	\!\!
	\left(
	u_{\vec{k}} a_{-\vec{k} \downarrow}
	+
	v_{\vec{k}} a_{\vec{k} \uparrow}^{\dagger}
	\right)
	\right\rangle
	\nonumber \\ && \hspace{5mm} +
	\theta(t)
	\left\langle
	\left(
	u_{\vec{k}} a_{-\vec{k} \downarrow}
	+
	v_{\vec{k}} a_{\vec{k} \uparrow}^{\dagger}
	\right)
	\!\!
	\left(
	u_{\vec{k}} e^{ - i E_{\vec{k}} t }
	a_{\vec{k} \uparrow}
	-
	v_{\vec{k}} e^{ i E_{\vec{k}} t }
	a_{-\vec{k} \downarrow}^{\dagger}
	\right)
	\right\rangle
	\nonumber \\[4mm]
	&=&
	\theta(t)
	u_{\vec{k}}
	v_{\vec{k}}
	\left(
	e^{ - i E_{\vec{k}} t }
	\left\langle
	a_{\vec{k} \uparrow}
	a_{\vec{k} \uparrow}^{\dagger}
	\right\rangle
	-
	e^{ i E_{\vec{k}} t }
	\left\langle
	a_{-\vec{k} \downarrow}^{\dagger}
	a_{-\vec{k} \downarrow}
	\right\rangle
	+
	e^{ - i E_{\vec{k}} t }
	\left\langle
	a_{\vec{k} \uparrow}^{\dagger}
	a_{\vec{k} \uparrow}
	\right\rangle
	-
	e^{ i E_{\vec{k}} t }
	\left\langle
	a_{-\vec{k} \downarrow}
	a_{-\vec{k} \downarrow}^{\dagger}
	\right\rangle
	\right)
	\nonumber \\[4mm]
	&=&
	\theta(t)
	u_{\vec{k}}
	v_{\vec{k}}
	\left[
		e^{ - i E_{\vec{k}} t }
		\dfrac{1}{2}
		\Big( 1 + {\rm tanh}\dfrac{\beta E_{\vec{k}}}{2} \Big)
		-
		e^{ i E_{\vec{k}} t }
		\dfrac{1}{2}
		\Big( 1 - {\rm tanh}\dfrac{\beta E_{\vec{k}}}{2} \Big)
		\right.
		\nonumber \\ && \hspace{5mm} +
		\left.
		e^{ - i E_{\vec{k}} t }
		\dfrac{1}{2}
		\Big( 1 - {\rm tanh}\dfrac{\beta E_{\vec{k}}}{2} \Big)
		-
		e^{ i E_{\vec{k}} t }
		\dfrac{1}{2}
		\Big( 1 + {\rm tanh}\dfrac{\beta E_{\vec{k}}}{2} \Big)
		\right]
	\nonumber \\[4mm] &=&
	\theta(t)
	u_{\vec{k}}
	v_{\vec{k}}
	\left(
	e^{ - i E_{\vec{k}} t }
	-
	e^{ i E_{\vec{k}} t }
	\right)
	,
\end{eqnarray}

\begin{eqnarray}
	&&
	F^{R}(k)
	\ \ = \ \
	- i
	u_{\vec{k}}
	v_{\vec{k}}
	\lim_{\eta \to +0}
	\int^{\infty}_{0} \!\! dt \
	\left(
	e^{ i E_{\vec{k}} t }
	-
	e^{ - i E_{\vec{k}} t }
	\right)
	e^{i \omega t - \eta t}
	\nonumber \\ &=&
	-
	u_{\vec{k}} v_{\vec{k}}
	\lim_{\eta \to +0}
	\left(
	\dfrac{1}{\omega - E_{\vec{k}} + i \eta }
	-
	\dfrac{1}{\omega + E_{\vec{k}} + i \eta }
	\right)
	,
\end{eqnarray}

異常グリーン関数に対応する状態密度は

\begin{eqnarray}
	&&
	-
	\dfrac{ 1 }{ \pi }
	\sum_{\vec{k}}
	{\rm Im} F^{R}(k)
	\nonumber \\ &=&
	\dfrac{ 1 }{ \pi }
	\lim_{\eta \to +0}
	\sum_{\vec{k}}
	u_{\vec{k}} v_{\vec{k}}
	\left(
	{\rm Im}
	\dfrac{1}{\omega - E_{\vec{k}} + i \eta }
	-
	{\rm Im}
	\dfrac{1}{\omega + E_{\vec{k}} + i \eta }
	\right)
	\nonumber \\ &=&
	\sum_{\vec{k}}
	u_{\vec{k}} v_{\vec{k}}
	\big[
		\delta( \omega + E_{\vec{k}} )
		-
		\delta( \omega - E_{\vec{k}} )
		\big]
	.
	\nonumber \\
\end{eqnarray}


遅延部分の $(2,1)$ 成分 $\bar{F}^{R}(k)$ は、遅延部分の $(1,2)$ 成分 $F^{R}(k)$ に等しい。

\begin{eqnarray}
	&&
	i
	\bar{F}_{\vec{k}}^{R}(t)
	\ \ = \ \
	\theta(t)
	\left\langle \left\{
	c_{-\vec{k} \downarrow}^{\dagger}(t), c_{\vec{k} \uparrow}^{\dagger}
	\right\} \right\rangle
	\ \ = \ \
	\theta(t)
	\Big[
		\left\langle
		c_{-\vec{k} \downarrow}^{\dagger}(t) c_{\vec{k} \uparrow}^{\dagger}
		\right\rangle
		+
		\left\langle
		c_{\vec{k} \uparrow}^{\dagger} c_{-\vec{k} \downarrow}^{\dagger}(t)
		\right\rangle
		\Big]
	\nonumber \\[4mm] &=&
	\theta(t)
	\left\langle
	\left(
	u_{\vec{k}} e^{ i E_{\vec{k}} t }
	a_{-\vec{k} \downarrow}^{\dagger}
	+
	v_{\vec{k}} e^{ - i E_{\vec{k}} t }
	a_{\vec{k} \uparrow}
	\right)
	\!\!
	\left(
	u_{\vec{k}} a_{\vec{k} \uparrow}^{\dagger}
	-
	v_{\vec{k}} a_{-\vec{k} \downarrow}
	\right)
	\right\rangle
	\nonumber \\ && \hspace{5mm} +
	\theta(t)
	\left\langle
	\left(
	u_{\vec{k}} a_{\vec{k} \uparrow}^{\dagger}
	-
	v_{\vec{k}} a_{-\vec{k} \downarrow}
	\right)
	\!\!
	\left(
	u_{\vec{k}} e^{ i E_{\vec{k}} t }
	a_{-\vec{k} \downarrow}^{\dagger}
	+
	v_{\vec{k}} e^{ - i E_{\vec{k}} t }
	a_{\vec{k} \uparrow}
	\right)
	\right\rangle
	\nonumber \\[4mm]
	&=&
	\theta(t)
	u_{\vec{k}}
	v_{\vec{k}}
	\left(
	-
	e^{ i E_{\vec{k}} t }
	\left\langle
	a_{-\vec{k} \downarrow}^{\dagger}
	a_{-\vec{k} \downarrow}
	\right\rangle
	+
	e^{ - i E_{\vec{k}} t }
	\left\langle
	a_{\vec{k} \uparrow}
	a_{\vec{k} \uparrow}^{\dagger}
	\right\rangle
	+
	e^{ - i E_{\vec{k}} t }
	\left\langle
	a_{\vec{k} \uparrow}^{\dagger}
	a_{\vec{k} \uparrow}
	\right\rangle
	-
	e^{ i E_{\vec{k}} t }
	\left\langle
	a_{-\vec{k} \downarrow}
	a_{-\vec{k} \downarrow}^{\dagger}
	\right\rangle
	\right)
	\nonumber \\[4mm]
	&=&
	\theta(t)
	u_{\vec{k}}
	v_{\vec{k}}
	\left[
		-
		e^{ i E_{\vec{k}} t }
		\dfrac{1}{2}
		\Big( 1 - {\rm tanh}\dfrac{\beta E_{\vec{k}}}{2} \Big)
		+
		e^{ - i E_{\vec{k}} t }
		\dfrac{1}{2}
		\Big( 1 + {\rm tanh}\dfrac{\beta E_{\vec{k}}}{2} \Big)
		\right.
		\nonumber \\ && \hspace{5mm} +
		\left.
		e^{ - i E_{\vec{k}} t }
		\dfrac{1}{2}
		\Big( 1 - {\rm tanh}\dfrac{\beta E_{\vec{k}}}{2} \Big)
		-
		e^{ i E_{\vec{k}} t }
		\dfrac{1}{2}
		\Big( 1 + {\rm tanh}\dfrac{\beta E_{\vec{k}}}{2} \Big)
		\right]
	\nonumber \\[4mm] &=&
	\theta(t)
	u_{\vec{k}}
	v_{\vec{k}}
	\left(
	-
	e^{ i E_{\vec{k}} t }
	+
	e^{ - i E_{\vec{k}} t }
	\right)
	\nonumber \\[4mm] &=&
	i
	F_{\vec{k}}^{R}(t)
\end{eqnarray}


遅延部分の $(2,2)$ 成分 $\bar{G}^{R}(k)$ もまた $G^{R}(k)$ に等しい。

\begin{eqnarray}
	&&
	i
	\bar{G}_{\vec{k}}^{R}(t)
	\ \ = \ \
	\theta(t)
	\left\langle \left\{
	c_{-\vec{k} \downarrow}^{\dagger}(t), c_{-\vec{k} \downarrow}
	\right\} \right\rangle
	\ \ = \ \
	\theta(t)
	\Big[
		\left\langle
		c_{-\vec{k} \downarrow}^{\dagger}(t) c_{-\vec{k} \downarrow}
		\right\rangle
		+
		\left\langle
		c_{-\vec{k} \downarrow} c_{-\vec{k} \downarrow}^{\dagger}(t)
		\right\rangle
		\Big]
	\nonumber \\[4mm] &=&
	\theta(t)
	\left\langle
	\left(
	u_{\vec{k}} e^{ i E_{\vec{k}} t }
	a_{-\vec{k} \downarrow}^{\dagger}
	+
	v_{\vec{k}} e^{ - i E_{\vec{k}} t }
	a_{\vec{k} \uparrow}
	\right)
	\!\!
	\left(
	u_{\vec{k}}
	a_{-\vec{k} \downarrow}
	+
	v_{\vec{k}}
	a_{\vec{k} \uparrow}^{\dagger}
	\right)
	\right\rangle
	\nonumber \\ && \hspace{5mm} +
	\theta(t)
	\left\langle
	\left(
	u_{\vec{k}}
	a_{-\vec{k} \downarrow}
	+
	v_{\vec{k}}
	a_{\vec{k} \uparrow}^{\dagger}
	\right)
	\!\!
	\left(
	u_{\vec{k}} e^{ i E_{\vec{k}} t }
	a_{-\vec{k} \downarrow}^{\dagger}
	+
	v_{\vec{k}} e^{ - i E_{\vec{k}} t }
	a_{\vec{k} \uparrow}
	\right)
	\right\rangle
	\nonumber \\[4mm]
	&=&
	\theta(t)
	\left(
	u_{\vec{k}}^{2}
	e^{ i E_{\vec{k}} t }
	\left\langle
	a_{-\vec{k} \downarrow}^{\dagger}
	a_{-\vec{k} \downarrow}
	\right\rangle
	+
	v_{\vec{k}}^{2}
	e^{ - i E_{\vec{k}} t }
	\left\langle
	a_{\vec{k} \uparrow}
	a_{\vec{k} \uparrow}^{\dagger}
	\right\rangle
	\right.
	\nonumber \\ && \hspace{5mm} +
	\left.
	u_{\vec{k}}^{2}
	e^{ i E_{\vec{k}} t }
	\left\langle
	a_{-\vec{k} \downarrow}
	a_{-\vec{k} \downarrow}^{\dagger}
	\right\rangle
	+
	v_{\vec{k}}^{2}
	e^{ - i E_{\vec{k}} t }
	\left\langle
	a_{\vec{k} \uparrow}^{\dagger}
	a_{\vec{k} \uparrow}
	\right\rangle
	\right)
	\nonumber \\[4mm]
	&=&
	\theta(t)
	\left[
		u_{\vec{k}}^{2}
		e^{ i E_{\vec{k}} t }
		\dfrac{1}{2}
		\Big( 1 - {\rm tanh}\dfrac{\beta E_{\vec{k}}}{2} \Big)
		+
		v_{\vec{k}}^{2}
		e^{ - i E_{\vec{k}} t }
		\dfrac{1}{2}
		\Big( 1 + {\rm tanh}\dfrac{\beta E_{\vec{k}}}{2} \Big)
		\right.
		\nonumber \\ && \hspace{5mm} +
		\left.
		u_{\vec{k}}^{2}
		e^{ i E_{\vec{k}} t }
		\dfrac{1}{2}
		\Big( 1 + {\rm tanh}\dfrac{\beta E_{\vec{k}}}{2} \Big)
		+
		v_{\vec{k}}^{2}
		e^{ - i E_{\vec{k}} t }
		\dfrac{1}{2}
		\Big( 1 - {\rm tanh}\dfrac{\beta E_{\vec{k}}}{2} \Big)
		\right]
	\nonumber \\[4mm] &=&
	\theta(t)
	\left(
	u_{\vec{k}}^{2}
	e^{ i E_{\vec{k}} t }
	+
	v_{\vec{k}}^{2}
	e^{ - i E_{\vec{k}} t }
	\right)
	\nonumber \\[4mm] &=&
	i
	G_{\vec{k}}^{R}(t)
\end{eqnarray}


最終的に、$\hat{G}^{R}$ を既知の値で表す式にたどり着く。

\begin{eqnarray}
	i
	\left(
	\begin{array}{cc}
			G^{R}_{\vec{k}}(t)       & F^{R}_{\vec{k}}(t)
			\\[3mm]
			\bar{F}^{R}_{\vec{k}}(t) & \bar{G}^{R}_{\vec{k}}(t)
		\end{array}
	\right)
	&=&
	\theta(t)
	\left(
	\begin{array}{cc}
			u_{\vec{k}}^{2}
			e^{ i E_{\vec{k}} t }
			+
			v_{\vec{k}}^{2}
			e^{ - i E_{\vec{k}} t }
			 &
			-
			u_{\vec{k}}
			v_{\vec{k}}
			\left(
			e^{ i E_{\vec{k}} t }
			-
			e^{ - i E_{\vec{k}} t }
			\right)
			\\[3mm]
			-
			u_{\vec{k}}
			v_{\vec{k}}
			\left(
			e^{ i E_{\vec{k}} t }
			-
			e^{ - i E_{\vec{k}} t }
			\right)
			 &
			u_{\vec{k}}^{2}
			e^{ i E_{\vec{k}} t }
			+
			v_{\vec{k}}^{2}
			e^{ - i E_{\vec{k}} t }
		\end{array}
	\right)
	\\[4mm]
	\left(
	\begin{array}{cc}
			G^{R}(k)       & F^{R}(k)
			\\[3mm]
			\bar{F}^{R}(k) & \bar{G}^{R}(k)
		\end{array}
	\right)
	&=&
	\lim_{\eta \to +0}
	\left(
	\begin{array}{cc}
			\dfrac{u_{\vec{k}}^{2}}{ \omega - E_{\vec{k}} + i \eta }
			+
			\dfrac{v_{\vec{k}}^{2}}{ \omega + E_{\vec{k}} + i \eta }
			 &
			-
			u_{\vec{k}} v_{\vec{k}}
			\left(
			\dfrac{1}{\omega - E_{\vec{k}} + i \eta }
			-
			\dfrac{1}{\omega + E_{\vec{k}} + i \eta }
			\right)
			\\[3mm]
			-
			u_{\vec{k}} v_{\vec{k}}
			\left(
			\dfrac{1}{\omega - E_{\vec{k}} + i \eta }
			-
			\dfrac{1}{\omega + E_{\vec{k}} + i \eta }
			\right)
			 &
			\dfrac{u_{\vec{k}}^{2}}{ \omega - E_{\vec{k}} + i \eta }
			+
			\dfrac{v_{\vec{k}}^{2}}{ \omega + E_{\vec{k}} + i \eta }
		\end{array}
	\right)
\end{eqnarray}
%


\subsection{一般化されたスピンの場合}

南部空間におけるグリーン関数($4 \times 4$ 行列)は、次のように定義される。

\begin{eqnarray}
	i
	\hat{G}(k)
	&=&
	\int \! dx
	\left\langle \hat{\rm T} \left[ \vec{c}_{\vec{k}}(x) \vec{c}_{\vec{k}}^{\ \dagger} \right] \right\rangle
	e^{ i k \cdot x }
	\nonumber \\ &=&
	\int \! dx
	\left\langle \hat{\rm T} \left[
		\left(
		\begin{array}{c}
				c_{\vec{k} \uparrow}(x)            \\[2mm]
				c_{\vec{k} \downarrow}(x)          \\[2mm]
				c^{\dagger}_{-\vec{k} \uparrow}(x) \\[2mm]
				c^{\dagger}_{-\vec{k} \downarrow}(x)
			\end{array}
		\right)
		\big( c^{\dagger}_{\vec{k} \uparrow} \ , \ c^{\dagger}_{\vec{k} \downarrow} \ , \ c_{-\vec{k} \uparrow} \ , \ c_{-\vec{k} \downarrow}  \big)
		\right]
	\right\rangle
	e^{ i k \cdot x }
\end{eqnarray}
%
ここで $\hat{\rm T} \left[ \cdots \right]$ は時間順序積演算子である。
$x^{\mu}=(t,\vec{r})$ および $k^{\mu}=(\omega,\vec{k})$
( i.e., $k \cdot x = g_{\mu \nu} k^{\mu} x^{\nu} = \omega t- \vec{k} \cdot \vec{r}$ )
は、簡略記法で書かれた4元運動量ベクトルである。
これらの関数の遅延(retarded)部分も、次のように定義される。

\begin{eqnarray}
	&&
	i
	\hat{G}^{R}(k)
	\nonumber \\ &=&
	\int \! dx
	\left\langle
	\left(
	\begin{array}{cccc}
			\{ c_{\vec{k}\uparrow}(x), c_{\vec{k}\uparrow}^{\dagger} \}                & \{ c_{\vec{k}\uparrow}(x), c^{\dagger}_{\vec{k} \downarrow} \}               & \{ c_{\vec{k} \uparrow}(x), c_{-\vec{k} \uparrow} \}              & \{ c_{\vec{k} \uparrow}(x), c_{-\vec{k} \downarrow} \}
			\\[3mm]
			\{ c_{\vec{k} \downarrow}(x), c_{\vec{k}\uparrow}^{\dagger} \}             & \{ c_{\vec{k} \downarrow}(x), c^{\dagger}_{\vec{k} \downarrow} \}            & \{ c_{\vec{k} \downarrow}(x), c_{-\vec{k} \uparrow} \}            & \{ c_{\vec{k} \downarrow}(x), c_{-\vec{k} \downarrow} \}
			\\[3mm]
			\{ c^{\dagger}_{-\vec{k} \uparrow}(x), c^{\dagger}_{\vec{k} \uparrow} \}   & \{ c^{\dagger}_{-\vec{k} \uparrow}(x), c^{\dagger}_{\vec{k} \downarrow} \}   & \{ c^{\dagger}_{-\vec{k} \uparrow}(x), c_{-\vec{k} \uparrow} \}   & \{ c^{\dagger}_{-\vec{k} \uparrow}(x), c_{-\vec{k} \downarrow} \}
			\\[3mm]
			\{ c^{\dagger}_{-\vec{k} \downarrow}(x), c^{\dagger}_{\vec{k} \uparrow} \} & \{ c^{\dagger}_{-\vec{k} \downarrow}(x), c^{\dagger}_{\vec{k} \downarrow} \} & \{ c^{\dagger}_{-\vec{k} \downarrow}(x), c_{-\vec{k} \uparrow} \} & \{ c^{\dagger}_{-\vec{k} \downarrow}(x), c_{-\vec{k} \downarrow} \}
		\end{array}
	\right)
	\right\rangle
	e^{ i k \cdot x }
\end{eqnarray}
%

対角化された基底(準粒子) $a_{\vec{k} \sigma}^{(\dagger)}$
に対する運動方程式は以下で得られる。:

\begin{eqnarray}
	i \dfrac{d a_{\vec{k}\uparrow}(t)}{dt}
	&=&
	\left[ a_{\vec{k}\uparrow} , H \right]
	\nonumber \\[2mm] &=&
	\dfrac{1}{2}
	\sum_{\vec{q}}
	\left[ a_{\vec{k}\uparrow} , \vec{c}_{\vec{q}}^{\ \dagger} \hat{H} \vec{c}_{\vec{q}} \right]
	\nonumber \\[2mm] &=&
	\dfrac{1}{2}
	\sum_{\vec{q}}
	\left[ a_{\vec{k}\uparrow}
		\ \ , \ \
		\big( c^{\dagger}_{\vec{q} \uparrow} \ , \ c^{\dagger}_{\vec{q} \downarrow} \ , \ c_{-\vec{q} \uparrow} \ , \ c_{-\vec{q} \downarrow}  \big)
		\!\!\!
		\left(
		\begin{array}{cc}
				\xi_{\vec{q}} \hat{1}_{2 \times 2} & \hat{\Delta}_{\vec{q}}               \\[3mm]
				\hat{\Delta}^{*}_{\vec{q}}         & - \xi_{\vec{q}} \hat{1}_{2 \times 2}
			\end{array}
		\right)
		\!\!\!
		\left(
		\begin{array}{c}
				c_{\vec{q} \uparrow}            \\[2mm]
				c_{\vec{q} \downarrow}          \\[2mm]
				c^{\dagger}_{-\vec{q} \uparrow} \\[2mm]
				c^{\dagger}_{-\vec{q} \downarrow}
			\end{array}
		\right)
		\right]
	\nonumber \\[2mm] &=&
	\dfrac{1}{2}
	\sum_{\vec{q}}
	\left[ a_{\vec{k}\uparrow} , \vec{a}_{\vec{q}}^{\ \dagger} \hat{U}^{\dagger} \hat{H} \hat{U} \vec{a}_{\vec{q}} \right]
	\nonumber \\[2mm] &=&
	\dfrac{1}{2}
	\sum_{\vec{q}}
	\left[ a_{\vec{k}\uparrow}
		\ \ , \ \
		\big( a^{\dagger}_{\vec{q} \uparrow} \ , \ a^{\dagger}_{\vec{q} \downarrow} \ , \ a_{-\vec{q} \uparrow} \ , \ a_{-\vec{q} \downarrow}  \big)
		\!\!\!
		\left(
		\begin{array}{cc}
				E_{\vec{q}} \hat{1}_{2 \times 2} &
				\\[3mm]
				                                 & - E_{\vec{q}} \hat{1}_{2 \times 2}
			\end{array}
		\right)
		\!\!\!
		\left(
		\begin{array}{c}
				a_{\vec{q} \uparrow}            \\[2mm]
				a_{\vec{q} \downarrow}          \\[2mm]
				a^{\dagger}_{-\vec{q} \uparrow} \\[2mm]
				a^{\dagger}_{-\vec{q} \downarrow}
			\end{array}
		\right)
		\right]
	\nonumber \\[2mm] &=&
	\dfrac{1}{2}
	\sum_{\vec{q}}
	E_{\vec{q}}
	\left[
		a_{\vec{k}\uparrow}
		\ \ , \ \
		a_{\vec{q}\uparrow}^{\dagger} a_{\vec{q}\uparrow}
		+
		a_{\vec{q}\downarrow}^{\dagger} a_{\vec{q}\downarrow}
		-
		a_{-\vec{q}\uparrow} a_{-\vec{q}\uparrow}^{\dagger}
		-
		a_{-\vec{q}\downarrow} a_{-\vec{q}\downarrow}^{\dagger}
		\right]
	\nonumber \\[2mm] &=&
	\dfrac{1}{2}
	\sum_{\vec{q}}
	E_{\vec{q}}
	\left(
	\left\{ a_{\vec{k}\uparrow} \ , \ a_{\vec{q} \uparrow}^{\dagger} \right\} a_{\vec{q} \uparrow}
	-
	a_{\vec{q} \uparrow}^{\dagger} \left\{ a_{\vec{k}\uparrow} \ , \ a_{\vec{q} \uparrow} \right\}
	+
	\left\{ a_{\vec{k}\uparrow} \ , \ a_{\vec{q} \downarrow}^{\dagger} \right\} a_{\vec{q} \downarrow}
	-
	a_{\vec{q} \downarrow}^{\dagger} \left\{ a_{\vec{k}\uparrow} \ , \ a_{\vec{q} \downarrow} \right\}
	\right.
	\nonumber \\[2mm] && \hspace{15mm}
	\left.
	-
	\left\{ a_{\vec{k}\uparrow} \ , \ a_{-\vec{q} \uparrow} \right\} a_{-\vec{q} \uparrow}^{\dagger}
	+
	a_{-\vec{q} \uparrow} \left\{ a_{\vec{k}\uparrow} \ , \ a_{-\vec{q} \uparrow}^{\dagger} \right\}
	-
	\left\{ a_{\vec{k}\uparrow} \ , \ a_{-\vec{q} \downarrow} \right\} a_{-\vec{q} \downarrow}^{\dagger}
	+
	a_{-\vec{q} \downarrow} \left\{ a_{\vec{k}\uparrow} \ , \ a_{-\vec{q} \downarrow}^{\dagger} \right\}
	\right)
	\nonumber \\[2mm] &=&
	E_{\vec{k}}
	a_{\vec{k} \uparrow}
\end{eqnarray}
%
これを積分すると、

\begin{eqnarray}
	a_{\vec{k} \uparrow}(t)
	&=&
	e^{ - i E_{\vec{k}} t }
	a_{\vec{k} \uparrow}(0)
\end{eqnarray}
%
同様にして、$a_{-\vec{k} \downarrow}^{\dagger}$ に対する方程式を得る。

\begin{eqnarray}
	i \dfrac{d a_{-\vec{k} \downarrow}^{\dagger}(t) }{dt}
	&=&
	\left[ a_{-\vec{k} \downarrow}^{\dagger} , H \right]
	\ = \
	-
	E_{\vec{k}}
	a_{-\vec{k} \downarrow}^{\dagger}
\end{eqnarray}
%
そして、

\begin{eqnarray}
	a_{-\vec{k} \downarrow}^{\dagger}(t)
	&=&
	e^{ i E_{\vec{k}} t }
	a_{-\vec{k} \downarrow}^{\dagger}(0)
	.
\end{eqnarray}
%
スピノル $\vec{c}$ と $\vec{a}$ の間の関係は以下で与えられる。

\begin{eqnarray}
	\vec{c}_{\vec{k}}
	&=&
	\left(
	\begin{array}{cc}
			\hat{u}_{\vec{k}}       & - \hat{v}_{\vec{k}} \\[3mm]
			\hat{v}_{- \vec{k}}^{*} & \hat{u}_{- \vec{k}}
		\end{array}
	\right)
	\!\!\!
	\left(
	\begin{array}{c}
			a_{\vec{k} \uparrow}            \\[2mm]
			a_{\vec{k} \downarrow}          \\[2mm]
			a^{\dagger}_{-\vec{k} \uparrow} \\[2mm]
			a^{\dagger}_{-\vec{k} \downarrow}
		\end{array}
	\right)
	\ \ = \ \
	\hat{U}
	\vec{a}_{\vec{k}}
	,
\end{eqnarray}
%
ここで、

\begin{eqnarray}
	\left(
	\begin{array}{cc}
			\hat{u}_{\vec{k}}       & - \hat{v}_{\vec{k}} \\[2mm]
			\hat{v}_{- \vec{k}}^{*} & \hat{u}_{-\vec{k}}
		\end{array}
	\right)
	&=&
	\dfrac{ 1 }{ \sqrt{ \big( E_{\vec{k}} + \xi_{\vec{k}} \big)^{2} + \dfrac{1}{2} {\rm Tr} \big[ \hat{\Delta}_{\vec{k}} \hat{\Delta}^{\dagger}_{\vec{k}} \big] } }
	\left(
	\begin{array}{cc}
			( E_{\vec{k}} + \xi_{\vec{k}} ) \hat{1}_{2 \times 2} & - \hat{\Delta}_{\vec{k}}                             \\[2mm]
			\hat{\Delta}_{\vec{k}}^{\dagger}                     & ( E_{\vec{k}} + \xi_{\vec{k}} ) \hat{1}_{2 \times 2}
		\end{array}
	\right)
	.
\end{eqnarray}
%
したがって、次のように置き換えることができる。

\begin{eqnarray}
	( \hat{u}_{\vec{k}} )_{\sigma \sigma'}
	\ = \
	( \hat{u}_{- \vec{k}} )_{\sigma \sigma'}
	\ = \
	u_{\vec{k}} \delta_{\sigma \sigma'}
	,
	\\[2mm]
	( \hat{v}_{\vec{k}} )_{\sigma \sigma'}
	\ = \
	v_{\vec{k} \sigma \sigma'}
	\ \ \ , \ \ \
	( \hat{v}_{\vec{k}}^{*} )_{\sigma \sigma'}
	\ = \
	v_{\vec{k} \sigma \sigma'}^{*}
	.
\end{eqnarray}
%
これで、スカラー量 $u_{\vec{k}}$, $v_{\vec{k} \sigma \sigma'}$ および $v_{\vec{k} \sigma \sigma'}^{*}$ が定義された。
この関係を用いて、グリーン関数についての表現を得ることができる。

\begin{eqnarray}
	c_{\vec{k} \sigma}(t)
	&=&
	(\hat{u}_{\vec{k}})_{\sigma \uparrow} a_{\vec{k} \uparrow}(t)
	+
	(\hat{u}_{\vec{k}})_{\sigma \downarrow} a_{\vec{k} \downarrow}(t)
	-
	(\hat{v}_{\vec{k}})_{\sigma \uparrow} a_{-\vec{k} \uparrow}^{\dagger}(t)
	-
	(\hat{v}_{\vec{k}})_{\sigma \downarrow} a_{-\vec{k} \downarrow}^{\dagger}(t)
	\nonumber \\[2mm] &=&
	u_{\vec{k}} \delta_{\sigma \uparrow} a_{\vec{k} \uparrow} e^{- i E_{\vec{k}} t}
	+
	u_{\vec{k}} \delta_{\sigma \downarrow} a_{\vec{k} \downarrow} e^{- i E_{\vec{k}} t}
	-
	v_{\vec{k} \sigma \uparrow} a_{-\vec{k} \uparrow}^{\dagger} e^{ i E_{\vec{k}} t}
	-
	v_{\vec{k} \sigma \downarrow} a_{-\vec{k} \downarrow}^{\dagger} e^{ i E_{\vec{k}} t}
	,
	\\[4mm]
	c_{-\vec{k} \sigma}^{\dagger}(t)
	&=&
	(\hat{v}_{\vec{k}}^{*})_{\sigma \uparrow} a_{\vec{k} \uparrow}(t)
	+
	(\hat{v}_{\vec{k}}^{*})_{\sigma \downarrow} a_{\vec{k} \downarrow}(t)
	+
	(\hat{u}_{\vec{k}})_{\sigma \uparrow} a_{-\vec{k} \uparrow}^{\dagger}(t)
	+
	(\hat{u}_{\vec{k}})_{\sigma \downarrow} a_{-\vec{k} \downarrow}^{\dagger}(t)
	\nonumber \\[2mm] &=&
	v_{\vec{k} \sigma \uparrow}^{*} a_{\vec{k} \uparrow} e^{- i E_{\vec{k}} t}
	+
	v_{\vec{k} \sigma \downarrow}^{*} a_{\vec{k} \downarrow} e^{- i E_{\vec{k}} t}
	+
	u_{\vec{k}} \delta_{\sigma \uparrow} a_{-\vec{k} \uparrow}^{\dagger} e^{ i E_{\vec{k}} t}
	+
	u_{\vec{k}} \delta_{\sigma \downarrow} a_{-\vec{k} \downarrow}^{\dagger} e^{ i E_{\vec{k}} t}
	.
\end{eqnarray}
%例えば、(1,1) 成分の遅延グリーン関数は次のように書ける。

\begin{eqnarray}
	&&
	i
	G_{\vec{k} \uparrow \uparrow}^{R}(t)
	\ \ = \ \
	\theta(t)
	\left\langle \left\{
	c_{\vec{k} \uparrow}(t), c_{\vec{k} \uparrow}^{\dagger}
	\right\} \right\rangle
	\ \ = \ \
	\theta(t)
	\left\langle
	c_{\vec{k} \uparrow}(t) c_{\vec{k} \uparrow}^{\dagger}
	+
	c_{\vec{k} \uparrow}^{\dagger} c_{\vec{k} \uparrow}(t)
	\right\rangle
	\nonumber \\[4mm] &=&
	\theta(t)
	\left\langle
	\left(
	u_{\vec{k}} a_{\vec{k} \uparrow} e^{- i E_{\vec{k}} t}
	-
	v_{\vec{k} \uparrow \uparrow} a_{-\vec{k} \uparrow}^{\dagger} e^{ i E_{\vec{k}} t}
	-
	v_{\vec{k} \uparrow \downarrow} a_{-\vec{k} \downarrow}^{\dagger} e^{ i E_{\vec{k}} t}
	\right)
	\!\!
	\left(
	u_{\vec{k}} a_{\vec{k} \uparrow}^{\dagger}
	-
	v_{\vec{k} \uparrow \uparrow}^{*} a_{-\vec{k} \uparrow}
	-
	v_{\vec{k} \uparrow \downarrow}^{*} a_{-\vec{k} \downarrow}
	\right)
	\right.
	\nonumber \\[2mm] && \hspace{12mm} +
	\left.
	\left(
	u_{\vec{k}} a_{\vec{k} \uparrow}^{\dagger}
	-
	v_{\vec{k} \uparrow \uparrow}^{*} a_{-\vec{k} \uparrow}
	-
	v_{\vec{k} \uparrow \downarrow}^{*} a_{-\vec{k} \downarrow}
	\right)
	\!\!
	\left(
	u_{\vec{k}} a_{\vec{k} \uparrow} e^{- i E_{\vec{k}} t}
	-
	v_{\vec{k} \uparrow \uparrow} a_{-\vec{k} \uparrow}^{\dagger} e^{ i E_{\vec{k}} t}
	-
	v_{\vec{k} \uparrow \downarrow} a_{-\vec{k} \downarrow}^{\dagger} e^{ i E_{\vec{k}} t}
	\right)
	\right\rangle
	\nonumber \\[4mm]
	&=&
	\theta(t)
	\left(
	u_{\vec{k}}^{2}
	e^{ - i E_{\vec{k}} t }
	\left\langle
	a_{\vec{k} \uparrow}
	a_{\vec{k} \uparrow}^{\dagger}
	\right\rangle
	+
	|v_{\vec{k} \uparrow \uparrow}|^{2}
	e^{ i E_{\vec{k}} t }
	\left\langle
	a_{-\vec{k} \uparrow}^{\dagger}
	a_{-\vec{k} \uparrow}
	\right\rangle
	+
	|v_{\vec{k} \uparrow \downarrow}|^{2}
	e^{ i E_{\vec{k}} t }
	\left\langle
	a_{-\vec{k} \downarrow}^{\dagger}
	a_{-\vec{k} \downarrow}
	\right\rangle
	\right.
	\nonumber \\[1mm] && \hspace{12mm} +
	\left.
	u_{\vec{k}}^{2}
	e^{ - i E_{\vec{k}} t }
	\left\langle
	a_{\vec{k} \uparrow}^{\dagger}
	a_{\vec{k} \uparrow}
	\right\rangle
	+
	|v_{\vec{k} \uparrow \uparrow}|^{2}
	e^{ i E_{\vec{k}} t }
	\left\langle
	a_{-\vec{k} \uparrow}
	a_{-\vec{k} \uparrow}^{\dagger}
	\right\rangle
	+
	|v_{\vec{k} \uparrow \downarrow}|^{2}
	e^{ i E_{\vec{k}} t }
	\left\langle
	a_{-\vec{k} \downarrow}
	a_{-\vec{k} \downarrow}^{\dagger}
	\right\rangle
	\right)
	\nonumber \\[4mm]
	&=&
	\theta(t)
	\left[
		u_{\vec{k}}^{2}
		e^{ - i E_{\vec{k}} t }
		\dfrac{1}{2}
		\Big( 1 + {\rm tanh}\dfrac{\beta E_{\vec{k}}}{2} \Big)
		+
		|v_{\vec{k} \uparrow \uparrow}|^{2}
		e^{ i E_{\vec{k}} t }
		\dfrac{1}{2}
		\Big( 1 - {\rm tanh}\dfrac{\beta E_{\vec{k}}}{2} \Big)
		+
		|v_{\vec{k} \uparrow \downarrow}|^{2}
		e^{ i E_{\vec{k}} t }
		\dfrac{1}{2}
		\Big( 1 - {\rm tanh}\dfrac{\beta E_{\vec{k}}}{2} \Big)
		\right.
		\nonumber \\[1mm] && \hspace{12mm} +
	\left.
	u_{\vec{k}}^{2}
	e^{ - i E_{\vec{k}} t }
	\dfrac{1}{2}
	\Big( 1 - {\rm tanh}\dfrac{\beta E_{\vec{k}}}{2} \Big)
	+
	|v_{\vec{k} \uparrow \uparrow}|^{2}
	e^{ i E_{\vec{k}} t }
	\dfrac{1}{2}
	\Big( 1 + {\rm tanh}\dfrac{\beta E_{\vec{k}}}{2} \Big)
	+
	|v_{\vec{k} \uparrow \downarrow}|^{2}
	e^{ i E_{\vec{k}} t }
	\dfrac{1}{2}
	\Big( 1 + {\rm tanh}\dfrac{\beta E_{\vec{k}}}{2} \Big)
	\right]
	\nonumber \\[4mm]
	&=&
	\theta(t)
	\left[
		u_{\vec{k}}^{2}
		e^{ - i E_{\vec{k}} t }
		+
		\Big(
		|v_{\vec{k} \uparrow \uparrow}|^{2}
		+
		|v_{\vec{k} \uparrow \downarrow}|^{2}
		\Big)
		e^{ i E_{\vec{k}} t }
		\right]
\end{eqnarray}


あるいは、フーリエ空間では、

\begin{eqnarray}
	G^{R}_{\uparrow \uparrow}(k)
	&=&
	\int \! dt \
	G^{R}_{\vec{k}}(t)
	e^{i \omega t}
	\nonumber \\[2mm] &=&
	-
	i
	\lim_{\eta \to +0}
	\int^{\infty}_{0} \!\! dt \
	\left[
		u_{\vec{k}}^{2}
		e^{ - i E_{\vec{k}} t }
		+
		\Big(
		|v_{\vec{k} \uparrow \uparrow}|^{2}
		+
		|v_{\vec{k} \uparrow \downarrow}|^{2}
		\Big)
		e^{ i E_{\vec{k}} t }
		\right]
	e^{i \omega t - \eta t}
	\nonumber \\[2mm] &=&
	\lim_{\eta \to +0}
	\left(
	\dfrac{u_{\vec{k}}^{2}}{ \omega - E_{\vec{k}} + i \eta }
	+
	\dfrac{
		|v_{\vec{k} \uparrow \uparrow}|^{2}
		+
		|v_{\vec{k} \uparrow \downarrow}|^{2}
	}{ \omega + E_{\vec{k}} + i \eta }
	\right)
	.
\end{eqnarray}
%
虚数部は状態密度に関連している。

\begin{eqnarray}
	&&
	- \dfrac{ 1 }{ \pi }
	\sum_{\vec{k}}
	{\rm Im} G^{R}_{\uparrow \uparrow}(k)
	\nonumber \\ &=&
	-
	\dfrac{1}{\pi}
	\lim_{\eta \to +0}
	\sum_{\vec{k}}
	\left(
	{\rm Im}
	\dfrac{u_{\vec{k}}^{2}}{ \omega - E_{\vec{k}} + i \eta }
	+
	{\rm Im}
	\dfrac{ |v_{\vec{k} \uparrow \uparrow}|^{2}
		+
		|v_{\vec{k} \uparrow \downarrow}|^{2}}{ \omega + E_{\vec{k}} + i \eta }
	\right)
	\nonumber \\[2mm] &=&
	\sum_{\vec{k}}
	\left[
		u_{\vec{k}}^{2}
		\delta( \omega - E_{\vec{k}} )
		+
		\left( |v_{\vec{k} \uparrow \uparrow}|^{2}
		+
		|v_{\vec{k} \uparrow \downarrow}|^{2} \right)
		\delta( \omega + E_{\vec{k}} )
		\right]
	.
\end{eqnarray}
%一般に、スピン空間における遅延グリーン関数(通常部分、原文: normalous)は、

\begin{eqnarray}
	\Big(
	G_{\vec{k} \sigma \sigma'}^{R}
	\Big)
	&=&
	\left(
	\begin{array}{cc}
			G_{\vec{k} \uparrow \uparrow}^{R}   & G_{\vec{k} \uparrow \downarrow}^{R}   \\[3mm]
			G_{\vec{k} \downarrow \uparrow}^{R} & G_{\vec{k} \downarrow \downarrow}^{R}
		\end{array}
	\right)
	,
\end{eqnarray}
%
次のように書き下すことができる。

\begin{eqnarray}
	&&
	i
	G_{\vec{k} \sigma \sigma'}^{R}(t)
	\ \ = \ \
	\theta(t)
	\left\langle \left\{
	c_{\vec{k} \sigma}(t), c_{\vec{k} \sigma'}^{\dagger}
	\right\} \right\rangle
	\ \ = \ \
	\theta(t)
	\left\langle
	c_{\vec{k} \sigma}(t) c_{\vec{k} \sigma'}^{\dagger}
	+
	c_{\vec{k} \sigma'}^{\dagger} c_{\vec{k} \sigma}(t)
	\right\rangle
	\nonumber \\[4mm] &=&
	\theta(t)
	\left\langle
	\left(
	u_{ \vec{k} } \delta_{ \sigma \uparrow } a_{\vec{k} \uparrow} e^{- i E_{\vec{k}} t}
	+
	u_{ \vec{k} } \delta_{ \sigma \downarrow } a_{\vec{k} \downarrow} e^{- i E_{\vec{k}} t}
	-
	v_{\vec{k} \sigma \uparrow} a_{-\vec{k} \uparrow}^{\dagger} e^{ i E_{\vec{k}} t}
	-
	v_{\vec{k} \sigma \downarrow} a_{-\vec{k} \downarrow}^{\dagger} e^{ i E_{\vec{k}} t}
	\right)
	\right.
	\nonumber \\ && \times
	\left.
	\left(
	u_{ \vec{k} } \delta_{ \sigma' \uparrow } a_{\vec{k} \uparrow}^{\dagger}
	+
	u_{ \vec{k} } \delta_{ \sigma' \downarrow } a_{\vec{k} \downarrow}^{\dagger}
	-
	v_{\vec{k} \sigma' \uparrow}^{*} a_{-\vec{k} \uparrow}
	-
	v_{\vec{k} \sigma' \downarrow}^{*} a_{-\vec{k} \downarrow}
	\right)
	\right.
	\nonumber \\[2mm] && \hspace{5mm} +
	\left.
	\left(
	u_{ \vec{k} } \delta_{ \sigma' \uparrow } a_{\vec{k} \uparrow}^{\dagger}
	+
	u_{ \vec{k} } \delta_{ \sigma' \downarrow } a_{\vec{k} \downarrow}^{\dagger}
	-
	v_{\vec{k} \sigma' \uparrow}^{*} a_{-\vec{k} \uparrow}
	-
	v_{\vec{k} \sigma' \downarrow}^{*} a_{-\vec{k} \downarrow}
	\right)
	\right.
	\nonumber \\ && \times
	\left.
	\left(
	u_{ \vec{k} } \delta_{ \sigma \uparrow } a_{\vec{k} \uparrow} e^{- i E_{\vec{k}} t}
	+
	u_{ \vec{k} } \delta_{ \sigma \downarrow } a_{\vec{k} \downarrow} e^{- i E_{\vec{k}} t}
	-
	v_{\vec{k} \sigma \uparrow} a_{-\vec{k} \uparrow}^{\dagger} e^{ i E_{\vec{k}} t}
	-
	v_{\vec{k} \sigma \downarrow} a_{-\vec{k} \downarrow}^{\dagger} e^{ i E_{\vec{k}} t}
	\right)
	\right\rangle
	\nonumber \\[4mm]
	&=&
	\theta(t)
	\left(
	u_{\vec{k}}^{2}
	\delta_{\sigma \uparrow}
	\delta_{\sigma' \uparrow}
	e^{ - i E_{\vec{k}} t }
	\left\langle
	a_{\vec{k} \uparrow}
	a_{\vec{k} \uparrow}^{\dagger}
	\right\rangle
	+
	u_{\vec{k}}^{2}
	\delta_{\sigma \downarrow}
	\delta_{\sigma' \downarrow}
	e^{ - i E_{\vec{k}} t }
	\left\langle
	a_{\vec{k} \downarrow}
	a_{\vec{k} \downarrow}^{\dagger}
	\right\rangle
	\right.
	\nonumber \\ && \hspace{5mm} +
	\left.
	v_{\vec{k} \sigma \uparrow}
	v_{\vec{k} \sigma' \uparrow}^{*}
	e^{ i E_{\vec{k}} t }
	\left\langle
	a_{-\vec{k} \uparrow}^{\dagger}
	a_{-\vec{k} \uparrow}
	\right\rangle
	+
	v_{\vec{k} \sigma \downarrow}
	v_{\vec{k} \sigma' \downarrow}^{*}
	e^{ i E_{\vec{k}} t }
	\left\langle
	a_{-\vec{k} \downarrow}^{\dagger}
	a_{-\vec{k} \downarrow}
	\right\rangle
	\right.
	\nonumber \\[1mm] && \hspace{12mm} +
	\left.
	u_{\vec{k}}^{2}
	\delta_{\sigma \uparrow}
	\delta_{\sigma' \uparrow}
	e^{ - i E_{\vec{k}} t }
	\left\langle
	a_{\vec{k} \uparrow}^{\dagger}
	a_{\vec{k} \uparrow}
	\right\rangle
	+
	u_{\vec{k}}^{2}
	\delta_{\sigma \downarrow}
	\delta_{\sigma' \downarrow}
	e^{ - i E_{\vec{k}} t }
	\left\langle
	a_{\vec{k} \downarrow}^{\dagger}
	a_{\vec{k} \downarrow}
	\right\rangle
	\right.
	\nonumber \\ && \hspace{5mm} +
	\left.
	v_{\vec{k} \sigma \uparrow}
	v_{\vec{k} \sigma' \uparrow}^{*}
	e^{ i E_{\vec{k}} t }
	\left\langle
	a_{-\vec{k} \uparrow}
	a_{-\vec{k} \uparrow}^{\dagger}
	\right\rangle
	+
	v_{\vec{k} \sigma \downarrow}
	v_{\vec{k} \sigma' \downarrow}^{*}
	e^{ i E_{\vec{k}} t }
	\left\langle
	a_{-\vec{k} \downarrow}
	a_{-\vec{k} \downarrow}^{\dagger}
	\right\rangle
	\right)
	\nonumber \\[4mm]
	&=&
	\theta(t)
	\left[
		u_{\vec{k}}^{2}
		\Big(
		\delta_{\sigma \uparrow}
		\delta_{\sigma' \uparrow}
		+
		\delta_{\sigma \downarrow}
		\delta_{\sigma' \downarrow}
		\Big)
		e^{ - i E_{\vec{k}} t }
		+
		\Big(
		v_{\vec{k} \sigma \uparrow}
		v_{\vec{k} \sigma' \uparrow}^{*}
		+
		v_{\vec{k} \sigma \downarrow}
		v_{\vec{k} \sigma' \downarrow}^{*}
		\Big)
		e^{ i E_{\vec{k}} t }
		\right]
	\nonumber \\[4mm]
	&=&
	\theta(t)
	\left[
		u_{\vec{k}}^{2}
		\delta_{\sigma \sigma'}
		e^{ - i E_{\vec{k}} t }
		+
		\Big(
		v_{\vec{k} \sigma \uparrow}
		v_{\vec{k} \sigma' \uparrow}^{*}
		+
		v_{\vec{k} \sigma \downarrow}
		v_{\vec{k} \sigma' \downarrow}^{*}
		\Big)
		e^{ i E_{\vec{k}} t }
		\right]
	,
\end{eqnarray}

フーリエ変換で以下を得る。

\begin{eqnarray}
	&&
	G^{R}_{\sigma \sigma'}(k)
	\ = \
	\int \! dt \
	G^{R}_{\vec{k} \sigma \sigma'}(t)
	e^{i \omega t}
	\nonumber \\[4mm] &=&
	-
	i
	\lim_{\eta \to +0}
	\int^{\infty}_{0} \!\! dt \
	\left[
		u_{\vec{k}}^{2}
		\delta_{\sigma \sigma'}
		e^{ - i E_{\vec{k}} t }
		+
		\Big(
		v_{\vec{k} \sigma \uparrow}
		v_{\vec{k} \sigma' \uparrow}^{*}
		+
		v_{\vec{k} \sigma \downarrow}
		v_{\vec{k} \sigma' \downarrow}^{*}
		\Big)
		e^{ i E_{\vec{k}} t }
		\right]
	e^{i \omega t - \eta t}
	\nonumber \\[4mm] &=&
	\lim_{\eta \to +0}
	\left[
		\dfrac{
			u_{\vec{k}}^{2}
			\delta_{\sigma \sigma'}
		}{ \omega - E_{\vec{k}} + i \eta }
		+
		\dfrac{
			v_{\vec{k} \sigma \uparrow}
			v_{\vec{k} \sigma' \uparrow}^{*}
			+
			v_{\vec{k} \sigma \downarrow}
			v_{\vec{k} \sigma' \downarrow}^{*}
		}{ \omega + E_{\vec{k}} + i \eta }
		\right]
	,
\end{eqnarray}

\begin{eqnarray}
	&&
	- \dfrac{ 1 }{ \pi }
	\sum_{\vec{k}}
	{\rm Im} G^{R}_{\sigma \sigma'}(k)
	\nonumber \\[2mm] &=&
	-
	\dfrac{1}{\pi}
	\lim_{\eta \to +0}
	\sum_{\vec{k}}
	\left[
	{\rm Im}
	\dfrac{
		u_{\vec{k}}^{2}
		\delta_{\sigma \sigma'}
	}{ \omega - E_{\vec{k}} + i \eta }
	+
	{\rm Im}
	\dfrac{
		v_{\vec{k} \sigma \uparrow}
		v_{\vec{k} \sigma' \uparrow}^{*}
		+
		v_{\vec{k} \sigma \downarrow}
		v_{\vec{k} \sigma' \downarrow}^{*}
	}{ \omega + E_{\vec{k}} + i \eta }
	\right]
	\nonumber \\[2mm] &=&
	\sum_{\vec{k}}
	\left[
		u_{\vec{k}}^{2}
		\delta_{\sigma \sigma'}
		\delta( \omega - E_{\vec{k}} )
		+
		\Big(
		v_{\vec{k} \sigma \uparrow}
		v_{\vec{k} \sigma' \uparrow}^{*}
		+
		v_{\vec{k} \sigma \downarrow}
		v_{\vec{k} \sigma' \downarrow}^{*}
		\Big)
		\delta( \omega + E_{\vec{k}} )
		\right]
	.
\end{eqnarray}


以上から次を得る。

\begin{eqnarray}
	&&
	\left(
	\begin{array}{cc}
			G_{\uparrow \uparrow}^{R}(k)   & G_{\uparrow \downarrow}^{R}(k)   \\[3mm]
			G_{\downarrow \uparrow}^{R}(k) & G_{\downarrow \downarrow}^{R}(k)
		\end{array}
	\right)
	\nonumber \\[4mm] &=&
	\lim_{\eta \to +0}
	\left(
	\begin{array}{cc}
			\dfrac{
				u_{\vec{k}}^{2}
			}{ \omega - E_{\vec{k}} + i \eta }
			+
			\dfrac{
				|v_{\vec{k} \uparrow \uparrow}|^{2}
				+
				|v_{\vec{k} \uparrow \downarrow}|^{2}
			}{ \omega + E_{\vec{k}} + i \eta }
			 &
			\dfrac{
				v_{\vec{k} \uparrow \uparrow}
				v_{\vec{k} \downarrow \uparrow}^{*}
				+
				v_{\vec{k} \uparrow \downarrow}
				v_{\vec{k} \downarrow \downarrow}^{*}
			}{ \omega + E_{\vec{k}} + i \eta }
			\\[3mm]
			\dfrac{
				v_{\vec{k} \downarrow \uparrow}
				v_{\vec{k} \uparrow \uparrow}^{*}
				+
				v_{\vec{k} \downarrow \downarrow}
				v_{\vec{k} \uparrow \downarrow}^{*}
			}{ \omega + E_{\vec{k}} + i \eta }
			 &
			\dfrac{
				u_{\vec{k}}^{2}
			}{ \omega - E_{\vec{k}} + i \eta }
			+
			\dfrac{
				|v_{\vec{k} \downarrow \uparrow}|^{2}
				+
				|v_{\vec{k} \downarrow \downarrow}|^{2}
			}{ \omega + E_{\vec{k}} + i \eta }
		\end{array}
	\right)
	.
\end{eqnarray}
%
この結果は、$(v_{\vec{k} \sigma \sigma'}) = v_{\vec{k}} \hat{\sigma}^{x}$
、
$(v_{\vec{k} \sigma \sigma'}^{*}) = v_{\vec{k}}^{*} \hat{\sigma}^{x}$
とおくことで、一重項の場合に帰着できる。

\begin{eqnarray}
	\left(
	\begin{array}{cc}
			G_{\uparrow \uparrow}^{R}(k)   & G_{\uparrow \downarrow}^{R}(k)   \\[3mm]
			G_{\downarrow \uparrow}^{R}(k) & G_{\downarrow \downarrow}^{R}(k)
		\end{array}
	\right)
	&=&
	\lim_{\eta \to +0}
	\left(
	\dfrac{
		u_{\vec{k}}^{2}
	}{ \omega - E_{\vec{k}} + i \eta }
	+
	\dfrac{ |v_{ \vec{k} }|^{2} }{ \omega + E_{\vec{k}} + i \eta }
	\right)
	\hat{1}_{2 \times 2}
	.
\end{eqnarray}
%


スピン空間における遅延異常(原文: anormalous)グリーン関数は、

\begin{eqnarray}
	\Big(
	F_{\vec{k} \sigma \sigma'}^{R}
	\Big)
	&=&
	\left(
	\begin{array}{cc}
			F_{\vec{k} \uparrow \uparrow}^{R}   & F_{\vec{k} \uparrow \downarrow}^{R}   \\[3mm]
			F_{\vec{k} \downarrow \uparrow}^{R} & F_{\vec{k} \downarrow \downarrow}^{R}
		\end{array}
	\right)
	,
\end{eqnarray}
%
そして、

\begin{eqnarray}
	&&
	i
	F_{\vec{k} \sigma \sigma'}^{R}(t)
	\ \ = \ \
	\theta(t)
	\left\langle \left\{
	c_{\vec{k} \sigma}(t), c_{-\vec{k} \sigma'}
	\right\} \right\rangle
	\ \ = \ \
	\theta(t)
	\left\langle
	c_{\vec{k} \sigma}(t) c_{-\vec{k} \sigma'}
	+
	c_{-\vec{k} \sigma'} c_{\vec{k} \sigma}(t)
	\right\rangle
	\nonumber \\[4mm] &=&
	\theta(t)
	\left\langle
	\left(
	u_{ \vec{k} } \delta_{ \sigma \uparrow } a_{\vec{k} \uparrow} e^{- i E_{\vec{k}} t}
	+
	u_{ \vec{k} } \delta_{ \sigma \downarrow } a_{\vec{k} \downarrow} e^{- i E_{\vec{k}} t}
	-
	v_{\vec{k} \sigma \uparrow} a_{-\vec{k} \uparrow}^{\dagger} e^{ i E_{\vec{k}} t}
	-
	v_{\vec{k} \sigma \downarrow} a_{-\vec{k} \downarrow}^{\dagger} e^{ i E_{\vec{k}} t}
	\right)
	\right.
	\nonumber \\ && \times
	\left.
	\left(
	v_{\vec{k} \sigma' \uparrow} a_{\vec{k} \uparrow}^{\dagger}
	+
	v_{\vec{k} \sigma' \downarrow} a_{\vec{k} \downarrow}^{\dagger}
	+
	u_{\vec{k}} \delta_{\sigma' \uparrow} a_{-\vec{k} \uparrow}
	+
	u_{\vec{k}} \delta_{\sigma' \downarrow} a_{-\vec{k} \downarrow}
	\right)
	\right.
	\nonumber \\[2mm] && \hspace{5mm} +
	\left.
	\left(
	v_{\vec{k} \sigma' \uparrow} a_{\vec{k} \uparrow}^{\dagger}
	+
	v_{\vec{k} \sigma' \downarrow} a_{\vec{k} \downarrow}^{\dagger}
	+
	u_{\vec{k}} \delta_{\sigma' \uparrow} a_{-\vec{k} \uparrow}
	+
	u_{\vec{k}} \delta_{\sigma' \downarrow} a_{-\vec{k} \downarrow}
	\right)
	\right.
	\nonumber \\ && \times
	\left.
	\left(
	u_{ \vec{k} } \delta_{ \sigma \uparrow } a_{\vec{k} \uparrow} e^{- i E_{\vec{k}} t}
	+
	u_{ \vec{k} } \delta_{ \sigma \downarrow } a_{\vec{k} \downarrow} e^{- i E_{\vec{k}} t}
	-
	v_{\vec{k} \sigma \uparrow} a_{-\vec{k} \uparrow}^{\dagger} e^{ i E_{\vec{k}} t}
	-
	v_{\vec{k} \sigma \downarrow} a_{-\vec{k} \downarrow}^{\dagger} e^{ i E_{\vec{k}} t}
	\right)
	\right\rangle
	\nonumber \\[4mm]
	&=&
	\theta(t)
	\left(
	u_{\vec{k}}
	v_{\vec{k} \sigma' \uparrow}
	\delta_{\sigma \uparrow}
	e^{ - i E_{\vec{k}} t }
	\left\langle
	a_{\vec{k} \uparrow}
	a_{\vec{k} \uparrow}^{\dagger}
	\right\rangle
	+
	u_{\vec{k}}
	v_{\vec{k} \sigma' \downarrow}
	\delta_{\sigma \downarrow}
	e^{ - i E_{\vec{k}} t }
	\left\langle
	a_{\vec{k} \downarrow}
	a_{\vec{k} \downarrow}^{\dagger}
	\right\rangle
	\right.
	\nonumber \\ && \hspace{4mm} -
	\left.
	u_{\vec{k}}
	v_{\vec{k} \sigma \uparrow}
	\delta_{\sigma' \uparrow}
	e^{ i E_{\vec{k}} t }
	\left\langle
	a_{-\vec{k} \uparrow}^{\dagger}
	a_{-\vec{k} \uparrow}
	\right\rangle
	-
	u_{\vec{k}}
	v_{\vec{k} \sigma \downarrow}
	\delta_{\sigma' \downarrow}
	e^{ i E_{\vec{k}} t }
	\left\langle
	a_{-\vec{k} \downarrow}^{\dagger}
	a_{-\vec{k} \downarrow}
	\right\rangle
	\right.
	\nonumber \\[1mm] && \hspace{5mm} +
	\left.
	u_{\vec{k}}
	v_{\vec{k} \sigma' \uparrow}
	\delta_{\sigma \uparrow}
	e^{ - i E_{\vec{k}} t }
	\left\langle
	a_{\vec{k} \uparrow}^{\dagger}
	a_{\vec{k} \uparrow}
	\right\rangle
	+
	u_{\vec{k}}
	v_{\vec{k} \sigma' \downarrow}
	\delta_{\sigma \downarrow}
	e^{ - i E_{\vec{k}} t }
	\left\langle
	a_{\vec{k} \downarrow}^{\dagger}
	a_{\vec{k} \downarrow}
	\right\rangle
	\right.
	\nonumber \\ && \hspace{4mm} -
	\left.
	u_{ \vec{k} }
	v_{\vec{k} \sigma \uparrow}
	\delta_{\sigma' \uparrow}
	e^{ i E_{\vec{k}} t }
	\left\langle
	a_{-\vec{k} \uparrow}
	a_{-\vec{k} \uparrow}^{\dagger}
	\right\rangle
	-
	u_{\vec{k}}
	v_{\vec{k} \sigma \downarrow}
	\delta_{\sigma' \downarrow}
	e^{ i E_{\vec{k}} t }
	\left\langle
	a_{-\vec{k} \downarrow}
	a_{-\vec{k} \downarrow}^{\dagger}
	\right\rangle
	\right)
	\nonumber \\[4mm]
	&=&
	u_{\vec{k}}
	\theta(t)
	\left[
		\Big(
		v_{\vec{k} \sigma' \uparrow}
		\delta_{\sigma \uparrow}
		+
		v_{\vec{k} \sigma' \downarrow}
		\delta_{\sigma \downarrow}
		\Big)
		e^{ - i E_{\vec{k}} t }
		-
		\Big(
		v_{\vec{k} \sigma \uparrow}
		\delta_{\sigma' \uparrow}
		+
		v_{\vec{k} \sigma \downarrow}
		\delta_{\sigma' \downarrow}
		\Big)
		e^{ i E_{\vec{k}} t }
		\right]
	,
\end{eqnarray}
\begin{eqnarray}
	&&
	F_{\sigma \sigma'}^{R}(k)
	\ = \
	\int \! dt \ F_{\vec{k} \sigma \sigma'}^{R}(t) e^{i \omega t}
	\nonumber \\[4mm] &=&
	-
	i
	u_{\vec{k}}
	\lim_{\eta \to +0}
	\int^{\infty}_{0} \!\! dt
	\left[
		\Big(
		v_{\vec{k} \sigma' \uparrow}
		\delta_{\sigma \uparrow}
		+
		v_{\vec{k} \sigma' \downarrow}
		\delta_{\sigma \downarrow}
		\Big)
		e^{ - i E_{\vec{k}} t }
		-
		\Big(
		v_{\vec{k} \sigma \uparrow}
		\delta_{\sigma' \uparrow}
		+
		v_{\vec{k} \sigma \downarrow}
		\delta_{\sigma' \downarrow}
		\Big)
		e^{ i E_{\vec{k}} t }
		\right]
	e^{i \omega t - \eta t}
	\nonumber \\[4mm] &=&
	-
	u_{\vec{k}}
	\lim_{\eta \to +0}
	\left[
		\dfrac{
			v_{\vec{k} \sigma' \uparrow}
			\delta_{\sigma \uparrow}
			+
			v_{\vec{k} \sigma' \downarrow}
			\delta_{\sigma \downarrow}
		}{ \omega - E_{\vec{k}} + i \eta }
		-
		\dfrac{
			v_{\vec{k} \sigma \uparrow}
			\delta_{\sigma' \uparrow}
			+
			v_{\vec{k} \sigma \downarrow}
			\delta_{\sigma' \downarrow}
		}{ \omega + E_{\vec{k}} + i \eta }
		\right]
	.
\end{eqnarray}
%
次を得る。

\begin{eqnarray}
	&&
	\left(
	\begin{array}{cc}
			F_{\uparrow \uparrow}^{R}(k)   & F_{\uparrow \downarrow}^{R}(k)   \\[3mm]
			F_{\downarrow \uparrow}^{R}(k) & F_{\downarrow \downarrow}^{R}(k)
		\end{array}
	\right)
	\nonumber \\ &=&
	-
	u_{ \vec{k} }
	\lim_{\eta \to +0}
	\left[
		\dfrac{ 1 }{ \omega - E_{\vec{k}} + i \eta }
		\left(
		\begin{array}{cc}
				v_{ \vec{k} \uparrow \uparrow }
				 &
				v_{ \vec{k} \downarrow \uparrow }
				\\[3mm]
				v_{ \vec{k} \uparrow \downarrow }
				 &
				v_{ \vec{k} \downarrow \downarrow }
			\end{array}
		\right)
		-
		\dfrac{ 1 }{ \omega + E_{\vec{k}} + i \eta }
		\left(
		\begin{array}{cc}
				v_{ \vec{k} \uparrow \uparrow }
				 &
				v_{ \vec{k} \uparrow \downarrow }
				\\[3mm]
				v_{ \vec{k} \downarrow \uparrow }
				 &
				v_{ \vec{k} \downarrow \downarrow }
			\end{array}
		\right)
		\right]
	.
\end{eqnarray}
%
この結果は、$(v_{\vec{k} \sigma \sigma'}) = v_{\vec{k}} \hat{\sigma}^{x}$
、
$(v_{\vec{k} \sigma \sigma'}^{*}) = v_{\vec{k}}^{*} \hat{\sigma}^{x}$
とおくことで、一重項の場合に帰着できる。

\begin{eqnarray}
	&&
	\left(
	\begin{array}{cc}
			F_{\uparrow \uparrow}^{R}(k)   & F_{\uparrow \downarrow}^{R}(k)   \\[3mm]
			F_{\downarrow \uparrow}^{R}(k) & F_{\downarrow \downarrow}^{R}(k)
		\end{array}
	\right)
	\nonumber \\ &=&
	-
	u_{ \vec{k} }
	v_{ \vec{k} }
	\lim_{\eta \to +0}
	\left(
	\dfrac{ 1 }{ \omega - E_{\vec{k}} + i \eta }
	-
	\dfrac{ 1 }{ \omega + E_{\vec{k}} + i \eta }
	\right)
	\hat{\sigma}^{x}
	.
\end{eqnarray}
%

${}$スピン空間における遅延・反異常グリーン関数は、

\begin{eqnarray}
	\Big(
	\bar{F}_{\vec{k} \sigma \sigma'}^{R}
	\Big)
	&=&
	\left(
	\begin{array}{cc}
			\bar{F}_{\vec{k} \uparrow \uparrow}^{R}   & \bar{F}_{\vec{k} \uparrow \downarrow}^{R}   \\[3mm]
			\bar{F}_{\vec{k} \downarrow \uparrow}^{R} & \bar{F}_{\vec{k} \downarrow \downarrow}^{R}
		\end{array}
	\right)
	,
\end{eqnarray}
%
次のように得られる。

\begin{eqnarray}
	&&
	i
	\bar{F}_{\vec{k} \sigma \sigma'}^{R}(t)
	\ \ = \ \
	\theta(t)
	\left\langle \left\{
	c_{-\vec{k} \sigma}^{\dagger}(t), c_{\vec{k} \sigma'}^{\dagger}
	\right\} \right\rangle
	\ \ = \ \
	\theta(t)
	\left\langle
	c_{- \vec{k} \sigma}^{\dagger}(t) c_{\vec{k} \sigma'}^{\dagger}
	+
	c_{\vec{k} \sigma'}^{\dagger} c_{- \vec{k} \sigma}^{\dagger}(t)
	\right\rangle
	\nonumber \\[4mm] &=&
	\theta(t)
	\left\langle
	\left(
	v_{\vec{k} \sigma \uparrow}^{*} a_{\vec{k} \uparrow} e^{- i E_{\vec{k}} t}
	+
	v_{\vec{k} \sigma \downarrow}^{*} a_{\vec{k} \downarrow} e^{- i E_{\vec{k}} t}
	+
	u_{\vec{k}} \delta_{\sigma \uparrow} a_{- \vec{k} \uparrow}^{\dagger} e^{ i E_{\vec{k}} t}
	+
	u_{\vec{k}} \delta_{\sigma \downarrow} a_{- \vec{k} \downarrow}^{\dagger} e^{ i E_{\vec{k}} t}
	\right)
	\right.
	\nonumber \\[2mm] && \hspace{5mm} \times
	\left(
	u_{\vec{k}} \delta_{\sigma' \uparrow} a_{\vec{k} \uparrow}^{\dagger}
	+
	u_{\vec{k}} \delta_{\sigma' \downarrow} a_{\vec{k} \downarrow}^{\dagger}
	-
	v_{\vec{k} \sigma' \uparrow}^{*} a_{-\vec{k} \uparrow}
	-
	v_{\vec{k} \sigma' \downarrow}^{*} a_{-\vec{k} \downarrow}
	\right)
	\nonumber \\[2mm] && \hspace{5mm} +
	\left(
	u_{\vec{k}} \delta_{\sigma' \uparrow} a_{\vec{k} \uparrow}^{\dagger}
	+
	u_{\vec{k}} \delta_{\sigma' \downarrow} a_{\vec{k} \downarrow}^{\dagger}
	-
	v_{\vec{k} \sigma' \uparrow}^{*} a_{-\vec{k} \uparrow}
	-
	v_{\vec{k} \sigma' \downarrow}^{*} a_{-\vec{k} \downarrow}
	\right)
	\nonumber \\[2mm] && \hspace{5mm} \times
	\left.
	\left(
	v_{\vec{k} \sigma \uparrow}^{*} a_{\vec{k} \uparrow} e^{- i E_{\vec{k}} t}
	+
	v_{\vec{k} \sigma \downarrow}^{*} a_{\vec{k} \downarrow} e^{- i E_{\vec{k}} t}
	+
	u_{\vec{k}} \delta_{\sigma \uparrow} a_{- \vec{k} \uparrow}^{\dagger} e^{ i E_{\vec{k}} t}
	+
	u_{\vec{k}} \delta_{\sigma \downarrow} a_{- \vec{k} \downarrow}^{\dagger} e^{ i E_{\vec{k}} t}
	\right)
	\right\rangle
	\nonumber \\[4mm]
	&=&
	\theta(t)
	\left(
	v_{\vec{k} \sigma \uparrow}^{*}
	u_{\vec{k}} \delta_{\sigma' \uparrow}
	e^{- i E_{\vec{k}} t}
	\left\langle
	a_{\vec{k} \uparrow}
	a_{\vec{k} \uparrow}^{\dagger}
	\right\rangle
	+
	v_{\vec{k} \sigma \downarrow}^{*}
	u_{\vec{k}}
	\delta_{\sigma' \downarrow}
	e^{- i E_{\vec{k}} t}
	\left\langle
	a_{\vec{k} \downarrow}
	a_{\vec{k} \downarrow}^{\dagger}
	\right\rangle
	\right.
	\nonumber \\[2mm] && \hspace{5mm} -
	u_{\vec{k}}
	\delta_{\sigma \uparrow}
	v_{\vec{k} \sigma' \uparrow}^{*}
	e^{ i E_{\vec{k}} t}
	\left\langle
	a_{- \vec{k} \uparrow}^{\dagger}
	a_{-\vec{k} \uparrow}
	\right\rangle
	-
	u_{\vec{k}} \delta_{\sigma \downarrow}
	v_{\vec{k} \sigma' \downarrow}^{*}
	e^{ i E_{\vec{k}} t}
	\left\langle
	a_{- \vec{k} \downarrow}^{\dagger}
	a_{- \vec{k} \downarrow}
	\right\rangle
	\nonumber \\[1mm] && \hspace{5mm} +
	\left.
	u_{\vec{k}}
	v_{\vec{k} \sigma \uparrow}^{*}
	\delta_{\sigma' \uparrow}
	e^{- i E_{\vec{k}} t}
	\left\langle
	a_{\vec{k} \uparrow}^{\dagger}
	a_{\vec{k} \uparrow}
	\right\rangle
	+
	u_{\vec{k}}
	v_{\vec{k} \sigma \downarrow}^{*}
	\delta_{\sigma' \downarrow}
	e^{- i E_{\vec{k}} t}
	\left\langle
	a_{\vec{k} \downarrow}^{\dagger}
	a_{\vec{k} \downarrow}
	\right\rangle
	\right.
	\nonumber \\[2mm] && \hspace{5mm} -
	\left.
	v_{\vec{k} \sigma' \uparrow}^{*}
	u_{\vec{k}}
	\delta_{\sigma \uparrow}
	e^{ i E_{\vec{k}} t}
	\left\langle
	a_{-\vec{k} \uparrow}
	a_{- \vec{k} \uparrow}^{\dagger}
	\right\rangle
	-
	v_{\vec{k} \sigma' \downarrow}^{*}
	u_{\vec{k}} \delta_{\sigma \downarrow}
	e^{ i E_{\vec{k}} t}
	\left\langle
	a_{-\vec{k} \downarrow}
	a_{- \vec{k} \downarrow}^{\dagger}
	\right\rangle
	\right)
	\nonumber \\[4mm]
	&=&
	u_{\vec{k}}
	\theta(t)
	\left[
		\Big(
		v_{\vec{k} \sigma \uparrow}^{*}
		\delta_{\sigma' \uparrow}
		+
		v_{\vec{k} \sigma \downarrow}^{*}
		\delta_{\sigma' \downarrow}
		\Big)
		e^{ - i E_{\vec{k}} t }
		-
		\Big(
		v_{\vec{k} \sigma' \uparrow}^{*}
		\delta_{\sigma \uparrow}
		+
		v_{\vec{k} \sigma' \downarrow}^{*}
		\delta_{\sigma \downarrow}
		\Big)
		e^{ i E_{\vec{k}} t }
		\right]
	,
\end{eqnarray}

\begin{eqnarray}
	&&
	F_{\sigma \sigma'}^{R}(k)
	\nonumber \\[4mm] &=&
	\int \! dt \ F_{\vec{k} \sigma \sigma'}^{R}(t) e^{i \omega t}
	\nonumber \\[4mm] &=&
	-
	i
	u_{\vec{k}}
	\lim_{\eta \to +0}
	\int^{\infty}_{0} \!\! dt
	\left[
		\Big(
		v_{\vec{k} \sigma \uparrow}^{*}
		\delta_{\sigma' \uparrow}
		+
		v_{\vec{k} \sigma \downarrow}^{*}
		\delta_{\sigma' \downarrow}
		\Big)
		e^{ - i E_{\vec{k}} t }
		-
		\Big(
		v_{\vec{k} \sigma' \uparrow}^{*}
		\delta_{\sigma \uparrow}
		+
		v_{\vec{k} \sigma' \downarrow}^{*}
		\delta_{\sigma \downarrow}
		\Big)
		e^{ i E_{\vec{k}} t }
		\right]
	e^{i \omega t - \eta t}
	\nonumber \\[4mm] &=&
	-
	u_{\vec{k}}
	\lim_{\eta \to +0}
	\left[
		\dfrac{
			v_{\vec{k} \sigma \uparrow}^{*}
			\delta_{\sigma' \uparrow}
			+
			v_{\vec{k} \sigma \downarrow}^{*}
			\delta_{\sigma' \downarrow}
		}{ \omega - E_{\vec{k}} + i \eta }
		-
		\dfrac{
			v_{\vec{k} \sigma' \uparrow}^{*}
			\delta_{\sigma \uparrow}
			+
			v_{\vec{k} \sigma' \downarrow}^{*}
			\delta_{\sigma \downarrow}
		}{ \omega + E_{\vec{k}} + i \eta }
		\right]
	.
\end{eqnarray}
%
次を得る。

\begin{eqnarray}
	&&
	\left(
	\begin{array}{cc}
			\bar{F}_{\uparrow \uparrow}^{R}(k)   & \bar{F}_{\uparrow \downarrow}^{R}(k)   \\[3mm]
			\bar{F}_{\downarrow \uparrow}^{R}(k) & \bar{F}_{\downarrow \downarrow}^{R}(k)
		\end{array}
	\right)
	\nonumber \\[4mm] &=&
	-
	u_{ \vec{k} }
	\lim_{\eta \to +0}
	\left[
		\dfrac{ 1 }{ \omega - E_{\vec{k}} + i \eta }
		\left(
		\begin{array}{cc}
				v_{\vec{k} \uparrow \uparrow}^{*}
				 &
				v_{\vec{k} \uparrow \downarrow}^{*}
				\\[3mm]
				v_{\vec{k} \downarrow \uparrow}^{*}
				 &
				v_{\vec{k} \downarrow \downarrow}^{*}
			\end{array}
		\right)
		-
		\dfrac{ 1 }{ \omega + E_{\vec{k}} + i \eta }
		\left(
		\begin{array}{cc}
				v_{\vec{k} \uparrow \uparrow}^{*}
				 &
				v_{\vec{k} \downarrow \uparrow}^{*}
				\\[3mm]
				v_{\vec{k} \uparrow \downarrow}^{*}
				 &
				v_{\vec{k} \downarrow \downarrow}^{*}
			\end{array}
		\right)
		\right]
	.
\end{eqnarray}
%
この結果は、$(v_{\vec{k} \sigma \sigma'}) = v_{\vec{k}} \hat{\sigma}^{x}$
、
$(v_{\vec{k} \sigma \sigma'}^{*}) = v_{\vec{k}}^{*} \hat{\sigma}^{x}$
とおくことで、一重項の場合に帰着できる。

\begin{eqnarray}
	\left(
	\begin{array}{cc}
			F_{\uparrow \uparrow}^{R}(k)   & F_{\uparrow \downarrow}^{R}(k)   \\[3mm]
			F_{\downarrow \uparrow}^{R}(k) & F_{\downarrow \downarrow}^{R}(k)
		\end{array}
	\right)
	&=&
	-
	u_{ \vec{k} }
	v_{ \vec{k} }^{*}
	\lim_{\eta \to +0}
	\left(
	\dfrac{ 1 }{ \omega - E_{\vec{k}} + i \eta }
	-
	\dfrac{ 1 }{ \omega + E_{\vec{k}} + i \eta }
	\right)
	\hat{\sigma}^{x}
	.
\end{eqnarray}
\end{document}
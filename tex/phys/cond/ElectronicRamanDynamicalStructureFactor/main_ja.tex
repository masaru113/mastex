\documentclass[uplatex,a4j,12pt,dvipdfmx]{jsarticle}
\usepackage{amsmath,amsthm,amssymb,bm,color,enumitem,mathrsfs,url,epic,eepic,ascmac,ulem,here,ascmac}
\usepackage[letterpaper,top=2cm,bottom=2cm,left=3cm,right=3cm,marginparwidth=1.75cm]{geometry}
\usepackage[english]{babel}
\usepackage[dvipdfm]{graphicx}
\usepackage[hypertex]{hyperref}
\title{
電子ラマン散乱における動的構造因子
}
\author{Masaru Okada}
\date{\today}
\begin{document}

\maketitle

\thispagestyle{empty}
無摂動のハミルトニアンを $H_{0}$ とし、有効摂動ハミルトニアン $H'$ は
\begin{eqnarray}
	H'
	&=&
	-e \dfrac{ 4 \pi n ( \vec{q} , \omega ) }{ q^{2} }
	e^{- i \omega t}
	\sum_{\vec{k},\sigma}
	\gamma_{\vec{k}}
	c^{\dagger}_{\vec{k}+\vec{q} \sigma} c_{\vec{p} \sigma}
	\nonumber \\ &=&
	- e \phi(\vec{q},\omega) e^{- i \omega t} \tilde{\rho}^{\dagger}_{\vec{q}}
\end{eqnarray}
とする。

ここで
$\phi(\vec{q},\omega) = \dfrac{ 4 \pi n_{t} ( \vec{q} , \omega ) }{ q^{2} }$
は、電荷
$n ( \vec{q} , \omega )$
によって誘起されるポテンシャルを表す。

全ハミルトニアンを $H=H_{0}+H'$ としたとき、
電子ラマン散乱実験によって次の差分が得られる。
$\Delta \tilde{\rho}_{\vec{q}} = \langle \tilde{\rho}_{\vec{q}} \rangle_{H} - \langle \tilde{\rho}_{\vec{q}} \rangle_{H_{0}}$


これは外力
$F(t) = e \phi(\vec{q},\omega) e^{- i \omega t}$
を用いて次のように表される。
\begin{eqnarray}
	\Delta \tilde{\rho}_{\vec{q}}
	&=&
	- \int^{t}_{-\infty} dt' F(t') \chi^{(R)}_{\tilde{\rho} \tilde{\rho}}(t - t')
\end{eqnarray}


ここで
$\chi^{(R)}_{\tilde{\rho} \tilde{\rho}}$
は(遅延部分の)応答関数としばしば呼ばれ、線形応答の範囲において R. Kubo によって示され、久保公式と呼ばれている。
\begin{eqnarray}
	\chi^{(R)}_{\tilde{\rho} \tilde{\rho}}(t)
	&=&
	- i \theta(t) \big\langle [ \tilde{\rho}_{\vec{q}}(t) , \tilde{\rho}_{\vec{q}}^{\dagger} ] \big\rangle
\end{eqnarray}


ここで、熱平均 $\langle \cdots \rangle$ は、熱力学ポテンシャル $\Omega$
と $H | n \rangle = E_{n} | n \rangle$ を満たす正規直交ベクトル $\big\{ | n \rangle \big\}$ を用いて展開できる。
\begin{eqnarray}
	\big\langle [ \tilde{\rho}_{\vec{q}}(t) , \tilde{\rho}_{\vec{q}}^{\dagger} ] \big\rangle
	&=&
	{\rm Tr}
	\Big[ e^{\beta ( \Omega - H ) }
		\big(
		e^{i H t} \tilde{\rho}_{\vec{q}} e^{-iHt} \tilde{\rho}_{\vec{q}}^{\dagger}
		-
		\tilde{\rho}_{\vec{q}}^{\dagger} e^{i H t} \tilde{\rho}_{\vec{q}} e^{-iHt}
		\big) \Big]
	\nonumber \\[2mm] &=&
	e^{\beta \Omega}
	\sum_{m,n} e^{i ( \beta E_{n} - E_{m} ) t }
	\big| \langle n | \tilde{\rho}_{\vec{q}} | m \rangle \big|^{2} \Big( e^{- \beta E_{n}} - e^{- \beta E_{m}}  \Big)
\end{eqnarray}


フーリエ変換すると、これは次のように書くこともできる。
\begin{eqnarray}
	\chi^{(R)}_{\tilde{\rho} \tilde{\rho}}(\vec{q},\omega)
	&=&
	- i \int^{\infty}_{0} \!\! dt \
	\big\langle [ \tilde{\rho}_{\vec{q}}(t) , \tilde{\rho}_{\vec{q}}^{\dagger} ] \big\rangle
\end{eqnarray}


一般に、ボルン近似において、散乱確率は動的構造因子に比例する。
\begin{eqnarray}
	S(\vec{q},\omega)
	&=&
	e^{\beta \Omega}
	\sum_{m,n} e^{- \beta E_{n}}
	\big| \langle n | \rho_{\vec{q}} | m \rangle \big|^{2} \delta (E_{n} - E_{m} + \omega)
\end{eqnarray}


$\chi^{(R)}_{\tilde{\rho} \tilde{\rho}}$ と動的構造因子の対応から、
以下のように電子ラマン動的構造因子 $\tilde{S}(\vec{q},\omega)$ を定義すればよい。
\begin{eqnarray}
	\tilde{S} (\vec{q},\omega)
	&=&
	e^{\beta \Omega}
	\sum_{m,n} e^{- \beta E_{n}}
	\big| \langle n | \tilde{\rho}_{\vec{q}} | m \rangle \big|^{2} \delta (E_{n} - E_{m} + \omega)
	\nonumber \\ &=&
	- \dfrac{1 + {\rm coth} (\beta \omega/2) }{2 \pi}
	{\rm Im} \chi^{(R)}_{\tilde{\rho} \tilde{\rho}} (\vec{q},\omega)
\end{eqnarray}
\end{document}
\documentclass[uplatex,a4j,12pt,dvipdfmx]{jsarticle}
\usepackage[english]{babel}
\usepackage[letterpaper,top=2cm,bottom=2cm,left=3cm,right=3cm,marginparwidth=1.75cm]{geometry}
\usepackage{amsmath, amssymb, bm}
\usepackage{graphicx}
\usepackage[colorlinks=true, allcolors=blue]{hyperref}
\usepackage{tikz-cd}

\title{
Anisotropic BCS Theory
}

\author{
Okada Masaru
}

\begin{document}
\maketitle

\section{Outline of BCS Theory}

In 1957, Bardeen, Cooper, and Schrieffer successfully established the first microscopic theory to approach the origin of superconductivity.
This theory is known as the \textbf{BCS theory}.

In 1956, Cooper discovered that a Fermi gas becomes unstable in the presence of an attractive interaction between fermions.

Specifically, when two fermions on the Fermi surface experience an attractive interaction, a ground state with lower energy than the Fermi gas exists.

This ground state becomes a coherent state, and electrons form pairs.
These pairs are called \textbf{Cooper pairs}.
These pairs are spatially more extended than the correlation length of the Fermi gas state, and their total momentum is zero.

It is more natural to think of the pairing as occurring in momentum space rather than real space.

The potential that originates from the electron-lattice interaction contains a region of attractive interaction in a thin layer near the Fermi surface in momentum space.

Within this potential, the channel functions as an isotropic attractive interaction, and the electron pair wavefunction takes on an orbital angular momentum of $l=0$.
This is an s-wave spin-singlet state.

Most known superconducting materials have conclusions that are consistent with the BCS theory.


\section{An Outline of BCS Theory Generalized to Non-S-Wave Pair Wavefunctions}

The electron-lattice interaction is not the only source of attractive interaction between fermions.

Other mechanisms can lead to anisotropic pair formation.
For example, in the case of spin-triplet superconductivity, the pair wavefunction is also anisotropic.

In 1961, Anderson and Morel, and in 1963, Balian and Werthamer, investigated this type of anisotropically generalized BCS theory.

This work was a precursor to major questions of the 1960s.
These questions later led to the construction of the theory of superfluid $^3$He, where it was discovered that p-wave triplet pairs are formed when assuming a spin fluctuation mechanism.

Below, I will briefly outline this extended BCS theory for anisotropic pairs.
In particular, I will summarize the symmetry of superconducting pairs, which is expected to be foundational for future research.

\section{Formulation}

\subsection{Mean-Field Hamiltonian}

First, let's consider the following effective Hamiltonian:

$$
	\mathcal{H}
	=
	\sum_{\bm{k},s}
	\varepsilon(\bm{k})
	a_{\bm{k} s}^{\dagger}
	a_{\bm{k} s}
	\ + \
	\dfrac{1}{2}
	\sum_{\bm{k},\bm{k}',s_{1},s_{2},s_{3},s_{4}}
	V_{s_{1} s_{2} s_{3} s_{4}}(\bm{k},\bm{k}')
	a_{-\bm{k} s_{1}}^{\dagger}
	a_{\bm{k} s_{2}}^{\dagger}
	a_{\bm{k}' s_{3}}
	a_{-\bm{k}' s_{4}}
	\ .
$$

Here, $\varepsilon(\bm{k})$ is the band energy measured from the chemical potential $\mu$, and

\[
	V_{s_{1} s_{2} s_{3} s_{4}}(\bm{k},\bm{k}')
	=
	\langle
	- \bm{k} , s_{1} ; \bm{k} , s_{2}
	|
	\hat{V}
	|
	- \bm{k}' , s_{4} ; \bm{k}' , s_{3}
	\rangle
	\nonumber
\]


represents the matrix element.
It's important to note that it has the following symmetries:

\[
	V_{s_{1} s_{2} s_{3} s_{4}}(\bm{k},\bm{k}')
	\ = \
	- V_{s_{2} s_{1} s_{3} s_{4}}( - \bm{k},\bm{k}')
	\ = \
	- V_{s_{1} s_{2} s_{4} s_{3}}(\bm{k}, - \bm{k}')
	\ = \
	V_{s_{4} s_{3} s_{2} s_{1}}( \bm{k}' , \bm{k})
	\ .
	\nonumber
\]

This $\hat{V}$ represents a general effective electron-electron interaction that functions as an attraction in a narrow region on the Fermi surface.
Specifically, by setting the cutoff energy to $\varepsilon_{c}$, we can consider

\[
	-
	\varepsilon_{c}
	\leq
	\varepsilon(\bm{k})
	\leq
	\varepsilon_{c}
	\nonumber
\]

The physical origin of this is omitted here.
The presence of an attractive interaction causes the degenerate Fermi gas to become unstable.

The Hamiltonian in equation (2.1) is treated using many-body techniques.

The first step is to use the mean-field approach.
We define the mean field, or pair potential, later called the gap function, as follows:

\[
	\Delta_{s s'}(\bm{k})
	=
	- \sum_{ \bm{k}' , s_{3} , s_{4} }
	V_{s s' s_{3} s_{4} }(\bm{k},\bm{k}')
	\langle
	a_{ \bm{k}' s_{3} }
	a_{ - \bm{k}' s_{4} }
	\rangle
	\ ,
\]\[
	\Delta_{s s'}^{*}( - \bm{k} )
	=
	\sum_{ \bm{k}' , s_{1} , s_{2} }
	V_{ s_{1} s_{2} s s' }(\bm{k}',\bm{k})
	\langle
	a_{ - \bm{k}' s_{1} }^{\dagger}
	a_{ \bm{k}' s_{2} }^{\dagger}
	\rangle
	\ .
\]

Here, the bracket notation $\langle A \rangle$ represents the expectation value

\[
	\langle A \rangle
	=
	\dfrac{ {\rm tr} [ {\rm exp} ( - \beta \mathcal{H} ) A ]  }{ {\rm tr} [ {\rm exp} ( - \beta \mathcal{H} ) ] }
	\ ,
	\nonumber
\]

where $\beta = (k_{B} T)^{-1}$ is the inverse temperature.

Now, let's consider replacing the factor $a_{\bm{k}s}^{(\dagger)} a_{-\bm{k}s'}^{(\dagger)}$ within the Hamiltonian (2.1).
First, consider the simple identity,

\[
	a_{\bm{k}s}^{(\dagger)}
	a_{-\bm{k}s'}^{(\dagger)}
	\ = \
	\langle
	a_{\bm{k}s}^{(\dagger)}
	a_{-\bm{k}s'}^{(\dagger)}
	\rangle
	\ + \
	\Big(
	a_{\bm{k}s}^{(\dagger)}
	a_{-\bm{k}s'}^{(\dagger)}
	\ - \
	\langle
	a_{\bm{k}s}^{(\dagger)}
	a_{-\bm{k}s'}^{(\dagger)}
	\rangle
	\Big)
	\nonumber
	\ .
\]

The term enclosed in parentheses represents the fluctuations of the operator around its mean-field expectation value.

We assume these fluctuations are negligible compared to the mean field, treating them as higher-order (infinitesimal) terms that can be ignored.

The Hamiltonian $\mathcal{H}$ is then reduced to a single-particle Hamiltonian, $\tilde{\mathcal{H}}$.

\[
	\tilde{\mathcal{H}}
	=
	\sum_{\bm{k},s}
	\varepsilon(\bm{k})
	a_{\bm{k} s}^{\dagger}
	a_{\bm{k} s}
	\ + \
	\dfrac{1}{2}
	\sum_{\bm{k},s_{1},s_{2}}
	\Big[
		\Delta_{ s_{1} s_{2} }( \bm{k} )
		a_{\bm{k} s_{1}}^{\dagger}
		a_{-\bm{k} s_{2}}^{\dagger}
		\ - \
		\Delta_{ s_{1} s_{2} }^{*}( - \bm{k} )
		a_{-\bm{k} s_{1}}
		a_{\bm{k} s_{2}}
		\Big]
	\ .
\]

Here, we've ignored the constant term containing only the mean field.
This constant only shifts the ground state energy, meaning it is renormalized into the chemical potential, so it can be omitted.
\subsection{Bogoliubov Transformation (Diagonalization)}
It's straightforward to diagonalize this single-particle effective Hamiltonian to find its eigenvalues and eigenoperators.
Let's call these eigenoperators $\alpha_{a}^{\dagger}$ and $\alpha_{b}$, which obey the following equations of motion:

\[
	\partial_{t} \alpha_{a}^{\dagger}
	=
	i [ \tilde{\mathcal{H}} , \alpha_{a}^{\dagger} ]
	\ = \
	E_{a} \alpha_{a}^{\dagger}
	\ ,
\]\[
	\partial_{t} \alpha_{b}
	=
	i [ \tilde{\mathcal{H}} , \alpha_{b} ]
	\ = \
	- E_{b} \alpha_{b}
	\ .
	\nonumber
\]

The constants $E_{a}$ and $-E_{b}$ are the eigenvalues.

The eigenvalues and eigenoperators can be obtained via the following canonical unitary transformation (the \textbf{Bogoliubov transformation}):

\[
	a_{\bm{k} s}
	=
	\sum_{s'}
	(
	u_{\bm{k} s s'}
	\alpha_{\bm{k} s'}
	\ + \
	v_{\bm{k} s s'}
	\alpha_{ - \bm{k} s'}^{\dagger}
	)
	\ .
\]


The new operators $\alpha_{\bm{k}s}^{(\dagger)}$ are fermions and obey anticommutation relations, corresponding to the elementary excitations of the new ground state.

If we define the spinors with operator components as follows:

\[
	\bm{a}_{\bm{k} s}
	=
	(
	a_{\bm{k} \uparrow}
	,
	a_{\bm{k} \downarrow}
	,
	a_{- \bm{k} \uparrow}^{\dagger}
	,
	a_{- \bm{k} \downarrow}^{\dagger}
	)
	\ , \]\[
	\bm{\alpha}_{\bm{k} s}
	=
	(
	\alpha_{\bm{k} \uparrow}
	,
	\alpha_{\bm{k} \downarrow}
	,
	\alpha_{- \bm{k} \uparrow}^{\dagger}
	,
	\alpha_{- \bm{k} \downarrow}^{\dagger}
	)
	\ .
	\nonumber
\]

The Bogoliubov transformation (equation 2.4) can be written more compactly as:

\[
	\bm{a}_{\bm{k}}
	=
	U_{\bm{k}}
	\bm{\alpha}_{\bm{k}}
	\ ,
	\nonumber
\]

where the $4 \times 4$ matrix $U_{\bm{k}}$ has the form

\[
	U_{\bm{k}}
	=
	\left(
	\begin{array}{cc}
			\hat{u}_{\bm{k}}      & \hat{v}_{\bm{k}}      \\
			\hat{v}_{-\bm{k}}^{*} & \hat{u}_{-\bm{k}}^{*}
		\end{array}
	\right)
	\ ,
\]

and satisfies the unitary condition $U_{\bm{k}} U_{\bm{k}}^{\dagger} = 1$.
The $2 \times 2$ matrices $\hat{u}_{\bm{k}}$ and $\hat{v}_{\bm{k}}$ are defined in equation (2.4).

Using this representation, we can diagonalize $\tilde{\mathcal{H}}$.

\[
	\hat{E}_{\bm{k}}
	=
	U_{\bm{k}}^{\dagger}
	\hat{\mathcal{E}}_{\bm{k}}
	U_{\bm{k}}
	\ .
\]

The $4 \times 4$ matrices on both sides are expressed as:

\[
	\hat{E}_{\bm{k}}
	=
	\left(
	\begin{array}{cccc}
			E_{\bm{k} +} & 0            & 0               & 0               \\
			0            & E_{\bm{k} -} & 0               & 0               \\
			0            & 0            & - E_{-\bm{k} +} & 0               \\
			0            & 0            & 0               & - E_{-\bm{k} -}
		\end{array}
	\right)
	\ , \]\[
	\hat{\mathcal{E}}_{\bm{k}}
	=
	\left(
	\begin{array}{cc}
			\varepsilon(\bm{k}) \hat{\sigma}_{0} & \hat{\Delta}(\bm{k})                   \\[2mm]
			- \hat{\Delta}^{*}(-\bm{k})          & - \varepsilon(\bm{k}) \hat{\sigma}_{0}
		\end{array}
	\right)
	\ .
\]

The components of $\hat{E}_{\bm{k}}$ correspond to the elementary excitation spectrum of the system.
$\hat{\mathcal{E}}_{\bm{k}}$ is the matrix representation of $\tilde{\mathcal{H}}$, $\hat{\sigma}_{0}$ is the $2 \times 2$ identity matrix, and $\hat{\Delta}(\bm{k})$ is the matrix defined in equation 2.2.


\subsection{Bogoliubov Transformation (Matrix Representation)}
To find the general form of the transformation matrix $U_{\bm{k}}^{\dagger}$, we need to consider $\hat{\Delta}(\bm{k})$.

From the fermionic anticommutation relations and equations 2.2 and 2.3, $\hat{\Delta}(\bm{k})$ obviously has the following symmetry:

\[
	\hat{\Delta}(\bm{k})
	=
	- \hat{\Delta}^{T}(-\bm{k})
	\ .
\]

For spin-singlet pairs, $\hat{\Delta}(\bm{k})$ must have even parity with respect to $\bm{k}$.
Therefore, $\hat{\Delta}(\bm{k})$ is an antisymmetric matrix and can be expressed using an even function of $\bm{k}$, $\psi(\bm{k})$, as follows:

\[
	\hat{\Delta}(\bm{k})
	=
	i \hat{\sigma}_{y} \psi(\bm{k})
	\ = \
	\left(
	\begin{array}{cc}
			0              & \psi(\bm{k}) \\
			- \psi(\bm{k}) & 0
		\end{array}
	\right)
	\ = \
	- \hat{\Delta}^{T}(\bm{k})
	\ .
\]

On the other hand, for spin-triplet pairs, $\hat{\Delta}(\bm{k})$ must have odd parity with respect to $\bm{k}$.
In this case, $\hat{\Delta}(\bm{k})$ is a symmetric matrix.
According to the 1963 paper by Balian and Werthamer, it can be written using a vector-valued odd function $\bm{d}(\bm{k})$ as a parameter:

\[
	\hat{\Delta}(\bm{k})
	=
	i \Big( \bm{d}(\bm{k}) \cdot \hat{\bm{\sigma}} \Big) \hat{\sigma}_{y}
	\ = \
	\left(
	\begin{array}{cc}
			- d_{x}(\bm{k}) + i d_{y}(\bm{k}) & d_{z}(\bm{k})                   \\
			d_{z}(\bm{k})                     & d_{x}(\bm{k}) + i d_{y}(\bm{k})
		\end{array}
	\right)
	\ = \
	\hat{\Delta}^{T}(\bm{k})
	\ .
\]

$\hat{\bm{\sigma}}$ is a vector with Pauli matrices as its components.

This matrix representation is quite practical.
Under spin rotation operations, $\bm{d}(\bm{k})$ behaves like a rotation of a 3D vector.
It also simplifies other transformations, as will be shown in Table 1 at the end of the chapter.

The matrix $\hat{\Delta}(\bm{k})$ can be classified into two patterns.
First, a matrix $\hat{\Delta}(\bm{k})$ is called \textbf{unitary} if the product $\hat{\Delta}(\bm{k}) \hat{\Delta}^{\dagger}(\bm{k})$ is proportional to the identity matrix $\hat{\sigma}_{0}$.
If not, the matrix $\hat{\Delta}(\bm{k})$ is called \textbf{non-unitary}.
This only occurs in the case of spin triplets and yields the following equality:

\[
	\hat{\Delta}(\bm{k}) \hat{\Delta}^{\dagger}(\bm{k})
	=
	|\bm{d}(\bm{k})|^{2} \hat{\sigma}_{0}
	\ + \
	\bm{q} \cdot \hat{\bm{\sigma}}
	\ .
\]

The vector $\bm{q}=i [ \bm{d}(\bm{k}) \times \bm{d}^{*}(\bm{k}) ]$ is non-zero only if $\bm{d}(\bm{k}) \neq \bm{d}^{*}(\bm{k})$, which is the case when not all components of the d-vector are real.

${}$

Looking ahead slightly, according to Table 1 at the end of the chapter, $\bm{d}(\bm{k})$ is not invariant under time-reversal operations.

$\bm{q}(\bm{k})$ is described by the mean value of the total spin of the paired state specified by $\bm{k}$, ${\rm tr}[\hat{\Delta}^{\dagger}(\bm{k}) \hat{\bm{\sigma}} \hat{\Delta}(\bm{k}) ]$.
This shows the intriguing fact that the average value of $\bm{q}(\bm{k})$ over the entire Fermi surface can be non-zero.

Typically, the average value over the entire Fermi surface should be zero.
A non-zero $\bm{q}(\bm{k})$ indicates that the strength of the interaction for electron pairs at a given momentum $\bm{k}$ differs depending on the spin direction.
In this case, time-reversal symmetry is clearly broken.
In the next section (Chapter 5), we will discuss the case where a non-zero $\bm{q}(\bm{k})$ causes the electron pairs to couple with a magnetic field.
Such effects are already known phenomena in non-unitary A-phase superfluid $^3$He.


The solution to equation 2.6 is divided into two cases depending on whether the matrix $\hat{\Delta}(\bm{k})$ is unitary or not.

When the matrix $\hat{\Delta}(\bm{k})$ is unitary, the elements of the Bogoliubov transformation's matrix representation become simple.

\[
	\hat{u}_{\bm{k}}
	=
	\dfrac{ [ E_{\bm{k}} + \varepsilon(\bm{k}) ] \hat{\sigma}_{0} }
	{ \left\{ [ E_{\bm{k}} + \varepsilon(\bm{k}) ]^{2} + \dfrac{1}{2} {\rm tr} \hat{\Delta} \hat{\Delta}^{\dagger}(\bm{k}) \right\}^{1/2} }
	\ \ ,
	\nonumber
\]\[
	\hat{v}_{\bm{k}}
	=
	\dfrac{ - \hat{\Delta}(\bm{k}) }
	{ \left\{ [ E_{\bm{k}} + \varepsilon(\bm{k}) ]^{2} + \dfrac{1}{2} {\rm tr} \hat{\Delta} \hat{\Delta}^{\dagger}(\bm{k}) \right\}^{1/2} }
	\ \ .
\]

In the unitary case, the energy spectrum of the elementary excitations $E_{\bm{k}}$ is doubly degenerate.

\[
	E_{\bm{k}+}
	\ = \
	E_{\bm{k}-}
	\ = \
	E_{\bm{k}}
	\ = \
	\left[ \varepsilon^{2}(\bm{k}) + \dfrac{1}{2} {\rm tr} \hat{\Delta} \hat{\Delta}^{\dagger}(\bm{k}) \right]^{1/2}
	\nonumber
\]

A momentum-dependent energy gap exists for the elementary excitations:
$$
	\left[ \dfrac{1}{2} {\rm tr} \hat{\Delta} \hat{\Delta}^{\dagger}(\bm{k}) \right]^{1/2}
$$

${}$

In the non-unitary case, the elements of the Bogoliubov transformation's matrix representation become more complex.

\[
	\hat{u}_{\bm{k}}
	\!\!
	=
	\!\!
	Q
	\left\{
	\left[
		\dfrac{ E_{\bm{k} + } + \varepsilon(\bm{k}) }
		{ E_{\bm{k} + } }
		\right]^{1/2}
	(
	| \bm{q} | \hat{\sigma}_{0} + \bm{q} \cdot \hat{\bm{\sigma}}
	)
	(
	\hat{\sigma}_{0} + \hat{\sigma}_{z}
	)
	+
	\left[
		\dfrac{ E_{\bm{k} - } + \varepsilon(\bm{k}) }
		{ E_{\bm{k} - } }
		\right]^{1/2}
	(
	| \bm{q} | \hat{\sigma}_{0} - \bm{q} \cdot \hat{\bm{\sigma}}
	)
	(
	\hat{\sigma}_{0} - \hat{\sigma}_{z}
	)
	\right\}
	\ \ ,
	\nonumber
\]\[
	\hat{v}_{\bm{k}}
	\!\!
	=
	\!\!
	- i Q
	\left\{
	\dfrac{1}{\sqrt{ E_{\bm{k} +} [ E_{\bm{k} +} + \varepsilon(\bm{k}) ] }}
	\left[
		| \bm{q} | \bm{d} - i ( \bm{d} \times \bm{q} )
		\right]
	\cdot
	\hat{\bm{\sigma}}
	\hat{\sigma}_{y}
	(
	\hat{\sigma}_{0} + \hat{\sigma}_{z}
	)
	+
	\dfrac{1}{\sqrt{ E_{\bm{k} -} [ E_{\bm{k} -} + \varepsilon(\bm{k}) ] }}
	\left[
		| \bm{q} | \bm{d} + i ( \bm{d} \times \bm{q} )
		\right]
	\cdot
	\hat{\bm{\sigma}}
	\hat{\sigma}_{y}
	(
	\hat{\sigma}_{0} - \hat{\sigma}_{z}
	)
	\right\}
	\ \ .
	\\
\]

Here, $Q$ and $E_{\bm{k} \pm}$ are shorthand notations for:

\[
	Q^{-2}
	=
	8 | \bm{q} | ( | \bm{q} | + q_{z} )
	\ \ ,
\]\[
	E_{\bm{k} \pm}
	=
	\sqrt{ \varepsilon^{2}(\bm{k}) + | \bm{d}(\bm{k}) |^{2} \pm | \bm{q}(\bm{k}) |^{2} }
	\ \ ,
	\nonumber
\]

In the non-unitary case, $\bm{q}$ is non-zero, and the energy spectrum that was doubly degenerate in the unitary state splits.

There are now two momentum-dependent energy gaps:

\[
	| \bm{d}(\bm{k}) |^{2} \pm | \bm{d}(\bm{k}) \times \bm{d}^{*}(\bm{k}) |
	\nonumber
\]

This lifting of the degeneracy in the energy spectrum can be understood as a consequence of broken time-reversal symmetry.
This is consistent with $\bm{q}$ being non-zero, which means the strength of the interaction for a Cooper pair depends on its spin orientation.
\subsection{Self-Consistent Gap Equation}

Using the fact that $\alpha_{\bm{k}s}^{(\dagger)}$ are fermions and the definition of the mean-field potential (equation 2.2), we can derive a self-consistent equation for the energy gap (the gap equation).

\[
	\Delta_{ss'}(\bm{k})
	=
	- \sum_{\bm{k}',s_{3},s_{4}}
	V_{s' s s_{3} s_{4}}
	(\bm{k},\bm{k}')
	\mathcal{F}_{s_{3} s_{4}}(\bm{k}',\beta)
	\ \ .
\]

The matrix $\hat{\mathcal{F}}(\bm{k},\beta)$ depends on whether the pair potential's matrix representation, $\hat{\Delta}(\bm{k})$, is unitary or non-unitary.
For the unitary case,

\[
	\hat{\mathcal{F}}(\bm{k},\beta)
	=
	\dfrac{ \hat{\Delta}(\bm{k}) }{ 2 E_{\bm{k}} }
	{\rm tanh}\dfrac{\beta E_{\bm{k}}}{2}
	\ \ ,
\]

and for the non-unitary case,

\[
	\hat{\mathcal{F}}(\bm{k},\beta)
	=
	\left[
		\dfrac{ 1 }{ 2 E_{\bm{k} +} }
		\left(
		\bm{d}(\bm{k})
		+
		\dfrac{\bm{q}(\bm{k}) \times \bm{d}(\bm{k})}{|\bm{q}(\bm{k})|}
		\right)
		{\rm tanh}\dfrac{\beta E_{\bm{k}+}}{2}
		\ + \
		\dfrac{ 1 }{ 2 E_{\bm{k} -} }
		\left(
		\bm{d}(\bm{k})
		-
		\dfrac{\bm{q}(\bm{k}) \times \bm{d}(\bm{k})}{|\bm{q}(\bm{k})|}
		\right)
		{\rm tanh}\dfrac{\beta E_{\bm{k}-}}{2}
		\right]
	i \hat{\bm{\sigma}} \hat{\sigma}_{y}
	\ \ .
\]

These self-consistent equations determine the temperature dependence of the pair potential's matrix representation $\hat{\Delta}(\bm{k})$ and the superconducting transition temperature $T_{c}$.

Near the transition temperature $T_{c}$, the magnitude of the gap is sufficiently small, which allows us to linearize and simplify the equation.

\[
	v \Delta_{s_{1} s_{2}}(\bm{k})
	=
	- \sum_{s_{3} s_{4}}
	\left\langle
	V_{ s_{2} s_{1} s_{3} s_{4} } (\bm{k} , \bm{k}')
	\Delta_{s_{3} s_{4}} (\bm{k}')
	\right\rangle_{\bm{k}'}
	\ \ .
\]

Here, the bracket notation with the momentum subscript $\langle \cdots \rangle_{\bm{k}}$ denotes an average over the entire Fermi surface.
In the weak-coupling limit, the constant $v$ can be approximated using the density of states at the Fermi surface, $N(0)$, as follows:

\[
	\dfrac{1}{v}
	=
	N(0)
	\int^{\varepsilon_{c}}_{0}
	\!\!\! d \varepsilon
	\dfrac{{\rm tanh}\dfrac{\beta_{c} \varepsilon(\bm{k})}{2}}{\varepsilon(\bm{k})}
	\ = \
	{\rm ln}(1.14 \beta_{c} \varepsilon_{c})
	\ \ .
\]


Equation (2.18) is the eigenvalue equation for the pair potential's matrix representation $\hat{\Delta}(\bm{k})$.

The largest eigenvalue $v$ defines the superconducting instability, the real $T_{c}$, and the nature of the state, $\hat{\Delta}(\bm{k})$.

To solve equation (2.18), we might naturally think we need the explicit expression for the pair-creation potential $\hat{V}$.

However, the next section will show that we can actually find the solution without knowing the precise form of $\hat{V}$.
In this case, group theory becomes an incredibly important tool.
\subsection{Symmetry of the Pair Potential}

The Hamiltonian described in equation (2.1) has symmetry.
Let's call the group whose elements are the operations that leave this symmetry invariant $\mathcal{G}$.

This group $\mathcal{G}$ is composed of subgroups: the point group of the crystal $G$, the 3D spin rotation symmetry SU(2), the time-reversal symmetry $\mathcal{K}$, and the gauge symmetry U(1).

If we write the elements of group $\mathcal{G}$ as $g$, then

\[
	g
	\mathcal{H}
	=
	\mathcal{H}
	\ \ ,
	\ \]\[
	g
	\ \in \
	\mathcal{G}
	=
	G \ \! \times \ \! {\rm SU(2)} \ \! \times \ \! \mathcal{K} \ \! \times \ \! {\rm U(1)}
	\ \ .
	\nonumber
\]

The same letter $g$ will be used for elements of other groups below, but the group it belongs to will be specified in the text.

To know how $\hat{\Delta}(\bm{k})$ transforms under a symmetry operation that is an element of $\mathcal{G}$, we need to know how $\tilde{\mathcal{H}}$ transforms.

An element $g \in G$ acts only on the vector $\bm{k}$.

\[
	g a_{\bm{k},s}^{\dagger}
	\ = \
	a_{ \hat{D}_{(G)}^{(-)}(g) \bm{k},s}^{\dagger}
	\ \ \longrightarrow \ \
	g \hat{\Delta}(\bm{k})
	\ = \
	\hat{\Delta}(\hat{D}_{(G)}^{(-)}(g) (\bm{k}))
\]

Here, the symbol $\hat{D}_{(G)}^{(-)}(g)$ is the 3D representation of the group $G$ in $\bm{k}$-space.

For the 3D spin rotation symmetry group SU(2), let's again write the matrix representation of a symmetry operation as $g$.
The transformation of $\hat{\Delta}(\bm{k})$ under SU(2) is:

\[
	g a_{\bm{k},s}^{\dagger}
	\ = \
	\sum_{s'}
	\hat{D}_{(S)} (g)_{s,s'}
	a_{\bm{k},s}^{\dagger}
	\ \ \longrightarrow \ \
	g \hat{\Delta}(\bm{k})
	\ = \
	\hat{D}_{(S)}^{T} (g) \ \!
	\hat{\Delta}(\bm{k}) \ \!
	\hat{D}_{(S)} (g)
\]

The $2 \times 2$ matrix $\hat{D}_{S}$ is the representation matrix of SU(2) in spin-1/2 space.
$\hat{D}^{T}$ represents the transpose of $\hat{D}$.
\ \\

\ \\

Work in Progress...

\begin{thebibliography}{9}
	\bibitem{SigristUeda1991} M. Sigrist and K. Ueda (1991) Rev. Mod. Phys.

\end{thebibliography}

\end{document}
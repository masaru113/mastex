\documentclass[uplatex,a4j,10.5pt,dvipdfmx]{jsarticle}
\usepackage{amsmath,amsthm,amssymb,bm,color,enumitem,mathrsfs,url,epic,eepic,ascmac,ulem,here,ascmac}
\usepackage[letterpaper,top=2cm,bottom=2cm,left=1.5cm,right=1.5cm,marginparwidth=1.75cm]{geometry}
\usepackage[dvipdfm]{graphicx}
\usepackage[hypertex]{hyperref}
\usepackage{array}
\usepackage{longtable}

\title{\sffamily \bfseries 職務経歴書}
\author{M. O.}
\date{2025年6月 現在}

\begin{document}

\maketitle

\section{職務要約}

\begin{longtable}{|c|p{14cm}|}
	\hline
	\multicolumn{1}{|c|}{\textbf{期間}} & \multicolumn{1}{c|}{\textbf{職務内容}} \\
	\hline
	\endhead

	\hline
	2014年4月 ~ 2015年3月                 & 東京大学 物性研究所 研究員(LASORセンター)          \\
	\hline
	2015年4月 ~ 2018年3月                 & 東京大学 日本学術振興会 特別研究員(DC1)            \\
	\hline
	2018年4月 ~ 2019年5月                 & コンサルティング会社 ミドルクオンツ・システム開発          \\
	\hline
	2019年6月 ~ 2024年2月                 & 大手証券会社 エクイティ・デリバティブ・クオンツ           \\
	\hline
	2024年3月 ~ 現在                      & メガバンク eFX(電子為替取引)クオンツ・マネージャー       \\
    \hline
    2022年6月 ~ 現在 & Quant Marketing Lab 代表(個人事業主) \\
     & (人材紹介業のDX・マーケティング基盤開発を統括) \\
    \hline
\end{longtable}

\section{テクニカルサマリー}
\begin{itemize}
	\item \textbf{プログラミング言語:} Python (Pandas, NumPy, scikit-learn), Java, SQL, VBA, shell script
	\item \textbf{プラットフォーム・ツール:} Tableau, Looker Studio, Git, JIRA, Docker
	\item \textbf{クラウド:} AWS (EC2, Lambda, Batch, S3), Google API (Gmail, Calendar, Gemini API), OpenAI API (GPT)
	\item \textbf{データベース:} Microsoft SQL Server, PostgreSQL
	\item \textbf{金融工学・機械学習:} モンテカルロ法, ブラック・ショールズモデル (および派生), SVM, クラスタリング分析
\end{itemize}

\section{職務詳細}

\subsection{2014年4月~2015年3月 東京大学物性研究所極限コヒーレント光科学研究(LASOR)センター 研究員 (東京大学大学院・教職員)}

\noindent\textbf{組織概要:} 極限コヒーレント光科学研究(LASOR)センターでは、超精密レーザーや極短パルス、大強度レーザーなどの極限的なレーザーを開発している。テラヘルツから真空紫外線、軟X線までの広いエネルギー範囲の極限的な光源を用いて、超高分解能光電子分光、時間分解分光、スピン偏極分光、顕微分光、回折や光散乱、イメージング、発光分光などの新しい最先端分光計測を開発している。一方、これらの極限的な光源や分光手法を用いて半導体、強相関物質、有機物質、表面、界面などの幅広い物性研究とその共同利用を行っている。

\noindent\textbf{担当業務:} 大強度レーザーを用いた真空紫外~軟X線レベルの極限コヒーレント光源によるフェムト秒時間分解および角度分解光電子分光実験を担当。量子場理論に基づく数値シミュレーションによる実験結果の検証を行った。

\noindent\textbf{成果:} 査読付きジャーナル投稿、査読付き国際会議発表

\newpage

\begin{longtable}{|l|p{14cm}|}
	\hline
	\multicolumn{1}{|c|}{\textbf{期間}} & \multicolumn{1}{c|}{\textbf{プロジェクト内容}}                                                                                                                                                                                                                                                                               \\
	\hline
	\endhead
	2014年4月~2015年3月                   & \textbf{■ 東京大学物性研究所 研究員}                                                                                                                                                                                                                                                                                             \\
	                                  & 【担当職務】実験結果の数値シミュレーションによる妥当性検証。超高真空および極限コヒーレント光の実験装置の新規開発。修士課程の理論物理学の研究を査読付き国際会議で発表。査読付き論文執筆。                                                                                                                                                                                                                         \\[15mm]
	                                  & 【実績】                                                                                                                                                                                                                                                                                                                 \\[-5mm]
	                                  & \begin{itemize}
		                                    \item 日本物理学会第69回年次大会における発表「磁場中角度分解電子ラマン散乱による超伝導ギャップの異方性の解析」 30aCA-8 (2014)
		                                    \item 査読付き国際会議(The international conference on strongly correlated electron systems)における発表 「Analysis of Magnetic Field-Angle Dependent Electronic Raman Scattering to Probe the Superconducting Gap」Contribution
		                                    \item (プレプリント)“Enhancement and termination of the superconducting proximity effect due to atomic-scale defects visualized by scanning tunneling microscopy”(実験家のアシスタントとして超伝導/金属界面における表面電子系の秩序状態の様子を数値シミュレーションし、実験結果をサポートした。)
		                                    \item 査読付きジャーナル発表論文(筆頭著者) “Analysis of Magnetic Field-Angle Dependent Electronic Raman Scattering to Probe the Superconducting Gap”JPS Conf. Proc. 3, 015045 (2014). (\url{https://journals.jps.jp/doi/10.7566/JPSCP.3.015045})(場の量子論を応用して磁場中の異方的超伝導体におけるラマン散乱の応答について新しい公式を導出した。その結果から実際に数値シミュレーションを行い、新しい実験手法を提案した。)
	                                    \end{itemize} \\
	\hline
\end{longtable}

\subsection{2015年4月~2018年3月 東京大学 研究員「独立行政法人日本学術振興会 特別研究員(DC1)」}

\noindent\textbf{概要:} 優れた若手研究者に、その研究生活の初期において、自由な発想のもとに主体的に研究課題等を選びながら研究に専念する機会を与え、研究者の養成・確保を図る制度。大学院博士課程在学者で、将来研究者となることを目指す者を「特別研究員-DC1」に採用し、3年間フェローシップが支給される。

\noindent\textbf{成果:} 査読付き国際ジャーナル投稿論文4本

\begin{longtable}{|c|p{14cm}|}
	\hline
	\multicolumn{1}{|c|}{\textbf{期間}} & \multicolumn{1}{c|}{\textbf{プロジェクト内容}}                                                                                                                                                                                    \\
	\hline
	\endhead

	\hline
	2015年4月~2017年3月                   & \textbf{■ Suppression of supercollision carrier cooling in high mobility graphene on $\mathrm{SiC}(0001)$}                                                                                                                \\
	                                  & 【担当職務】シリコンカーバイド薄膜$\mathrm{SiC}(0001)$上に蒸着させたグラフェンの2次元電子状態の非平衡緩和過程を観測した。フェムト秒($10^{-15}\mathrm{sec}$)で電子系を励起状態にし、ピコ秒($10^{-12}\mathrm{sec}$)で緩和する電子系の状態を時間分解及び角度分解光電子分光の手法を用いた。私はその主任実験遂行者としての役割を担った。また、理論計算によるモデルの検証も行った。 \\
	                                  & 【実績】査読付き国際ジャーナル発表論文 \textnormal{[Phys. Rev. B 95, 165303(1-7) (2017). Editors' Suggestion]} (\url{https://journals.aps.org/prb/abstract/10.1103/PhysRevB.95.165303})                                                      \\
	\hline
	2015年4月~2017年4月                   & \textbf{■ Ultrafast Melting of Spin Density Wave Order in $\mathrm{BaFe}_2\mathrm{As}_2$ Observed by Time- and Angle-Resolved Photoemission Spectroscopy with Extreme-Ultraviolet Higher Harmonic Generation}             \\
	                                  & 【担当職務】鉄系超伝導体$\mathrm{BaFe}_2\mathrm{As}_2$のスピン密度波状態におけるフェムト秒時間分解及び角度分解光電子分光の手法を用いた世界初の観測。私はその主任実験遂行者としての役割を担った。これまでの理論(数値計算結果)から予想されていた電子状態を実際に観測することに成功した。                                                               \\
	                                  & 【実績】査読付き国際ジャーナル発表論文 \textnormal{[Phys. Rev. B 95, 165112(1-6) (2017)]} (\url{https://journals.aps.org/prb/abstract/10.1103/PhysRevB.95.165112})                                                                           \\
	\hline
	2015年4月~2017年11月                  & \textbf{■ Femtosecond to picosecond transient effects in $\mathrm{WSe}_2$ observed by pump-probe angle-resolved photoemission spectroscopy}                                                                               \\
	                                  & 【担当職務】タングステンセレナイド$\mathrm{WSe}_2$はいわゆるトポロジカル絶縁体であり、非平衡電子緩和過程が通常の絶縁体とは異なり、レプリカバンド構造が現れることが理論的に予想されていた。この系の非平衡緩和過程を実際にフェムト秒時間分解及び角度分解光電子分光の手法を用いた観測を世界で初めて成功した。                                                             \\
	                                  & 【実績】査読付き国際ジャーナル発表論文 \textnormal{[Sci. Rep. 7, 15981(1-7) (2017)]} (\url{https://www.nature.com/articles/s41598-017-16076-z})                                                                                              \\
	\hline
	2015年4月~2018年3月                   & \textbf{■ Antiphase Fermi-surface modulations accompanying displacement excitation in a parent compound of iron-based superconductors}                                                                                    \\
	                                  & 【担当職務】鉄系超伝導体のブリルアンゾーン境界における電子面とホール面のそれぞれで、電子系の非平衡緩和過程に特徴的な差があることをフェムト秒時間分解及び角度分解光電子分光の手法を用いることで実験的に示した。私はその主任実験遂行者としての役割を担った。また、電子系の過渡状態における数値シミュレーションで実験結果の検証も行った。                                                       \\
	                                  & 【実績】査読付き国際ジャーナル発表論文 \textnormal{[Phys. Rev. B 97, 121107(R)(1-6) (2018) Rapid Communication]} (\url{https://journals.aps.org/prb/abstract/10.1103/PhysRevB.97.121107})                                                    \\
	\hline
\end{longtable}

\subsection{2018年4月~2019年5月 コンサルティング会社}

\noindent\textbf{会社概要:} 金融機関のフロント業務に関わるシステムのコンサルティング業務、システム開発業務を行う企業。

\noindent\textbf{担当部署:} 金融フロンティアディビジョン

\noindent\textbf{担当業務:} 国際金融規制(FRTB)対応

\begin{longtable}{|c|p{14cm}|}
	\hline
	\multicolumn{1}{|c|}{\textbf{期間}} & \multicolumn{1}{c|}{\textbf{プロジェクト内容}}                                                                                                         \\
	\hline
	\endhead

	\hline
	2018年4月~7月                        & \textbf{■ 研修期間}                                                                                                                                \\
	                                  & Pyhon、Java、shell script、SQLを用いた債券管理システムの構築。                                                                                                    \\
	\hline
	2018年4月~7月                        & 研修生代表として米UCLAに留学:UCLA(カルフォルニア大学ロサンゼルス校)においてFrancis A. Longstaff の下でMBAコースの金融工学クラス修了。                                                           \\
	\hline
	2018年8月~2019年5月                   & \textbf{■ 国内最大手銀行におけるFRTB規制対応システムの新規導入}                                                                                                        \\
	                                  & 【担当職務】システム開発テストツール開発(Java、VBA、Python、SQL、shell script)システム間I/F検証(Java、VBA、SQL、Javascript)数値検証(VBA、Python、SQL、shell script)データベース定義の管理(VBA、SQL) \\
	                                  & 【実績】テストケース打鍵数13万件以上。テスト自動化ツールの作成して効率化。成果は同期間に他の同僚の10倍以上。                                                                                       \\
	\hline
\end{longtable}

\subsection{2019年6月~2024年2月 大手証券会社}

\noindent\textbf{担当部署:} クオンツ部

\noindent\textbf{担当業務:} エクイティ・デリバティブ業務周り全般

\begin{longtable}{|c|p{14cm}|}
	\hline
	\multicolumn{1}{|c|}{\textbf{期間}} & \multicolumn{1}{c|}{\textbf{プロジェクト内容}}                                                                                                                                  \\
	\hline
	\endhead

	\hline

	2019年6月~現在                        & \textbf{■ 株系 仕組債(EB/ELB)取引}                                                                                                                                             \\
	                                  & 【担当職務】仕組債インディケーション自動化対応仕組債カバートレーディング自動化対応仕組債 社内オフィシャルBooking 対応                                                                                                         \\
	                                  & 【実績】インディケーション業務について、自社および他社から試算依頼されるフォーマットを整理して社内で一元的に管理できるようにした。カバートレードの試算依頼までトレーダーの手を介さずに自動で取引できるようにした。(従来、手動で取引ごとに5~10分かかっていたインディケーション提示を完全自動化し、提示可能件数を10倍以上に増加させた。) \\
	\hline
	2019年6月~2021年3月                   & \textbf{■ トータルリターンレートオブスワップ(TRS)取引}                                                                                                                                     \\
	                                  & 銀行とのTRS取引業務の自動化を行った。【担当職務】社内オフィシャル時価レポート自動化                                                                                                                             \\
	                                  & 【実績】ミドルクオンツ部署とポートフォリオトレーディング課の業務を自動化。                                                                                                                                   \\
	\hline
	2019年6月~現在                        & \textbf{■ ショートプットオプション(SPO)取引}                                                                                                                                          \\
	                                  & 【担当職務】エキゾチックオプション取引についての社内オフィシャル約定管理ツールの作成                                                                                                                              \\
	                                  & 【実績】これまで取引できなかったスキームの取引が可能になった。KnockOut/KnockIn付エイジアンオプション Knock Out リセット付きスキーム                                                                                         \\
	\hline
	2020年3月~2021年3月                   & \textbf{■ コロナ感染者数の予測と感染者数推移予測分析}                                                                                                                                        \\
	                                  & 機械学習(SVM)を用いて感染者数の予測を行い、クオンツアナリストの株価予測をサポートした。                                                                                                                          \\
	\hline
	2020年10月~2021年1月                  & \textbf{■ LIBOR廃止対応}                                                                                                                                                    \\
	                                  & $\cdot$ 既存のエキゾチックデリバティブの約定についてのロールフォワード対応と、新規約定についても新RFRで登録できるように切り替えの同時システム対応。                                                                                         \\
	\hline
	2021年4月~2023年3月                   & \textbf{■ プリペイドバリアブルフォワードの商品開発}                                                                                                                                         \\
	                                  & ソフトバンクGのアリババ株売却に伴い商品の設計、実装、約定管理まで行った。                                                                                                                                   \\
	                                  & 【実績】ソフトバンクGのアリババ株 72億ドル分の売却取引成功。 (\url{https://www.bloomberg.co.jp/news/articles/2023-04-12/RT0UTTDWRGG001})                                                            \\
	\hline
\end{longtable}

\subsection{2024年3月~現在 メガバンク}

\noindent\textbf{担当部署:} 為替トレーディング部

\noindent\textbf{担当業務:} eFXクオンツチームのマネージャーとして、電子為替取引(eFX)のプライシング業務、およびマーケティングによるセールスチームサポートを統括。

\begin{longtable}{|c|p{14cm}|}
	\hline
	\multicolumn{1}{|c|}{\textbf{期間}} & \multicolumn{1}{c|}{\textbf{プロジェクト内容}}                                                                       \\
	\hline
	\endhead

	\hline
	2024年3月~2024年6月                   & \textbf{■ eTrading業務効率化}                                                                                     \\
	                                  & 【担当職務】オペレーション負荷削減のため俗人的になっている定常作業を明確化して整理。                                                                   \\
	                                  & 【実績】オペレーションを標準化・自動化したことで、トレーディングデスクの24時間カバレッジを30 \% 少ない人員で達成可能にし、クオンツもトレーダーとしてシフトに入れる体制を構築した。                \\
	\hline
	2024年6月~現在                        & \textbf{■ スプレッド・キャリブレーション}                                                                                   \\
	                                  & 【担当職務】為替レートのbid/askスプレッドをマーケット適正水準を分析して求めるバッチの作成、自動化。(Dockerコンテナ化し、AWS BatchおよびLambdaを利用したサーバレス環境で実行。)       \\
	                                  & 【実績】前担当者が日次1.5時間をかけていた手作業のキャリブレーション業務を完全自動化し、月30時間分の工数を削減。定時実行により、東京だけでなくロンドン市場のプライシング精度も向上させた。              \\
	\hline
	2024年6月~現在                        & \textbf{■ 顧客動向分析}                                                                                            \\
	                                  & 【担当職務】セールスチームのサポートツールとして、顧客動向をクラスタリング分析を用いてカテゴリー分けし、取引量、取引時間帯、通貨ペアなどの観点から分析できるTableauダッシュボードの作成とそのメンテナンスを担当。 \\
	                                  & 【実績】トレーダーやセールスチームとのミーティングにおいて、ダッシュボードが頻繁に活用されるようになった。                                                        \\
	\hline
	2024年3月~現在                        & \textbf{■ レポート業務の自動化}                                                                                        \\
	                                  & 【担当職務】OpenAI APIを活用し、日次のルーチン分析と朝会発表資料の作成を自動化。                                                            \\
	                                  & 【実績】従来、人の手で日次30分(分析・資料作成)かかっていた作業をゼロ化。月間15~20時間の工数削減を達成。                                                     \\
	\hline
\end{longtable}



\subsection{2022年6月~現在 Quant Marketing Lab 代表(個人事業主)}
\noindent\textbf{概要:} メガバンクのクオンツ業務、MBA(在学中)と並行し、「Quant Marketing Lab」の代表(創設者 兼 開発責任者)として人材紹介業のDX・定量分析基盤の構築をリード。

\noindent\textbf{担当業務:} 定量分析業務全般のシステム化(企画、設計、開発、運用)。
\begin{longtable}{|c|p{14cm}|}
	\hline
	\multicolumn{1}{|c|}{\textbf{期間}} & \multicolumn{1}{c|}{\textbf{プロジェクト内容}} \\
	\hline
	\endhead
	\hline
	2022年6月~現在                        & \textbf{■ 人材紹介業向け 定量分析・自動化基盤の構築}       \\
	                                  & 【担当職務】
	\begin{itemize}[leftmargin=*,noitemsep]
		\item \textbf{定量分析・可視化:} 売上・進捗のリアルタイム可視化(PORTERSのDBミラーリング(PostgreSQL), Google Spread Sheet, Looker Studio連携)。キャンセル(辞退)率の要因分析と定量的施策の提案。
		\item \textbf{業務自動化:} Google API (Calendar, Gmail) を活用したスカウトメールの自動化。Gemini APIおよびOpenAI API利用した朝会レポート等の自動作成。
		\item \textbf{インフラ構築:} AWS (Lambda, Batch) 上でのPython分析バッチの自動実行環境を構築。
	\end{itemize}
	\\
	                                  & 【実績】
	\begin{itemize}[leftmargin=*,noitemsep]
		\item 担当したシステム基盤が事業の急成長(売上1億円 $\to$ 36億円)に寄与。
		\item スカウトメール送信業務の効率化(1000倍以上)を達成。
		\item レポート自動化により、日次30分のルーチン分析・報告業務をゼロ化(月間15時間以上の工数削減)。
		\item 自動化・システム化による高い生産性が評価され、スカウトメール部門が分社化(新規SES企業設立)するに至った。
	\end{itemize}
	\\
	\hline
\end{longtable}





\section{取得資格等}

\begin{longtable}{|c|l|}
	\hline
	\centering\textbf{期間} & \textbf{資格}                                     \\
	\hline
	\endhead

	\hline
	2008年5月               & 普通自動車免許                                         \\
	\hline
	2010年4月               & 実用数学技能検定(数検)(1級)                                \\
	\hline
	2012年2月               & 大型自動二輪免許                                        \\
	\hline
	2015年7月               & 国家公務員試験総合職区分(旧・国家I種)合格(採用候補者名簿登録)               \\
	\hline
	2016年7月               & 同上(連続合格・名簿更新)                                   \\
	\hline
	2018年4月               & ITパスポート(新卒研修)                                   \\
	\hline
	2018年4月               & TOEIC 765(新卒研修)                                 \\
	\hline
	2019年5月               & 証券外務員I種                                         \\
	\hline
	2019年7月               & 日本証券業協会 内部管理責任者                                 \\
	\hline
	2020年1月               & ディープラーニングG検定                                    \\
	\hline
	2021年10月              & ビジネス実務法務検定3級                                    \\
	\hline
	2022年2月               & ビジネス会計検定3級                                      \\
	\hline
	2022年10月              & ファイナンシャルプランナー3級                                 \\
	\hline
	2023年2月               & マーケティング検定3級                                     \\
	\hline
	2024年4月               & NFA Swaps Proficiency Requirements (Long Track) \\
	\hline
\end{longtable}

\section{学歴}
\begin{itemize}
	\item SBI大学院大学 経営管理研究科(MBA)在学中(オンライン)
	\item 東京大学大学院 理学系研究科 物理学専攻 博士課程(日本学術振興会 特別研究員DC1)
	\item 東京大学大学院 理学系研究科 物理学専攻 修士課程(首席卒業 / 48人中1位, 全取得単位「優」評価, GPA 4.0)
	\item UCLA Anderson School of Management(金融工学 MBAコース修了, 研修生代表)
	\item (下記「表彰・その他」の業績により国公立大学に早期特別合格(飛び入学))
\end{itemize}

\newpage

\section{表彰・その他}
\begin{itemize}[leftmargin=*]
	\item 日本数学コンクールで金賞受賞(読売新聞に掲載)。
	\item スーパーサイエンスハイスクールの全国大会は全学代表として3年間連続で出場。
\end{itemize}

\section{活かせる経験・知識・技術}

修士課程(首席卒業)では理論物理学、助手・博士課程では実験物理学を専攻し、物理学の深い専門性を有しています。

コンサルティング会社でシステムエンジニアからプロジェクトマネージャーまで経験し、国内大手証券会社ではトレーディング部門に隣接するクオンツとしても活動してきました。

国内最大手の銀行では、人的介入を伴う取引業務の効率化と完全自動化を推進し、クオンツだけでなくトレーディング部門やセールスチームにも貢献。現在、マネージャーとしてその取り組みをリードしています。

さらに、「Quant Marketing Lab」の代表としてゼロから事業会社のDXを推進し、売上を36倍にグロースさせた実績(システム開発・マーケティング自動化)も有しており、MBA(在学中)で学ぶ経営理論と実践を両立しています。

\vspace{1\baselineskip}
何卒よろしくお願い申し上げます。

\vspace{1\baselineskip}
\hfill 以上

\end{document}
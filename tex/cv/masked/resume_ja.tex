\documentclass[uplatex,a4j,10.5pt,dvipdfmx]{jsarticle}
\usepackage{amsmath,amsthm,amssymb,bm,color,enumitem,mathrsfs,url,epic,eepic,ascmac,ulem,here,ascmac}
\usepackage[letterpaper,top=2cm,bottom=2cm,left=1.5cm,right=1.5cm,marginparwidth=1.75cm]{geometry}
\usepackage[dvipdfm]{graphicx}
\usepackage[hypertex]{hyperref}
\usepackage{array}
\usepackage{longtable}

\title{\ \\[-20mm] \sffamily \bfseries 職務経歴書 (サマリー)}
\author{M. O.}
\date{2025年6月 現在}

\begin{document}

\maketitle

\section{職務要約}

\begin{longtable}{|c|p{14cm}|}
	\hline
	\multicolumn{1}{|c|}{\textbf{期間}} & \multicolumn{1}{c|}{\textbf{職務内容}} \\
	\hline
	\endhead

	\hline
	2014年4月 ~ 2015年3月                 & 東京大学 物性研究所 研究員(LASORセンター)          \\
	\hline
	2015年4月 ~ 2018年3月                 & 日本学術振興会 特別研究員(DC1)@ 東京大学大学院        \\
	\hline
	2018年4月 ~ 2019年5月                 & コンサルティング会社 ミドルクオンツ・システム開発          \\
	\hline
	2019年6月 ~ 2024年2月                 & 大手証券会社 エクイティ・デリバティブ・クオンツ           \\
	\hline
	2022年6月 ~ 現在                      & Quant Marketing Lab 代表             \\
	                                  & (人材紹介業のDX・マーケティング基盤開発を統括)          \\
	\hline
	2024年3月 ~ 現在                      & メガバンク eFX(電子為替取引)クオンツ・マネージャー       \\
	\hline
\end{longtable}

\section{テクニカルサマリー}
\begin{itemize}
	\item \textbf{プログラミング言語:} Python (Pandas, NumPy, scikit-learn), Java, SQL, VBA, shell script
	\item \textbf{プラットフォーム・ツール:} Tableau, Looker Studio, Git, JIRA, Docker
	\item \textbf{クラウド:} AWS (EC2, Lambda, Batch, S3, Bedrock), Google API (Gmail, Calendar, Gemini API), OpenAI API
	\item \textbf{データベース:} Microsoft SQL Server, PostgreSQL, kdb+ q database
	\item \textbf{金融工学:} デリバティブ評価 (バリア/エキゾチック・オプション/株系仕組債), 最小二乗法モンテカルロ (LSM), ブラック・ショールズモデル (派生含む), ポジション自動照合モデル
	\item \textbf{機械学習:} 深層学習 (PyTorch)、Transformerモデル派生 (BERT等)、時系列分析(異常検知、ARIMA、GARCH)、特徴量エンジニアリング(テクニカル指標開発)、SVM、クラスタリング分析
\end{itemize}

\section{主な職務経歴(概要)}
\begin{itemize}[leftmargin=*]
	\item \textbf{2024年3月~現在: メガバンク}
	      \begin{itemize}
		      \item eFX(電子為替取引)クオンツ・マネージャー
		      \item 担当業務: グローバル拠点(東京・ロンドン)へのマーケットメイク戦略展開、スプレッド最適化アルゴリズムの自動実行基盤(AWS/Docker)構築、生成AI(Bedrock/OpenAI API)を活用したリアルタイム約定レポートの自動作成、トキシックフロー対応リスク管理アルゴリズム開発(進行中)、プライシング業務及びセールスサポート統括(Tableau)
	      \end{itemize}
	\item \textbf{2022年6月~現在: Quant Marketing Lab}
	      \begin{itemize}
		      \item 代表(個人事業主)
		      \item 担当業務: 人材紹介業のDX・定量分析基盤構築(AI活用)に加え、PyTorchベースのRNN派生モデル(BERT等)を用いた高度なデータ分析・マーケティング予測モデルの開発と事業グロース支援
	      \end{itemize}
	\item \textbf{2019年6月~2024年2月: 大手証券会社}
	      \begin{itemize}
		      \item エクイティ・デリバティブ・クオンツ
		      \item 担当業務: エキゾチック・デリバティブ取引の約定管理システム統括(データ項目設計、時価管理含む)、LSM適用モデルを用いた仕組債取引・カバートレード自動化、フロント/ミドルポジション自動照合バッチ開発、システム管理(自動実行スケジューラー)
	      \end{itemize}
	\item \textbf{2018年4月~2019年5月: コンサルティング会社}
	      \begin{itemize}
		      \item ミドルクオンツ・システム開発
		      \item 担当業務: FRTB規制対応システム導入、テスト自動化
	      \end{itemize}
	\item \textbf{2015年4月~2018年3月: 東京大学}
	      \begin{itemize}
		      \item 日本学術振興会 特別研究員(DC1)
		      \item 成果: 査読付き国際ジャーナル投稿論文4本(筆頭著者含む)
	      \end{itemize}
	\item \textbf{2014年4月~2015年3月: 東京大学 物性研究所}
	      \begin{itemize}
		      \item 研究員(LASORセンター)
		      \item 担当業務: フェムト秒時間分解光電子分光実験、数値シミュレーション
	      \end{itemize}
\end{itemize}


\section{取得資格等}

\begin{longtable}{|c|l|}
	\hline
	\centering\textbf{期間} & \textbf{資格}                                     \\
	\hline
	\endhead

	\hline
	2008年5月               & 普通自動車免許                                         \\
	\hline
	2010年4月               & 実用数学技能検定(数検)(1級)                                \\
	\hline
	2012年2月               & 大型自動二輪免許                                        \\
	\hline
	2015年7月               & 国家公務員試験総合職区分(旧・国家I種)合格(採用候補者名簿登録)               \\
	\hline
	2016年7月               & 同上(連続合格・名簿更新)                                   \\
	\hline
	2018年4月               & ITパスポート(新卒研修)                                   \\
	\hline
	2018年4月               & TOEIC 765(新卒研修)                                 \\
	\hline
	2019年5月               & 証券外務員I種                                         \\
	\hline
	2019年7月               & 日本証券業協会 内部管理責任者                                 \\
	\hline
	2020年1月               & ディープラーニングG検定                                    \\
	\hline
	2021年10月              & ビジネス実務法務検定3級                                    \\
	\hline
	2022年2月               & ビジネス会計検定3級                                      \\
	\hline
	2022年10月              & ファイナンシャルプランナー3級                                 \\
	\hline
	2023年2月               & マーケティング検定3級                                     \\
	\hline
	2024年4月               & NFA Swaps Proficiency Requirements (Long Track) \\
	\hline
\end{longtable}

\section{学歴}
\begin{itemize}
	\item (2025年~現在): 経営管理研究科 アントレプレナー専攻(MBA)(在学中)
	\item (2018年~2018年): UCLA Anderson School of Management(金融工学 MBAコース修了, 研修生代表)
	\item (2015年~2018年): 東京大学大学院 博士課程(日本学術振興会 特別研究員DC1)
	\item (2012年~2014年): 東京大学大学院 修士課程
\end{itemize}

\end{document}

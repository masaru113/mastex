\documentclass[uplatex]{jsarticle}
\usepackage[english]{babel}
\usepackage[letterpaper,top=2cm,bottom=2cm,left=3cm,right=3cm,marginparwidth=1.75cm]{geometry}
\usepackage{amsmath, amssymb}
\usepackage{graphicx}
\usepackage{here}

\title{
\textbf{Arbitrage and Gauge Theory}
}

\author{
Masaru Okada
}

\begin{document}
\maketitle

\begin{abstract}
	I found a strange thought experiment about arbitrage, so I'm writing it down as a memo.
\end{abstract}

\section{\textbf{Denomination}}

Suppose a government performs a denomination, lowering the value of the current 10,000 yen note to 1,000 yen.

An orange that previously cost 10,000 yen for 100 pieces will now cost 1,000 yen for 100 pieces after the denomination.

To denominate money by a factor of $\alpha$ means to equate the value of one unit of pre-denomination money with $\alpha$ units of post-denomination money.

A denomination by a factor of $\alpha$ causes the price of goods to increase by a factor of $\alpha$.

A denomination by a factor of $\alpha$ causes the value of one unit of money to become $1/\alpha$.

In the example of lowering 10,000 yen to 1,000 yen, $\alpha=1/10$.



\section{\textbf{Impact of Denomination on Exchange Rates}}

Let $R_{ij}$ be the price of the currency of country $j$ in terms of one unit of the currency of country $i$.

When a country $i$ performs a currency denomination, the exchange rate changes to
$$
	R_{ij} \to R_{ij} / \alpha
$$
For the inverse transaction,
$$
	R_{ji} \to \alpha R_{ji}
$$
Consider the currencies of four countries, $i,j,k,l$, and trade them in the order of $i \to j \to k \to l \to i$.

In a transaction between $i$ and $j$, one unit of country $i$'s currency can be traded for $R_{ij}$.

In a transaction between $i,j,$ and $k$, one unit of country $i$'s currency can be traded for
$R_{ij} R_{jk}$.

In a transaction between $i,j,k,$ and $j$, one unit of country $i$'s currency can be traded for
$R_{ij} R_{jk} R_{kl}$.

Finally, in a transaction between $i,j,k,j,$ and $i$, one unit of country $i$'s currency can be traded for
$R_{ij} R_{jk} R_{kl} R_{li}$.

The profit margin for a transaction in the order $i \to j \to k \to l \to i$ is written as
$M_{ijkl} = R_{ij} R_{jk} R_{kl} R_{li}$, which serves as an index for arbitrage.

Here, arbitrage refers to a transaction where, by owning cash in a certain currency A and successively converting it into currencies B, C, D, ... according to the exchange rates at a specific time, and finally converting it back to currency A, the final amount is greater than or less than the initial amount.

If $M_{ijkl}=1$, no arbitrage opportunity exists.
If $M_{ijkl} \neq 1$, an arbitrage opportunity exists.

When country $i$ denominates its own currency by a factor of $\alpha$, the affected rates are
$R_{ij} \to R_{ij} / \alpha , \ R_{li} \to \alpha R_{li}$.
However, the profit margin for the currency transaction is
$$
	M_{ijkl} \to (R_{ij} / \alpha) \times R_{jk} R_{kl} \times (\alpha R_{li}) = M_{ijkl}
$$
and is therefore unaffected.

While the absolute value of a currency can be set arbitrarily by each country, it can be seen that the profit margin from exchange transactions is independent of denomination.



\section{\textbf{Lattice Model Setup}}

Imagine a grid laid over a globe, with each country arranged on a lattice.

The world is a $d+1$ dimensional space, and countries exist on the lattice points.

The coordinates of a lattice point are represented by $n = (t, n_{1},n_{2}, \cdots, n_{d})$, where $t$ is time.

The unit vector in the $\mu$ direction is denoted by $e_{\mu}$.

$C(n)$ represents the currency of the country located at lattice point $n$.

A quantity $R_{\mu}(n)$ is defined as $C(n) / C(n + e_{\mu})$, which represents the exchange rate of the currency of the country at $n + e_{\mu}$ relative to the country at $n$.

Furthermore, a quantity $A_{\mu}(n) = \log (R_{\mu}(n))$ is defined.

Also,
$$
	M_{\mu \nu}(n)
	=
	R_{\mu}(n) R_{\nu}(n+e_{\mu})
	/ R_{\mu}(n+e_{\nu})
	/ R_{\nu}(n)
$$
is defined.
$M_{\mu \nu}(n)$ is the profit margin when trading in the order of $\mu \to \nu \to \mu$.

Furthermore, a quantity $F_{\mu \nu}(n) = \log (M_{\mu \nu}(n))$ is defined.



\section{\textbf{Continuum Limit}}

Until now, we have not considered the distance between countries. However, let's consider what happens when we take the limit as the spacing between lattice points, say $a$, approaches 0 in every direction.
$$
	F_{\mu \nu}(n) =
	A_{\mu}(n) + A_{\nu}(n+e_{\mu})
	- A_{\mu}(n+e_{\nu})
	- A_{\nu}(n)
$$
In this limit, each lattice point becomes dense and continuous, like an ordinary spacetime point $(x_{0},x_{1},x_{2},\cdots,x_{d})$.
In this limit, the lattice point coordinate that was written as $n$ is now written as $x$.

In this case,
$
	A_{\mu}(n) - A_{\mu}(n+e_{\nu})
$
and
$
	A_{\nu}(n+e_{\mu}) - A_{\nu}(n)
$
become increasingly small quantities.

In this limit, by Taylor expanding $A_{\mu}(n)$, and writing the partial derivative in the $\mu$ direction as $\partial_{\mu}$, we get
$$
	A_{\mu}(n) - A_{\mu}(n+e_{\nu})
	\ \to \
	- a \partial_{\nu} A_{\mu} (x)
$$
$$
	A_{\nu}(n+e_{\mu}) - A_{\nu}(n)
	\ \to \
	a \partial_{\mu} A_{\nu} (x)
$$
Using this,
$$
	F_{\mu \nu}(n)
	\ \to \
	F_{\mu \nu}(x)
	=
	\partial_{\mu} A_{\nu} (x)
	-
	\partial_{\nu} A_{\mu} (x)
$$
Here, the lattice point spacing $a$ before taking the continuum limit is an arbitrary constant, so it was set to 1.

In this case,
$A_{\mu} (x)$
represents the "logarithm of the exchange rate between the currency of the country at coordinate $x$ and the currency of the country at a tiny distance away in the $\mu$ direction," and
$F_{\mu} (x)$
represents the "logarithm of the profit margin when trading between two countries a tiny distance apart relative to the currency of the country at coordinate $x$ in the order of $\mu \to \nu \to \mu$."



\section{\textbf{Local Denomination Invariance}}

When the country at $x$ performs a denomination by a factor of $\Omega(x) = \log (\theta (x))$, the exchange rate of the country at $x$ is affected as follows:
$$
	R_{\mu}(x) \ \to \ \Omega(x + e_{\mu}) R_{\mu}(x) / \Omega(x)
$$
In this case,
$$
	\exp (A_{\mu}(x))
	\ \to \
	\exp (A_{\mu}(x) + \theta(x+e_{\mu}) - \theta(x))
$$
$$
	\to \ \exp (A_{\mu}(x) + \partial_{\mu} \theta (x))
$$
On the other hand, $F_{\mu \nu}$ is unaffected by local denomination.
$$
	F_{\mu \nu}
	=
	\partial_{\mu} A_{\nu}
	-
	\partial_{\nu} A_{\mu}
$$
$$
	\to \
	\partial_{\mu} ( A_{\nu} + \partial_{\nu} \theta )
	-
	\partial_{\nu} ( A_{\mu} + \partial_{\mu} \theta )
$$
$$
	= \
	\partial_{\mu} A_{\nu}
	-
	\partial_{\nu} A_{\mu}
$$
$$
	= \
	F_{\mu \nu}
$$
While the exchange rates themselves are affected by the denomination $\theta(x)$, the profits from currency exchange are not.

Recalling that the transactions that actually affect profits are arbitrage transactions, the transactions affected by $\theta(x)$ are non-arbitrage transactions.



\section{\textbf{Correspondence to the Picture of Gauge Theory in Physics}}

Not limited to denomination, in general, a change in exchange rates that does not fundamentally affect value may keep $F_{\mu \nu}$ invariant. This idea is similar to gauge theory in physics.

When arbitrage occurs, a flow of value is generated in that transaction loop.

And through non-arbitrage transactions, the purchased currency instantly becomes more expensive, and the sold currency becomes cheaper. This non-arbitrage transaction creates a distortion in value, but in a perfect market, the arbitrage opportunity is immediately corrected.

This picture is as if a distortion $F_{\mu \nu}$ is created in the field $A_{\mu}$, a force acts on the value, and value flows due to arbitrage.

In this picture, the flow of the matter field (current) in gauge theory seems to correspond to the flow of value.

\begin{table}[H]
	\centering
	\begin{tabular}{|c|c|}
		\hline
		\textbf{Arbitrage} & \textbf{Physics} \\ \hline \hline
		Log of the exchange rate $A_{\mu}$ & Gauge field \\ \hline
		Log of the arbitrage index $F_{\mu \nu}$ & Field strength \\ \hline
		Log of the denomination ratio $\theta(x)$ & Gauge transformation function \\ \hline
		Flow of "value" due to arbitrage opportunity & Current \\ \hline
		Exchange rate $R_{\mu}$ & Lattice link variable \\ \hline
		Arbitrage index $M_{\mu \nu}$ & Lattice plaquette \\ \hline
	\end{tabular}
	\caption{Correspondence table to the picture of gauge theory in physics}
	\label{tab:tab202502281751}
\end{table}


\section{\textbf{Things I'd like to write about if I feel like it}}

If we restrict the gauge symmetry to $U(1)$, we can make it correspond to condensed matter physics based on electromagnetism.

I am curious what economic interpretations can be given to phenomena in condensed matter physics.

For example, I wonder what economic interpretations can be given to anomalies in gauge theory (e.g., superconductivity) within this framework.

\begin{thebibliography}{9}

	\bibitem{Rudin:Schwichtenberg}
	J.~Schwichtenberg.
	\newblock Demystifying Gauge Symmetry
	\newblock arXiv:1901.10420v1 (2019).

	\bibitem{Rudin:Ilinski}
	K.~Ilinski.
	\newblock Physics of Finance.
	\newblock hep-th/9710148 (1997).

	\bibitem{Rudin:Young}
	K.~Young.
	\newblock Foreign exchange market as a lattice gauge theory.
	\newblock American Journal of Physics 67, 862 (1999).

	\bibitem{Rudin:IlinskiKalinin}
	K.~Ilinski., G.~Kalinin.
	\newblock Black-Scholes equation from Gauge Theory of Arbitrage
	\newblock hep-th/9712034 (1997).

\end{thebibliography}

\end{document}
\documentclass[uplatex,a4j,12pt,dvipdfmx]{jsarticle}
\usepackage[english]{babel}
\usepackage[letterpaper,top=2cm,bottom=2cm,left=3cm,right=3cm,marginparwidth=1.75cm]{geometry}
\usepackage{amsmath}
\usepackage{amssymb}
\usepackage{amsthm}
\usepackage{graphicx}
\usepackage{hyperref}
\usepackage{enumitem}

\title{
Stochastic Volatility and Local Volatility
}
\author{Masaru Okada}
\date{\today}

\begin{document}

\maketitle

\section{Stochastic Volatility}

\subsection{Derivation of the Equation}

Following the arguments of Wilmott(2000), let's proceed.

Assume that the stock price $S_{t}$ and its variance $\nu_{t}$ at time $t$ follow the following equations:

\begin{eqnarray*}
	dS_{t}
	&=&
	\mu_{t} S_{t} dt
	\ + \
	\sqrt{\nu_{t}} S_{t} dZ_{1}
	\\
	d \nu_{t}
	&=&
	\alpha(S_{t},\nu_{t},t) dt
	\ + \
	\eta \beta(S_{t},\nu_{t},t) \sqrt{\nu_{t}} dZ_{2}
	\\
	\langle
	dZ_{1} dZ_{2}
	\rangle
	&=&
	\rho dt
\end{eqnarray*}

Here, $\mu_{t}$ is a deterministic function, representing the instantaneous drift of the stock's return.

$\eta$ represents the volatility of volatility.

$dZ_{1}$ and $dZ_{2}$ are Wiener processes, and $\rho$ is the correlation between the stock's return and the change in $\nu_{t}$.

The first equation was assumed by Black and Scholes (1973).

In fact, Wilmott(2000) shows in Section 8.3 that this system of equations becomes the Black-Scholes model if we take the limit $\eta \to 0$ in the second equation.

\ \\

Assuming the variance $\nu_{t}$ follows this second equation allows for a very general discussion.

For now, the functional forms of $\alpha$ and $\beta$ are not determined.

Furthermore, we haven't specified the process that $\sqrt{\nu_{t}}$ follows (e.g., assuming it's a Wiener process).

\ \\

When constructing a risk-free portfolio, the source of randomness in the Black-Scholes framework was solely the stock price, so only the stock itself was needed for hedging.

In this case, Wilmott(2000)'s equations show that we must also hedge the volatility.

\ \\

Consider a portfolio $\Pi$ that includes an asset with value $V(S,\nu,t)$. Let the holdings of stock $S$ be $(-\Delta)$ and the holdings of a volatility-dependent asset $V_{1}$ be $(-\Delta_{1})$.

\begin{eqnarray*}
	\Pi
	&=&
	V - \Delta S - \Delta_{1} V_{1}
\end{eqnarray*}

We want to find the change in the portfolio's value, $d\Pi$, over a small time interval $dt$. Using Itô's lemma, we get:

\begin{eqnarray*}
	d \Pi
	&=&
	dV - d(\Delta S) - d(\Delta_{1} V_{1})
	\\ &=&
	dV - \Delta dS - \Delta_{1} dV_{1}
\end{eqnarray*}

Let's expand each term.

First, expanding $dS^{2}$, $d\nu^{2}$, and $dSd\nu$:

\begin{eqnarray*}
	dS^{2}
	&=&
	(\sqrt{\nu} S dZ_{1} + \mu S dt)^{2}
	\\ &=&
	\nu S^{2} dZ_{1}^{2}
	+
	2 \mu S \sqrt{\nu} S dt dZ_{1}
	+
	\mu^{2} S^{2} dt^{2}
	\\ &=&
	\nu S^{2} dt
\end{eqnarray*}

We use $dZ_{1}^{2} = dt$, as higher-order infinitesimals like $dZ_{1}dt$ and $dt^{2}$ are of order $o(dt)$ and therefore vanish. Similarly,

\begin{eqnarray*}
	d \nu^{2}
	&=&
	( \alpha dt + \eta \beta \sqrt{\nu} dZ_{2} )^{2}
	\\ &=&
	\eta^{2} \nu \beta^{2} dt
	\\[3mm]
	dS d \nu
	&=&
	(\sqrt{\nu} S dZ_{1} + \mu S dt)
	( \alpha dt + \eta \beta \sqrt{\nu} dZ_{2} )
	\\ &=&
	\rho \eta \nu \beta dt
\end{eqnarray*}

From these, we can express $dV$ as:

\begin{eqnarray*}
	dV
	\!\!\!\!
	&=&
	\dfrac{\partial V}{\partial t}
	dt
	+
	\dfrac{\partial V}{\partial S}
	dS
	+
	\dfrac{\partial V}{\partial \nu}
	d\nu
	\\ && \ + \
	\dfrac{1}{2}
	\dfrac{\partial^{2} V}{\partial S^{2}}
	dS^{2}
	+
	\dfrac{\partial^{2} V}{\partial S \partial \nu}
	dSd\nu
	+
	\dfrac{1}{2}
	\dfrac{\partial^{2} V}{\partial \nu^{2}}
	d\nu^{2}
	\\ &=&
	\dfrac{\partial V}{\partial t}
	dt
	+
	\dfrac{\partial V}{\partial S}
	dS
	+
	\dfrac{\partial V}{\partial \nu}
	d\nu
	\\ && \ + \
	\dfrac{1}{2}
	\nu S^{2}
	\dfrac{\partial^{2} V}{\partial S^{2}}
	dt
	+
	\rho \eta \nu \beta
	\dfrac{\partial^{2} V}{\partial S \partial \nu}
	dt
	+
	\dfrac{1}{2}
	\eta^{2} \nu \beta^{2}
	\dfrac{\partial^{2} V}{\partial \nu^{2}}
	dt
	\\ &&
	\hspace{-7.7mm}
	\begin{array}{l}
		= \
		\left(
		\dfrac{\partial V}{\partial t}
		+
		\dfrac{1}{2}
		\nu S^{2}
		\dfrac{\partial^{2} V}{\partial S^{2}}
		\right.
		\\ \hspace{10mm}
		\left.
		+
		\rho \eta \nu \beta
		\dfrac{\partial^{2} V}{\partial S \partial \nu}
		+
		\dfrac{1}{2}
		\eta^{2} \nu \beta^{2}
		\dfrac{\partial^{2} V}{\partial \nu^{2}}
		\right)dt
	\end{array}
	\\ && \ + \
	\dfrac{\partial V}{\partial S}
	dS
	+
	\dfrac{\partial V}{\partial \nu}
	d\nu
\end{eqnarray*}

Similarly, for $V_{1}=V_{1}(S,\nu,t)$, we get:

\begin{eqnarray*}
	&&
	\begin{array}{l}
		dV_{1} \ = \
		\left(
		\dfrac{\partial V_{1}}{\partial t}
		+
		\dfrac{1}{2}
		\nu S^{2}
		\dfrac{\partial^{2} V_{1}}{\partial S^{2}}
		\right.
		\\ \hspace{10mm}
		\left.
		+
		\rho \eta \nu \beta
		\dfrac{\partial^{2} V_{1}}{\partial S \partial \nu}
		+
		\dfrac{1}{2}
		\eta^{2} \nu \beta^{2}
		\dfrac{\partial^{2} V_{1}}{\partial \nu^{2}}
		\right)dt
	\end{array}
	\\ && \hspace{10mm} + \
	\dfrac{\partial V_{1}}{\partial S}
	dS
	+
	\dfrac{\partial V_{1}}{\partial \nu}
	d\nu
\end{eqnarray*}

Substituting these into the portfolio's stochastic differential, we have:

\begin{eqnarray*}
	&&
	dV - \Delta dS - \Delta_{1} dV_{1}
	\\[2mm]
	&&
	\hspace{-10mm}
	\begin{array}{l}
		= \
		\left(
		\dfrac{\partial V}{\partial t}
		+
		\dfrac{1}{2}
		\nu S^{2}
		\dfrac{\partial^{2} V}{\partial S^{2}}
		\right.
		\\ \hspace{10mm}
		\left.
		+
		\rho \eta \nu \beta
		\dfrac{\partial^{2} V}{\partial S \partial \nu}
		+
		\dfrac{1}{2}
		\eta^{2} \nu \beta^{2}
		\dfrac{\partial^{2} V}{\partial \nu^{2}}
		\right)dt
	\end{array}
	\\
	&&
	\hspace{-5mm}
	+
	\dfrac{\partial V}{\partial S}
	dS
	+
	\dfrac{\partial V}{\partial \nu}
	d\nu
	\\
	&&
	\hspace{-5mm}
	-
	\Delta dS
	\\
	&&
	\hspace{-6.6mm}
	\begin{array}{l}
		- \
		\Delta_{1}
		\left(
		\dfrac{\partial V_{1}}{\partial t}
		+
		\dfrac{1}{2}
		\nu S^{2}
		\dfrac{\partial^{2} V_{1}}{\partial S^{2}}
		\right.
		\\ \hspace{10mm}
		\left.
		+
		\rho \eta \nu \beta
		\dfrac{\partial^{2} V_{1}}{\partial S \partial \nu}
		+
		\dfrac{1}{2}
		\eta^{2} \nu \beta^{2}
		\dfrac{\partial^{2} V_{1}}{\partial \nu^{2}}
		\right)dt
	\end{array}
	\\
	&&
	\hspace{-5mm}
	-
	\Delta_{1}
	\dfrac{\partial V_{1}}{\partial S}
	dS
	-
	\Delta_{1}
	\dfrac{\partial V_{1}}{\partial \nu}
	d\nu
\end{eqnarray*}

Combining the terms:

\begin{eqnarray*}
	&&
	\begin{array}{l}
		d \Pi \ = \
		\left(
		\dfrac{\partial V}{\partial t}
		+
		\dfrac{1}{2}
		\nu S^{2}
		\dfrac{\partial^{2} V}{\partial S^{2}}
		\right.
		\\ \hspace{10mm}
		\left.
		+
		\rho \eta \nu \beta
		\dfrac{\partial^{2} V}{\partial S \partial \nu}
		+
		\dfrac{1}{2}
		\eta^{2} \nu \beta^{2}
		\dfrac{\partial^{2} V}{\partial \nu^{2}}
		\right)dt
	\end{array}
	\\ && \hspace{10mm}
	\begin{array}{l}
		- \
		\Delta_{1}
		\left(
		\dfrac{\partial V_{1}}{\partial t}
		+
		\dfrac{1}{2}
		\nu S^{2}
		\dfrac{\partial^{2} V_{1}}{\partial S^{2}}
		\right.
		\\ \hspace{10mm}
		\left.
		+
		\rho \eta \nu \beta
		\dfrac{\partial^{2} V_{1}}{\partial S \partial \nu}
		+
		\dfrac{1}{2}
		\eta^{2} \nu \beta^{2}
		\dfrac{\partial^{2} V_{1}}{\partial \nu^{2}}
		\right)dt
	\end{array}
	\\ &&
	\hspace{10mm} +
	\left(
	\dfrac{\partial V}{\partial S}
	-
	\Delta
	-
	\Delta_{1}
	\dfrac{\partial V_{1}}{\partial S}
	\right)
	dS
	\\ &&
	\hspace{10mm} +
	\left(
	\dfrac{\partial V}{\partial \nu}
	-
	\Delta_{1}
	\dfrac{\partial V_{1}}{\partial \nu}
	\right)
	d\nu
\end{eqnarray*}

We now hedge this portfolio to make it instantaneously risk-free. To eliminate the $dS$ and $d\nu$ terms, we choose the following constraints:

\begin{eqnarray*}
	\dfrac{\partial V}{\partial S}
	-
	\Delta
	-
	\Delta_{1}
	\dfrac{\partial V_{1}}{\partial S}
	&=&
	0
	\\
	\dfrac{\partial V}{\partial \nu}
	-
	\Delta_{1}
	\dfrac{\partial V_{1}}{\partial \nu}
	&=&
	0
\end{eqnarray*}

Solving these yields the hedge quantities we need:

\begin{eqnarray*}
	\Delta
	&=&
	\dfrac{\partial V}{\partial S}
	\ - \
	\dfrac{
		\dfrac{\partial V}{\partial \nu}
		\dfrac{\partial V_{1}}{\partial S}
	}{
		\dfrac{\partial V_{1}}{\partial \nu}
	}
	\\
	\Delta_{1}
	&=&
	\dfrac{ \ \
		\dfrac{\partial V}{\partial \nu}
		\
	}{ \ \
		\dfrac{\partial V_{1}}{\partial \nu}
		\
	}
\end{eqnarray*}

Under these conditions, the portfolio can be expressed using the risk-free rate $r$:

\begin{eqnarray*}
	d \Pi
	&=&
	r \Pi dt
	\\[1mm] &&
	\hspace{-10mm}
	= r (V - \Delta S - \Delta_{1} V_{1}) dt
	\\ &&
	\hspace{-10mm}
	= r \left\{
	V -
	\left(
	\dfrac{\partial V}{\partial S}
	\ - \
	\dfrac{
		\dfrac{\partial V}{\partial \nu}
		\dfrac{\partial V_{1}}{\partial S}
	}{
		\dfrac{\partial V_{1}}{\partial \nu}
	}
	\right)
	S
	-
	\left(
	\dfrac{ \
		\dfrac{\partial V}{\partial \nu}
		\
	}{ \ \
		\dfrac{\partial V_{1}}{\partial \nu}
		\ \
	}
	\right)
	V_{1}
	\right\} dt
\end{eqnarray*}

At the same time, we have:

\begin{eqnarray*}
	&&
	\hspace{-5mm}
	d \Pi
	\\[3mm] &&
	\hspace{-10mm}
	=
	\left(
	\dfrac{\partial V}{\partial t}
	+
	\dfrac{1}{2}
	\nu S^{2}
	\dfrac{\partial^{2} V}{\partial S^{2}}
	+
	\rho \eta \nu \beta
	\dfrac{\partial^{2} V}{\partial S \partial \nu}
	+
	\dfrac{1}{2}
	\eta^{2} \nu \beta^{2}
	\dfrac{\partial^{2} V}{\partial \nu^{2}}
	\right)dt
	\\ && \hspace{-10mm} -
	\Delta_{1}
	\!\!
	\left(
	\dfrac{\partial V_{1}}{\partial t}
	+
	\dfrac{1}{2}
	\nu S^{2}
	\dfrac{\partial^{2} V_{1}}{\partial S^{2}}
	+
	\rho \eta \nu \beta
	\dfrac{\partial^{2} V_{1}}{\partial S \partial \nu}
	+
	\dfrac{1}{2}
	\eta^{2} \nu \beta^{2}
	\dfrac{\partial^{2} V_{1}}{\partial \nu^{2}}
	\right)
	\!
	dt
	\\[3mm] &&
	\hspace{-10mm}
	=
	\left(
	\dfrac{\partial V}{\partial t}
	+
	\dfrac{1}{2}
	\nu S^{2}
	\dfrac{\partial^{2} V}{\partial S^{2}}
	+
	\rho \eta \nu \beta
	\dfrac{\partial^{2} V}{\partial S \partial \nu}
	+
	\dfrac{1}{2}
	\eta^{2} \nu \beta^{2}
	\dfrac{\partial^{2} V}{\partial \nu^{2}}
	\right)dt
	\\ &-&
	\!\!\!\!
	\left(
	\dfrac{ \
		\dfrac{\partial V}{\partial \nu}
		\
	}{ \ \
		\dfrac{\partial V_{1}}{\partial \nu}
		\ \
	}
	\right)
	\\ &&
	\hspace{-5mm}
	\times
	\left(
	\dfrac{\partial V_{1}}{\partial t}
	+
	\dfrac{1}{2}
	\nu S^{2}
	\dfrac{\partial^{2} V_{1}}{\partial S^{2}}
	+
	\rho \eta \nu \beta
	\dfrac{\partial^{2} V_{1}}{\partial S \partial \nu}
	+
	\dfrac{1}{2}
	\eta^{2} \nu \beta^{2}
	\dfrac{\partial^{2} V_{1}}{\partial \nu^{2}}
	\right)
	\!
	dt
\end{eqnarray*}

Since these expressions for $d\Pi$ are equal, we can rearrange the terms by moving the terms related to $V$ to the left side and those related to $V_{1}$ to the right side:

\begin{eqnarray*}
	&&
	\dfrac{
		\left.
		\begin{array}{l}
			\dfrac{\partial V}{\partial t}
			+
			\dfrac{1}{2}
			\nu S^{2}
			\dfrac{\partial^{2} V}{\partial S^{2}}
			+
			\rho \eta \nu \beta
			\dfrac{\partial^{2} V}{\partial S \partial \nu}
			\\ \hspace{10mm}
			+
			\dfrac{1}{2}
			\eta^{2} \nu \beta^{2}
			\dfrac{\partial^{2} V}{\partial \nu^{2}}
			+
			rS \dfrac{\partial V}{\partial S}
			-rV
		\end{array}
		\right.
	}{
		\dfrac{\partial V}{\partial \nu}
	}
	\\ &=&
	\dfrac{
		\left.
		\begin{array}{l}
			\dfrac{\partial V_{1}}{\partial t}
			+
			\dfrac{1}{2}
			\nu S^{2}
			\dfrac{\partial^{2} V_{1}}{\partial S^{2}}
			+
			\rho \eta \nu \beta
			\dfrac{\partial^{2} V_{1}}{\partial S \partial \nu}
			\\ \hspace{10mm}
			+
			\dfrac{1}{2}
			\eta^{2} \nu \beta^{2}
			\dfrac{\partial^{2} V_{1}}{\partial \nu^{2}}
			+
			rS \dfrac{\partial V_{1}}{\partial S}
			-rV_{1}
		\end{array}
		\right.
	}{
		\dfrac{\partial V_{1}}{\partial \nu}
	}
\end{eqnarray*}

The left side of the equation depends only on $V$, and the right side only on $V_{1}$. For this equation to hold identically, both sides must equal an arbitrary function that is independent of $S, \nu,$ and $t$. Without loss of generality, we can set this function equal to $- (\alpha - \phi \beta \sqrt{\nu})$. Thus, we have:

\begin{eqnarray*}
	\begin{array}{l}
		\dfrac{\partial V}{\partial t}
		+
		\dfrac{1}{2}
		\nu S^{2}
		\dfrac{\partial^{2} V}{\partial S^{2}}
		+
		\rho \eta \nu \beta
		\dfrac{\partial^{2} V}{\partial S \partial \nu}
		\\[3mm] \hspace{10mm}
		+
		\dfrac{1}{2}
		\eta^{2} \nu \beta^{2}
		\dfrac{\partial^{2} V}{\partial \nu^{2}}
		+
		rS \dfrac{\partial V}{\partial S}
		-rV
		\ = \
		- (\alpha - \phi \beta \sqrt{\nu})
		\dfrac{\partial V}{\partial \nu}
	\end{array}
\end{eqnarray*}

Here, $\phi = \phi(S,\nu,t)$ is called the \textbf{market price of volatility risk}.

$\\[-15mm]$

\subsection{Market Price of Volatility Risk}

Let's examine why Wilmott's discussion gives $\phi$ the name 'market price of volatility risk'.

Consider a portfolio $\Pi_{1}$ composed of $V$, which is delta-hedged but not vega-hedged:

\begin{eqnarray*}
	\Pi_{1}
	&=&
	V
	-
	\dfrac{\partial V}{\partial S} S
\end{eqnarray*}

Applying Itô's lemma just as before, we find the change in the portfolio's value, $d\Pi_{1}$:

\begin{eqnarray*}
	&&
	d\Pi_{1}
	\\[3mm] &=&
	dV
	-
	S d \left( \dfrac{\partial V}{\partial S} \right)
	-
	\dfrac{\partial V}{\partial S} dS
	\\ &=& \!\!\!
	\left\{
	\left(
	\dfrac{\partial V}{\partial t}
	+
	\dfrac{1}{2}
	\nu S^{2}
	\dfrac{\partial^{2} V}{\partial S^{2}}
	+
	\rho \eta \nu \beta
	\dfrac{\partial^{2} V}{\partial S \partial \nu}
	+
	\dfrac{1}{2}
	\eta^{2} \nu \beta^{2}
	\dfrac{\partial^{2} V}{\partial \nu^{2}}
	\right)dt
	\right.
	\\ && +
	\left.
	\dfrac{\partial V}{\partial S} dS
	+
	\dfrac{\partial V}{\partial \nu} d \nu
	\right\}
	\\ && -
	\left(
	\Delta dS
	-
	\dfrac{\partial V}{\partial S} dS
	\right)
	\ \
	-
	\dfrac{\partial V}{\partial S} dS
	\\ &=&
	\left(
	\dfrac{\partial V}{\partial t}
	+
	\dfrac{1}{2}
	\nu S^{2}
	\dfrac{\partial^{2} V}{\partial S^{2}}
	+
	\rho \eta \nu \beta
	\dfrac{\partial^{2} V}{\partial S \partial \nu}
	+
	\dfrac{1}{2}
	\eta^{2} \nu \beta^{2}
	\dfrac{\partial^{2} V}{\partial \nu^{2}}
	\right)dt
	\\ && +
	\left(
	\dfrac{\partial V}{\partial S}
	-
	\Delta
	\right)
	dS
	+
	\dfrac{\partial V}{\partial \nu} d \nu
\end{eqnarray*}

\footnote{
	A point of confusion. The textbook seems to handle the product rule for $S d \left( \frac{\partial V}{\partial S} \right)$ in a non-standard way. I can't figure out the exact reasoning behind the sudden appearance of the $\Delta dS$ term. To follow the text's flow for now, it's assumed that $S d \left( \frac{\partial V}{\partial S} \right) = \Delta dS - \frac{\partial V}{\partial S} dS$.
}

Since this portfolio is delta-hedged, the $dS$ term should vanish, meaning $ \Delta dS = \frac{\partial V}{\partial S} dS $. This simplifies the expression to:

\begin{eqnarray*}
	&&
	d\Pi_{1}
	\\[3mm] &=&
	\left(
	\dfrac{\partial V}{\partial t}
	+
	\dfrac{1}{2}
	\nu S^{2}
	\dfrac{\partial^{2} V}{\partial S^{2}}
	+
	\rho \eta \nu \beta
	\dfrac{\partial^{2} V}{\partial S \partial \nu}
	+
	\dfrac{1}{2}
	\eta^{2} \nu \beta^{2}
	\dfrac{\partial^{2} V}{\partial \nu^{2}}
	\right)dt
	\\ &&
	+
	\dfrac{\partial V}{\partial \nu} d \nu
\end{eqnarray*}

Now, let's consider the difference in price between this portfolio, which is only delta-hedged (and therefore not vega-hedged), and a fully hedged risk-free portfolio, whose price change is $r \Pi_{1} dt$:

\begin{eqnarray*}
	&&
	d\Pi_{1} - r \Pi_{1} dt
	\\[3mm] &=&
	\left(
	\dfrac{\partial V}{\partial t}
	+
	\dfrac{1}{2}
	\nu S^{2}
	\dfrac{\partial^{2} V}{\partial S^{2}}
	+
	\rho \eta \nu \beta
	\dfrac{\partial^{2} V}{\partial S \partial \nu}
	+
	\dfrac{1}{2}
	\eta^{2} \nu \beta^{2}
	\dfrac{\partial^{2} V}{\partial \nu^{2}}
	\right)dt
	\\ &&
	+
	\dfrac{\partial V}{\partial \nu} d \nu
	\\ && -
	\left(
	rV
	-
	r
	\dfrac{\partial V}{\partial S} S
	\right)
	dt
	\\ &&
	\begin{array}{l}
		\hspace{-7mm}
		= \
		\left(
		\dfrac{\partial V}{\partial t}
		+
		\dfrac{1}{2}
		\nu S^{2}
		\dfrac{\partial^{2} V}{\partial S^{2}}
		+
		\rho \eta \nu \beta
		\dfrac{\partial^{2} V}{\partial S \partial \nu}
		\right.
		\\
		\left.
		+
		\dfrac{1}{2}
		\eta^{2} \nu \beta^{2}
		\dfrac{\partial^{2} V}{\partial \nu^{2}}
		-
		rV
		+
		r
		\dfrac{\partial V}{\partial S} S
		\right)dt
	\end{array}
	\\ &&
	+
	\dfrac{\partial V}{\partial \nu} d \nu
	\\ &=&
	-
	( \alpha - \phi \beta \sqrt{\nu} )
	\dfrac{\partial V}{\partial \nu}
	dt
	\ \
	+
	\dfrac{\partial V}{\partial \nu} d \nu
\end{eqnarray*}

\footnote{The textbook has an incorrect sign for the $rS\dfrac{\partial V}{\partial S}$ term in the equation between the second and third equality signs in my notes. }

Using the second equation from the fundamental set, $ d \nu = \alpha dt + \eta \beta \sqrt{\nu} dZ_{2} $, we can substitute it into the expression:

\begin{eqnarray*}
	&&
	d\Pi_{1} - r \Pi_{1} dt
	\\ &=&
	-
	( \alpha - \phi \beta \sqrt{\nu} )
	\dfrac{\partial V}{\partial \nu}
	dt
	\ + \
	\dfrac{\partial V}{\partial \nu}
	( \alpha dt + \eta \beta \sqrt{\nu} dZ_{2} )
	\\ &=&
	\beta \sqrt{\nu}
	\dfrac{\partial V}{\partial \nu}
	(\phi dt + \eta dZ_{2})
\end{eqnarray*}

Recalling the Capital Asset Pricing Model (CAPM):

$ \ \ $
Expected Return $ \ = \ $ Risk-Free Rate $ \ + \ $ Risk Premium

This leads to the following equation:

$$
	d\Pi_{1}
	\ = \
	r \Pi_{1} dt
	\ + \
	\beta \sqrt{\nu}
	\dfrac{\partial V}{\partial \nu}
	(\phi dt + \eta dZ_{2})
$$

By analogy with CAPM, $\phi(S,\nu,t)$ is called the \textbf{market price of volatility risk}.

\ \\

The risk-neutral drift is defined as:

$$
	\alpha'
	\ = \
	\alpha - \beta \sqrt{\nu} \phi
$$

(Work in progress ......).

\if0

	$\\$

	With this definition, the price difference between the stochastic differential of the portfolio and the risk-free portfolio is:


	\begin{eqnarray*}
		d\Pi_{1} - r \Pi_{1} dt
		&=&
		-
		( \alpha - \phi \beta \sqrt{\nu} )
		\dfrac{\partial V}{\partial \nu}
		dt
		\ \
		+
		\dfrac{\partial V}{\partial \nu} d \nu
		\\ &=&
		\dfrac{\partial V}{\partial \nu}
		(
		-\tilde{\alpha} dt
		+
		d \nu
		)
	\end{eqnarray*}


	At the same time, by analogy with CAPM,


	$ \ \ $
	Expected Return $ \ - \ $ Risk-Free Rate
	$ \ = \ $
	Risk Premium
	and we had


	\begin{eqnarray*}
		d\Pi_{1}
		-
		r \Pi_{1} dt
		&=&
		\beta \sqrt{\nu}
		\dfrac{\partial V}{\partial \nu}
		(\phi dt + \eta dZ_{2})
	\end{eqnarray*}


	Equating the right-hand sides, we get:


	\begin{eqnarray*}
		\dfrac{\partial V}{\partial \nu}
		(
		-\tilde{\alpha} dt
		+
		d \nu
		)
		&=&
		\beta \sqrt{\nu}
		\dfrac{\partial V}{\partial \nu}
		(\phi dt + \eta dZ_{2})
		\\[3mm]
		d \nu
		&=&
		(
		\tilde{\alpha}
		+
		\beta \sqrt{\nu}
		)
		\phi dt
		+
		\beta \sqrt{\nu}
		\eta dZ_{2}
	\end{eqnarray*}



	$\\$



	\begin{eqnarray*}
		d \nu_{t}
		&=&
		\alpha dt
		\ + \
		\eta \beta \sqrt{\nu_{t}} dZ_{2}
		\\ &=&
		(\alpha' + \beta \sqrt{\nu} \phi)
		dt
		\ + \
		\eta \beta \sqrt{\nu_{t}} dZ_{2}
	\end{eqnarray*}






	\begin{eqnarray*}
		\beta \sqrt{\nu}
		\dfrac{\partial V}{\partial \nu}
		(\phi dt + \eta dZ_{2})
		&=&
		\dfrac{\partial V}{\partial \nu}
		((\alpha' - \alpha) dt + \beta \sqrt{\nu} \eta dZ_{2})
	\end{eqnarray*}



	\footnote{I need to keep in mind why this definition makes it risk-neutral as I continue reading. }

\fi

\begin{thebibliography}{9}

	\bibitem{Rudin:Schwichtenberg}
	Jim Gatheral
	\newblock (2006)
	\newblock The Volatility Surface: A Practitioner's Guide (Wiley Finance)

\end{thebibliography}


\end{document}
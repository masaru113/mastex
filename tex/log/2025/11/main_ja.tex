\documentclass[dvipdfmx, autodetect-engine, aspectratio=169, 10.5pt]{beamer}

\usepackage{amsmath}
\usepackage{amssymb}
\usepackage{amsthm}
\usepackage{graphicx}
\usepackage{hyperref}
\usepackage{enumitem}
\usepackage[english]{babel}

\usetheme{Boadilla}
\usetheme{Marburg}
\usecolortheme{orchid}
\usefonttheme{professionalfonts}

\title{
Log.
}

\author{
M. O.
}

\date{Nov. 2025}

\begin{document}
\maketitle

\section{2025年11月1日(土)} 
\begin{frame}{2025年11月1日(土)}  
	\tiny  

	\textbf{Fact(事実)}  
	\begin{itemize}
		\item 朝から職務経歴書を作成し、Geminiと協働して高品質な成果物を完成させた。  
		\item 午後は書店を訪れ、生成AI関連のライフハック本を探したが、求める内容のものは見つからなかった。  
		\item 夜は友人と渋谷で南インド料理を食べた。ハロウィンの時期と重なり、街は人で溢れていた。  
		\item 新たに個人事業に「Quant Marketing Lab」という屋号を命名した。  
	\end{itemize}

	\textbf{Event(出来事・印象)}  
	\begin{itemize}
		\item 生成AIとの共同作業に強い感動を覚え、その可能性を実感した。  
		\item 渋谷の人混みの中を歩くのは体力的に非常に疲れたが、目的の店に辿り着いて食事を楽しめたときには安堵した。  
		\item 友人と本の話をしながら帰路につき、読書への意欲が再び高まった。  
	\end{itemize}

	\textbf{Reflection(内省・分析)}  
	\begin{itemize}
		\item AIを使いこなしたいという欲求が日に日に強まっており、焦りにも似た感情が芽生えている。  
		\item 都会の喧騒や人混みには消耗を感じ、自分は静かな環境でこそ力を発揮できるタイプだと再確認した。  
		\item AIの進化速度を考えると、今の学習がどれほど長期的に有効かという不安もある。  
	\end{itemize}

	\textbf{Insight(発見)}  
	\begin{itemize}
		\item 生成AIは単なるツールではなく、思考の質を高める創造的なパートナーである。  
		\item 学びは「成果を生むための訓練」ではなく「変化に適応するための連続的な更新」であることを実感した。  
		\item 「Quant Marketing Lab」という屋号の誕生が、自分の知的活動に一貫した方向性を与えてくれた。  
	\end{itemize}

	\textbf{Next Step(次への展開)}  
	\begin{itemize}
		\item 生成AIとの実践的な協働を半年以上継続し、操作スキルと発想力を磨く。  
		\item 明日は読書デーとし、『幼年期の終り』や手元の生成AI関連書籍を読み進める。  
		\item 静かに集中できる時間と空間を確保し、自分の思索の深さを取り戻す。  
	\end{itemize}
\end{frame}

\end{document}
\documentclass[dvipdfmx, autodetect-engine, aspectratio=169, 10.5pt]{beamer}

\usepackage{amsmath}
\usepackage{amssymb}
\usepackage{amsthm}
\usepackage{graphicx}
\usepackage{hyperref}
\usepackage{enumitem}
\usepackage[english]{babel}

\usetheme{Boadilla}
\usetheme{Marburg}
\usecolortheme{orchid}
\usefonttheme{professionalfonts}

\title{
Log.
}

\author{
M. O.
}

\date{Nov. 2025}

\begin{document}
\maketitle

\section{2025年11月1日(土)}
\begin{frame}{2025年11月1日(土)}
	\tiny

	\textbf{Fact(事実)}
	\begin{itemize}
		\item 朝から職務経歴書を作成し、Geminiと協働して高品質な成果物を完成させた。
		\item 午後は書店を訪れ、生成AI関連のライフハック本を探したが、求める内容のものは見つからなかった。
		\item 夜は友人と渋谷で南インド料理を食べた。ハロウィンの時期と重なり、街は人で溢れていた。
		\item 新たに個人事業に「Quant Marketing Lab」という屋号を命名した。
	\end{itemize}

	\textbf{Event(出来事・印象)}
	\begin{itemize}
		\item 生成AIとの共同作業に強い感動を覚え、その可能性を実感した。
		\item 渋谷の人混みの中を歩くのは体力的に非常に疲れたが、目的の店に辿り着いて食事を楽しめたときには安堵した。
		\item 友人と本の話をしながら帰路につき、読書への意欲が再び高まった。
	\end{itemize}

	\textbf{Reflection(内省・分析)}
	\begin{itemize}
		\item AIを使いこなしたいという欲求が日に日に強まっており、焦りにも似た感情が芽生えている。
		\item 都会の喧騒や人混みには消耗を感じ、自分は静かな環境でこそ力を発揮できるタイプだと再確認した。
		\item AIの進化速度を考えると、今の学習がどれほど長期的に有効かという不安もある。
	\end{itemize}

	\textbf{Insight(発見)}
	\begin{itemize}
		\item 生成AIは単なるツールではなく、思考の質を高める創造的なパートナーである。
		\item 学びは「成果を生むための訓練」ではなく「変化に適応するための連続的な更新」であることを実感した。
		\item 「Quant Marketing Lab」という屋号の誕生が、自分の知的活動に一貫した方向性を与えてくれた。
	\end{itemize}

	\textbf{Next Step(次への展開)}
	\begin{itemize}
		\item 生成AIとの実践的な協働を半年以上継続し、操作スキルと発想力を磨く。
		\item 明日は読書デーとし、『幼年期の終り』や手元の生成AI関連書籍を読み進める。
		\item 静かに集中できる時間と空間を確保し、自分の思索の深さを取り戻す。
	\end{itemize}
\end{frame}

\section{2025年11月2日(日)}
\begin{frame}{2025年11月2日(日)}
	\tiny

	\textbf{Fact(事実)}
	\begin{itemize}
		\item 一日中、自宅で作業に集中していた。
		\item MBA講義の講義録生成を自動化するスクリプトを開発。
		\item OpenAI Whisperで音声を文字起こしし、プロンプトテンプレートからLaTeX形式のノートを自動生成する仕組みを構築。
		\item 昼から夜まで約12時間、休みなしで作業に没頭した。
		\item 朝は家族と恒例の朝マックを楽しみ、2歳の娘がパソコンに興味を示していた。
	\end{itemize}

	\textbf{Event(出来事・印象)}
	\begin{itemize}
		\item スクリプトが想定以上にうまく動き、作業の自動化と高品質な出力に感動した。
		\item 外からの指示ではなく、自分の意志で効率を突き詰める過程に充実感を覚えた。
		\item 娘がパソコンを触ろうとする姿がかわいらしく、忙しい中にも温かい時間があった。
	\end{itemize}

	\textbf{Reflection(内省・分析)}
	\begin{itemize}
		\item 一度集中すると止まらず、体力の限界まで作業してしまう傾向を再確認した。
		\item AIを使いこなすには、単なる操作ではなく「いかに良質なデータと指示を与えるか」が核心であると実感した。
		\item 作業の効率化が進む一方で、自分自身の創造的思考の使い方も問われていると感じた。
	\end{itemize}

	\textbf{Insight(発見)}
	\begin{itemize}
		\item AIと人間は異なる役割を担いながらも、補い合う関係にあることを体感した。
		\item 「AIとの共生」とは、人間がいかに明確な意図と構造を設計できるかという挑戦でもある。
		\item 技術と家庭生活が自然に共存できる日常の中に、これからの働き方のヒントがあると感じた。
	\end{itemize}

	\textbf{Next Step(次への展開)}
	\begin{itemize}
		\item 生成AIとの協働範囲をさらに広げるため、利活用に関する専門書を読み進める。
		\item 長時間作業でも意識的に休憩を取り、集中と回復のリズムを整える。
		\item 今回の仕組みを発展させ、学びや業務に再利用できる汎用的なワークフローへと昇華させたい。
	\end{itemize}
\end{frame}


\section{2025年11月3日(月)}
\begin{frame}{2025年11月3日(月)}
	\tiny

	\textbf{Fact(事実)}
	\begin{itemize}
		\item 一日を通して静かに忙しく、MBAの講義ノートを生成AIとともに作成した。
		\item データの取り方や指示文の作成方法が洗練され、効率的に学習が進められる感覚を得た。
		\item 講義ノートを会社でも共有し、学びをチーム全体で高めたいと考えた。
	\end{itemize}

	\textbf{Event(出来事・印象)}
	\begin{itemize}
		\item 上手い指示出しができた瞬間に達成感を感じた。
		\item 学びの過程が自分の中で少しずつ体系化されていく充実感があった。
		\item 誰かに見られているほうが継続のモチベーションになり、質を保てるという実感を得た。
	\end{itemize}

	\textbf{Reflection(内省・分析)}
	\begin{itemize}
		\item クイズ形式で出力してもらうことで、受動的な講義の内容が能動的な問題意識に変わると気づいた。
		\item 「学ぶ」ことを仕組みとして設計することで、集中と理解が深まる自分の特性を再認識した。
	\end{itemize}

	\textbf{Insight(発見)}
	\begin{itemize}
		\item 生成AIへの的確な指示は、そのまま自分の思考整理の訓練にもなる。
		\item AIを通じて学習構造を外化することで、思考を効率的に鍛えられることを体感した。
	\end{itemize}

	\textbf{Next Step(次への展開)}
	\begin{itemize}
		\item 生成AIの応用的な使い方をさらに学び、より深いレベルで活用できるようにしたい。
		\item MBAの学びだけでなく、生成AIそのものの学習にも時間を投資し、知的生産性を高めていく。
	\end{itemize}
\end{frame}


\end{document}
\documentclass[dvipdfmx, autodetect-engine, aspectratio=169, 10.5pt]{beamer}

\usepackage{amsmath}
\usepackage{amssymb}
\usepackage{amsthm}
\usepackage{graphicx}
\usepackage{hyperref}
\usepackage{enumitem}
\usepackage[english]{babel}

\usetheme{Boadilla}
\usetheme{Marburg}
\usecolortheme{orchid}
\usefonttheme{professionalfonts}

\title{
Log.
}

\author{
Author
}

\date{Nov. 2025}

\begin{document}
\maketitle

\section{2025年11月1日(土)}
\begin{frame}{2025年11月1日(土)}
	\tiny

	\textbf{Fact(事実)}
	\begin{itemize}
		\item 朝から職務経歴書を作成し、Geminiと協働して高品質な成果物を完成させた。
		\item 午後は書店を訪れ、生成AI関連のライフハック本を探したが、求める内容のものは見つからなかった。
		\item 夜は友人と都内の繁華街で南インド料理を食べた。ハロウィンの時期と重なり、街は人で溢れていた。
		\item 新たに個人事業に「Quant Marketing Lab」という屋号を命名した。
	\end{itemize}

	\textbf{Event(出来事・印象)}
	\begin{itemize}
		\item 生成AIとの共同作業に強い感動を覚え、その可能性を実感した。
		\item 都内の繁華街の人混みの中を歩くのは体力的に非常に疲れたが、目的の店に辿り着いて食事を楽しめたときには安堵した。
		\item 友人と本の話をしながら帰路につき、読書への意欲が再び高まった。
	\end{itemize}

	\textbf{Reflection(内省・分析)}
	\begin{itemize}
		\item AIを使いこなしたいという欲求が日に日に強まっており、焦りにも似た感情が芽生えている。
		\item 都会の喧騒や人混みには消耗を感じ、自分は静かな環境でこそ力を発揮できるタイプだと再確認した。
		\item AIの進化速度を考えると、今の学習がどれほど長期的に有効かという不安もある。
	\end{itemize}

	\textbf{Insight(発見)}
	\begin{itemize}
		\item 生成AIは単なるツールではなく、思考の質を高める創造的なパートナーである。
		\item 学びは「成果を生むための訓練」ではなく「変化に適応するための連続的な更新」であることを実感した。
		\item 「Quant Marketing Lab」という屋号の誕生が、自分の知的活動に一貫した方向性を与えてくれた。
	\end{itemize}

	\textbf{Next Step(次への展開)}
	\begin{itemize}
		\item 生成AIとの実践的な協働を半年以上継続し、操作スキルと発想力を磨く。
		\item 明日は読書デーとし、『幼年期の終り』や手元の生成AI関連書籍を読み進める。
		\item 静かに集中できる時間と空間を確保し、自分の思索の深さを取り戻す。
	\end{itemize}
\end{frame}

\section{2025年11月2日(日)}
\begin{frame}{2025年11月2日(日)}
	\tiny

	\textbf{Fact(事実)}
	\begin{itemize}
		\item 一日中、自宅で作業に集中していた。
		\item MBA講義の講義録生成を自動化するスクリプトを開発。
		\item OpenAI Whisperで音声を文字起こしし、プロンプトテンプレートからLaTeX形式のノートを自動生成する仕組みを構築。
		\item 昼から夜まで約12時間、休みなしで作業に没頭した。
		\item 朝は家族と恒例のファストフード店の朝食を楽しみ、2歳の娘がパソコンに興味を示していた。
	\end{itemize}

	\textbf{Event(出来事・印象)}
	\begin{itemize}
		\item スクリプトが想定以上にうまく動き、作業の自動化と高品質な出力に感動した。
		\item 外からの指示ではなく、自分の意志で効率を突き詰める過程に充実感を覚えた。
		\item 娘がパソコンを触ろうとする姿がかわいらしく、忙しい中にも温かい時間があった。
	\end{itemize}

	\textbf{Reflection(内省・分析)}
	\begin{itemize}
		\item 一度集中すると止まらず、体力の限界まで作業してしまう傾向を再確認した。
		\item AIを使いこなすには、単なる操作ではなく「いかに良質なデータと指示を与えるか」が核心であると実感した。
		\item 作業の効率化が進む一方で、自分自身の創造的思考の使い方も問われていると感じた。
	\end{itemize}

	\textbf{Insight(発見)}
	\begin{itemize}
		\item AIと人間は異なる役割を担いながらも、補い合う関係にあることを体感した。
		\item 「AIとの共生」とは、人間がいかに明確な意図と構造を設計できるかという挑戦でもある。
		\item 技術と家庭生活が自然に共存できる日常の中に、これからの働き方のヒントがあると感じた。
	\end{itemize}

	\textbf{Next Step(次への展開)}
	\begin{itemize}
		\item 生成AIとの協働範囲をさらに広げるため、利活用に関する専門書を読み進める。
		\item 長時間作業でも意識的に休憩を取り、集中と回復のリズムを整える。
		\item 今回の仕組みを発展させ、学びや業務に再利用できる汎用的なワークフローへと昇華させたい。
	\end{itemize}
\end{frame}


\section{2025年11月3日(月)}
\begin{frame}{2025年11月3日(月)}
	\tiny

	\textbf{Fact(事実)}
	\begin{itemize}
		\item 一日を通して静かに忙しく、MBAの講義ノートを生成AIとともに作成した。
		\item データの取り方や指示文の作成方法が洗練され、効率的に学習が進められる感覚を得た。
		\item 講義ノートを会社でも共有し、学びをチーム全体で高めたいと考えた。
	\end{itemize}

	\textbf{Event(出来事・印象)}
	\begin{itemize}
		\item 上手い指示出しができた瞬間に達成感を感じた。
		\item 学びの過程が自分の中で少しずつ体系化されていく充実感があった。
		\item 誰かに見られているほうが継続のモチベーションになり、質を保てるという実感を得た。
	\end{itemize}

	\textbf{Reflection(内省・分析)}
	\begin{itemize}
		\item クイズ形式で出力してもらうことで、受動的な講義の内容が能動的な問題意識に変わると気づいた。
		\item 「学ぶ」ことを仕組みとして設計することで、集中と理解が深まる自分の特性を再認識した。
	\end{itemize}

	\textbf{Insight(発見)}
	\begin{itemize}
		\item 生成AIへの的確な指示は、そのまま自分の思考整理の訓練にもなる。
		\item AIを通じて学習構造を外化することで、思考を効率的に鍛えられることを体感した。
	\end{itemize}

	\textbf{Next Step(次への展開)}
	\begin{itemize}
		\item 生成AIの応用的な使い方をさらに学び、より深いレベルで活用できるようにしたい。
		\item MBAの学びだけでなく、生成AIそのものの学習にも時間を投資し、知的生産性を高めていく。
	\end{itemize}
\end{frame}

\section{2025年11月4日(火)}
\begin{frame}{2025年11月4日(火)}
	\tiny

	\textbf{Fact(事実)}
	\begin{itemize}
		\item 夜遅かったため、朝はゆっくり起床し約8時間の睡眠をとった。
		\item ブランチに妻が買ってきてくれたパン屋のパンを食べた。2日連続だったが美味しかった。
		\item 12:30からマーケティングチームとのミーティングを実施。
		\item MBAの講義動画を整理し、講義ノートを作成した。
		\item 開発作業ではリアルタイム更新されるデータのヒストリカルテーブルを構築。AWS LambdaやEventBridgeにも触れた。
		\item 夜はシュンペーターの著作を読み進め、「創造的破壊」と「新結合」について考察した。
		\item 夕食に蒸し鶏を作って食べ、21時頃に就寝した。
	\end{itemize}

	\textbf{Event(出来事・印象)}
	\begin{itemize}
		\item 充実感と落ち着きに包まれた一日で、静かな環境の中で集中して多くの作業を進められた。
		\item MBAノートを作る中で自然と疑問が湧き、問題意識が育っていく感覚が心地よかった。
		\item パンの美味しさのような小さな幸福と、知的探究の充実が同居していた。
		\item 達成感と好奇心の両方を強く感じた。
	\end{itemize}

	\textbf{Reflection(内省・分析)}
	\begin{itemize}
		\item 長時間の集中作業に没頭できたのは自分らしさの表れだった。
		\item 一方で、内向的な集中に偏りすぎず、外との交流を取り入れる余地も感じた。
		\item 知識を吸収するだけでなく、形にしていくプロセスそのものが自分にとっての充実の源泉であると再認識した。
	\end{itemize}

	\textbf{Insight(発見)}
	\begin{itemize}
		\item シュンペーターの理論を通して、「破壊→新結合→創造」という本来のイノベーションの流れに触れた。
		\item マルクスとは異なる視点から資本主義を捉える思考に興味が深まり、この世界の仕組みをより広い視野で理解したいと感じた。
		\item 学びは静かな内省と実践の循環の中で最も豊かに育つ。
	\end{itemize}

	\textbf{Next Step(次への展開)}
	\begin{itemize}
		\item 明日から始まる新しいMBA講義を受講し、講義ノートを通して思索をさらに深めたい。
		\item 事務的なタスク(会社からの連絡、郵送作業など)を整理し、心を整えた状態で新しい学びに臨む。
	\end{itemize}
\end{frame}

\section{2025年11月5日(水)}
\begin{frame}{2025年11月5日(水)}
	\tiny

	\textbf{Fact(事実)}
	\begin{itemize}
		\item 早朝4時に起床し、終日開発業務に没頭。
		\item 蓄積していたMBAの知見メモを、チームのレベル向上を意図して社内に共有。
		\item 完成したBIダッシュボードカタログを社員に共有できて、好意的な反応を得る。
		\item 午後の会議にて、新規マーケティングシステムの開発依頼を受け、開発担当になる。
		\item 高負荷な状態を自覚し、業務後に本屋へ立ち寄り「時系列分析の教科書」を購入。
	\end{itemize}

	\textbf{Event(出来事・印象)}
	\begin{itemize}
		\item 「大金が入る」という印象的な夢で一日が始まる。
		\item 自身の成果物が承認されたことに対し、タスク完了以上の「価値が認められた」という深い満足感を覚える。
		\item 連続した業務により「働き過ぎ」を自覚し、あえて物理的にPCから離れるという積極的な「切断」行動をとる。
	\end{itemize}

	\textbf{Reflection(内省・分析)}
	\begin{itemize}
		\item 朝から続いた「夢中」な状態は、時間を忘れるほどの没入感であったが、それは同時に自身のキャパシティへの客観的な懸念も生じさせた。
		\item 新規依頼を受けた際、挑戦心よりも先にキャパシティへの懸念が浮かんだのは、連続した高負荷状態の証左である。
		\item 役割分担が整理されてもなお「中核を担うプレッシャー」が残ることから、責任の重さはタスクの量だけでは測れないと再認識する。
	\end{itemize}

	\textbf{Insight(発見)}
	\begin{itemize}
		\item 高い集中力による成果(ダッシュボード)は、すぐに次の大きな期待(新規システム開発)となって還流することを実感する。
		\item 精神的な飽和状態に対し、物理的な場所の移動(本屋へ行く)と、本業と地続きの新たな知的刺激(時系列分析)が、有効なリセット手段となり得る。
	\end{itemize}

	\textbf{Next Step(次への展開)}
	\begin{itemize}
		\item 購入した時系列分析の知見を、まずは本業である新規マーケティングシステムの開発に応用する道筋を立てる。
		\item 並行して、株価分析などへの応用という純粋な知的好奇心も探求し、思考の幅を広げる。
	\end{itemize}
\end{frame}
\section{2025年11月6日(木)}
\begin{frame}{2025年11月6日(木)}
	\tiny

	\textbf{Fact(事実)}
	\begin{itemize}
		\item 朝、分析ダッシュボードのバグ修正を行い、その後、外出準備で慌ただしくなる。
		\item 友人と都内で異国の珍しい料理(煮込み料理)を食し、カフェで歓談。
		\item 本屋にて量子コンピュータ、確率微分方程式、生成AI投資に関する専門書を複数購入。
		\item 都内(複数箇所)のオフィスを訪問し、会社のロゴTシャツを受け取る。
		\item 夜、都内某所にてある数学者による圏論の講演会に参加し、講演後には著書へのサインと短い対話の機会を得る。
	\end{itemize}

	\textbf{Event(出来事・印象)}
	\begin{itemize}
		\item 友人との会話もさることながら、体験した異国の珍しい料理の「珍しさと美味しさ」自体が、感覚的に強く印象に残る一日となった。
		\item 講演会後の著者との短い対話は、知的な刺激に満ちた時間として記憶される。
	\end{itemize}

	\textbf{Reflection(内省・分析)}
	\begin{itemize}
		\item 専門家との直接的な接触は、純粋な「知的な興奮」を呼び起こすと同時に、「もっと勉強しなければならない」という自省的な学習意欲を強く喚起させる。
		\item 朝の慌ただしさは、その後の精神状態に影響を及ぼすものではなく、単なる事実の断片として記録される。
	\end{itemize}

	\textbf{Insight(発見)}
	\begin{itemize}
		\item 自身の知的好奇心が、特定のイベント(講演会など)に触発されるものではなく、常に拡散モデルの背景にある数学や一般相対性理論など、より根源的な数理へ向かっていることを再確認する。
		\item 旺盛な探究心ゆえに、自ら「消化不良」を危惧し、リーマン幾何学の参考書の購入を見送るなど、インプットを意識的に制御(マネジメント)する必要性を自覚している。
	\end{itemize}

	\textbf{Next Step(次への展開)}
	\begin{itemize}
		\item 新たに購入した専門書、特に確率微分方程式や生成AIに関する書籍の読解を進め、関心のある分野の数学的な背景理解を深めていく。
	\end{itemize}
\end{frame}

\section{2025年11月7日(金)}
\begin{frame}{2025年11月7日(金)}
	\tiny

	\textbf{Fact(事実)}
	\begin{itemize}
		\item 妻の公的支援の更新手続きのため、都内某所の担当窓口へ二人で外出する。
		\item 道中、公園で休憩し、妻に黒ビールを紹介したところ、好意的な反応が得られた。
		\item 妻はメンタルの周期的に最も調子の良い時期であり、YouTubeでの活動意欲を語る。
		\item 帰宅後、娘を迎え、その後は友人を地元に迎え、居酒屋と寿司屋の二軒を巡り、半年ぶりに飲酒する。
		\item 飲酒の結果、翌朝まで体調不良が残る。
	\end{itemize}

	\textbf{Event(出来事・印象)}
	\begin{itemize}
		\item 懸案だった事務手続き(公的支援の更新)を完了し、精神的な負荷が解消されたことによる「達成感」を得る。
		\item 妻との穏やかな時間を公園で共有し、妻が新しいお酒のレパートリー(黒ビール)を見つけたことにささやかな満足感を覚える。
	\end{itemize}

	\textbf{Reflection(内省・分析)}
	\begin{itemize}
		\item 妻の「新しいことを始めたい」という意欲に対し、純粋に応援したい気持ちと、バイオリズムの影響を理解しているからこぞの「不安」が、半分ずつ同居している。
		\item 友人を地元に招くことに「申し訳なさ」を感じていたが、当人がその場所を楽しんでくれたことは、自身の思い込みを覆す嬉しい驚きであった。
	\end{itemize}

	\textbf{Insight(発見)}
	\begin{itemize}
		\item 半年ぶりにアルコールを摂取したことで、酔いやすく体調を崩すという自身の体質を「やはり合わない」と明確に再確認する。
		\item 妻の意欲的な期間における言動に対しては、積極的な介入ではなく、本人のリズムを尊重し「見守る」姿勢が最適なサポートであると判断する。
	\end{itemize}

	\textbf{Next Step(次への展開)}
	\begin{itemize}
		\item 妻の活動意欲については、過度に干渉せず、本人の主体性を尊重する形でのサポート(見守り)を継続する。
		\item 今回の体調不良という明確なフィードバックに基づき、今後はアルコールを「絶対に飲まない」という決断を維持する。
	\end{itemize}
\end{frame}

\section{2025年11月8日(土)}
\begin{frame}{2025年11月8日(土)}
    \tiny

    \textbf{Fact(事実)}
    \begin{itemize}
        \item 昨夜の少量のアルコールにより、午前中は二日酔いであったが、昼過ぎには回復。
        \item 担当医が来年から院長として別の場所へ移ることを知り、引き続きその医師の診察を受ける意向を伝える。
        \item 「生成AIで英語学習をする教科書」および「高速取引の新書」を読了。
        \item 夜は『幼年期の終り』を読み進め、物語はクライマックスに差し掛かる。
    \end{itemize}

    \textbf{Event(出来事・印象)}
    \begin{itemize}
        \item 体調不良から回復した午後は、複数の書籍を読破することに集中し、知的好奇心が満たされる感覚を得た。
    \end{itemize}

    \textbf{Reflection(内省・分析)}
    \begin{itemize}
        \item 生成AIのHow-to本を読了したが、「実践なくして定着なし」という自身の学習スタイルに照らし、実際に手を動かせていないことへの焦りを感じる。
        \item 通院先を変更してでも同じ医師を頼るのは、長年の「信頼感」に加え、相手が経営者となることで将来的な「協業の可能性」も期待しているためである。
    \end{itemize}

    \textbf{Insight(発見)}
    \begin{itemize}
        \item ごく少量の飲酒が明確な二日酔いを引き起こすという事実を体験し、アルコールが自身の体質に合わないという昨日の決断を、身体的証拠をもって再確認した。
        \item 知的活動を「加速」させるフェーズに入っており、そのためには無料ツールの範囲を超え、有料の生成AIサービスへ「投資」することが必要であると判断する。
    \end{itemize}

    \textbf{Next Step(次への展開)}
    \begin{itemize}
        \item 読了した生成AIの教科書の内容を、プロンプト入力など具体的に手を動かして実践に移す。
        \item 知的生産性を高めるための投資として、生成AIの有料版への課金を具体的に検討し、導入する。
    \end{itemize}
\end{frame}

\end{document}
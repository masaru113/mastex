\documentclass[dvipdfmx, autodetect-engine, aspectratio=169, 10.5pt]{beamer}

\usepackage{amsmath}
\usepackage{amssymb}
\usepackage{amsthm}
\usepackage{graphicx}
\usepackage{hyperref}
\usepackage{enumitem}
\usepackage[english]{babel}

\usetheme{Boadilla}
\usetheme{Marburg}
\usecolortheme{orchid}
\usefonttheme{professionalfonts}

\title{
Log.
}

\author{
M. O.
}

\date{Nov. 2025}

\begin{document}
\maketitle

\section{November 1, 2025 (Sat)}
\begin{frame}{November 1, 2025 (Sat)}
    \tiny

    \textbf{Fact}
    \begin{itemize}
        \item Started the morning by drafting my professional resume, collaborating with Gemini to produce a high-quality final document.
        \item Visited a bookstore in the afternoon searching for life-hack books related to generative AI, but couldn't find anything that matched what I was looking for.
        \item Met a friend for South Indian food in Shibuya in the evening. The area was overflowing with people, coinciding with the Halloween season.
        \item Decided on a trade name for my new sole proprietorship: 'Quant Marketing Lab'.
    \end{itemize}

    \textbf{Event (Impressions)}
    \begin{itemize}
        \item Felt deeply impressed by the collaborative process with generative AI, truly realizing its potential.
        \item Navigating the Shibuya crowds was physically exhausting, but it was a relief to finally reach the restaurant and enjoy the meal.
        \item Headed home while discussing books with my friend, which rekindled my motivation to read.
    \end{itemize}

    \textbf{Reflection (Analysis)}
    \begin{itemize}
        \item The desire to master AI is growing stronger by the day, sparking a feeling akin to impatience.
        \item The hustle and crowds of the city feel draining; a reaffirmation that I'm the type who performs best in quiet environments.
        \item Considering the sheer speed of AI evolution, there's also anxiety about how effective my current learning will be in the long term.
    \end{itemize}

    \textbf{Insight}
    \begin{itemize}
        \item Generative AI is not merely a tool; it is a creative partner that enhances the quality of thought.
        \item Realized that learning is not 'training to produce results' but 'a continuous update process to adapt to change'.
        \item The creation of the 'Quant Marketing Lab' name has given a consistent direction to my intellectual pursuits.
    \end{itemize}

    \textbf{Next Step}
    \begin{itemize}
        \item Continue practical collaboration with generative AI for at least six months to refine operational skills and ideation abilities.
        \item Set aside tomorrow as a reading day to make progress on 'Childhood's End' and the generative AI books I have on hand.
        \item Secure time and space for quiet concentration to regain depth in my own thinking.
    \end{itemize}
\end{frame}

\section{November 2, 2025 (Sun)}
\begin{frame}{November 2, 2025 (Sun)}
    \tiny

    \textbf{Fact}
    \begin{itemize}
        \item Spent the entire day focused on work at home.
        \item Developed a script to automate the generation of lecture notes for my MBA courses.
        \item Built a system that transcribes audio using OpenAI Whisper and automatically generates LaTeX-formatted notes from a prompt template.
        \item Became completely engrossed in the work for about 12 hours straight, from afternoon until night, without a break.
        \item Enjoyed our customary 'Asa Mac' (Morning McDonald's) with the family; my two-year-old daughter showed interest in the computer.
    \end{itemize}

    \textbf{Event (Impressions)}
    \begin{itemize}
        \item The script worked better than expected; deeply impressed by the automation and the high-quality output.
        \item Felt a strong sense of fulfillment in the process of pursuing efficiency based on my own initiative, not external directives.
        \item My daughter trying to touch the computer was adorable, a warm moment amidst the busy day.
    \end{itemize}

    \textbf{Reflection (Analysis)}
    \begin{itemize}
        \item Reconfirmed my tendency to work to the point of physical exhaustion once focused, finding it hard to stop.
        \item Realized that mastering AI is not just about operation; the core lies in 'how to provide high-quality data and instructions'.
        \item While operational efficiency is improving, it feels like how I utilize my own creative thinking is also being called into question.
    \end{itemize}

    \textbf{Insight}
    \begin{itemize}
        \item Experienced firsthand that AI and humans, while holding different roles, exist in a complementary relationship.
        \item 'Coexistence with AI' is also a challenge of how clearly humans can design intent and structure.
        \item Felt that hints for future ways of working lie in the everyday, where technology and family life can naturally coexist.
    \end{itemize}

    \textbf{Next Step}
    \begin{itemize}
        \item Read more specialized books on generative AI utilization to further expand the scope of collaboration.
        \item Consciously take breaks even during long work sessions to establish a rhythm of concentration and recovery.
        \item Want to develop this system further, refining it into a versatile workflow reusable for learning and professional tasks.
    \end{itemize}
\end{frame}


\section{November 3, 2025 (Mon)}
\begin{frame}{November 3, 2025 (Mon)}
    \tiny

    \textbf{Fact}
    \begin{itemize}
        \item A quietly busy day spent creating MBA lecture notes in collaboration with generative AI.
        \item Gained a sense that learning can proceed efficiently as methods for data handling and prompt creation become more refined.
        \item Thought about sharing the lecture notes at work to enhance learning across the entire team.
    \end{itemize}

    \textbf{Event (Impressions)}
    \begin{itemize}
        \item Felt a sense of accomplishment at moments when I was able to craft effective prompts.
        \item There was a sense of fulfillment as the learning process gradually became more systematized within my mind.
        \item Realized that knowing someone might see the output provides motivation to continue and maintain quality.
    \end{itemize}

    \textbf{Reflection (Analysis)}
    \begin{itemize}
        \item Discovered that having the AI output content in a quiz format transforms passive lecture material into active, critical questions.
        \item Reaffirmed my characteristic of deepening concentration and understanding by designing 'learning' as a system.
    \end{itemize}

    \textbf{Insight}
    \begin{itemize}
        \item Giving precise instructions to generative AI simultaneously serves as training for organizing my own thoughts.
        \item Experienced how externalizing the learning structure through AI allows for the efficient training of one's thinking.
    \end{itemize}

    \textbf{Next Step}
    \begin{itemize}
        \item Want to learn more about advanced applications of generative AI to utilize it at a deeper level.
        \item Invest time not only in MBA studies but also in learning about generative AI itself to enhance intellectual productivity.
    \end{itemize}
\end{frame}

\section{November 4, 2025 (Tue)}
\begin{frame}{November 4, 2025 (Tue)}
    \tiny

    \textbf{Fact}
    \begin{itemize}
        \item Woke up slowly in the morning after a late night, getting about 8 hours of sleep.
        \item Had bakery bread for brunch that my wife bought. It was the second day in a row, but still delicious.
        \item Held a meeting with the marketing team starting at 12:30.
        \item Organized MBA lecture videos and created notes.
        \item In development work, built a historical table for real-time updated data. Also worked with AWS Lambda and EventBridge.
        \item Read more of Schumpeter's work in the evening, reflecting on 'creative destruction' and 'new combinations'.
        \item Made and ate steamed chicken for dinner, then went to bed around 21:00.
    \end{itemize}

    \textbf{Event (Impressions)}
    \begin{itemize}
        \item A day enveloped in fulfillment and calm; was able to focus and advance many tasks in a quiet environment.
        \item The comfortable feeling of questions naturally arising and a critical awareness developing while creating the MBA notes.
        \item A coexistence of small joys, like the delicious bread, and the fulfillment of intellectual inquiry.
        \item Felt a strong sense of both accomplishment and curiosity.
    \end{itemize}

    \textbf{Reflection (Analysis)}
    \begin{itemize}
        \item Being able to immerse myself in long hours of focused work was an expression of my nature.
        \item On the other hand, also felt there is room to incorporate external interaction, rather than leaning too heavily on introverted concentration.
        \item Reaffirmed that the process of giving form to knowledge, not just acquiring it, is the very source of my fulfillment.
    \end{itemize}

    \textbf{Insight}
    \begin{itemize}
        \item Through Schumpeter's theory, encountered the original flow of innovation: 'destruction -> new combination -> creation'.
        \item Grew more interested in the thinking that captures capitalism from a perspective different from Marx, feeling a desire to understand the mechanisms of this world with a broader view.
        \item Learning flourishes most richly within a cycle of quiet reflection and practice.
    \end{itemize}

    \textbf{Next Step}
    \begin{itemize}
        \item Attend the new MBA lectures starting tomorrow, aiming to deepen my reflections through the note-taking process.
        \item Organize administrative tasks (company communications, mailing, etc.) to approach new studies with a clear and prepared mind.
    \end{itemize}
\end{frame}

\end{document}
\documentclass[uplatex]{jsarticle}
\usepackage[english]{babel}
\usepackage[letterpaper,top=2cm,bottom=2cm,left=3cm,right=3cm,marginparwidth=1.75cm]{geometry}
\usepackage{amsmath, amssymb}
\usepackage{graphicx}
\usepackage{here}

\title{
Log.
}

\author{
M. O.
}

\date{Oct. 2025}

\begin{document}
\maketitle

\section{2025年10月1日(水)}

朝はファミリーレストランのモーニングを食べながら圏論のテキストと柄谷行人の『力と交換様式』を読んでいた。

昼過ぎになったのでそのままランチも注文した。

優雅に暮らせている。

妻に今月のお小遣いを渡したが、朝の優雅さとは裏腹に家計はぎりぎり。

副業の新卒既卒分析についてはAWS Lambdaで動かせるように作ったが、リファクタリングしないとえらいことになっている。

マックレーンの圏論の教科書の序盤で、圏論の可換図式を用いることで群、位相群、Lie群が統一的に理解できるということが書いてあって感激した。

物理の美しさにも感動ができて、数学の美しさにも感動ができて、自分は恵まれていると思う。





\section{2025年10月2日(木)}

昼は友人とケニア料理を食べに行く予定だったが、
ケニア料理のお店が閉まっていた。

仕方なく、いつもの五反田のペルー料理を食べに行った。
いつも通り美味しかったので結果的にとても良かった。

食後にカフェでコーヒーを飲みながら友人とお話をした。

学生時代の体育会系の部活ががいかに社会適応に役に立つかという話が出た。

他にも、予備校で教えている数学の苦手な生徒は参考書に書かれている以前に、数式をきちんと写せていないだとか、という話も聞いた。

友人は無事に医学部専門予備校に受かって、受かっただけでなく早速生徒を3名指導しているらしい。

カフェの後に解散して、それから新宿の会社に出社した。

友人と話せたり、チームメンバーと話せたりして良かった。

夜に新宿の古本屋に立ち寄ってから帰宅したら20時すぎになってしまった。



\section{2025年10月3日(金)}

1日中、ひたすら副業の開発をしていた。

Yomi Sheetの分析バッチがおおよそできてきた。

CAの稼働分析についてリーダーの2人、別件でCAのパフォーマンスについての分析システムを開発している人と話した。

どうも内定承諾数が1桁合わない。

おそらく30日や90日の区切りの刈り取りを先に行うか、最も進んだプロセスのフラグを先に立てるか、その順番によるものではないかと疑っている。

夜は友人と北千住にご飯を食べに出かけた。

北千住はビリヤニを食べられる可能性のあるお店は多いものの、不定休が多く、なかなか食べられない。

今日もラーメン二郎系の豚山を食べた。

ハンナ・アーレントの本、因果推論の本などを友人に渡せた。

帰り際にいつものビールの立ち飲み屋でゆっくり話した。

博士課程に入り直したい旨、具体的な研究室まで決まってきている話をした。

友人はコーポレートファイナンス理論に興味が出ているそう。

返ってからKeldysh Green Functionのノートをアップロードした。

\section{2025年10月4日(土)}

ほとんど1日中寝ていた日。

リチャード・ローティの『偶然性・アイロニー・連帯』が届いた。読みたい。

日付が変わるタイミング、つまり年齢が変わるタイミングは、グロダンディーク構成と等しくなるコンマ圏が、要素の圏として圏の積分記号を用いて表されるところを勉強していた。
圏論にも積分が出てきて、フビニの定理まであるとは驚いた。


\section{2025年10月5日(日)}

朝は少し睡眠不足気味。

フェルミ液体のノートを少し修正した。

圏論の極限の勉強をしている。

今日は誕生日なのでお寿司を食べに行く予定だったが、娘が家の中で暴れまわっていて、妻が疲れてしまっていて、お弁当屋さんでお惣菜を買って食べた。

AmmonとErdmengerのゲージ重力対応の本がたまたま中古で出ているがあまり安くない。
以前から買おうか悩んでいたが、誕生日なので買おうかと思っていた。

ひょんなことから入手できた。

ついでにAltrandとSimonsの教科書、中原のトポロジーの教科書、Xiao Gang Wenの教科書も入手できた。
ラッキー。

古本屋ではオードリー・タンの『プルラリティ』、
ミルトン・フリードマンの『資本主義と自由』
も手に入れた。

このMisc.のノートを作ってみた。

妻と娘は早く寝たので、夜の一人の時間が長くなった。

ゆっくりとくつろぎながら読書ができる贅沢な時間を味わっている。

Emily Riehlの圏論の教科書がかなり良さそうに思えてきた。
米田の補題のあたりをノートにまとめたいなと思う。

買い物のために少し家の外を歩いたくらいで、特に大きなイベントもなく、とても平穏で、これまでの人生で最高に良い誕生日になった。

\section{2025年10月6日(月)}

朝は5:55に起きた。
8時間半睡眠できた。

8時くらいまで圏論の極限の基本事項について勉強してまとめていた。

9時くらいから久しぶりに近所のファミレスでモーニングを楽しんだ。

ハンナ・アーレントの本を読んでいた。

家の鍵を忘れたことに気が付いたが、妻が帰宅してくれていて助かった。

昼は米田の補題の証明を追って、自分なりにまとめてみた。

自分なりに分かるように証明を書いたつもりが、やっぱりまだややこしさを感じるので、ノートの出来はあまりよくないのかもしれない。
もしくはまだ圏論に慣れていないのかもしれない。

昼ごはんは弁当屋さんのお弁当にした。
妻と一緒にくつろいで食べた。

タスク整理をして、妻に事務作業を手伝ってもらって会社の書類を出した。

MBAの講義も受講した。

最近パソコンの前に座っていることが多いので明日はゆっくりサウナにでも行きたい。


\section{2025年10月7日(火)}

23時に就寝して6時に起床で7時間睡眠。

家にゴキブリが出たので、家では過ごしたくなく、個室サウナで1日過ごした。

会社でルービックキューブと群論が話題に出て、そういえば群論の勉強を久しくしていないことに気づいた。

群の表現論とか、その他基本的ないくつかの定理の証明は追えるようにしておきたい。

時系列解析についての本を買った。

ARIMA過程とかはPメジャーの人は使うのかなあ。

ARモデルから勉強したい。

今年もノーベル物理学賞は中継で見た。

ジョセフソン接合を用いた量子計算の基礎のお話だった。嬉しい。

超弦理論、ゲージ理論、くりこみ群の本を買った。


\section{2025年10月8日(水)}

昨夜は21:30すぎに寝て朝は8時半くらいに起きた。

11時間睡眠。よく眠れて調子が良い。

中原の幾何学とトポロジーが届いたので朝からまったり読書をした。

昼前はCAのビジネスプロセスについて教えていただくミーティングをした。

各CAそれぞれに最適な数値を提供できればみんな安心して業務ができそうという展望が得られた。

昼ご飯はお惣菜屋さんで唐揚げとかを食べた。

子どものオムツを買うついでに事務用品、消耗品をいくつか買った。

本屋でゆっくりウィンドウショッピングを楽しんだ。

気持ちよく散歩もできて気分良く帰宅したが、買い物の目的の子どもオムツを買うのを忘れてしまったことに帰宅してから気付いた。

昼から時系列解析の教科書を読んでいた。

ARモデルについてざっくり仕組みを掴めた。

ARモデルの次数pが大きすぎるときはMAモデルが有効なようで、その仕組みについて次に学びたい。

CP対称性の破れの教科書も買った。

2025年はノーベル化学賞も日本人が選ばれたらしい。おめでたい。

多孔の金属錯体による成果とのことで、物性分野であることもおめでたい。嬉しい。

今日も良い1日になった。


\section{2025年10月9日(木)}

この頃、朝起きて即、目をこすりながらプログラミングするというのが習慣化してきている。

午前中はひたすらコーディングをしていた。

上野のウイグル料理屋で妻と友人と3人でランチをした。

ラグメンとクワスというお酒を楽しんだ。

食べ過ぎたか香辛料でやられたか、お腹の調子が悪くなってしまい、夜まで何も手を付けられず。

ノーベル賞を受賞したのをきっかけに量子情報理論について復習した。

複素多様体の教科書が届いた。

複素幾何について見たことのある言葉は多いものの全然そのコンセプトが分からないし、読み進められない。

代数の基本的な知識が足りてないなあと感じる。

中原の幾何学とトポロジーでホモロジー代数の大枠を掴んだりする必要がありそう。

ガロア理論について勉強をするのも、その中で代数のコンセプトが得られるなら良いのかもしれない。

代数が分かれば圏論の理解も深まるし、代数がボトルネックになっていると感じる。


\section{2025年10月10日(金)}

朝は超伝導のノート2つと金融工学の記事、配当支払いの無い原資産のアメリカンオプションは常にヨーロピアンオプションよりも安いという話のメモを掘り起こしてgit pushした。

昼ご飯は妻と駅前のインドカレー屋さんに行った。

辛口のほうれん草カレーとチーズナンは美味しかった。

ナンのおかわりをしてしまって、食べ過ぎた。

超弦理論、ゲージ理論、複素解析の教科書たちがたくさん届いた。

散髪に行こうとしたが、また散髪屋が閉まっていた。

散髪屋に行くついでに寄った本屋で、今日発売の雑誌、数学セミナーの今月版は圏論特集だったのでそれも購入した。

夕方はこれまでまとめた伊藤の補題やブラック・ショールズの話などを整理してgithubに上げた。

今日は娘は保育園で運動会があったらしく、迎えに行くと金メダルの折り紙を首から提げていた。

昼からずっとお腹いっぱいなので晩ご飯は抜いた。

夜は雑誌の圏論特集を眺めて過ごした。

寝る直前にGithub Pageを作ってみた。

\section{2025年10月11日(土)}

朝は娘が布団の部屋で牛乳をこぼした騒ぎで起きた。

朝から群馬に行く支度をして、支度をざっくり終えて、少し横になったつもりが昼前まで寝てしまった。

結局、13:30に家を出た。
雨の中、キャリーケースと子どもをベビーカーに乗せての移動はなかなかハードだった。

電車の中で子どもが牛乳をこぼした。
外人に拾ってもらった。
「覆水盆に返らず」の意味のIt's no use crying over split milkを咄嗟に言えれば良いジョークになったのに、それが出ずに普通に感謝の気持ちだけを伝えるまでになってしまった。

高崎に着くと寒かった。
電車に乗ってる間に季節が変わったのかと思えるほど。

義母に車で迎えに来てもらって、お食事もいただいた。
栗ご飯と白菜のお吸い物と柿。

義母に娘と合わせることができてよかった。

娘は広い家ではしゃいでいた。

寝床は少し埃っぽくて、咳が出る。

\section{2025年10月12日(日)}

朝は8時くらいに起きた。
妻と娘はとっくに起きていたらしかった。
花粉症みたいになってしまって眠りが浅かった。
口はカラカラで鼻水が止まらない。
田舎には田舎ならではの困難さがあることを思い出した。

朝食は昨日の煮物にうどんを入れたものと、手作りのたまごサンドをいただいた。

義母の株式と投資信託の運用のサポートをした。

昼前に軽食で栗ご飯のおにぎりとバナナとポメロのドライフルーツをいただいた。

娘を寝かし付けるために添い寝していたら自分だけが昼寝をしてしまって、起きたら夜になってしまった。

焼き肉を作っていただいた。

今日は食べて寝るだけの1日になった。

\section{2025年10月13日(月・祝)}

昨日は14時間近く寝たので早起きになった。朝は5時くらいに起きた。
コーポレートファイナンス理論の教科書を読んでいたらまた寝ていた。

7時に起きて庭で妻と柿の木から柿を採った。
たくさん庭に柿がなっていて、たくさん採れた。
3,40個ほど採れた。

食べきれないので大阪の実家に送ることにした。

9時40分頃に帰りの電車に乗った。

10:40頃に新幹線に乗った。
自由席で座れず。
立ち乗りで過ごした。

ガロア群の定義を理解しようとしていた。

新幹線内で外国人と思わしき人から切符を見せられてどこの席か分からないから教えてほしいと日本語で尋ねられた。
その切符は自由席だったので、この新幹線では1〜5号車の空いている席に座ることができると英語で教えてあげた。
とても上手な日本語で「なるほど、分かりました。ありがとう。」と返ってきた。
見ていた妻曰く、日本人相手に英語で説明していてコントみたいだったとのこと。

\section{2025年10月14日(火)}

場の理論とトポロジーについて思いを馳せた1日になった。

昼はミーティングが2件続いた。
マーケティングチームのミーティングではARモデルについて話した。
新しい施策の効果を外場を摂動として入れることで因果効果を説明できるという話が出た。

PORTERSからデータを取得する周りについてのミーティングもした。

それから開発が進んで、登録者データ2年分のバルク取得ができるようになった。

新卒・既卒の登録者分析バッチも定期実行できるようにした。

夕方は病院に行った。

昨日はあまりにも疲れてしまったからか、鬱っぽくなってしまった。


\section{2025年10月15日(水)}

朝は7時くらいに起きてシャワーを浴びてゴミ出しをして丸の内の丸善に行った。

「加群とホモロジー代数」、「ファイバー束とホモトピー」が圏論をベースにした数学の教科書で、まだ加群もホモトピーも知らないので、とても良いなあと思って眺めていた。

時間ギリギリになって会社の面談をした。
3ヶ月間、延長することを伝えた。
2026年1月からプログラムが再開するかもしれないらしい。

とみ田でつけ麺をお腹いっぱい食べた。

もう一度丸善に戻って、「加群とホモロジー代数」、「ファイバー束とホモトピー」 、「線形代数対話2巻」の3冊を買ってしまった。
1万2千円なり。

凝縮系物理における場の量子論の教科書も良いものが複数出ていたなあと思った。

帰宅してから開発の続きをした。

MBAの講義も受講した。
マーケティングとオペレーションマネジメント。

Radon-Nikodym 微分と Cameron-Martin-Girsanov 定理についてのノートを書いた。

21時くらいに就寝。


\section{2025年10月16日(木)} 

朝から外国為替に対するデリバティブの価格付けのノートを作っていた。

久しぶりにヒゲを剃って、シャワーを浴びて9:30くらいに家を出た。

10:30くらいに高田馬場に着いて本屋にいた。

11:30ちょうどにロシア料理屋に着いたが、友人が開店前行列に並んでいてくれたおかげで入店できた。

ビーフストロガノフを食べた。ここしばらく食べたものの中で一番おいしかった。

喫茶店でケーキを食べて、解散した。

帰宅してからは開発業務に勤しんだ。

マーケ担当者とミーティングをしたが、高度に複雑なものをさらに複雑にするようなシステム開発の依頼だったので、他のソリューションを提案した。

ガロア理論の本を読んでいるが、一向に理解した気にならない。
読書自体は進んでいるが、自分の言葉で説明する練習をしていないので、理解した気にならないのかもしれない。

群と加群についても勉強を進めたい。
加群についての知識がないのでホモロジー代数についての読書が進まない。

\end{document}
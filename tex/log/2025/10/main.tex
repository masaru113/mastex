\documentclass[uplatex]{jsarticle}
\usepackage[english]{babel}
\usepackage[letterpaper,top=2cm,bottom=2cm,left=3cm,right=3cm,marginparwidth=1.75cm]{geometry}
\usepackage{amsmath, amssymb}
\usepackage{graphicx}
\usepackage{here}

\title{
Log.
}

\author{
M. O.
}

\date{Oct. 2025}

\begin{document}
\maketitle

\section{Wednesday, October 1, 2025}

Spent the morning at a family restaurant, reading a textbook on category theory and Kojin Karatani's "Power and Modes of Exchange."

As noon approached, ordered lunch there as well.

Feeling that life is quite graceful these days.

Handed my wife her monthly allowance, though our household finances remain tight despite the morning's serenity.

For my side project on new graduate analysis, built a system that runs on AWS Lambda, but without refactoring it, things are getting rather messy.

Was deeply moved by a passage in Mac Lane's category theory textbook, which explained how commutative diagrams can unify the understanding of groups, topological groups, and Lie groups.

It's a blessing to be able to feel moved by both the beauty of physics and that of mathematics.



\section{Thursday, October 2, 2025}

Had planned to eat Kenyan food with a friend for lunch, but the restaurant was closed.

Instead, went to our usual Peruvian place in Gotanda. As always, it was excellent—so it turned out well.

After the meal, we had coffee at a café and talked.

The conversation turned to how much participation in athletic clubs during student years helps people adapt to society.

Also heard that some of the students he tutors in mathematics struggle not because of the textbooks themselves, but because they can't properly copy the equations in the first place.

My friend was recently hired at a prep school specializing in medical students, and not only got in but is already teaching three students.

After the café, we parted ways, and I headed to the company office in Shinjuku.

It felt good to have conversations—with him, and later, with team members.

On the way home, stopped by a used bookstore in Shinjuku; by the time I returned, it was past eight.



\section{Friday, October 3, 2025}

Spent the entire day developing my side project.

The analysis batch for Yomi Sheet is nearly complete.

Talked with two team leads about CA's operational analysis, as well as with another person who's developing a performance analysis system for CA.

The number of accepted offers seems off by an order of magnitude.

It might be due to whether we harvest 30- or 90-day intervals first, or whether we flag the most advanced process first—suspecting the mismatch comes from that sequencing.

In the evening, went out for dinner with a friend in Kitasenju.

Although there are quite a few restaurants that serve biryani, many have irregular schedules, so it's often hard to find one open.

Ended up eating at Butayama, a Jiro-style ramen place, again.

Gave my friend some books—Hannah Arendt's and one on causal inference.

On the way back, we talked leisurely over beers at our usual standing bar.

Told him about my plan to re-enter a doctoral program, and that I've already narrowed down a specific lab.

He, on the other hand, has recently developed an interest in corporate finance theory.

After coming home, uploaded my notes on the Keldysh Green function.



\section{Saturday, October 4, 2025}

Spent almost the entire day sleeping.

Richard Rorty's "Contingency, Irony, and Solidarity" arrived—looking forward to reading it.

Around midnight, as the date (and thus my age) changed, I was studying how the comma category corresponding to the Grothendieck construction can be expressed as a category of elements using the integral symbol over categories.
It's astonishing that even category theory includes an integral concept—and even a Fubini theorem.



\section{Sunday, October 5, 2025}

Slept a bit less than usual and felt slightly short on rest in the morning.

Spent some time revising my notes on Fermi liquids.

Currently studying limits in category theory.

It was supposed to be a birthday celebration with sushi, but my daughter was running wild around the house, and my wife was too exhausted.
So we ended up buying some prepared dishes from a bento shop and eating at home instead.

Found a used copy of Ammon and Erdmenger's book on gauge/gravity duality—not cheap, but I had been debating whether to buy it for a while.
Since it's my birthday, I decided to go for it.

By chance, I managed to get a copy.

Along with it, I also obtained textbooks by Altland and Simons, Nakahara's book on topology, and Xiao-Gang Wen's text.
Quite lucky.

At a secondhand bookstore, I picked up Audrey Tang's "Plurality" and Milton Friedman's "Capitalism and Freedom" as well.

Started putting together this miscellaneous notebook.

My wife and daughter went to bed early, leaving me with a long stretch of quiet nighttime.

Enjoying the luxury of slow, peaceful reading.

Emily Riehl's textbook on category theory is turning out to be excellent; I'm thinking of summarizing the section on Yoneda's lemma in my notes.

Aside from a short walk outside for some shopping, there was no particular event today—just a calm, peaceful day.
It turned out to be the best birthday I've ever had.



\section{Monday, October 6, 2025}

Woke up at 5:55 in the morning after getting eight and a half hours of sleep.

Spent time until around 8 a.m. studying and summarizing the basic concepts of limits in category theory.

Around 9 a.m., went to a nearby family restaurant and enjoyed breakfast for the first time in a while.

Read a book by Hannah Arendt during the meal.

Noticed that I had forgotten my house key, but thankfully my wife had already come home, which saved me from trouble.

In the afternoon, followed the proof of Yoneda's lemma and tried to summarize it in my own words.

Although the proof was meant to be written in a way I could understand, it still felt somewhat complicated; perhaps my notes are not very good—or maybe I'm simply not yet used to category theory.

Had a boxed lunch from a local bento shop and enjoyed it together with my wife.

Later, organized tasks and, with my wife's help, submitted some company documents.

Also attended an MBA lecture.

Since I've been spending so much time in front of the computer lately, I'd like to take it easy tomorrow and go to the sauna.


\section{Tuesday, October 7, 2025}

Went to bed at 11 p.m. and woke up at 6 a.m.—a seven-hour sleep.

A cockroach appeared in the house, so I didn't feel like staying there and spent the whole day in a private sauna instead.

At work, Rubik's Cubes and group theory came up in conversation, and it struck me that I hadn't studied group theory in quite some time.

Would like to be able to follow proofs of basic theorems and group representation theory again.

Bought a book on time series analysis.

I wonder if people in P-major fields actually use ARIMA processes.

Planning to start from AR models.

Watched the Nobel Prize in Physics broadcast this year as well.

It was about the fundamentals of quantum computation using Josephson junctions—delightful.

Also bought books on superstring theory, gauge theory, and the renormalization group.


\section{Wednesday, October 8, 2025}

Went to bed around 9:30 p.m. last night and woke up around 8:30 in the morning.

Eleven hours of sleep. Slept deeply and felt great.

A new copy of Nakahara's \textit{Geometry, Topology and Physics} arrived, so spent a slow morning reading it.

Before noon, had a meeting to learn about CA's business processes.

Gained a clear vision that if each CA could be provided with the optimal numerical parameters, everyone could work with greater confidence and peace of mind.

Lunch was simple—fried chicken and a few other dishes from a local delicatessen.

Stopped by to buy some office and daily supplies, along with diapers for the baby.

Then spent some quiet time browsing books at a bookstore, enjoying a slow, relaxed pace.

Returned home in a pleasant mood after a refreshing walk, only to realize I had forgotten the very purpose of the shopping trip—the baby diapers.

Spent the afternoon reading a textbook on time series analysis.

Got a rough understanding of how the AR model works.

When the order $p$ of the AR model becomes too large, the MA model seems to be effective; that mechanism will be the next thing to learn.

Also purchased a textbook on CP symmetry breaking.

It seems that a Japanese scientist has been awarded the 2025 Nobel Prize in Chemistry—what wonderful news.

The achievement involves porous metal complexes, which belong to the field of condensed matter physics. That makes it even more delightful.

All in all, another good day.


\section{Thursday, October 9, 2025}

Lately, it's becoming a habit to start programming right after waking up, rubbing my eyes as I reach for the keyboard.

Spent the entire morning immersed in coding.

Had lunch at a Uyghur restaurant in Ueno with my wife and a friend.

We enjoyed laghman and kvass, a fermented drink with a unique, earthy sweetness.

Perhaps I ate too much, or maybe the spices got to me — my stomach started to feel off, and I couldn’t touch any food for the rest of the day.

Took some time to revisit quantum information theory, prompted by the recent Nobel Prize announcement.

A textbook on complex manifolds arrived in the mail.

Although I recognize many of the terms related to complex geometry, the concepts themselves remain elusive, and I can hardly make progress reading it.

It’s becoming clear that my foundation in algebra is still insufficient.

I may need to go back to Nakahara’s \textit{Geometry, Topology and Physics} to get a broader grasp of homological algebra.

Studying Galois theory might also help, especially if it allows me to internalize some of those algebraic concepts along the way.

A deeper understanding of algebra would naturally enrich my grasp of category theory — it feels like algebra has become the bottleneck.

\section{Friday, October 10, 2025}

In the morning, two notes on superconductivity, an article on financial engineering, and a memo about the fact that an American option on a non-dividend-paying underlying is always cheaper than its European counterpart were dug up and pushed to Git.

Lunch was spent with my wife at an Indian curry restaurant near the station.

The spicy spinach curry paired with cheese naan was especially delicious.

A second naan was ordered, resulting in a bit of overeating.

A stack of textbooks on superstring theory, gauge theory, and complex analysis arrived.

Plans for a haircut were thwarted again when the barbershop was closed.

At a bookstore visited along the way, this month's issue of \textit{Mathematics Seminar} featured category theory, so it was picked up as well.

In the evening, summaries on Ito's lemma and the Black-Scholes model were organized and uploaded to GitHub.

Apparently, my daughter had a sports day at her daycare, and she greeted me wearing a paper gold medal around her neck.

Having been full since lunch, dinner was skipped.

The night was spent leafing through the magazine's category theory feature.

Created github pages.


\section{Saturday, October 11, 2025}

Woke up this morning to a commotion: my daughter had spilled milk all over the futon room.

Got ready to head to Gunma, but after a quick nap, I ended up sleeping until noon.

We finally left the house at 1:30 PM.
Traveling in the rain with a suitcase and a stroller was pretty tough.

On the train, my daughter spilled milk again.
A foreigner helped pick it up.
I wish I had been quick enough to say, 'It's no use crying over split milk.' Would've been a great joke, but instead, I just ended up saying a simple thank you.

Arriving in Takasaki, it was chilly.
Felt like the seasons had changed during our short train ride.

My mother-in-law picked us up in her car, and we had a lovely meal.
Chestnut rice, Chinese cabbage soup, and persimmons.

It was so nice to finally introduce my daughter to her grandmother.

The little one was so excited, running all over the big house.

The sleeping area was a bit dusty, and it made me cough.



\section{Sunday, October 12, 2025}

Woke up around 8 AM.
Turns out my wife and daughter had already been up for a while.
My sleep was shallow, like I had hay fever.
My mouth was dry, and my nose wouldn't stop running.
Remembered that the countryside has its own unique set of challenges.

For breakfast, we had yesterday's simmered vegetables with udon and some homemade egg salad sandwiches.

Helped my mother-in-law with her stocks and investment trusts.

Had a light lunch of chestnut rice balls, a banana, and dried pomelo.

I lay down with my daughter to get her to nap, but I fell asleep myself. Woke up, and it was already evening.

We had yakiniku for dinner.

Today was a day of just eating and sleeping.


\section{Monday (Holiday), October 13, 2025}

After nearly 14 hours of sleep, I woke up early today. Around 5 AM.

Was reading a corporate finance theory textbook and fell asleep again.

Woke up at 7 AM and went to the garden with my wife to pick persimmons from the tree.
There were so many, we picked about 30 to 40.
Since we couldn't possibly eat them all, we decided to send some to my parents' house in Osaka.

Got on the train for the trip home around 9:40 AM.

Transferred to the shinkansen around 10:40 AM.

The unreserved cars were packed, so we couldn't get a seat.

Had to stand the whole way.

I was trying to understand the definition of Galois groups.

On the shinkansen, a person who seemed to be a foreigner showed me their ticket and asked in Japanese for help finding their seat.
The ticket was for an unreserved seat, so I explained in English that they could sit in any available seat in cars 1 through 5.
They replied with a very fluent 'I see, thank you very much' in Japanese.
My wife, who was watching, said it looked like a skit, with me explaining in English to a Japanese person.


\section{Tuesday, October 14, 2025}

Spent the day thinking a lot about field theory and topology.

In the afternoon, had back-to-back meetings. During the marketing team's meeting, we discussed AR models. A point came up about how we could explain causal effects by introducing new campaign results as perturbations from an external field.

Also had a meeting about fetching data from PORTERS.

After that, continued with my development work and finally got the bulk acquisition of two years' worth of registrant data working.

Set up a batch process to regularly analyze data for new and recent graduates.

Later in the evening, went to the hospital.

Felt a little down and depressed, perhaps because yesterday was just so exhausting.


\section{Wednesday, October 15, 2025}

Woke up around 7 a.m., took a shower, and took out the trash before heading to Maruzen in Marunouchi.

There were a couple of math textbooks, 'Modules and Homological Algebra' and 'Fiber Bundles and Homotopy,' that were based on category theory. Since I don't know much about modules or homotopy yet, I found myself really drawn to them.

Made it to my company meeting just in the nick of time. Told them I'd be extending my contract for another three months. It sounds like the program might resume in January 2026.

Had a huge, satisfying bowl of tsukemen at Tomita.

Went back to Maruzen again and ended up buying those three books: 'Modules and Homological Algebra,' 'Fiber Bundles and Homotopy,' and 'Linear Algebra Dialogues Vol. 2.' It came out to 12,000 yen.

I also noticed there were a number of good textbooks on quantum field theory for condensed matter physics.

Got home and continued with my development work.

Attended some MBA lectures as well. They covered marketing and operations management.

Wrote some notes on the Radon-Nikodym derivative and the Cameron-Martin-Girsanov theorem.

Finally, went to bed around 9 p.m.

\end{document}

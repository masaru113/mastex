\documentclass{article}
\usepackage[english]{babel}
\usepackage[letterpaper,top=2cm,bottom=2cm,left=3cm,right=3cm,marginparwidth=1.75cm]{geometry}
\usepackage{amsmath, amssymb}
\usepackage{graphicx}
\usepackage{here}

\title{
Log.
}

\author{
M. O.
}

\date{Oct. 2025}

\begin{document}
\maketitle

\section{October 1, 2025 (Wednesday)}

Started the morning at a family restaurant, reading a category theory textbook and Kojin Karatani's 'Power and Modes of Exchange' over breakfast.

It turned to afternoon, so I just ordered lunch there as well.

A taste of the elegant life.

Gave my wife this month's allowance, but contrary to the morning's elegance, the household finances are tight.

The side-job analysis of new and recent graduates is now built to run on AWS Lambda, but it's a mess and desperately needs refactoring.

Was deeply impressed reading the beginning of Mac Lane's category theory textbook, which mentioned how commutative diagrams in category theory can provide a unified understanding of groups, topological groups, and Lie groups.

To be moved by the beauty of physics, and also by the beauty of mathematics... I feel fortunate.

\section{October 2, 2025 (Thursday)}

Had plans to eat Kenyan food with a friend for lunch, but the restaurant was closed.

We ended up at our usual Peruvian place in Gotanda instead. It was delicious, as always, so it all worked out well.

Chatted with my friend over coffee at a cafe afterward.

We talked about how much those intense, sports-focused clubs in school help with social adaptation later.

Also heard that his students at the prep school who struggle with math often can't even copy the formulas correctly, long before they get to what's written in the textbook.

My friend apparently got the job at the medical school prep academy, and not only that, he's already teaching three students.

We parted ways after the cafe, and I headed to the office in Shinjuku.

It was good to talk with my friend and the team members.

Stopped by a used bookstore in Shinjuku on the way home, didn't get back until after 8 PM.

\section{October 3, 2025 (Friday)}

Spent the entire day just grinding on development for my side job.

The analysis batch for Yomi Sheet is mostly complete.

Spoke with the two CA leaders about workload analysis, and also with the person developing the CA performance analysis system for a different matter.

The number of job offers accepted seems to be off by an order of magnitude.

I suspect it's due to the order of operations---whether the 30-day or 90-day cutoffs are processed first, or if the flag for the most advanced process is set first.

Went out to Kitasenju for dinner with a friend at night.

Kitasenju has many places that might serve biryani, but they have irregular holidays, making it hard to actually get any.

Ended up at Butayama, a Jiro-style ramen shop, again today.

Managed to give my friend the Hannah Arendt book and the causal inference book.

Chatted for a while at our usual standing beer bar on the way back.

Talked about wanting to re-enroll in a doctoral program, and how I've even narrowed it down to specific research labs.

My friend, it seems, is getting interested in corporate finance theory.

Got home and uploaded my notes on the Keldysh Green Function.

\section{October 4, 2025 (Saturday)}

A day spent almost entirely sleeping.

Richard Rorty's 'Contingency, Irony, and Solidarity' arrived. Want to read it.

Right as the date changed---my birthday---I was studying the part where the comma category, which becomes equivalent to the Grothendieck construction, is represented using the integral sign for categories as the category of elements.
Surprised to find that category theory also has integration, and even Fubini's theorem.

\section{October 5, 2025 (Sunday)}

Woke up feeling a bit sleep-deprived.

Made some minor corrections to my notes on Fermi liquids.

Studying limits in category theory.

Today's my birthday. We planned to go out for sushi, but my daughter was rampaging around the house, and my wife got exhausted, so we just bought some prepared food from a bento shop and ate at home.

The Ammon and Erdmenger book on gauge/gravity duality is available used, but not very cheaply. I'd been debating buying it for a while and was thinking of getting it for my birthday.

Managed to acquire it through a stroke of luck.

Also got my hands on the Altland and Simons textbook, Nakahara's topology textbook, and Xiao-Gang Wen's textbook. Lucky.

At the used bookstore, I also picked up Audrey Tang's 'Plurality' and Milton Friedman's 'Capitalism and Freedom'.

Tried creating this 'Misc.' note.

My wife and daughter went to bed early, leaving me with a long evening to myself.

Savoring the luxury of relaxing and reading at my own pace.

Emily Riehl's category theory textbook is starting to look quite good.
Feel like I want to summarize the section on the Yoneda lemma in my notes.

Aside from stepping out briefly for shopping, there were no major events. It was an extremely peaceful, and the best, birthday I've ever had.

\section{October 6, 2025 (Monday)}

Woke up at 5:55 AM. Got eight and a half hours of sleep.

Until about 8:00, I was studying and summarizing the fundamentals of limits in category theory.

Around 9:00, I enjoyed breakfast at the nearby family restaurant for the first time in a while.

Was reading Hannah Arendt.

Realized I'd forgotten my house key, but thankfully my wife was home to let me in.

In the afternoon, I traced the proof of the Yoneda lemma and tried to summarize it in my own way.

I tried to write the proof to be understandable to myself, but it still feels complicated. Maybe the notes aren't very good, or maybe I'm just not used to category theory yet.

Lunch was from the bento shop. Ate relaxing with my wife.

Organized my tasks, and my wife helped me with some administrative work to submit company documents.

Also attended an MBA lecture.

Been spending too much time in front of the computer lately. I'd like to take a slow day at the sauna tomorrow.

\section{October 7, 2025 (Tuesday)}

Bed at 11 PM, up at 6 AM. Seven hours of sleep.

A cockroach appeared in the house, so I couldn't stand being there and spent the day at a private sauna.

Rubik's Cubes and group theory came up at work, making me realize I haven't studied group theory in a long time.

I want to make sure I can still follow the proofs for group representation theory and other basic theorems.

Bought a book on time-series analysis.

Wonder if P-measure folks use things like ARIMA processes.

Want to start by studying AR models.

Watched the Nobel Prize in Physics announcement live again this year.

It was about the fundamentals of quantum computing using Josephson junctions. Happy about that.

Bought books on superstring theory, gauge theory, and the renormalization group.

\section{October 8, 2025 (Wednesday)}

Went to bed just after 9:30 PM last night and woke up around 8:30 AM.

Eleven hours of sleep. Feeling good after sleeping so well.

Nakahara's 'Geometry, Topology and Physics' arrived, so I spent the morning reading leisurely.

Before lunch, attended a meeting to learn about the CA business process.

We gained a perspective that if we can provide optimized metrics for each CA, everyone could work with more peace of mind.

Lunch was fried chicken and such from the prepared food shop.

Picked up some office supplies and consumables while out buying diapers for my child.

Enjoyed some slow window shopping at the bookstore.

Had a pleasant walk and returned home in good spirits, only to realize I'd forgotten the diapers, the main purpose of the shopping trip.

Reading the time-series analysis textbook since the afternoon.

Got a rough grasp of how AR models work.

It seems MA models are effective when the order $p$ of the AR model is too large; want to learn that mechanism next.

Also bought a textbook on CP symmetry violation.

Apparently, a Japanese researcher also won the Nobel Prize in Chemistry in 2025. Congratulations. It's for work on porous metal complexes, which is great to see as it's in the field of condensed matter physics. Thrilled.

Today was another good day.

\section{October 9, 2025 (Thursday)}

Lately, the habit of waking up and immediately, while rubbing my eyes, starting to program has taken root.

Spent the whole morning coding.

Had lunch at a Uyghur restaurant in Ueno with my wife and a friend.

Enjoyed laghman and kvass.

My stomach felt off, maybe from overeating or the spices, and I couldn't get anything done for the rest of the evening.

Inspired by the Nobel Prize, I reviewed some quantum information theory.

The textbook on complex manifolds arrived.

I've seen many terms related to complex geometry, but I just can't grasp the concepts, and I'm not making progress reading it.

Feel like I'm lacking fundamental knowledge of algebra.

It seems necessary to grasp the general framework of homological algebra using Nakahara's 'Geometry, Topology and Physics'.

Studying Galois theory might also be good if it helps build algebraic concepts.

If I understand algebra, my understanding of category theory will deepen. I feel algebra is the bottleneck.

\section{October 10, 2025 (Friday)}

In the morning, I dug up and git push'ed two notes on superconductivity, an article on financial engineering, and a memo about how American options on an underlying asset with no dividend payments are always cheaper than European options.

Went to the Indian curry place in front of the station with my wife for lunch.

The spicy spinach curry and cheese naan were delicious.

Got a refill on the naan and ate way too much.

A bunch of textbooks on superstring theory, gauge theory, and complex analysis arrived.

Tried to go for a haircut, but the barber shop was closed again.

Stopped by a bookstore on the way to the barber, and this month's issue of 'Sugaku Seminar' magazine, released today, was a special feature on category theory, so I bought that too.

In the evening, I organized my notes on Ito's lemma and Black-Scholes and uploaded them to GitHub.

Apparently, my daughter had a sports day at nursery school; when I picked her up, she was wearing an origami gold medal around her neck.

I've been full since lunch, so I skipped dinner.

Spent the evening looking through the category theory feature in the magazine.

Tried making a GitHub Page right before bed.

\section{October 11, 2025 (Saturday)}

Woke up in the morning to a commotion: my daughter had spilled milk in the futon room.

Started packing for Gunma, finished the rough packing, and then lay down for what I thought would be a short rest, but ended up sleeping until just before noon.

We finally left the house at 1:30 PM.
Traveling in the rain with a suitcase and a child in a stroller was quite tough.

My child spilled milk on the train. A foreigner helped pick it up.
It would have been a good joke if I could have immediately said 'It's no use crying over split milk,' but the phrase wouldn't come to me, and I just ended up expressing simple gratitude.

Arriving in Takasaki, it was cold.
So cold it felt like the season had changed while we were on the train.

My mother-in-law picked us up by car, and she had a meal ready for us.
Chestnut rice, clear soup with Chinese cabbage, and persimmons.

It was good to be able to let my mother-in-law see our daughter.

My daughter was excitedly running around the spacious house.

The bedding is a bit dusty; it's making me cough.

\section{October 12, 2025 (Sunday)}

Woke up around 8 AM.
My wife and daughter had apparently been up for ages.
My sleep was shallow, as if I had hay fever. My mouth is parched, and my nose won't stop running.
Remembered that the countryside has its own unique difficulties.

Breakfast was udon noodles added to yesterday's stew, and homemade egg sandwiches.

Helped my mother-in-law with managing her stocks and investment trusts.

Had a light meal before noon: chestnut rice balls, a banana, and dried pomelo fruit.

I lay down with my daughter to put her to sleep, but I was the only one who napped, and when I woke up, it was already evening.

She made yakiniku for us.

Today ended up being just a day of eating and sleeping.

\section{October 13, 2025 (Monday, Holiday)}

Having slept nearly 14 hours yesterday, I was up early. Woke around 5 AM.
Started reading a corporate finance theory textbook, only to fall asleep again.

Woke up at 7 and picked persimmons from the tree in the garden with my wife.
There were so many persimmons on the tree, and we picked a lot.
About 30 or 40.

More than we can eat, so we decided to send them to my parents in Osaka.

Got on the train home around 9:40.

Boarded the Shinkansen around 10:40.
Couldn't get a seat in the unreserved car.
Spent the ride standing.

Was trying to understand the definition of a Galois group.

On the Shinkansen, someone who appeared to be a foreigner showed me their ticket and asked in Japanese where their seat was, as they couldn't tell.
The ticket was for unreserved seating, so I explained in English that they could sit in any available seat in cars 1 through 5.
They replied in very good Japanese, 'Naruhodo, wakarimashita. Arigatou.' (I see, I understand. Thank you.)
According to my wife, who was watching, it looked like a comedy sketch: me explaining in English to someone who was speaking perfectly good Japanese.

\section{October 14, 2025 (Tuesday)}

A day spent contemplating field theory and topology.

Had two back-to-back meetings in the afternoon.
In the marketing team meeting, we discussed AR models.
The idea came up that we could explain the causal effect of new measures by treating them as an external field perturbation.

Also had a meeting about data acquisition from PORTERS.

Since then, development has progressed, and I'm now able to bulk-fetch two years' worth of registrant data.

Also set up the analysis batch for new and recent graduates to run periodically.

Went to the hospital in the evening.

I must have been completely exhausted from yesterday; I've been feeling depressed.

\section{October 15, 2025 (Wednesday)}

Woke up around 7 AM, showered, took out the trash, and went to Maruzen in Marunouchi.

'Modules and Homological Algebra' and 'Fiber Bundles and Homotopy' are math textbooks based on category theory. Since I don't know modules or homotopy yet, I was browsing them thinking they looked excellent.

Had my company review meeting right at the last minute.
I told them I would extend for another three months.
The program might restart from January 2026, apparently.

Ate tsukemen at Tomita until I was stuffed.

Went back to Maruzen again and ended up buying all three: 'Modules and Homological Algebra,' 'Fiber Bundles and Homotopy,' and 'Linear Algebra Dialogue Vol. 2.'
12,000 yen.

Noted that there were also several good textbooks on quantum field theory in condensed matter physics.

Continued with development work after returning home.

Also attended MBA lectures.
Marketing and Operations Management.

Wrote notes on the Radon-Nikodym derivative and the Cameron-Martin-Girsanov theorem.

Bed around 9 PM.

\section{October 16, 2025 (Thursday)}

Started the morning creating notes on pricing derivatives for foreign exchange.

Shaved for the first time in a while, showered, and left the house around 9:30 AM.

Arrived in Takadanobaba around 10:30 and browsed a bookstore.

Got to the Russian restaurant exactly at 11:30, but my friend was already queueing before it opened, so we got in.

Had the beef stroganoff. It was the most delicious thing I've eaten in some time.

Ate cake at a coffee shop, then parted ways.

Devoted myself to development work after getting home.

Had a meeting with the marketing manager, but the request was for a system development that would make something highly complex even more complex, so I proposed alternative solutions.

Reading a book on Galois theory, but I just can't seem to grasp it.
The reading itself is progressing, but I'm not practicing explaining it in my own words, which might be why it doesn't feel like I understand it.

I want to study groups and modules as well.
My reading on homological algebra is stalled because I have no knowledge of modules.

\section{October 17, 2025 (Friday)}

A day spent relaxing at home all day, for the first time in a while.

Had a light chat with my senior colleague in the morning.

I asked if Lovelock's theory of gravity would be a good next step after learning the basics of general relativity.
Apparently, it's quite a leap to get to Lovelock's theory.

He recommended the black hole textbook 'A Relativist's Toolkit.'

My daughter's medical insurance card was missing from the house, so I had no choice but to go to the health center and get it reissued.

Ate Uzbek food with my wife while we were out.

Devoted myself to development work after returning home.

Chatted with a different developer from marketing than yesterday.
This, too, was a request for a system development that would make something highly complex even more complex.

After agonizing over it, I spoke with my senior colleague again and had him help sort out the tasks.

\section{October 18, 2025 (Saturday)}

My daughter seemed unwell from the morning, having trouble breathing.

Her temperature was normal in the morning.

Took her to a nearby hospital and got some cold medicine.

When they took her temperature at the hospital, it was nearly 39 degrees (C).

We all had kebabs for lunch.

It took about two hours to get the medicine from the pharmacy.

While waiting, I was reading a book on renormalization theory.

The correspondence between gauge theory and $\phi^{4}$ theory was interesting.

I've never studied the renormalization group equations, so I'm thinking I'd like to summarize them in my notes at some point.

Perhaps thanks to the medicine, my daughter's condition was much better by night.

\section{October 19, 2025 (Sunday)}

Went for our customary 'Asa Mac' (McDonald's breakfast).

Studied and summarized adjunctions, an unavoidable topic in learning category theory.

Awodey's textbook is quite clear and good.

Now that I've summarized the unit of an adjunction, I want to write about the counit and the zigzag identities next.

My daughter is fully recovered and incredibly energetic, running wild in our small house with excess energy.

\section{October 20, 2025 (Monday)}

Read a book on statistics before bed.

A thought occurred to me: perhaps statistics is what allows things like category theory, physics, and financial engineering to be brought down to the real world.

In other words, mathematics like category theory isn't normally used in society, but when it's connected to statistics---and statistics \textit{is} connected to society---perhaps statistics can serve as the 'glue' that binds abstract math to society.

Organizing these disjointed thoughts leads to the following:

Observation: On the mediating role of statistics.
\begin{itemize}
    \item Idea: Statistics might act as the intermediary (glue) for applying abstract mathematics (e.g., category theory) and theoretical disciplines (e.g., physics, financial engineering) to the 'real world.'
    \item Background assumption:
    \begin{itemize}
        \item Advanced mathematics like category theory is rarely applied directly in society.
        \item Statistics, on the other hand, is intimately connected to various societal phenomena (data).
    \end{itemize}
    \item Inference:
    \begin{itemize}
        \item By functioning as the 'mediator' connecting abstract theory and real-world society, statistics may make it possible to leverage these theories for practical problem-solving.
    \end{itemize}
    \item Conclusion:
    \begin{itemize}
        \item Derived the hypothesis that statistics can fulfill the role of 'glue' connecting abstract mathematics/theory and real-world society.
    \end{itemize}
\end{itemize}

12:56 PM
Arrived at Butayama, the ramen shop in Kitasenju.

Luckily, got into Butayama without waiting.

Should have scanned the visit stamp \textit{before} buying the meal ticket.

In the evening, I wrote up practice notes on using the fRG Wetterich equation and git push'ed them.

I was already interested in the renormalization group, but now I'm even more interested and want to study it systematically.

\section{October 21, 2025 (Tuesday)}

There was a water outage in the apartment building this morning.

Woke up and immediately took a shower, did laundry, and finished all water-related chores.

Also made progress on development and posted a status update in the meeting minutes.

Had a network connection issue and needed to manually run all of today's batch jobs.
I really want to migrate to AWS EC2 soon.

Transferred money between bank accounts.

After that, went to the family restaurant right near my house, ate Peperoncino and chicken steak, and read Clarke's 'Childhood's End.'
Incredibly interesting.
It made me think that I, too, am ultimately just doing the work of transferring values from one ledger account to another.

\section{October 22, 2025 (Wednesday)}

A day spent at the Nishi-Shinjuku office.

Heard that a friend is mentally exhausted.

The Looker Studio report I'd been developing for a while is finally complete.

At night, I talked about data science at the request of a friend.

Gained some new challenges, too. All in all, a good day.

\section{October 23, 2025 (Thursday)}

Had soup-less dandan noodles for breakfast.

Attended an MBA lecture on research ethics.

Had a chilling realization, remembering that even simple notes must properly cite their sources.

The Looker Studio report seems to be getting used well.

Scheduled a meeting to lecture the company staff on it.

In the afternoon, went to Shinjuku with a friend to eat Romanian food.
For such an impressive-looking exterior, the prices were normal, which was nice.

Chatted at a cafe, then parted ways.

Spent the evening relaxing at home.

\section{October 24, 2025 (Friday)}

Went to eat tsukemen with my wife before noon.

'Submanifolds in Theoretical Physics' arrived.

Dug up my old notes from my student days on the Usadel equation at superconductor/metal interfaces and git push'ed them.

Ten hours of sleep at night plus a two-hour nap makes 12 hours of sleep.

My daughter's cough won't stop; it's worrying.

The book on high-frequency trading is starting to seem interesting.
Maybe my health is starting to improve.

\section{October 25, 2025 (Saturday)}

Been reading about algorithmic trading since 4 AM.

My daughter's cough wouldn't stop, so we went to the hospital in the morning.

We all had kebabs for lunch.
There's a kebab shop near the hospital, so we tend to get them when we have a hospital visit.

My wife had her own hospital appointment in the afternoon.

I took a nap.

Summarized a bit about algorithmic trading in a mind map, and that's it for reading today.

'Statistics is the Strongest Science (Practical Edition)' was a very good book.
Want to summarize this too.

Purchased Ian Goodfellow's 'Deep Learning,' Nawa's 'Schumpeter,' Takamura's 'Introduction to Functional Analysis,' and Kazuo Matsuzaka's 'Introduction to Algebraic Systems.'

Also bought 'Quantum Many-Body Physics and Artificial Neural Networks.'

\section{October 26, 2025 (Sunday)}

The whole family was up around 5 AM.

Was trying to make slides in \LaTeX.
It's my first time making \LaTeX slides.
It's fun to experiment with different things.

Had our weekly 'Asa Mac.'

To stop my daughter's habit of putting her fingers in her mouth, we tried a bitter-tasting nail polish that toddlers are supposed to hate licking.

She licked her fingers, found it bitter, but kept licking them anyway, and then she couldn't stop throwing up all day. It was tough for everyone.

My wife is interested in statistics, so I created some simple slides on statistics for her.

\end{document}
\documentclass[dvipdfmx, autodetect-engine, aspectratio=169, 10.5pt]{beamer}

\usepackage{amsmath}
\usepackage{amssymb}
\usepackage{amsthm}
\usepackage{graphicx}
\usepackage{hyperref}
\usepackage{enumitem}
\usepackage[english]{babel}

\usetheme{Boadilla}
\usetheme{Marburg}
\usecolortheme{orchid}
\usefonttheme{professionalfonts}

\title{
Log.
}

\author{
M. O.
}

\date{Oct. 2025}

\begin{document}
\maketitle

\section{October 1, 2025 (Wed)}

\begin{frame}{October 1, 2025 (Wed)}
Started the morning at a family restaurant, reading a category theory text and Kojin Karatani's 'Powers and Modes of Exchange' over breakfast.

It got to be past noon, so I ordered lunch there as well.

Living quite elegantly.

Gave my wife her allowance for the month, but contrary to the morning's elegance, our household budget is tight.

For my side job's analysis of new and recent graduates, I've made it runnable on AWS Lambda, but it's a mess and desperately needs refactoring.

Was deeply impressed reading the beginning of Mac Lane's category theory textbook, which explained how commutative diagrams in category theory can unify the understanding of groups, topological groups, and Lie groups.

Feeling fortunate to be able to appreciate the beauty of physics and the beauty of mathematics.
\end{frame}

\section{October 2, 2025 (Thu)}

\begin{frame}{October 2, 2025 (Thu)}
Planned to go for Kenyan food with a friend for lunch, but the restaurant was closed.

We ended up going to our usual Peruvian place in Gotanda instead.
It was delicious as always, so it turned out very well.

Chatted with my friend over coffee at a cafe after the meal.

We talked about how much being in a jock-style sports club in school helps with adapting to society.

Also heard about how students at the prep school who struggle with math often can't even copy the formulas correctly, let alone understand what's in the textbook.

My friend apparently got accepted into the medical school prep academy and is already teaching three students.

We parted ways after the cafe, and then I headed to the office in Shinjuku.

It was good to talk with my friend and also with my team members.

Stopped by a used bookstore in Shinjuku on the way home, getting back after 8 PM.
\end{frame}

\section{October 3, 2025 (Fri)}

\begin{frame}{October 3, 2025 (Fri)}
\scriptsize
Spent the entire day immersed in development for my side job.

The analysis batch for Yomi Sheet is mostly complete.

Talked with the two leaders about CA activity analysis, and with another person developing a separate CA performance analysis system.

It seems the number of job offers accepted is off by an order of magnitude.

Suspecting it's due to the order of operations: whether to first process the 30-day or 90-day cutoffs, or to first set the flag for the most advanced process.

Went out for dinner with a friend in Kita-Senju.

Kita-Senju has many potential biryani places, but they have irregular holidays, making it hard to catch them open.

Ended up eating at Butayama, a Ramen Jiro-style place, again today.

Managed to give my friend the Hannah Arendt book and the causal inference book.

On the way back, we stopped at our usual standing bar for a beer and a long talk.

Shared my desire to re-enroll in a doctoral program, mentioning I've even narrowed down specific labs.

My friend seems to be getting interested in corporate finance theory.

Uploaded my notes on the Keldysh Green Function after getting home.
\end{frame}

\section{October 4, 2025 (Sat)}

\begin{frame}{October 4, 2025 (Sat)}
A day spent mostly sleeping.

Richard Rorty's 'Contingency, Irony, and Solidarity' arrived. Want to read it.

As the date changed---meaning, as my age changed---I was studying the part where the comma category, which becomes equal to the Grothendieck construction, is expressed as the category of elements using the integral sign for categories.

Surprised to find that integration, and even Fubini's theorem, appears in category theory.
\end{frame}

\section{October 5, 2025 (Sun)}

\begin{frame}{October 5, 2025 (Sun)}
\scriptsize
Woke up feeling a bit sleep-deprived.

Made some corrections to my Fermi liquid notes.

Studying limits in category theory.

It's my birthday, so we planned to go out for sushi, but my daughter was rampaging around the house, and my wife got exhausted. We ended up just buying prepared food from a bento shop.

Ammon and Erdmenger's book on gauge/gravity duality happens to be available used, but it's not that cheap.
Had been debating buying it for a while, but thought I might as it's my birthday.

Managed to obtain it through some unexpected channel.

Along with it, also got Altland and Simons' textbook, Nakahara's topology textbook, and Xiao-Gang Wen's textbook. Lucky.

Also picked up Audrey Tang's 'Plurality' and Milton Friedman's 'Capitalism and Freedom' at a used bookstore.

Tried creating this 'Misc.' note.

My wife and daughter went to bed early, giving me a long evening to myself.

Savoring the luxury of relaxing and reading.

Emily Riehl's category theory textbook is starting to look very good.
Thinking I'd like to summarize the Yoneda lemma in my notes.

Barely did anything eventful, just walked outside a bit for shopping. It was a very peaceful day, and the best birthday I've had in my entire life.
\end{frame}

\section{October 6, 2025 (Mon)}

\begin{frame}{October 6, 2025 (Mon)}
\scriptsize
Woke up at 5:55 AM. Got eight and a half hours of sleep.

Studied and summarized the basics of limits in category theory until about 8:00.

Around 9:00, enjoyed breakfast at the nearby family restaurant for the first time in a while.

Was reading Hannah Arendt.

Realized I'd forgotten my house key, but my wife was home, which was a relief.

In the afternoon, I traced the proof of the Yoneda lemma and tried to summarize it in my own way.

Though I tried to write the proof to make it understandable for myself, it still feels complicated. Maybe the notes aren't very good, or perhaps I'm just not used to category theory yet.

Lunch was a bento from the bento shop, eaten relaxing with my wife.

Organized my tasks, and my wife helped me with some administrative work to submit company documents.

Attended an MBA lecture as well.

Have been sitting in front of the computer a lot lately. Want to go to a sauna tomorrow and relax.
\end{frame}

\section{October 7, 2025 (Tue)}

\begin{frame}{October 7, 2025 (Tue)}
To bed at 11 PM, up at 6 AM. Seven hours of sleep.

A cockroach appeared at home, so I didn't want to stay there and spent the day in a private sauna room.

The Rubik's Cube and group theory came up in conversation at work, making me realize I haven't studied group theory in a long time.

Want to make sure I can still follow proofs for group representation theory and other basic theorems.

Bought a book on time series analysis.

Wondering if P-measure folks use things like ARIMA processes.

Want to start by studying AR models.

Watched the Nobel Prize in Physics announcement live again this year.

It was for the foundations of quantum computing using Josephson junctions. Thrilled.

Bought books on superstring theory, gauge theory, and the renormalization group.
\end{frame}

\section{October 8, 2025 (Wed)}

\begin{frame}{October 8, 2025 (Wed)}
\scriptsize
Slept around 9:30 PM last night and woke up around 8:30 AM.

Eleven hours of sleep. Feeling good after sleeping so well.

Nakahara's 'Geometry, Topology and Physics' arrived, so I spent the morning reading it leisurely.

Before noon, had a meeting to learn about the business processes for CAs.

Got a sense that if we can provide optimized metrics for each CA, everyone could work with more peace of mind.

Lunch was fried chicken and such from a prepared food shop.

Bought some office supplies and consumables while out buying diapers for my daughter.

Enjoyed some window shopping at a bookstore.

Came home in a good mood after a pleasant walk, only to realize I'd forgotten the one thing I went out to buy: the diapers.

Spent the afternoon reading the time series analysis textbook.

Got a rough grasp of how AR models work.

It seems MA models are useful when the order $p$ of the AR model is too large; want to learn about that mechanism next.

Also bought a textbook on CP symmetry violation.

Apparently, a Japanese researcher also won the Nobel Prize in Chemistry this year. Congratulations.

It was for work on porous metal-organic frameworks, and it's wonderful that it's in the field of condensed matter. Happy about that.

Today was another good day.
\end{frame}

\section{October 9, 2025 (Thu)}

\begin{frame}{October 9, 2025 (Thu)}
\scriptsize
Lately, it's become a habit to wake up and, while rubbing my eyes, start programming immediately.

Spent the morning just coding.

Had lunch with my wife and a friend at a Uyghur restaurant in Ueno.

Enjoyed Laghman and Kvass, a type of alcoholic drink.

My stomach felt off, either from overeating or the spices, and I couldn't do anything until the evening.

Reviewed quantum information theory, prompted by the Nobel Prize.

A textbook on complex manifolds arrived.

I've seen many of the terms in complex geometry, but I don't get the concepts at all and can't make progress reading.

Feeling a lack of basic knowledge in algebra.

It seems necessary to grasp the outline of homological algebra using Nakahara's book.

Studying Galois theory might also be good if it helps build algebraic concepts.

Understanding algebra would deepen my understanding of category theory, too. Feeling like algebra is the bottleneck.
\end{frame}

\section{October 10, 2025 (Fri)}

\begin{frame}{October 10, 2025 (Fri)}
\scriptsize
In the morning, I dug up and git-pushed two notes on superconductivity, an article on financial engineering, and a memo about American options on non-dividend-paying assets always being cheaper than European options.

Went for lunch with my wife at an Indian curry place near the station.

The spicy spinach curry and cheese naan were delicious.

Got a second helping of naan and ate way too much.

A big batch of textbooks on superstring theory, gauge theory, and complex analysis arrived.

Tried to go for a haircut, but the barbershop was closed again.

On the way to the barbershop, I stopped at a bookstore and bought this month's issue of 'Sugaku Seminar' magazine, which came out today and features category theory.

In the evening, I organized and uploaded my notes on Ito's lemma and Black-Scholes to GitHub.

My daughter apparently had a sports day at daycare; when I picked her up, she was wearing a gold medal origami around her neck.

Still full from lunch, so I skipped dinner.

Spent the night looking through the category theory feature in the magazine.

Tried setting up a GitHub Page right before bed.
\end{frame}

\section{October 11, 2025 (Sat)}

\begin{frame}{October 11, 2025 (Sat)}
Woke up to a commotion: my daughter had spilled milk in the futon room.

Started packing for Gunma, roughly finished, and intended to lie down for just a bit, but ended up sleeping until just before noon.

We finally left the house at 1:30 PM.

Traveling in the rain with a suitcase and a child in a stroller was quite tough.

My daughter spilled milk on the train. A foreigner helped clean it up.
It would have been a good joke if I could have instantly said 'It's no use crying over spilt milk', but the phrase didn't come to me, and I just ended up expressing my gratitude normally.

It was cold when we arrived in Takasaki. So cold it felt like the season had changed during the train ride.

My mother-in-law picked us up by car, and we had a meal.
Chestnut rice, Chinese cabbage soup, and persimmons.

It was good that my mother-in-law got to see my daughter.

My daughter was running around excitedly in the spacious house.

The bedding is a bit dusty, making me cough.
\end{frame}

\section{October 12, 2025 (Sun)}

\begin{frame}{October 12, 2025 (Sun)}
Woke up around 8 AM. My wife and daughter were apparently up long before.

My allergies flared up like hay fever, so I slept poorly.
My mouth is dry and my nose won't stop running.
Remembered that the countryside has its own unique difficulties.

For breakfast, we had udon added to yesterday's stew and homemade egg sandwiches.

Helped my mother-in-law with managing her stocks and mutual funds.

Before noon, had a snack of chestnut rice balls, a banana, and dried pomelo.

I lay down with my daughter to put her to sleep but ended up taking a nap myself. Woke up, and it was already evening.

My mother-in-law made yakiniku for us.

Today turned into a day of just eating and sleeping.
\end{frame}

\section{October 13, 2025 (Mon)}

\begin{frame}{October 13, 2025 (Mon)}
\scriptsize
Slept nearly 14 hours yesterday, so I woke up early. Got up around 5 AM.
Was reading a corporate finance theory textbook and fell asleep again.

Woke up again at 7 AM and went into the garden with my wife to pick persimmons from the tree.
There were so many on the tree, and we picked a lot. About 30 or 40.
More than we can eat, so we decided to send them to my parents in Osaka.

Caught the train back around 9:40 AM.

Got on the Shinkansen around 10:40. No seats in the non-reserved section.
Had to stand.

Was trying to understand the definition of a Galois group.

On the Shinkansen, someone who appeared to be a foreigner showed me their ticket and asked in Japanese where their seat was, as they couldn't tell.
The ticket was for the non-reserved section, so I explained in English that on this train, they could sit in any available seat in cars 1 through 5.
They replied in perfect Japanese, 'Naruhodo, wakarimashita. Arigatou.' (I see, I understand. Thank you.)
My wife, who was watching, said it was like a comedy skit, me explaining in English to someone who was clearly Japanese.
\end{frame}

\section{October 14, 2025 (Tue)}

\begin{frame}{October 14, 2025 (Tue)}
A day spent lost in thought about field theory and topology.

Had two back-to-back meetings in the afternoon.
In the marketing team meeting, we discussed AR models.
The idea came up that we could explain the causal effect of new measures by treating them as an external field perturbation.

Also had a meeting about retrieving data from PORTERS.

Made progress on development, and it's now possible to bulk-fetch two years' worth of registrant data.

Also set up the analysis batch for new and recent graduate registrants to run periodically.

Went to the hospital in the evening.

Perhaps from being so exhausted yesterday, I've fallen into a bit of a depression.
\end{frame}

\section{October 15, 2025 (Wed)}

\begin{frame}{October 15, 2025 (Wed)}
\scriptsize
Woke up around 7 AM, showered, took out the trash, and went to Maruzen in Marunouchi.

Was browsing 'Modules and Homological Algebra' and 'Fiber Bundles and Homotopy', which are math textbooks based on category theory. Since I don't know modules or homotopy yet, I thought they looked great.

Right at the last minute, had my company interview.
Told them I would extend for another three months.
Apparently, the program might restart in January 2026.

Ate my fill of tsukemen at Tomita.

Went back to Maruzen and ended up buying three books: 'Modules and Homological Algebra', 'Fiber Bundles and Homotopy', and 'Linear Algebra Dialogue Vol. 2'.
12,000 yen.

There were several other good textbooks on quantum field theory in condensed matter physics, too.

Returned home and continued with development.

Also attended MBA lectures: Marketing and Operations Management.

Wrote notes on the Radon-Nikodym derivative and the Cameron-Martin-Girsanov theorem.

Went to bed around 9 PM.
\end{frame}

\section{October 16, 2025 (Thu)}

\begin{frame}{October 16, 2025 (Thu)}
\scriptsize
Spent the morning making notes on pricing foreign exchange derivatives.

Shaved and showered for the first time in a while, leaving the house around 9:30 AM.

Arrived in Takadanobaba around 10:30 and browsed a bookstore.

Got to the Russian restaurant right at 11:30. My friend had already been waiting in line before it opened, so we got in.

Had the beef stroganoff. It was the most delicious thing I've eaten in a while.

We had cake at a cafe and then parted ways.

After returning home, I buckled down on development work.

Had a meeting with someone from marketing, but the request was for a system development that would make something already highly complex even more complicated, so I proposed alternative solutions.

Reading a book on Galois theory, but I just can't seem to grasp it.
The reading itself is progressing, but I haven't been practicing explaining it in my own words, which might be why I don't feel like I understand it.

Want to study groups and modules as well.
My reading on homological algebra is stalled because I have no knowledge of modules.
\end{frame}

\section{October 17, 2025 (Fri)}

\begin{frame}{October 17, 2025 (Fri)}
A day spent relaxing at home all day for the first time in a while.

Had a light chat with a senior colleague in the morning.

I asked if Lovelock's theory of gravity would be a good next step after learning the basics of general relativity.
Apparently, there's quite a gap to bridge before tackling Lovelock's theory.
He recommended the black hole textbook 'A Relativist Toolkit'.

My daughter's medical certificate was missing from the house, so I had no choice but to go to the health center and get it reissued.

While we were out, my wife and I had Uzbek food.

After returning home, I got back to development work.

Chatted with a different marketing developer than yesterday.
This was, likewise, a request to develop a system that would make something highly complex even more complex.

After agonizing over it, I spoke with my senior colleague again and had him help sort out the tasks.
\end{frame}

\section{October 18, 2025 (Sat)}

\begin{frame}{October 18, 2025 (Sat)}
My daughter seemed unwell since the morning, having trouble breathing.

Her temperature was normal in the morning.

Took her to a nearby clinic and got cold medicine.
When they checked her temperature at the clinic, it was nearly 39 degrees.

We all had kebab for lunch.

It took about two hours to get the medicine from the pharmacy.
While waiting, I read a book on renormalization theory.

The correspondence between gauge theory and $\phi^{4}$ theory was interesting.

I've never studied the renormalization group equations, so I'm thinking I'd like to summarize them in my notes at some point.

The medicine must have worked, as my daughter's condition had improved greatly by the evening.
\end{frame}

\section{October 19, 2025 (Sun)}

\begin{frame}{October 19, 2025 (Sun)}
Went for our customary morning McDonald's.

Studied and summarized adjunctions, an unavoidable topic in learning category theory.

Awodey's textbook is very clear and good.

Managed to summarize the unit of an adjunction, so next I want to cover the counit and the zigzag identities.

My daughter has fully recovered and is incredibly energetic, running wild in our small house with too much energy.
\end{frame}

\section{October 20, 2025 (Mon)}

\begin{frame}{October 20, 2025 (Mon)}
\tiny
Fell asleep reading a book on statistics.

A thought occurred to me: maybe statistics can be what grounds category theory, physics, and financial engineering in the real world.

That is, while advanced mathematics like category theory isn't typically used in society directly, statistics is deeply connected to society. So, when mathematics is linked with statistics, statistics could act as the 'glue' connecting abstract math to society.

Organizing these disjointed thoughts leads to the following:

\textbf{Consideration: On the Mediating Role of Statistics}
\begin{itemize}
    \item \textbf{Idea:} Statistics might serve as a bridge (or 'glue') for applying abstract mathematics (e.g., category theory) and theoretical disciplines (e.g., physics, financial engineering) to the 'real world'.
    \item \textbf{Background Perception:}
    \begin{itemize}
        \item Advanced mathematics like category theory is rarely applied directly in society.
        \item Statistics, on the other hand, is intimately linked to various societal phenomena (data).
    \end{itemize}
    \item \textbf{Inference:}
    \begin{itemize}
        \item By functioning as a 'mediator' connecting abstract theory and the real world, statistics may enable the use of these theories for practical problem-solving.
    \end{itemize}
\end{itemize}

\textbf{Conclusion:}
Derived the hypothesis that statistics can fulfill the role of 'glue' connecting abstract mathematics and theory with real-world society.

12:56 PM
At the ramen shop Butayama in Kita-Senju.

Luckily, got into Butayama without waiting.

Should have scanned the loyalty stamp reader \textit{before} buying the meal ticket.

From the evening, wrote some practice notes on using the Wetterich equation for fRG and git-pushed them.

I was originally interested in the renormalization group, but my interest has grown even stronger, so I want to study it systematically.
\end{frame}

\section{October 21, 2025 (Tue)}

\begin{frame}{October 21, 2025 (Tue)}
There was a water outage in the apartment building this morning for construction.

Woke up and immediately took a shower and did the laundry to get water-related chores done.

Made progress on development and posted a progress report in the meeting minutes.

There was a network connection issue, forcing me to run all of today's batch jobs manually.
Really want to migrate to AWS EC2 soon.

Transferred some money between bank accounts.

Then, went to the family restaurant right near my building, eating peperoncino and a chicken steak while reading Clarke's 'Childhood's End'.
It's incredibly good.
Made me think that I, too, am ultimately just doing work that transfers values from one ledger account to another.
\end{frame}

\section{October 22, 2025 (Wed)}

\begin{frame}{October 22, 2025 (Wed)}
A day I went to the Nishi-Shinjuku office.

Heard that a friend is feeling mentally drained.

The Looker Studio report I had been developing for a while is finally complete.

In the evening, I gave a talk on data science at a friend's request.

Gained some new challenges to think about, so all in all, it was a good day.
\end{frame}

\section{October 23, 2025 (Thu)}

\begin{frame}{October 23, 2025 (Thu)}
Had soup-less dandan noodles for breakfast.

Attended an MBA lecture on research ethics.

Had a minor scare remembering that even simple notes require citations.

The Looker Studio report seems to be getting used well.

Also set a meeting date to give a lecture on it to people at the company.

In the afternoon, went with a friend to eat Romanian food in Shinjuku.
It was reasonably priced for its appearance, which was nice.

Chatted at a cafe and then went our separate ways.

Spent the evening relaxing at home.
\end{frame}

\section{October 24, 2025 (Fri)}

\begin{frame}{October 24, 2025 (Fri)}
Went to eat tsukemen with my wife before noon.

'Submanifolds in Theoretical Physics' arrived.

Dug up and git-pushed some notes I wrote in my student days about the Usadel equation at superconductor-metal interfaces.

Got 12 hours of sleep: 10 hours at night plus a 2-hour nap.

My daughter's cough won't stop, which is worrying.

A book on high-frequency trading is starting to seem interesting.
Maybe my health is on the mend.
\end{frame}

\section{October 25, 2025 (Sat)}

\begin{frame}{October 25, 2025 (Sat)}
Reading about algorithmic trading since 4 AM.

My daughter's cough still hasn't stopped, so we went to the clinic in the morning.

We all ate kebab for lunch.
There's a kebab shop near the clinic, so we tend to get kebab on clinic days.

In the afternoon, my wife went to the doctor.

I took a nap.

Summarized some thoughts on algorithmic trading in a mind map, and that's it for reading today.

'Statistics is the Strongest Science (Practical Edition)' was a very good book. Want to summarize that, too.

Bought Ian Goodfellow's 'Deep Learning', Nawa's 'Schumpeter', Takamura's 'Introduction to Functional Analysis', and Kazuo Matsuzaka's 'Introduction to Algebraic Systems'.

Also bought 'Quantum Many-Body Physics and Artificial Neural Networks'.
\end{frame}

\section{October 26, 2025 (Sun)}

\begin{frame}{October 26, 2025 (Sun)}
The whole family woke up around 5 AM.

Tried making slides with \LaTeX.
It was my first time making \LaTeX\ slides. It's fun to experiment with.

Had our weekly customary morning McDonald's.

To stop my daughter's habit of putting her fingers in her mouth, we tried a bitter-tasting nail polish that toddlers are supposed to hate.

She licked her fingers, found it bitter, but kept licking them anyway. As a result, she couldn't stop throwing up all day, which was rough on everyone.

My wife is interested in statistics, so I created a simple presentation on it for her.
\end{frame}

\end{document}
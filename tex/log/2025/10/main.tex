\documentclass[dvipdfmx, autodetect-engine, aspectratio=169, 10.5pt]{beamer}

\usepackage{amsmath}
\usepackage{amssymb}
\usepackage{amsthm}
\usepackage{graphicx}
\usepackage{hyperref}
\usepackage{enumitem}
\usepackage[english]{babel}

\usetheme{Boadilla}
\usetheme{Marburg}
\usecolortheme{orchid}
\usefonttheme{professionalfonts}

\title{
Log.
}

\author{
M. O.
}

\date{Oct. 2025}

\begin{document}
\maketitle

\section{Wednesday, October 1, 2025}

\begin{frame}{Wednesday, October 1, 2025}
	The morning was spent at a family restaurant, enjoying a morning set while reading a Category Theory textbook and Kojin Karatani's 'The Structure of Exchange and Power'.

	The time shifted to afternoon, so lunch was ordered right there.

	Truly, living a refined life.

	Handed my wife this month's allowance, yet the household budget is tight, contrary to the morning's elegance.

	The analysis tool for new grads/recent grads for the side job was built to run on AWS Lambda, but the code desperately needs refactoring.

	Was deeply impressed by the opening of Mac Lane's category theory textbook, where it mentions how groups, topological groups, and Lie groups can be unifiedly understood using commutative diagrams.

	Feeling fortunate to be moved by both the beauty of physics and the elegance of mathematics.
\end{frame}

\section{Thursday, October 2, 2025}

\begin{frame}{Thursday, October 2, 2025}
	Had plans to have Kenyan cuisine with a friend for lunch, but the restaurant was closed.

	Instead, went to the usual Peruvian place in Gotanda.
	As always, it was delicious, so the outing turned out great in the end.

	Enjoyed a coffee and a chat with a friend at a cafe after the meal.

	The topic of how much participating in sports clubs during student days helps with social adaptation came up.

	Also heard about a student struggling with math at the prep school, noting they can't even correctly transcribe formulas written in the textbook.

	My friend successfully passed the exam for a medical school prep school, and not only that, has already started tutoring three students.

	Parted ways after the cafe and headed to the company office in Shinjuku.

	It was good to talk with my friend and team members.

	Stopping by a secondhand bookstore in Shinjuku before heading home meant arriving back past 8 PM.
\end{frame}

\section{Friday, October 3, 2025}

\begin{frame}{Friday, October 3, 2025}
	Spent the entire day focused on development for the side job.

	The analysis batch for Yomi Sheet is mostly complete.

	Discussed the CA activity analysis with two leaders, and separately with a person developing an analysis system for CA performance.

	The number of accepted offers seems to be off by a single digit.

	Suspecting the discrepancy is due to the order of operations: whether the 30-day or 90-day cut-off harvesting is done first, or if the flag for the most advanced process is set first.

	Went out for dinner with a friend in Kita-Senju in the evening.

	Kita-Senju has many places that potentially serve Biryani, but they often have irregular closing days, making it hard to find.

	Ended up eating Butayama (a Jiro-style ramen) again today.

	Handed over books to my friend, including one by Hannah Arendt and one on Causal Inference.

	Had a relaxing conversation at our usual standing bar for beer on the way back.

	Shared my intention to re-enter a doctoral program, even mentioning specific labs I'm considering.

	The friend seems to have developed an interest in corporate finance theory.

	After returning, uploaded my notes on the Keldysh Green Function.
\end{frame}

\section{Saturday, October 4, 2025}

\begin{frame}{Saturday, October 4, 2025}
	Slept nearly the whole day.

	Richard Rorty's 'Contingency, Irony, and Solidarity' arrived. Looking forward to reading it.

	As the date changed, marking a change in age, I was studying how the comma category, which is equivalent to the Grothendieck construction, can be represented using the category integral notation as the category of elements.
	It was surprising to find integrals in category theory, even including Fubini's theorem.
\end{frame}

\section{Sunday, October 5, 2025}

\begin{frame}{Sunday, October 5, 2025}
	\scriptsize
	Woke up slightly sleep-deprived this morning.

	Made a small correction to the Fermi Liquid notes.

	Studying limits in category theory.

	Today being my birthday, planned to go out for sushi, but my daughter was running rampant in the house, exhausting my wife, so ended up buying side dishes from a deli instead.

	Ammon and Erd-menger's book on Gauge/Gravity Duality popped up used, but it's not cheap.
	Had been considering buying it for a while and thought about getting it for my birthday.

	Fortunately, managed to acquire it somehow.

	Also managed to get Altrand and Simons' textbook, Nakahara's Topology textbook, and Xiao-Gang Wen's textbook.
	Lucky day.

	At the secondhand bookstore, also picked up Audrey Tang's 'Plurality' and
	Milton Friedman's 'Capitalism and Freedom'.

	Created this Misc. log document.

	My wife and daughter went to bed early, giving me an extended period of solitary time in the evening.

	Savoring this luxurious time to relax and read.

	Emily Riehl's category theory textbook is starting to look quite promising.
	Feel like I want to summarize the section on the Yoneda Lemma in my notes.

	Beyond a short walk outside for shopping, there were no major events; it was a very peaceful day and the best birthday of my life so far.
\end{frame}

\section{Monday, October 6, 2025}

\begin{frame}{Monday, October 6, 2025}
	Woke up at 5:55 AM.
	Slept for eight and a half hours.

	Studied and summarized the basics of limits in category theory until about 8 AM.

	Around 9 AM, enjoyed breakfast at a nearby family restaurant for the first time in a while.

	Was reading Hannah Arendt's book.

	Realized I had forgotten my house key, but my wife returned home and saved the day.

	In the afternoon, traced the proof of the Yoneda Lemma and summarized it in my own way.

	Even though the proof was written to be understandable to myself, it still felt complicated; perhaps the notes aren't very good,
	or perhaps still not entirely used to category theory.

	Lunch was a bento box from the deli.
	Enjoyed it relaxing with my wife.

	Organized tasks, and with my wife's help on administrative work, submitted company documents.

	Also attended an MBA lecture.

	As I've been spending a lot of time in front of the computer lately, planning to go to the sauna tomorrow for a break.
\end{frame}

\section{Tuesday, October 7, 2025}

\begin{frame}{Tuesday, October 7, 2025}
	Went to bed at 11 PM and woke up at 6 AM, seven hours of sleep.

	A cockroach appeared at home, so spent the entire day at a private sauna to avoid being home.

	Rubik's Cubes and group theory came up in a conversation at the office, reminding me it's been a while since studying group theory.

	Want to ensure I can follow the proofs of group representation theory and some other fundamental theorems.

	Purchased a book on time-series analysis.

	Wondering if AR-IMA processes are used by P-measure people.

	Plan to start by studying the AR model.

	Watched the live broadcast of the Nobel Prize in Physics again this year.

	It was about the fundamentals of quantum computing using Josephson junctions. Happy about that.

	Bought books on string theory, gauge theory, and the renormalization group.
\end{frame}

\section{Wednesday, October 8, 2025}

\begin{frame}{Wednesday, October 8, 2025}
	\scriptsize
	Went to bed just past 9:30 PM last night and woke up around 8:30 AM.

	Slept for 11 hours. Feeling great after a good rest.

	Nakahara's 'Geometry and Topology' arrived, so spent the morning leisurely reading it.

	Before noon, had a meeting to learn about the CA business process.

	Got the outlook that if optimal metrics can be provided to each CA, they'd all be able to work with confidence.

	Lunch consisted of fried chicken and other items from the deli.

	Bought some office supplies and consumables, along with diapers for the child.

	Enjoyed some relaxed window shopping at a bookstore.

	Felt refreshed after a pleasant walk and returned home in a good mood, only to realize I'd forgotten to buy the diapers I went out for.

	Read the time-series analysis textbook from the afternoon.

	Gained a rough understanding of the AR model's mechanism.

	When the AR model's order $p$ is too large, the MA model seems to be effective, so that's what I want to learn about next.

	Also bought a textbook on CP symmetry breaking.

	Apparently, a Japanese person was also selected for the Nobel Prize in Chemistry in 2025. Congratulations!

	The achievement involved porous metal complexes, which is also great news as it's in the field of condensed matter physics. Thrilled.

	It was another good day.
\end{frame}

\section{Thursday, October 9, 2025}

\begin{frame}{Thursday, October 9, 2025}
	\scriptsize
	Lately, the habit of waking up and immediately programming while rubbing my eyes is taking hold.

	Spent the entire morning coding.

	Had lunch with my wife and a friend at a Uyghur restaurant in Ueno.

	Enjoyed Laghman and a drink called Kvass.

	Either ate too much or was affected by the spices; my stomach was upset, and couldn't touch anything until the evening.

	Took the Nobel Prize announcement as a chance to review quantum information theory.

	A textbook on complex manifolds arrived.

	Recognize many terms in complex geometry but completely fail to grasp the concepts and can't progress through the reading.

	Feeling a lack of fundamental knowledge in algebra.

	Might need to grasp the big picture of homological algebra using Nakahara's 'Geometry and Topology'.

	Studying Galois theory might be helpful if it provides the necessary algebraic concepts.

	Algebra feels like the bottleneck; better understanding of algebra would deepen the understanding of category theory.
\end{frame}

\section{Friday, October 10, 2025}

\begin{frame}{Friday, October 10, 2025}
	\scriptsize
	In the morning, dug up two sets of notes on superconductivity, an article on financial engineering, and a memo about how an American option on a non-dividend-paying asset is always cheaper than a European option, then pushed them to git.

	For lunch, went to the Indian curry place near the station with my wife.

	The spicy spinach curry and cheese naan were delicious.

	Ended up having a naan refill and overate.

	Many textbooks on string theory, gauge theory, and complex analysis arrived.

	Tried to go for a haircut, but the barbershop was closed again.

	At the bookstore visited on the way to the barbershop, I saw that this month's issue of the magazine 'Sūgaku Seminar' had a feature on category theory, so bought that too.

	In the evening, organized the notes on Itô's Lemma and the Black-Scholes model and uploaded them to GitHub.

	My daughter apparently had a sports day at nursery school today; when I picked her up, she was wearing a gold medal origami around her neck.

	Skipped dinner because my stomach was full since noon.

	Spent the evening looking through the category theory feature in the magazine.

	Created a GitHub Page just before going to bed.
\end{frame}

\section{Saturday, October 11, 2025}

\begin{frame}{Saturday, October 11, 2025}
	Woke up to the commotion of my daughter spilling milk in the futon room.

	Started packing for Gunma in the morning, finished the rough preparations, and lay down for a moment, only to sleep until just before noon.

	Finally left the house at 1:30 PM.
	Traveling in the rain with a suitcase and my child in a stroller was quite a struggle.

	The child spilled milk on the train.
	A foreigner helped to clean it up.
	It would have been a good joke to spontaneously say 'It's no use crying over split milk,' but that didn't come to mind, and simply expressed my gratitude.

	It was cold upon arriving at Takasaki.
	Felt as if the season had changed while on the train.

	My mother-in-law picked us up by car and served us a meal.
	It was chestnut rice, clear soup with Chinese cabbage, and persimmons.

	Glad to have the opportunity to see my daughter interact with my mother-in-law.

	My daughter was excited running around the spacious house.

	The bedding felt a bit dusty, causing a cough.
\end{frame}

\section{Sunday, October 12, 2025}

\begin{frame}{Sunday, October 12, 2025}
	Woke up around 8 AM.
	My wife and daughter had apparently been awake for a long time.
	My sleep was shallow, feeling like hay fever.
	My mouth was dry, and my nose was running non-stop.
	This reminded me of the unique difficulties of the countryside.

	Breakfast was yesterday's simmered dish with Udon noodles added, and homemade egg sandwiches.

	Assisted my mother-in-law with her stock and investment trust management.

	Had a light meal before noon: an onigiri with chestnut rice, a banana, and dried pomelo.

	Fell asleep myself while trying to put my daughter down for a nap, and when I woke up, it was already night.

	A yakiniku dinner was prepared for us.

	It turned out to be a day spent entirely on eating and sleeping.
\end{frame}

\section{Monday (Holiday), October 13, 2025}

\begin{frame}{Monday (Holiday), October 13, 2025}
	Woke up early, around 5 AM, having slept for nearly 14 hours yesterday.
	Was reading a corporate finance theory textbook and fell back asleep again.

	Woke up at 7 AM and picked persimmons from the tree in the garden with my wife.
	There were many persimmons on the tree, and we picked a lot.
	Gathered about 30 to 40 pieces.

	Decided to send them to my family home in Osaka since we couldn't eat them all.

	Caught the train home around 9:40 AM.

	Got on the Shinkansen around 10:40 AM.
	Couldn't find a seat in the non-reserved section.
	Stood for the journey.

	Was trying to understand the definition of the Galois group.

	A person who appeared to be a foreigner showed me their ticket on the Shinkansen and asked in Japanese for help finding their seat.
	Since the ticket was for a non-reserved seat, explained in English that they could sit in any empty seat in cars 1 through 5 on this train.
	They replied with very good Japanese, 'I see, I understand. Thank you.'
	According to my wife, who saw the interaction, it looked like a comedy sketch: me explaining in English to a Japanese-speaking person.
\end{frame}

\section{Tuesday, October 14, 2025}

\begin{frame}{Tuesday, October 14, 2025}
	The day was spent contemplating field theory and topology.

	Had two consecutive meetings around noon.
	In the marketing team meeting, discussed the AR model.
	The idea came up that the causal effect of a new measure could be explained by treating the measure as an external field perturbation.

	Also had a meeting about the processes for acquiring data from a certain SaaS system.

	Development progressed, and bulk acquisition of two years' worth of registrant data is now possible.

	The analysis batch for new grads/recent grads was set up for scheduled execution.

	Went to the hospital in the evening.

	Felt somewhat depressed, perhaps due to being overly exhausted yesterday.
\end{frame}

\section{Wednesday, October 15, 2025}

\begin{frame}{Wednesday, October 15, 2025}
	\scriptsize
	Woke up around 7 AM, took a shower, took out the trash, and went to Maruzen in Marunouchi.

	Browsed the textbooks 'Modules and Homological Algebra' and 'Fibre Bundles and Homotopy', thinking they looked excellent as category-theory-based math books, even though I'm not familiar with modules or homotopy yet.

	Had a company interview just as time was running out.
	Informed them of my intention to extend for three months.
	The program might resume in January 2026.

	Ate a satisfying bowl of Tsukemen at Tomita.

	Returned to Maruzen and bought the three books: 'Modules and Homological Algebra', 'Fibre Bundles and Homotopy', and 'Linear Algebra Dialogues Vol. 2'.
	That totaled 12,000 yen.

	Noticed that several good textbooks on quantum field theory in condensed matter physics were also available.

	Continued with development after returning home.

	Also attended MBA lectures: Marketing and Operations Management.

	Wrote notes on the Radon-Nikodym derivative and the Cameron-Martin-Girsanov theorem.

	Went to bed around 9 PM.
\end{frame}

\section{Thursday, October 16, 2025}

\begin{frame}{Thursday, October 16, 2025}
	Spent the morning creating notes on the pricing of derivatives on foreign exchange.

	Shaved for the first time in a while, took a shower, and left the house around 9:30 AM.

	Arrived at Takadanobaba around 10:30 AM and stayed at a bookstore.

	Arrived at the Russian restaurant exactly at 11:30 AM, and thanks to my friend waiting in the pre-opening line, we were able to get in.

	Ate Beef Stroganoff. It was the most delicious thing I've had in a while.

	Had cake at a cafe before parting ways.

	Dedicated the time after returning home to development work.

	Had a meeting with the marketing representative, who requested a system development that would further complicate an already highly complex structure, so proposed an alternative solution.

	Reading the Galois theory book, but don't feel like I'm understanding it at all.
	While the reading itself progresses, perhaps the lack of practice explaining it in my own words is why the sense of comprehension is missing.

	Also need to make progress in studying groups and modules.
	The lack of knowledge about modules is hindering progress with the homological algebra reading.
\end{frame}

\section{Friday, October 17, 2025}

\begin{frame}{Friday, October 17, 2025}
	A rare day spent relaxing at home all day.

	Had a casual conversation with a senior colleague in the morning.

	Asked if Lovelock's theory of gravity would be a good next step after learning the basics of general relativity, and learned that it's apparently quite a leap.

	Was recommended the black hole textbook 'A Relativist Toolkit'.

	The daughter's medical certificate was missing, so had to go to the Public Health Center to get it reissued.

	While out, my wife and I had Uzbekistani food.

	Dedicated the time after returning home to development work.

	Had a chat with a different developer from the marketing team than yesterday.
	This was also a request for a system that would further complicate an already highly complex structure.

	After much deliberation, spoke with the senior colleague again to help sort out the task priorities.
\end{frame}

\section{Saturday, October 18, 2025}

\begin{frame}{Saturday, October 18, 2025}
	My daughter seemed unwell this morning and was having trouble breathing.

	Her temperature was normal in the morning.

	Took her to a nearby hospital and got cold medicine.

	A temperature check at the hospital showed she was nearly 39 degrees Celsius.

	Had kebab for lunch with the family.

	It took about two hours to get the prescription filled at the pharmacy.

	Read a book on renormalization theory during the wait.

	The correspondence between gauge theory and $\phi^{4}$ theory was interesting.

	Since I haven't studied the renormalization group equations, I'd like to summarize them in my notes at some point.

	Perhaps thanks to the medicine, my daughter's condition was much better by nightfall.
\end{frame}

\section{Sunday, October 19, 2025}

\begin{frame}{Sunday, October 19, 2025}
	Had the usual Sunday morning McDonald's breakfast.

	Studied adjoint functors, which are unavoidable in category theory, and summarized them in my notes.

	Awodey's textbook is quite clear and helpful.

	Having summarized the unit of the adjunction, the next step is to document the counit and the zig-zag identities.

	My daughter has made a complete recovery and is incredibly energetic, running around our small home with too much stamina.
\end{frame}

\section{Monday, October 20, 2025}

\begin{frame}{Monday, October 20, 2025}
	\scriptsize
	Read a statistics book before sleeping.

	A thought crossed my mind: perhaps statistics could serve to connect category theory, physics, financial engineering, and other abstract fields to the real world.

	Specifically, while mathematics like category theory is rarely used directly in society, when it is linked with statistics, which is deeply connected to societal phenomena (data), statistics could act as the glue.

	The above fragmented thought process can be organized as follows:

	Consideration: The Mediating Role of Statistics

	$\cdot$ Insight: Statistics might act as a bridge (glue) to apply abstract mathematics (e.g., category theory) and theoretical disciplines (e.g., physics, financial engineering) to the 'real world'.

	$\cdot$ Background Recognition:

	\begin{itemize}
		\item Advanced mathematics like category theory is rarely utilized directly in society.
		\item Statistics, conversely, is closely related to various social phenomena (data).
	\end{itemize}

	$\cdot$ Inference:

	\begin{itemize}
		\item By functioning as a 'mediating term' between abstract theory and real society, statistics could enable these theories to be applied to practical problem-solving.
	\end{itemize}

	Conclusion:
	Hypothesized that statistics can act as 'glue' connecting abstract mathematics and theory with the real world.

	12:56 PM
	Arrived at Butayama, a ramen shop in Kita-Senju.

	Luckily, got in without a queue at Butayama.

	Realized I should have scanned the loyalty stamp reader before buying the food ticket.

	In the evening, wrote practice notes using the fRG Wetterich Equation and pushed them to git.

	Always had an interest in the renormalization group, but a stronger desire has emerged to study it systematically.
\end{frame}

\section{Tuesday, October 21, 2025}

\begin{frame}{Tuesday, October 21, 2025}
	The apartment complex had a water shut-off scheduled for the morning today.

	Woke up, took a shower, and did laundry first thing to finish all water-using chores.

	Advanced development and reported the achievements in the meeting minutes.

	A network connection trouble necessitated manually running all batch jobs for the day.
	Need to migrate to AWS EC2 soon.

	Transferred money between bank accounts.

	Then came to the family restaurant right near the house, had a Peperoncino and chicken steak, while reading Clark's 'Childhood's End'.
	It's incredibly fascinating.
	Couldn't help but think that, at heart, all I do is transfer values from one account item to another.
\end{frame}

\section{Wednesday, October 22, 2025}

\begin{frame}{Wednesday, October 22, 2025}
	A day spent working at the office in Nishi-Shinjuku.

	Heard that a friend is feeling mentally exhausted.

	The Looker Studio report that had been under development for some time is finally complete.

	In the evening, talked about data science at the request of a friend.

	Gained new insights and, overall, it was a good day.
\end{frame}

\section{Thursday, October 23, 2025}

\begin{frame}{Thursday, October 23, 2025}
	Had Tan Tan Noodles without soup for breakfast.

	Attended the MBA lecture on research ethics.

	Was slightly panicked, reminded that even for simple notes, the sources must be clearly cited.

	The Looker Studio report is being used effectively.

	Set a meeting date to give a lecture to company colleagues.

	In the afternoon, went with a friend to Shinjuku for Romanian food.
	The price was reasonable considering the restaurant's appearance.

	Chatted at a cafe before heading home.

	Spent the evening relaxing at home.
\end{frame}

\section{Friday, October 24, 2025}

\begin{frame}{Friday, October 24, 2025}
	Went for Tsukemen with my wife before noon.

	'Submanifolds in Theoretical Physics' arrived.

	Dug up and pushed to git the notes on the Superconductor/Metal interface using the Usadel equation, which I wrote during my student days.

	Sleeping 12 hours total: 10 hours at night and a 2-hour nap during the day.

	Worried that my daughter's cough won't stop.

	The book on high-speed trading is starting to feel interesting.
	This might be a sign that I'm recovering physically.
\end{frame}

\section{Saturday, October 25, 2025}

\begin{frame}{Saturday, October 25, 2025}
	Started reading about algorithmic trading at 4 AM.

	My daughter's cough continued, so went to the hospital in the morning.

	Had kebab for lunch with the family.
	Since there's a kebab shop near the hospital, we often end up having kebab after a visit.

	My wife went to the hospital in the afternoon.

	Took a nap.

	Summarized algorithmic trading slightly on a mind map, and finished reading for the day.

	'Statistics is the Strongest Subject (Practical Edition)' was a very good book.
	Want to summarize this one too.

	Purchased 'Deep Learning' by Ian Goodfellow, Nawa's 'Schumpeter', Takamura's 'Introduction to Functional Analysis', and Matsuzaka's 'Introduction to Algebraic Systems'.

	Also bought 'Quantum Many-Body Physics and Artificial Neural Networks'.
\end{frame}

\section{Sunday, October 26, 2025}

\begin{frame}{Sunday, October 26, 2025}
	The whole family woke up around 5 AM.

	Experimented with creating slides using LaTeX.
	It was my first time making LaTeX slides.
	It's fun to try different things.

	Had the usual weekly Sunday McDonald's breakfast.

	Tried using a bitter-tasting nail polish for toddlers to stop my daughter's thumb-sucking habit.

	Despite the bitter taste when she licked her finger, she kept sucking, and the resulting non-stop gagging caused trouble for everyone all day.

	My wife is interested in statistics, so created a simple presentation slide on the topic.
\end{frame}

\section{Monday, October 27, 2025}

\begin{frame}{Monday, October 27, 2025}

	The world of High-Frequency Trading (HFT) is fascinating.

	Slept until around 9 AM.
	About 11 hours of sleep.

	Investigated a report where the numbers didn't match at 9:30 AM, and from 10:30 AM...

	Bought the textbook 'Seiberg-Witten Equations'.
	Hoping to be able to read it someday.

	Had rich, concentrated Niboshi ramen for the first time in a while.

	Submitted the MBA assignment just under the wire.
\end{frame}

\section{Tuesday, October 28, 2025}
\begin{frame}{Tuesday, October 28, 2025}
	\scriptsize
	Slept until around 9 AM again today.
	About 12 hours of sleep.

	Spent the morning briefly studying number theory and function theory.
	Want to summarize Gauss's fundamental theorem.

	Read a bit further in the category theory textbook about adjoint functors.
	Was impressed by the concept of Kan extension (Awodey section 9.17) which suggests that, in a sense, every functor has an adjoint.

	Lunch was steamed chicken and instant noodles.

	Had a meeting with the marketing team in the afternoon.

	Also had a meeting with my superior.

	Attended an MBA lecture: marketing differentiation theory.
	The discussion centered on how differentiation can create inelastic demand.
	Something like this, I believe, could be neatly stated and proven (in a Bourbaki-like style) using mathematical economics.

	Also attended an MBA lecture on Operations Management.
	The class covered the historical background of operations and IT, bringing up topics like the von Neumann architecture.

	Spent the evening entirely on development, and a deliverable was completed quickly.
\end{frame}

\section{Wednesday, October 29, 2025}
\begin{frame}{Wednesday, October 29, 2025}
	\scriptsize
	Spent all of last night developing.

	Went to bed late, but woke up around 7 AM, resulting in only about 6 hours of sleep.

	A bookshelf arrived.

	It was good timing, as books had piled up in mountains around my feet, leaving no walking space.

	'Probability Theory and Function Theory' arrived.
	It seems promising, going beyond the basics of stochastic differential equations.

	The morning was spent organizing tasks and experimenting with Google Sites after being shown how to use it.

	Lunch was chicken pasta and pizza at a nearby family restaurant.

	Brought a statistics book to the family restaurant to read.

	Bought chicken at the supermarket and made more steamed chicken to eat.
	For some reason, eating chicken seems to make me feel incredibly better.

	Had a meeting from 3 PM to learn about the integration of Google Sites and Gemini.
	Learning how to use generative AI feels very cost-effective.

	Continued development for the analysis, creating a differential survey, but the report turned out uninspiring.
	The data is good, but struggling to find a compelling way to present it, so it might be better to postpone the report.

	Recently, I've been working too much and haven't had enough time for independent study, which is a regret.

	Want to study more math and physics while I have the chance.
\end{frame}

\section{Thursday, October 30, 2025}
\begin{frame}{Thursday, October 30, 2025}
	\tiny

	\textbf{Fact}
	\begin{itemize}
		\item Learned how to use generative AI and felt it could be applied to knowledge organization and information dissemination.
		\item Went out for African cuisine with a friend, and discussed a Georgian restaurant as the next option.
		\item Felt physically exhausted after commuting for about two hours round trip on a crowded train.
		\item Exchanged the topic '80\% of worries never come true' with a friend.
		\item Read a book about HSP (Highly Sensitive Person) and felt validated about my high sensitivity.
		\item Read a math book and deepened my understanding of the concept of adjunction in category theory.
	\end{itemize}

	\textbf{Event (Occurrence / Impression)}
	\begin{itemize}
		\item While learning generative AI applications, felt a new potential for idea visualization and text generation.
		\item Felt the energy of a different culture through African cuisine, and received a pleasant stimulation.
		\item Recognized through the stress of the crowded train that my sensory acuity contributes to daily fatigue.
		\item The topic '80\% of worries never come true' was striking, serving as a trigger to notice my daily thought patterns.
	\end{itemize}

	\textbf{Reflection (Introspection / Analysis)}
	\begin{itemize}
		\item A day where I felt physically tired yet intellectually fulfilled.
		\item Accepting my sensitivity (HSP-like tendency) made me feel I could respect my own pace instead of trying to 'get used to' things.
		\item Realized that AI can function not only as an 'efficiency tool' but also as an 'extension of thought' and a 'mirror for self-understanding'.
		\item The idea that '80\% of worries never come true' felt like a cognitive reset that could alleviate excessive risk avoidance.
	\end{itemize}

	\textbf{Insight (Discovery)}
	\begin{itemize}
		\item Learning deepens the most on days when external fatigue and internal stimulation coexist.
		\item Treating my sensitivity not as a weakness but as a source of insight and creativity can lead to richer expression.
		\item Externalizing thought through AI can help reveal my structural thinking patterns.
		\item The majority of anxiety is merely a 'projection of thought' and can be mitigated through observation.
	\end{itemize}

	\textbf{Next Step}
	\begin{itemize}
		\item Continue reading 'Childhood's End'.
		\item Summarize the 'Adjunction' in category theory in my notes.
		\item Try to improve the look and feel of my homepage using generative AI.
		\item Continue reading the generative AI application book and hone practical utilization skills.
		\item Write a note on using the renormalization group in Cooper's theory of superconductivity.
	\end{itemize}
\end{frame}

\section{Friday, October 31, 2025}  
\begin{frame}{Friday, October 31, 2025}  
	\tiny  

	\textbf{Fact}  
	\begin{itemize}
		\item Slept in until late morning and spent the day in calm tranquility.  
		\item Went out with my wife to a nearby ramen shop famous for its rich dried-fish broth, and enjoyed a lively conversation with the owner.  
		\item While talking with my wife about investment strategies, the discussion became too technical and ended up upsetting her.  
		\item Work on the development project progressed smoothly, and the data cleansing phase finally began to take shape.  
		\item Received several new books: one on nonequilibrium statistical mechanics, another on machine learning through Bayesian statistics, and a textbook on neutrino physics.  
		\item My child played happily with a handmade Halloween toy from daycare and has now learned to catch a ball during our playtime together.  
	\end{itemize}

	\textbf{Event}  
	\begin{itemize}
		\item A slight awkwardness lingered after the conversation with my wife, leaving me with a sense of regret and reflection.  
		\item Felt genuine surprise and joy the moment my child caught the ball.  
		\item While facing the growing pile of unread books, was struck by both intellectual curiosity and the awareness of limited time.  
	\end{itemize}

	\textbf{Reflection}  
	\begin{itemize}
		\item In the conversation with my wife, wanted to convey the idea that money ultimately belongs to the family as a whole, but failed to express it clearly.  
		\item Realized once again the importance of being mindful of how others receive and interpret one's values.  
		\item Observing my child's cognitive development led to contemplation about how much parental involvement is appropriate.  
	\end{itemize}

	\textbf{Insight}  
	\begin{itemize}
		\item In family dialogue, creating a safe and open atmosphere matters more than being right.  
		\item A child's growth often exceeds a parent's expectations, and the act of simply watching over them requires courage.  
		\item Even within the calmness of everyday life, subtle emotions and quiet signs of growth are always present.  
	\end{itemize}

	\textbf{Next Step}  
	\begin{itemize}
		\item In conversations with my wife, aim to respect the flow of her feelings and interests while communicating.  
		\item Continue reading the newly arrived books at a steady pace without haste.  
		\item Keep observing my child's development and explore natural ways to stay involved.  
	\end{itemize}
\end{frame}


\end{document}
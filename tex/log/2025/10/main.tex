\documentclass[uplatex]{jsarticle}
\usepackage[english]{babel}
\usepackage[letterpaper,top=2cm,bottom=2cm,left=3cm,right=3cm,marginparwidth=1.75cm]{geometry}
\usepackage{amsmath, amssymb}
\usepackage{graphicx}
\usepackage{here}

\title{
Log.
}

\author{
M. O.
}

\date{Oct. 2025}

\begin{document}
\maketitle

\section{October 1, 2025 (Wed)}

The morning started at a family restaurant, eating breakfast while reading a category theory textbook and Kojin Karatani's 'Powers and Modes of Exchange'.

It got to be past noon, so I just stayed and ordered lunch.

Living quite elegantly.

Gave my wife her allowance for the month. Despite the morning's elegance, our household finances are tight.

For my side job, I built the analysis tool for new and recent graduates to run on AWS Lambda, but it's in a terrible state and desperately needs refactoring.

Was deeply impressed reading the beginning of Mac Lane's category theory textbook. It explained how commutative diagrams can be used to understand groups, topological groups, and Lie groups in a unified way.

I feel fortunate to be moved by the beauty of physics and the beauty of mathematics alike.

\section{October 2, 2025 (Thu)}

Had plans to meet a friend for Kenyan food for lunch, but the restaurant was closed.

We ended up going to our usual Peruvian place in Gotanda instead.
It was delicious as always, so it turned out great in the end.

Talked with my friend over coffee at a cafe after the meal.

We discussed how much being in a serious athletic club during school helps one adapt to society.

Also heard about his work at a prep school, how students who struggle with math often can't even copy formulas correctly, let alone understand the textbook.

My friend successfully got the job at the medical school prep academy. Not only that, but he's apparently already teaching three students.

We parted ways after the cafe, and from there I headed to the office in Shinjuku.

It was good to be able to talk with my friend and my team members.

Stopped by a used bookstore in Shinjuku on the way home, not getting back until after 8 PM.

\section{October 3, 2025 (Fri)}

Spent the entire day focused on development for my side job.

The analysis batch for 'Yomi Sheet' is mostly complete.

Spoke with the two leaders about CA activity analysis, and also with the person developing a separate system for CA performance analysis.

It seems the count of job offer acceptances is off by an order of magnitude.

I suspect it's due to the order of operations: whether the 30- or 90-day cutoff aggregation is performed first, or if the flag for the most advanced process is set first.

Went out to Kita-Senju for dinner with a friend in the evening.

Kita-Senju has many places that might serve biryani, but they often have irregular holidays, making it hard to actually get any.

Today, too, we ended up at Butayama, a Ramen Jiro-style place.

Was able to give my friend the Hannah Arendt book and the causal inference book.

On the way back, we stopped at our usual standing beer bar and had a long talk.

Talked about wanting to re-enter a doctoral program, and how I've even narrowed it down to a specific research lab.

My friend, meanwhile, is apparently getting interested in corporate finance theory.

After getting home, uploaded my notes on the Keldysh Green Function.

\section{October 4, 2025 (Sat)}

A day spent almost entirely sleeping.

Richard Rorty's 'Contingency, Irony, and Solidarity' arrived. Want to read it.

Right as the date was changing--that is, right as my age was changing--I was studying the part where the comma category, which is equivalent to the Grothendieck construction, is represented as the category of elements using the integral sign for categories.
Surprised to find that category theory also has integration, and even a Fubini's theorem.

\section{October 5, 2025 (Sun)}

Woke up feeling a bit sleep-deprived.

Made some slight revisions to my Fermi liquid notes.

Studying limits in category theory.

It's my birthday, so we had planned to go out for sushi, but my daughter was rampaging around the house, and my wife was exhausted. We ended up just buying some prepared dishes from a bento shop.

Ammon and Erdmenger's book on gauge/gravity duality happens to be available used, but it's not very cheap.
Had been debating whether to buy it for a while, but was thinking of getting it for my birthday.

Managed to obtain it through a stroke of luck.

Along with it, also got my hands on Altland and Simons' textbook, Nakahara's topology textbook, and Xiao-Gang Wen's textbook. Lucky.

At a used bookstore, also picked up Audrey Tang's 'Plurality' and Milton Friedman's 'Capitalism and Freedom'.

Tried creating this 'Misc.' notebook.

My wife and daughter went to bed early, leaving me with a long evening to myself.

Savoring the luxury of time to relax and read slowly.

Emily Riehl's category theory textbook is starting to seem quite good.
Feeling like I want to summarize the section on the Yoneda lemma in my notes.

Aside from a short walk outside for shopping, there were no major events. It was extremely peaceful, and became the best birthday I've had in my entire life.

\section{October 6, 2025 (Mon)}

Woke up at 5:55 AM. Got eight and a half hours of sleep.

Until about 8:00, studied and summarized the fundamentals of limits in category theory.

From around 9:00, enjoyed breakfast at the nearby family restaurant for the first time in a while.

Was reading the Hannah Arendt book.

Realized I had forgotten my house key, but my wife happened to come home, which was a relief.

In the afternoon, traced the proof of the Yoneda lemma and tried to summarize it in my own way.

I tried to write the proof in a way that I myself could understand, but it still feels complicated. Maybe the notes aren't very good, or perhaps I'm just not used to category theory yet.

Lunch was a bento from the bento shop. Ate it relaxing with my wife.

Organized my tasks, and my wife helped me with the administrative work to submit some company documents.

Also attended an MBA lecture.

Been sitting in front of the computer a lot lately. I'd like to go for a relaxing sauna tomorrow.

\section{October 7, 2025 (Tue)}

Bed at 11 PM, up at 6 AM. Seven hours of sleep.

A cockroach appeared in the house. Not wanting to stay home, I spent the day at a private sauna.

Rubik's Cubes and group theory came up in conversation at work, which made me realize I haven't studied group theory in a long time.

I'd like to get to the point where I can follow proofs for group representation theory and some other basic theorems.

Bought a book on time series analysis.

I wonder if people using P-measures actually use things like ARIMA processes.

Want to start by studying AR models.

Watched the Nobel Prize in Physics announcement live again this year.

It was for the fundamentals of quantum computation using Josephson junctions. Happy about that.

Bought books on superstring theory, gauge theory, and the renormalization group.

\section{October 8, 2025 (Wed)}

Went to bed just after 9:30 PM last night and woke up around 8:30 AM.

Eleven hours of sleep. Slept well and am feeling good.

Nakahara's 'Geometry, Topology and Physics' arrived, so I spent the morning reading it leisurely.

Before noon, had a meeting where I was taught about the CA business process.

Gained the perspective that if we can provide optimal metrics for each CA, everyone should be able to work with more peace of mind.

Lunch was karaage and such from the prepared food shop.

While out buying diapers for the kid, also picked up some office supplies and consumables.

Enjoyed some leisurely window shopping at the bookstore.

Had a pleasant walk and returned home in a good mood, only to realize I had forgotten the one thing I went out to buy: the diapers.

From the afternoon, I was reading the time series analysis textbook.

Got a rough grasp of the mechanism of AR models.

It seems MA models are effective when the order $p$ of the AR model is too large; I want to learn about that mechanism next.

Also bought a textbook on CP symmetry violation.

Apparently, a Japanese person also won the Nobel Prize in Chemistry for 2025. Congratulations.

The award was for work on porous metal-organic frameworks, so it's also a condensed matter field. That's great. Happy to hear it.

Today ended up being another good day.

\section{October 9, 2025 (Thu)}

Lately, it's becoming a habit to wake up and immediately start programming, rubbing my eyes.

Spent the entire morning coding.

Had lunch with my wife and a friend at a Uyghur restaurant in Ueno.

Enjoyed laghman and an alcoholic drink called kvass.

Maybe I ate too much, or maybe the spices did me in. My stomach felt awful, and I couldn't get anything done until evening.

The Nobel Prize prompted me to review some quantum information theory.

The textbook on complex manifolds arrived.

Regarding complex geometry, while I've seen many of the terms before, I don't understand the concepts at all and can't make progress reading.

I feel I'm lacking fundamental knowledge of algebra.

It seems necessary to grasp the general framework of homological algebra using Nakahara's 'Geometry, Topology and Physics'.

Studying Galois theory might also be good, if I can gain algebraic concepts from it.

Understanding algebra would deepen my understanding of category theory. I feel like algebra is the bottleneck.

\section{October 10, 2025 (Fri)}

In the morning, dug up and git pushed two sets of notes on superconductivity, an article on financial engineering, and a memo regarding the idea that American options on a non-dividend-paying underlying asset are always less valuable than European options.

Went for lunch with my wife at the Indian curry place in front of the station.

The spicy spinach curry and cheese naan were delicious.

Got a refill on the naan and ended up eating too much.

A pile of textbooks arrived: superstring theory, gauge theory, and complex analysis.

Tried to go for a haircut, but the barbershop was closed again.

On the way to the barber, I stopped by a bookstore and bought this month's issue of 'Suri Kagaku Seminar', which was released today. It's a special feature on category theory.

In the evening, organized my notes so far on Ito's lemma and Black-Scholes, then uploaded them to GitHub.

Apparently, my daughter had a sports day at nursery school. When I went to pick her up, she was wearing an origami gold medal around her neck.

I was still full from lunch, so I skipped dinner.

Spent the evening browsing the category theory feature in the magazine.

Tried making a GitHub Page right before bed.

\section{October 11, 2025 (Sat)}

Woke up to a commotion: my daughter had spilled milk in the futon room.

Started packing for Gunma in the morning. After finishing the rough packing, I intended to lie down for just a bit but ended up sleeping until just before noon.

We finally left the house at 1:30 PM.

Traveling in the rain with a suitcase and a child in a stroller was pretty tough.

On the train, the kid spilled milk.

A foreigner helped out.
It would have been a good joke if I could have immediately said 'It's no use crying over split milk', but the phrase didn't come to me, and I just ended up expressing my thanks normally.

It was cold when we arrived in Takasaki.
So cold it felt as if the season had changed while we were on the train.

My mother-in-law picked us up by car, and we had a meal at her place.
Chestnut rice, clear soup with napa cabbage, and persimmons.

It was good that my mother-in-law got to see my daughter.

My daughter was overjoyed, running around the spacious house.

The bedding was a bit dusty, making me cough.

\section{October 12, 2025 (Sun)}

Woke up around 8:00 AM.
My wife and daughter were apparently up long before me.
I felt like I was having allergies, so my sleep was shallow.
My mouth was parched and my nose wouldn't stop running.
Was reminded that the countryside has its own unique difficulties.

For breakfast, we had udon added to yesterday's stew, along with homemade egg sandwiches.

Helped my mother-in-law with managing her stocks and investment trusts.

Had a light meal before noon: chestnut rice balls, a banana, and dried pomelo.

I lay down with my daughter to get her to nap, but I ended up falling asleep myself. When I woke up, it was already evening.

She made us yakiniku for dinner.

Today ended up being a day of just eating and sleeping.

\section{October 13, 2025 (Mon, Holiday)}

Having slept nearly 14 hours yesterday, I woke up early. Got up around 5:00 AM.
Started reading a corporate finance theory textbook and fell asleep again.

Woke up at 7:00 and went into the garden with my wife to pick persimmons from the tree.
There were so many persimmons on the tree; we picked a lot. About 30 or 40.
More than we could eat, so we decided to send them to my parents in Osaka.

Boarded the train home around 9:40.
Got on the Shinkansen around 10:40. Couldn't get a seat in the non-reserved car, so we had to stand.

Was trying to understand the definition of a Galois group.

On the Shinkansen, someone who appeared to be a foreigner showed me their ticket and asked me in Japanese where their seat was, as they couldn't tell.
The ticket was for non-reserved, so I explained in English that they could sit in any empty seat in cars 1 through 5.
They replied in perfect Japanese, 'Naruhodo, wakarimashita. Arigatou.' (I see, I understand. Thank you.)
According to my wife, who was watching, it looked like a comedy sketch, with me explaining in English to someone who was speaking Japanese.

\section{October 14, 2025 (Tue)}

A day spent pondering field theory and topology.

Had two back-to-back meetings in the afternoon.
In the marketing team meeting, we discussed AR models.
The idea came up that we could explain the causal effect of new measures by treating them as an external field perturbation.

Also had a meeting about the processes for acquiring data from PORTERS.

After that, made progress on development, and it's now possible to bulk-fetch two years' worth of registrant data.
Also set up the analysis batch for new and recent graduate registrants to run on a regular schedule.

Went to the hospital in the evening.

Perhaps from being so exhausted yesterday, I've become somewhat depressed.

\section{October 15, 2025 (Wed)}

Woke up around 7:00, showered, took out the trash, and went to Maruzen in Marunouchi.

Was looking at 'Modules and Homological Algebra' and 'Fiber Bundles and Homotopy'. They're math textbooks based on category theory, and since I don't know modules or homotopy yet, I thought they looked excellent.

Had a meeting with my company right at the last minute.
I told them I would be extending for another three months.
Apparently, the program might restart from January 2026.

Ate my fill of tsukemen at Tomita.

Went back to Maruzen again and ended up buying all three: 'Modules and Homological Algebra', 'Fiber Bundles and Homotopy', and 'Linear Algebra Dialogue Vol. 2'. 12,000 yen.

Noticed there were several other good textbooks out on quantum field theory in condensed matter physics.

Continued with development work after returning home.

Attended MBA lectures as well. Marketing and Operations Management.

Wrote notes on the Radon-Nikodym derivative and the Cameron-Martin-Girsanov theorem.

Went to bed around 9:00 PM.

\section{October 16, 2025 (Thu)}

Started the morning by making notes on pricing derivatives for foreign exchange.

Shaved for the first time in a while, showered, and left the house around 9:30.

Arrived in Takadanobaba around 10:30 and spent time at a bookstore.

Got to the Russian restaurant right at 11:30. My friend had already been queueing up before it opened, so we were able to get in.

Had the beef stroganoff. It was the most delicious thing I've eaten in some time.

We had cake at a coffee shop and then parted ways.

After returning home, I buckled down on development work.

Had a meeting with the person in charge of marketing. The request was for system development that would make something already highly complex even more complicated, so I proposed alternative solutions.

Reading a book on Galois theory, but I just can't seem to feel like I understand it.
The reading itself is progressing, but I haven't been practicing explaining it in my own words, which might be why it doesn't feel like it's sinking in.

I want to make progress studying groups and modules, too.
My lack of knowledge about modules is holding up my reading on homological algebra.

\section{October 17, 2025 (Fri)}

A day spent relaxing at home all day for the first time in a while.

Had a light chat with a senior colleague in the morning.

I asked if Lovelock's theory of gravity would be a good next step after learning the basics of general relativity. It turns out there's apparently quite a gap to bridge before I can tackle Lovelock's theory.
He recommended the black hole text 'A Relativist's Toolkit'.

We couldn't find my daughter's medical insurance card at home. Had no choice but to go to the health center and get it reissued.
While we were out, my wife and I had Uzbek food.

After returning home, I got back to development work.

Chatted with a different developer from marketing than yesterday.
This, too, was a request for system development that would make something highly complex even more complicated.

After agonizing over it, I spoke with my senior colleague again and had him help sort out and prioritize the tasks.

\section{October 18, 2025 (Sat)}

My daughter seemed unwell from the morning; she looked like she was having trouble breathing.
Her temperature was normal in the morning.

Took her to a nearby clinic and got some cold medicine.
When they took her temperature at the clinic, it was nearly 39 degrees Celsius.

We all had kebabs for lunch.

It ended up taking about two hours to get the medicine from the pharmacy.

While waiting, I read a book on renormalization theory.
The correspondence between gauge theory and $\phi^4$ theory was interesting.
I've never studied the renormalization group equations, so I'm thinking I'd like to summarize them in my notes at some point.

The medicine must have worked; my daughter's condition was much better by the evening.

\section{October 19, 2025 (Sun)}

Went with the customary 'Asa Mac' (McDonald's breakfast).

Studied and summarized adjunctions, an unavoidable topic when learning category theory.

Awodey's textbook is very clear and quite good.

Now that I've summarized the unit of an adjunction, I want to write about the counit and the zigzag identities next.

My daughter has made a full recovery. She's incredibly energetic, with too much stamina for our small apartment, and is tearing around the place.

\end{document}
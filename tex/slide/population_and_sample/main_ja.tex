\documentclass[dvipdfmx, autodetect-engine, aspectratio=169, 10.5pt]{beamer}

\usepackage{amsmath}
\usepackage{amssymb}
\usepackage{amsthm}
\usepackage{graphicx}
\usepackage{hyperref}
\usepackage{enumitem}
\usepackage[english]{babel}

\usetheme{Boadilla}
\usetheme{Marburg}
\usecolortheme{orchid}
\usefonttheme{professionalfonts}

\title{統計的推定の基本事項}
\author{M. O.}
\date{\today}

\begin{document}

\begin{frame}[plain]
	\titlepage
\end{frame}

\begin{frame}{目次}
	\tableofcontents
\end{frame}

\section{母集団と標本}

\subsection{母集団}

\begin{frame}{母集団}
	推定統計において、分析の対象とする母集団と、母集団から抽出された標本を区別する必要がある。

	${}$

	母集団とは分析の結果をあてはめる対象となる全体の集合のことである。

	推定統計において、問題設定上、「知りたいが、知り得ないもの」となっているのが普通である。
\end{frame}

\subsection{標本}

\begin{frame}{標本}

	一方で、標本とは、本来は母集団から抽出した部分集合のことである。

	言葉の濫用によって、母集団から抽出した特性値(例えば身長や体重といった量)の部分集合を指すことが多い。

	標本として特性値の部分集合の場合は、例えば
	$\{ x_{1}, x_{2}, ..., x_{n} \} \in \mathbb{R}^{n}$
	のように表現ができるので定量的な分析・推測が可能になる。

	${}$

	問題の設定上、母集団を全て調査する「全数調査」が可能になるケースもある。

	しかし一般には母集団はサイズが大きく全数調査が困難であり、
	標本を分析することで母集団の特性を知ろうとする。
	このような調査方法を「標本調査」と呼び、
	これに基づく推定や検定が「統計的推測」である。
	(推「定」や検「定」が統計的推「測」である。)
\end{frame}

\subsection{調査と母集団の特徴付け}

\begin{frame}{調査と母集団の特徴付け}
	母集団の分布について知りたいケースもあるが、
	その母集団を特徴づける定数の値を知りたいというケースもある。

	母集団を特徴づける定数を母数という。

	母数の例としては、母平均、母分散、母標準偏差、母比率などがある。

	${}$

	それに対して標本からつくられる「関数」を統計量と呼ぶ。

	統計量の例としては、標本平均、標本分散、標本標準偏差、標本比率などがある。

	推測統計では、観測値から計算された統計量の値に基づいて母数の値を知ろうとする。
\end{frame}

\section{統計的研究の種類}

\begin{frame}{統計的な研究の種類}
	統計学の役割は、平均や分散を推測することによって母集団の特性を知るための方法を提供するだけではない。

	母集団から標本を正しく取り出すための方法を提供するのも統計学の役割である。

	よく知られた例として、「単純無作為抽出法」(後述)があるが、それ以外にも多くの実験デザインがなされている。

	${}$

	母集団の特性を知るための研究には大きく分けて2つの種類、実験研究と観察研究がある。

	実験研究は研究対象に介入を行う研究である。

	観察研究は研究対象に介入できない研究である。
\end{frame}

\subsection{観察研究と実験研究}

\begin{frame}{観察研究と実験研究}
	統計学の役割は、平均や分散を推測することにより母集団の特性を知るための方法を提供するだけではない。

	母集団から標本を正しく取り出すための方法を提供することも統計学の役割である。

	例えば新薬開発のための動物実験や臨床試験、研究室や工場における実験のデザインを行うことも統計学の役割の一つである。

	${}$

	母集団の特性を知るための研究として実験研究と観察研究がある。

	\begin{itemize}
		\item 実験研究:研究対象に介入できる
		\item 観察研究:研究対象に介入しない
	\end{itemize}

	いずれの場合でも得られる標本は母集団の一部である。

	標本が偏りなく母集団の特徴を示すように研究プロセスをデザインをする必要がある。

\end{frame}

\subsection{実験研究のデザイン}

\begin{frame}{実験研究のデザイン}

	実験と名付けられているが実験室で行う研究だけが実験研究ではない。

	条件の設定が研究者自らの手でできるものを実験という。

	新薬の効果を評価するための臨床試験や、肥料の効果を知るための農事試験なども実験に含まれる。

	${}$

	例えば新薬の効果の評価をするケースを考える。

	患者を新薬に投与する群(実験群)と新薬ではない対照薬を投与する群(対照群)に分けて、
	それらの差を測定する。

	${}$

	実験研究では次に述べるフィッシャーの3原則が重要になる。

\end{frame}

\subsection{フィッシャーの3原則}

\begin{frame}{フィッシャーの3原則}

	フィッシャーの3原則

	${}$

	\begin{itemize}
		\item ・無作為化
		\item ・繰り返し
		\item ・局所管理
	\end{itemize}

\end{frame}

\begin{frame}{フィッシャーの3原則(無作為化)}

	無作為化はランダム化とも呼ばれる。

	新薬の臨床試験で新薬(試験薬)と既存薬(対照薬)とを比較する場合に、どの患者にどちらの薬剤を投与するかをランダムに決めることが重要になる。(無作為割り付けと呼ぶ。)

	試験薬と対照薬の間の比較をするとき、実験単位(この場合は被験者)に課される実験条件を「処理」という。

	処理以外の条件はなるべく均一にする。

	予期できる偏りは対処できるが、予期できない偏りにつき均一にできないときは無作為に割り付けることが必要になる。

\end{frame}

\begin{frame}{フィッシャーの3原則(繰り返し)}

	データにはばらつきがある。

	まったく同じ条件で実験してもデータはばらつく。

	そのようなばらつきを超えて処理の効果が見られるかが知りたいことであるので、ばらつきの大きさを見積もる必要がある。

	そのために実験を繰り返す必要がある。

	人間を対象とした臨床試験の場合は、個体差があるため多くの被験者に対するデータを取る必要があり、これも繰り返しとみなされる。

	何回の繰り返しが必要か、計画の段階で見積もることは、統計分析をする者の大きな仕事の一つである。

\end{frame}

\begin{frame}{フィッシャーの3原則(局所管理)}

	前述のとおり、
	処理の効果をばらつきの大きさと比較するためには、
	処理以外のばらつきは極力小さいことが望ましい。

	そのため、実験の条件が均一ないくつかのブロックに分けて行う。

	このブロック化を局所管理と呼ぶ。

	${}$

	例えば新薬の臨床試験で、新薬を60代男性のみ投与し、対照薬を20代女性にのみ投与して比較する実験は処理の意味がない。

	年代や性別をブロックとして実験をすることが望ましい。

	${}$

	ブロックの設定は、同じブロック内ではなるべく均一に、異なるブロック間での違いは大きくすることが望ましい。


\end{frame}

\subsection{観察研究のデザイン}

\begin{frame}{観察研究のデザイン}

	処理の効果の立証には実験研究が欠かせないが、問題によっては実験ができないケースも多い。

	例えば喫煙が健康に対して与える影響を調べる研究を考えると、
	喫煙の影響を調べるために被験者に喫煙を強いることはできない。

	こういった場合に観察研究に頼らざるを得ない。

	${}$

	観察研究では実験研究とは異なり、フィッシャー3原則のうち無作為化がなされない。

	被験者が自ら処理を選択している点が実験と異なる。

	被験者が処理の選択を意識的に行うため、
	被験者の特性により処理の選択に偏りを生じる可能性がある。

	観察研究による処置効果の解釈には注意が必要になる。


\end{frame}

\section{標本調査と抽出方法}

\subsection{単純無作為抽出法}

\begin{frame}{単純無作為抽出法}
	aaa
\end{frame}

\subsection{系統抽出法}

\begin{frame}{系統抽出法}
	aaa
\end{frame}

\subsection{層化無作為抽出法}

\begin{frame}{層化無作為抽出法}
	aaa
\end{frame}

\section{aaa}

\begin{frame}{aaa}
	aaa
\end{frame}


\begin{frame}[allowframebreaks]{References}
	\begin{thebibliography}{9}
		\bibitem{Vickers}
		What is a p-value anyway $?$ - Vickers, Andrew J.
		\bibitem{toukei_saikyou}
		Statistics is the most powerful discipline (Practical Edition) - Kei Nishiuci
	\end{thebibliography}
\end{frame}

\end{document}
\documentclass[dvipdfmx, autodetect-engine, aspectratio=169, 10.5pt]{beamer}

\usepackage{amsmath}
\usepackage{amssymb}
\usepackage{amsthm}
\usepackage{graphicx}
\usepackage{hyperref}
\usepackage{enumitem}
\usepackage[english]{babel}

\usetheme{Boadilla}
\usetheme{Marburg}
\usecolortheme{orchid}
\usefonttheme{professionalfonts}

\title{Algorithmic Trading}
\author{Masaru Okada}
\date{\today}

\begin{document}

\begin{frame}[plain]
	\titlepage
\end{frame}

\begin{frame}{Table of Contents}
	\tableofcontents
\end{frame}

\section{Algorithmic Trading}

\subsection{What is Algorithmic Trading?}

\begin{frame}{What is Algorithmic Trading?}
	Trading executed automatically by computers
	\begin{itemize}
		\item ■ Also known as automated trading or system trading
	\end{itemize}

	Not all such trading is mathematically sophisticated
	\begin{itemize}
		\item ■ Mathematically sophisticated trading, using standard item marker
		      \begin{itemize}
			      \item ・ Statistical methods
			      \item ・ Time series analysis
			      \item ・ Machine learning
			      \item ・ Deep learning
		      \end{itemize}
		\item ■ Trading that isn't mathematically sophisticated
		      \begin{itemize}
			      \item ・ Trading that merely automates routine procedures
			            \begin{itemize}
				            \item - Simple agency trades
				            \item - Trades targeting momentary arbitrage opportunities
			            \end{itemize}
		      \end{itemize}
	\end{itemize}

\end{frame}

\subsection{Objectives of Algorithmic Trading}
\begin{frame}{Objectives of Algorithmic Trading}
	\scriptsize
	Expanding the stability of profits
	\begin{itemize}
		\item ■ Pursuit of returns
		      \begin{itemize}
			      \item ・ Automatically buying low and selling high
			      \item ・ Pursuit of profit opportunities
		      \end{itemize}
		\item ■ Cost reduction
		      \begin{itemize}
			      \item ・ Automation of routine execution tasks
			            \begin{itemize}
				            \item - Automated market making
				            \item - Automated basket trading
			            \end{itemize}
			      \item ・ Automatic selection of markets with optimal fees
			      \item ・ Reduction of costs due to market impact
			            \begin{itemize}
				            \item - Market impact refers to the increase in one's own trading costs when one's own execution causes prices to move unfavorably. definition
			            \end{itemize}
		      \end{itemize}
		\item ■ Risk control
		      \begin{itemize}
			      \item ・ Controlling the execution probability for desired quantities
			      \item ・ Controlling the market risk of one's own positions
		      \end{itemize}
	\end{itemize}
	\vspace{5ex}
	The pursuit of returns, cost reduction, and risk control involve trade-offs with each other.
\end{frame}

\subsection{Types of Algorithmic Trading}
\begin{frame}{Types of Algorithmic Trading}
	\scriptsize
	\begin{itemize}
		\item ■ Algorithms aiming to reduce trading costs
		      \begin{itemize}
			      \item ・ Market impact reduction
			      \item ・ Trading cost reduction
			            \begin{itemize}
				            \item - Execution algorithms
				                  \begin{itemize}
					                  \item - Iceberg, etc.
				                  \end{itemize}
				            \item - Benchmark execution algorithms
				                  \begin{itemize}
					                  \item - VWAP, etc.
				                  \end{itemize}
				            \item - These aim to reduce trading costs through effects such as splitting trades finely to hide (make less obvious) the execution volume from the market.
			            \end{itemize}
		      \end{itemize}
		\item ■ Algorithms aiming for profit opportunities
		      \begin{itemize}
			      \item ・ Market-making algorithms
			            \begin{itemize}
				            \item - Placing both sell and buy orders in the market, using the price difference as a source of profit.
			            \end{itemize}
			      \item ・ Arbitrage algorithms
			            \begin{itemize}
				            \item - Algorithms that earn profit by detecting when financial instruments of identical value can be traded at different prices.
			            \end{itemize}
			      \item ・ Directional algorithms
			            \begin{itemize}
				            \item - Algorithms that aim for profit from the price difference by predicting the market to buy low and sell high.
			            \end{itemize}
		      \end{itemize}
		\item ■ Market manipulation algorithms
		      \begin{itemize}
			      \item ・ Algorithms aiming to move the market in a favorable direction by misleading the market about the liquidity or trading intentions they provide.
		      \end{itemize}
	\end{itemize}
\end{frame}

\subsection{Users of Algorithms}
\begin{frame}{Users of Algorithms}
	\scriptsize
	\begin{itemize}
		\item ■ Some individual investors
		      \begin{itemize}
			      \item ・ Directional algorithms are used.
			      \item ・ Due to constraints of the system environment, algorithms advantaged by high speed and high frequency are difficult to use.
			            \begin{itemize}
				            \item - Market-making and arbitrage algorithms are difficult to use.
			            \end{itemize}
		      \end{itemize}
		\item ■ Institutional investors
		      \begin{itemize}
			      \item ・ Agency execution departments
			            \begin{itemize}
				            \item - Execute based on client requests.
				            \item - Execution for the fulfillment of their own best execution obligations.
			            \end{itemize}
			      \item ・ Proprietary trading departments
			            \begin{itemize}
				            \item - The objective is to obtain trading profits from the market.
				                  \begin{itemize}
					                  \item - Market-making algorithms
					                  \item - Arbitrage algorithms
					                  \item - Directional algorithms
				                  \end{itemize}
			            \end{itemize}
			      \item ・ Specific examples
			            \begin{itemize}
				            \item - Index managers use execution-type algorithms.
				            \item - If pursuing trading profits, algorithms similar to those in proprietary trading are used.
			            \end{itemize}
		      \end{itemize}
	\end{itemize}
\end{frame}

\subsection{HFT}
\begin{frame}{HFT \textnormal{(High-Frequency Trading)}}
	\scriptsize
	\begin{itemize}
		\item ■ HFT holds an advantage in market-making and arbitrage algorithms.
		\item ■ To conduct high-frequency and high-speed trading requires:
		      \begin{itemize}
			      \item 1. Speeding up information acquisition for trading decisions
			            \begin{itemize}
				            \item - Narrowing down the information utilized.
				            \item - Accelerating one's own systems.
			            \end{itemize}
			      \item 2. Speeding up the process from information processing to execution
			            \begin{itemize}
				            \item - Accelerating the algorithms.
			            \end{itemize}
			      \item 3. Speeding up the arrival of order information at the trading system
			            \begin{itemize}
				            \item - Installation of dedicated lines.
				            \item - Shortening transit time through the use of DMA (Direct Market Access).
				            \item - Placing orders from within the same network as the exchange via exchange colocation.
			            \end{itemize}
			      \item 4. Speeding up the processing speed per unit time of exchange and other trading systems
			            \begin{itemize}
				            \item - Speed is increasing due to the competition for market share among exchanges.
				            \item - In Japan as well, HFT became possible in 2010 when Arrowhead started operating on the TSE.
			            \end{itemize}
			      \item ・ Since bottlenecks must be avoided, acceleration is necessary in all four areas mentioned above.
		      \end{itemize}
	\end{itemize}
\end{frame}

\subsection{Pros and Cons of HFT}
\begin{frame}{Pros and Cons of HFT}
	\begin{itemize}
		\item ■ High-frequency and high-speed trading is thought to utilize market-making algorithms.
		      \begin{itemize}
			      \item ・ Therefore, there is an argument that it benefits general investors by supplying liquidity to the market.
			      \item ・ On the other hand, only HFT firms capture profit opportunities, impairing fairness among investors.
			      \item ・ There is also an argument that it harms market stability by repeatedly placing, modifying, and canceling orders at speeds invisible to general investors.
		      \end{itemize}
	\end{itemize}
\end{frame}

\begin{frame}[allowframebreaks]{References}
	\begin{thebibliography}{9}
		\bibitem{NTTData}
		The Essence of Algorithmic Trading: Strategies and Execution - NTT Data Financial Technology
		\bibitem{Adachi}
		Algorithmic Trading - Takanori Adachi
		\bibitem{Marcos}
		Advances in Financial Machine Learning - Lopez de Prado, Marcos
	\end{thebibliography}
\end{frame}

\end{document}
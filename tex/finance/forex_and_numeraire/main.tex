\documentclass[uplatex,a4j,12pt,dvipdfmx]{jsarticle}
\usepackage{amsmath,amsthm,amssymb,bm,color,enumitem,mathrsfs,url,epic,eepic,ascmac,ulem,here}
\usepackage[letterpaper,top=2cm,bottom=2cm,left=3cm,right=3cm,marginparwidth=1.75cm]{geometry}
\usepackage[english]{babel}
\usepackage[dvipdfm]{graphicx}
\usepackage[hypertex]{hyperref}
\title{Foreign Exchange and Numeraire}
\author{Masaru Okada
}
\date{ \today }

\begin{document}

\maketitle

\begin{abstract}
	These are my self-study notes for Chapter 3 of 'Financial Calculus: An Introduction to Derivative Pricing' by Martin Baxter and Andrew Rennie. Written on June 3, 2020.
\end{abstract}

The fundamental assets in the foreign exchange market are currencies.

Holding currency, just like holding a stock, involves risk.

For example, the exchange rate for one Japanese yen to US dollars, like a stock price, fluctuates from moment to moment.

This inherent risk gives rise to the demand for derivatives.

\section{The Black-Scholes Currency Model}

Let's denote the US dollar bond as $B_{t}$, the Japanese yen bond as $D_{t}$, and the exchange rate as $C_{t}$ (where 1 JPY $= C_{t}$ USD).

The Black-Scholes currency model then gives us the following equations:
%
%
%
\begin{eqnarray*}
	B_{t}
	&=&
	e^{rt}
	\\
	D_{t}
	&=&
	e^{ut}
	\\
	C_{t}
	&=&
	C_{0} \exp (\sigma W_{t} + \mu t )
\end{eqnarray*}
%
%
%
Here, $W_{t}$ is a $\mathbb{P}$-Brownian motion, and $r,u,\sigma,\mu$ are constants.

\subsection{For the US Dollar-Based Investor}

A US dollar-based investor can trade two types of assets: the US dollar bond $B_{t}$ and the yen bond converted to dollars, $C_{t}D_{t}$. Just as in the standard Black-Scholes model for stocks and bonds, a replicating portfolio can be constructed.

${}$

While $C_{t}$ represents the dollar price of one yen, the dollar-based investor cannot trade the yen in its raw cash form. If this were possible, it would create an arbitrage opportunity against holding yen bonds. Cash has a zero interest rate, while the yen bond yields an interest rate of $u$. As a result, market participants could make infinite profits by going long on an arbitrary amount of yen bonds and shorting the cash itself.

${}$

The asset $C_{t} D_{t}$ is tradable in dollars. It represents the dollar-denominated price of the yen bond, $D_{t}$.

With these two stochastic processes, $B_{t}$ and $C_{t}D_{t}$, we can construct a replicating portfolio.

\subsubsection{Constructing the Replicating Portfolio}

Using the tradable assets $B_{t}$ and $C_{t}D_{t}$, we can construct a replicating portfolio for a contract $X$ and determine its price using the no-arbitrage principle. This is done in three main steps:

${}$

\ \ \ \ 1. \ \ Find a measure $\mathbb{Q}^{\$}$ under which the process for the yen bond discounted by the dollar bond, $Z_{t} = B^{-1}_{t} C_{t} D_{t}$, becomes a martingale.

\ \ \ \ 2. \ \ Transform the contract $X$ into a process $E_{t} = \mathbb{E}_{\mathbb{Q}^{\$}}(B_{T}^{-1}X|\mathcal{F}_{t})$.

\ \ \ \ 3. \ \ Find a predictable process $\phi_{t}$ such that $dE_{t}=\phi_{t} dZ_{t}$.

${}$
The process for the yen bond discounted by the dollar bond, $Z_{t}$, is:
%
%
\begin{eqnarray*}
	Z_{t}
	&=&
	B^{-1}_{t} C_{t} D_{t}
	\\ &=&
	e^{-rt} e^{ut} C_{0} \exp (\sigma W_{t} + \mu t )
	\\ &=&
	C_{0} \exp \big[ \sigma W_{t} + (\mu + u - r)t \big]
\end{eqnarray*}
%
%
Its stochastic differential is:
%
%
\begin{eqnarray*}
	d Z_{t}
	&=&
	\left(
	\dfrac{\partial Z_{t}}{\partial t}
	\right)
	dt
	+
	\left(
	\dfrac{\partial Z_{t}}{\partial x}
	\right)
	d W_{t}
	+
	\dfrac{1}{2!}
	\left(
	\dfrac{\partial^{2} Z_{t}}{\partial x^{2}}
	\right)
	(d W_{t})^{2}
	\\
	\dfrac{dZ_{t}}{Z_{t}}
	&=&
	\left(
	\mu + u - r + \dfrac{1}{2} \sigma^{2}
	\right)
	dt
	+
	\sigma d W_{t}
\end{eqnarray*}
%
%

Now, we apply Girsanov's theorem.
%
%
\begin{eqnarray*}
	\gamma
	\ = \
	\dfrac{ \mu + u - r + \dfrac{1}{2} \sigma^{2} }{\sigma}
\end{eqnarray*}
%
%
By using this $\gamma$, the stochastic differential for the Brownian motion $W^{\$}_{t}$ under the measure $\mathbb{Q}^{\$}$ that makes $Z_{t}$ a martingale should be:
%
%
\begin{eqnarray*}
	d W^{\$}_{t}
	&=&
	d W_{t}
	+
	\gamma dt
\end{eqnarray*}
%
%
Furthermore, according to the Radon-Nikodym theorem, such a measure $\mathbb{Q}^{\$}$ is defined by:
%
%
\begin{eqnarray*}
	\dfrac{ d \mathbb{Q}^{\$} }{ d \mathbb{P} }
	&=&
	\exp
	\left(
	- \int^{T}_{0} \gamma dW_{t}
	- \dfrac{1}{2} \int^{T}_{0} \gamma^{2} dt
	\right)
	\\ &=&
	\exp
	\left(
	-\gamma W_{T}
	- \dfrac{1}{2} \gamma^{2}T
	\right)
\end{eqnarray*}
%
%
Under this measure $\mathbb{Q}^{\$}$, we have:
%
%
\begin{eqnarray*}
	\dfrac{ d Z_{t} }{ Z_{t} }
	&=&
	\sigma dW^{\$}_{t}
	\\
	Z_{t}
	&=&
	Z_{0} \exp
	\left(
	\int^{t}_{0} \sigma dW^{\$}_{s}
	-
	\dfrac{1}{2}
	\int^{t}_{0} \sigma^{2} ds
	\right)
	\\ &=&
	C_{0}
	\exp
	\left(
	\sigma W^{\$}_{t}
	-
	\dfrac{1}{2}
	\sigma^{2} t
	\right)
	\\
	C_{t}
	&=&
	B_{t} Z_{t} D^{-1}_{t}
	\\ &=&
	e^{rt}
	C_{0}
	\exp
	\left(
	\sigma W^{\$}_{t}
	-
	\dfrac{1}{2}
	\sigma^{2} t
	\right)
	e^{-ut}
	\\ &=&
	C_{0}
	\exp
	\left[
		\sigma W^{\$}_{t}
		+
		\left(
		r - u -
		\dfrac{1}{2}
		\sigma^{2}
		\right) t
		\right]
\end{eqnarray*}
%
%
This holds.

We define the conditional expectation under the measure $\mathbb{Q}^{\$}$ and the filtration $\mathcal{F}_{t}$ as:
$$
	E_{t}
	\ = \
	\mathbb{E}_{\mathbb{Q}^{ \$ }}
	( B_{T}^{-1} X | \mathcal{F}_{t} )
$$
For $s(<t)$, we get:
%
%
\begin{eqnarray*}
	\mathbb{E}_{\mathbb{Q}^{ \$ }}
	( E_{t} | \mathcal{F}_{s} )
	&=&
	\mathbb{E}_{\mathbb{Q}^{ \$ }}
	\Big(
	\mathbb{E}_{\mathbb{Q}^{ \$ }}
	( B_{T}^{-1} X | \mathcal{F}_{t} )
	\Big| \mathcal{F}_{s} \Big)
	\\ &=&
	\mathbb{E}_{\mathbb{Q}^{ \$ }}
	( B_{T}^{-1} X | \mathcal{F}_{s} )
	\\ &=&
	E_{s}
\end{eqnarray*}
%
%
So, $E_{t}$ is a $\mathbb{Q}^{ \$ }$-martingale. By the martingale representation theorem, a predictable process exists such that:
$$
	dE_{t} \ = \ \phi_{t} dZ_{t}
$$
We want to find the holdings of dollar-denominated currency, $S_{t} = C_{t} D_{t}$, and the holdings of the dollar bond, $B_{t}$, needed to construct the replicating portfolio at time $t$. Let these be $\phi_{t}$ and $\psi_{t}$, respectively.

The value of the replicating portfolio, $V_{t}$, is:
$$
	V_{t}
	\ = \
	\phi_{t} S_{t} + \psi_{t} B_{t}
$$
At maturity, the portfolio is identical to the contract, so:
$$
	X
	\ = \
	\phi_{T} S_{T} + \psi_{T} B_{T}
$$
The $\mathbb{Q}^{\$}$-martingale $E_{t}$ we constructed earlier becomes at $t=T$:
%
%
\begin{eqnarray*}
	E_{T}
	&=&
	\mathbb{E}_{\mathbb{Q}^{ \$ }}
	( B_{T}^{-1} X | \mathcal{F}_{T} )
	\\ &=&
	B_{T}^{-1} X
\end{eqnarray*}
%
%
Which means:
$$
	B_{T} E_{T}
	\ = \
	X
	\ = \
	\phi_{T} S_{T} + \psi_{T} B_{T}
$$
If this equality, $B_{t} E_{t} = V_{t} = \phi_{t} S_{t} + \psi_{t} B_{t}$, holds for all $t$, not just at $T$, then the dollar bond holdings $\psi_{t}$ of the replicating portfolio can be found by rearranging the equation:
$$
	\psi_{t}
	\ = \
	E_{t} - \phi_{t} Z_{t}
$$
Let's check if this assumption is correct.

${}$

Taking the stochastic differential of $V_{t} = B_{t} E_{t}$, and using the facts that $dE_{t} = \phi_{t} dZ_{t}$ and $E_{t} = \phi_{t} Z_{t} + \psi_{t}$, we get:
%
%
\begin{eqnarray*}
	d V_{t}
	&=&
	B_{t} d E_{t}
	+
	E_{t} d B_{t}
	\\ &=&
	B_{t} ( \phi_{t} dZ_{t} )
	+
	( \phi_{t} Z_{t} + \psi_{t} ) d B_{t}
	\\ &=&
	\phi_{t}
	(B_{t} dZ_{t} + Z_{t} dB_{t})
	+
	\psi_{t}
	dB_{t}
	\\ &=&
	\phi_{t}
	dS_{t}
	+
	\psi_{t}
	dB_{t}
\end{eqnarray*}
%
%
The portfolio is indeed self-financing. This confirms that we can construct a replicating portfolio by assuming dollar bond holdings are $\psi_{t} = E_{t} - \phi_{t} Z_{t}$.

${}$

The value of the portfolio that replicates contract $X$ is $V_{t}$. We've found that it can be expressed using the measure $\mathbb{Q}^{ \$ }$ that makes the dollar-denominated currency price $Z_{t}$ a martingale, as follows:
$$
	V_{t}
	=
	B_{t} E_{t}
	\ = \
	B_{t}
	\mathbb{E}_{\mathbb{Q}^{ \$ }}
	( B_{T}^{-1} X | \mathcal{F}_{t} )
$$
\subsection{The Forward Contract}

Let's consider a contract to buy one yen for $k$ dollars at a future time $T(>t)$.

The payoff at time $T$ is:
%
%
\begin{eqnarray*}
	X
	&=&
	C_{T} - k
	\\ &=&
	C_{0} \exp (\sigma W_{T} + \mu T) - k
	\\ &=&
	C_{0}
	\exp
	\left[
		\sigma W^{\$}_{T}
		+
		\left(
		r - u -
		\dfrac{1}{2}
		\sigma^{2}
		\right) T
		\right]
	- k
\end{eqnarray*}
%
%
The value at any time $t$ is:
%
%
\begin{eqnarray*}
	V_{t}
	&=&
	B_{t}
	\mathbb{E}_{\mathbb{Q}^{ \$ }}
	( B_{T}^{-1} X | \mathcal{F}_{t} )
	\\ &=&
	e^{-r(T-t)}
	\mathbb{E}_{\mathbb{Q}^{ \$ }}
	( C_{T} - k | \mathcal{F}_{t} )
	\\ &=&
	e^{-r(T-t)}
	\mathbb{E}_{\mathbb{Q}^{ \$ }}
	\Big(
	C_{0}
	\exp
	\left[
		\sigma W^{\$}_{T}
		+
		\left(
		r - u -
		\dfrac{1}{2}
		\sigma^{2}
		\right) T
		\right]
	- k
	\Big| \mathcal{F}_{t} \Big)
\end{eqnarray*}
%
%
The value of the contract at time zero (the present) should be zero under the no-arbitrage condition.
%
%
\begin{eqnarray*}
	0
	&=&
	V_{0}
	\\ &=&
	e^{-rT}
	\mathbb{E}_{\mathbb{Q}^{ \$ }}
	C_{0}
	\exp
	\left[
		\sigma W^{\$}_{T}
		+
		\left(
		r - u -
		\dfrac{1}{2}
		\sigma^{2}
		\right) T
		\right]
	- e^{-rT} k
\end{eqnarray*}
%
%
Therefore, the no-arbitrage delivery price $F$ (the value of $k$ for which $V_{0}=0$ at $t=0$) is:
%
%
\begin{eqnarray*}
	F
	&=&
	\mathbb{E}_{\mathbb{Q}^{ \$ }}
	C_{0}
	\exp
	\left[
		\sigma W^{\$}_{T}
		+
		\left(
		r - u -
		\dfrac{1}{2}
		\sigma^{2}
		\right) T
		\right]
	\\ &=&
	C_{0}
	\exp
	\left[
		\left(
		r - u -
		\dfrac{1}{2}
		\sigma^{2}
		\right) T
		\right]
	\times
	\mathbb{E}_{\mathbb{Q}^{ \$ }}
	\exp
	\sigma W^{\$}_{T}
	\\ &=&
	C_{0}
	\exp
	\left[
		\left(
		r - u -
		\dfrac{1}{2}
		\sigma^{2}
		\right) T
		\right]
	\times
	\exp
	\left(
	\dfrac{1}{2}
	\sigma^{2} T
	\right)
	\\ &=&
	C_{0}
	e^{(r-u)T}
\end{eqnarray*}
%
%
This value is equal to the yen-dollar exchange rate discounted by the interest rate difference between the two currencies.

Using $F$, we can also find the value of the forward contract at time $t$, $V_{t}$.

%
%
\begin{eqnarray*}
	V_{t}
	&=&
	e^{-r(T-t)}
	\mathbb{E}_{\mathbb{Q}^{ \$ }}
	(
	C_{T} - F
	| \mathcal{F}_{t} )
	\\ &=&
	e^{-r(T-t)}
	\mathbb{E}_{\mathbb{Q}^{ \$ }}
	(
	C_{T}
	| \mathcal{F}_{t} )
	- e^{-r(T-t)} F
	\\ &=&
	B_{t}
	\mathbb{E}_{\mathbb{Q}^{ \$ }}
	(B_{T}^{-1} C_{T} | \mathcal{F}_{t} )
	- e^{-r(T-t)} C_{0} e^{(r-u)T}
	\\ &=&
	C_{t} - e^{-uT} e^{rt} C_{0}
	\\ &=&
	e^{-uT}
	\left(
	e^{uT} C_{t}
	- e^{rt} C_{0}
	\right)
\end{eqnarray*}
%
%

The discounted value of the portfolio is:
%
%
\begin{eqnarray*}
	E_{t}
	&=&
	B^{-1}_{t} V_{t}
	\\ &=&
	e^{-rt}
	e^{-uT}
	\left(
	e^{uT} C_{t}
	- e^{rt} C_{0}
	\right)
	\\ &=&
	e^{-rt} C_{t} - e^{-uT} C_{0}
	\\ &=&
	e^{uT} Z_{t} - e^{-uT} C_{0}
\end{eqnarray*}
%
%
\footnote{The final equation isn't right. Working backward, we see $e^{uT}Z_{t}=e^{uT}(B_{t}^{-1} C_{t} D_{t})=e^{uT} e^{-rt} C_{t} e^{ut}$. Some calculation must be incorrect.}
The stochastic differential is $dE_{t} = e^{-uT} dZ_{t}$.
The holdings of stock $\phi_{t}$ and bonds $\psi_{t}$ needed to construct the replicating portfolio are constant with respect to $t$:
%
%
\begin{eqnarray*}
	\phi_{t} &=& e^{-uT} = D_{T}^{-1}
	\\[3mm]
	\psi_{t} &=& E_{t} - \phi_{t} Z_{t}
	\\ &=&
	(e^{uT} Z_{t} - e^{-uT} C_{0}) - e^{-uT} Z_{t}
	\\ &=&
	- e^{-uT} C_{0}
	\\ &=&
	- D_{T}^{-1} C_{0}
\end{eqnarray*}
%
%
\footnote{The value for $\psi_{t}$ doesn't match the textbook. The exponent sign for the coefficient of $Z_{t}$ is different.}

\subsection{For the Japanese Yen-Based Investor}

Unlike the US dollar-based investor, the Japanese yen-based investor is interested in the yen-denominated prices of tradable assets.

First, the yen bond $D_{t} = e^{ut}$ is tradable.

Furthermore, the dollar bond denominated in yen, $C^{-1}_{t} B_{t}$, is also tradable.

If we consider the exchange rate of one dollar to yen, $C^{-1}_{t}$, we have:
%
%
\begin{eqnarray*}
	C^{-1}_{t}
	&=&
	C_{0}^{-1}
	\exp ( - \sigma W_{t} - \mu t )
\end{eqnarray*}
%
%
With these two assets, the yen bond $D_{t}$ and the yen-denominated dollar bond $C^{-1}_{t} B_{t}$, we can replicate a risk-free portfolio.

The price of the dollar bond discounted by the yen bond is:
%
%
\begin{eqnarray*}
	Y_{t}
	&=&
	D^{-1}_{t}
	C^{-1}_{t}
	B_{t}
	\\ &=&
	e^{-ut}
	C_{0}^{-1}
	\exp ( - \sigma W_{t} - \mu t )
	e^{rt}
	\\ &=&
	C_{0}^{-1}
	\exp ( - \sigma W_{t} - (\mu + u -r) t )
\end{eqnarray*}
%
%
The stochastic differential is:
%
%
\begin{eqnarray*}
	d Y_{t}
	&=&
	\dfrac{\partial Y_{t}}{\partial t} dt
	+
	\dfrac{\partial Y_{t}}{\partial x} dW_{t}
	+
	\dfrac{1}{2!}
	\dfrac{\partial^{2} Y_{t}}{\partial x^{2}} (dW_{t})^{2}
	\\ &=&
	- (\mu + u -r) Y_{t} dt
	- \sigma Y_{t} d W_{t}
	+ \dfrac{1}{2} \sigma^{2} Y_{t} dt
	\\[3mm]
	\dfrac{dY_{t}}{Y_{t}}
	&=&
	- \sigma d W_{t}
	- \left( \mu + u -r + \dfrac{1}{2} \sigma^{2} \right) dt
\end{eqnarray*}
%
%
Therefore, for $W^{\tilde{\mathbb{Q}}}_{t}$ to be a $\tilde{\mathbb{Q}}$-Brownian motion, we need to introduce a new measure $\tilde{\mathbb{Q}}$ such that the discounted price $Y_{t}$ becomes a martingale.
%
%
\begin{eqnarray*}
	d W^{\tilde{\mathbb{Q}}}_{t}
	&=&
	d W_{t}
	+
	\dfrac{\mu + u -r + \dfrac{1}{2} \sigma^{2}}{\sigma} dt
	\\
	W^{\tilde{\mathbb{Q}}}_{t}
	&=&
	W_{t}
	+
	\dfrac{\mu + u -r + \dfrac{1}{2} \sigma^{2}}{\sigma} t
\end{eqnarray*}
%
%
\subsection*{Option Pricing in the Yen World}

A yen-denominated payoff $X$ at time $T$ has a value at time $t$ of:
$$
	U_{t}
	\ = \
	D_{t}
	\mathbb{E}_{\tilde{\mathbb{Q}}}
	( D^{-1}_{T} X | \mathcal{F}_{t} )
$$
Here, $\tilde{\mathbb{Q}}$ is the martingale measure for $Y_{t}$, the asset value discounted by the yen bond.

\subsection{Changing the Numeraire}

A concern arises: will the US dollar-based investor and the Japanese yen-based investor value the same security differently?

In the dollar world, the value of a payoff $X$ at time $t$ is:
$$
	V_{t}
	\ = \
	B_{t}
	\mathbb{E}_{\mathbb{Q}^{\$}}
	( B^{-1}_{T} X | \mathcal{F}_{t} )
$$
The unit is dollars.

In the yen world, the same contract is a payment of $C^{-1}_{T} X$ yen, not $X$ dollars. Therefore, its value at time $t$ is:
$$
	U_{t}
	\ = \
	D_{t}
	\mathbb{E}_{\tilde{\mathbb{Q}}}
	( D^{-1}_{T} ( C^{-1}_{T} X ) | \mathcal{F}_{t} )
$$
The unit is yen.

Do these two values actually coincide?

Is the dollar-equivalent value of the price determined in the yen world, $C_{t} U_{t}$, equal to the original $V_{t}$?

${}$

The $\mathbb{Q}^{\$}$-Brownian motion $W^{\mathbb{Q}^{\$}}_{t}$ and the $\tilde{\mathbb{Q}}$-Brownian motion $W^{\tilde{\mathbb{Q}}}_{t}$ are expressed using the $\mathbb{P}$-Brownian motion $W_{t}$ as follows:
%
%
\begin{eqnarray*}
	W^{\mathbb{Q}^{\$}}_{t}
	&=&
	W_{t}
	+
	\dfrac{\mu + u -r - \dfrac{1}{2} \sigma^{2}}{\sigma} t
	\\
	W^{\tilde{\mathbb{Q}}}_{t}
	&=&
	W_{t}
	+
	\dfrac{\mu + u -r + \dfrac{1}{2} \sigma^{2}}{\sigma} t
\end{eqnarray*}
%
%
This means:
%
%
\begin{eqnarray*}
	W^{\tilde{\mathbb{Q}}}_{t}
	&=&
	W^{\mathbb{Q}^{\$}}_{t}
	-
	\sigma t
	\\
	d W^{\tilde{\mathbb{Q}}}_{t}
	&=&
	d W^{\mathbb{Q}^{\$}}_{t}
	-
	\sigma d t
\end{eqnarray*}
%
%
Therefore, by the reverse of Girsanov's theorem, the Radon-Nikodym derivative must be:
%
%
\begin{eqnarray*}
	\dfrac{d \tilde{\mathbb{Q}}}{d \mathbb{Q}^{\$}}
	&=&
	\exp
	\left(
	- \int^{T}_{0} (- \sigma) d W^{\mathbb{Q}^{\$}}_{t}
	- \dfrac{1}{2} \int^{T}_{0} ( - \sigma)^{2} d t
	\right)
	\\ &=&
	\exp
	\left(
	\sigma W^{\mathbb{Q}^{\$}}_{T}
	- \dfrac{1}{2} \sigma^{2} T
	\right)
\end{eqnarray*}
%
%
Taking the conditional expectation of this Radon-Nikodym derivative under the measure $\mathbb{Q}^{\$}$ and the filtration $\mathcal{F}_{t}$, we get:
%
%
\begin{eqnarray*}
	\xi_{t}
	&=&
	\mathbb{E}_{Q^{\$}}
	\left(
	\left.
	\dfrac{d \tilde{\mathbb{Q}}}{d \mathbb{Q}^{\$}}
	\right|
	\mathcal{F}_{t}
	\right)
	\\ &=&
	\mathbb{E}_{Q^{\$}}
	\left[
		\left.
		\exp
		\left(
		\sigma W^{\mathbb{Q}^{\$}}_{T}
		- \dfrac{1}{2} \sigma^{2} T
		\right)
		\right|
		\mathcal{F}_{t}
		\right]
	\\ &=&
	\exp
	\left(
	- \dfrac{1}{2} \sigma^{2} T
	\right)
	\\ && \hspace{-5mm} \times \
	\mathbb{E}_{Q^{\$}}
	\left[
		\left.
		\exp
		\left(
		\sigma W^{\mathbb{Q}^{\$}}_{t}
		\right)
		\exp
		\left\{
		\sigma
		\left(
		W^{\mathbb{Q}^{\$}}_{T} - W^{\mathbb{Q}^{\$}}_{t}
		\right)
		\right\}
		\right|
		\mathcal{F}_{t}
		\right]
\end{eqnarray*}
%
%
The factor inside the expectation is:
%
%
\begin{eqnarray*}
	&&
	\exp
	\left\{
	\sigma
	\left(
	W^{\mathbb{Q}^{\$}}_{T} - W^{\mathbb{Q}^{\$}}_{t}
	\right)
	\right\}
	\\ &=&
	\exp
	\left\{
	\sigma \sqrt{T-t}
	\dfrac{
		W^{\mathbb{Q}^{\$}}_{T} - W^{\mathbb{Q}^{\$}}_{t}
	}
	{
		\sqrt{T-t}
	}
	\right\}
\end{eqnarray*}
%
%
This factor,
$$
	\dfrac{
		W^{\mathbb{Q}^{\$}}_{T} - W^{\mathbb{Q}^{\$}}_{t}
	}
	{
		\sqrt{T-t}
	}
$$
is a standard normal random variable following an $N(0,1)$ distribution under the measure $\mathbb{Q}^{\$}$. Let's call this variable $Z$.

The expectation can then be broken down into a product of a $\mathcal{F}_{t}$-measurable factor,
$$
	\exp
	\left(
	\sigma W^{\mathbb{Q}^{\$}}_{t}
	\right)
$$
and an $\mathcal{F}_{t}$-independent random variable,
$$
	\exp
	\left(
	Z \sigma \sqrt{T-t}
	\right)
$$
Thus:
%
%
\begin{eqnarray*}
	&&
	\hspace{-10mm}
	\mathbb{E}_{Q^{\$}}
	\left[
		\left.
		\exp
		\left(
		\sigma W^{\mathbb{Q}^{\$}}_{t}
		\right)
		\exp
		\left(
		\sigma
		\sqrt{T-t}
		\dfrac{
			W^{\mathbb{Q}^{\$}}_{T} - W^{\mathbb{Q}^{\$}}_{t}
		}
		{
			\sqrt{T-t}
		}
		\right)
		\right|
		\mathcal{F}_{t}
		\right]
	\\ &=&
	\mathbb{E}
	\left[
		\exp
		\left(
		\sigma W^{\mathbb{Q}^{\$}}_{t}
		\right)
		\exp
		\left(
		Z \sigma \sqrt{T-t}
		\right)
		\right]
	\\ &=&
	\exp
	\left(
	\sigma W^{\mathbb{Q}^{\$}}_{t}
	\right)
	\dfrac{1}{\sqrt{2 \pi}}
	\int^{\infty}_{- \infty}
	\exp
	\left(
	z \sigma \sqrt{T-t}
	\right)
	e^{- \frac{1}{2} z^{2}}
	dz
	\\ &=&
	\exp
	\left(
	\sigma W^{\mathbb{Q}^{\$}}_{t}
	\right)
	\exp
	\left(
	\dfrac{1}{2} \left( \sigma \sqrt{T-t} \right)^{2}
	\right)
\end{eqnarray*}
%
%
From the above, we can derive:
%
%
\begin{eqnarray*}
	\xi_{t}
	&=&
	\mathbb{E}_{Q^{\$}}
	\left(
	\left.
	\dfrac{d \tilde{\mathbb{Q}}}{d \mathbb{Q}^{\$}}
	\right|
	\mathcal{F}_{t}
	\right)
	\\ &=&
	\exp
	\left(
	- \dfrac{1}{2} \sigma^{2} T
	\right)
	\\ && \hspace{-15mm} \times \
	\mathbb{E}_{Q^{\$}}
	\left[
		\left.
		\exp
		\left(
		\sigma W^{\mathbb{Q}^{\$}}_{t}
		\right)
		\exp
		\left\{
		\sigma
		\left(
		W^{\mathbb{Q}^{\$}}_{T} - W^{\mathbb{Q}^{\$}}_{t}
		\right)
		\right\}
		\right|
		\mathcal{F}_{t}
		\right]
	\\ &=&
	\exp
	\left(
	- \dfrac{1}{2} \sigma^{2} T
	\right)
	\\ && \hspace{-15mm} \times \
	\exp
	\left(
	\sigma W^{\mathbb{Q}^{\$}}_{t}
	\right)
	\exp
	\left(
	\dfrac{1}{2} \left( \sigma \sqrt{T-t} \right)^{2}
	\right)
	\\ &=&
	\exp
	\left(
	\sigma W^{\mathbb{Q}^{\$}}_{t} - \dfrac{1}{2} \sigma^{2} t
	\right)
\end{eqnarray*}
%
%
We have already shown that the price of the yen bond discounted by the dollar bond is:
%
%
\begin{eqnarray*}
	Z_{t}
	&=&
	B^{-1}_{t}
	C_{t}
	D_{t}
	\\ &=&
	C_{0}
	\exp
	\left(
	\sigma W^{\mathbb{Q}^{\$}}_{t} - \dfrac{1}{2} \sigma^{2} t
	\right)
	\\ &=&
	C_{0} \xi_{t}
\end{eqnarray*}
%
%
Therefore, the price of the yen bond discounted by the dollar bond, $Z_{t}$, is proportional to the Radon-Nikodym process $\xi_{t}$.

${}$

Using this process, the dollar-converted value of the price set in the yen world, $C_{t} U_{t}$ (where $U_{t}$ is the price for the payoff $C^{-1}_{T} X$ described earlier), is:
%
%
\begin{eqnarray*}
	C_{t} U_{t}
	&=&
	C_{t}
	D_{t}
	\mathbb{E}_{\tilde{\mathbb{Q}}}
	( D^{-1}_{T} ( C^{-1}_{T} X ) | \mathcal{F}_{t} )
	\\ &=&
	C_{t} D_{t} \xi^{-1}_{t}
	\mathbb{E}_{\mathbb{Q}^{\$}}
	( \xi^{-1}_{T} D^{-1}_{T}  C^{-1}_{T} X | \mathcal{F}_{t} )
	\\ &=&
	B_{t}
	\mathbb{E}_{\mathbb{Q}^{\$}}
	( B^{-1}_{T} X | \mathcal{F}_{t} )
	\\ &=&
	V_{t}
\end{eqnarray*}
%
%
This shows that a dollar-denominated payoff $X$ at time $T$ has the same value at any time $t(<T)$, regardless of whether it is priced from the dollar world or the yen world.
\begin{thebibliography}{9}
	\bibitem{BaxterRennie}
	Financial Calculus - An Introduction to Derivative Pricing - Martin Baxter, Andrew Rennie
\end{thebibliography}
\end{document}
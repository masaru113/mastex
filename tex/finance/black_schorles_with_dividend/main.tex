\documentclass[uplatex,a4j,12pt,dvipdfmx]{jsarticle}
\usepackage{amsmath,amsthm,amssymb,bm,color,enumitem,mathrsfs,url,epic,eepic,ascmac,ulem,here}
\usepackage[letterpaper,top=2cm,bottom=2cm,left=3cm,right=3cm,marginparwidth=1.75cm]{geometry}
\usepackage[english]{babel}
\usepackage[dvipdfm]{graphicx}
\usepackage[hypertex]{hyperref}
\title{Stocks with Dividends}
\author{Masaru Okada
}
\date{ \today }

\begin{document}

\maketitle

\begin{abstract}
	These notes provide a mathematical framework for pricing financial derivatives on dividend-paying stocks by adapting the Black-Scholes model. They explore models for both continuous and discrete periodic dividends, demonstrating how to price instruments like forwards, call options, and structured products by creating a new tradable asset that accounts for dividend reinvestment. This text originates from a self-study session on Chapter 3 of 'Financial Calculus - An Introduction to Derivative Pricing' by Martin Baxter and Andrew Rennie, written on June 3, 2020.
\end{abstract}

\ \\

In the models considered so far, stocks have been treated as pure assets.

Here, we will see how to modify the model to incorporate dividends.

\section{A Model for Stocks with Continuous Dividends}
Let the stock price $S_{t}$ and the cash bond $B_{t}$ follow the Black-Scholes model
%
\begin{eqnarray*}
	S_{t}
	&=&
	S_{0} \exp (\sigma W_{t} + \mu t )
	\\
	B_{t}
	&=&
	B_{0} \exp (rt)
\end{eqnarray*}
%
Furthermore, let the dividend paid between time $t$ and $t+dt$ be
$$
	\delta S_{t} dt
	\ = \
	\delta S_{0} \exp (\sigma W_{t} + \mu t ) dt
$$
using a constant of proportionality $\delta$.

${}$

It's important to note that in this model, $S_{t}$ is not a tradable asset.

If one were to buy a stock priced at $S_{0}$ now and hold it until time $t$, the existence of dividends means that the present value $S_{0}$ would account for a value greater than $S_{t}$.

Under this model, the stochastic process $S_{t}$ no longer represents the value of an asset.
${}$

The goal is to transform this non-tradable stochastic process $S_{t}$ into the form of a tradable asset through some clever correspondence, and then to construct a replicating portfolio based on this newly created asset.

${}$

Since the dividend per share over an infinitesimal time $dt$ is $\delta S_{t} dt$, consider a strategy of continuously buying an additional $\delta dt$ units of the stock during that infinitesimal time.

In this case, if the number of shares held is $a_{t}$, then
$$
	d a_{t} \ = \ a_{t} \times \delta dt
$$
so, solving this differential equation under $a_{0}=1$ gives,
$$
	a_{t} \ = \ \exp ( \delta t )
$$
meaning that at time $t$, the number of shares held has become $\exp ( \delta t )$ units.

The value of the portfolio under this strategy (which is simply to continuously buy more stock) is
%
\begin{eqnarray*}
	\tilde{S}_{t}
	&=&
	S_{0} \exp (\sigma W_{t} + \mu t ) \times \exp ( \delta t )
	\\ &=&
	S_{0} \exp (\sigma W_{t} + ( \mu + \delta ) t )
\end{eqnarray*}
%
This new stochastic process $\tilde{S}_{t}$ is the 'price of the reinvested stock' and becomes a tradable asset in this model.

${}$

Under the assumption that the dividend is a constant proportion of the stock price, the reinvested stock becomes a tradable asset. If, however, the dividend amount were predetermined and independent of the stock price, it would be standard to reinvest it into the cash bond.

\subsection{Replication Strategy using Stocks with Continuous Dividends}

Consider a replication strategy by replacing the stock and cash bond portfolio $(\phi_{t},\psi_{t})$ with a reinvested stock and cash bond portfolio $(\tilde{\phi}_{t},\psi_{t})$.

The value of this portfolio is
$$
	V_{t}
	\ = \
	\phi_{t} S_{t} + \psi_{t} B_{t}
	\ = \
	\tilde{\phi}_{t} \tilde{S}_{t} + \psi_{t} B_{t}
$$
where $\tilde{\phi}_{t}$ is defined as
$$
	\tilde{\phi}_{t} \ = \
	e^{- \delta t} \phi_{t}
$$
In this case, $dV_{t}$ becomes
%
\begin{eqnarray*}
	dV_{t}
	&=&
	\tilde{\phi}_{t} d \tilde{S}_{t} + \psi_{t} d B_{t}
\end{eqnarray*}
%
\footnote{Derivation unclear.}

On the other hand, in the portfolio with the original stock and bond, it seems to become
%
\begin{eqnarray*}
	dV_{t}
	&=&
	\phi_{t} d S_{t} + \psi_{t} d B_{t}
	\ + \
	\phi_{t} \delta S_{t} dt
\end{eqnarray*}
%
\footnote{Derivation unclear.} which would mean that gains and losses arise not only from trading (the $dS_{t}$ and $dB_{t}$ terms) but also from the dividend $\phi_{t} \delta S_{t} dt$.

${}$

Now, to find a measure under which the discounted value of the reinvested stock
$$
	\tilde{Z}_{t}
	\ = \
	B^{-1}_{t}
	\tilde{S}_{t}
$$
is a martingale.
%
%
\begin{eqnarray*}
	\tilde{Z}_{t}
	&=&
	Z_{0} \exp (\sigma W_{t} + ( \mu + \delta -r ) t )
	\\
	d \tilde{Z}_{t}
	&=&
	\tilde{Z}_{t}
	\left( \sigma dW_{t} + \left( \mu + \delta -r + \dfrac{1}{2} \sigma^{2} \right) dt \right)
\end{eqnarray*}
%
%

Therefore, by introducing a new measure $\mathbb{Q}$ under which the Brownian motion $W^{\mathbb{Q}}$ is
$$
	d W^{\mathbb{Q}}
	\ = \
	d W_{t} + \dfrac{\mu + \delta -r + \dfrac{1}{2} \sigma^{2}}{\sigma} dt
$$
under this martingale measure $\mathbb{Q}$, we have
%
%
\begin{eqnarray*}
	d \tilde{Z}_{t}
	&=&
	\sigma \tilde{Z}_{t} d W^{\mathbb{Q}}_{t}
	\\
	\tilde{Z}_{t}
	&=&
	\tilde{Z}_{0}
	\exp
	\left(
	\sigma W^{\mathbb{Q}}_{t}
	- \dfrac{1}{2} \sigma^{2} t
	\right)
	\\
	\tilde{S}_{t}
	&=&
	B_{t} \tilde{Z}_{t}
	\\ &=&
	\tilde{S}_{0}
	\exp
	\left(
	\sigma W^{\mathbb{Q}}_{t}
	+
	\left(
	r
	-
	\dfrac{1}{2} \sigma^{2}
	\right)
	t
	\right)
	\\
	S_{t}
	&=&
	e^{-\delta t} \tilde{S}_{t}
	\\ &=&
	S_{0}
	\exp
	\left(
	\sigma W^{\mathbb{Q}}_{t}
	+
	\left(
	r
	- \delta
	- \dfrac{1}{2} \sigma^{2}
	\right)
	t
	\right)
\end{eqnarray*}
%
%
(For readability, $S_{0}=\tilde{S}_{0}=\tilde{Z}_{0}$ is used.)
${}$

To hedge a contract $X$ at maturity $T$, we use the martingale representation theorem. That is, there exists a predictable process $\tilde{\phi}_{t}$ that satisfies the following equation.
%
%
\begin{eqnarray*}
	E_{t}
	&=&
	\mathbb{E}_{\mathbb{Q}} ( B_{T}^{-1} X | \mathcal{F}_{t} )
	\\
	&=&
	\mathbb{E}_{\mathbb{Q}} ( B_{T}^{-1} X )
	+
	\int^{t}_{0} \tilde{\phi}_{s} d \tilde{Z}_{s}
\end{eqnarray*}
%
%

Using this, the trading strategy is to hold $\tilde{\phi}_{t}$ units of the reinvested stock and
$\psi_{t}=E_{t} - \tilde{\phi}_{t} \tilde{Z}_{t}$
units of the bond.
\subsection{Forwards}

A contract $X$ to purchase one unit of a stock with continuous dividends at time $T$ for $k$ yen has a payoff of
$$
	X \ = \ S_{T} - k
$$
The price of this $X$ at time $t$, denoted $V_{t}$, is
%
%
\begin{eqnarray*}
	V_{t}
	&=&
	B_{t}
	\mathbb{E}_{\mathbb{Q}} ( B^{-1}_{T} X | \mathcal{F}_{t} )
	\\ &=&
	\mathbb{E}_{\mathbb{Q}} ( e^{-r(T-t)} (S_{T} - k) | \mathcal{F}_{t} )
	\\ &=&
	e^{-r(T-t)} \mathbb{E}_{\mathbb{Q}} ( S_{T}  | \mathcal{F}_{t} ) - e^{-r(T-t)} k
\end{eqnarray*}
%
%
Here,
%
%
\begin{eqnarray*}
	&&
	S_{T}
	\\ &=&
	S_{0}
	\exp
	\left(
	\sigma W^{\mathbb{Q}}_{T}
	+
	\left(
	r
	- \delta
	- \dfrac{1}{2} \sigma^{2}
	\right)
	T
	\right)
	\\ &=&
	S_{t}
	\exp
	\left\{
	\sigma ( W^{\mathbb{Q}}_{T} - W^{\mathbb{Q}}_{t} )
	+
	\left(
	r
	- \delta
	- \dfrac{1}{2} \sigma^{2}
	\right)
	(T-t)
	\right\}
	\\ &=&
	S_{t}
	\exp
	\left\{
	\sigma \sqrt{T-t}
	\dfrac{ W^{\mathbb{Q}}_{T} - W^{\mathbb{Q}}_{t} }
	{\sqrt{T-t}}
	+
	\left(
	r
	- \delta
	- \dfrac{1}{2} \sigma^{2}
	\right)
	(T-t)
	\right\}
\end{eqnarray*}
%
%
but the factor
$$
	\dfrac{ W^{\mathbb{Q}}_{T} - W^{\mathbb{Q}}_{t} }
	{\sqrt{T-t}}
$$
is a standard normal random variable $N(0,1)$ under the measure $\mathbb{Q}$.

Letting this variable be $Y$,
$\mathbb{E}_{\mathbb{Q}} ( S_{T}  | \mathcal{F}_{t} )$
is the product of the $\mathcal{F}_{t}$-predictable factor $S_{t}$ and the $\mathcal{F}_{t}$-independent factor
$$
	\exp
	\left\{
	\sigma \sqrt{T-t}
	Y
	+
	\left(
	r
	- \delta
	- \dfrac{1}{2} \sigma^{2}
	\right)
	(T-t)
	\right\}
$$
Therefore,
%
%
\begin{eqnarray*}
	&&
	\hspace{-10mm}
	\mathbb{E}_{\mathbb{Q}} ( S_{T}  | \mathcal{F}_{t} )
	\\[2mm] &&
	\hspace{-10mm}
	= \
	S_{t}
	\dfrac{1}{\sqrt{2\pi}}
	\int^{\infty}_{-\infty}
	\!\!\!
	e^{-\frac{1}{2} y^{2}}
	\exp
	\left\{
	\sigma \sqrt{T-t}
	y
	+
	\left(
	r
	- \delta
	- \dfrac{1}{2} \sigma^{2}
	\right)
	(T-t)
	\right\}
	dy
	\\ &&
	\hspace{-10mm}
	= \
	S_{t}
	\exp
	\left(
	r
	- \delta
	- \dfrac{1}{2} \sigma^{2}
	\right)
	(T-t)
	\times
	\dfrac{1}{\sqrt{2\pi}}
	\int^{\infty}_{-\infty}
	e^{-\frac{1}{2} y^{2}}
	e^{\sigma \sqrt{T-t} y}
	dy
	\\ &&
	\hspace{-10mm}
	= \
	S_{t}
	\exp
	\left(
	r
	- \delta
	- \dfrac{1}{2} \sigma^{2}
	\right)
	(T-t)
	\times
	\exp
	\left(
	\dfrac{1}{2} (\sigma \sqrt{T-t})^{2}
	\right)
	\\ &&
	\hspace{-10mm}
	= \
	S_{t} e^{(r-\delta)(T-t)}
\end{eqnarray*}
%
%

From the above,
%
%
\begin{eqnarray*}
	V_{t}
	&=&
	\mathbb{E}_{\mathbb{Q}} ( e^{-r(T-t)} (S_{T} - k) | \mathcal{F}_{t} )
	\\ &=&
	e^{-r(T-t)} \mathbb{E}_{\mathbb{Q}} ( S_{T}  | \mathcal{F}_{t} ) - e^{-r(T-t)} k
	\\ &=&
	e^{-r(T-t)} S_{t} e^{(r-\delta)(T-t)} - e^{-r(T-t)} k
	\\ &=&
	e^{-\delta(T-t)}S_{t} - e^{-r(T-t)} k
\end{eqnarray*}
%
%
was obtained.

The forward price $F$ is for a contract $X$ whose price is zero at present ($t=0$) due to the no-arbitrage condition, so
%
%
\begin{eqnarray*}
	0\
	( \ = \
	V_{0})
	&=&
	e^{- \delta ( T - 0 ) } S_{0} - e^{-r(T-0)} F
	\\
	\Longleftrightarrow \ \ \
	F &=&
	e^{(r - \delta) T } S_{0}
\end{eqnarray*}
%
%

In the case of no dividends, $F=e^{rT}S_{0}$, so it was sufficient to hold the stock at time zero and keep it until maturity $T$.

With continuous dividends, one must adopt a strategy of holding the stock at $t=0$ and, furthermore, using the dividend income from the purchased stock to continuously reinvest until $t=T$.

\subsection{Call Options}

The payoff of a call option with strike price $k$ and exercise time $T$ is
$$
	X
	\ = \
	(S_{T} - k)^{+}
$$
and its price at time $t(<T)$ is
%
%
\begin{eqnarray*}
	V_{t}
	&=&
	B_{t}
	\mathbb{E}_{\mathbb{Q}} ( B^{-1}_{T} X | \mathcal{F}_{t} )
	\\ &=&
	e^{-r(T-t)}
	\mathbb{E}_{\mathbb{Q}} \left[ \left. \left( S_{T} - k \right)^{+} \right| \mathcal{F}_{t} \right]
\end{eqnarray*}
%
%
The calculation of an expectation containing a Max of a 'log-normal random variable + constant' like this is well-known, and the following formula is often used.

${}$

\subsubsection*{Formula for the Expected Value of Max(log-normal variable + constant, 0)}

When $Z$ is a random variable following $N(0,1)$, the following formula holds.
%
%
\begin{eqnarray*}
	&&
	\mathbb{E}
	\left\{
	F
	\exp
	\left(
	\bar{\sigma} Z - \dfrac{1}{2} \bar{\sigma}^{2}
	\right)
	-k
	\right\}^{+}
	\\ &=&
	F
	\ \! \Phi
	\left(
	\dfrac{
		\log \dfrac{F}{k} + \dfrac{1}{2} \bar{\sigma}^{2}
	}
	{\bar{\sigma}}
	\right)
	-
	k
	\ \! \Phi
	\left(
	\dfrac{
		\log \dfrac{F}{k} - \dfrac{1}{2} \bar{\sigma}^{2}
	}
	{\bar{\sigma}}
	\right)
\end{eqnarray*}
%
%
Alternatively, writing
$$
	d_{\pm}
	\ = \
	\dfrac{
		\log (F/k)
	}
	{\bar{\sigma}}
	\pm \dfrac{1}{2} \bar{\sigma}
$$
gives
$$
	\mathbb{E}
	\left(
	F e^{\bar{\sigma}Z - \frac{1}{2} \bar{\sigma}^{2} }
	-
	k
	\right)^{+}
	\ = \
	F
	\ \! \Phi
	\left(
	d_{+}
	\right)
	-
	k
	\ \! \Phi
	\left(
	d_{-}
	\right)
$$
where $F,\bar{\sigma},k$ are constants. Also, the function $\Phi(x)$ is the probability that a $N(0,1)$ variable is less than or equal to $x$, expressed specifically as:
$$
	\Phi(x)
	\ = \
	\dfrac{1}{\sqrt{2 \pi}}
	\int^{x}_{- \infty} \exp \left( - \dfrac{y^{2}}{2} \right) dy
$$

${}$

Using the above formula, the following factor
$$
	\dfrac{W^{\mathbb{Q}}_{T} - W^{\mathbb{Q}}_{t}}{\sqrt{T-t}}
$$
becomes a standard normal random variable under the measure $\mathbb{Q}$.
Letting this variable be $Y$,
%
%
\begin{eqnarray*}
	&&
	S_{T}
	\\ && \hspace{-15mm} = \
	S_{0}
	\exp
	\left(
	\sigma W^{\mathbb{Q}}_{T}
	+
	\left(
	r
	- \delta
	- \dfrac{1}{2} \sigma^{2}
	\right)
	T
	\right)
	\\ && \hspace{-15mm} = \
	S_{t}
	\exp
	\left\{
	\sigma ( W^{\mathbb{Q}}_{T} - W^{\mathbb{Q}}_{t} )
	+
	\left(
	r
	- \delta
	- \dfrac{1}{2} \sigma^{2}
	\right)
	(T-t)
	\right\}
	\\ && \hspace{-15mm} = \
	S_{t}
	e^{(r - \delta )(T-t)}
	\exp
	\left\{
	\sigma \sqrt{T-t}
	\dfrac{ W^{\mathbb{Q}}_{T} - W^{\mathbb{Q}}_{t} }
	{\sqrt{T-t}}
	-
	\dfrac{1}{2}
	\left( \sigma \sqrt{T-t} \right)^{2}
	\right\}
\end{eqnarray*}
%
%
so, if we set $\sigma \sqrt{T-t}$ as $\tilde{\sigma}$, then
%
%
\begin{eqnarray*}
	S_{T}
	&=&
	S_{t}
	e^{(r - \delta )(T-t)}
	\exp
	\left(
	\tilde{\sigma} Y
	-
	\dfrac{1}{2}
	\tilde{\sigma}^{2}
	\right)
\end{eqnarray*}
%
%
The process $S_{T}$ is separated into the product of the $\mathcal{F}_{t}$-predictable process $S_{t}$ and the $\mathcal{F}_{t}$-independent process
$$
	\exp
	\left(
	\tilde{\sigma} Y
	-
	\dfrac{1}{2}
	\tilde{\sigma}^{2}
	\right)
$$
so,
%
%
\begin{eqnarray*}
	&&
	\mathbb{E}_{\mathbb{Q}} \left[ \left. \left(
		S_{T}
		-k
		\right)^{+} \right| \mathcal{F}_{t} \right]
	\\ &=&
	\mathbb{E} \left[ \left(
		S_{t}
		e^{(r - \delta )(T-t)}
		\exp
		\left(
		\tilde{\sigma} Y
		-
		\dfrac{1}{2}
		\tilde{\sigma}^{2}
		\right)
		-k
		\right)^{+} \right]
\end{eqnarray*}
%
%
In this way, the expectation integral can be expressed as an integral independent of the measure and filtration (an integral with respect to $Y$, which is independent of $S_{t}$; in other words, for this integral, $S_{t}$ is a constant).
Applying the calculation formula directly, and setting $S_{t} e^{(r - \delta)(T-t)}=\tilde{F}_{t}$, we get
%
%
\begin{eqnarray*}
	&&
	e^{r(T-t)}V_{t}
	\\ &=&
	\mathbb{E} \left[ \left(
		\tilde{F}_{t}
		\exp
		\left(
		\tilde{\sigma} Y
		-
		\dfrac{1}{2}
		\tilde{\sigma}^{2}
		\right)
		-k
		\right)^{+} \right]
	\\ &=&
	\tilde{F}_{t}
	\ \! \Phi
	\left(
	\dfrac{
		\log \dfrac{\tilde{F}_{t}}{k} + \dfrac{1}{2} \tilde{\sigma}^{2}
	}
	{\tilde{\sigma}}
	\right)
	-
	k
	\ \! \Phi
	\left(
	\dfrac{
		\log \dfrac{\tilde{F}_{t}}{k} - \dfrac{1}{2} \tilde{\sigma}^{2}
	}
	{\tilde{\sigma}}
	\right)
\end{eqnarray*}
%
%

Here too, the Black-Scholes formula appears, but what was the stock price becomes the forward price
$$
	\tilde{F}_{t}=S_{t} e^{(r - \delta)(T-t)}
$$
and it can be hedged by going long
$$
	\Phi
	\left(
	\dfrac{
		\log \dfrac{\tilde{F}_{t}}{k} + \dfrac{1}{2} \tilde{\sigma}^{2}
	}
	{\tilde{\sigma}}
	\right)
$$
units of stock (reinvested) and shorting
$$
	e^{-r(T-t)} k
	\ \! \Phi
	\left(
	\dfrac{
		\log \dfrac{\tilde{F}_{t}}{k} - \dfrac{1}{2} \tilde{\sigma}^{2}
	}
	{\tilde{\sigma}}
	\right)
$$
units of the bond.
\section{Index-Linked Product with a Cap and Floor}

Let the FTSE index be the process $S_{t}$.

Consider a contract that matures in 5 years, with a payoff based on 90\% of the price relative to the current price. That is, the payoff is $0.9S_{T}$.

Furthermore, set a floor at 130\% and a cap at 180\%.

Specifically, consider the payoff $X$ defined as follows.
$$
	X
	\ = \
	{\rm Min}
	\Big\{
	{\rm Max}
	( 0.9 S_{T} , \ 1.3 ) , \ 1.8
	\Big\}
$$
Since the FTSE index is composed of 100 stocks, the sum of the dividends from each stock can be approximated as a continuous dividend.

The parameter set is as follows:

FTSE drift $\mu = 7 \%$

FTSE volatility $\sigma = 15 \%$

FTSE dividend yield $\delta = 4 \%$

Pound interest rate $r = 6.5 \%$

Expressing the payoff using only Max$( \cdot , 0)$ instead of Min gives,
\footnote{I haven't been able to follow this algebraic transformation. Though it's easy to show that the equality holds.}
%
%
\begin{eqnarray*}
	X
	&=&
	1.3
	+
	0.9
	\left\{
	\left( S_{T} - \dfrac{1.3}{0.9} \right)^{+}
	-
	\left( S_{T} - \dfrac{1.8}{0.9} \right)^{+}
	\right\}
\end{eqnarray*}
%
%
In other words, it can be written solely in terms of cash and the difference between two call options on the FTSE as the underlying asset.

The forward price $F_{T}$ of $S_{T}$ is
$$
	F_{0,T}
	\ = \
	e^{(r-\delta)T}S_{0}
	\ = \
	1.133 S_0
$$
Assuming $S_0=1$, the forward price is 1.133.
Thus, substituting the numerical values into the call option price formula
$$
	e^{-rT} \left[
		F_{0,T}
		\ \! \Phi
		\left(
		\dfrac{
			\log \dfrac{F_{0,T}}{k} + \dfrac{1}{2} ( \sigma \sqrt{T} )^{2}
		}
		{\sigma \sqrt{T}}
		\right)
		-
		k
		\ \! \Phi
		\left(
		\dfrac{
			\log \dfrac{F_{0,T}}{k} - \dfrac{1}{2} ( \sigma \sqrt{T} )^{2}
		}
		{\sigma \sqrt{T}}
		\right)
		\right]
$$
yields,
%
%
\begin{eqnarray*}
	V_{0}
	&=&
	1.3 e^{-rT}
	\ + \
	0.9
	(0.0422 - 0.0067)
	\ = \
	0.9712
\end{eqnarray*}
%
%
If one forgets that the dividends of the stocks constituting the FTSE are not reflected in the index and calculates, one gets the incorrect value of 1.0183.
\footnote{Skipped this part. Does setting $\delta=0$ give this result? I'll check when I have time.}
\section{A Model for Stocks with Periodic Dividends}

So far, we have looked at cases where dividends are paid continuously.

Below, we will look at cases where dividends are paid discretely over time.

Assume that the dividend amount at predetermined times $T_{1},T_{2},\cdots$ is $\delta$ times the stock price just before the dividend payment.

The stock price itself is modeled as
%
%
\begin{eqnarray*}
	S_{t}
	&=&
	S_{0} (1-\delta)^{n[t]}
	\exp(\sigma W_{t} + \mu t)
\end{eqnarray*}
%
%
where $n[t]={\rm Max} \{ i | T_{i} \le t\}$ represents the number of times a dividend has been paid by time $t$.

The cash bond remains $B_{t}=e^{rt}$.

${}$

There are two points of departure from the Black-Scholes model.

One, as in the previous section, is that $S_{t}$ is not a tradable asset. There is a prospect of resolving this by creating a tradable asset through a transformation, similar to the previous section.

The other is that $S_{t}$ experiences a discontinuous jump at $t=T_{i}$.

At times other than $t=T_{i}$, it follows the usual stochastic differential equation
$$
	\dfrac{dS_{t}}{S_{t}}
	\ = \
	\sigma d W_{t}
	+
	\left( \mu + \dfrac{1}{2} \sigma^{2} \right)
	dt
$$
but at $t=T_{i}$, it becomes discontinuous and does not fall within the scope of stochastic processes in the usual sense.

However, it can be shown that this problem due to discontinuity can also be resolved by cleverly transforming $S_{t}$.

${}$

Let's consider a trading strategy.

Starting with one unit of stock (that is, assuming one holds one unit of stock at $t=0$), consider a strategy where each time a dividend is paid, that dividend is used to buy more stock.

In this strategy, at time $t$, one holds $(1-\delta)^{-n[t]}$ units of stock, so the value of the portfolio $\tilde{S}_{t}$ is
%
%
\begin{eqnarray*}
	\tilde{S}_{t}
	&=&
	(1-\delta)^{-n[t]} S_{t}
	\\ &=&
	S_{0}
	\exp(\sigma W_{t} + \mu t)
\end{eqnarray*}
%
%
Thus, $\tilde{S}_{t}$ represents the value of a tradable asset.

We have assumed that the ex-dividend drop factor $\delta$ and the dividend amount factor $\delta$ are equal; this is a condition for no arbitrage.
\subsection{Replication Strategy}

In this strategy, at time $t$, one holds $\tilde{\phi}_{t}$ units of $\tilde{S}_{t}$ (not the stock $S_{t}$) and $\psi_{t}$ units of $B_{t}$.
The value of this portfolio, $V_{t}$, is
%
%
\begin{eqnarray*}
	V_{t}
	&=&
	\tilde{\phi}_{t} \tilde{S}_{t} + B_{t} \psi_{t}
\end{eqnarray*}
%
%
Here, it should be noted that holding $\tilde{\phi}_{t}$ units of $\tilde{S}_{t}$ is equivalent to holding $(1-\delta)^{-n[t]} \tilde{\phi}_{t}$ units of the original stock $S_{t}$.

Letting the discounted value of $\tilde{S}_{t}$ by $B_{t}$ be represented as $\tilde{Z}_{t} = B^{-1}_{t} \tilde{S}_{t}$, the discounted value of this portfolio $(\tilde{\phi}_{t},\psi_{t})$, $E_{t}=B^{-1}_{t} V_{t}$, is
%
%
\begin{eqnarray*}
	E_{t}
	&=&
	\tilde{\phi}_{t} \tilde{Z}_{t} + \psi_{t}
\end{eqnarray*}
%
%

If $dE_{t} = \tilde{\phi}_{t} d \tilde{Z}_{t}$ holds, this portfolio is self-financing, a point to be confirmed later.

First, let's find the martingale measure $\mathbb{Q}$ for $\tilde{Z}_{t}$.
%
%
\begin{eqnarray*}
	\tilde{Z}_{t}
	&=&
	B^{-1}_{t} \tilde{S}_{t}
	\\ &=&
	S_{0} \exp \left\{ \sigma W_{t} + ( \mu - r ) t \right\}
\end{eqnarray*}
%
%
Therefore,
%
%
\begin{eqnarray*}
	\dfrac{d \tilde{Z}_{t}}{\tilde{Z}_{t}}
	&=&
	\sigma dW_{t} + \left( \mu - r + \dfrac{1}{2} \sigma^{2} \right) dt
\end{eqnarray*}
%
%
Hence, $\tilde{W}_{t}$ that satisfies
%
%
\begin{eqnarray*}
	d \tilde{W}_{t}
	&=&
	dW_{t}
	+
	\dfrac{\mu - r + \dfrac{1}{2} \sigma^{2} }{\sigma} dt
\end{eqnarray*}
%
%
is a $\mathbb{Q}$-martingale, and using the Radon-Nikodym derivative with respect to the real-world measure $\mathbb{P}$, it is defined by
%
%
\begin{eqnarray*}
	\dfrac{d \mathbb{Q} }{ d \mathbb{P} }
	&=&
	\exp \left(
	-
	\int^{T}_{0} \theta dW_{t}
	-
	\dfrac{1}{2}
	\int^{T}_{0} \theta^{2} dt
	\right)
	\\ &=&
	\exp \left(
	-
	\theta W_{T}
	-
	\dfrac{1}{2}
	\theta^{2} T
	\right)
\end{eqnarray*}
%
%
where $\theta = (\mu - r + \frac{1}{2}\sigma^2)/\sigma$.

Using this $\tilde{W}_{t}$,
%
\footnote{More simply, it's good to remember that when differentiating a stochastic process with a factor like $\sigma W_{t}$ in the exponent, you add $\frac{1}{2}\sigma^2$, and conversely, when integrating, you subtract $\frac{1}{2}\sigma^2$. (Since this comes up over and over again.)}
%
%
\begin{eqnarray*}
	\dfrac{d \tilde{Z}_{t}}{\tilde{Z}_{t}}
	&=&
	\sigma d \tilde{W}_{t}
	\\
	\tilde{Z}_{t}
	&=&
	\tilde{Z}_{0}
	\exp \left(
	\int^{t}_{0} \sigma d\tilde{W}_{s}
	-
	\dfrac{1}{2}
	\int^{t}_{0} \sigma^{2} ds
	\right)
	\\ &=&
	\tilde{Z}_{0}
	\exp \left(
	\sigma \tilde{W}_{t}
	-
	\dfrac{1}{2}
	\sigma^{2} t
	\right)
\end{eqnarray*}
%
%

To hedge a stock option $X$, we define a stochastic process by
$$
	E_{t}
	\ = \
	\mathbb{E}_{\mathbb{Q}}(B^{-1}_{T} X | \mathcal{F}_{t} )
$$
and by the martingale representation theorem, the existence of a predictable process $\tilde{\phi}_{t}$ such that
$$
	dE_{t}
	\ = \
	\tilde{\phi}_{t}
	d \tilde{Z}_{t}
$$
is guaranteed.

This predictable process $\tilde{\phi}_{t}$ is the hedge for $\tilde{Z}_{t}$, and the hedge for the bond $B_{t}$, $\psi_{t}$, is
$$
	E_{t}
	\ = \
	B^{-1}_{t} V_{t}
	\ = \
	B^{-1}_{t} ( \tilde{\phi}_{t} \tilde{S}_{t} + \psi_{t} B_{t} )
	\ = \
	\tilde{\phi}_{t} \tilde{Z}_{t} + \psi_{t}
$$
from which
$$
	\psi_{t}
	\ = \
	E_{t} - \tilde{\phi}_{t} \tilde{Z}_{t}
$$
is derived.

Also, the value of this contract $X$ at time zero (present value) is
$$
	V_{0}
	\ = \
	\mathbb{E}_{\mathbb{Q}}(B^{-1}_{T} X )
$$

Finally, let's consider the case where $X=(S_{T}-K)^{+}$, i.e., the contract $X$ is a call option with maturity $T$ and strike price $K$.

The stock price is
%
%
\begin{eqnarray*}
	S_{T}
	&=&
	S_{0}
	(1-\delta)^{n[T]}
	\exp
	\left\{
	\sigma \tilde{W}_{T}
	+
	\left(
	r
	-
	\dfrac{1}{2}
	\sigma^{2}
	\right)
	T
	\right\}
\end{eqnarray*}
%
%
In other words, it follows a log-normal distribution under the measure $\mathbb{Q}$, so from the Black-Scholes formula,
%
%
\begin{eqnarray*}
	V_{0}
	&=&
	e^{-rT} (F_{T}
	\ \! \Phi_{+}
	-
	K
	\ \! \Phi_{-})
\end{eqnarray*}
%
%
where
%
%
\begin{eqnarray*}
	\Phi_{\pm}
	&=&
	\Phi
	\left(
	\dfrac{
		\log \dfrac{F_{T}}{K} \pm \dfrac{1}{2} ( \sigma \sqrt{T} )^{2}
	}
	{\sigma \sqrt{T}}
	\right)
\end{eqnarray*}
%
%
is used as a shorthand. The forward price $F_{T}$ is
%
%
\begin{eqnarray*}
	F_{T}
	&=&
	\mathbb{E}_{\mathbb{Q}}[S_T]
	\ = \
	S_{0}
	(1-\delta)^{n[T]} e^{rT}
\end{eqnarray*}
%
%
where $n[T]$ is the total number of dividends paid up to time $T$.


\begin{thebibliography}{9}
	\bibitem{BaxterRennie}
	Financial Calculus - An Introduction to Derivative Pricing - Martin Baxter, Andrew Rennie
\end{thebibliography}
\end{document}

\end{document}

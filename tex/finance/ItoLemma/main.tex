\documentclass[uplatex,a4j,12pt,dvipdfmx]{jsarticle}
\usepackage[english]{babel}
\usepackage[letterpaper,top=2cm,bottom=2cm,left=3cm,right=3cm,marginparwidth=1.75cm]{geometry}
\usepackage{amsmath}
\usepackage{amssymb}
\usepackage{amsthm}
\usepackage{graphicx}
\usepackage{hyperref}
\usepackage{enumitem}

\title{Ito's Lemma}
\author{Masaru Okada}
\date{\today}

\begin{document}

\maketitle

The price of a stock option is a function of the underlying asset's price and time. Generally, it can be said that all derivatives are a function of the underlying's price process and time.

To understand this, one needs a solid grasp of the properties of functions of stochastic processes. Here I will summarize Ito's Lemma, a foundational concept discovered by the mathematician Kiyoshi Ito in 1951.

Consider a function $G=G(x,t)$. The change in $G$, denoted by $\Delta G$, can be expressed in terms of the changes in its variables, $\Delta x$ and $\Delta t$, as follows. (This is simply a Taylor expansion for a two-variable function.)
\begin{align}
	\Delta G
	\ = & \
	\dfrac{\partial G}{\partial x} \Delta x
	\ + \
	\dfrac{1}{2!} \dfrac{\partial^{2} G}{\partial x^{2}} \Delta x^{2}
	\ + \
	\dfrac{1}{3!} \dfrac{\partial^{3} G}{\partial x^{3}} \Delta x^{3}
	\ + \
	\cdots
	\\
	    & \ \ +
	\dfrac{\partial G}{\partial t} \Delta t
	\ + \
	\dfrac{1}{2!} \dfrac{\partial^{2} G}{\partial t^{2}} \Delta t^{2}
	\ + \
	\dfrac{1}{3!} \dfrac{\partial^{3} G}{\partial t^{3}} \Delta t^{3}
	\ + \
	\cdots
	\\
	    & \ \ +
	\dfrac{ {}_2 {\rm C}_{1} }{2!}
	\dfrac{\partial^{2} G}{\partial x \partial t } \Delta x \Delta t
	\ + \
	\dfrac{ {}_3 {\rm C}_{1} }{3!}
	\dfrac{\partial^{3} G}{\partial x^{2} \partial t } \Delta x^{2} \Delta t
	\ + \
	\dfrac{ {}_3 {\rm C}_{2} }{3!}
	\dfrac{\partial^{3} G}{\partial x \partial t^{2} } \Delta x \Delta t^{2}
	\ + \
	\cdots
	\\
	\ = & \
	\sum_{n=1}^{\infty}
	\left(
	\Delta x
	\dfrac{\partial}{\partial x}
	\ + \
	\Delta t
	\dfrac{\partial}{\partial t}
	\right)^{n}
	G
\end{align}
This equation is a general one, holding true for changes of any magnitude, not just infinitesimal ones. Building on this, let's now consider the case where the variable $x$ is a stochastic process and the changes in $x$ and $t$ are infinitesimal.

Let's assume the variable $x$ follows an Ito process, satisfying the equation:
$$
	dx
	\ = \
	a(x,t)dt
	\ + \
	b(x,t)dz
$$
Here, $dz$ represents a Wiener process. The functions $a=a(x,t)$ and $b=b(x,t)$ are the drift rate and the square root of the variance ($b$ is the volatility) of the variable $x$, respectively.

Discretizing this equation gives us:
$$
	\Delta x
	\ = \
	a \Delta t
	\ + \
	b \varepsilon \sqrt{\Delta t}
$$
Here, $\varepsilon$ is a random variable that follows a standard normal distribution. (A standard normal distribution has a mean of zero and a standard deviation of 1.0.)

Squaring both sides of the equation yields:
$$
	\Delta x^{2}
	\ = \
	b^{2} \varepsilon^{2} \Delta t
	\ + \
	2 a b \varepsilon \Delta t^{\frac{3}{2}}
	\ + \
	a^{2} \Delta t^{2}
$$
In the limit as $\Delta t \to 0$, we have:
\begin{align}
	d x^{2}
	\ = & \
	b^{2} \varepsilon^{2} dt
	\ + \
	2 a b \varepsilon dt^{\frac{3}{2}}
	\ + \
	a^{2} d t^{2}
	\\
	\ = & \
	b^{2} \varepsilon^{2} dt
\end{align}
Here, we've dropped the higher-order terms in $dt$, keeping only the lowest-order term. Even with this simplification, $dx^{2}$ remains of the first order (the lowest order) with respect to $dt$, so it's a non-negligible term. (If even the lowest-order term were ignored, the resulting equation would not be a stochastic differential equation but a trivial conclusion.)

Now, let's consider this result, $dx^{2} = b^{2} \varepsilon^{2} dt$. As mentioned before, $\varepsilon$ is a random variable following a standard normal distribution. Since a standard normal distribution has a mean (expected value) of zero and a variance of 1, we can write $E(\varepsilon)=0$. With the variance being 1, we have:
$$
	E(\varepsilon^{2})
	\ - \
	E(\varepsilon)^{2}
	\ = \
	1
$$
Since $E(\varepsilon)=0$, it follows that $E(\varepsilon^{2})=1$.
The expected value of $\varepsilon^{2} \Delta t$ is $E(\varepsilon^{2} \Delta t) = \Delta t$.
The variance of $\varepsilon^{2} \Delta t$ is:
$$
	E(\varepsilon^{4} \Delta t^{2})
	\ - \
	E(\varepsilon^{2} \Delta t)^{2}
$$
It can be shown that this is on the order of $\Delta t^{2}$. (For now, let's accept this fact without proof and move on. This is a topic for a future discussion.) As the order of $d t^{2}$ drops, the variance of $\varepsilon^{2} d t$ also drops to zero in the limit as $\Delta t \to 0$. With a variance of zero, $dx^{2}$ is no longer a stochastic variable. Furthermore, since its expected value is $E(\varepsilon^{2} d t) = d t$, the result is:
$$
	dx^{2} = b^{2} dt
$$
This is a key insight.

Using the result $dx^{2} = b^{2} dt$, we can expand $G=G(x,t)$ to the lowest order of $dx$ and $dt$:
\begin{align}
	d G
	\ =      & \
	\dfrac{\partial G}{\partial x} d x
	\ + \
	\dfrac{1}{2!} \dfrac{\partial^{2} G}{\partial x^{2}} d x^{2}
	\ + \
	\dfrac{1}{3!} \dfrac{\partial^{3} G}{\partial x^{3}} d x^{3}
	\ + \
	\cdots
	\\
	         & \ \ +
	\dfrac{\partial G}{\partial t} d t
	\ + \
	\dfrac{1}{2!} \dfrac{\partial^{2} G}{\partial t^{2}} d t^{2}
	\ + \
	\dfrac{1}{3!} \dfrac{\partial^{3} G}{\partial t^{3}} d t^{3}
	\ + \
	\cdots
	\\
	         & \ \ +
	\dfrac{ {}_2 {\rm C}_{1} }{2!}
	\dfrac{\partial^{2} G}{\partial x \partial t } d x d t
	\ + \
	\dfrac{ {}_3 {\rm C}_{1} }{3!}
	\dfrac{\partial^{3} G}{\partial x^{2} \partial t } d x^{2} d t
	\ + \
	\dfrac{ {}_3 {\rm C}_{2} }{3!}
	\dfrac{\partial^{3} G}{\partial x \partial t^{2} } d x d t^{2}
	\ + \
	\cdots
	\\
	\ \simeq & \
	\dfrac{\partial G}{\partial x} d x
	\ + \
	\dfrac{\partial G}{\partial t} d t
	\ + \
	\dfrac{1}{2!} \dfrac{\partial^{2} G}{\partial x^{2}} d x^{2}
	\\
	\ =      & \
	\dfrac{\partial G}{\partial x} d x
	\ + \
	\dfrac{\partial G}{\partial t} d t
	\ + \
	\dfrac{1}{2} \dfrac{\partial^{2} G}{\partial x^{2}} b^{2}dt
	\\
	\ =      & \
	\dfrac{\partial G}{\partial x} d x
	\ + \
	\left(
	\dfrac{\partial G}{\partial t}
	\ + \
	\dfrac{1}{2} \dfrac{\partial^{2} G}{\partial x^{2}} b^{2}
	\right)
	dt
\end{align}
This equation is known as Ito's Lemma.

Since $x$ is an Ito process
$$
	dx
	\ = \
	adt
	\ + \
	bdz
$$
we can substitute this into the equation above to change the expression from terms of $dx, dt$ to terms of $dt, dz$:
\begin{align}
	d G
	\ = & \
	\dfrac{\partial G}{\partial x} d x
	\ + \
	\left(
	\dfrac{\partial G}{\partial t}
	\ + \
	\dfrac{1}{2} \dfrac{\partial^{2} G}{\partial x^{2}} b^{2}
	\right)
	dt
	\\
	\ = & \
	\dfrac{\partial G}{\partial x} (adt+bdz)
	\ + \
	\left(
	\dfrac{\partial G}{\partial t}
	\ + \
	\dfrac{1}{2} \dfrac{\partial^{2} G}{\partial x^{2}} b^{2}
	\right)
	dt
	\\
	\ = &
	\dfrac{\partial G}{\partial x} bdz
	\ + \
	\left(
	\dfrac{\partial G}{\partial x} a
	\ + \
	\dfrac{\partial G}{\partial t}
	\ + \
	\dfrac{1}{2} \dfrac{\partial^{2} G}{\partial x^{2}} b^{2}
	\right)
	dt
\end{align}
Thus, the drift (the coefficient of $dt$) for a general derivative $G=G(x,t)$ is:
$$
	\dfrac{\partial G}{\partial x} a
	\ + \
	\dfrac{\partial G}{\partial t}
	\ + \
	\dfrac{1}{2} \dfrac{\partial^{2} G}{\partial x^{2}} b^{2}
$$
and its volatility (the coefficient of $dz$) is:
$$
	\dfrac{\partial G}{\partial x} b
$$
Now, let's consider the case where the stochastic process $x$ is the stock price $S$. So, the derivative $G=G(S,t)$ is a stock derivative with $S$ as the underlying asset. Since the stock price follows $dS = \mu S dt + \sigma S dz$, by comparing the coefficients with $dx=adt+bdz$, we see that $a= \mu S$ and $b=\sigma S$.
This leads to:
$$
	dG
	\ = \
	\dfrac{\partial G}{\partial x} \sigma S dz
	\ + \
	\left(
	\dfrac{\partial G}{\partial x} \mu S
	\ + \
	\dfrac{\partial G}{\partial t}
	\ + \
	\dfrac{1}{2} \dfrac{\partial^{2} G}{\partial x^{2}} \sigma^{2} S^{2}
	\right)
	dt
$$
This is the initial expression used in the derivation of the Black-Scholes-Merton equation.
\end{document}
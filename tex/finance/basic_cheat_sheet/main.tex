\documentclass[uplatex]{jsarticle}
\usepackage[english]{babel}
\usepackage[letterpaper,top=2cm,bottom=2cm,left=3cm,right=3cm,marginparwidth=1.75cm]{geometry}
\usepackage{amsmath, amssymb}
\usepackage{graphicx}
\usepackage{here}

\title{
Finance Cheatsheet (Basic)
}

\author{
M. O.
}

\begin{document}
\maketitle

\begin{abstract}
	A collection of sad formulas that, despite hating rote memorization, we must be able to state reflexively and unconsciously for our profession.

	I will add more as I realize they are necessary.
\end{abstract}

\ \\

\begin{enumerate}
	\item BS Model?
	\item BS Equation?
	\item Solution for European Call Option?
	\item Solution for European Call Option using Forward?
	\item Local Volatility Model?
	\item Dupire's Local Volatility?
	\item The inverse of Girsanov's Theorem?
	\item Exponential Martingale SDE and its solution?
\end{enumerate}

\newpage



\section{BS Model}

\[
	\left\{
	\begin{array}{rcl}
		\displaystyle \frac{dS}{S} & = & \mu dt + \sigma dW \\
		\displaystyle \frac{dB}{B} & = & r dt
	\end{array}
	\right.
\]

\section{BS Equation}

\[
	rf
	=
	\frac{\partial f}{\partial t}
	+
	(r-q) S \frac{\partial f}{\partial S}
	+
	\frac{1}{2} \sigma^{2} S^{2}
	\frac{\partial^{2} f}{\partial S^{2}}
\]


\section{Solution for European Call Option}

\[
	C = e^{-q(T-t)} S \Phi(d_{+}) - e^{-r(T-t)} K \Phi(d_{-})
\]

Here, $d_{\pm}$ and $\Phi$ are, respectively,

\[
	d_{\pm} = \frac{ \displaystyle \log \frac{S}{K} + (r-q \pm \frac{1}{2} \sigma^{2} ) (T-t) }{ \sigma \sqrt{ T - t } }
\]

\[
	\Phi(x) = \frac{1}{ \sqrt{2 \pi} } \int^{x}_{- \infty} e^{-y^{2}/2} dy
\]


\section{Solution for European Call Option using Forward}

Using forward not only provides a simpler expression but also a formula that can be used even if the interest rate is not constant.

\[
	C = e^{-\int^{T}_{0} r(s) ds} \Big( F \Phi(d_{+}) - K \Phi(d_{-}) \Big)
\]

Here, $F$ and $d_{\pm}$ are, respectively,

\[
	F = S e^{\int^{T}_{0} r(s) ds }
\]

\[
	d_{\pm} = \frac{ \displaystyle \log \frac{F}{K} \pm \frac{1}{2} \sigma^{2} T }{ \sigma \sqrt{ T } }
\]

\section{Dupire's Local Volatility}

\[
	\sigma_{\rm LV}^{2}(T,K)
	=
	\frac{
		\displaystyle
		\frac{\partial C}{\partial T} + r(T) K \frac{\partial C}{\partial K}
	}{
		\displaystyle
		\frac{1}{2} K^{2}
		\frac{\partial^{2} C}{\partial K^{2}}
	}
\]

This resembles the BS equation, so you can recall it by mentally transforming the following.

\[
	\Longleftrightarrow
	\ \hspace{5mm}
	\frac{\partial C}{\partial T} + r(T) K \frac{\partial C}{\partial K}
	-
	\frac{1}{2} K^{2}
	\sigma_{\rm LV}^{2}(T,K)
	\frac{\partial^{2} C}{\partial K^{2}}
	=0
\]

\[
	c.f., \
	\text{BS eqn:}
	\ \hspace{5mm}
	rf
	=
	\frac{\partial f}{\partial t}
	+
	r S \frac{\partial f}{\partial S}
	+
	\frac{1}{2} \sigma^{2} S^{2}
	\frac{\partial^{2} f}{\partial S^{2}}
\]



\section{The inverse of Girsanov's Theorem}

\[
	W^{\mathbb{Q}}_{t} = W^{\mathbb{R}}_{t} + \int^{t}_{0} \gamma_{s} ds
\]
\[
	\Longleftrightarrow
	\ \hspace{5mm}
	\frac{d \mathbb{Q}}{d \mathbb{P}}
	=
	\exp \left( - \int^{T}_{0} \gamma_{t} dW_{t} - \frac{1}{2} \int^{T}_{0} \gamma_{t}^{2} dt \right)
\]

Pay attention to the signs.


\section{Exponential Martingale SDE and its solution}

\[
	\frac{d X_{t}}{ dt}
	=
	\sigma_{t} d W_{t}
\]
\[
	\Longleftrightarrow
	\ \hspace{5mm}
	X_{t}
	=
	X_{0}
	\exp \left( \int^{t}_{0} \sigma_{s} dW_{s} - \frac{1}{2} \int^{T}_{0} \sigma_{s}^{2} dt \right)
\]

Pay attention to the signs.

\end{document}
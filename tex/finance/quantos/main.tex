\documentclass[uplatex,a4j,12pt,dvipdfmx]{jsarticle}
\usepackage{amsmath,amsthm,amssymb,bm,color,enumitem,mathrsfs,url,epic,eepic,ascmac,ulem,here}
\usepackage[letterpaper,top=2cm,bottom=2cm,left=3cm,right=3cm,marginparwidth=1.75cm]{geometry}
\usepackage[english]{babel}
\usepackage[dvipdfm]{graphicx}
\usepackage[hypertex]{hyperref}
\title{
Quanto
}
\author{Masaru Okada}

\date{\today}

\begin{document}

\maketitle

% \begin{abstract}
% 	Notes for a study group on Chapter 3 of 'Financial Calculus - An Introduction to Derivative Pricing' by Martin Baxter and Andrew Rennie.
% 	Written on November 3, 2019.
% \end{abstract}

\begin{abstract}
	This document serves as a self-contained guide to the valuation of derivatives with cross-currency exposure, known as \textbf{quantos}. It addresses the challenge of pricing a contract where the underlying asset is denominated in a foreign currency, but the payoff is settled in a domestic currency at a fixed nominal value, thereby removing direct exchange rate risk from the payoff function.
	Based on the methodology from Chapter 4 of 'Financial Calculus' by Baxter and Rennie, we develop a two-factor stochastic model. The core of the analysis is the derivation of the risk-neutral pricing measure, which allows us to compute the fair value of these path-dependent instruments. The practical application of the theory is demonstrated through the explicit pricing of a \textbf{quanto forward} and a \textbf{quanto digital option}.
\end{abstract}

\tableofcontents

\ \\

Let's consider a type of contract known as a 'quanto'.

On Friday, November 1, 2019, the closing price of the SPX was 3066.91 dollars.

For an investor holding Japanese yen, this figure of 3066.91 is in dollars.
If the exchange rate is, for example, 108.18 JPY/USD,
the value is converted to $3066.91 \times 108.18 = 331778.3238$ yen for the transaction.

However, it's also possible to conceive of a derivative that, in a sense, ignores the 'dollar' unit,
paying out $S_{t}$ yen when the SPX is at $S_{t}$ dollars.

In other words, this is a contract where the exchange rate is not applied to the payoff.
When the SPX is at 3066.91 dollars, a payment of 3066.91 yen is made.
From the investor's perspective, this simplifies the transaction to tracking a single factor of movement.

Such a contract is called a quanto.

${}$

Let's look at three examples of quantos.

$\hspace{2mm} \cdot$
1. Quanto Forward

$\hspace{2mm} \cdot$
2. Quanto Digital Option

$\hspace{2mm} \cdot$
3. Quanto Call Option

${}$

A quanto forward is a contract where, for instance, at maturity $T$,
when the SPX stock price is $S_{T}$ dollars,
a predetermined price of $K$ yen (not $K$ dollars) is paid.

A quanto digital is a contract where, for example,
if the SPX stock price at maturity $T$
exceeds a predetermined price of $K$ dollars,
1 yen (not 1 dollar) is paid.

A quanto call option is a contract where, for example,
at maturity $T$, if the SPX stock price $S_{T}$ dollars
exceeds a predetermined strike price of $K$ dollars,
a payment of $(S_{T}-K)$ yen is made.

${}$

The idea of changing the currency for payment while keeping the numerical value
of the underlying asset (like the SPX) the same is fundamentally different
from simply multiplying by an exchange rate.

An underlying asset multiplied by an exchange rate is actually traded as the same underlying asset.

But a security where only the unit of the underlying has been changed is a conceptual entity.
It is no longer the actually traded underlying asset; it is a derivative product.

\section{Quanto Model}

Here, we have two sources of randomness: the exchange rate and the stock price.

Given two $\mathbb{P}$-Brownian motions $W_{1}(t),W_{2}(t)$,
using a constant $\rho \in (-1,1)$,
$$
	W_{3}(t)
	\ = \
	\rho W_{1}(t) +
	\sqrt{1 - \rho^{2} } W_{2}(t)
$$
also becomes a $\mathbb{P}$-Brownian motion.

And $W_{1}$ and $W_{3}$ have a correlation of $\rho$.
Similarly, $W_{2}$ and $W_{3}$ have a correlation of $\sqrt{1 - \rho^{2}}$.

With this in mind, consider the following two-factor model of $\mathbb{P}$-Brownian motion
%
\footnote{In the following, the currencies from the textbook have been substituted (dollar $\to$ yen, pound $\to$ dollar).}
%
.
%
%
\begin{eqnarray*}
	S_{t}
	&=&
	S_{0}
	\exp \left( \sigma_{1} W_{1}(t) + \mu t \right)
	\\
	C_{t}
	&=&
	C_{0}
	\exp \left(
	\rho \sigma_{2} W_{1}(t) +
	\bar{\rho} \sigma_{2} W_{2}(t) + \nu t \right)
	\\
	B_{t}
	&=&
	\exp (rt)
	\\
	D_{t}
	&=&
	\exp (ut)
\end{eqnarray*}
%
%
where we have used the abbreviation $\bar{\rho} = \sqrt{1 - \rho^{2}}$.

$S_{t}$ is the price of a US stock in dollars at time $t$,

$C_{t}$ is the exchange rate (1 dollar $=C_{t}$ yen),

$B_{t}$ is the yen cash bond,

and $D_{t}$ is the dollar cash bond.

${}$

(Review)

Given a vector of $n$-dimensional random variables $X =(X_{1},X_{2},\cdots,X_{n})$,
the $n \times n$ matrix whose $(i,j)$-th element is given by the following
is called the covariance (matrix) of $X$.
$$
	\mathbb{\sigma}_{ij}
	\ = \
	\mathbb{E}
	(X_{i} - \mathbb{E} X_{i})
	(X_{j} - \mathbb{E} X_{j})
$$

${}$

Now, we want to find the covariance of the vector $(S_{t},C_{t})$.
To simplify, let's consider the logarithms of the components, $(\log S_{t},\log C_{t})$.
%
%
\begin{eqnarray*}
	\mathbb{E}_{\mathbb{P}} \log S_{t}
	&=&
	\mathbb{E}_{\mathbb{P}} (\log S_{0} + \sigma_{1} W_{1}(t) + \mu t)
	\\
	&=&
	\log S_{0} + \mu t
	\\
	\mathbb{E}_{\mathbb{P}} \log C_{t}
	&=&
	\mathbb{E}_{\mathbb{P}} (\log C_{0} + \rho \sigma_{2} W_{1}(t) + \bar{\rho} \sigma_{2} W_{2}(t) + \nu t)
	\\
	&=&
	\log C_{0} + \nu t
	\\
	\log S_{t} - \mathbb{E}_{\mathbb{P}} \log S_{t}
	&=&
	\log S_{0} + \sigma_{1} W_{1}(t) + \mu t
	-
	(\log S_{0} + \mu t)
	\\
	&=&
	\sigma_{1} W_{1}(t)
	\\
	\log C_{t} - \mathbb{E}_{\mathbb{P}} \log C_{t}
	&=&
	\rho \sigma_{2} W_{1}(t) + \bar{\rho} \sigma_{2} W_{2}(t)
\end{eqnarray*}
%
%
Therefore, the covariance $\mathbb{\sigma}$ is
%
%
\begin{eqnarray*}
	&&
	\hspace{-10mm} \mathbb{\sigma}
	\\[1mm] && \hspace{-20mm} \ =
	\left(
	\hspace{-2mm}
	\begin{array}{lr}
		\mathbb{E}[\sigma_{1} W_{1}(t)]^{2}
		 &
		\hspace{-20mm}
		\mathbb{E}[\sigma_{1} W_{1}(t) (\rho \sigma_{2} W_{1}(t) + \bar{\rho} \sigma_{2} W_{2}(t))]
		\\
		\mathbb{E}[(\rho \sigma_{2} W_{1}(t) + \bar{\rho} \sigma_{2} W_{2}(t))\sigma_{1} W_{1}(t)]
		 &
		\mathbb{E}[\rho \sigma_{2} W_{1}(t) + \bar{\rho} \sigma_{2} W_{2}(t)]^{2}
	\end{array}
	\hspace{-2mm}
	\right)
	\\[1mm] && \hspace{-20mm} \ =
	\left(
	\begin{array}{cc}
			\sigma_{1}^{2} t             & \rho \sigma_{1} \sigma_{2} t
			\\
			\rho \sigma_{1} \sigma_{2} t & \rho^{2} \sigma_{2}^{2} t + \bar{\rho}^{2} \sigma^{2}_{2} t
		\end{array}
	\right)
	\\[1mm] && \hspace{-20mm} \ =
	\left(
	\begin{array}{cc}
			\sigma_{1}^{2}             & \rho \sigma_{1} \sigma_{2}
			\\
			\rho \sigma_{1} \sigma_{2} & \sigma^{2}_{2}
		\end{array}
	\right)
	t
\end{eqnarray*}
%
%
Its correlation matrix $R$ is obtained by pre- and post-multiplying by $P$, a diagonal matrix with $(1/\sqrt{\Sigma_{ii}})$ as its $(i,i)$-th element.
%
%
\begin{eqnarray*}
	&& \hspace{-10mm}
	R \ = \ P \mathbb{\sigma} P
	\\[1mm] && \hspace{-20mm} \ =
	\left(
	\begin{array}{cc}
			\dfrac{1}{\sigma_{1} \sqrt{t}} & 0
			\\
			0                              & \dfrac{1}{\sigma_{2} \sqrt{t}}
		\end{array}
	\right)
	\left(
	\begin{array}{cc}
			\sigma_{1}^{2}             & \rho \sigma_{1} \sigma_{2}
			\\
			\rho \sigma_{1} \sigma_{2} & \sigma^{2}_{2}
		\end{array}
	\right)
	t
	\left(
	\begin{array}{cc}
			\dfrac{1}{\sigma_{1} \sqrt{t}} & 0
			\\
			0                              & \dfrac{1}{\sigma_{2} \sqrt{t}}
		\end{array}
	\right)
	\\ && \hspace{-20mm} \ =
	\left(
	\begin{array}{cc}
			\dfrac{1}{\sigma_{1}} & 0
			\\
			0                     & \dfrac{1}{\sigma_{2}}
		\end{array}
	\right)
	\left(
	\begin{array}{cc}
			\sigma_{1}^{2}             & \rho \sigma_{1} \sigma_{2}
			\\
			\rho \sigma_{1} \sigma_{2} & \sigma^{2}_{2}
		\end{array}
	\right)
	\left(
	\begin{array}{cc}
			\dfrac{1}{\sigma_{1}} & 0
			\\
			0                     & \dfrac{1}{\sigma_{2}}
		\end{array}
	\right)
	\\ && \hspace{-20mm} \ =
	\left(
	\begin{array}{cc}
			\sigma_{1}      & \rho \sigma_{2}
			\\
			\rho \sigma_{1} & \sigma_{2}
		\end{array}
	\right)
	\left(
	\begin{array}{cc}
			\dfrac{1}{\sigma_{1}} & 0
			\\
			0                     & \dfrac{1}{\sigma_{2}}
		\end{array}
	\right)
	\\ && \hspace{-20mm} \ =
	\left(
	\begin{array}{cc}
			1    & \rho
			\\
			\rho & 1
		\end{array}
	\right)
\end{eqnarray*}
%
%

From the above, we can see that:

\ \ \ \ $\cdot$ The volatility of $\log S_{t}$ is $\sigma_{1}$

\ \ \ \ $\cdot$ The volatility of $\log C_{t}$ is $\sigma_{2}$

\ \ \ \ $\cdot$ The correlation between $\log S_{t}$ and $\log C_{t}$ is $\rho$

\section{Tradable Assets}

Here, there are three assets that can be traded in yen.

\ \ \ 1. The dollar cash bond, denominated in yen: $C_{t} D_{t}$

\ \ \ 2. The US stock, denominated in yen: $C_{t} S_{t}$

\ \ \ 3. The yen cash bond: $B_{t}$

If we choose the yen cash bond as the numeraire, the discounted asset processes for 1. and 2. become as follows:
%
%
\begin{eqnarray*}
	Y_{t}
	&=&
	B^{-1}_{t} C_{t} D_{t}
	\\ &=&
	C_{0} D_{0}
	e^{-rt}
	\exp \left(
	\rho \sigma_{2} W_{1}(t) +
	\bar{\rho} \sigma_{2} W_{2}(t) + \nu t \right)
	e^{ut}
	\\ &=&
	Y_{0}
	\exp \left\{
	\rho \sigma_{2} W_{1}(t) +
	\bar{\rho} \sigma_{2} W_{2}(t) + (\nu + u - r) t \right\}
\end{eqnarray*}
%
%
%
%
\begin{eqnarray*}
	Z_{t}
	&=&
	B^{-1}_{t} C_{t} S_{t}
	\\ &=&
	C_{0} S_{0}
	e^{-rt}
	\exp \left(
	\rho \sigma_{2} W_{1}(t) +
	\bar{\rho} \sigma_{2} W_{2}(t) + \nu t \right)
	\\ && \times
	\exp \left( \sigma_{1} W_{1}(t) + \mu t \right)
	\\ &=&
	Z_{0}
	\exp \left\{
	(\sigma_{1} + \rho \sigma_{2}) W_{1}(t) +
	\bar{\rho} \sigma_{2} W_{2}(t) + ( \mu + \nu - r ) t \right\}
\end{eqnarray*}
%
%

The stochastic differential equations (SDEs) that these satisfy are
%
%
\begin{eqnarray*}
	&&
	\dfrac{dY_{t}}{Y_{t}}
	\\ &=&
	\rho \sigma_{2} dW_{1}(t) +
	\bar{\rho} \sigma_{2} dW_{2}(t) + \left( \nu + u - r \right) dt
	\\ && \
	+ \
	\dfrac{1}{2}(\rho \sigma_{2})^{2} dt
	+ \
	\dfrac{1}{2}(\bar{\rho} \sigma_{2})^{2} dt
	\\ &=&
	\rho \sigma_{2} dW_{1}(t) +
	\bar{\rho} \sigma_{2} dW_{2}(t) + \left( \nu + u + \dfrac{1}{2} \sigma_{2}^{2} - r \right) dt
\end{eqnarray*}
%
%
Even in an $n$-factor model, Ito's lemma is the same; one just needs to look at the coefficient of each Brownian motion, square it, take half, multiply by $dt$, and sum up all $n$ terms.

Similarly for $Z_{t}$,
%
%
\begin{eqnarray*}
	&&
	\dfrac{dZ_{t}}{Z_{t}}
	\\[2mm] &=&
	(\sigma_{1} + \rho \sigma_{2}) dW_{1}(t)
	+ \bar{\rho} \sigma_{2} dW_{2}(t)
	+ ( \mu + \nu - r ) dt
	\\ && \
	+ \
	\dfrac{1}{2} (\sigma_{1} + \rho \sigma_{2})^{2} dt
	+ \
	\dfrac{1}{2} (\bar{\rho} \sigma_{2})^{2} dt
	\\[2mm] &=&
	(\sigma_{1} + \rho \sigma_{2}) dW_{1}(t)
	+ \bar{\rho} \sigma_{2} dW_{2}(t)
	\\ && \ +
	\left(
	\mu + \nu +
	\dfrac{1}{2} \sigma_{1}^{2} + \rho \sigma_{1} \sigma_{2} + \dfrac{1}{2} \rho^{2} \sigma_{2}^{2}
	+ \
	\dfrac{1}{2} \bar{\rho}^{2} \sigma_{2}^{2}
	- r
	\right) dt
	\\[2mm] &=&
	(\sigma_{1} + \rho \sigma_{2}) dW_{1}(t)
	+ \bar{\rho} \sigma_{2} dW_{2}(t)
	\\ && \ +
	\left(
	\mu + \nu +
	\dfrac{1}{2} \sigma_{1}^{2} + \rho \sigma_{1} \sigma_{2} + \dfrac{1}{2} \sigma_{2}^{2}
	- r
	\right) dt
\end{eqnarray*}
%
%

${}$

The SDEs to be satisfied by the tradable assets $Y_{t},Z_{t}$ have been found.

Next is to find a measure that makes $Y_{t},Z_{t}$ martingales.

This is equivalent to finding the market price of risk, which represents the drift transformation.

Since we are considering a 2-factor problem, the market price of risk will be a two-dimensional vector, not a scalar.

In matrix notation, the SDE is
%
%
\begin{eqnarray*}
	&&
	\left(
	\begin{array}{c}
		d Y_{t} / Y_{t}
		\\[2mm]
		d Z_{t} / Z_{t}
	\end{array}
	\right)
	\\ &=&
	\left(
	\begin{array}{ccc}
		\rho \sigma_{2}              & \bar{\rho} \sigma_{2} & \nu + u + \dfrac{1}{2} \sigma_{2}^{2} - r
		\\
		\sigma_{1} + \rho \sigma_{2} & \bar{\rho} \sigma_{2} & \mu + \nu + \dfrac{1}{2} \sigma_{1}^{2} + \rho \sigma_{1} \sigma_{2} + \dfrac{1}{2} \sigma_{2}^{2} - r
	\end{array}
	\right)
	\\ && \ \times \
	\left(
	\begin{array}{c}
		dW_{1}(t)
		\\
		dW_{2}(t)
		\\
		dt
	\end{array}
	\right)
	\\ &=&
	\left(
	\begin{array}{cc}
		\rho \sigma_{2}              & \bar{\rho} \sigma_{2}
		\\
		\sigma_{1} + \rho \sigma_{2} & \bar{\rho} \sigma_{2}
	\end{array}
	\right)
	\left(
	\begin{array}{c}
		dW_{1}(t)
		\\
		dW_{2}(t)
	\end{array}
	\right)
	\\ && \ + \
	\left(
	\begin{array}{c}
		\nu + u + \dfrac{1}{2} \sigma_{2}^{2} - r
		\\
		\mu + \nu + \dfrac{1}{2} \sigma_{1}^{2} + \rho \sigma_{1} \sigma_{2} + \dfrac{1}{2} \sigma_{2}^{2} - r
	\end{array}
	\right)
	dt
\end{eqnarray*}
%
%
This separates the terms for the Brownian differential and the time differential.

The coefficient matrix of the Brownian differential terms is called the volatility matrix, denoted by ${\bm \Sigma}$.
%
%
\begin{eqnarray*}
	{\bm \Sigma}
	&=&
	\left(
	\begin{array}{cc}
			\rho \sigma_{2}              & \bar{\rho} \sigma_{2}
			\\
			\sigma_{1} + \rho \sigma_{2} & \bar{\rho} \sigma_{2}
		\end{array}
	\right)
\end{eqnarray*}
%
%
The drift vector ${\bm \mu}$ is defined as follows.
%
%
\begin{eqnarray*}
	{\bm \mu}
	&=&
	\left(
	\begin{array}{c}
			\nu + u + \dfrac{1}{2} \sigma_{2}^{2}
			\\
			\mu + \nu + \dfrac{1}{2} \sigma_{1}^{2} + \rho \sigma_{1} \sigma_{2} + \dfrac{1}{2} \sigma_{2}^{2}
		\end{array}
	\right)
\end{eqnarray*}
%
%
Using the volatility matrix ${\bm \Sigma}$ and the drift vector ${\bm \mu}$, the SDE can be written transparently as follows.
%
%
\begin{eqnarray*}
	\left(
	\begin{array}{c}
			d Y_{t} / Y_{t}
			\\
			d Z_{t} / Z_{t}
		\end{array}
	\right)
	&=&
	{\bm \Sigma}
	\left(
	\begin{array}{c}
			dW_{1}(t)
			\\
			dW_{2}(t)
		\end{array}
	\right)
	+
	( {\bm \mu} - r {\bm 1} ) dt
\end{eqnarray*}
%
%
To cancel this drift term, using a Brownian motion $(\tilde{W}_{1}(t),\tilde{W}_{2}(t))$ under some measure $\mathbb{Q}$,
%
%
\begin{eqnarray*}
	\left(
	\begin{array}{c}
			d Y_{t} / Y_{t}
			\\
			d Z_{t} / Z_{t}
		\end{array}
	\right)
	&=&
	{\bm \Sigma}
	\left(
	\begin{array}{c}
			d \tilde{W}_{1}(t)
			\\
			d \tilde{W}_{2}(t)
		\end{array}
	\right)
\end{eqnarray*}
%
%
is what we want, and comparing the right-hand sides gives
%
%
\begin{eqnarray*}
	{\bm \Sigma}
	\left(
	\begin{array}{c}
			d \tilde{W}_{1}(t)
			\\
			d \tilde{W}_{2}(t)
		\end{array}
	\right)
	&=&
	{\bm \Sigma}
	\left(
	\begin{array}{c}
			dW_{1}(t)
			\\
			dW_{2}(t)
		\end{array}
	\right)
	+
	( {\bm \mu} - r {\bm 1} ) dt
\end{eqnarray*}
%
%
If the volatility matrix ${\bm \Sigma}$ is invertible,
%
%
\begin{eqnarray*}
	\left(
	\begin{array}{c}
			d \tilde{W}_{1}(t)
			\\
			d \tilde{W}_{2}(t)
		\end{array}
	\right)
	&=&
	\left(
	\begin{array}{c}
			dW_{1}(t)
			\\
			dW_{2}(t)
		\end{array}
	\right)
	+
	{\bm \Sigma}^{-1}
	( {\bm \mu} - r {\bm 1} ) dt
	\\ &=&
	\left(
	\begin{array}{c}
			dW_{1}(t)
			\\
			dW_{2}(t)
		\end{array}
	\right)
	+
	{\bm \gamma}dt
\end{eqnarray*}
%
%
where ${\bm \gamma}$ is the market price of risk corresponding to $(W_{1}(t),W_{2}(t))$, with ${\bm \gamma}^{T} = (\gamma_{1}(t),\gamma_{2}(t))$.
%
%
\begin{eqnarray*}
	{\bm \gamma}
	&=&
	{\bm \Sigma}^{-1}
	( {\bm \mu} - r {\bm 1} )
\end{eqnarray*}
%
%
To describe the SDE using the $\mathbb{Q}$-Brownian motion $(\tilde{W}_{1}(t),\tilde{W}_{2}(t))$, we just need to calculate the market price of risk.

First, the inverse of the volatility matrix is,
%
%
\begin{eqnarray*}
	{\bm \Sigma}^{-1}
	&=&
	\left(
	\begin{array}{cc}
		\rho \sigma_{2}              & \bar{\rho} \sigma_{2}
		\\
		\sigma_{1} + \rho \sigma_{2} & \bar{\rho} \sigma_{2}
	\end{array}
	\right)^{-1}
	\\ &=&
	\dfrac{1}{
		\rho \sigma_{2} \bar{\rho} \sigma_{2} - \bar{\rho} \sigma_{2}(\sigma_{1} + \rho \sigma_{2}) }
	\left(
	\!\!
	\begin{array}{cc}
			\bar{\rho} \sigma_{2}          & - \bar{\rho} \sigma_{2}
			\\
			- \sigma_{1} - \rho \sigma_{2} & \rho \sigma_{2}
		\end{array}
	\!\!
	\right)
	\\ &=&
	\dfrac{1}{
		- \bar{\rho} \sigma_{1} \sigma_{2} }
	\left(
	\!\!
	\begin{array}{cc}
			\bar{\rho} \sigma_{2}          & - \bar{\rho} \sigma_{2}
			\\
			- \sigma_{1} - \rho \sigma_{2} & \rho \sigma_{2}
		\end{array}
	\!\!
	\right)
\end{eqnarray*}
%
%
and so,

%
%
\begin{eqnarray*}
	{\bm \gamma}
	&=&
	{\bm \Sigma}^{-1}
	( {\bm \mu} - r {\bm 1} )
	\\ &=&
	\dfrac{1}{ - \bar{\rho} \sigma_{1} \sigma_{2} }
	\left(
	\!\!
	\begin{array}{cc}
			\bar{\rho} \sigma_{2}          & - \bar{\rho} \sigma_{2}
			\\
			- \sigma_{1} - \rho \sigma_{2} & \rho \sigma_{2}
		\end{array}
	\!\!
	\right)
	\\ && \ \times
	\left(
	\begin{array}{c}
			\nu + u + \dfrac{1}{2} \sigma_{2}^{2} - r
			\\
			\mu + \nu + \dfrac{1}{2} \sigma_{1}^{2} + \rho \sigma_{1} \sigma_{2} + \dfrac{1}{2} \sigma_{2}^{2} - r
		\end{array}
	\right)
\end{eqnarray*}
%
%
The expression becomes long, so let's look at each component separately.
\footnote{
	After a long calculation, the result for $\gamma_{2}$ was exactly -1 times the one in the textbook (the sign was reversed, but all subsequent terms were the same).
}
%
%
\begin{eqnarray*}
	\gamma_{1}
	&=&
	\dfrac{
		\begin{array}{l}
			\bar{\rho} \sigma_{2}
			( \nu + u + \dfrac{1}{2} \sigma_{2}^{2} - r )
			\\ \hspace{7mm}
			- \bar{\rho} \sigma_{2}
			( \mu + \nu + \dfrac{1}{2} \sigma_{1}^{2} + \rho \sigma_{1} \sigma_{2} + \dfrac{1}{2} \sigma_{2}^{2} - r )
		\end{array}
	}
	{ - \bar{\rho} \sigma_{1} \sigma_{2} }
	\\ &=&
	\dfrac{
		u
		-
		\mu - \dfrac{1}{2} \sigma_{1}^{2} - \rho \sigma_{1} \sigma_{2}
	}
	{ - \sigma_{1} }
\end{eqnarray*}
%
%

%
%
\begin{eqnarray*}
	\gamma_{2}
	&=&
	\dfrac{
		\begin{array}{l}
			( - \sigma_{1} - \rho \sigma_{2} )
			( \nu + u + \dfrac{1}{2} \sigma_{2}^{2} - r )
			\\ \hspace{7mm} + \
			\rho \sigma_{2}
			( \mu + \nu + \dfrac{1}{2} \sigma_{1}^{2} + \rho \sigma_{1} \sigma_{2} + \dfrac{1}{2} \sigma_{2}^{2} - r )
		\end{array}
	}
	{ - \bar{\rho} \sigma_{1} \sigma_{2} }
	\\[2mm] &=&
	\dfrac{
		\begin{array}{l}
			- \sigma_{1}
			(  \nu + u + \dfrac{1}{2} \sigma_{2}^{2} - r )
			\\[2mm] \hspace{5mm} + \
			\rho \sigma_{2}
			(\mu + \dfrac{1}{2} \sigma_{1}^{2} + \rho \sigma_{1} \sigma_{2} - u)
		\end{array}
	}
	{ - \bar{\rho} \sigma_{1} \sigma_{2} }
	\\ &=&
	-
	\dfrac{
		\nu + u + \dfrac{1}{2} \sigma_{2}^{2} - r
	}
	{ - \bar{\rho} \sigma_{2} }
	+
	\dfrac{
		\mu + \dfrac{1}{2} \sigma_{1}^{2} + \rho \sigma_{1} \sigma_{2} - u
	}
	{ - \bar{\rho} \sigma_{1} }
	\\ &=&
	\dfrac{
		\nu + u + \dfrac{1}{2} \sigma_{2}^{2} - r
	}
	{ \bar{\rho} \sigma_{2} }
	-
	\dfrac{
		1
	}
	{ \bar{\rho} }
	\gamma_{1}
	\\ &=&
	\dfrac{
		\nu + u + \dfrac{1}{2} \sigma_{2}^{2} - r - \rho \sigma_{2} \gamma_{1}
	}
	{ \bar{\rho} \sigma_{2} }
\end{eqnarray*}
%
%

Using the market price of risk just found, the SDE can be rewritten as follows.
\footnote{This also serves as the solution to Exercise 4.2.}
%
%
\begin{eqnarray*}
	\left(
	\begin{array}{c}
			d Y_{t} / Y_{t}
			\\
			d Z_{t} / Z_{t}
		\end{array}
	\right)
	&=&
	{\bm \Sigma}
	\left(
	\begin{array}{c}
			dW_{1}(t)
			\\
			dW_{2}(t)
		\end{array}
	\right)
	+
	( {\bm \mu} - r {\bm 1} ) dt
	\\ &=&
	{\bm \Sigma}
	\left\{
	\left(
	\begin{array}{c}
			dW_{1}(t)
			\\
			dW_{2}(t)
		\end{array}
	\right)
	+
	{\bm \Sigma}^{-1}
	( {\bm \mu} - r {\bm 1} ) dt
	\right\}
	\\ &=&
	{\bm \Sigma}
	\left\{
	\left(
	\begin{array}{c}
			dW_{1}(t)
			\\
			dW_{2}(t)
		\end{array}
	\right)
	-
	{\bm \gamma} dt
	\right\}
	\\ &=&
	{\bm \Sigma}
	\left(
	\begin{array}{c}
			d \tilde{W}_{1}(t)
			\\
			d \tilde{W}_{2}(t)
		\end{array}
	\right)
\end{eqnarray*}
%
%

In component form,
%
%
\begin{eqnarray*}
	&&
	\left(
	\begin{array}{c}
		d Y_{t} / Y_{t}
		\\
		d Z_{t} / Z_{t}
	\end{array}
	\right)
	\\ &=&
	\left(
	\begin{array}{cc}
		\rho \sigma_{2}              & \bar{\rho} \sigma_{2}
		\\
		\sigma_{1} + \rho \sigma_{2} & \bar{\rho} \sigma_{2}
	\end{array}
	\right)
	\left(
	\begin{array}{c}
		d \tilde{W}_{1}(t)
		\\
		d \tilde{W}_{2}(t)
	\end{array}
	\right)
	\\ &=&
	\left(
	\begin{array}{l}
		\rho \sigma_{2} d \tilde{W}_{1}(t) \ + \ \bar{\rho} \sigma_{2} d \tilde{W}_{2}(t)
		\\
		( \sigma_{1} + \rho \sigma_{2} ) d \tilde{W}_{1}(t) \ + \ \bar{\rho} \sigma_{2} d \tilde{W}_{2}(t)
	\end{array}
	\right)
\end{eqnarray*}
%
%
This is in the form of a system of differential equations, so trying to solve it directly seems complicated, possibly requiring diagonalization of the volatility matrix.

However, using the concept of the market price of risk, we only need to find the respective $(\mu,\nu)$ that make ${\bm \gamma} = {\bm 0}$ and substitute them into the drift of the original asset processes.

From $\gamma_{1} = 0$,
$$
	u
	-
	\mu - \dfrac{1}{2} \sigma_{1}^{2} - \rho \sigma_{1} \sigma_{2}
	\ = \ 0
$$
which means
$$
	\mu
	\ = \
	u - \dfrac{1}{2} \sigma_{1}^{2} - \rho \sigma_{1} \sigma_{2}
$$

From $\gamma_{2} = 0$,
$$
	\nu + u + \dfrac{1}{2} \sigma_{2}^{2} - r - \rho \sigma_{2} \gamma_{1}
	\ = \
	0
$$
which means,
$$
	\nu
	\ = \
	r - u - \dfrac{1}{2} \sigma_{2}^{2}
$$

The original asset processes were expressed using $\mathbb{P}$-Brownian motion as
%
%
\begin{eqnarray*}
	S_{t}
	&=&
	S_{0}
	\exp \left( \sigma_{1} W_{1}(t) + \mu t \right)
	\\
	C_{t}
	&=&
	C_{0}
	\exp \left(
	\rho \sigma_{2} W_{1}(t) +
	\bar{\rho} \sigma_{2} W_{2}(t) + \nu t \right)
\end{eqnarray*}
%
%
so describing them with $\mathbb{Q}$-Brownian motion, substituting the ${\bm \mu}$ we just found, yields:
%
%
\begin{eqnarray*}
	S_{t}
	&=&
	S_{0}
	\exp \left( \sigma_{1} \tilde{W}_{1}(t) +
	\left(
		u - \dfrac{1}{2} \sigma_{1}^{2} - \rho \sigma_{1} \sigma_{2}
		\right)
	t \right)
	\\
	C_{t}
	&=&
	C_{0}
	\exp \left(
	\rho \sigma_{2} \tilde{W}_{1}(t) +
	\bar{\rho} \sigma_{2} \tilde{W}_{2}(t) +
	\left(
		r - u - \dfrac{1}{2} \sigma_{2}^{2}
		\right)
	t \right)
\end{eqnarray*}
%
%
となる。

${}$

For $C_{t}$, by using the $\mathbb{Q}$-Brownian motion
$$
	\tilde{W}_{3}(t)
	\ = \
	\rho \tilde{W}_{1}(t) +
	\bar{\rho} \tilde{W}_{2}(t)
$$
it can be written as
%
%
\begin{eqnarray*}
	C_{t}
	&=&
	e^{(r-u)t}
	C_{0}
	\exp \left(
	\sigma_{2} \tilde{W}_{3}(t) -
	\dfrac{1}{2} \sigma_{2}^{2}
	t \right)
\end{eqnarray*}
%
%
Thus, if we choose the numeraire $e^{(r-u)t} = B_{t} D^{-1}_{t}$, it becomes a $\mathbb{Q}$-martingale and hence tradable.

On the other hand, for $S_t$,
%
%
\begin{eqnarray*}
	S_{t}
	&=&
	e^{ut}
	S_{0}
	\exp \left( \sigma_{1} \tilde{W}_{1}(t) -
	\dfrac{1}{2} \sigma_{1}^{2}
	t \right)
	e^{- \rho \sigma_{1} \sigma_{2} t}
\end{eqnarray*}
%
%
it contains a factor like $e^{- \rho \sigma_{1} \sigma_{2} t}$, so even if we choose the dollar cash bond $e^{ut} = D_{t}$ as the numeraire, it does not become tradable.

\section{Quanto Forward}

Now that we have found the measure $\mathbb{Q}$ that makes tradable assets into martingales, we can proceed to price quanto derivatives.

Regarding the equation for $S_t$ described using $\mathbb{Q}$-Brownian motion just above, using the forward price at maturity $T$, $F = S_{0} e^{uT}$,
%
%
\begin{eqnarray*}
	S_{T}
	&=&
	F e^{-uT}
	\exp \left( \sigma_{1} \tilde{W}_{1}(T) +
	\left(
		u - \dfrac{1}{2} \sigma_{1}^{2} - \rho \sigma_{1} \sigma_{2}
		\right)
	T \right)
\end{eqnarray*}
%
%
The present value of a forward contract with delivery price = $K$ (yen) can be calculated as follows.
%
%
\begin{eqnarray*}
	V_{0}
	&=&
	e^{-rT}
	\mathbb{E}_{\mathbb{Q}}(S_{T}-K)
	\\ &=&
	e^{-rT}
	\mathbb{E}_{\mathbb{Q}}
	\left[
		S_{0}
		\exp \left( \sigma_{1} \tilde{W}_{1}(T) +
		\left(
			u - \dfrac{1}{2} \sigma_{1}^{2} - \rho \sigma_{1} \sigma_{2}
			\right)
		T \right)
		-
		K
		\right]
	\\ &=&
	S_{0}
	e^{-rT}
	e^{ (u - \frac{1}{2} \sigma_{1}^{2} - \rho \sigma_{1} \sigma_{2}) T}
	\mathbb{E}_{\mathbb{Q}}
		[e^{\sigma_{1} \tilde{W}_{1}(T)}]
	-
	Ke^{-rT}
\end{eqnarray*}
%
%
Here, the expectation value $\mathbb{E}_{\mathbb{Q}}[e^{\sigma_{1} \tilde{W}_{1}(T)}]$ can be calculated, since $\dfrac{\tilde{W}_{1}(T)}{\sqrt{T}}$ follows a standard normal random variable $Z$ under the measure $\mathbb{Q}$.
%
%
\begin{eqnarray*}
	\mathbb{E}_{\mathbb{Q}}
		[e^{\sigma_{1} \tilde{W}_{1}(T)}]
	&=&
	\mathbb{E}_{\mathbb{Q}}
	\left[ \exp \left( \sigma_{1} \sqrt{T} \dfrac{\tilde{W}_{1}(T)}{\sqrt{T}} \right) \right]
	\\ &=&
	\mathbb{E}
	\left[ \exp \left( \sigma_{1} \sqrt{T} Z \right) \right]
	\\ &=&
	\dfrac{1}{ \sqrt{2 \pi} }
	\int^{\infty}_{-\infty}
	\exp \left( \sigma_{1} \sqrt{T} z \right)
	\
	e^{-\frac{1}{2}z^{2}}
	dz
	\\ &=&
	\exp \left( \dfrac{1}{2} (\sigma_{1} \sqrt{T})^{2} \right)
\end{eqnarray*}
%
%
This can be integrated as follows
\footnote{
	$$
		\dfrac{1}{ \sqrt{2 \pi} }
		\int^{\infty}_{-\infty}
		e^{-\frac{1}{2} x^{2} + ax}
		dx
		\ = \
		e^{\frac{1}{2} a^{2}}
	$$
}.
To summarize,
%
%
\begin{eqnarray*}
	&& V_{0}
	\\ &=&
	e^{-rT}
	\mathbb{E}_{\mathbb{Q}}(S_{T}-K)
	\\ &=&
	S_{0}
	e^{-rT}
	e^{(u - \frac{1}{2} \sigma_{1}^{2} - \rho \sigma_{1} \sigma_{2})T}
	\mathbb{E}_{\mathbb{Q}}
		[e^{\sigma_{1} \tilde{W}_{1}(T)}]
	-
	Ke^{-rT}
	\\ &=&
	S_{0}
	e^{-rT}
	e^{(u - \frac{1}{2} \sigma_{1}^{2} - \rho \sigma_{1} \sigma_{2})T}
	\exp \left(\dfrac{1}{2} \sigma_{1}^{2} T \right)
	-
	Ke^{-rT}
	\\ &=&
	S_{0} e^{uT}
	e^{-rT}
	e^{- \rho \sigma_{1} \sigma_{2} T}
	-
	Ke^{-rT}
	\\ &=&
	F
	e^{-rT}
	e^{- \rho \sigma_{1} \sigma_{2} T}
	-
	Ke^{-rT}
\end{eqnarray*}
%
%

The delivery price $K$ is set so that the present value becomes zero (so that neither party to the trade has an initial profit or loss).

Therefore, the value of $K$ when $V_{0}=0$ is,
%
%
\begin{eqnarray*}
	0 &=& V_{0}
	\ = \
	F
	e^{-rT}
	e^{- \rho \sigma_{1} \sigma_{2} T}
	-
	Ke^{-rT}
	\\
	\Longleftrightarrow
	\
	K
	&=&
	F
	e^{- \rho \sigma_{1} \sigma_{2} T}
\end{eqnarray*}
%
%

Since the values of $\sigma_{1}$ and $\sigma_{2}$ are both positive, only when the stock price and the exchange rate have a negative correlation ($\rho < 0$) will the price of the quanto forward be higher than the regular forward price $F$, by a factor of $e^{- \rho \sigma_{1} \sigma_{2} T}$.


\section{Quanto Digital Option}

A quanto digital option is a contract where, for instance, if the stock price $S_{T}$ (in dollars) at maturity $T$ exceeds a predetermined price of $K$ dollars, 1 yen (not 1 dollar) is paid.

The contract is
$$
	X
	\ = \
	1_{ \{ S_{T}>K \} }
$$
and its present value $V_{0}$ is
%
%
\begin{eqnarray*}
	\hspace{-10mm}
	V_{0}
	& = &
	e^{-rT}
	\mathbb{E}_{\mathbb{Q}}
		[X]
	\\ & = &
	e^{-rT}
	\mathbb{E}_{\mathbb{Q}}
		[1_{ \{ S_{T}>K \} }]
	\\ & = &
	e^{-rT}
	\mathbb{Q}
	\{ S_{T}>K \}
\end{eqnarray*}
%
%
Here, $\mathbb{Q}\{B\}$ is the probability of condition $B$ being met under the measure $\mathbb{Q}$.

Recall that $S_{T}$ could be written using $\mathbb{Q}$-Brownian motion as follows.
%
%
\begin{eqnarray*}
	S_{T}
	&=&
	S_{0}
	\exp \left( \sigma_{1} \tilde{W}_{1}(T) +
	\left(
		u - \dfrac{1}{2} \sigma_{1}^{2} - \rho \sigma_{1} \sigma_{2}
		\right)
	T \right)
	\\ &=&
	S_{0}e^{uT}
	\exp \left( \sigma_{1} \sqrt{T} \dfrac{ \tilde{W}_{1}(T) }{ \sqrt{T} } -
	\dfrac{1}{2} \sigma_{1}^{2} T -
	\rho \sigma_{1} \sigma_{2} T \right)
	\\ &=&
	F
	\exp \left( \sigma_{1} \sqrt{T} Z -
	\dfrac{1}{2} \sigma_{1}^{2} T  -
	\rho \sigma_{1} \sigma_{2} T \right)
\end{eqnarray*}
%
%
where $F$ is the forward price at maturity $T$, $F = S_{0} e^{uT}$. Also, $Z$ is a standard normal random variable under $\mathbb{Q}$, following $N(0,1)$.

Furthermore, for notational simplicity, if we set
$$
	F_{Q} \ = \ F e^{- \rho \sigma_{1} \sigma_{2} T}
$$
it takes on a familiar form.
$$
	S_{T}
	\ = \
	F_{Q}
	\exp \left( \sigma_{1} \sqrt{T} Z -
	\dfrac{1}{2} \sigma_{1}^{2} T
	\right)
$$
Transforming the condition $S_{T}>K$,
%
%
\begin{eqnarray*}
	S_{T}
	&>&
	K
	\\
	F_{Q}
	\exp \left( \sigma_{1} \sqrt{T} Z -
	\dfrac{1}{2} \sigma_{1}^{2} T
	\right)
	&>&
	K
	\\
	\exp \left( \sigma_{1} \sqrt{T} Z -
	\dfrac{1}{2} \sigma_{1}^{2} T
	\right)
	&>&
	\dfrac{K}{F_{Q}}
	\\
	\sigma_{1} \sqrt{T} Z -
	\dfrac{1}{2} \sigma_{1}^{2} T
	&>&
	\log
	\dfrac{K}{F_{Q}}
	\\
	Z
	&>&
	\dfrac{
		\dfrac{1}{2} \sigma_{1}^{2} T +
		\log
		\dfrac{K}{F_{Q}}
	}{\sigma_{1} \sqrt{T}}
	\ = \
	z_{0}
\end{eqnarray*}
%
%
The right-hand side is cumbersome, so for now, let's set it to $z_{0}$.

Based on the above, continuing the calculation of the present value,
%
%
\begin{eqnarray*}
	V_{0}
	&=&
	e^{-rT}
	\mathbb{Q}
	\{ S_{T} > K \}
	\\ &=&
	e^{-rT}
	\mathbb{Q}
	\{ Z > z_{0} \}
	\\ &=&
	e^{-rT}
	\dfrac{1}{ \sqrt{2 \pi} }
	\int^{\infty}_{z_{0}}
	e^{
			- \frac{1}{2} z^{2}
		} dz
	\\ &=&
	e^{-rT}
	\left( 1 - \Phi(z_0) \right)
	\\ &=&
	e^{-rT}
	\Phi(-z_{0})
	\\ &=&
	e^{-rT}
	\Phi
	\left(
	- \dfrac{
		\dfrac{1}{2} \sigma_{1}^{2} T +
		\log
		\dfrac{K}{F_{Q}}
	}{\sigma_{1} \sqrt{T}}
	\right)
	\\ &=&
	e^{-rT}
	\Phi
	\left(
	\dfrac{
		\log
		\dfrac{F_{Q}}{K}
		-
		\dfrac{1}{2} \sigma_{1}^{2} T
	}{\sigma_{1} \sqrt{T}}
	\right)
	\\ &=&
	e^{-rT}
	\Phi
	\left(
	\dfrac{
		\log
		\dfrac{F}{K}
		-
		\dfrac{1}{2} \sigma_{1}^{2} T
		-
		\rho \sigma_{1} \sigma_{2} T
	}{\sigma_{1} \sqrt{T}}
	\right)
\end{eqnarray*}
%
%


\begin{thebibliography}{9}
	\bibitem{BaxterRennie}
	Financial Calculus - An Introduction to Derivative Pricing - Martin Baxter, Andrew Rennie
\end{thebibliography}

\end{document}
\documentclass[uplatex,a4j,12pt,dvipdfmx]{jsarticle}
\usepackage[english]{babel}
\usepackage[letterpaper,top=2cm,bottom=2cm,left=3cm,right=3cm,marginparwidth=1.75cm]{geometry}
\usepackage{amsmath}
\usepackage{amssymb}
\usepackage{amsthm}
\usepackage{graphicx}
\usepackage{hyperref}
\usepackage{enumitem}

\title{
二項ツリーモデルにおけるリスク中立確率
}
\author{Masaru Okada}
\date{\today}

\begin{document}

\maketitle

\tableofcontents

\newpage

\section{はじめに}
本書の最終目標は、裁定取引の限界を探求すること。そのために、実際の金融市場の現実的なモデルとして機能し、なおかつ価格構築の技術を生み出すことができるような数学的な枠組みを発展させていく。

\section{二項枝モデル}
我々の目的に最低限必要なものは、次の二つである。
\begin{itemize}
	\item お金の時間的価値を表すもの
	\item 株式を表すランダムな要素
\end{itemize}
この二つがなければ、いかなるモデルも実際の金融市場との関連性を考えることはできない。まずは、債券と株式から構成される、可能な限りシンプルなモデルを検討することから始めよう。

\subsection{株式}
時刻$t=0$で始まり$t=\delta t$で終わる、非常に短い期間を考えよう。株式の価格を表す変数は、予測不可能なランダムな要素を持っているのが理想的だ。この短い期間$\delta t$では、株価は上昇するか、もしくは下降するかの2通りの変化しか起こらないと仮定する。

このランダムさに何らかの構造を持たせることを考える。つまり、価格の上昇と下降に確率を割り当てるのだ。短い期間$\delta t$での上昇確率を$p$、下降確率を$(1-p)$とする。時刻$t=0$の初期価格を$s_1$としよう。この価格は、我々が量に制限なく売買できる価格を表している。そこで、$\delta t$の間、その株式を保有することができる。株式を保有している間は何も起こらない。言い換えれば、正負いずれの量を保有してもコストはかからないとしよう。しかし、$\delta t$の時間が経過した後、株式は新しい価格を持つと仮定する。$\delta t$の後、下降すれば価格は$s_2$に、上昇すれば$s_3$になるとする。

\subsection{債券}
お金の時間的価値を表す何か、すなわち、キャッシュボンドも考慮に入れる必要がある。株式の場合と同様に、時刻$t=0$で始まり$t=\delta t$で終わる非常に短い期間を考える。$\delta t$が経過した後、時刻$t=0$における1ドルが$\exp(r\delta t)$になるような連続複利$r$が存在すると仮定しよう。この金利で任意の金額を貸し借りできるとする。これを表現する変数として、時刻$t=0$で価格$B_0$で売買できるキャッシュボンドBを導入する。つまり、このキャッシュボンドは、$\delta t$が経過した後、価格が$B_0 \exp(r\delta t)$になるのだ。

これらたった2つの道具が、我々の金融の世界のすべてだ。単純化されているとはいえ、この世界は投資家にとっては相変わらず不確実性をもたらす。市場参加者にとって、$\delta t$が経過した後の株価が取りうる2つの値のうち、どちらか一方が都合が良い。(例えば、株を空売りしている場合は、価格が下落する方が都合が良い。)都合が良いか悪いかは、ランダムな結果に左右される。投資家が将来の株価に基づいた支払いを要求する場合、$\delta t$における株価の取りうる2つの可能性である$s_2$と$s_3$に対して、支払額は関数$f$を用いてそれぞれ$f(2)$と$f(3)$に対応させるような関数で表現できる。例えば、受渡価格が$k$と契約されたフォワード契約は、$f(2)=s_2-k$および$f(3)=s_3-k$で表される。

\section{無リスク構築}
適切な戦略によって、具体的にどのような関数$f$が提供されるかを考えてみよう。第1章で述べたように、フォワード契約では、時刻$t=0$で$s_1$の価格で株式を購入し、その購入代金を賄うために同額$s_1$のキャッシュボンドを売却し、満期$\delta t$までそのポジションを保有する。したがって、その代金としては$s_1 \exp(r\delta t)$を請求すればよい。このことから、フォワード契約の価格は\[k = s_1 \exp(r\delta t)\]となる。これが裁定を通した価格決定である。

より複雑な$f$について、価格構築戦略を見つけることは一見できないように思われる。非常に短い時間$\delta t$の後、株価はランダムに2つの値を取り、同様に、デリバティブも一般的に$t=0$の価格とは異なる価格になる。デリバティブの各支払額に対する確率がわかっていれば、同様に満期時の$f$の期待価格、すなわち$(1-p)f(2) + pf(3)$もわかるだろう。しかし実際には、時刻$t=0$の時点では、将来の時刻$t=\delta t$の間に価格が遷移する確率$p$はわかっていない。

\subsection{債券のみの戦略}
キャッシュボンドだけのポートフォリオを考えてみよう。$\delta t$の経過で、このキャッシュボンドは$\exp(r\delta t)$倍に増える。つまり、時刻$t=0$で$\exp(-r\delta t)[(1-p)f(2) + pf(3)]$の価格でキャッシュボンドを購入すると、$\delta t$後には$(1-p)f(2) + pf(3)$の価格になる。この価格は、それがデリバティブの期待価格となるため、目指すべき目標として選択される。

Sを、時刻$t=0$における最初の価格$s_1$、下降価値$s_2$、上昇価値$s_3$を持つ二項枝の過程とする。上昇確率$p$のもとで、時刻1におけるSの期待値(または期待価格)$E_p(S_1)$は、\[E_p(S_1) = (1-p)s_2 + ps_3\]となる。

Sに対する契約$f$は、$S_1$と同様に確率変数である。つまり、契約$f$も同様に期待値が定義される: \[E_p[f(1)] = (1-p)f(2) + pf(3)\]

また、確率$p$による期待値に割引率をかけたもの\[\exp(-r\delta t)[(1-p)f(2) + pf(3)]\]を、割引期待値(または割引期待価格)と呼ぶ。

しかし、この価格構築戦略が良いものとなる可能性はほとんどない。これは、第1章で見た大数の法則が形を変えて再び現れたものであり、第1章の時と同様に、裁定による価格の強制を見落としている。そして、この期待値は少なくともフォワード取引には適用できないことを、すでに第1章で見てきた。フォワード価格は$f(2), f(3)$(つまり$s_2, s_3$)では表されず、むしろ債券に伴う金利$r$によって$s_1 \exp(r\delta t)$に強制されるのだ。

\subsection{債券と株式の組み合わせ}
もっと良い組み合わせを考えていこう。非常に短い期間のポートフォリオを構築するために、債券と株式の両方を利用することを考える。株式とデリバティブのパフォーマンスにより強く結びついている道具は、単なるキャッシュボンドではなく、実は株式そのものなのだ。

一般的なポートフォリオを$(\phi, \psi)$として、$\phi$単位の株式S(その価格は$\phi s_1$)と$\psi$単位のキャッシュボンドB(価格は$\psi B_0$)を保有すると仮定する。このポートフォリオを$t=0$で購入すると、その取得コストは$\phi s_1 + \psi B_0$となる。しかし、$t=\delta t$では、このポートフォリオは以下の2つの値のうちいずれか一つになる。
株価が上昇した場合の契約$f$の価格: \[f(3) = \phi s_3 + \psi B_0\exp(r\delta t)\]
株価が下降した場合の契約$f$の価格: \[f(2) = \phi s_2 + \psi B_0\exp(r\delta t)\]
今、我々には2つの価格の可能性と、2つの自由な変数$\phi$と$\psi$がある。株式が適切に動くという条件下で、一致させたい2つの価格$f(3)$と$f(2)$が手元にある。(※この問題は、契約$f$の値$f(2), f(3)$と将来の株式の値$s_2, s_3$がすでに決まっている場合に、その株式と現金(債券)をどれだけ保有すればよいかという問題であり、$s_2, s_3, f(2), f(3)$の値が分かっている場合の$\phi$と$\psi$の値を知りたい。)$s_3 \neq s_2$の場合、これらは次のように等式変形できる。
\[\phi = \frac{f(3) - f(2)}{s_3 - s_2}\]
\[\psi = B_0^{-1} \exp(-r\delta t) \left\{ f(3) - s_3 \frac{f(3) - f(2)}{s_3 - s_2} \right\}\]

\section{価格の適正性}
債券と株式から成る適切なポートフォリオを使えば、いかなるデリバティブ$f$をも構築することができる。この事実は契約に何らかの影響を及ぼすはずであり、実際に市場は、期待価格ではなく、これを合理的な価格として認めている。このポートフォリオを$t=0$で購入すると、その取得コストは$V = \phi s_1 + \psi B_0$であり、先ほどの結果を代入すると、
\[V = \frac{f(3) - f(2)}{s_3 - s_2} s_1 + \exp(-r\delta t) \left\{ f(3) - s_3 \frac{f(3) - f(2)}{s_3 - s_2} \right\}\]
となる。したがって、連続複利金利$r$、契約$f(2), f(3)$、将来の株価$s_2, s_3$の値が分かっていれば、任意のデリバティブを複製できる。

\subsection{この価格より安い価格でデリバティブの売買を提案した場合}
この$V$より安い価格$P$でデリバティブの売買が提案されたと仮定しよう。別の市場参加者は、彼らから安いデリバティブを価格$P$で任意の量購入することが可能だ。また、その価格$P$で任意の量購入した分と同じ金額の$(\phi, \psi)$のポートフォリオを売却することもできる。極短期間$\delta t$が経過した後、株価がどうなっていようと、デリバティブの価格$P$はポートフォリオの値段を相殺できる。この取引は、購入されたデリバティブとポートフォリオの取引単位ごとに$V-P$の利益を生み出す。つまり、誰でもリスクのない利益をいくらでも得ることができるのだ。したがって、$P$は市場参加者が認めるには合理的な価格ではなく、このタダで得られる利益を狙って、市場は素早く正しい価格$V$へと是正されるだろう。

\subsection{この価格より高い価格でデリバティブの売買を提案した場合}
この$V$より高い価格$P$でデリバティブの売買が提案されたと仮定しよう。別の市場参加者は、彼らから高いデリバティブを価格$P$で任意の量売却することが可能だ。また、その価格$P$で任意の量売却した分と同じ金額の$(\phi, \psi)$のポートフォリオを購入することもできる。極短期間$\delta t$が経過した後、株価がどうなっていようと、デリバティブの価格$P$はポートフォリオの値段を相殺できる。この取引は、売却されたデリバティブとポートフォリオの取引単位ごとに$P-V$の利益を生み出す。つまり、誰でもリスクのない利益をいくらでも得ることができるのだ。したがって、$P$は市場参加者が認めるには合理的な価格ではなく、このタダで得られる利益を狙って、市場は素早く正しい価格$V$へと是正されるだろう。

契約の相手方がリスクのない利益を手にするのを避けるには、価格$V$を提示することだけが唯一の方法である。こうして、$V$だけが時刻$t=0$におけるデリバティブの合理的な価格となる。

\subsection{1段階での全体像}
\subsubsection{問題}
利息の付かない債券と株式があり、どちらも$t=0$で1ドルだった。将来の$t=\delta t$では、株価は2ドルまたは0.5ドルになる。株価が上昇した場合に1ドルもらえる賭けに見合う賭け金はいくらだろうか?

\subsubsection{解答}
債券の価格をB、株価をS、求めたい賭けの支払いをXとする。先ほどの$V$の式を参照すると、今回の問題では$r=0$、$f(3)=1$、$f(2)=0$、$s_3=2$、$s_2=0.5$、$B_0=1$である。
\[X = \frac{f(3) - f(2)}{s_3 - s_2} s_1 + \exp(-r\delta t) \left\{ f(3) - s_3 \frac{f(3) - f(2)}{s_3 - s_2} \right\}\]
代入すると$X = 1/3$となり、この問題の答えは0.33ドルである。
また、
\[\phi = \frac{f(3) - f(2)}{s_3 - s_2}\]
\[\psi = B_0^{-1} \exp(-r\delta t) \left\{ f(3) - s_3 \frac{f(3) - f(2)}{s_3 - s_2} \right\}\]
なので、それぞれ代入すると、$\phi =2/3$、$\psi=-1/3$となる。すなわち、このデリバティブ(というか「賭け」)を複製するには、債券を2/3単位購入し、株式を1/3単位売却すればよい。実際、$t=0$では$2/3 \times 1 - 1/3 \times 1 = 1/3$(ドル)であり、$t=\delta t$において株価が上昇した場合は$2/3 \times 2 - 1/3 \times 1 = 1$(ドル)、株価が下降した場合は$2/3 \times 0.5 - 1/3 \times 1 = 0$となり、今回の問題の「賭け」の支払額と全く同じ結果となる。

\section{復活した期待値}
大数の法則に基づくアプローチは無駄であり、偶然の一致は別として、確率$p$と確率$(1-p)$で計算された期待価格は裁定機会を生むことになる、とすでに述べた。ここで、唐突に以下の数値$q$を考えてみよう。
\[q = \frac{s_1 \exp(r\delta t) - s_2}{s_3 - s_2}\]
以降、$s_3 > s_2$と仮定する。($s_3 > s_2$としても一般性は失われない。)

\subsection{$q < 0$の場合}
もし$q<0$であれば、不等式$s_1 \exp(r\delta t) < s_2 < s_3$が成り立つことになる。しかし、$s_1 \exp(r\delta t)$は、$t=0$で$s_1$の価格分の債券を購入したときに得られる価格である。したがって、$s_1 \exp(r\delta t) < s_2$が成り立つ場合、時刻$t=0$で任意の量の債券を売って、そのお金で価格$s_1$の株式を任意の量購入することで、無限の利益が得られてしまう。つまり、無裁定の仮定のもとでは、$q<0$は成立しない。

\subsection{$q > 1$の場合}
もし$q>1$であれば、不等式$s_2 < s_3 < s_1 \exp(r\delta t)$が成り立つことになる。$s_1 \exp(r\delta t)$は、時刻$t=0$で$s_1$の価格分の債券を購入したときに得られる価格である。したがって、$s_3 < s_1 \exp(r\delta t)$が成り立つ場合、時刻$t=0$で任意の量の債券を購入し、そのお金で価格$s_1$の株式を任意の量売却することで、無限の利益が得られてしまう。つまり、無裁定の仮定のもとでは、$q>1$は成立しない。

以上のことから、合理的な市場構造(すなわち無裁定条件)では、数値$q$は$0 \le q \le 1$を満たすため、数値$q$を何らかの確率と見なすことができる。

驚くべきことに、$(\phi, \psi)$のポートフォリオの価格$V$\[V = \frac{f(3) - f(2)}{s_3 - s_2} s_1 + \exp(-r\delta t) \left\{ f(3) - s_3 \frac{f(3) - f(2)}{s_3 - s_2} \right\},\]は、$q = \frac{s_1 \exp(r\delta t) - s_2}{s_3 - s_2}$を用いて等式変形すると、次のように表現できる。
\[V = \exp(-r\delta t) \left\{ (1-q)f(2) + qf(3) \right\}\]
これは、以前出てきた期待値の演算子$E_p[f(1)] = \exp(-r\delta t) \left\{ (1-p)f(2) + pf(3) \right\}$を使って表現すると、\[V = E_q[f(1)]\]となる。つまり、$(\phi, \psi)$ポートフォリオの裁定条件によって決定された妥当な価格$V$は、実は確率$q$における割引期待価格だったのだ。

\subsection{受渡価格がkのフォワード取引のペイオフ}
受渡価格が$k$のフォワード取引のペイオフは、
\[f(2) = s_2 - k\]
\[f(3) = s_3 - k\]
である。正しい受渡価格$k$を求めよう。この取引では、$t=0$でポートフォリオを保有する必要はない($t=0$で債券を借りて株を買い、そのまま満期を迎えたときに契約相手に株を引き渡してお金を得て、最後はそのお金で$t=0$で借りた借金を返済する)。$V=0$であることから、
\[V = \frac{f(3) - f(2)}{s_3 - s_2} s_1 + \exp(-r\delta t) \left\{ f(3) - s_3 \frac{f(3) - f(2)}{s_3 - s_2} \right\} \]
\[= \frac{(s_3-k) - (s_2-k)}{s_3 - s_2} s_1 + \exp(-r\delta t) \left\{ (s_3-k) - s_3 \frac{(s_3-k) - (s_2-k)}{s_3 - s_2} \right\}\]
\[= \frac{s_3 - s_2}{s_3 - s_2} s_1 + \exp(-r\delta t) \left\{ s_3 - k - s_3 \frac{s_3 - s_2}{s_3 - s_2} \right\} = s_1 + \exp(-r\delta t) (s_3 - k - s_3) = s_1 - k \exp(-r\delta t)\]
したがって、$V=0$のとき、$k = s_1 \exp(r\delta t)$となる。

\begin{thebibliography}{9}
	\bibitem{BaxterRennie}
	Financial Calculus - An Introduction to Derivative Pricing - Martin Baxter, Andrew Rennie
\end{thebibliography}


\end{document}
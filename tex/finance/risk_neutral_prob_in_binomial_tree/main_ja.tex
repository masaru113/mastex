\documentclass[uplatex,a4j,12pt,dvipdfmx]{jsarticle}
\usepackage[english]{babel}
\usepackage[letterpaper,top=2cm,bottom=2cm,left=3cm,right=3cm,marginparwidth=1.75cm]{geometry}
\usepackage{amsmath}
\usepackage{amssymb}
\usepackage{amsthm}
\usepackage{graphicx}
\usepackage{hyperref}
\usepackage{enumitem}

\title{
二項ツリーモデルにおけるリスク中立確率
}
\author{Masaru Okada}
\date{\today}

\begin{document}

\maketitle

\tableofcontents

\newpage

\section{はじめに}
本書の究極的な目的は、裁定取引の限界を探ることです。そのために、実際の金融市場の現実的なモデルとして機能し、価格構築の技術を生み出すことができる数学的枠組みを構築していきます。

\section{二項枝モデル}
私たちの目的に必要な最小限の要件は、次の2つです。
\begin{itemize}
	\item お金の時間的価値を表すもの
	\item 株式を表すランダムな要素
\end{itemize}
これら2つがなければ、いかなるモデルも実際の金融市場と関連性を持つとは考えられません。まずは、債券と株式から成る、可能な限りシンプルなモデルを検討することから始めましょう。

\subsection{株式}
時刻$t=0$で始まり$t=\delta t$で終わる、非常に短い期間を考えます。株価を表す変数は、予測不可能なランダムな要素を持っているのが理想的です。この短い期間$\delta t$では、株価は上昇するか、もしくは下降するかの2通りの変化しか起こらないと仮定します。

このランダムさに何らかの構造を持たせることを考えます。つまり、価格の上昇と下降に確率を割り当てるのです。短い期間$\delta t$での上昇確率を$p$、下降確率を$(1-p)$とします。$t=0$の初期価格を$s_1$とします。この価格は、私たちが量に制限なく売買できる価格を表しています。私たちは$\delta t$の間、その株式を保有できます。株式を保有している間は何も起こりません。つまり、正負いずれの量を保有してもコストはかからないとします。しかし、$\delta t$の時間が経過した後、株式は新しい価格を持つと仮定します。$\delta t$の後、下降すれば価格は$s_2$に、上昇すれば$s_3$になるとします。

\subsection{債券}
お金の時間的価値を表す何か、すなわちキャッシュボンドも考慮に入れる必要があります。株式の場合と同様に、$t=0$で始まり$t=\delta t$で終わる非常に短い期間を考えます。$\delta t$が経過した後、時刻$t=0$における1ドルが$\exp(r\delta t)$になるような連続複利$r$が存在すると仮定します。この金利で任意の金額を貸し借りできるとします。これを表現する変数として、$t=0$において価格$B_0$で売買できるキャッシュボンドBを導入します。つまり、このキャッシュボンドは、$\delta t$が経過した後、価格が$B_0 \exp(r\delta t)$になるのです。

これらたった2つの道具が、私たちの金融の世界のすべてです。単純化されているとはいえ、この世界は投資家にとっては相変わらず不確実性をもたらします。市場参加者にとって、$\delta t$が経過した後の株価が取り得る2つの値のうち、どちらか一方が都合が良いのです。(例えば、株を空売りしている場合は、価格が下落する方が都合が良い。)都合が良いか悪いかは、ランダムな結果に左右されます。投資家が将来の株価に基づいた支払いを要求する場合、$\delta t$における株価の取り得る2つの可能性である$s_2$と$s_3$に対して、支払額は関数$f$を用いてそれぞれ$f(2)$と$f(3)$に対応させるような関数で表現できます。例えば、受渡価格が$k$と契約されたフォワード契約は、$f(2)=s_2-k$および$f(3)=s_3-k$で表されます。

\section{無リスク構築}
適切な戦略によって、具体的にどのような関数$f$が提供されるかを考えてみましょう。第1章で述べたように、フォワード契約では、時刻$t=0$で$s_1$の価格で株式を購入し、その購入代金を賄うために同額$s_1$のキャッシュボンドを売却し、満期$\delta t$までそのポジションを保有します。したがって、その代金としては$s_1 \exp(r\delta t)$を請求すればよいのです。このことから、フォワード契約の価格は\[k = s_1 \exp(r\delta t)\]となります。これが裁定を通した価格決定です。

より複雑な$f$について、価格構築戦略を見つけることは一見できないように思われます。非常に短い時間$\delta t$の後、株価はランダムに2つの値を取り、同様に、デリバティブも一般的に$t=0$の価格とは異なる価格になります。デリバティブの各支払額に対する確率が分かっていれば、同様に満期時の$f$の期待価格、すなわち$(1-p)f(2) + pf(3)$もわかるでしょう。しかし実際には、時刻$t=0$の時点では、将来の時刻$t=\delta t$の間に価格が遷移する確率$p$はわかっていません。

\subsection{債券のみの戦略}
キャッシュボンドだけのポートフォリオを考えてみましょう。$\delta t$の経過で、このキャッシュボンドは$\exp(r\delta t)$倍に増えます。つまり、時刻$t=0$で$\exp(-r\delta t)[(1-p)f(2) + pf(3)]$の価格でキャッシュボンドを購入すると、$\delta t$後には$(1-p)f(2) + pf(3)$の価格になります。この価格は、それがデリバティブの期待価格となるため、目指すべき目標として選択されます。

$S$を、時刻$t=0$における最初の価格$s_1$、下降価値$s_2$、上昇価値$s_3$を持つ二項枝の過程とします。上昇確率$p$のもとで、時刻1における$S$の期待値(または期待価格)$E_p(S_1)$は、\[E_p(S_1) = (1-p)s_2 + ps_3\]となります。

$S$に対する契約$f$は、$S_1$と同様に確率変数です。つまり、契約$f$も同様に期待値が定義されます: \[E_p[f(1)] = (1-p)f(2) + pf(3)\]

また、確率$p$による期待値に割引率をかけたもの\[\exp(-r\delta t)[(1-p)f(2) + pf(3)]\]を、割引期待値(または割引期待価格)と呼びます。

しかし、この価格構築戦略が良いものとなる可能性はほとんどありません。これは、第1章で見た大数の法則が形を変えて再び現れたものであり、第1章の時と同様に、裁定による価格の強制を見落としています。そして、この期待値は少なくともフォワード取引には適用できないことを、すでに第1章で見てきました。フォワード価格は$f(2), f(3)$(つまり$s_2, s_3$)では表されず、むしろ債券に伴う金利$r$によって$s_1 \exp(r\delta t)$に強制されるのです。

\subsection{債券と株式の組み合わせ}
もっと良い組み合わせを考えていきましょう。非常に短い期間のポートフォリオを構築するために、債券と株式の両方を利用することを考えます。株式とデリバティブのパフォーマンスにより強く結びついている道具は、単なるキャッシュボンドではなく、実は株式そのものなのです。

一般的なポートフォリオを$(\phi, \psi)$として、$\phi$単位の株式$S$(その価格は$\phi s_1$)と$\psi$単位のキャッシュボンド$B$(価格は$\psi B_0$)を保有すると仮定します。このポートフォリオを$t=0$で購入すると、その取得コストは$\phi s_1 + \psi B_0$となります。しかし、$t=\delta t$では、このポートフォリオは以下の2つの値のうちいずれか一つになります。
株価が上昇した場合の契約$f$の価格: \[f(3) = \phi s_3 + \psi B_0\exp(r\delta t)\]
株価が下降した場合の契約$f$の価格: \[f(2) = \phi s_2 + \psi B_0\exp(r\delta t)\]
今、私たちには2つの価格の可能性と、2つの自由な変数$\phi$と$\psi$があります。株式が適切に動くという条件下で、一致させたい2つの価格$f(3)$と$f(2)$が手元にあります。(※この問題は、契約$f$の値$f(2), f(3)$と将来の株式の値$s_2, s_3$がすでに決まっている場合に、その株式と現金(債券)をどれだけ保有すればよいかという問題であり、$s_2, s_3, f(2), f(3)$の値が分かっている場合の$\phi$と$\psi$の値を知りたい。)$s_3 \neq s_2$の場合、これらは次のように等式変形できます。
\[\phi = \frac{f(3) - f(2)}{s_3 - s_2}\]
\[\psi = B_0^{-1} \exp(-r\delta t) \left\{ f(3) - s_3 \frac{f(3) - f(2)}{s_3 - s_2} \right\}\]

\section{価格の適正性}
債券と株式から成る適切なポートフォリオを使えば、いかなるデリバティブ$f$をも構築することができます。この事実は契約に何らかの影響を及ぼすはずであり、実際に市場は、期待価格ではなく、これを合理的な価格として認めています。このポートフォリオを$t=0$で購入すると、その取得コストは$V = \phi s_1 + \psi B_0$であり、先ほどの結果を代入すると、
\[V = \frac{f(3) - f(2)}{s_3 - s_2} s_1 + \exp(-r\delta t) \left\{ f(3) - s_3 \frac{f(3) - f(2)}{s_3 - s_2} \right\}\]
となります。したがって、連続複利金利$r$、契約$f(2), f(3)$、将来の株価$s_2, s_3$の値が分かっていれば、任意のデリバティブを複製できます。

\subsection{この価格より安い価格でデリバティブの売買を提案した場合}
この$V$より安い価格$P$でデリバティブの売買が提案されたと仮定しましょう。別の市場参加者は、彼らから安いデリバティブを価格$P$で任意の量購入することが可能です。また、その価格$P$で任意の量購入した分と同じ金額の$(\phi, \psi)$のポートフォリオを売却することもできます。極短期間$\delta t$が経過した後、株価がどうなっていようと、デリバティブの価格$P$はポートフォリオの値段を相殺できます。この取引は、購入されたデリバティブとポートフォリオの取引単位ごとに$V-P$の利益を生み出します。つまり、誰でもリスクのない利益をいくらでも得ることができるのです。したがって、$P$は市場参加者が認めるには合理的な価格ではなく、このタダで得られる利益を狙って、市場は素早く正しい価格$V$へと是正されるでしょう。

\subsection{この価格より高い価格でデリバティブの売買を提案した場合}
この$V$より高い価格$P$でデリバティブの売買が提案されたと仮定しましょう。別の市場参加者は、彼らから高いデリバティブを価格$P$で任意の量売却することが可能です。また、その価格$P$で任意の量売却した分と同じ金額の$(\phi, \psi)$のポートフォリオを購入することもできます。極短期間$\delta t$が経過した後、株価がどうなっていようと、デリバティブの価格$P$はポートフォリオの値段を相殺できます。この取引は、売却されたデリバティブとポートフォリオの取引単位ごとに$P-V$の利益を生み出します。つまり、誰でもリスクのない利益をいくらでも得ることができるのです。したがって、$P$は市場参加者が認めるには合理的な価格ではなく、このタダで得られる利益を狙って、市場は素早く正しい価格$V$へと是正されるでしょう。

契約の相手方がリスクのない利益を手にするのを避けるには、価格$V$を提示することだけが唯一の方法です。こうして、$V$だけが時刻$t=0$におけるデリバティブの合理的な価格となります。

\subsection{1段階での全体像}
\subsubsection{問題}
利息の付かない債券と株式があり、どちらも$t=0$で1ドルでした。将来の$t=\delta t$では、株価は2ドルまたは0.5ドルになります。株価が上昇した場合に1ドルもらえる賭けに見合う賭け金はいくらでしょうか?

\subsubsection{解答}
債券の価格を$B$、株価を$S$、求めたい賭けの支払いを$X$とします。先ほどの$V$の式を参照すると、今回の問題では$r=0$、$f(3)=1$、$f(2)=0$、$s_3=2$、$s_2=0.5$、そして$B_0=1$です。
\[X = \frac{f(3) - f(2)}{s_3 - s_2} s_1 + \exp(-r\delta t) \left\{ f(3) - s_3 \frac{f(3) - f(2)}{s_3 - s_2} \right\}\]
代入すると$X = 1/3$となり、この問題の答えは0.33ドルです。
また、
\[\phi = \frac{f(3) - f(2)}{s_3 - s_2}\]
\[\psi = B_0^{-1} \exp(-r\delta t) \left\{ f(3) - s_3 \frac{f(3) - f(2)}{s_3 - s_2} \right\}\]
なので、それぞれ代入すると、$\phi =2/3$、$\psi=-1/3$となります。すなわち、このデリバティブ(というか「賭け」)を複製するには、債券を2/3単位購入し、株式を1/3単位売却すればよいのです。実際、$t=0$では$2/3 \times 1 - 1/3 \times 1 = 1/3$(ドル)であり、$t=\delta t$において株価が上昇した場合は$2/3 \times 2 - 1/3 \times 1 = 1$(ドル)、株価が下降した場合は$2/3 \times 0.5 - 1/3 \times 1 = 0$となり、今回の問題の「賭け」の支払額と全く同じ結果となります。

\section{復活した期待値}
大数の法則に基づくアプローチは無駄であり、偶然の一致は別として、確率$p$と確率$(1-p)$で計算された期待価格は裁定機会を生むことになる、とすでに述べました。ここで、唐突に以下の数値$q$を考えてみましょう。
\[q = \frac{s_1 \exp(r\delta t) - s_2}{s_3 - s_2}\]
以降、$s_3 > s_2$と仮定します。($s_3 > s_2$としても一般性は失われません。)

\subsection{$q < 0$の場合}
もし$q<0$であれば、不等式$s_1 \exp(r\delta t) < s_2 < s_3$が成り立つことになります。しかし、$s_1 \exp(r\delta t)$は、$t=0$で$s_1$の価格分の債券を購入したときに得られる価格です。したがって、$s_1 \exp(r\delta t) < s_2$が成り立つ場合、時刻$t=0$で任意の量の債券を売って、そのお金で価格$s_1$の株式を任意の量購入することで、無限の利益が得られてしまいます。つまり、無裁定の仮定のもとでは、$q<0$は成立しません。

\subsection{$q > 1$の場合}
もし$q>1$であれば、不等式$s_2 < s_3 < s_1 \exp(r\delta t)$が成り立つことになります。$s_1 \exp(r\delta t)$は、時刻$t=0$で$s_1$の価格分の債券を購入したときに得られる価格です。したがって、$s_3 < s_1 \exp(r\delta t)$が成り立つ場合、時刻$t=0$で任意の量の債券を購入し、そのお金で価格$s_1$の株式を任意の量売却することで、無限の利益が得られてしまいます。つまり、無裁定の仮定のもとでは、$q>1$は成立しません。

以上のことから、合理的な市場構造(すなわち無裁定条件)では、数値$q$は$0 \le q \le 1$を満たすため、数値$q$を何らかの確率と見なすことができます。

驚くべきことに、$(\phi, \psi)$のポートフォリオの価格$V$\[V = \frac{f(3) - f(2)}{s_3 - s_2} s_1 + \exp(-r\delta t) \left\{ f(3) - s_3 \frac{f(3) - f(2)}{s_3 - s_2} \right\},\]は、$q = \frac{s_1 \exp(r\delta t) - s_2}{s_3 - s_2}$を用いて等式変形すると、次のように表現できます。
\[V = \exp(-r\delta t) \left\{ (1-q)f(2) + qf(3) \right\}\]
これは、以前出てきた期待値の演算子$E_p[f(1)] = \exp(-r\delta t) \left\{ (1-p)f(2) + pf(3) \right\}$を使って表現すると、\[V = E_q[f(1)]\]となります。つまり、$(\phi, \psi)$ポートフォリオの裁定条件によって決定された妥当な価格$V$は、実は確率$q$における割引期待価格だったのです。

\subsection{受渡価格がkのフォワード取引のペイオフ}
受渡価格が$k$のフォワード取引のペイオフは、
\[f(2) = s_2 - k\]
\[f(3) = s_3 - k\]
です。正しい受渡価格$k$を求めましょう。この取引では、$t=0$でポートフォリオを保有する必要はありません($t=0$で債券を借りて株を買い、そのまま満期を迎えたときに契約相手に株を引き渡してお金を得て、最後はそのお金で$t=0$で借りた借金を返済します)。$V=0$であることから、
\[V = \frac{f(3) - f(2)}{s_3 - s_2} s_1 + \exp(-r\delta t) \left\{ f(3) - s_3 \frac{f(3) - f(2)}{s_3 - s_2} \right\} \]
\[= \frac{(s_3-k) - (s_2-k)}{s_3 - s_2} s_1 + \exp(-r\delta t) \left\{ (s_3-k) - s_3 \frac{(s_3-k) - (s_2-k)}{s_3 - s_2} \right\}\]
\[= \frac{s_3 - s_2}{s_3 - s_2} s_1 + \exp(-r\delta t) \left\{ s_3 - k - s_3 \frac{s_3 - s_2}{s_3 - s_2} \right\} = s_1 + \exp(-r\delta t) (s_3 - k - s_3) = s_1 - k \exp(-r\delta t)\]
したがって、$V=0$のとき、$k = s_1 \exp(r\delta t)$となります。

\section{二項ツリー}

二項枝から二項ツリーへと議論を進めていきます。1期間だけの経過は分析しやすいですが、モデルとしては最小限のものを示すにすぎません。二項枝の場合、キャッシュボンドとランダムな株式を持っていても、株価の時間経過後の価格は2パターンに絞られていました。二項枝から二項ツリーへと話を展開していくと何が言えるかを見ていきましょう。

以下では1期間ではなく、多時間の時間発展を見ます。ここでもキャッシュボンドとランダムな株式しか持たないものとし、取引費用や債務不履行に陥るリスク、ビッドとオファーのスプレッド無しに無限に売り買いできるものとします。

\subsection{株式}
株価$S$は前節と同様に、$t=0$における$S$の値が$S_0 = s_1$のとき、時間の最小単位$\delta t$だけ経過した後、$S_1=s_2$または$S_1=s_3$の2値しか取らないとします。しかし、$2\delta t$時間経過後、$s_2$からは$S_2=s_4$または$s_5$に、$s_3$からは$S_2=s_6$または$s_7$になり得ます。自然数$i$として、時間$i\delta t$経過後、株式は$2^i$個の値を取ります。しかし、時間$(i-1)$における値が与えられると、考えられる可能性はたった2つしかありません。$s_j$の次の値は$s_{2j}$、または$s_{(2j+1)}$です。また、株価の上昇確率(つまり$s_j$が$s_{(2j+1)}$になる確率)を$p_j$とすると、下降確率(つまり$s_j$が$s_{2j}$になる確率)は$1-p_j$です。

\subsection{キャッシュボンド}
\subsubsection{一般論}
キャッシュボンドのパフォーマンスにランダムな可能性があることは認めることができます。(ただし、その実際の正確な形状については今は特に興味を持っていません。)キャッシュボンドの場合は、株式とは全く違った種類のランダムです。キャッシュボンドはお金の時間的価値と同じ構造を持っています。支払われるべき金利は時間によってのみ変化し得ます。しかし次の瞬間のキャッシュボンドの価値はいつも分かっています。なぜならば、この値は$t=0$で既知である金利にのみ依存するからです。
\subsubsection{仮定}
ここではツリーのどこでも一定である金利$r$を導入します。つまり、時点$n\delta t$におけるキャッシュボンドの値は$B_0 \exp(rn\delta t)$です。

\subsection{ツリーは複雑}
私たちの最終目的は、原資産が連続時間の中で連続した値を取る場合に、無リスク構築の限界を理解するということです。そして$\delta t \to 0$の極限では、この私たちのモデルはこの最終目的に非常に一致します。ツリーが単純だと諦めてしまうのではなく、その分析を進めていく目的に対しては、それほど複雑ではないと言うべきです。

\subsection{後ろ向き帰納法}
実は大半の準備はすでに1期間の二項枝の議論で終わっています。この多期間の二項ツリーの鍵となるポイントは、支払いが発生する最終時点から価格を求めたい期首の時間まで、ごく短い時間ごとに遡って構築ポートフォリオを広げていく後ろ向き帰納法です。1期間二項枝の議論と同様に最終時点(満期)での支払額(ペイオフ)の関数$f$がありました。多期間の二項ツリーでは期初$t=0$から満期に至るまでの途中の契約にまで考えを拡張することができます。

二項ツリーの構造は、「交点」と「株式がその交点に至るまでの経路」との間に1対1の関係を持つものがあります。ある交点へ至る経路とは異なる別の経路をたどればその交点へはたどり着きません。この条件により、「ある契約の支払額$f$」と「ある特定の終点」とを対応付けることができます。一般的な契約は、この契約の満期における交点上の関数$f$と考えることができます。

\subsection{二段階}
期待値オペレータ$E_p[]$は、一つの枝(1期間二項枝)に対して定義されました。ここでは二段階、すなわち3つの枝を持つツリーについて計算していきます。どの枝にも適用される金利も金利$r$で一定であるとします。1期間二項枝の議論より、単位時間$i$における交点$j$のデリバティブの価格$f(j)$を示す、適切な$q_j$の組み合わせが存在し、
\[f(j) = \exp(-r\delta t) ( q_j f(2j+1) + ( 1 -q_j )f(2j) )\]
となります。これは時点$(i+1)$における契約価格$f(2j+1)$と$f(2j)$の確率(測度)$q_i$の下での割引期待値です。$j=3,2$でそれぞれ
\[f(3) = \exp(-r\delta t) ( q_3 f(7) + ( 1 -q_3 )f(6) )\]
\[f(2) = \exp(-r\delta t) ( q_2 f(5) + ( 1 -q_2 )f(4) )\]
ここで確率$q_j$は、
\[q_j = \frac{s_j \exp(r\delta t) - s_{2j}}{s_{(2j+1)} - s_{2j}}\]
すなわち、
\[q_3 = \frac{s_3 \exp(r\delta t) - s_6}{s_7 - s_6}\]
\[q_2 = \frac{s_2 \exp(r\delta t) - s_4}{s_5 - s_4}\]
です。
\[f(1) = \exp(-r\delta t) ( q_1f(3) + ( 1 -q_1 )f(2) )\]
なので、全て代入していくと、
\[f(1) = \exp(-2r\delta t) \{ q_1 q_3 f(7) + q_1 (1 - q_3) f(6) + (1 - q_1) q_2 f(5) + (1 - q_1) (1 - q_2) f(4) \}\]
となります。

ここから次のことが分かります。
\subsection{ツリーについての期待値}
ツリーの最終点における請求$f$の期待値は、各々の経路の確率に支払額を掛けて合計したものです。

2段階のツリーでは最終点に至るまでに4通りの経路がありました。したがって$f(1)$は4項から構成されています。経路の確率はそれぞれ$q_1 q_3$, $q_1 (1 - q_3)$, $(1 - q_1) q_2$, $(1 - q_1) (1 - q_2)$でした。そのそれぞれに対応する請求は各終点$f(7),f(6),f(5),f(4)$でした。期間は$2\delta t$あり、割引率として$\exp(-2r\delta t)$が乗じられています。

\subsection{帰納的な段階}
時点$(n-1)$における1交点から時点$n$における2つの交点(合計で3点)を考えます。前節の結果により、時点$n$においては株式と債券からなるリスクの無いポートフォリオ$(\phi,\psi)$によって複製することができます。そして、この3点のみ考えると1期間の二項枝モデルと区別がつきません。(1期間の二項枝モデルと全く同じです。)

\subsection{帰納による結果}
帰納法によって、ツリーを時間発展と逆向きに(つまり契約満期から$t=0$へ)と辿っていくことで、それぞれの枝におけるポートフォリオ$(\phi,\psi)$を構築(複製)することができます。ツリー全体の根幹である$t=0$にただ一つの値だけが到達します。これは$t=0$におけるデリバティブの価格です。これは、1期間二項枝モデルの場合と同様に、その株式が実際にどんな経路を辿ろうとも、私たちは裁定という強い制約の下で唯一のポートフォリオ$(\phi,\psi)$を構築(複製)することができるためです。このポートフォリオ$(\phi,\psi)$は各単位時間ごとに変動します。この構築ポートフォリオでは、一つの交点における一つの株式保有量$\phi$だけでなく、全ての交点におけるそれぞれ全ての株式保有量$\phi$が求まっています。つまり、運命のコイン(表か裏しか出ないコイン)が投げられ、ツリー上で株式が変動したとき、この保有量(保有すべき量)$\phi$も変動します。この構築ポートフォリオ$(\phi,\psi)$は株式と同様にランダムではありますが、株式と異なる最も重要な点は、「株式とは異なり、構築ポートフォリオ$(\phi,\psi)$はちょうどそのタイミングに売買できるような量が、その単位時間分1つ事前に分かる」ということです。(例えば前節のp.28の例題)全ての契約は株式と債券から成るポートフォリオで複製することができます。そして全ての契約は無裁定価格を持ちます。

\subsection{再び期待値}
今まで確率$p_j$(前述の株価の上昇確率、すなわち$s_j$が$s_{(2j+1)}$になる確率)を必要としませんでした。たまたま立式上で$p_j$を必要としなかったのではなく、ここにはしっかりとしたロジックがあります。前節の結論は、適切な「確率」に関して期待値オペレータが正しいローカル(各時における、時間に依存する)ヘッジを与えるということと同値であるということをこれから見ていきます。

\subsection{具体例}
図2.5のように各点における株価の上昇確率は3/4、下降確率は1/4という3期間二項ツリーを考えます。問題は、時点3において値段100で株式を購入できるというオプションの価格を求めよ、というものです。

図2.5の最終時点(満期)における価格はそれぞれ160,120,80,40であり、100より大きければ権利行使をして、100より小さければ権利放棄することから、オプション価格の二項ツリーの満期時の価格はそれぞれ$60(=160-100)$,$20(=120-100)$,$0$,$0$となります。

ここで新しい確率$q$と契約価格$f$のための新しい方程式が必要になります。それぞれは、
\[q = \frac{s_{now} - s_{down}}{s_{up} - s_{down}}\]
\[f_{now} = qf_{up} + (1-q)f_{down}\]
でそれぞれ求まります。

この新しい確率$q$は全ての交点において1/2となります。実際、例えば時点2で株価が140の枝については、
\[q = \frac{140 - 120}{160 - 120} = 1/2\]
時点1で株価が80の枝については、
\[q = \frac{80 - 60}{100 - 60} = 1/2\]

契約価格は、例えば
時点2で株価が140だった交点については、
\[f_{now} = (1/2) \times 60 + (1/2) \times 20 = 40\]
時点1で株価が80の枝については、
\[f_{now} = (1/2) \times 10 + (1/2) \times 0 = 5\]

以上の手続きを進めることで全ての時点(交点)でのリスク中立確率、契約価格が分かります。

\subsubsection{qとfの実際の利用方法について}
具体的なヘッジ戦略を考えます。以下で定義される$\phi$を計算していきます。
\[\phi = \frac{f_{up} - f_{down}}{s_{up} - s_{down}}\]
これは二項枝モデルにおける当初($t=0$)の複製ポートフォリオの株式の保有量です。

\begin{itemize}
	\item \textbf{$t=0$における戦略(株価=100、オプション価格=15の交点)}
	      先ほどの手続きで時点0におけるオプションの価格は15と求まりました。(この「オプションの価格」は例題における「ブックメーカー(胴元、今回は私たちの立場)に支払われるゲームの参加料」と同じ意味です。)まず最初はオプションには15の価値があります。(つまりこの「ゲーム」の参加料は15です。)さて、この$t=0$においては$\phi = (25 - 5) / ( 120 - 80 ) = 1/2$です。$t=0$時点では株式の値段は100であり、これを1/2単位購入するには50のコストがかかります。つまり、(期初のオプション価格) - (保有すべき株価) = $15 - 50 = -35$であり、$\phi(= 0.5)$単位の株式を保有するには35の借り入れが必要になります。以上から、時点0における借り入れは35です。

	\item \textbf{時点1における戦略(株価=120、オプション価格=25の交点に到達した場合)}
	      時点1のこの交点における$\phi$は$\phi = ( 40 - 10 ) / ( 140 - 100 ) = 3/4$です。$t=0$では$\phi=1/2$だったので、新たな$\phi$に一致させるためには$\phi$を(つまり株式の保有量を)$3/4 - 1/2 = 1/4$だけ増やさなければなりません。この交点での株価は120なので、新たな$\phi$に一致させるために$120 \times (1/4)=30$だけ株式を買い増ししたいです。したがって、新たな$\phi$に一致させるためにさらに30だけ借り入れを行います。$t=0$で35だけ借り入れていたので、この交点においてトータルの借り入れは65になりました。

	\item \textbf{時点2における戦略(株価=140、オプション価格=40の交点に到達した場合)}
	      時点2のこの交点における$\phi$は$\phi = ( 60 - 20 ) / ( 160 - 120 ) = 1$です。1つ前の交点では$\phi=3/4$だったので、新たな$\phi$に一致させるためには$\phi$を(つまり株式の保有量を)$1 - 3/4 = 1/4$だけ増やさなければなりません。この交点での株価は140なので、新たな$\phi$に一致させるために$140 \times (1/4)=35$だけ株式を買い増ししたいです。したがって、新たな$\phi$に一致させるためにさらに35だけ借り入れを行います。1つ前の交点で65だけ借り入れていたので、この交点においてトータルの借り入れは100になりました。

	\item \textbf{時点3(満期時)における戦略(株価=120、オプション価格=20の交点に到達した場合)}
	      $120 > 100$(満期時の株価 $>$ オプションの権利行使価格)なので、オプションは権利行使されます。私たちは$\phi = 1$だけ株式を受け渡します。1つ前の時点でトータル100だけ借り入れた分を返済するために、株式を受け渡した相手から現金を100受け取ります。以上から、オプションの価格は$t=0$において15でなければオプションを売る側が損をしてしまうので、15が妥当な価格であることが分かりました。
\end{itemize}

以上が$\phi$の利用方法です。もう一度同じツリーで今の例とは別の経路の例を考えてみます。

\begin{itemize}
	\item \textbf{$t=0$における戦略(株価=100、オプション価格=15の交点)}
	      $t=0$においては$\phi = (25 - 5) / ( 120 - 80 ) = 1/2$であり、オプションの価格 = 複製ポートフォリオの債券 + 複製ポートフォリオの株式 債券 = オプション - 株式 = オプション - 株価保有量$\phi$ * 株価 = $15 - (1/2) \times 100 = -35$です。つまり$t=0$でまずは$\phi=1/2$にするために35の借り入れを行います。(さっきの例と全く同じ議論です。)

	\item \textbf{時点1における戦略(株価=80、オプション価格=5の交点に到達した場合)}
	      $\phi = (10 - 0) / ( 100 - 60 ) = 1/4$です。債券 = オプション - 株式 = $5 - 80 \times 1/4 = -15$です。前時点で35の借り入れでしたが、この交点においては15まで借り入れを縮小します。(つまり、借り入れている額を15まで縮小するように、保有していた株式を部分的に値段20で売るのです。)

	\item \textbf{時点2における戦略(株価=100、オプション価格=10の交点に到達した場合)}
	      $\phi = ( 20 - 0 ) / ( 120 - 80 )$です。債券 = オプション - 株式 = $10 - 100 \times (1/2) = -40$です。前時点で15の借り入れでしたが、さらに25だけ追加で借り入れて借り入れ額を40にします。(その追加した25の借り入れ額は株式の購入金額に費やしています。今、世界は債券か株式しか存在しないので、債券か株式のどちらかに交換されています。つまり保有している債券の資産量が減少しているとするならば、保有している株式の量が増加しているのです。)

	      (なぜ「株式」と「債券」の「2つ」を考えるのか。例えば「石ころ」と「水」と「ごはん」の「3つ」ではだめなのか?この問いは2つに分けられます。まず、種類の個数がなぜ2個なのか。それから、種類は「株式」と「債券」のみに限られるのか。→まず、今は最も簡単なモデルを考えています。何か物と物を交換するので、交換を行うには1よりも大きい最小の自然数個の種類(つまり2種類のもの)があれば良いのです。さらに、期初のデリバティブの価格は原資産のランダムさを反映した契約の支払価格ですが、ランダムというのは債券の立場に立ったときに債券の価格から測るとランダムに動くという意味です。逆に、株式の立場からすれば債券がランダムに動いているのです。例えば、1株の値段が債券100円→ある日には110円→また別の日には105円と変わるように。何かランダムに動く資産と、一つ価値基準になる資産があれば、「水」と「石ころ」でも良いのです。水1リットルを買うにはダイヤモンド1g→ある日には2g→また別の日には0.5gと変わるように。何か一つの価値基準から見ると、もう一つがランダムに動くものであればキャッシュフローが複製できます。もし3種類のものがあれば、そのうち1つを使わないことによって2種類の資産のモデルに還元できます。なので、3以上の自然数$n$種類の商品を考えても別にいいですが、そのうち$n-2$種類は考察に使う必要がありません。)

	\item \textbf{時点3(満期)における戦略(株価=80、オプション価格=0の交点に到達した場合)}
	      オプションの権利行使はされません。(株価=80 $<$ 権利行使価格=100なので)保有している株数は前時点の$\phi=1/2$なので、株式の資産は$80 \times (1/2)$で40です。借り入れ額は前時点で40なので、損得なく取引を終了できます。
\end{itemize}

ここでも期待値を用いることができます。リスク中立上昇確率$q$のもとで、最終交点に到達する各確率は、満期時に取り得る価格4点で、$1/8, 3/8, 3/8, 1/8$です。この確率$q$の下での$f$の期待値はオプションの無裁定価格の15になります。しかし元の株式での最終交点に到達する各確率は、満期時に取り得る価格4点で、$27/64, 27/64, 9/64, 1/64$です。この確率$p$の下での$f$の期待値は33.75となります。
この確率$p$の下での$S$の期待値は$(160 \times 27+120 \times 27+80 \times 9+40 \times 1)/64=130$となり、別に今までの計算のどこかに出てきた値というわけでもなく、今回初めて登場した意味の無い値でした。
どうやら確率$q$の下での$f$の期待値という$(q,f)$の組み合わせだけがどうやら意味を持ちそうです。


\begin{thebibliography}{9}
	\bibitem{BaxterRennie}
	Financial Calculus - An Introduction to Derivative Pricing - Martin Baxter, Andrew Rennie
\end{thebibliography}

\end{document}
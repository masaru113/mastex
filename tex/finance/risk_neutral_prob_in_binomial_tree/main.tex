\documentclass[uplatex,a4j,12pt,dvipdfmx]{jsarticle}
\usepackage[english]{babel}
\usepackage[letterpaper,top=2cm,bottom=2cm,left=3cm,right=3cm,marginparwidth=1.75cm]{geometry}
\usepackage{amsmath}
\usepackage{amssymb}
\usepackage{amsthm}
\usepackage{graphicx}
\usepackage{hyperref}
\usepackage{enumitem}

\title{
Risk-Neutral Probabilities in a Binomial Tree Model
}
\author{Masaru Okada}
\date{\today}

\begin{document}

\maketitle

\tableofcontents

\newpage

\section{Introduction}
The ultimate goal of this book is to explore the limits of arbitrage. To that end, we'll develop a mathematical framework that serves as a realistic model for actual financial markets and allows us to build pricing techniques.

\section{The Binomial Branch Model}
There are two minimum requirements for our purposes:
\begin{itemize}
	\item A representation of the time value of money
	\item A random element to represent stocks
\end{itemize}
Without these two things, no model can be considered relevant to actual financial markets. So, we begin by considering the simplest possible model, one consisting of bonds and stocks.

\subsection{Stocks}
Let's consider a very short period starting at time $t=0$ and ending at $t=\delta t$. A variable representing a stock price should ideally have an unpredictable, random component. Over this short interval $\delta t$, we'll assume the stock price can only change in two ways: it can either go up or go down.

We'll introduce some structure to this randomness by assigning probabilities to the price going up or down. Let the probability of an upward movement over the short interval $\delta t$ be $p$, and the probability of a downward movement be $(1-p)$. The initial price at $t=0$ is denoted as $s_1$. This is the price at which we can buy or sell any quantity of the stock without limit. We can then hold the stock for the duration of $\delta t$. Nothing happens while we hold the stock; that is, there is no cost to holding any quantity, whether positive or negative. However, after the elapsed time $\delta t$, the stock is assumed to have a new price. The price becomes $s_2$ if it goes down, and $s_3$ if it goes up.

\subsection{Bonds}
We also need to consider something that represents the time value of money, namely, a cash bond. Just as with the stock, we consider a very short period starting at $t=0$ and ending at $t=\delta t$. We'll assume there is a continuous compound interest rate $r$ such that one dollar at time $t=0$ becomes $\exp(r\delta t)$ after the elapsed time $\delta t$. Any amount can be lent or borrowed at this interest rate. To represent this, we introduce a cash bond B, which can be traded at a price $B_0$ at time $t=0$. The value of this bond becomes $B_0 \exp(r\delta t)$ after the elapsed time $\delta t$.

These two tools are all that our financial world contains. Although simplified, this world still presents uncertainty to investors. For a market participant, only one of the two possible stock prices after $\delta t$ is favorable. (For example, if one is shorting the stock, a price drop is favorable.) Whether the outcome is favorable or not depends on a random result. An investor's demand for a payoff based on the future stock price can be expressed as a function $f$, which maps the two possible stock prices $s_2$ and $s_3$ at $t=\delta t$ to payoffs $f(2)$ and $f(3)$, respectively. For instance, a forward contract with a delivery price of $k$ is represented by $f(2)=s_2 -k$ and $f(3)=s_3 -k$.

\section{Risk-Free Construction}
Let's consider what specific function $f$ might be provided by a suitable strategy. As mentioned in Chapter 1, in a forward contract, you buy the stock at a price of $s_1$ at $t=0$, sell cash bonds for the same amount $s_1$ to finance the purchase, and hold the position until maturity at $\delta t$. Therefore, the amount to be claimed should be $s_1 \exp(r\delta t)$. This leads to the forward contract price being \[k = s_1 \exp(r\delta t).\] This is price determination through arbitrage.

It may seem at first glance that we can't find a price-building strategy for a more complex $f$. After the very short time $\delta t$, the stock price takes on one of two random values, and similarly, derivatives generally take on a price different from their $t=0$ price. If we knew the probabilities for each derivative payout, we could also find the expected price of $f$ at maturity, which would be $(1-p)f(2) + pf(3)$. But in reality, at time $t=0$, we don't know the probability $p$ of the price transitioning to its future state at time $t=\delta t$.

\subsection{A Bond-Only Strategy}
Let's consider a portfolio consisting only of cash bonds. Over the elapsed time $\delta t$, this cash bond increases by a factor of $\exp(r\delta t)$. This means that if you buy a cash bond at a price of $\exp(- r\delta t)[(1-p)f(2) + pf(3)]$ at $t=0$, its price will become $(1-p)f(2) + pf(3)$ after $\delta t$. This price is chosen as the target because it is the expected price of the derivative.

Let S be a binomial process with an initial price $s_1$ at time $t=0$, a downward value of $s_2$, and an upward value of $s_3$. Under the upward probability $p$, the expected value (or expected price) of $S_1$ at time 1, $E_p(S_1)$, is:
\[E_p(S_1) = (1-p)s_2 + ps_3.\]
The contract $f$ on S is a random variable, just like $S_1$. This means the expected value of the contract $f$ is similarly defined:
\[E_p[f(1)] = (1-p)f(2) + pf(3).\]
Also, the expected value discounted by the interest rate, under probability $p$, is \[\exp(-r\delta t)[(1-p)f(2) + pf(3)].\] This is called the discounted expected value (or discounted expected price).

However, this price-building strategy is highly unlikely to be a good one. It's the law of large numbers from Chapter 1 appearing again in a different form, and just as before, it overlooks the price enforcement that comes from arbitrage. We already saw in Chapter 1 that this expectation cannot be applied to a forward contract, at least. The forward price is not expressed by $f(2), f(3)$ (i.e., $s_2, s_3$); rather, it's enforced by the interest rate $r$ associated with the bond, resulting in $s_1 \exp(r\delta t)$.

\subsection{Combining Bonds and Stocks}
Let's think about a better combination. We'll use both bonds and stocks to construct a short-term portfolio. The tool that is more strongly linked to the performance of both stocks and derivatives than a simple cash bond is, in fact, the stock itself.

As a general portfolio ($\phi$,$\psi$), we assume we hold $\phi$ units of stock S (at a price of $\phi s_1$) and $\psi$ units of cash bond B (at a price of $\psi B_0$). If we purchase this portfolio at $t=0$, the acquisition cost would be $\phi s_1 + \psi B_0$. However, at $t=\delta t$, this portfolio would take one of two possible values:
Price of contract $f$ if the stock price goes up: \[f(3) = \phi s_3 + \psi B_0\exp(r\delta t)\]
Price of contract $f$ if the stock price goes down: \[f(2) = \phi s_2 + \psi B_0\exp(r\delta t)\]
Now, we have two possible prices and two free variables, $\phi$ and $\psi$. Under the condition that the stock moves appropriately, we have the two prices we want to match, $f(3)$ and $f(2)$, in hand. (Note: This is a problem of determining how much stock and cash (bonds) to hold if the values $s_2, s_3, f(2), f(3)$ are already known, and we want to find the values of $\phi$ and $\psi$.) If $s_3 \neq s_2$, we can transform the equations as follows:
\[\phi = \frac{f(3) - f(2)}{s_3 - s_2}\]
\[\psi = B_0^{-1} \exp(-r\delta t) \left\{ f(3) - s_3 \frac{f(3) - f(2)}{s_3 - s_2} \right\}\]

\section{Fair Pricing}
Any derivative $f$ can be constructed from a suitable portfolio of bonds and stocks. This fact should have some influence on the contract, and indeed, the market recognizes this as a rational price, not an expected price. If we purchase this portfolio at $t=0$, its acquisition cost is $V = \phi s_1 + \psi B_0$. Substituting the previous results, we get:
\[V = \frac{f(3) - f(2)}{s_3 - s_2} s_1 + \exp(-r\delta t) \left\{ f(3) - s_3 \frac{f(3) - f(2)}{s_3 - s_2} \right\}\]
Therefore, if we know the continuous compound interest rate $r$, the contract values $f(2), f(3)$, and the future stock prices $s_2, s_3$, we can replicate any derivative.

\subsection{If a derivative is offered for sale at a price lower than V}
Let's assume a derivative is offered for sale at a price $P$ which is lower than this $V$. Another market participant can buy any quantity of the cheap derivative from them at price $P$. They can also sell a portfolio of ($\phi$, $\psi$) for the same amount as the purchased derivative. After the short period $\delta t$ has passed, no matter what the stock price is, the price $P$ of the derivative can offset the value of the portfolio. This transaction generates a profit of $V-P$ for each unit of the purchased derivative and portfolio traded. This means anyone can earn a risk-free profit of any size. Therefore, $P$ is not a rational price for market participants to accept, and the market will quickly correct to the correct price $V$ in an effort to capture this free profit.

\subsection{If a derivative is offered for sale at a price higher than V}
Let's assume a derivative is offered for sale at a price $P$ which is higher than this $V$. Another market participant can sell any quantity of the expensive derivative to them at price $P$. They can also buy a portfolio of ($\phi$, $\psi$) for the same amount as the sold derivative. After the short period $\delta t$ has passed, no matter what the stock price is, the price $P$ of the derivative can offset the value of the portfolio. This transaction generates a profit of $P-V$ for each unit of the sold derivative and portfolio traded. This means anyone can earn a risk-free profit of any size. Therefore, $P$ is not a rational price for market participants to accept, and the market will quickly correct to the correct price $V$ in an effort to capture this free profit.

The only way to prevent the counterparty from making a risk-free profit is to offer the price $V$. Thus, only $V$ is the rational price for the derivative at time $t=0$.

\subsection{Example - The Big Picture in One Step}
\subsubsection{Problem}
There's a non-interest-bearing bond and a stock, both at 1 dollar at $t=0$. In the future, at $t=\delta t$, the stock price will be either 2 dollars or 0.5 dollars. What is the fair price for a bet that pays 1 dollar if the stock price goes up?

\subsubsection{Solution}
Let B be the bond price, S be the stock price, and X be the desired payoff for the bet. Referring to the expression for V from earlier, for this problem, we have $r=0$, $f(3)=1$, $f(2)=0$, $s_3=2$, $s_2=0.5$, and $B_0=1$.
\[X = \frac{f(3) - f(2)}{s_3 - s_2} s_1 + \exp(-r\delta t) \left\{ f(3) - s_3 \frac{f(3) - f(2)}{s_3 - s_2} \right\}\]
Substituting the values, we find $X = 1/3$, so the answer to this problem is 0.33 dollars.
Also, since
\[\phi = \frac{f(3) - f(2)}{s_3 - s_2}\]
\[\psi = B_0^{-1} \exp(-r\delta t) \left\{ f(3) - s_3 \frac{f(3) - f(2)}{s_3 - s_2} \right\}\]
substituting gives us $\phi =2/3$ and $\psi=-1/3$. This means that to replicate this derivative (or "bet"), you should buy 2/3 units of the bond and sell 1/3 units of the stock. Indeed, at $t=0$, the value is $2/3 \times 1 - 1/3 \times 1 = 1/3$ (dollars). If the stock price goes up at $t=\delta t$, the value becomes $2/3 \times 2 - 1/3 \times 1 = 1$ (dollar), and if the stock price goes down, the value becomes $2/3 \times 0.5 - 1/3 \times 1 = 0$, which are exactly the same payouts as the "bet" in this problem.

\section{The Return of Expected Value}
I've already stated that the approach based on the law of large numbers is futile, and that the expected price calculated using probabilities $p$ and $(1-p)$ would provide an arbitrage opportunity, except by sheer coincidence. Here, let's unexpectedly consider the following number $q$:
\[q = \frac{s_1 \exp(r\delta t) - s_2}{s_3 - s_2}\]
We'll assume $s_3 > s_2$ from now on. (This doesn't result in any loss of generality.)

\subsection{The case where $q < 0$}
If $q<0$, then the inequality $s_1 \exp(r\delta t) < s_2 < s_3$ would hold. However, $s_1 \exp(r\delta t)$ is the price you get when you buy a bond worth $s_1$ at $t=0$. Therefore, if $s_1 \exp(r\delta t) < s_2$, you could sell any quantity of bonds at $t=0$ and use the proceeds to buy any quantity of stock at price $s_1$, thereby earning an infinite profit. In other words, under the assumption of no arbitrage, $q<0$ cannot hold.

\subsection{The case where $q > 1$}
If $q>1$, then the inequality $s_2 < s_3 < s_1 \exp(r\delta t)$ would hold. $s_1 \exp(r\delta t)$ is the price you get when you buy a bond worth $s_1$ at $t=0$. Therefore, if $s_3 < s_1 \exp(r\delta t)$, you could buy any quantity of bonds at $t=0$ and use the proceeds to sell any quantity of stock at price $s_1$, thereby earning an infinite profit. In other words, under the assumption of no arbitrage, $q>1$ cannot hold.

From the above, in a rational market structure (i.e., under the no-arbitrage condition), the value $q$ must satisfy $0 \le q \le 1$, so we can view $q$ as some kind of probability.

Surprisingly, the price $V$ of the $(\phi,\psi)$ portfolio,
\[V = \frac{f(3) - f(2)}{s_3 - s_2} s_1 + \exp(-r\delta t) \left\{ f(3) - s_3 \frac{f(3) - f(2)}{s_3 - s_2} \right\},\]
can be transformed using $q = \frac{s_1 \exp(r\delta t) - s_2}{s_3 - s_2}$ into the following expression:
\[V = \exp(-r\delta t) \left\{ (1-q)f(2) + qf(3) \right\}\]
It's the very same thing, but when we express it using the expected value operator $E_p[f(1)] = \exp(-r\delta t) \left\{ (1-p)f(2) + pf(3) \right\}$ that appeared earlier, we get:
\[V = E_q[f(1)]\]
In other words, the rational price $V$ determined by the arbitrage condition of the $(\phi, \psi)$ portfolio was, in fact, the discounted expected price under probability $q$.

\subsection{The payoff for a forward contract with a delivery price of k}
The payoff for a forward contract with a delivery price of $k$ is:
\[f(2) = s_2 - k\]
\[f(3) = s_3 - k\]
Let's find the correct delivery price $k$. In this transaction, we don't need to hold the portfolio at $t=0$ (we borrow a bond and buy a stock at $t=0$, then get money by delivering the stock to the counterparty at maturity, and finally repay the debt made at $t=0$ with that money). Since $V=0$, we have:
\[V = \frac{f(3) - f(2)}{s_3 - s_2} s_1 + \exp(-r\delta t) \left\{ f(3) - s_3 \frac{f(3) - f(2)}{s_3 - s_2} \right\} \]
\[
	= \frac{(s_3-k) - (s_2-k)}{s_3 - s_2} s_1 + \exp(-r\delta t) \left\{ (s_3-k) - s_3 \frac{(s_3-k) - (s_2-k)}{s_3 - s_2} \right\}\]
\[= \frac{s_3 - s_2}{s_3 - s_2} s_1 + \exp(-r\delta t) \left\{ s_3 - k - s_3 \frac{s_3 - s_2}{s_3 - s_2} \right\} = s_1 + \exp(-r\delta t) (s_3 - k - s_3) = s_1 - k \exp(-r\delta t)\]
Therefore, when $V=0$, we get $k = s_1 \exp(r\delta t)$.

\begin{thebibliography}{9}
	\bibitem{BaxterRennie}
	Financial Calculus - An Introduction to Derivative Pricing - Martin Baxter, Andrew Rennie
\end{thebibliography}

\end{document}
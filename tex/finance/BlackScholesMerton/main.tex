\documentclass[uplatex]{jsarticle}
\usepackage[english]{babel}
\usepackage[letterpaper,top=2cm,bottom=2cm,left=3cm,right=3cm,marginparwidth=1.75cm]{geometry}
\usepackage{amsmath, amssymb}
\usepackage{graphicx}
\usepackage{here}

\title{The Black-Scholes-Merton Equation}

\author{
Okada Masaru
}

\begin{document}
\maketitle

\begin{abstract}
    This document provides a detailed derivation of the Black-Scholes-Merton differential equation and the corresponding European option pricing formula. It explains the core concepts of constructing a risk-free portfolio and demonstrates the mathematical steps using Ito's lemma and no-arbitrage conditions.
\end{abstract}

\section{Underlying Concepts of the Black-Scholes-Merton Differential Equation}

The Black-Scholes-Merton differential equation is the equation satisfied by the price of a derivative on a non-dividend-paying stock.

The derivation of the Black-Scholes-Merton differential equation proceeds with an argument that, by considering a portfolio consisting of a derivative and its underlying asset (a stock), the portfolio's return must equal the risk-free interest rate $r$ in the absence of arbitrage opportunities.

Why is it possible to construct a risk-free portfolio from a derivative and its underlying asset? The reason is that price fluctuations of both the derivative and the underlying asset are influenced by the same source of uncertainty. In other words, over a short period, the price of a derivative and its underlying asset are perfectly correlated. This is because the same Wiener process appears in both of their expressions. Since the same variable is present, we can eliminate the Wiener process term from the portfolio expression (or the derivative's price expression) using a method similar to elimination or substitution for systems of two-variable linear equations taught in junior high school.

Unlike the binomial model, the Black-Scholes-Merton model is only risk-free over an infinitesimal time period. This is the crucial difference between the two models. In the binomial model, a risk-free state can be constructed at each point in time. In the Black-Scholes-Merton model, a risk-free portfolio can only be constructed instantaneously, over an infinitesimal time $dt$.

\bigskip

For example, suppose the relationship between an infinitesimal change in stock price $dS$ at a certain point in time and the resulting infinitesimal change in the price of a European call option $dc$ is given by
$$
	dc = 0.4 dS
$$
In this case, a risk-free portfolio can be constructed from:
\begin{enumerate}
	\item a long position of 0.4 units of the stock
	\item a short position of 1 unit of the call option
\end{enumerate}

If, for instance, the stock price rises by 10 cents in the next instant, the call option's price will rise by 4 cents. The profit from the stock is $0.4 \times 10 = 4$ cents, and the loss from the short call option is also 4 cents, making the profit and loss of this portfolio zero.

Now, let's say some time passes, and after two weeks, for example, the relationship changes to:
$$
	dc = 0.5 dS
$$
At this point, a risk-free portfolio can be constructed from:
\begin{enumerate}
	\item a long position of 0.5 units of the stock
	\item a short position of 1 unit of the call option
\end{enumerate}
This means that to make the portfolio that was risk-free when $dc = 0.4 dS$ risk-free again, an additional 0.1 $(=0.5-0.4)$ units of the stock must be purchased for every unit of the call option sold.

Even if rebalancing is necessary, the return of a risk-free portfolio must be equal to the risk-free interest rate over any infinitesimal time period.

\section{Deriving the Black-Scholes-Merton Differential Equation}

\subsection{Deriving the Risk-Free Portfolio Using Ito's Lemma}

We start from the following equation, which has already been shown through Ito's lemma:
$$
	df = \dfrac{\partial f}{\partial S} \sigma S dz + \left( \dfrac{\partial f}{\partial S} \mu S + \dfrac{\partial f}{\partial t} + \dfrac{1}{2} \dfrac{\partial^{2} f}{\partial S^{2}} \sigma^{2} S^{2} \right) dt
$$
Here, $f=f(S,t)$ is a derivative whose underlying asset is $S$, which depends on time $t$. The underlying asset $S$ is a stock price that follows the process:
$$
	dS = \mu S dt + \sigma S dz
$$
Here, $dz$ is a Wiener process and can be expressed as $dz = \varepsilon \sqrt{dt}$ using a random sample $\varepsilon$ from a standard normal distribution. $\mu$ is the drift and $\sigma$ is the volatility (standard deviation).

\bigskip

The stochastic component of the derivative $f$ and its underlying asset $S$, the Wiener process $dz$, is the same. Using this fact, we can eliminate the Wiener process component by constructing a portfolio $\Pi$ with a short position of 1 unit of the derivative and a long position of $\dfrac{\partial f}{\partial S}$ units of the stock.

Specifically, we construct the following portfolio $\Pi$:
\begin{align}
	\Pi & = - f + \dfrac{\partial f}{\partial S} S
\end{align}
The infinitesimal change in this portfolio is:
\begin{align}
	d \Pi & = - df + \dfrac{\partial f}{\partial S} dS\\
	& = - \dfrac{\partial f}{\partial S} \sigma S dz - \left( \dfrac{\partial f}{\partial S} \mu S + \dfrac{\partial f}{\partial t} + \dfrac{1}{2} \dfrac{\partial^{2} f}{\partial S^{2}} \sigma^{2} S^{2} \right) dt + \dfrac{\partial f}{\partial S} (\mu S dt + \sigma S dz) \\
	& = \left( - \dfrac{\partial f}{\partial t} - \dfrac{1}{2} \dfrac{\partial^{2} f}{\partial S^{2}} \sigma^{2} S^{2} \right) dt
\end{align}
This expression does not contain the Wiener process $dz$. This means that the portfolio $\Pi$ has a drift (the coefficient of $dt$) of
$$
	- \dfrac{\partial f}{\partial t} - \dfrac{1}{2} \dfrac{\partial^{2} f}{\partial S^{2}} \sigma^{2} S^{2}
$$
and a volatility (the coefficient of $dz$) of
$$
	0
$$
Since the volatility is zero, by definition, this is a risk-free portfolio.

\subsection{Deriving the Risk-Free Portfolio Under No-Arbitrage Conditions}

We now impose the condition that the return of this portfolio must be equal to the return of a risk-free security.

As a quick review, let's summarize the no-arbitrage conditions:
\begin{itemize}
	\item If the portfolio's return is higher than the risk-free security's return: one could borrow money to buy the portfolio, making infinite risk-free profit.
	\item If the portfolio's return is lower than the risk-free security's return: one could sell the portfolio, invest the proceeds in the risk-free security, and make infinite risk-free profit.
\end{itemize}
Naturally, it's not possible to make infinite risk-free profit in a real market. (If it were, everyone would become infinitely wealthy.)

By contradiction, the risk-free rate of return must be equal to the return of the risk-free security (the risk-free interest rate).

From the above, if the risk-free interest rate is $r$, we know that:
\begin{align}
	d \Pi & = r \Pi dt \\
	& = r \left( - f + \dfrac{\partial f}{\partial S} S \right) dt
\end{align}
must be satisfied.

\subsection{The Black-Scholes-Merton Differential Equation}

We equate the $d \Pi$ derived from Ito's lemma with the $d \Pi$ derived from the no-arbitrage condition.
\begin{align}
	\left( - \dfrac{\partial f}{\partial t} - \dfrac{1}{2} \dfrac{\partial^{2} f}{\partial S^{2}} \sigma^{2} S^{2} \right) dt & = r \left( - f + \dfrac{\partial f}{\partial S} S \right) dt
\end{align}
Dividing both sides by $dt$ and rearranging the terms gives us the Black-Scholes-Merton differential equation:
\begin{align}
	\dfrac{\partial f}{\partial t} + rS \dfrac{\partial f}{\partial S} + \dfrac{1}{2} \sigma^{2} S^{2} \dfrac{\partial^{2} f}{\partial S^{2}} & = rf
\end{align}
This differential equation has many solutions. The solution $f$ to this equation represents any derivative of the underlying asset $S$. The specific type of derivative is determined by the boundary conditions of the differential equation. For example, to find the price of a European call option, we apply the boundary condition that at time $t=T$:
$$
	f = \max (S-K,0)
$$

\bigskip

The risk-free portfolio $\Pi$ is only risk-free over an infinitesimal time period $dt$. As $t$ changes, $S$ changes, and consequently, $\dfrac{\partial f}{\partial S}$ also changes. To keep $\Pi$ in a risk-free state, the ratio of the holdings of the derivative $f$ and the underlying asset $S$ must be continuously adjusted.

\bigskip

Any function that is a solution to the Black-Scholes-Merton differential equation represents a tradable derivative that does not create an arbitrage opportunity. Conversely, any function that is not a solution to the Black-Scholes-Merton differential equation is not tradable if no arbitrage opportunities exist.

A simple counterexample is the price $e^{S}$. This price is not a solution to the Black-Scholes-Merton differential equation. (It's easy to see this by substituting $f = e^{S}$ into the differential equation and comparing both sides.) Therefore, it's not the price of any derivative dependent on the stock price $S$. If a product with a price of always $e^{S}$ existed, it would create an arbitrage opportunity.

Another example is
$$
	\dfrac{\exp{[(\sigma^{2} - 2r)(T-t)]}}{S}
$$
Although it looks like a strange function, this function actually satisfies the Black-Scholes-Merton differential equation, so it is theoretically a tradable product. (This price is, in fact, the price of a derivative with a payoff of $1/S_{T}$ at time $T$.)

\section{The Black-Scholes-Merton Call Option Pricing Formula}

\subsection{Main Result}

If $V$ follows a lognormal distribution and the standard deviation of $\log(V)$ is $\sigma$, the expected value of $\max(V-K,0)$ is:
\begin{align}
	E \left[ \max(V-K,0) \right] & = E(V) \Phi (d_{+}) - K \Phi (d_{-})
\end{align}
where $d_{+}$ and $d_{-}$ satisfy:

\begin{align}
	d_{+} & = \dfrac{ \log \frac{E(V)}{K} + \frac{\sigma^{2}}{2} }{ \sigma } \\
	d_{-} & = \dfrac{ \log \frac{E(V)}{K} - \frac{\sigma^{2}}{2} }{ \sigma }
\end{align}

\subsection{Confirmation}

\subsubsection{Useful Facts About Lognormal Distributions}

The random variable $V$ follows a lognormal distribution. This means that $X = \log(V)$ follows a normal distribution. In other words, the probability density function $f(X)$ of $X$ is a normal distribution, and it satisfies:

\begin{align}
	f(X) & = \dfrac{1}{\sqrt{2 \pi} \sigma } \exp \left( - \dfrac{(X - \mu )^{2}}{2 \sigma^{2}} \right)
\end{align}

Therefore, if the probability density function of $V$ is $k(V)$, since
$$
	\dfrac{dX}{dV} = \dfrac{1}{V}
$$
we have:

\begin{align}
	k(V) & = \dfrac{1}{\sqrt{2 \pi} \sigma V} \exp \left( - \dfrac{(\log{(V)}- \mu )^{2}}{2 \sigma^{2}} \right)
\end{align}
Let's consider the $n$-th moment of $V$.

\begin{align}
	\int^{\infty}_{0} V^{n} k(V) dV
\end{align}

By substituting $V = \exp(X)$, we get:

\begin{align}
	 & \int^{\infty}_{0} V^{n} k(V) dV\\
	 & = \int^{\infty}_{-\infty} \dfrac{\exp(nX)}{\sqrt{2 \pi} \sigma V} \exp \left( - \dfrac{(\log{(V)}- \mu )^{2}}{2 \sigma^{2}} \right) (VdX)\\
	 & = \int^{\infty}_{-\infty} \dfrac{\exp(nX)}{\sqrt{2 \pi} \sigma} \exp \left( - \dfrac{(X- \mu )^{2}}{2 \sigma^{2}} \right) dX \\
	 & = \exp(n \mu + \dfrac{1}{2} n^{2} \sigma^{2} ) \int^{\infty}_{-\infty} \dfrac{1}{\sqrt{2 \pi} \sigma } \exp \left( - \dfrac{(X- \mu -n \sigma^{2})^{2}}{2 \sigma^{2}} \right) dX \\
	 & = \exp(n \mu + \dfrac{1}{2} n^{2} \sigma^{2} )
\end{align}

When $n=1$, the first moment is the expected value:
$$
	E(V) = \exp(\mu + \dfrac{1}{2} \sigma^{2} )
$$
When $n=2$, the second moment is:
$$
	E(V^{2}) = \exp(2\mu + 2 \sigma^{2} )
$$
From this, the variance is:

\begin{align}
	E(V^{2}) - E(V)^{2} & = \exp(2\mu + 2 \sigma^{2} ) - \left[ \exp(\mu + \dfrac{1}{2} \sigma^{2} ) \right]^{2} \\
	& = \exp(2 \mu + \sigma^{2} ) \left[ \exp( \sigma^{2}) -1 \right]
\end{align}

\subsubsection{European Call Option Pricing Formula}

Let $g(V)$ be the probability density function of $V$. Then,

\begin{align}
	E \left[ \max(V-K,0) \right] & = \int^{\infty}_{-\infty} \max(V-K,0) g(V) dV \\
	 & = \int^{\infty}_{K} (V-K) g(V) dV
\end{align}

The mean (expected value) $\mu$ of the log-normally distributed variable $V = \exp(X)$ is, from the previous section:

\begin{align}
	E(V) & = \exp( \mu + \dfrac{\sigma^{2}}{2} ) \\
	\mu& = \log[E(V)] - \dfrac{\sigma^{2}}{2}
\end{align}

Using this $\mu$, we define a new variable:

\begin{align}
	Q & = \dfrac{\log(V) - \mu}{\sigma}
\end{align}

This random variable $Q$ follows a normal distribution with a standard deviation of 1.0. This means that if the density function of $Q$ is $h(Q)$, then:

\begin{align}
	h(Q) & = \dfrac{ 1 }{ \sqrt{2 \pi} } e^{ - Q^{2} / 2 }
\end{align}

Using this variable, we transform from $V$ to $Q$.

\begin{align}
	V & = \exp(\sigma Q + \mu )
\end{align}

Noting that the integration interval $V: [K , \infty)$ becomes $Q: [ \dfrac{\log(K) - \mu}{\sigma} , \infty)$, we get:

\begin{align}
	E \left[ \max(V-K,0) \right] & = \int^{\infty}_{K} (V-K) g(V) dV \\
	 & = \int^{\infty}_{\frac{\log(K) - \mu}{\sigma}} ( e^{\sigma Q + \mu } -K) h(Q) dQ\\
	 & = \int^{\infty}_{\frac{\log(K) - \mu}{\sigma}} e^{\sigma Q + \mu } h(Q) dQ - K \int^{\infty}_{\frac{\log(K) - \mu}{\sigma}} h(Q) dQ
\end{align}

Now, let's transform the integrand of the first term by completing the square in the exponent:

\begin{align}
	e^{\sigma Q + \mu } h(Q) & = \dfrac{1}{\sqrt{2 \pi}} \exp(\sigma Q + \mu ) \exp(-\dfrac{1}{2} Q^{2}) \\
	 & = \dfrac{1}{\sqrt{2 \pi}} \exp( -\dfrac{ Q^{2} - 2 \sigma Q - 2 \mu }{2}) \\
	 & = \dfrac{1}{\sqrt{2 \pi}} \exp( -\dfrac{ (Q - \sigma)^{2} - 2 \mu - \sigma^{2} }{2})\\
	 & = \exp( \mu + \dfrac{\sigma^{2} }{2}) \dfrac{1}{\sqrt{2 \pi}} \exp( -\dfrac{ (Q - \sigma)^{2} }{2}) \\
	 & = \exp( \mu + \dfrac{\sigma^{2} }{2}) h(Q - \sigma)
\end{align}

Therefore,

\begin{align}
	E \left[ \max(V-K,0) \right] & = \int^{\infty}_{\frac{\log(K) - \mu}{\sigma}} e^{\sigma Q + \mu } h(Q) dQ - K \int^{\infty}_{\frac{\log(K) - \mu}{\sigma}} h(Q) dQ\\
	 & = \exp( \mu + \dfrac{\sigma^{2} }{2}) \int^{\infty}_{\frac{\log(K) - \mu}{\sigma}} h(Q - \sigma) dQ - K \int^{\infty}_{\frac{\log(K) - \mu}{\sigma}} h(Q) dQ
\end{align}

Let's evaluate the second integral.

Let $\Phi (x)$ be the cumulative density function of the standard normal distribution, such that:

\begin{align}
	\int^{x}_{-\infty} h(x) dx & = \Phi (x)
\end{align}

Using this, the second integral becomes:

\begin{align}
	K \int^{\infty}_{\frac{\log K - \mu}{\sigma}} h(Q) dQ & = K \int_{-\frac{\log K - \mu}{\sigma}}^{-\infty} h(-Q) d(-Q)\\
	& = K \int^{\frac{ \mu - \log K }{\sigma}}_{-\infty} h(Q) dQ \\
	& = K \Phi \left( \dfrac{ \mu - \log K }{\sigma} \right) \\
	& = K \Phi \left( \dfrac{ \left( \log E(V) - \frac{\sigma^{2}}{2} \right) - \log K }{\sigma} \right) \\
	& = K \Phi \left( \frac{\log\frac{E(V)}{K} - \frac{\sigma^{2}}{2} }{\sigma} \right)
\end{align}

Similarly, we evaluate the integral of the first term. By substituting $Q-\sigma \to -Q$, we get:

\begin{align}
	\int^{\infty}_{\frac{\log(K) - \mu}{\sigma}} h(Q - \sigma) dQ & = \int^{-\infty}_{\sigma - \frac{\log(K) - \mu}{\sigma}} h(-Q) d(-Q) \\
	& = \int_{-\infty}^{\sigma - \frac{\log(K) - \mu}{\sigma}} h(Q) dQ \\
	& = \Phi \left( \sigma - \frac{\log(K) - \mu}{\sigma} \right)\\
	& = \Phi \left( \sigma - \frac{\log(K) - \left( \log E(V) - \frac{\sigma^{2}}{2} \right) }{\sigma} \right) \\
	& = \Phi \left( \frac{ \log \frac{E(V)}{K} + \frac{\sigma^{2}}{2} }{\sigma} \right)
\end{align}

From the above,

\begin{align}
	E \left[ \max(V-K,0) \right] & = \exp( \mu + \dfrac{\sigma^{2} }{2}) \int^{\infty}_{\frac{\log(K) - \mu}{\sigma}} h(Q - \sigma) dQ - K \int^{\infty}_{\frac{\log(K) - \mu}{\sigma}} h(Q) dQ\\
	 & = \exp( \mu + \dfrac{\sigma^{2} }{2}) \Phi \left( \frac{ \log \frac{E(V)}{K} + \frac{\sigma^{2}}{2} }{\sigma} \right) - K \Phi \left( \frac{ \log \frac{E(V)}{K} - \frac{\sigma^{2}}{2} }{\sigma} \right)
\end{align}

For ease of notation, let's substitute variables as follows:

\begin{align}
	d_{\pm} & = \dfrac{ \log \frac{E(V)}{K} \pm \frac{\sigma^{2}}{2} }{ \sigma }
\end{align}

This can be expressed simply as:

\begin{align}
	E \left[ \max(V-K,0) \right] & = E(V) \Phi (d_{+}) - K \Phi (d_{-})
\end{align}

\subsubsection{European Option Pricing Formula Based on Black-Scholes-Merton}

The price of a call option, $c$, is expressed in terms of the strike price $K$, the risk-free interest rate $r$, the stock price at time $t$, $S_{t}$, the expiration time $t=T$, the volatility $\sigma$, and the risk-neutral expectation $E( \cdot )$ as:

\begin{align}
	c & = e^{-rT} E \left[ \max(S_{T}-K,0) \right] \\
	& = e^{-rT} \left[ S_{0} e^{rT} \Phi (d_{1}) - K \Phi(d_{2}) \right] \\
	& = S_{0} \Phi (d_{+}) - e^{-rT} K \Phi(d_{-})
\end{align}

The discount factor appears only in the strike price term. This is because the stock price term's growth at the risk-free rate cancels out the discount factor, causing it to disappear.

Here, since $E(S_{T}) = S_{0} e^{rT}$ and the volatility of $\log(S_{T})$ is $\sigma \sqrt{T}$, we make the substitutions:

\begin{align}
	\sigma & \to \sigma \sqrt{T} \\
	E(V) & \to E(S_T)
\end{align}

Therefore, we find:

\begin{align}
	d_{\pm} & = \dfrac{ \log \frac{E(V)}{K} \pm \frac{\sigma^{2}}{2} }{ \sigma } \\
	& \to \dfrac{ \log \frac{E(S_{T})}{K} \pm \frac{(\sigma \sqrt{T})^{2}}{2} }{ (\sigma \sqrt{T}) } \\
	& = \dfrac{ \log \frac{ e^{rT} S_{0} }{K} \pm \frac{\sigma^{2}}{2}T }{ \sigma \sqrt{T} } \\
	& = \dfrac{ \log \frac{S_{0}}{K} + ( r \pm \frac{\sigma^{2}}{2})T }{ \sigma \sqrt{T} }
\end{align}


\end{document}
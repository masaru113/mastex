\documentclass[uplatex]{jsarticle}
\usepackage[english]{babel}
\usepackage[letterpaper,top=2cm,bottom=2cm,left=3cm,right=3cm,marginparwidth=1.75cm]{geometry}
\usepackage{amsmath, amssymb}
\usepackage{graphicx}
\usepackage{here}

\title{Notes on Black-Scholes-Merton Equation}

\author{
Masaru Okada
}

\begin{document}
\maketitle

\maketitle

\section{Underlying Concepts of the Black-Scholes-Merton Differential Equation}

The Black-Scholes-Merton differential equation is the equation satisfied by the price of a derivative on a non-dividend-paying stock.

The derivation of the Black-Scholes-Merton differential equation proceeds by considering a portfolio consisting of a derivative and its underlying asset (the stock). The argument posits that, in the absence of arbitrage opportunities, the return on this portfolio must be equal to the risk-free interest rate, $r$.

Why is it possible to construct a risk-free portfolio from a derivative and its underlying asset? This is because the price fluctuations of both the derivative and the underlying asset are subject to the same source of uncertainty. In other words, over a short period, the prices of a derivative and its underlying asset are perfectly correlated. This is because the same Wiener process appears in both of their expressions. Since they share a common variable, the Wiener process term in the portfolio's expression (or the derivative's price expression) can be eliminated using the same methods as solving a system of two linear equations with two variables, like the elimination or substitution methods you learn in middle school.

Unlike the binomial model, the Black-Scholes-Merton model becomes risk-free only over an infinitesimal time period. This is the crucial difference between the two models. In the binomial model, a risk-free state can be constructed at each discrete point in time. In the Black-Scholes-Merton model, a risk-free portfolio can only be constructed instantaneously over a microscopic time interval, $dt$.

\bigskip

For instance, consider a situation where the infinitesimal change in a stock's price, $dS$, at a specific moment in time is related to the infinitesimal change in a European call option's price, $dc$, by the following:
$$
	dc = 0.4 dS
$$
A risk-free portfolio can then be constructed by taking:
\begin{enumerate}
	\item A long position of 0.4 units of the stock.
	\item A short position of 1 unit of the call option.
\end{enumerate}

If the stock price were to rise by 10 cents in the next instant, the call option's price would increase by 4 cents. The profit from the stock position would be $0.4 \times 10 = 4$ cents, while the loss from the short call option position would also be 4 cents, resulting in a net profit/loss of zero for the portfolio.

As more time elapses, say two weeks later, the relationship changes to:
$$
	dc = 0.5 dS
$$
At this point, a new risk-free portfolio can be constructed by taking:
\begin{enumerate}
	\item A long position of 0.5 units of the stock.
	\item A short position of 1 unit of the call option.
\end{enumerate}
To rebalance the portfolio that was risk-free when $dc = 0.4 dS$, you would need to buy an additional 0.1 $(=0.5-0.4)$ units of the stock for every unit of the short call option position.

Even though this rebalancing is required, the return on a risk-free portfolio must equal the risk-free interest rate for any infinitesimal period.

\section{Derivation of the Black-Scholes-Merton Differential Equation}

\subsection{Deriving the Risk-Free Portfolio Using Ito's Lemma}

We begin with the following equation, already established through Ito's Lemma:
$$
	df = \dfrac{\partial f}{\partial S} \sigma S dz + \left( \dfrac{\partial f}{\partial S} \mu S + \dfrac{\partial f}{\partial t} + \dfrac{1}{2} \dfrac{\partial^{2} f}{\partial S^{2}} \sigma^{2} S^{2} \right) dt
$$
Here, $f=f(S,t)$ is a derivative whose underlying asset, $S$, depends on time, $t$. The underlying asset's price, $S$, follows the process:
$$
	dS = \mu S dt + \sigma S dz
$$
In this equation, $dz$ is a Wiener process, which can be expressed as $dz = \varepsilon \sqrt{dt}$ using a random draw $\varepsilon$ from a standard normal distribution. $\mu$ represents the drift, and $\sigma$ is the volatility (standard deviation).

\bigskip

The random fluctuation component, the Wiener process $dz$, is the same for both the derivative $f$ and its underlying asset $S$. Using this fact, we can eliminate the Wiener process component by constructing a portfolio $\Pi$ that is long $\dfrac{\partial f}{\partial S}$ units of the stock and short one unit of the derivative.

The portfolio $\Pi$ is thus defined as:
\begin{align}
	\Pi & = - f + \dfrac{\partial f}{\partial S} S
\end{align}
The infinitesimal change in this portfolio, $d \Pi$, is:
\begin{align}
	d \Pi & = - df + \dfrac{\partial f}{\partial S} dS                                                                                                                                                                                                                                               \\
	      & = - \left[ \dfrac{\partial f}{\partial S} \sigma S dz + \left( \dfrac{\partial f}{\partial S} \mu S + \dfrac{\partial f}{\partial t} + \dfrac{1}{2} \dfrac{\partial^{2} f}{\partial S^{2}} \sigma^{2} S^{2} \right) dt \right] + \dfrac{\partial f}{\partial S} (\mu S dt + \sigma S dz) \\
	      & = \left( - \dfrac{\partial f}{\partial t} - \dfrac{1}{2} \dfrac{\partial^{2} f}{\partial S^{2}} \sigma^{2} S^{2} \right) dt
\end{align}
This expression no longer contains the Wiener process $dz$. This means the portfolio $\Pi$ has a drift (the coefficient of $dt$) of:
$$
	- \dfrac{\partial f}{\partial t} - \dfrac{1}{2} \dfrac{\partial^{2} f}{\partial S^{2}} \sigma^{2} S^{2}
$$
and a volatility (the coefficient of $dz$) of:
$$
	0
$$
With a volatility of zero, this process is, by definition, a risk-free portfolio.

\subsection{Deriving the Risk-Free Portfolio Using the No-Arbitrage Condition}

We now impose the condition that the return on this portfolio must be equal to the return on a risk-free security.

As a quick review, the no-arbitrage condition states:
\begin{itemize}
	\item If the portfolio's return is higher than the risk-free rate: You could borrow money to buy the portfolio and generate unlimited, risk-free profit.
	\item If the portfolio's return is lower than the risk-free rate: You could sell the portfolio, use the proceeds to buy risk-free securities, and also generate unlimited, risk-free profit.
\end{itemize}
Naturally, you can't actually make unlimited, risk-free profit in the market. (If you could, everyone would be a billionaire!)

By contradiction, the risk-free return must be equal to the return on a risk-free security, which is the risk-free interest rate.

Therefore, if the risk-free interest rate is $r$, the following condition must hold:
\begin{align}
	d \Pi & = r \Pi dt                                                   \\
	      & = r \left( - f + \dfrac{\partial f}{\partial S} S \right) dt
\end{align}

\subsection{The Black-Scholes-Merton Differential Equation}

Now we equate the expression for $d \Pi$ derived from Ito's Lemma with the one derived from the no-arbitrage condition:
\begin{align}
	\left( - \dfrac{\partial f}{\partial t} - \dfrac{1}{2} \dfrac{\partial^{2} f}{\partial S^{2}} \sigma^{2} S^{2} \right) dt & = r \left( - f + \dfrac{\partial f}{\partial S} S \right) dt
\end{align}
Dividing both sides by $dt$, expanding, and rearranging the terms gives us the Black-Scholes-Merton differential equation:
\begin{align}
	\dfrac{\partial f}{\partial t} + rS \dfrac{\partial f}{\partial S} + \dfrac{1}{2} \sigma^{2} S^{2} \dfrac{\partial^{2} f}{\partial S^{2}} & = rf
\end{align}
This differential equation has many solutions. A solution $f$ to this equation represents any derivative of the underlying asset $S$. The specific derivative is determined by the boundary conditions of the differential equation. For example, to find the price of a European call option, we would impose the boundary condition at time $t=T$:
$$
	f = \max (S-K,0)
$$

\bigskip

The risk-free portfolio $\Pi$ is only risk-free over an infinitesimal time interval $dt$. As time $t$ changes, $S$ also changes, and consequently, so does $\dfrac{\partial f}{\partial S}$. To keep $\Pi$ in a risk-free state, the ratio of the derivative $f$ to the underlying asset $S$ must be constantly adjusted.

\bigskip

A function that is a solution to the Black-Scholes-Merton differential equation represents a tradable derivative that creates no arbitrage opportunities. Conversely, a function that is not a solution to the Black-Scholes-Merton differential equation is not tradable if no arbitrage opportunities exist.

A simple counterexample is the price $e^{S}$. This price is not a solution to the Black-Scholes-Merton differential equation. (You can easily verify this by substituting $f = e^{S}$ into the equation and comparing both sides.) Therefore, it's not the price of any derivative that depends on the stock price $S$. If a product whose price was always $e^{S}$ existed, it would create an arbitrage opportunity.

As another example, consider the function:
$$
	\dfrac{\exp{[(\sigma^{2} - 2r)(T-t)]}}{S}
$$
Although it looks strange at first, this function actually satisfies the Black-Scholes-Merton differential equation, so it is, in theory, a tradable product. (In fact, this is the price of a derivative that has a payoff of $1/S_{T}$ at time $T$.)

\section{Proof of the Black-Scholes-Merton Formula}

\subsection{Main Result}

If a random variable $V$ is log-normally distributed with the standard deviation of $\log(V)$ being $\sigma$, the expected value of $\max(V-K,0)$ is:
\begin{align}
	E \left[ \max(V-K,0) \right] & = E(V) \Phi (d_{+}) - K \Phi (d_{-})
\end{align}
where $d_{+}$ and $d_{-}$ are defined as:

\begin{align}
	d_{+} & = \dfrac{ \log \frac{E(V)}{K} + \frac{\sigma^{2}}{2} }{ \sigma } \\
	d_{-} & = \dfrac{ \log \frac{E(V)}{K} - \frac{\sigma^{2}}{2} }{ \sigma }
\end{align}


\subsection{Proof}

\subsubsection{Useful Facts About the Lognormal Distribution}

Let a random variable $V$ be log-normally distributed. This means that $X = \log(V)$ is normally distributed. In other words, the probability density function of $X$, $f(X)$, is a normal distribution:

\begin{align}
	f(X) & = \dfrac{1}{\sqrt{2 \pi} \sigma } \exp \left( - \dfrac{(X - \mu )^{2}}{2 \sigma^{2}} \right)
\end{align}

Therefore, if $k(V)$ is the probability density function of $V$, then from the change of variables formula,
$$
	\dfrac{dX}{dV} = \dfrac{1}{V}
$$
so,

\begin{align}
	k(V) & = \dfrac{1}{\sqrt{2 \pi} \sigma V} \exp \left( - \dfrac{(\log{(V)}- \mu )^{2}}{2 \sigma^{2}} \right)
\end{align}
Now, let's consider the $n$-th moment of $V$:

\begin{align}
	\int^{\infty}_{0} V^{n} k(V) dV
\end{align}

By substituting $V = \exp(X)$, we get:

\begin{align}
	 & \int^{\infty}_{0} V^{n} k(V) dV                                                                                                                                                  \\
	 & = \int^{\infty}_{-\infty} \dfrac{\exp(nX)}{\sqrt{2 \pi} \sigma V} \exp \left( - \dfrac{(\log{(V)}- \mu )^{2}}{2 \sigma^{2}} \right) (VdX)                                        \\
	 & = \int^{\infty}_{-\infty} \dfrac{\exp(nX)}{\sqrt{2 \pi} \sigma} \exp \left( - \dfrac{(X- \mu )^{2}}{2 \sigma^{2}} \right) dX                                                     \\
	 & = \exp(n \mu + \dfrac{1}{2} n^{2} \sigma^{2} ) \int^{\infty}_{-\infty} \dfrac{1}{\sqrt{2 \pi} \sigma } \exp \left( - \dfrac{(X- \mu -n \sigma^{2})^{2}}{2 \sigma^{2}} \right) dX \\
	 & = \exp(n \mu + \dfrac{1}{2} n^{2} \sigma^{2} )
\end{align}

For $n=1$, the first moment is the expected value:
$$
	E(V) = \exp(\mu + \dfrac{1}{2} \sigma^{2} )
$$
For $n=2$, the second moment is:
$$
	E(V^{2}) = \exp(2\mu + 2 \sigma^{2} )
$$
Therefore, the variance is:

\begin{align}
	E(V^{2}) - E(V)^{2} & = \exp(2\mu + 2 \sigma^{2} ) - \left[ \exp(\mu + \dfrac{1}{2} \sigma^{2} ) \right]^{2} \\
	                    & = \exp(2 \mu + \sigma^{2} ) \left[ \exp( \sigma^{2}) -1 \right]
\end{align}


\subsubsection{European Call Option Pricing Formula}

Let $g(V)$ be the probability density function of $V$. Then,
\begin{align}
	E \left[ \max(V-K,0) \right] & = \int^{\infty}_{-\infty} \max(V-K,0) g(V) dV \\
	                             & = \int^{\infty}_{K} (V-K) g(V) dV
\end{align}

From the previous section, the mean (expected value) $\mu$ of the log-normally distributed variable $V = \exp(X)$ is:

\begin{align}
	E(V) & = \exp( \mu + \dfrac{\sigma^{2}}{2} ) \\
	\mu  & = \log[E(V)] - \dfrac{\sigma^{2}}{2}
\end{align}

Using this $\mu$, we define a new variable:

\begin{align}
	Q & = \dfrac{\log(V) - \mu}{\sigma}
\end{align}

This random variable $Q$ follows a normal distribution with a standard deviation of 1.0. If $h(Q)$ is the density function of $Q$, then:

\begin{align}
	h(Q) & = \dfrac{ 1 }{ \sqrt{2 \pi} } e^{ - Q^{2} / 2 }
\end{align}

Using this variable, we change variables from $V$ to $Q$.
\begin{align}
	V & = \exp(\sigma Q + \mu )
\end{align}
Noting that the integration range changes from $V: [K , \infty)$ to $Q: [ \dfrac{\log(K) - \mu}{\sigma} , \infty)$, we get:

\begin{align}
	E \left[ \max(V-K,0) \right] & = \int^{\infty}_{K} (V-K) g(V) dV                                                                                                   \\
	                             & = \int^{\infty}_{\frac{\log(K) - \mu}{\sigma}} ( e^{\sigma Q + \mu } -K) h(Q) dQ                                                    \\
	                             & = \int^{\infty}_{\frac{\log(K) - \mu}{\sigma}} e^{\sigma Q + \mu } h(Q) dQ - K \int^{\infty}_{\frac{\log(K) - \mu}{\sigma}} h(Q) dQ
\end{align}

Let's transform the integrand of the first term. We complete the square of the exponent:

\begin{align}
	e^{\sigma Q + \mu } h(Q) & = \dfrac{1}{\sqrt{2 \pi}} \exp(\sigma Q + \mu ) \exp(-\dfrac{1}{2} Q^{2})                           \\
	                         & = \dfrac{1}{\sqrt{2 \pi}} \exp( -\dfrac{ Q^{2} - 2 \sigma Q - 2 \mu }{2})                           \\
	                         & = \dfrac{1}{\sqrt{2 \pi}} \exp( -\dfrac{ (Q - \sigma)^{2} - 2 \mu - \sigma^{2} }{2})                \\
	                         & = \exp( \mu + \dfrac{\sigma^{2} }{2}) \dfrac{1}{\sqrt{2 \pi}} \exp( -\dfrac{ (Q - \sigma)^{2} }{2}) \\
	                         & = \exp( \mu + \dfrac{\sigma^{2} }{2}) h(Q - \sigma)
\end{align}

Therefore,
\begin{align}
	E \left[ \max(V-K,0) \right] & = \int^{\infty}_{\frac{\log(K) - \mu}{\sigma}} e^{\sigma Q + \mu } h(Q) dQ - K \int^{\infty}_{\frac{\log(K) - \mu}{\sigma}} h(Q) dQ                          \\
	                             & = \exp( \mu + \dfrac{\sigma^{2} }{2}) \int^{\infty}_{\frac{\log(K) - \mu}{\sigma}} h(Q - \sigma) dQ - K \int^{\infty}_{\frac{\log(K) - \mu}{\sigma}} h(Q) dQ
\end{align}

Let's evaluate the integral of the second term.

Let $\Phi (x)$ be the cumulative density function of the standard normal distribution. That is,
\begin{align}
	\int^{x}_{-\infty} h(x) dx & = \Phi (x)
\end{align}
Using this, the integral of the second term becomes:

\begin{align}
	K \int^{\infty}_{\frac{\log K - \mu}{\sigma}} h(Q) dQ & = K \int_{-\frac{\log K - \mu}{\sigma}}^{-\infty} h(-Q) d(-Q)                                      \\
	                                                      & = K \int^{\frac{ \mu - \log K }{\sigma}}_{-\infty} h(Q) dQ                                         \\
	                                                      & = K \Phi \left( \dfrac{ \mu - \log K }{\sigma} \right)                                             \\
	                                                      & = K \Phi \left( \dfrac{ \left( \log E(V) - \frac{\sigma^{2}}{2} \right) - \log K }{\sigma} \right) \\
	                                                      & = K \Phi \left( \frac{\log\frac{E(V)}{K} - \frac{\sigma^{2}}{2} }{\sigma} \right)
\end{align}

We evaluate the first term's integral in a similar way, performing the substitution $Q-\sigma \to -Q$:

\begin{align}
	\int^{\infty}_{\frac{\log(K) - \mu}{\sigma}} h(Q - \sigma) dQ & = \int^{-\infty}_{\sigma - \frac{\log(K) - \mu}{\sigma}} h(-Q) d(-Q)                                     \\
	                                                              & = \int_{-\infty}^{\sigma - \frac{\log(K) - \mu}{\sigma}} h(Q) dQ                                         \\
	                                                              & = \Phi \left( \sigma - \frac{\log(K) - \mu}{\sigma} \right)                                              \\
	                                                              & = \Phi \left( \sigma - \frac{\log(K) - \left( \log E(V) - \frac{\sigma^{2}}{2} \right) }{\sigma} \right) \\
	                                                              & = \Phi \left( \frac{ \log \frac{E(V)}{K} + \frac{\sigma^{2}}{2} }{\sigma} \right)
\end{align}


Putting it all together, we have:

\begin{align}
	E \left[ \max(V-K,0) \right] & = \exp( \mu + \dfrac{\sigma^{2} }{2}) \int^{\infty}_{\frac{\log(K) - \mu}{\sigma}} h(Q - \sigma) dQ - K \int^{\infty}_{\frac{\log(K) - \mu}{\sigma}} h(Q) dQ                                              \\
	                             & = \exp( \mu + \dfrac{\sigma^{2} }{2}) \Phi \left( \frac{ \log \frac{E(V)}{K} + \frac{\sigma^{2}}{2} }{\sigma} \right) - K \Phi \left( \frac{ \log \frac{E(V)}{K} - \frac{\sigma^{2}}{2} }{\sigma} \right)
\end{align}

For ease of notation, let's redefine the variables:

\begin{align}
	d_{\pm} & = \dfrac{ \log \frac{E(V)}{K} \pm \frac{\sigma^{2}}{2} }{ \sigma }
\end{align}

The formula can be expressed concisely as:

\begin{align}
	E \left[ \max(V-K,0) \right] & = E(V) \Phi (d_{+}) - K \Phi (d_{-})
\end{align}


\subsubsection{European Option Pricing Formula Based on Black-Scholes-Merton}

The price of a call option, $c$, is given by the present value of the expected payoff at maturity. We can express this using the strike price $K$, risk-free rate $r$, stock price at time $t$ of $S_{t}$, maturity at time $t=T$, volatility $\sigma$, and the expectation in the risk-neutral world $E(\cdot)$.

\begin{align}
	c & = e^{-rT} E \left[ \max(S_{T}-K,0) \right]                         \\
	  & = e^{-rT} \left[ S_{0} e^{rT} \Phi (d_{1}) - K \Phi(d_{2}) \right] \\
	  & = S_{0} \Phi (d_{+}) - e^{-rT} K \Phi(d_{-})
\end{align}

Note that the discount factor only appears in the strike price term. This is because the stock price term grows at the risk-free rate, which cancels out the discount factor, causing it to disappear from the stock price term.

Here, $E(S_{T}) = S_{0} e^{rT}$, and the volatility of $\log(S_{T})$ is $\sigma \sqrt{T}$.
Thus, we make the substitutions:

\begin{align}
	\sigma & \to \sigma \sqrt{T} \\
	E(V)   & \to E(S_T)
\end{align}

This gives us the final expression for $d_{\pm}$:

\begin{align}
	d_{\pm} & = \dfrac{ \log \frac{E(V)}{K} \pm \frac{\sigma^{2}}{2} }{ \sigma }                             \\
	        & \to \dfrac{ \log \frac{E(S_{T})}{K} \pm \frac{(\sigma \sqrt{T})^{2}}{2} }{ (\sigma \sqrt{T}) } \\
	        & = \dfrac{ \log \frac{ e^{rT} S_{0} }{K} \pm \frac{\sigma^{2}}{2}T }{ \sigma \sqrt{T} }         \\
	        & = \dfrac{ \log \frac{S_{0}}{K} + ( r \pm \frac{\sigma^{2}}{2})T }{ \sigma \sqrt{T} }
\end{align}


\end{document}
\documentclass[uplatex,a4j,12pt,dvipdfmx]{jsarticle}
\usepackage{amsmath,amsthm,amssymb,bm,color,enumitem,mathrsfs,url,epic,eepic,ascmac,ulem,here}
\usepackage[letterpaper,top=2cm,bottom=2cm,left=3cm,right=3cm,marginparwidth=1.75cm]{geometry}
\usepackage[english]{babel}
\usepackage[dvipdfm]{graphicx}
\usepackage[hypertex]{hyperref}
\title{
マルチンゲール表現定理
}
\author{岡田 大(Okada Masaru)}

\date{ \today }
\begin{document}

\maketitle
\begin{abstract}
	Financial Calculus - An Introduction to Derivative Pricing - Martin Baxter, Andrew Rennie の3章の自主ゼミのノート。2020年5月20日に書いたもの。
\end{abstract}

\section{ここまでの現状と課題の整理}

2章3節で離散過程についてマルチンゲールとはどのような過程かを見た。

${}$

%
\begin{itembox}[l]{(復習)2.3節「図による定義」の中の定義(vii)}
	過程$S$は、もしすべての$i \leq j$に対して、
	$$
		{\bf E}_{\mathbb{P}}( S_{j} | \mathcal{F}_{i} ) = S_{i}
	$$
	であれば、
	確率測度$\mathbb{P}$と
	フィルトレーション$\mathcal{F}_{i}$
	に関してマルチンゲールであるという。
\end{itembox}
%

${}$

連続時間の過程においても同じことが成立することを見ていく。
\section{(連続時間過程の)マルチンゲールの条件}

確率過程$M_{t}$が測度$\mathbb{P}$に関してマルチンゲールであるとは、
以下の条件を満たすことである。

${}$

%
\begin{itembox}[l]{マルチンゲールの条件}
	%
	\begin{enumerate}
		\item 全ての$t$に対して${\bf E}_{\mathbb{P}}( | M_{t} | ) < \infty$
		\item $s(<t)$に対して${\bf E}_{\mathbb{P}}( M_{t} | \mathcal{F}_{s} ) = M_{s}$
	\end{enumerate}
	%
\end{itembox}
%

${}$

2つ目の条件が特に重要なマルチンゲールの条件になる。

離散過程と同様に、将来の期待値が現在と一致するということが表現されている。

実感を掴むためにテキストでは例が3つ挙げられている。
\subsection{例1: 定数過程}

すべての時刻$t$で$S_{t}=c$(一定)となる過程は任意の測度でマルチンゲールになっている。

ある将来時点$s,t$(ただし$s(<t)$とする)に対して、
$S_{t}=S_{s}=c$
なので、
%
\begin{eqnarray*}
	c
	&=&
	{\bf E}_{\mathbb{P}}( S_{t} | \mathcal{F}_{s} )
	\\ &=&
	{\bf E}_{\mathbb{P}}( S_{s} | \mathcal{F}_{s} )
	\\ &=&
	S_{s}
\end{eqnarray*}
%
が任意の測度$\mathbb{P}$に対して成り立つ。
\subsection{例2: 測度$\mathbb{P}$の下での$\mathbb{P}$-Brown運動}

$\mathbb{P}$-Brown運動は$\mathbb{P}$-マルチンゲールであることを確認する。

${}$

時点$s,t$($s<t$)、
$\mathbb{P}$-Brown運動を$W_{t}$として、
%
\begin{eqnarray*}
	{\bf E}_{\mathbb{P}}( W_{t} | \mathcal{F}_{s} )
	&=&
	{\bf E}_{\mathbb{P}}( W_{t} + W_{s} - W_{s} | \mathcal{F}_{s} )
	\\ &=&
	{\bf E}_{\mathbb{P}}( W_{s} | \mathcal{F}_{s} )
	+
	{\bf E}_{\mathbb{P}}( W_{t} - W_{s} | \mathcal{F}_{s} )
	\\ &=&
	W_{s}
	+
	0
\end{eqnarray*}
%
ここで
$
	{\bf E}_{\mathbb{P}}( W_{t} - W_{s} | \mathcal{F}_{s} )
	=0
$
については、Brown運動の性質;
「$W_{t} - W_{s}$は$\mathcal{F}_{s}$とは独立であり、
$\mathbb{P}$の下で
$N(0,t-s)$の分布をする。」
を用いた。

ゆえに
$W_{t}$はマルチンゲールである。

\subsection{例3: 積み重ねの法則:測度$\mathbb{P}$の下での条件付き期待値の過程}

満期$T$でペイオフが確定する契約$X$に対して、
過程
$
	N_{t} = {\bf E}_{\mathbb{P}}( X | \mathcal{F}_{t} )
$
は$\mathbb{P}$-マルチンゲールになることを確認する。

${}$

これを示すために時点$s,t$($s \leq t$)で成り立つ積み重ねの法則
$$
	{\bf E}_{\mathbb{P}}
	\Big(
	{\bf E}_{\mathbb{P}}
	\left(
	X | \mathcal{F}_{t}
	\right)
	\Big| \mathcal{F}_{s} \Big)
	\ = \
	{\bf E}_{\mathbb{P}}( X | \mathcal{F}_{s} )
$$
を利用する。

これは時刻$t$までの履歴で条件付けした後に時刻$s$までの履歴で条件付けした期待値は、
最初から$s$までの履歴で条件付けした期待値と等しくなるという離散過程で出てきたものと同じものである。

これを利用すれば
%
\begin{eqnarray*}
	{\bf E}_{\mathbb{P}}( N_{t} | \mathcal{F}_{s} )
	&=&
	{\bf E}_{\mathbb{P}}
	\Big(
	{\bf E}_{\mathbb{P}}
	\left(
	X | \mathcal{F}_{t}
	\right)
	\Big| \mathcal{F}_{s} \Big)
	\\ &=&
	{\bf E}_{\mathbb{P}}( X | \mathcal{F}_{s} )
	\\ &=&
	N_{s}
\end{eqnarray*}
%
となり、$N_{t}$がマルチンゲールであることが示すことができる。

\section{練習問題3.10}

\subsection{問題}

$W_{t}$は$\mathbb{P}$-Brown運動であるとする。
確率過程$X_{t}=W_{t} + \gamma t$は$\gamma=0$のときに$\mathbb{P}$-マルチンゲールとなることを示せ。

\subsection{解答}

測度$\mathbb{P}$、時点$s(<t)$における履歴$\mathcal{F}_{s}$の下で期待値を取ると
%
\begin{eqnarray*}
	{\bf E}_{\mathbb{P}}( X_{t} | \mathcal{F}_{s} )
	&=&
	{\bf E}_{\mathbb{P}}( W_{t} + \gamma t | \mathcal{F}_{s} )
	\\ &=&
	W_{s}+ \gamma t
	\\ &=&
	X_{s} + \gamma  ( t - s )
\end{eqnarray*}
%
となるので、$\gamma = 0$の場合にのみ$X_{t}$は$\mathbb{P}$-マルチンゲールになる。
\subsection{もう少し踏み込んでみる}

$X_{t} = \sigma W_{t} + \gamma t$のように正定数のボラティリティ$\sigma(>0)$を与えてみる。
このとき、
%
\begin{eqnarray*}
	{\bf E}_{\mathbb{P}}( X_{t} | \mathcal{F}_{s} )
	&=&
	{\bf E}_{\mathbb{P}}( \sigma W_{t} + \gamma t | \mathcal{F}_{s} )
	\\ &=&
	\sigma W_{s}+ \gamma t
	\\ &=&
	X_{s} + \gamma  ( t - s )
\end{eqnarray*}
%
この場合も同様にドリフト$\gamma = 0$でなければ$\mathbb{P}$-マルチンゲールにならない。

${}$

この後すぐに「マルチンゲールであることとドリフト項が無いことは同値」ということを見る。
ドリフトが定数でなく時間に依存する場合についてどういう条件が必要かが述べられる。
\subsection{(追記)Brown運動の2乗のマルチンゲール}
$\mathbb{P}$-Brown運動を$W_{t}$として$W_{t}^{2}$の
測度$\mathbb{P}$、時点$s(<t)$における履歴$\mathcal{F}_{s}$の下で期待値を取ると
%
\begin{eqnarray*}
	{\bf E}_{\mathbb{P}}( W_{t}^{2} | \mathcal{F}_{s} )
	&=&
	{\bf E}_{\mathbb{P}}[ \{ W_{s} + (W_{t} - W_{s}) \}^{2} | \mathcal{F}_{s} ]
	\\ &=&
	W_{s}^{2}
	\ + \
	2 W_{s}
		{\bf E}_{\mathbb{P}}( W_{t} - W_{s} | \mathcal{F}_{s} )
	\ + \
	{\bf E}_{\mathbb{P}}[ ( W_{t} - W_{s} )^{2} | \mathcal{F}_{s} ]
	\\ &=&
	W_{s}^{2}
	\ + \
	0
	\ + \
	(t-s)
\end{eqnarray*}
%
となるので、両辺から$t$を引くと
%
\begin{eqnarray*}
	{\bf E}_{\mathbb{P}}( W_{t}^{2} - t | \mathcal{F}_{s} )
	&=&
	W_{s}^{2} - s
\end{eqnarray*}
%
となって$W_{t}^{2} - t$はマルチンゲールになることがわかる。
($W_{t} - W_{s}$は$\mathcal{F}_{s}$とは独立で分散$(t-s)$を持つことを利用している。)

\subsection{(追記)定数ボラティリティの指数マルチンゲール}

$\mathbb{P}$-Brown運動を$W_{t}$として$\exp(\sigma W_{t})$の測度$\mathbb{P}$、時点$s(<t)$における履歴$\mathcal{F}_{s}$の下で期待値を取ると
%
\begin{eqnarray*}
	{\bf E}_{\mathbb{P}}[ e^{\sigma W_{t}} | \mathcal{F}_{s} ]
	&=&
	e^{\sigma W_{s}}
		{\bf E}_{\mathbb{P}}[ e^{\sigma (W_{t} -W_{s})} | \mathcal{F}_{s} ]
\end{eqnarray*}
%
ここで、前節の積率母関数の話を思い出すと、
($W_{t} - W_{s}$は$\mathcal{F}_{s}$とは独立で分散$(t-s)$を持つので)
%
\begin{eqnarray*}
	{\bf E}_{\mathbb{P}}[ e^{\sigma W_{t}} | \mathcal{F}_{s} ]
	&=&
	e^{\sigma W_{s}}
	\exp \left( 0 \times \sigma + \dfrac{1}{2} \sigma^{2} (t-s) \right)
	\\ &=&
	\exp \left( \dfrac{1}{2} \sigma^{2} t \right)
	\exp \left( \sigma W_{s} - \dfrac{1}{2} \sigma^{2} s \right)
\end{eqnarray*}
%
両辺に$\exp \left( - \dfrac{1}{2} \sigma^{2} t \right)$を掛けると
%
\begin{eqnarray*}
	{\bf E}_{\mathbb{P}}[ e^{ ( \sigma W_{t} - \frac{1}{2} \sigma^{2} t ) } | \mathcal{F}_{s} ]
	&=&
	e^{ ( \sigma W_{s} - \frac{1}{2} \sigma^{2} s ) }
\end{eqnarray*}
%
となるので、
$e^{ ( \sigma W_{t} - \frac{1}{2} \sigma^{2} t ) }$もマルチンゲールになる。
\section{マルチンゲール表現定理}
%
\begin{itembox}[l]{(復習)二項マルチンゲール表現定理}
	時点$i$における二項過程$M_{i}$が$\mathbb{Q}$-マルチンゲールであるとする。
	さらに別の過程$N_{i}$も$\mathbb{Q}$-マルチンゲールであるとき、
	ある可予測過程$\phi_{i}$が存在して、$N_{i}$は
	$$
		N_{i} \ = \ N_{0} + \sum_{k=1}^{i} \phi_{k} \Delta M_{k}
	$$
	で表される。

	($N_{i}$が上のように表現できるような可予測過程$\phi_{i}$が存在する。)
\end{itembox}
%

${}$

%
\begin{itembox}[l]{マルチンゲール表現定理}
	過程$M_{t}$が$\mathbb{Q}$-マルチンゲールであり、
	そのボラティリティ$\sigma_{t}$が常に0でないとする。
	さらに別の過程$N_{t}$も$\mathbb{Q}$-マルチンゲールであるとき、
	$$
		\int^{T}_{0} \phi_{t}^{2} \sigma_{t}^{2} d t < \infty
	$$
	を常に満たす可予測過程$\phi_{t}$が存在して、$N_{t}$は
	$$
		N_{t} \ = \ N_{0} + \int^{t}_{0} \phi_{s} d M_{s}
	$$
	で表される。
\end{itembox}
%

二項過程の定理の中の和の部分が積分に変わっただけで同じ定理である。

${}$

テキストで説明されている点
「
$M_{t}$が$\mathbb{Q}$-マルチンゲールとなるような測度$\mathbb{Q}$が存在するならば、
ほかのあらゆる$\mathbb{Q}$-マルチンゲールは$M_{t}$を用いて上の式のように表すことができる。
そして過程$\phi_{t}$は単にそれぞれボラティリティの比率となっている。
」
について考えてみる。

${}$

マルチンゲール表現定理を微分形で表現すると、
$d N_{t} = \phi_{t} dM_{t}$となる。

$M_{t}$は$\mathbb{Q}$-マルチンゲールなので
$\mathbb{Q}$-Brown運動$\tilde{W}_{t}$を用いて
$d M_{t} = \sigma^{(M)}_{t} d \tilde{W}_{t}$と書ける。
(ただし$M$のボラティリティ$\sigma^{(M)}_{t}$は$\sigma^{(M)}_{t}>0$とする。)
よって、これを用いると$dN_{t}$は
$dN_{t} = \phi_{t} \sigma^{(M)}_{t} d \tilde{W}_{t}$

一方で、$N_{t}$も$\mathbb{Q}$-マルチンゲールなので
$N$のボラティリティを$\sigma^{(N)}_{t}$と書くと、
$dN_{t} = \sigma_{t}^{(N)} d \tilde{W}_{t}$
と表すことができる。

ゆえに$dN_{t}$を$M$と$N$のボラティリティ$\sigma^{(M)}_{t},\sigma^{(N)}_{t}$を用いて表すと、
$
	\phi_{t} \sigma^{(M)}_{t} d \tilde{W}_{t}
	=
	\sigma_{t}^{(N)} d \tilde{W}_{t}
$
これを変形して
$$
	\phi_{t}
	\ = \
	\dfrac{\sigma_{t}^{(N)}}{\sigma^{(M)}_{t}}
$$
テキストの説明の通り、
「過程$\phi_{t}$は単にそれぞれボラティリティの比率となっている。」
ということが分かった。

\section{ドリフト無し}

練習問題3.10ですでに少し見たように、
ドリフトが無い過程はマルチンゲールである、
と言う為の条件がここでより詳細に述べられている。

${}$

$dX_{t}= \sigma_{t} dW_{t} + \mu_{t} dt$
を満たすような確率過程$X_{t}$は、
条件
$$
	{\bf E}
	\left[
		\left(
		\int^{t}_{0} \sigma_{s}^{2} ds
		\right)^{\frac{1}{2}}
		\right]
	\ < \
	\infty
$$
を満たすならば、
$X_{t}$がマルチンゲールであることと$\mu_{t} = 0$
であることは同値である。

${}$

上の条件を満たさない過程は局所マルチンゲールと呼ばれる。

\section{指数マルチンゲール}

ドリフトの無い幾何Brown運動
$ dX_{t} = \sigma_{t} X_{t} dW_{t} $
の場合、
$X_{t}$がマルチンゲールである為の
条件を上のままに当てはめると
$$
	{\bf E}
	\left[
		\left(
		\int^{t}_{0} \sigma_{s}^{2} X_{s}^{2} ds
		\right)^{\frac{1}{2}}
		\right]
	\ < \
	\infty
$$
になるが、
実際には条件はさらに簡潔にすることができて、
$$
	{\bf E}
	\left[
		\exp
		\left(
		\dfrac{1}{2}
		\int^{t}_{0} \sigma_{s}^{2} ds
		\right)
		\right]
	\ < \
	\infty
$$
が言えれば十分である、ということが書かれている。

加えてこの場合は解を陽に書き表すことができて、
$$
	X_{t}
	\ = \
	X_{0}
	\exp
	\left(
	\int^{t}_{0}
	\sigma_{s} dW_{s}
	-
	\dfrac{1}{2}
	\int^{t}_{0} \sigma_{s}^{2} ds
	\right)
$$
になる。

${}$

このことを確かめてみる。

$X_{t} = X_{0} e^{Y_{t}}$と置く。
$Y_{t} =
	\displaystyle
	\int^{t}_{0}
	\sigma_{s} dW_{s}
	-
	\dfrac{1}{2}
	\int^{t}_{0} \sigma_{s}^{2} ds
$
なので、微分は
$$
	d Y_{t}
	\ = \
	\sigma_{t} dW_{t}
	-
	\dfrac{1}{2}
	\sigma_{t}^{2} dt
$$
になる。
$dY_{t}$の二乗は、
%
\begin{eqnarray*}
	d Y_{t}^{2}
	&=&
	\left(
	\sigma_{t} dW_{t}
	-
	\dfrac{1}{2}
	\sigma_{t}^{2} dt
	\right)^{2}
	\\ &=&
	\sigma_{t}^{2} dt
\end{eqnarray*}
%

関数$f$を$f(y)=X_{0} e^{y}$と置いて伊藤の公式を利用する。
$f'(y) = f''(y) = X_{0} e^{y}$なので、
%
\begin{eqnarray*}
	d X_{t}
	&=&
	d f(y)
	\\ &=&
	f'(y) dY_{t}
	\ + \
	\dfrac{1}{2} f''(y) dY_{t}^{2}
	\\ &=&
	X_{0} e^{Y_{t}}
	\left(
	\sigma_{t} dW_{t}
	-
	\dfrac{1}{2}
	\sigma_{t}^{2} dt
	\right)
	\ + \
	\dfrac{1}{2}
	X_{0} e^{Y_{t}}
	\sigma_{t}^{2} dt
	\\ &=&
	X_{0} e^{Y_{t}}
	\sigma_{t} dW_{t}
	\\ &=&
	\sigma_{t} X_{t} dW_{t}
\end{eqnarray*}
%
以上から$dX_{t} = \sigma_{t} X_{t} dW_{t}$の解が
$$
	X_{t}
	\ = \
	X_{0}
	\exp
	\left(
	\int^{t}_{0}
	\sigma_{s} dW_{s}
	-
	\dfrac{1}{2}
	\int^{t}_{0} \sigma_{s}^{2} ds
	\right)
$$
になることが確認できた。

\section{練習問題 3.11}

\subsection{問題}

$\sigma$が時刻と経路の両方について有界な関数であるならば、
$dX_{t} = \sigma_{t} X_{t} dW_{t}$は$\mathbb{P}$-マルチンゲールであることを示せ。
\subsection{解答}

まず
$dX_{t} = \sigma_{t} X_{t} dW_{t}$
にはドリフト項が無いので、
指数マルチンゲールであるための条件
$$
	{\bf E}
	\left[
		\exp
		\left(
		\dfrac{1}{2}
		\int^{t}_{0} \sigma_{s}^{2} ds
		\right)
		\right]
	\ < \
	\infty
$$
を満たせばマルチンゲールであることが言える。

${}$

時点$t$、経路$\omega$を用いて$\sigma_{t}=\sigma(t,\omega)$とあらわす。
時刻と経路の両方について有界な関数なので、ある定数$K$をもって
任意の$(t,\omega)$について
$
	| \sigma(t,\omega) | < K
$
のように評価できる。
二乗して
$
	\sigma^{2}(t,\omega) < K^{2}
$
であるので、
%
\begin{eqnarray*}
	{\bf E}_{\mathbb{P}}
	\left[
		\exp
		\left(
		\dfrac{1}{2}
		\int^{t}_{0} \sigma_{s}^{2} ds
		\right)
		\
		\Big|
		\
		\omega
		\right]
	&<&
	\exp
	\left(
	\dfrac{1}{2}
	\int^{t}_{0} K^{2} ds
	\right)
	\ = \
	{\rm const.}
\end{eqnarray*}
%
これが有限値で押さえられることから、
問題の条件では$X_{t}$がマルチンゲールであることが示された。

${}$

\ \\

3章5節はここまで。

\begin{thebibliography}{9}
	\bibitem{BaxterRennie}
	Financial Calculus - An Introduction to Derivative Pricing - Martin Baxter, Andrew Rennie
\end{thebibliography}
\end{document}
\end{document}

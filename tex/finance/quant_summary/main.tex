\documentclass[uplatex,a4j,12pt,dvipdfmx]{jsarticle}
\usepackage{amsmath,amsthm,amssymb,bm,color,enumitem,mathrsfs,url,epic,eepic,ascmac,ulem,here}
\usepackage[letterpaper,top=2cm,bottom=2cm,left=3cm,right=3cm,marginparwidth=1.75cm]{geometry}
\usepackage[english]{babel}
\usepackage[dvipdfm]{graphicx}
\usepackage[hypertex]{hyperref}
\title{
    Summary Notes on \textit{Quanto Derivatives} \\
    \normalsize Reading from \texttt{Baxter \& Rennie}, Chapter 4.5
}
\author{Masaru Okada}

\date{\today}

\begin{document}

\maketitle

\tableofcontents

\ \\
\section{Quanto Contracts}

A contract in which the currency of the underlying asset differs from the currency of the contract's payoff.

For instance, a contract that pays $S_{T}$ yen when a product listed on a U.S. market is priced at $S_{T}$ dollars.

\section{The Idea of a Quanto (The Model We Will Consider)}

Using two non-independent $\mathbb{P}$-Brownian motions, $W_{1}(t)$ and $W_{2}(t)$, we consider the following model:
%
%
\begin{eqnarray}
	\left.
	\begin{array}{ll}
		S_{t}
		\ = \
		S_{0}
		\exp \left( \sigma_{1} W_{1}(t) + \mu t \right) \ ,
		 &
		B_{t}
		\ = \
		\exp (rt)
		\\
		C_{t}
		\ = \
		C_{0}
		\exp \left(
		\rho \sigma_{2} W_{1}(t) +
		\bar{\rho} \sigma_{2} W_{2}(t) + \nu t \right)\ ,
		 &
		D_{t}
		\ = \
		\exp (ut)
	\end{array}
	\right.
\end{eqnarray}
%
%
Here, $\mu, \nu, r, u$ are constants, with $r$ and $u$ being positive.
$\rho$ is a constant between 0 and 1, inclusive,
and we use the shorthand $\bar{\rho} = \sqrt{1 - \rho^{2}}$.
$S_{t}$ is the stock price in pounds at time $t$,
$C_{t}$ is the exchange rate (1 pound = $C_{t}$ dollars),
$B_{t}$ is the dollar cash bond,
and $D_{t}$ is the pound cash bond.

\section{Interpretation of the Model}

By calculating the variance-covariance matrix and correlation matrix of the vector-valued random variable
$( \log S_{t} , \log C_{t})$,
we find that:

\ \ \ \ $\cdot$ The volatility of $\log S_{t}$ is $\sigma_{1}$

\ \ \ \ $\cdot$ The volatility of $\log C_{t}$ is $\sigma_{2}$

\ \ \ \ $\cdot$ The correlation between $\log S_{t}$ and $\log C_{t}$ is $\rho$

\section{SDE of Tradable Assets}

In this model, there are three assets that can be traded in dollars.

\ \ \ 1. The pound cash bond, denominated in dollars: $C_{t} D_{t}$

\ \ \ 2. The stock price, denominated in dollars: $C_{t} S_{t}$

\ \ \ 3. The dollar cash bond: $B_{t}$

Taking $B_{t}$ as the numeraire, we consider the two discounted processes
$Y_{t} = B_{t}^{-1} C_{t} D_{t}$ and
$Z_{t} = B_{t}^{-1} C_{t} S_{t}$.
The stochastic differential equations (SDEs) that these processes satisfy are:
%
%
\begin{eqnarray}
	\dfrac{dY_{t}}{Y_{t}}
	&=&
	\rho \sigma_{2} dW_{1}(t) +
	\bar{\rho} \sigma_{2} dW_{2}(t) \ + \ \left( \nu + u + \dfrac{1}{2} \sigma_{2}^{2} - r \right) dt
	\\
	\dfrac{dZ_{t}}{Z_{t}}
	&=&
	(\sigma_{1} + \rho \sigma_{2}) dW_{1}(t)
	+ \bar{\rho} \sigma_{2} dW_{2}(t)
	\ + \
	\left(
	\mu + \nu +
	\dfrac{1}{2} \sigma_{1}^{2} + \rho \sigma_{1} \sigma_{2} + \dfrac{1}{2} \sigma_{2}^{2}
	- r
	\right) dt
\end{eqnarray}
%
%


\section{SDE of Tradable Assets (Matrix Representation)}
The stochastic differential equations can be expressed in matrix form as:
%
%
\begin{eqnarray}
	\left(
	\begin{array}{c}
		d Y_{t} / Y_{t}
		\\[2mm]
		d Z_{t} / Z_{t}
	\end{array}
	\right)
	&=&
	\left(
	\begin{array}{ccc}
		\rho \sigma_{2}              & \bar{\rho} \sigma_{2} & \nu + u + \dfrac{1}{2} \sigma_{2}^{2} - r
		\\
		\sigma_{1} + \rho \sigma_{2} & \bar{\rho} \sigma_{2} & \mu + \nu + \dfrac{1}{2} \sigma_{1}^{2} + \rho \sigma_{1} \sigma_{2} + \dfrac{1}{2} \sigma_{2}^{2} - r
	\end{array}
	\right)
	\left(
	\begin{array}{c}
		dW_{1}(t)
		\\
		dW_{2}(t)
		\\
		dt
	\end{array}
	\right)
	\\ &=&
	\left(
	\begin{array}{cc}
		\rho \sigma_{2}              & \bar{\rho} \sigma_{2}
		\\
		\sigma_{1} + \rho \sigma_{2} & \bar{\rho} \sigma_{2}
	\end{array}
	\right)
	\left(
	\begin{array}{c}
		dW_{1}(t)
		\\
		dW_{2}(t)
	\end{array}
	\right)
	\ + \
	\left(
	\begin{array}{c}
		\nu + u + \dfrac{1}{2} \sigma_{2}^{2} - r
		\\
		\mu + \nu + \dfrac{1}{2} \sigma_{1}^{2} + \rho \sigma_{1} \sigma_{2} + \dfrac{1}{2} \sigma_{2}^{2} - r
	\end{array}
	\right)
	dt
\end{eqnarray}
%
%
We have separated the terms involving the differentials of Brownian motion from the time differential term.

The coefficient matrix of the Brownian motion differential terms is called the volatility matrix, denoted by
${\bm \Sigma}$.
%
%
\begin{eqnarray}
	{\bm \Sigma}
	&=&
	\left(
	\begin{array}{cc}
			\rho \sigma_{2}              & \bar{\rho} \sigma_{2}
			\\
			\sigma_{1} + \rho \sigma_{2} & \bar{\rho} \sigma_{2}
		\end{array}
	\right)
\end{eqnarray}
%
%
The drift vector ${\bm \mu}$ is defined as follows.
%
%
\begin{eqnarray}
	{\bm \mu}
	&=&
	\left(
	\begin{array}{c}
			\nu + u + \dfrac{1}{2} \sigma_{2}^{2}
			\\
			\mu + \nu + \dfrac{1}{2} \sigma_{1}^{2} + \rho \sigma_{1} \sigma_{2} + \dfrac{1}{2} \sigma_{2}^{2}
		\end{array}
	\right)
\end{eqnarray}
%
%
Using the volatility matrix ${\bm \Sigma}$ and
the drift vector ${\bm \mu}$,
the SDE can be written more clearly as:
%
%
\begin{eqnarray}
	\left(
	\begin{array}{c}
			d Y_{t} / Y_{t}
			\\
			d Z_{t} / Z_{t}
		\end{array}
	\right)
	&=&
	{\bm \Sigma}
	\left(
	\begin{array}{c}
			dW_{1}(t)
			\\
			dW_{2}(t)
		\end{array}
	\right)
	+
	( {\bm \mu} - r {\bm 1} ) dt
\end{eqnarray}
%
%

\section{Condition for No Drift}
To eliminate the drift term, we need to be able to write the SDE using a Brownian motion
$(\tilde{W}_{1}(t),\tilde{W}_{2}(t))$
under some measure $\mathbb{Q}$, such that:
%
%
\begin{eqnarray}
	\left(
	\begin{array}{c}
			d Y_{t} / Y_{t}
			\\
			d Z_{t} / Z_{t}
		\end{array}
	\right)
	&=&
	{\bm \Sigma}
	\left(
	\begin{array}{c}
			d \tilde{W}_{1}(t)
			\\
			d \tilde{W}_{2}(t)
		\end{array}
	\right)
\end{eqnarray}
%
%
Comparing the right-hand sides gives us
%
%
\begin{eqnarray}
	{\bm \Sigma}
	\left(
	\begin{array}{c}
			d \tilde{W}_{1}(t)
			\\
			d \tilde{W}_{2}(t)
		\end{array}
	\right)
	&=&
	{\bm \Sigma}
	\left(
	\begin{array}{c}
			dW_{1}(t)
			\\
			dW_{2}(t)
		\end{array}
	\right)
	+
	( {\bm \mu} - r {\bm 1} ) dt
\end{eqnarray}
%
%
If the volatility matrix ${\bm \Sigma}$ is invertible, then
%
%
\begin{eqnarray}
	\left(
	\begin{array}{c}
			d \tilde{W}_{1}(t)
			\\
			d \tilde{W}_{2}(t)
		\end{array}
	\right)
	&=&
	\left(
	\begin{array}{c}
			dW_{1}(t)
			\\
			dW_{2}(t)
		\end{array}
	\right)
	+
	{\bm \Sigma}^{-1}
	( {\bm \mu} - r {\bm 1} ) dt
	\\ &=&
	\left(
	\begin{array}{c}
			dW_{1}(t)
			\\
			dW_{2}(t)
		\end{array}
	\right)
	+
	{\bm \gamma}dt
\end{eqnarray}
%
%
where ${\bm \gamma}$ is
the market price of risk
corresponding to
$(W_{1}(t),W_{2}(t))$,
${\bm \gamma}^{T} = (\gamma_{1}(t),\gamma_{2}(t))$.
%
%
\begin{eqnarray}
	{\bm \gamma}
	&=&
	{\bm \Sigma}^{-1}
	( {\bm \mu} - r {\bm 1} )
\end{eqnarray}
%
%
To describe the SDE using the $\mathbb{Q}$-Brownian motion
$(\tilde{W}_{1}(t),\tilde{W}_{2}(t))$,
we just need to calculate the market price of risk.
%
%
\begin{eqnarray}
	{\bm \gamma}
	&=&
	{\bm \Sigma}^{-1}
	( {\bm \mu} - r {\bm 1} )
	\\ &=&
	\dfrac{1}{ \bar{\rho} \sigma_{1} \sigma_{2} }
	\left(
	\!\!
	\begin{array}{cc}
			- \bar{\rho} \sigma_{2}      & \bar{\rho} \sigma_{2}
			\\
			\sigma_{1} + \rho \sigma_{2} & - \rho \sigma_{2}
		\end{array}
	\!\!
	\right)
	\!\!
	\left(
	\begin{array}{c}
			\nu + u + \dfrac{1}{2} \sigma_{2}^{2} - r
			\\
			\mu + \nu + \dfrac{1}{2} \sigma_{1}^{2} + \rho \sigma_{1} \sigma_{2} + \dfrac{1}{2} \sigma_{2}^{2} - r
		\end{array}
	\right)
\end{eqnarray}
%
%
From this, the components of the market price of risk are respectively
%
%
\begin{eqnarray}
	\gamma_{1}
	&=&
	\dfrac{
		u
		-
		\mu - \dfrac{1}{2} \sigma_{1}^{2} - \rho \sigma_{1} \sigma_{2}
	}
	{ \sigma_{1} }
	\\[3mm]
	\gamma_{2}
	&=&
	\dfrac{
		\nu + u + \dfrac{1}{2} \sigma_{2}^{2} - r - \rho \sigma_{2} \gamma_{1}
	}
	{ \bar{\rho} \sigma_{2} }
\end{eqnarray}
%
%

\subsection*{Using the Condition that $\bm{\gamma}$ Satisfies When Tradable}
The stochastic differential equations can be rewritten using the $\mathbb{Q}$-Brownian motion as follows.
%
%
\begin{eqnarray}
	\left(
	\begin{array}{c}
		d Y_{t} / Y_{t}
		\\
		d Z_{t} / Z_{t}
	\end{array}
	\right)
	&=&
	\left(
	\begin{array}{l}
		\rho \sigma_{2} d \tilde{W}_{1}(t) \ + \ \bar{\rho} \sigma_{2} d \tilde{W}_{2}(t)
		\\
		( \sigma_{1} + \rho \sigma_{2} ) d \tilde{W}_{1}(t) \ + \ \bar{\rho} \sigma_{2} d \tilde{W}_{2}(t)
	\end{array}
	\right)
\end{eqnarray}
%
%
This is a system of SDEs. Attempting to solve it directly in this form seems quite cumbersome, as it would require diagonalizing the volatility matrix and so on.

However, using the concept of the market price of risk from the previous section, we only need to find the values of ${\bm \mu} = (\mu,\nu)$ that make ${\bm \gamma} = {\bm 0}$ and substitute them back into the processes for the original assets.

From $\gamma_{1} = 0$, we have
%
%
\begin{eqnarray}
	\mu
	\ = \
	u - \dfrac{1}{2} \sigma_{1}^{2} - \rho \sigma_{1} \sigma_{2}
\end{eqnarray}
%
%

From $\gamma_{2} = 0$, we have
%
%
\begin{eqnarray}
	\nu
	\ = \
	r - u - \dfrac{1}{2} \sigma_{2}^{2}
\end{eqnarray}
%
%

The original asset processes were expressed using the $\mathbb{P}$-Brownian motion as
%
%
\begin{eqnarray}
	S_{t}
	&=&
	S_{0}
	\exp \left( \sigma_{1} W_{1}(t) + \mu t \right)
	\\
	C_{t}
	&=&
	C_{0}
	\exp \left(
	\rho \sigma_{2} W_{1}(t) +
	\bar{\rho} \sigma_{2} W_{2}(t) + \nu t \right)
\end{eqnarray}
%
%
so, describing them with the $\mathbb{Q}$-Brownian motion and substituting the values of ${\bm \mu}$ we just found gives:
%
%
\begin{eqnarray}
	S_{t}
	&=&
	S_{0}
	\exp \left( \sigma_{1} \tilde{W}_{1}(t) +
	\left(
		u - \dfrac{1}{2} \sigma_{1}^{2} - \rho \sigma_{1} \sigma_{2}
		\right)
	t \right)
	\\
	C_{t}
	&=&
	C_{0}
	\exp \left(
	\rho \sigma_{2} \tilde{W}_{1}(t) +
	\bar{\rho} \sigma_{2} \tilde{W}_{2}(t) +
	\left(
		r - u - \dfrac{1}{2} \sigma_{2}^{2}
		\right)
	t \right)
\end{eqnarray}
%
%

${}$

For $C_{t}$,
by defining a $\mathbb{Q}$-Brownian motion
$$
	\tilde{W}_{3}(t)
	\ = \
	\rho \tilde{W}_{1}(t) +
	\bar{\rho} \tilde{W}_{2}(t)
$$
we can write
%
%
\begin{eqnarray}
	C_{t}
	&=&
	e^{(r-u)t}
	C_{0}
	\exp \left(
	\sigma_{2} \tilde{W}_{3}(t) -
	\dfrac{1}{2} \sigma_{2}^{2}
	t \right)
\end{eqnarray}
%
%
Therefore, if we choose the numeraire $e^{(r-u)t} = B_{t} D^{-1}_{t}$,
the process becomes a $\mathbb{Q}$-martingale, and is thus tradable.

$S_{t}$, on the other hand, is given by
%
%
\begin{eqnarray}
	S_{t}
	&=&
	e^{ut}
	S_{0}
	e^{- \rho \sigma_{1} \sigma_{2}}
	\exp \left( \sigma_{1} \tilde{W}_{1}(t) -
	\dfrac{1}{2} \sigma_{1}^{2}
	t \right)
\end{eqnarray}
%
%
Because it contains a factor of
$e^{- \rho \sigma_{1} \sigma_{2}}$,
it does not become tradable even if we choose the pound cash bond $e^{ut} = D_{t}$ as the numeraire.



\ \\[-10mm]

\section{Quanto Forward}

We have successfully expressed the tradable assets using a Brownian motion under a measure $\mathbb{Q}$ that renders them martingales.
Our next step is to determine the price of a quanto contract by calculating the expected value under this $\mathbb{Q}$ measure.

Regarding the expression for $S_{t}$ written with the $\mathbb{Q}$-Brownian motion, and using the forward price at maturity $T$, $F = D_{T} S_{0} = e^{uT} S_{0}$, we can write:
%
%
\begin{eqnarray}
	S_{t}
	&=&
	F
	e^{- \rho \sigma_{1} \sigma_{2}}
	\exp \left( \sigma_{1} \tilde{W}_{1}(t) -
	\dfrac{1}{2} \sigma_{1}^{2}
	t \right)
\end{eqnarray}
%
%
The present value of a forward contract with a delivery price of $K$ (dollars) can be calculated as follows.
%
%
\begin{eqnarray}
	V_{0}
	&=&
	e^{-rT}
	\mathbb{E}_{\mathbb{Q}}(S_{T}-K)
	\\ &=&
	e^{-rT}
	\mathbb{E}_{\mathbb{Q}}
	\left[
		F
		e^{- \rho \sigma_{1} \sigma_{2}}
		\exp \left( \sigma_{1} \tilde{W}_{1}(T) -
		\dfrac{1}{2} \sigma_{1}^{2}
		T \right)
		-
		K
		\right]
	\\ &=&
	F
	e^{- \rho \sigma_{1} \sigma_{2}}
	e^{-rT}
	\exp \left( -\dfrac{1}{2} \sigma_{1}^{2} T \right)
	\mathbb{E}_{\mathbb{Q}}
	e^{\sigma_{1} \tilde{W}_{1}(T)}
	-
	Ke^{-rT}
\end{eqnarray}
%
%
Here, the expected value
$\mathbb{E}_{\mathbb{Q}}
	e^{\sigma_{1} \tilde{W}_{1}(T)}$
is calculated by noting that
$\dfrac{\tilde{W}_{1}(T)}{\sqrt{T}}$
follows a standard normal distribution (a random variable $Z$) under the measure $\mathbb{Q}$.
%
%
\begin{eqnarray}
	\mathbb{E}_{\mathbb{Q}}
	e^{\sigma_{1} \tilde{W}_{1}(T)}
	&=&
	\mathbb{E}_{\mathbb{Q}}
	\exp \left( \sigma_{1} \sqrt{T} \dfrac{\tilde{W}_{1}(T)}{\sqrt{T}} \right)
	\\ &=&
	\mathbb{E}
	\exp \left( \sigma_{1} \sqrt{T} Z \right)
	\\ &=&
	\dfrac{1}{ \sqrt{2 \pi} }
	\int^{\infty}_{-\infty}
	\exp \left( \sigma_{1} \sqrt{T} z \right)
	\
	e^{-\frac{1}{2}z^{2}}
	dz
	\\ &=&
	\exp \left( \dfrac{1}{2} (\sigma_{1} \sqrt{T})^{2} \right)
\end{eqnarray}
%
%

In summary,
%
%
\begin{eqnarray}
	V_{0}
	&=&
	e^{-rT}
	\mathbb{E}_{\mathbb{Q}}(S_{T}-K)
	\\ &=&
	F
	e^{-rT}
	e^{- \rho \sigma_{1} \sigma_{2}}
	\exp \left( -\dfrac{1}{2} \sigma_{1}^{2} T \right)
	\exp \left(\dfrac{1}{2} \sigma_{1}^{2} T \right)
	-
	Ke^{-rT}
	\\ &=&
	F
	e^{-rT}
	e^{- \rho \sigma_{1} \sigma_{2}}
	-
	Ke^{-rT}
\end{eqnarray}
%
%

The delivery price $K$ is set such that the present value is zero, ensuring that neither party to the trade makes a profit or loss at inception.

Thus, if we write the value of $K$ when $V_{0}=0$ as $F_{Q}$, we have:
%
%
\begin{eqnarray}
	0 &=& V_{0}
	\ = \
	F
	e^{-rT}
	e^{- \rho \sigma_{1} \sigma_{2}}
	-
	F_{Q} e^{-rT}
	\\
	\Longleftrightarrow
	\
	F_{Q}
	&=&
	F
	e^{- \rho \sigma_{1} \sigma_{2}}
\end{eqnarray}
%
%

Since the values of $\sigma_{1}$ and $\sigma_{2}$ are both positive,
the price of the quanto forward is higher than the standard forward price $F$ by a factor of $e^{- \rho \sigma_{1} \sigma_{2}}$
only when the stock price and the exchange rate are negatively correlated ($\rho < 0$).


\ \\[-10mm]

\section{Quanto Digital Option}

A quanto digital option is a contract that, for example, pays 1 dollar (not 1 pound) if the stock price $S_{T}$ (in pounds) at maturity $T$ exceeds a predetermined price of $K$ pounds.

The payoff is
$$
	X
	\ = \
	1_{ \{ S_{T}>K \} }
$$
and its present value $V_{0}$ is
%
%
\begin{eqnarray}
	\hspace{-10mm}
	V_{0}
	& = &
	B^{-1}_{T}
	\mathbb{E}_{\mathbb{Q}}
	X
	\\ & = &
	e^{-rT}
	\mathbb{E}_{\mathbb{Q}}
	1_{ \{ S_{T}>K \} }
	\\ & = &
	e^{-rT}
	\mathbb{Q}
	\{ S_{T}>K \}
\end{eqnarray}
%
%
Here, $\mathbb{Q}\{B\}$ is the probability of condition $B$ being met under the measure $\mathbb{Q}$.

$S_{T}$ could be written using the $\mathbb{Q}$-Brownian motion as follows.
%
%
\begin{eqnarray}
	S_{T}
	&=&
	F
	\exp \left( \sigma_{1} \tilde{W}_{1}(T) -
	\dfrac{1}{2} \sigma_{1}^{2} T -
	\rho \sigma_{1} \sigma_{2}
	\right)
	\\ &=&
	F
	\exp \left( \sigma_{1} \sqrt{T} \dfrac{ \tilde{W}_{1}(T) }{ \sqrt{T} } -
	\dfrac{1}{2} \sigma_{1}^{2} T -
	\rho \sigma_{1} \sigma_{2} \right)
	\\ &=&
	F
	\exp \left( \sigma_{1} \sqrt{T} Z -
	\dfrac{1}{2} \sigma_{1}^{2} T  -
	\rho \sigma_{1} \sigma_{2}\right)
\end{eqnarray}
%
%
where $F$ is the forward price at maturity $T$, $F = e^{uT} S_{0}$.
Also, $Z$ is a standard normal random variable under $\mathbb{Q}$, following $N(0,1)$.

Furthermore, for simplicity of notation, if we use
$$
	F_{Q} \ = \ F e^{- \rho \sigma_{1} \sigma_{2} }
$$
then,
$$
	S_{T}
	\ = \
	F_{Q}
	\exp \left( \sigma_{1} \sqrt{T} Z -
	\dfrac{1}{2} \sigma_{1}^{2} T
	\right)
$$
Rearranging the condition $S_{T}>K$, we get:
%
%
\begin{eqnarray}
	S_{T}
	&>&
	K
	\\
	F_{Q}
	\exp \left( \sigma_{1} \sqrt{T} Z -
	\dfrac{1}{2} \sigma_{1}^{2} T
	\right)
	&>&
	K
	\\
	Z
	&>&
	\dfrac{
		\dfrac{1}{2} \sigma_{1}^{2} T
		-
		\log
		\dfrac{F_{Q}}{K}
	}{\sigma_{1} \sqrt{T}}
	\ = \
	z_{0}
\end{eqnarray}
%
%
Since the right-hand side is complex, we will temporarily set it as $z_{0}$.
Continuing the calculation,
%
%
\begin{eqnarray}
	V_{0}
	&=&
	e^{-rT}
	\mathbb{Q}
	\{ S_{T} > K \}
	\\ &=&
	e^{-rT}
	\mathbb{Q}
	\{ Z > z_{0} \}
	\\ &=&
	e^{-rT}
	\dfrac{1}{ \sqrt{2 \pi} }
	\int^{\infty}_{z_{0}}
	e^{
			- \frac{1}{2} z^{2}
		} dz
	\\ &=&
	e^{-rT}
	\Phi
	\left(
	\dfrac{
		\log
		\dfrac{F_{Q}}{K}
		-
		\dfrac{1}{2} \sigma_{1}^{2} T
	}{\sigma_{1} \sqrt{T}}
	\right)
\end{eqnarray}
%
%


\ \\[-10mm]

\section{Quanto Call Option}
A quanto call option is a contract that, for example, pays
$S_{T} - k$ dollars (not $S_{T} - k$ pounds) if the stock price $S_{T}$ (in pounds) at maturity $T$ exceeds a predetermined strike price of $k$ pounds.

The payoff is
%
%
\begin{eqnarray}
	X
	\ = \
	( S_{T} - k )^{+}
\end{eqnarray}
%
%
and its present value $V_{0}$ is
%
%
\begin{eqnarray}
	V_{0}
	& = &
	B^{-1}_{T}
	\mathbb{E}_{\mathbb{Q}}
	X
	\\ & = &
	e^{-rT}
	\mathbb{E}_{\mathbb{Q}}
	( S_{T} - k  )^{+}
\end{eqnarray}
%
%

As in the previous section, using
$
	F_{Q} = F e^{- \rho \sigma_{1} \sigma_{2} }
$
and a standard normal random variable $Z$,
%
%
\begin{eqnarray}
	S_{T}
	\ = \
	F_{Q}
	\exp \left( \sigma_{1} \sqrt{T} Z -
	\dfrac{1}{2} \sigma_{1}^{2} T
	\right)
\end{eqnarray}
%
%
$S_T$ can be written as above, therefore:
%
%
\begin{eqnarray}
	V_{0}
	& = &
	e^{-rT}
	\mathbb{E}
	\left(
	F_{Q}
	\exp \left( \sigma_{1} \sqrt{T} Z -
	\dfrac{1}{2} \sigma_{1}^{2} T
	\right)
	- k
	\right)^{+}
\end{eqnarray}
%
%

Using the 'calculation formula for the price of a call option in the log-normal case', which was already discussed in Section 4.1, we get:

%
%
\begin{eqnarray}
	V_{0}
	&=&
	e^{-rT}
	\left\{
	F_{Q}
	\Phi
	\left(
	\dfrac{
		\log
		\dfrac{F_{Q}}{k}
		+
		\dfrac{1}{2} \sigma_{1}^{2} T
	}{\sigma_{1} \sqrt{T}}
	\right)
	-
	k
	\Phi
	\left(
	\dfrac{
		\log
		\dfrac{F_{Q}}{k}
		-
		\dfrac{1}{2} \sigma_{1}^{2} T
	}{\sigma_{1} \sqrt{T}}
	\right)
	\right\}
\end{eqnarray}
%
%


\section{Exercise 4.3}
Consider the case where the underlying asset is $S_{t}$ yen,
the exchange rate $C_{t}$ is in units of dollars per yen,
and the correlation coefficient is $\rho$. Determine the formula for the quanto forward price in this scenario.
In the previous model, the payoff currency was in the denominator of the exchange rate convention, such as 'pounds per dollar'.

In the current case, the payoff currency is in the numerator,
which means the correlation coefficient effectively becomes
$\rho \to - \rho$.

In this case, the formula becomes:
$
	F_{Q}
	=
	F
	e^{\rho \sigma_{1} \sigma_{2}}
$
.

The pricing formulas for the digital and call options are also modified in accordance with this change to the forward price formula.

\begin{thebibliography}{9}
	\bibitem{BaxterRennie}
	Financial Calculus - An Introduction to Derivative Pricing - Martin Baxter, Andrew Rennie
\end{thebibliography}

\end{document}
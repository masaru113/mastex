\documentclass[uplatex,a4j,12pt,dvipdfmx]{jsarticle}
\usepackage{amsmath,amsthm,amssymb,bm,color,enumitem,mathrsfs,url,epic,eepic,ascmac,ulem,here,ascmac}
\usepackage[letterpaper,top=2cm,bottom=2cm,left=3cm,right=3cm,marginparwidth=1.75cm]{geometry}
\usepackage[english]{babel}
\usepackage[dvipdfm]{graphicx}
\usepackage[hypertex]{hyperref}
\title{Research Ethics Lecture 1 Notes}
\author{M. O.}
\date{\today}
\begin{document}
\maketitle
\tableofcontents
\section{Overview of Research Ethics}
\subsection{Introduction}
This report is based on a lecture audio explaining the importance of research ethics in business school and the observance of norms in specific research activities. The objective is for students to deeply understand ethical norms and grasp related procedures and trouble-shooting, not only for academic papers but for a broad rangeof research activities (reports, business plans, surveys, etc.). In particular, this report will clarify the two pillars of guidelines for research ethics in general and guidelines specific to research involving human subjects, organizing their scope and importance.
\subsection{Key Concepts and Issues}
\subsubsection{Definition and Necessity of Research Ethics}
\paragraph{Research Ethics}
Research ethics are the norms that researchers must observe when conducting research.
\paragraph{Scope of Research}
Research in graduate school refers to all activities that clarify the truth of specific matters or discover and create new knowledge. This includes not only academic papers and research papers but also all reports and business plan exercises assigned during the course, which are themselves respectable research activities and must follow the norms of research ethics.
\paragraph{Necessity of Research Ethics}
Similar to corporate compliance, research ethics are indispensable for ensuring the credibility of research. While researcher autonomy is respected, research ethics are stipulated by each research institution to prevent research misconduct and maintain a sound research environment.
\subsubsection{Types and Examples of Research Misconduct}
Research misconduct is broadly divided into the following two categories:
\begin{enumerate}
	\item Misconduct in Research Activities:
	      \begin{itemize}
		      \item Fabrication: The act of artificially creating data that does not actually exist.
		      \item Falsification: The act of improperly manipulating data or results to distort the truth.
		      \item Plagiarism: The act of presenting others' research results (text, figures, ideas, etc.) as one's own without proper citation. Copying and pasting from others' works in reports or theses at graduate school falls under this category, and strict caution is required.
	      \end{itemize}
	\item Misuse of Research Funds:
	      \begin{itemize}
		      \item This includes fraudulent billing, use for unapproved purposes, etc.
	      \end{itemize}
\end{enumerate}
\subsubsection{Obligation to Comply with Research Ethics and Penalties for Violations}
\paragraph{Scope of Compliance}
Research ethics must be observed at all stages of research, from conception to planning, application, implementation, and reporting.
\paragraph{Penalties for Violations}
Ignorance ('I didn't know,' 'I didn't notice') is not an excuse for misconduct. Violations will lead to disciplinary action, such as dismissal or expulsion, and a loss of social credibility. Furthermore, cases that infringe upon the rights or property of others may lead to civil lawsuits or criminal prosecution. Therefore, researchers are expected to engage in their work with a strong senseof personal responsibility and awareness.
\subsubsection{Two Sets of Ethical Guidelines}
This lecture pointed out the existence of the following two sets of guidelines:
\begin{itemize}
	\item Research Ethics Guidelines: Describe general research ethics.
	\item Guidelines for Research Involving Human Subjects: Provisions specific to cases where humans are the research subjects, such as special considerations for personal information and the design of prior review processes.
\end{itemize}
\subsection{Application and Case Analysis}
\subsubsection{Case Studies of Plagiarism in Graduate School}
This lecture particularly warned against plagiarism, a pitfall that graduate students often fall into.
\begin{itemize}
	\item \textbf{Difference from casual copy-pasting on SNS}: While one might casually copy and paste others' text in general social media posts, submitting works written by others as one's own without adhering to citation rules in graduate school reports or papers is treated as serious plagiarism.
	\item \textbf{Scope of Application}: This applies equally to activities that may seem removed from academic research, such as \textbf{creating business plans}. For example, using descriptions from existing business plan documents or market analysis reports without permission constitutes a violation of research ethics.
\end{itemize}
\subsubsection{Systems and Structures for Misconduct Prevention}
To establish research ethics and prevent misconduct, many research institutions, as mentioned in this lecture, have established guidelines, set up ethics committees, and implemented prior review systems, especially for research involving human subjects. These function as mechanisms to help researchers understand ethical risks and ensure the transparency and fairness of research.
\subsection{Deeper Context and Lessons}
\textbf{\paragraph{The Parallel Between Research and Compliance}}
This lecture used the analogy that observing research ethics is essentially the same as observing compliance in business. Just as compliance is indispensable for sound corporate operations, research ethics are indispensable for the credibility of research. This comparison is highly effective for business school students in framing research ethics as a professional responsibility.
\textbf{\paragraph{Social Impact of Research Misconduct}}
Misconduct in research activities, such as fabrication and falsification, is not merely an internal problem for the research institution; its results, when applied to society, can cause serious damage to public health and economic activities. Furthermore, the misuse of research funds is an act that undermines trust in public funds, including taxes, and significantly erodes public trust in researchers.
\textbf{\subsubsection{AI Supplement: Expanding on Key Points}}
The transcript of this lecture omits specific mention of 'Informed Consent,' one of the core concepts of research ethics. As this is one of the most critical elements in the 'Guidelines for Research Involving Human Subjects,' it is supplemented below.
\paragraph{Informed Consent (IC)}
Informed Consent refers to 'agreement given after being fully informed.' It means that the individuals targeted for research (participants) are explained in advance and in detail the research purpose, methods, foreseeable benefits and risks, and the voluntary nature of participation (including the ability to withdraw consent at any time), and they consent to participate in the research based on their own free will. This procedure is strictly required as an ethical prerequisite before starting research, especially in studies involving human subjects such as surveys and interviews in business administration research.
\subsection{Conclusion}
This lecture reaffirmed the importance of research ethics in business school, emphasizing that its application extends beyond academic papers to all research activities, including reports and business plans. In particular, misconduct in research activities such as fabrication, falsification, and plagiarism can lead to severe disciplinary action and loss of social credibility, requiring students to approach their research with a strong senseof responsibility and awareness.
The practical lesson derived from the deeper context and supplementary information is the establishment of a professional code of ethics. Just like compliance in business, transparency and integrity are demanded in research activities. Especially in research involving human subjects, an attitude that fully respects the rights and dignity of the subjects, such as by thoroughly implementing informed consent, is essential. This is extremely important learning for building a foundation as managers and professionals who will be responsible for corporate ethics and CSR in the future.
\subsection{List of Important Keywords}
Research Ethics, Research Misconduct, Fabrication, Falsification, Plagiarism, Informed Consent
\vspace{\baselineskip}
Compliance, Research Paper, Disciplinary Action, Civil Lawsuit, Criminal Prosecution, Guidelines, Ethics Committee
\subsection{Comprehension Check Quiz}
\begin{enumerate}
	\item What are the norms that researchers must follow to ensure the fairness of research activities called?
	\item What is the term for the misconduct of artificially creating data that was not actually obtained and presenting it as research results?
	\item What is the term for the misconduct of presenting text from existing research papers or reports as if it were one's own work, without proper citation?
	\item Research misconduct is broadly divided into misconduct in research activities and one other type. What is the other type of misconduct?
	\item Are reports and business plans created in business school subject to the observance of research ethics? (Yes or No)
	\item In this lecture, what business equivalent was used to explain the necessity of research ethics for ensuring research credibility and preventing misconduct?
	\item If a violation of research ethics occurs, is the excuse 'I didn't know' valid? (Yes or No)
	\item If one receives disciplinary action for research misconduct, what social consequence was pointed out?
	\item Please state one example of misuse of research funds mentioned in this lecture.
	\item Compliance with research ethics is required from which stage of research to which stage?
	\item In research involving human subjects, what is the procedure of fully explaining the research content and risks to participants and obtaining their consent based on free will called?
	\item Is a survey in business administration research more relevant to the 'Research Ethics Guidelines' or the 'Guidelines for Research Involving Human Subjects'?
	\item In this lecture, the fundamental attitude always required of researchers was expressed as 'one's own responsibility and \underline{\hspace{1cm}}.' What word fills in the blank?
	\item If research misconduct not only leads to a loss of personal credibility but also causes damage to the rights or property of others, what two legal actions might it develop into?
	\item Was the necessity of research ethics explained with more emphasis on the principle of respecting researcher autonomy or on the objective of preventing research misconduct?
\end{enumerate}
\subsubsection*{Answer Key}
1. Research Ethics, 2. Fabrication, 3. Plagiarism, 4. Misuse of Research Funds, 5. Yes, 6. Compliance, 7. No, 8. Loss of social credibility, 9. Fraudulent billing or use for unapproved purposes, 10. From conception to reporting (or all stages), 11. Informed Consent, 12. Guidelines for Research Involving Human Subjects, 13. Awareness, 14. Civil lawsuit and criminal prosecution, 15. The objective of preventing research misconduct
\section{Research Ethics Guidelines}
\subsection{Introduction}
Research ethics are universal norms that should be observed even without explicit rules, but in recent years, the prevention of research misconduct has become a critical social issue. For this reason, it is essential for educational and research institutions to establish specific guidelines. These guidelines were established against this backdrop to \textbf{prevent research misconduct} at this university and ensure the soundness of research activities (Article 1). This note aims to organize the main points of these guidelines and reaffirm the ethical foundation for learning and practice in the MBA program.
\subsection{Key Concepts and Issues}
\subsubsection{Scope of Application (Article 2)}
The scope of application for these guidelines is defined for 'researchers' and 'research activities.'
\begin{itemize}
	\item \textbf{Researchers}: This includes not only the university's full-time faculty and students but all related parties engaged in research activities at the university, such as part-time lecturers, visiting faculty, research fellows, and even external \textbf{joint researchers}.
	\item \textbf{Research Activities}: Not only the development of lecture content but also degree-related research in the MBA program (such as business research projects and organizational change projects) are important research activities to which these guidelines apply.
\end{itemize}
\subsubsection{Responsibilities of the Graduate School (Article 3)}
The graduate school bears the responsibility for ensuring the soundness of research. This responsibility is primarily carried out by the \textbf{FD/SD Committee} (Faculty Development / Staff Development Committee). In case of anxiety or questions regarding research ethics, discovery of research misconduct, or other troubles, it is required to promptly contact the FD/SD Committee.
\subsubsection{Responsibilities of Researchers (Article 4)}
This is the core of the guidelines, stipulating the specific responsibilities (10 items in total) that researchers must observe.
\begin{itemize}
	\item \textbf{Research Conduct (Item 1)}: Basic ethics in the execution of research.
	\item \textbf{Use of Research Funds (Item 2)}: Fair and transparent execution of expenses.
	\item \textbf{Contract Conclusion (Item 3)}: Compliance with contractual obligations.
	\item \textbf{Research Publication (Item 4)}: \textbf{The most critical item requiring special attention}. When publishing papers or business plans, meticulous care must be paid to ensuring the \textbf{credibility of data} and \text{adherence to citation rules}.
	\item \textbf{Others}: Compliance with laws, appropriate handling of personal information, etc., are stipulated.
	\item \textbf{Research Involving Human Subjects (Item 7)}: Provisions concerning the protection of subjects' human rights, etc. (details omitted in this lecture).
\end{itemize}
\subsubsection{Responsibilities of Research Supervisors (Article 5)}
Faculty members responsible for supervision also bear the responsibility to advise and guide students to ensure their research activities are conducted ethically.
\subsubsection{Handling Troubles (Article 6)}
When receiving complaints from others (such as research subjects) during the research investigation stage, responding or acting carelessly on the spot carries the risk of further harming the other party or escalating the situation. If trouble occurs, it is strongly required to \textbf{report to the FD/SD Committee} immediately and discuss an organizational response.
\subsection{Application and Case Analysis}
The provisions of these guidelines apply to specific activities in the MBA program as follows:
\subsubsection{Case 1: Joint Research with an External Mentor}
Based on Article 2 (Scope of Application), if an MBA student invites an external expert as a joint researcher for their degree research (business plan), that expert is also considered subject to these guidelines as long as they use the university's resources (data, facilities, student results).
\subsubsection{Case 2: Data Citation in a Business Plan}
Based on Article 4 (Responsibilities of Researchers), even in a business plan created for an exercise, if data on market size, competitor analysis, or existing business models are referenced, the sources must be accurately cited, and \textbf{citation rules must be observed}. A presentation that cannot guarantee the credibility of its data may violate research ethics.
\subsubsection{Case 3: Responding to Complaints During Fieldwork}
Based on Article 6 (Handling Troubles), suppose a student receives a complaint during customer interviews or surveys, such as 'I am concerned about the handling of my personal information.' In this case, the student should not apologize or make promises (like data deletion) alone, but must first report the facts to their supervisor or the FD/SD Committee and discuss an official organizational response.
\subsection{Deeper Context and Lessons}
\textbf{\paragraph{Digression: The FD/SD Committee's 'Sanctuary' Function}}
In the lecture, in relation to Article 3 (Responsibilities of the Graduate School) and Article 6 (Handling Troubles), it was repeatedly emphasized that the \textbf{FD/SD Committee} is the point of contact for reports and consultations. This is intended to function as a safety net to prevent students and faculty from bearing ethical problems or troubles alone. Article 6, in particular, aims to prevent an individual's careless actions from escalating a situation, while also having an aspect of an organizational defense measure to protect the person reporting.
\textbf{\paragraph{Digression: The Lecturer's Intent in Emphasizing 'Item 4'}}
The lecturer emphasized the importance of Article 4 (Responsibilities of Researchers), especially Item 4 (data credibility, citation). Unlike clear illegal acts such as misuse of research funds (Item 2), \textbf{plagiarism} or improper citation and data interpretation errors are the ethical problems that MBA students are most likely to fall into in their daily academic activities of writing papers and reports. The lecturer's strong warning on this point reflects a recognition that it is extremely important for ensuring the quality of academic output.
\textbf{\subsubsection{AI Supplement: Expanding on Key Points}}
While the lecture text uses the term 'research misconduct,' in the context of research ethics, the most serious misconduct internationally is known as '\textbf{FFP}'.
\begin{itemize}
	\item \textbf{Fabrication}: Creating, recording, or reporting non-existent data or research results.
	\item \textbf{Falsification}: Manipulating research materials, equipment, or processes, or distorting data or results, to inaccurately represent the research.
	\item \textbf{Plagiarism}: Appropriating another person's ideas, processes, results, or words without giving appropriate credit (citation). This is directly linked to the 'observance of citation rules' emphasized in Item 4 of the lecture.
\end{itemize}
In addition to these, '\textbf{salami-slicing}' (unnecessarily splitting research results for publication) and '\textbf{inappropriate authorship}' (including individuals as co-authors who made no contribution) are also strictly questioned as broader research misconduct or 'questionable research practices (QRP)'.
\subsection{Conclusion}
The research ethics guidelines explained in this lecture are the foundation for preventing research misconduct and maintaining the soundness of the academic community. For MBA students in particular, ensuring \textbf{data credibility} and \textbf{observing citation rules} as stipulated in Article 4 are responsibilities that are directly tested in exercises and degree research.
As is clear from the 'Deeper Context' of this report, the practical lesson extends beyond 'knowing the ethical code.' The most important thing is to internalize the behavioral norm of \textbf{never bearing the burden alone when faced with ethical uncertainty or troubles as described in Article 6, but promptly reporting and consulting with the designated organizational contact point, such as the FD/SD Committee}. This is the only certain way to protect the researchers themselves and, at the same time, guarantee the credibility of the entire academic community.
\subsection{List of Important Keywords}
(No specific names other than the lecturer's appeared in the text of this lecture.)
\vspace{\baselineskip}
\noindent
\textbf{Research Ethics}, \textbf{Research Misconduct}, \textbf{Data Credibility}, \textbf{Citation}, \textbf{Legal Compliance}, \textbf{Personal Information Protection}, \textbf{Research Involving Human Subjects}, \textbf{Fabrication} (AI Supp.), \textbf{Falsification} (AI Supp.), \textbf{Plagiarism} (AI Supp.)
\subsection{Comprehension Check Quiz}
\begin{enumerate}
	\item What is the general term for the ethical norms and codes of conduct in research activities?
	\item Of the three 'F's' in FFP (Specific Misconduct), name two.
	\item What is the misconduct of creating non-existent experimental data or survey results from scratch called?
	\item What is the misconduct of modifying parts of existing experimental data to derive results favorable to oneself called?
	\item What is the misconduct of using others' papers or ideas as one's own without proper citation?
	\item What is the term for the rules and rights that determine the authors of a paper based on their contribution to the research results?
	\item What is the term for a situation where a researcher's financial interests (e.g., owning company stock) could potentially influence the fairness of research (e.g., judging a new drug's effectiveness)?
	\item In research involving human subjects (surveys, interviews, etc.), what is the consent obtained from research subjects after a full prior explanation called?
	\item What is the act of submitting substantially the same research content to multiple academic journals as separate papers called?
	\item What is the act of intentionally splitting a series of research data into small pieces to increase the number of publications colloquially called, based on its appearance?
	\item In an academic paper, besides demonstrating the 'novelty' of one's own claims, what is the other main role that 'citation' should play?
	\item What is the right that legally protects creative expressions of research output (papers, program code, business plans) and prohibits unauthorized use by others called?
	\item What is the obligation not to leak undisclosed information about individuals or companies obtained during the research process (e.g., interviews) to external parties called?
	\item When submitting a paper, what is the process where experts in the field anonymously review and evaluate the paper's content called?
	\item What is the practice of appropriately storing and managing raw data and lab notes for a certain period after research completion, in order to ensure research reproducibility and transparency, called?
\end{enumerate}
\subsubsection*{Answer Key}
1. Research Ethics, 2. Two of: Fabrication, Falsification, Plagiarism, 3. Fabrication, 4. Falsification, 5. Plagiarism, 6. Authorship, 7. Conflict of Interest (COI), 8. Informed Consent, 9. Duplicate Submission (or Multiple Submission), 10. Salami-slicing, 11. Expressing respect for prior research and providing evidence (basis) for one's own claims, 12. Copyright, 13. Confidentiality (or Duty of Confidentiality), 14. Peer Review, 15. Research Data Management (and/or Archiving)
\section{Guidelines for Research Involving Human Subjects}
\subsection{Introduction}
Research activities in an MBA program are often thought to be different from medical research. However, \textbf{questionnaire surveys} on people's behavior, the handling of \textbf{personal information}, and \textbf{interviews} often fall under 'research involving human subjects.' Even in these activities, research misconduct such as causing psychological distress to subjects, leaking sensitive personal information, and human rights violations can occur.
The \textbf{fundamental premise of the guidelines covered in this lecture is 'to protect the human subjects'}, and if this is not upheld, the research itself cannot proceed. Therefore, consideration for human rights issues, including social consent and the protection of subjects' privacy, is strongly demanded. This note aims to foster an understanding of the ethical regulations necessary for conducting research involving human subjects and the specific procedures (prior review system) at this university.
\subsection{Key Concepts and Issues}
\subsubsection{Definition of Research Involving Human Subjects (Article 3)}
'Research involving human subjects' in these guidelines refers to research activities conducted by collecting and analyzing information or data concerning individuals or groups, including their personal information, behavior, environment, or physical/mental state. As this definition shows, if the research subjects are human, a \textbf{wide range of fields in the humanities and social sciences} (including surveys and research in an MBA) are covered.
Article 3 also defines other important terms such as 'personal information,' 'research subject,' and '\textbf{anonymization}.' It is necessary to carefully check the content, including whether one's own research applies.
\subsubsection{Responsibilities of Researchers (Article 4)}
Researchers are required to respect the human rights of subjects, select appropriate research methods, and comply with related laws and guidelines. In particular, they have an obligation to thoroughly manage collected information and take sufficient measures to \textbf{prevent the leakage of personal information}.
\subsubsection{Explanation to and Consent from Research Subjects (Articles 5, 6)}
This is a core concept of research ethics, generally known as the '\textbf{Informed Consent}' process.
\begin{itemize}
	\item \textbf{Duty to Explain (Article 5)}: Researchers must provide a sufficient prior explanation of all matters that could affect the subjects' rights (research purpose, methods, handling of personal information, etc.). This requires a careful and detailed explanation tailored to the other party's position and situation, not just a template.
	\item \textbf{Consent by Free Will (Article 6)}: Based on the explanation, the \textbf{clear consent of the research subject, based on their free will}, must be obtained in advance.
\end{itemize}
\subsubsection{Rights of Research Subjects (Articles 7, 9)}
Important rights aimed at protecting the subjects are stipulated.
\begin{itemize}
	\item \textbf{Request for Disclosure (Article 7)}: Research subjects have the right to request disclosure of research content concerning themselves at any time.
	\item \textbf{Withdrawal of Consent (Article 9)}: Even after giving consent, research subjects can \textbf{withdraw their consent and stop cooperating at any time} during the research.
\end{itemize}
\subsubsection{Special Subjects and Third Parties (Articles 8, 11)}
\begin{itemize}
	\item \textbf{Subjects Requiring Special Consideration (Article 8)}: If the subject is a minor, ward, etc., \textbf{consent from their guardian or legal representative} is mandatory in addition to the subject's own consent.
	\item \textbf{Outsourcing to Third Parties (Article 11)}: When outsourcing tasks to a survey company or commissioning external transcription of interviews, that third party is also involved in the research and handles personal information, and thus bears the \textbf{obligation to comply} with these guidelines.
\end{itemize}
\subsubsection{Simplification of Explanation and Consent (Article 10)}
Strict consent procedures are not required in all cases; exception clauses exist.
\begin{itemize}
	\item \textbf{Implied Consent}: Cases where the questions do not place a particular burden on the subject, and the act of responding to the questionnaire itself can be deemed as consent (simplification).
	\item \textbf{Exemption from Procedures}: Cases that satisfy all the items listed in Article 10, Item 3 (e.g., research using only existing public information).
	\item \textbf{Exceptions in Special Fields (Item 4)}: For special research where conducting a prior explanation would make it difficult to achieve the research objective (e.g., certain observational studies), exceptions may be permitted under extremely strict requirements, but a check of the detailed provisions is essential.
\end{itemize}
\subsubsection{Application in Classes (Article 12)}
If students' personal information is collected and used for a specific research purpose in the course of group work or discussions in a graduate school class, these guidelines also apply.
\subsection{Application and Case Analysis: The Prior Review System Process}
In this lecture, the \textbf{prior review system} implemented at this university was explained in detail as a specific procedure for research involving human subjects.
\subsubsection{Principle of Prior Review}
If research falls under the category of research involving human subjects, it must, in principle, undergo a review before implementation to ensure the soundness of the research.
\subsubsection{Application Process (Before Starting Research)}
\begin{enumerate}
	\item \textbf{Application Deadline}: By the \textbf{20th of the month preceding} the planned research implementation.
	\item \textbf{Submission Method}: Submit the prescribed application form via email (as an attachment) to the \textbf{FD/SD Committee}.
\end{enumerate}
\subsubsection{Review Process}
\begin{enumerate}
	\item \textbf{Initial Judgment}: The FD/SD Committee determines 'whether the submission is subject to prior review' based on the submitted application.
	\item \textbf{Full Review}: If judged to be subject, the application is sent to the \textbf{Research Ethics Committee}, and a full review is conducted.
	\item \textbf{Notification of Results}: The review results are notified to the applicant via email from the FD/SD Committee, in principle, \textbf{within 3 weeks} from the application submission date.
\end{enumerate}
\subsubsection{Post-Research Procedures}
Approval for implementation by the review is not the end of the process. After the approved research is completed, a 'Completion Report' must be submitted via email to the FD/SD Committee \textbf{within one month}.
\subsubsection{Exceptions to Review}
There are also exceptions where prior review is not necessary, in cases where the soundness of the research is considered to be separately guaranteed.
\begin{itemize}
	\item Cases of joint research with a supervising faculty member.
	\item Cases where internal approval has already been obtained from the subject's affiliated organization (e.g., a company).
	\item Cases that fall outside the scope of this university's guidelines, such as research conducted in a personal capacity.
\end{itemize}
Regarding whether these exceptions apply, it is necessary to read Article 4's provisions carefully and confirm any unclear points.
\subsection{Deeper Context and Lessons}
\textbf{\paragraph{Digression: The Dual Role of the FD/SD Committee (Article 13)}}
In relation to Article 13 (Response to Complaints, etc.), the lecture emphasized that the \textbf{FD/SD Committee} serves as a consultation contact. Based on the premise that research involving human subjects is prone to complaints, it was suggested that the FD/SD Committee plays not only the role of an 'ethics guardian (ensuring compliance, protecting rights)' but also a research-support role as a '\textbf{follow-up actor for the smooth conduct of research}' by handling troubles smoothly. Contacting them promptly when trouble arises, or when there are signs of it, is important for the smooth execution of research.
\textbf{\paragraph{Digression: Risk Hedging for Consent Withdrawal (Article 9)}}
Regarding the fact that 'withdrawal of consent' is permitted under Article 9, the lecturer mentioned the \textbf{research-related risk} that 'it becomes difficult to continue the research.' As a countermeasure, the lecturer stated that 'one must \textbf{pay sufficient attention to the selection of subjects and the explanation of the situation in advance}.' This indicates the importance of practical \textbf{risk management}—designing careful communication from the initial stages of research design so that subjects do not harbor anxiety or misunderstandings—which goes beyond mere formal compliance with ethical rules.
\textbf{\subsubsection{AI Supplement: Expanding on Key Points}}
While 'Informed Consent' was a central topic in this lecture, related concepts that are ethically important, especially in MBA research (e.g., interviews with one's own company or clients), are '\textbf{Authenticity}' and '\textbf{Reciprocity}'.
\begin{itemize}
	\item \textbf{Authenticity}: Researchers must truthfully disclose their identity, the true purpose of the research, and how the data will be used (e.g., the possibility of it being published as a thesis). Especially in cases like research on one's own company's issues, using one's position or status to extract unofficial information can impede the subject's free will and is considered ethically problematic.
	\item \textbf{Reciprocity}: Research should not be a one-way street where information is merely 'taken' from subjects. It is desirable to have \textbf{bidirectionality}, where some benefit (not limited to money) is returned to the subjects, such as by ensuring the interview provides an opportunity for the subject's own reflection, or by feeding back a summary of the research results (to the extent possible).
\end{itemize}
\subsection{Conclusion}
As shown in this lecture, the surveys and interviews conducted by MBA students clearly fall under 'research involving human subjects,' which is premised on the protection of subjects' human rights. Therefore, researchers have a duty to deeply understand the guidelines and, in particular, to practice the proper acquisition of \textbf{Informed Consent} (Articles 5, 6) and the absolute guarantee of subjects' rights (Article 7: Disclosure, Article 9: Withdrawal of Consent).
Furthermore, it is required to accurately grasp and follow the procedures of this university's unique \textbf{prior review system} (Application deadline: 20th of the previous month; Completion report: within 1 month after completion).
The practical lesson from the 'Deeper Context' of this report is the importance of not just formally observing ethical rules, but also designing careful communication with subjects from the initial research design stage to avoid research-related risks like those in Article 9 (Withdrawal of Consent). Moreover, based on Article 13 (Complaint Handling), promptly consulting the \textbf{FD/SD Committee} at the sign of trouble and not bearing problems alone is the key to protecting both the researcher and the soundness of the research.
\subsection{List of Important Keywords}
(No specific names other than the lecturer's appeared in the text of this lecture.)
\vspace{\baselineskip}
\noindent
\textbf{Research Involving Human Subjects}, \textbf{Research Ethics}, \textbf{Informed Consent} (Explanation and Consent), \textbf{Privacy Protection}, \textbf{Human Rights}, \textbf{Personal Information}, \textbf{Anonymization}, \textbf{Withdrawal of Consent}, \textbf{Authenticity} (AI Supp.), \textbf{Reciprocity} (AI Supp.)
\subsection{Comprehension Check Quiz}
\begin{enumerate}
	\item What is the process of providing research subjects with a full prior explanation of the research purpose, methods, risks, etc., and obtaining their consent based on free will called?
	\item In research involving human subjects, what is the most important fundamental premise that research ethics must protect?
	\item If a research subject is a minor, whose consent is required in addition to the subject's own consent?
	\item What is the process of removing information that can identify a specific individual (such as name or address) from collected data so that the individual cannot be identified called?
	\item Research subjects can withdraw their consent at any time without suffering disadvantages, even after agreeing to cooperate in the research. What is this right called?
	\item When an MBA student interviews colleagues at their own company, which concept mentioned in the AI supplement does the ethical attitude of truthfully explaining their identity and the true purpose of the research correspond to?
	\item What did the AI supplement call the idea of considering the research to be beneficial for the subjects as well, such as by feeding back a summary of the research results to the participating company?
	\item In a questionnaire survey, what is the procedure called where the act of answering and submitting the questionnaire itself is deemed as consent to the research?
	\item If the transcription of interviews is outsourced to an external vendor, what obligation must that vendor also comply with?
	\item Do research subjects have the right to request disclosure of data concerning themselves held by the researcher?
	\item In research involving human subjects, should researchers strive to minimize the physical and psychological burden and distress on the subjects?
	\item Does the definition of 'research involving human subjects' include psychological surveys and sociological interviews?
	\item What is the system explained in this lecture, where approval must be obtained from a university committee before implementing research, called?
	\item If an ethical problem or trouble with a research subject occurs (or there are signs of it), what is the most appropriate action for the researcher to take?
	\item After research is approved and completed, what procedure must the researcher perform for the committee?
\end{enumerate}
\subsubsection*{Answer Key}
1. Informed Consent (Explanation and Consent), 2. Protecting the human rights of research subjects (or privacy protection), 3. A guardian or legal representative, 4. Anonymization (or De-identification), 5. The right to withdraw consent, 6. Authenticity, 7. Reciprocity, 8. Implied consent (or simplification of consent), 9. The obligation to comply with research ethics (or confidentiality), 10. Yes, they do, 11. Yes, they should, 12. Yes, it does, 13. Prior review system, 14. Promptly contact and consult the FD/SD Committee (or the designated contact point) and not respond individually, 15. Submit a completion report (within the specified deadline, e.g., within one month)
\end{document}
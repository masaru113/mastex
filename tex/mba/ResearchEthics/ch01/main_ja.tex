\documentclass[uplatex,a4j,12pt,dvipdfmx]{jsarticle}
\usepackage{amsmath,amsthm,amssymb,bm,color,enumitem,mathrsfs,url,epic,eepic,ascmac,ulem,here,ascmac}
\usepackage[letterpaper,top=2cm,bottom=2cm,left=3cm,right=3cm,marginparwidth=1.75cm]{geometry}
\usepackage[english]{babel}
\usepackage[dvipdfm]{graphicx}
\usepackage[hypertex]{hyperref}
\title{研究倫理 第1回 講義ノート}
\author{M. O.}
\date{\today}

\begin{document}
\maketitle
\tableofcontents

\section{研究倫理の概要}

\subsection{はじめに}
本レポートは、経営大学院における研究倫理の重要性と、具体的な研究活動における規範の遵守について解説した講義音声に基づき作成する。目的は、学術論文に限定されない広範な研究活動(レポート、事業計画、アンケート調査など)において、受講生が倫理的規範を深く理解し、関連する手続きやトラブル対応について把握することである。特に、研究倫理全般に関するガイドラインと、人を対象とする研究に特化したガイドラインの二つの柱を明確にし、その適用範囲と重要性を整理する。

\subsection{主要な概念と論点}
\subsubsection{研究倫理の定義と必要性}
\paragraph{研究倫理}
研究倫理とは、研究を行う際に研究者が遵守しなければならない規範である。
\paragraph{研究の範囲}
大学院における研究は、特定の物事の真実を明らかにしたり、新しい知識を発見・創造したりする活動全般を指す。学術論文やリサーチペーパーはもちろん、受講中に課される各レポートや事業計画演習なども立派な研究活動であり、研究倫理の規範に従う必要がある。
\paragraph{研究倫理の必要性}
企業におけるコンプライアンスと同様に、研究倫理は研究の信憑性を確保するために不可欠である。研究者の自律性は尊重されるが、研究不正を防ぎ、健全な研究環境を維持するために、各研究機関で研究倫理が規定されている。

\subsubsection{研究不正の類型と具体例}
研究の不正は大きく以下の2つに分けられる。
\begin{enumerate}
	\item 研究活動の不正:
	      \begin{itemize}
		      \item 捏造(Fabrication): 実際にはないデータを人為的に作り出す行為。
		      \item 改ざん(Falsification): データや結果を不当に加工し、真実を歪める行為。
		      \item 盗用(Plagiarism): 他人の研究成果(文章、図表、アイデアなど)を、適切な引用なしに自分のものとして発表する行為。大学院でのレポート・論文作成における他者著作物のコピー\&ペーストはこれに該当し、厳重に注意が必要である。
	      \end{itemize}
	\item 研究費の不正使用:
	      \begin{itemize}
		      \item 架空請求や目的外の使用などが該当する。
	      \end{itemize}
\end{enumerate}

\subsubsection{研究倫理の遵守義務と違反時の処分}
\paragraph{遵守の範囲}
研究倫理は、研究の立案から計画・申請・実施・報告までの全段階において守る必要がある。
\paragraph{違反時の処分}
不正行為は「知らなかった」「気づかなかった」では済まされない。違反した場合は、解雇・退学などの懲戒処分、社会における信用失墜を招く。また、他人の権利や財産に損害を与えた場合は、民事訴訟や刑事訴追に至るケースもあるため、自らの責任と自覚を持って取り組むことが求められる。

\subsubsection{二つの研究倫理規定}
本講義では、以下の二つの規定の存在が指摘された。
\begin{itemize}
	\item 研究倫理ガイドライン: 研究倫理全般について記載されている。
	\item 人を対象とする研究倫理ガイドライン: 個人情報など特別に配慮すべき事項や、事前の審査設計など、人間を研究対象とする場合に特化した規定。
\end{itemize}



\subsection{応用と事例分析}
\subsubsection{大学院における盗用事例}
本講義では、大学院生が日常的に陥りやすい盗用(Plagiarism)について特に注意が促された。
\begin{itemize}
	\item \textbf{日常的なSNSでのコピペとの違い}: 一般的なSNS投稿などでは他者の文章を気軽にコピー&ペーストすることがあるかもしれないが、大学院のレポートや論文において、他者が書いたものを引用ルールを守らずに自分の著作物として提出する行為は、重大な盗用行為として扱われる。
	\item \textbf{適用範囲}: これは、\textbf{事業計画の作成}など、一見すると学術研究から離れた活動においても同様に適用される。例えば、既存の事業計画書や市場分析レポートの記述を無断で流用することは、研究倫理違反となる。
\end{itemize}
\subsubsection{不正防止のための体制と仕組み}
研究機関が研究倫理を確立し、不正を防止するために、本講義で言及されたように、多くの研究機関では、ガイドラインの策定、倫理委員会の設置、そして特に人を対象とする研究に対する事前審査制度が設けられている。これらは、研究者が倫理的リスクを理解し、研究の透明性と公正性を確保するための仕組みとして機能する。



\subsection{深層背景と教訓}
\textbf{\paragraph{研究とコンプライアンスのパラレル}}
本講義では、研究倫理の遵守は、ビジネスにおけるコンプライアンスの遵守と本質的に同じであるというアナロジーが用いられた。健全な企業運営にコンプライアンスが不可欠であるように、研究の信憑性には研究倫理が不可欠である。この比較は、経営大学院生にとって研究倫理を専門職としての責任と捉える上で非常に有効である。

\textbf{\paragraph{研究不正の社会的影響}}
捏造や改ざんといった研究活動の不正は、単に研究機関内の問題に留まらず、その成果が社会に適用されることで公衆衛生や経済活動に深刻な損害を与える可能性がある。また、研究費の不正使用は、税金を含む公的資金の信頼を損なう行為であり、研究者に対する社会の信頼を大きく失墜させる。

\textbf{\subsubsection{AIによる補足:重要論点の拡張}}
本講義の文字起こしテキスト中では、研究倫理の核となる概念の一つであるインフォームド・コンセント(Informed Consent)に関する具体的な言及が漏れている。これは「人を対象とする研究倫理ガイドライン」において最も重要な要素の一つであるため、以下に補足する。
\paragraph{インフォームド・コンセント(IC)}
インフォームド・コンセントとは、「情報を十分に与えられた上での同意」を意味し、研究の対象となる人々(参加者)が、研究の目的、方法、予見される利益と不利益、参加の任意性(いつでも同意を撤回できること)などを事前に詳細に説明され、自由意思に基づいて研究への参加に同意することを指す。特に経営学研究におけるアンケート調査やインタビュー調査など、人を対象とする研究では、この手続きが研究開始前に適切に行われることが、倫理的な要件として厳格に求められる。



\subsection{結論}
本講義は、経営大学院における研究倫理の重要性を再確認するものであり、その適用範囲が学術論文に留まらず、レポートや事業計画を含む全ての研究活動に及ぶことを強調した。特に、捏造、改ざん、盗用といった研究活動の不正は、深刻な懲戒処分や社会的信用の失墜につながるため、受講生は自らの責任と自覚を持って研究に取り組む必要がある。

深層背景と補足情報から得られる実践的な教訓は、専門職としての倫理観の確立である。ビジネスにおけるコンプライアンスと同様に、研究活動においても透明性と誠実性が求められる。特に、人を対象とする研究では、インフォームド・コンセントを徹底するなど、対象者の権利と尊厳を最大限に尊重する姿勢が不可欠となる。これは、将来的に企業倫理やCSRを担う経営者・専門家としての基盤を築く上で、極めて重要な学習である。

\subsection{重要キーワード一覧}
研究倫理, 研究不正, 捏造, 改ざん, 盗用, インフォームド・コンセント
\vspace{\baselineskip}
コンプライアンス, リサーチペーパー, 懲戒処分, 民事訴訟, 刑事訴追, ガイドライン, 倫理委員会

\subsection{理解度確認クイズ}
\begin{enumerate}
	\item 研究活動の公正性を確保するために、研究者が守るべき規範を何と呼びますか。
	\item 実際には得られていないデータを人為的に作り出し、それを研究成果として発表する不正行為を何と呼びますか。
	\item 既存の研究論文やレポートの文章を、適切な引用をせずに自分の著作物であるかのように発表する不正行為を何と呼びますか。
	\item 研究活動の不正は、研究活動の不正ともう一つの不正に大別されます。もう一つはどのような不正ですか。
	\item 経営大学院で作成するレポートや事業計画は、研究倫理の遵守の対象となりますか、なりませんか。
	\item 研究不正を防止し、研究の信憑性を確保するために研究倫理が必要とされるのは、ビジネスにおける何に相当すると本講義では説明されましたか。
	\item 研究倫理に違反する行為があった場合、「知らなかった」という弁明は通用しますか、しませんか。
	\item 研究不正によって懲戒処分を受けた場合、社会的にどのような影響を受けることが指摘されましたか。
	\item 研究費の不正使用の一例として、本講義で挙げられたものを一つ述べてください。
	\item 研究倫理の遵守は、研究のどの段階からどの段階まで求められますか。
	\item 人を対象とする研究において、参加者に研究内容やリスクを十分に説明し、自由意思に基づいて同意を得る手続きを何と呼びますか。
	\item 経営学研究におけるアンケート調査は、「研究倫理ガイドライン」と「人を対象とする研究倫理ガイドライン」のどちらに特に関連性が高いですか。
	\item 研究者が研究を行う上で、常に求められる基本姿勢を、本講義では「自らの責任と\underline{\hspace{1cm}}」と表現しました。空欄に入る言葉は何ですか。
	\item 研究不正が、個人の信用失墜だけでなく、他人の権利や財産に損害を与えた場合に発展する可能性のある法的手段を2つ述べてください。
	\item 研究倫理の必要性は、研究者の自律性を尊重するという原則と、研究不正を防ぐという目的のどちらに重きを置いて説明されましたか。
\end{enumerate}

\subsubsection*{解答一覧}
1.研究倫理, 2.捏造, 3.盗用, 4.研究費の不正使用, 5.なります, 6.コンプライアンス, 7.しません, 8.社会における信用を失う, 9.架空請求または目的外の仕様, 10.立案から報告まで(または全段階), 11.インフォームド・コンセント, 12.人を対象とする研究倫理ガイドライン, 13.自覚, 14.民事訴訟と刑事訴追, 15.研究不正を防ぐという目的


\section{研究倫理ガイドライン}

\subsection{はじめに}
研究倫理は、本来規定がなくとも遵守すべき普遍的な規範であるが、近年、研究不正の防止が社会的な重要課題となっている。このため、各教育研究機関において具体的なガイドラインを設ける取り組みが不可欠である。本ガイドラインは、そうした背景のもと、本学における\textbf{研究不正を防止}し、研究活動の健全性を確保するために設けられた(第1条)。本ノートは、このガイドラインの主要な論点を整理し、MBAでの学習と実践における倫理的基盤を再確認することを目的とする。

\subsection{主要な概念と論点}

\subsubsection{適用対象(第2条)}
本ガイドラインの適用対象は「研究者」と「研究活動」について定義される。
\begin{itemize}
	\item \textbf{研究者}: 本学の専任教員や学生に限らず、非常勤講師、客員教員、研究員、さらには外部からの\textbf{共同研究者}など、本学において研究活動に従事するすべての関係者が含まれる。
	\item \textbf{研究活動}: 講義コンテンツの開発のみならず、MBAにおける学位研究(事業研究演習や組織変革演習など)も、本ガイドラインが適用される重要な研究活動である。
\end{itemize}

\subsubsection{大学院の責務(第3条)}
大学院は、研究の健全性を確保するための責務を負う。この責務は主に\textbf{FD・SD委員会}(Faculty Development / Staff Development 委員会)が中心となって遂行される。研究倫理に関する不安、質問、あるいは研究不正の発見、その他のトラブルが発生した場合は、速やかにFD・SD委員会に連絡することが求められる。

\subsubsection{研究者の責務(第4条)}
本ガイドラインの中核であり、研究者が遵守すべき具体的な責務(全10項)が定められている。
\begin{itemize}
	\item \textbf{研究行為(第1項)}: 研究の遂行における基本的な倫理。
	\item \textbf{研究費の使用(第2項)}: 公正かつ透明性のある経費執行。
	\item \textbf{契約締結(第3項)}: 契約上の義務の遵守。
	\item \textbf{研究発表(第4項)}: \textbf{特に注意すべき最重要項目}。論文や事業計画書を発表する際、\textbf{データの信憑性}の担保と、\textbf{引用ルールの遵守}に細心の注意を払う必要がある。
	\item \textbf{その他}: 法令遵守、個人情報の適切な取り扱いなどが定められている。
	\item \textbf{人を対象とする研究(第7項)}: 対象者の人権擁護などに関する規定(本講義では詳細割愛)。
\end{itemize}

\subsubsection{研究指導者の責務(第5条)}
指導に当たる教員も、学生の研究活動が倫理的に行われるよう、責任を持って助言・指導する責務を負う。

\subsubsection{トラブル対応(第6条)}
研究調査の段階で他人(調査対象者など)から苦情やクレームを受けた際、その場で安易に対応・行動することは、相手をさらに傷つけたり、事態を深刻化させたりするリスクがある。トラブルが発生した場合は、必ず速やかに\textbf{FD・SD委員会に報告}し、組織としての対応を協議することが強く求められる。

\subsection{応用と事例分析}
本ガイドラインの規定は、MBAにおける具体的な活動において以下のように適用される。

\subsubsection{事例1:外部メンターとの共同研究}
第2条(適用対象)に基づき、MBAの学生が学位研究(事業計画)のために外部の専門家を共同研究者として招聘した場合、その専門家も本学のリソース(データ、施設、学生の成果)を利用する限りにおいて、本ガイドラインの適用対象者とみなされる。

\subsubsection{事例2:事業計画書におけるデータ引用}
第4項(研究者の責務)に基づき、演習で作成する事業計画書であっても、市場規模のデータ、競合分析、あるいは既存のビジネスモデルを参照する場合、その出典を正確に明記し、\textbf{引用ルールを遵守}しなければならない。データの信憑性を担保できない発表は、研究倫理に反する可能性がある。

\subsubsection{事例3:フィールドワーク中の苦情対応}
第6条(トラブル対応)に基づき、顧客インタビューやアンケート調査の実施中に、調査対象者から「個人情報の扱いに不安がある」といった苦情を受けたとする。この場合、学生は単独で謝罪や(データ削除などの)約束をせず、まず指導教員やFD・SD委員会に事実を報告し、組織としての正式な対応を協議することが求められる。

\subsection{深層背景と教訓}

\textbf{\paragraph{本論から逸れた寄り道トピック名:FD・SD委員会の「駆け込み寺」機能}}
本講義では、第3条(大学院の責務)および第6条(トラブル対応)において、\textbf{FD・SD委員会}が報告・相談の窓口であることが繰り返し強調された。これは、研究倫理上の問題やトラブルを、学生や教員が個人で抱え込むことを防ぐためのセーフティネットとして機能させる意図がある。特に第6条のトラブル対応は、個人の安易な行動が事態を深刻化させることを防ぐと同時に、報告者を守るための組織的な防衛策としての側面も持っている。

\textbf{\paragraph{本論から逸れた寄り道トピック名:講師が特に強調する「第4項」の意図}}
講師は第4条(研究者の責務)の中でも、特に第4項(データの信憑性、引用)の重要性を強調した。第2項の研究費不正のような明確な違法行為とは異なり、\textbf{盗用(Plagiarism)}や不適切な引用、データの解釈ミスは、MBA学生が論文やレポート作成という日常的な学術活動において最も陥りやすい倫理的問題である。この点を強く警告することは、学術的な成果物の品質を担保する上で極めて重要であるとの講師の認識がうかがえる。

\textbf{\subsubsection{AIによる補足:重要論点の拡張}}
本講義テキストでは「研究不正」という言葉が使われているが、研究倫理の文脈で国際的に最も重い不正行為とされるのが「\textbf{特定不正行為(FFP)}」である。
\begin{itemize}
	\item \textbf{捏造 (Fabrication)}: 存在しないデータや研究結果を作成し、記録または報告すること。
	\item \textbf{改ざん (Falsification)}: 研究資料、機器、プロセスを操作し、データや結果を歪めて、研究を不正確に表現すること。
	\item \textbf{盗用 (Plagiarism)}: 他者のアイデア、プロセス、結果、または言葉を、適切なクレジット(引用)なしに流用すること。これは講義中の第4項で強調された「引用ルールの遵守」に直結する。
\end{itemize}
これらに加え、研究成果を不必要に分割して発表する「\textbf{サラミ・スライシング}」や、貢献のない人物を共著者にする「\textbf{不適切なオーサーシップ}」なども、広義の研究不正または「問題ある研究行為(QRP)」として厳しく問われる。

\subsection{結論}
本講義で解説された研究倫理ガイドラインは、研究不正を未然に防ぎ、学術コミュニティの健全性を維持するための基盤である。特にMBA学生にとっては、第4条で定められた「\textbf{データの信憑性}」の担保や「\textbf{引用ルールの遵守}」が、演習や学位研究において直接的に問われる責務となる。

本レポートの「深層背景」からも明らかなように、実践的な教訓は「倫理規定を熟知すること」に留まらない。最も重要なのは、倫理的な判断に迷った場合や、第6条にあるような\textbf{トラブルに直面した場合に、決して一人で抱え込まず、FD・SD委員会のような定められた組織的窓口へ速やかに報告・相談する}という行動規範を身につけることである。これが、研究者自身を守り、同時に学術コミュニティ全体の信頼性を担保する唯一確実な方法と言える。

\subsection{重要キーワード一覧}
(本講義のテキスト内には、講師以外の特定の人名は登場しなかった。)

\vspace{\baselineskip}
\noindent
\textbf{研究倫理}、\textbf{研究不正}、\textbf{データの信憑性}、\textbf{引用}、\textbf{法令遵守}、\textbf{個人情報保護}、\textbf{人を対象とする研究}、\textbf{捏造}(AI補足)、\textbf{改ざん}(AI補足)、\textbf{盗用}(AI補足)

\subsection{理解度確認クイズ}
\begin{enumerate}
	\item 研究活動における倫理的な規範や行動指針の総称を何と呼ぶか。
	\item 特定不正行為(FFP)の「F」が示す3つの不正行為(英語)のうち、2つを挙げよ。
	\item 存在しない実験データや調査結果をゼロから作成する不正行為を何と呼ぶか。
	\item 既存の実験データの一部に修正を加え、自身に都合の良い結果を導き出す不正行為を何と呼ぶか。
	\item 他者の論文やアイデアを、適切な引用表示なしに自分のものとして使用する不正行為を何と呼ぶか。
	\item 研究成果に対する貢献度に基づいて論文の著者を決定するルールや権利を何と呼ぶか。
	\item 研究者の経済的利益(例:企業の株式保有)が、研究の公正性(例:新薬の効果判定)に影響を及ぼす可能性がある状態を何と呼ぶか。
	\item 人を対象とする研究(アンケート、インタビューなど)において、研究対象者から事前に十分な説明の上で得る同意を何と呼ぶか。
	\item 実質的に同一の研究内容を、複数の学術雑誌に別々の論文として投稿する行為を何と呼ぶか。
	\item 一連の研究データを意図的に細かく分割し、論文数を増やすために発表する行為を、その見た目から俗に何と呼ぶか。
	\item 学術論文において「引用」が果たすべき、自身の主張の「新規性」を示す以外の主要な役割は何か。
	\item 研究成果(論文、プログラムコード、事業計画書)の表現物を法的に保護し、他者による無断利用を禁じる権利を何と呼ぶか。
	\item 研究プロセス(例:インタビュー)で得られた、個人や企業の非公開情報を外部に漏らさない義務を何と呼ぶか。
	\item 論文投稿時、その分野の専門家が論文の内容を匿名で審査・評価するプロセスを何と呼ぶか。
	\item 研究の再現性や透明性を担保するため、研究終了後も一定期間、生データや実験ノートを適切に保管・管理することを何と呼ぶか。
\end{enumerate}

\subsubsection*{解答一覧}
1. 研究倫理、2. Fabrication (捏造), Falsification (改ざん), Plagiarism (盗用) のうち2つ、3. 捏造 (Fabrication)、4. 改ざん (Falsification)、5. 盗用 (Plagiarism)、6. オーサーシップ、7. 利益相反 (COI: Conflict of Interest)、8. インフォームド・コンセント、9. 二重投稿(多重投稿)、10. サラミ・スライシング、11. 先行研究への敬意の表明と、自身の主張の論拠(証拠)の提示、12. 著作権、13. 守秘義務、14. ピア・レビュー(査読)、15. 研究データの管理(・保存)


\section{人を対象とする研究倫理ガイドライン}

\subsection{はじめに}
MBAにおける研究活動は、医学系研究とは異なると考えられがちである。しかし、人々の行動に関する\textbf{アンケート調査}、\textbf{個人情報}の取り扱い、\textbf{インタビュー取材}など、その多くが「人を対象とする研究」に該当する。これらの活動においても、対象者への精神的苦痛、機微な個人情報の漏洩、人権被害といった研究不正は生じ得る。

本講義で扱うガイドラインの\textbf{大前提は「対象者となる人を守る」}ことであり、これが守られなければ研究自体が進められなくなる。したがって、社会的な同意や被調査者のプライバシー保護を含む人権問題への配慮が強く求められる。本ノートは、人を対象とする研究の遂行に必要な倫理規定、および本学における具体的な手続き(事前審査制度)の理解を目的とする。

\subsection{主要な概念と論点}

\subsubsection{人を対象とする研究の定義(第3条)}
本ガイドラインにおける「人を対象とする研究」とは、個人または集団を対象に、個人情報、行動、環境、心身に関する情報やデータを収集・解析して行われる研究活動を指す。この定義からわかるように、研究対象者が人であれば、\textbf{人文社会科学の広範な領域}(MBAでの調査・研究を含む)が対象となる。
第3条では、その他「個人情報」「研究対象者」「\textbf{匿名化}」などの重要用語が定義されており、自身の研究が該当するかどうかを含め、内容を正確に確認する必要がある。

\subsubsection{研究者の責務(第4条)}
研究者には、対象者の人権の尊重、適切な研究方法の選定、関連法令・指針の遵守が求められる。特に、収集した情報の管理を徹底し、\textbf{個人情報の流出を防止}するために十分な措置を講じる義務を負う。

\subsubsection{研究対象者への説明と同意(第5条、第6条)}
研究倫理の中核をなす概念であり、一般に「\textbf{インフォームド・コンセント}」と呼ばれるプロセスである。
\begin{itemize}
	\item \textbf{説明義務(第5条)}: 研究者は、対象者の権利に影響を与え得る全事項(研究目的、方法、個人情報の扱い等)について、事前に十分な説明を行わなければならない。単なる雛形の説明ではなく、相手の立場や状況に合わせた、丁寧かつ詳細な説明が求められる。
	\item \textbf{自由意志による同意(第6条)}: 説明に基づき、研究対象者の\textbf{自由意志による明確な同意}を事前に得なければならない。
\end{itemize}

\subsubsection{研究対象者の権利(第7条、第9条)}
対象者の保護を目的とした重要な権利が定められている。
\begin{itemize}
	\item \textbf{開示請求(第7条)}: 研究対象者は、自身に関する研究内容について、いつでも開示を求める権利を有する。
	\item \textbf{同意の撤回(第9条)}: 研究対象者は、一度同意した後でも、研究の途中において\textbf{いつでも同意を撤回し、協力をやめる}ことができる。
\end{itemize}

\subsubsection{特殊な対象者と第三者(第8条、第11条)}
\begin{itemize}
	\item \textbf{配慮が必要な対象者(第8条)}: 対象者が未成年者や被後見人などである場合、本人の同意に加え、その\textbf{保護者または後見人による同意}が必須となる。
	\item \textbf{第三者への委託(第11条)}: 調査会社への業務委託や、インタビューの文字起こしを外部に依頼する場合、その第三者も研究に関与し個人情報を把握するため、本ガイドラインの\textbf{遵守義務}を負う。
\end{itemize}

\subsubsection{説明と同意の簡略化(第10条)}
すべての場合で厳格な同意手続きが必要なわけではなく、例外規定が存在する。
\begin{itemize}
	\item \textbf{みなし同意}: 質問内容が対象者に特に負担を与えず、アンケートへの回答という行為自体をもって同意したとみなせる場合(簡略化)。
	\item \textbf{手続きの免除}: 第10条第3項に挙げられる事項(例:既存の公開情報のみを用いる研究など)をすべて満たす場合。
	\item \textbf{特殊分野の例外(第4項)}: 事前説明を行うと研究目的の達成が困難になる特殊な研究(例:特定の行動観察など)については、極めて厳格な要件下で例外が認められる場合があるが、詳細な規定の確認が必須である。
\end{itemize}

\subsubsection{授業における適用(第12条)}
大学院の授業において、グループワークやディスカッションの過程で、特定の研究目的のために受講生の個人情報等を収集・利用する場合も、本ガイドラインの適用対象となる。

\subsection{応用と事例分析:事前審査制度のプロセス}
本講義では、人を対象とする研究の具体的な手続きとして、本学で導入されている\textbf{事前審査制度}が詳細に解説された。

\subsubsection{事前審査の原則}
人を対象とする研究に該当する場合、研究の健全性を確保するため、原則として研究実施前に審査を受ける必要がある。

\subsubsection{申請プロセス(研究開始前)}
\begin{enumerate}
	\item \textbf{申請期限}: 研究実施予定の\textbf{前の月の20日}まで。
	\item \textbf{提出方法}: 所定の申請書を、\textbf{FD・SD委員会}宛に電子メール(添付)にて提出する。
\end{enumerate}

\subsubsection{審査プロセス}
\begin{enumerate}
	\item \textbf{一次判断}: FD・SD委員会が、提出された申請書に基づき「事前審査の対象となるか」を判断する。
	\item \textbf{本審査}: 対象となると判断された場合、申請書は\textbf{研究倫理委員会}に送付され、本審査が実施される。
	\item \textbf{結果通知}: 審査結果は、申請提出日から原則として\textbf{3週間以内}に、FD・SD委員会から申請者へ電子メールで通知される。
\end{enumerate}

\subsubsection{研究終了後の手続き}
審査による実施承認は、手続きの終わりではない。承認された研究が終了した後、\textbf{1ヶ月以内}に「終了報告書」をFD・SD委員会に電子メールで提出しなければならない。

\subsubsection{審査対象の例外}
研究の健全性が別個に担保されていると考えられる場合、事前審査が不要となる例外も存在する。
\begin{itemize}
	\item 指導教員との共同研究である場合。
	\item 対象者の所属機関(例:企業)による内部の承認を既に得ている場合。
	\item 個人名義で行う研究など、本学のガイドライン適用対象外となる場合。
\end{itemize}
これらの例外に該当するか否かについても、第4条の規定を熟読し、不明点は確認する必要がある。

\subsection{深層背景と教訓}

\textbf{\paragraph{本論から逸れた寄り道トピック名:FD・SD委員会のデュアル・ロール(第13条)}}
講義では、第13条(苦情等受理事の対応)に関連し、\textbf{FD・SD委員会}が相談窓口であることが強調された。人を対象とする研究は苦情が生じやすいという前提に立ち、FD・SD委員会は単なる「倫理の番人(遵守確保、権利保護)」としての役割だけでなく、トラブルを円滑に処理し「\textbf{研究活動を順調に行うためのフォロー役}」という、研究支援的な役割も果たしている点が示唆された。トラブル発生時、あるいはその兆候がある場合に速やかに連絡することが、研究の円滑な遂行のために重要である。

\textbf{\paragraph{本論から逸れた寄り道トピック名:同意撤回のリスクヘッジ(第9条)}}
第9条で「同意の撤回」が認められている点について、講師は「研究を続けることが難しくなる」という\textbf{研究遂行上のリスク}に言及した。その対策として「\textbf{事前に対象者の選定や状況説明に十分に注意}しなければならない」と述べた。これは、単に倫理規定を守るという形式的な遵守を超え、研究デザインの初期段階において、対象者が不安や誤解を抱かぬよう丁寧なコミュニケーション設計を行うという、実践的な\textbf{リスクマネジメント}の重要性を示している。

\textbf{\subsubsection{AIによる補足:重要論点の拡張}}
本講義では「インフォームド・コンセント」が中心的な論点であったが、特にMBAの研究(例:自社や顧客へのインタビュー)において、倫理的に重要となる関連概念が「\textbf{真正性(Authenticity)}」と「\textbf{相互作用性(Reciprocity)}」である。
\begin{itemize}
	\item \textbf{真正性 (Authenticity)}: 研究者は、自身の身元、研究の真の目的、データの利用方法(例:学位論文として公開される可能性)を偽りなく開示する必要がある。特に自社の課題研究などの場合、自身の役職や立場を利用して非公式な情報を引き出すことは、対象者の自由意志を阻害する可能性があり、倫理的に問題視される。
	\item \textbf{相互作用性 (Reciprocity)}: 研究は、対象者から情報を「奪う」だけの一方的なものであってはならない。インタビューが対象者自身の内省の機会となるよう配慮する、あるいは研究成果のサマリーを(可能な範囲で)対象者にフィードバックするなど、対象者にも何らかの利益(金銭に限らない)が還元されるような\textbf{双方向性}が望ましいとされる。
\end{itemize}

\subsection{結論}
本講義で示された通り、MBA学生が実施するアンケートやインタビューも、対象者の人権保護を大前提とする「人を対象とする研究」に明確に該当する。したがって、研究者はガイドラインを深く理解し、特に\textbf{インフォームド・コンセント}(第5条、第6条)の適切な取得と、対象者の権利(第7条:開示、第9条:同意の撤回)の絶対的な保証を実践する責務を負う。

さらに、本学独自の\textbf{事前審査制度}のプロセス(申請期限:前月20日、終了報告:終了後1ヶ月以内)を正確に把握し、手続きを履践することが求められる。

本レポートの「深層背景」から得られる実践的な教訓は、倫理規定を形式的に守るだけでなく、第9条(同意の撤回)のような研究遂行上のリスクを回避するため、研究デザインの初期段階から対象者との丁寧なコミュニケーションを設計することの重要性である。また、第13条(苦情対応)に基づき、トラブルの兆候があれば速やかに\textbf{FD・SD委員会}に相談し、問題を個人で抱え込まないことが、研究者自身と研究の健全性を守る鍵となる。

\subsection{重要キーワード一覧}
(本講義のテキスト内には、講師以外の特定の人名は登場しなかった。)

\vspace{\baselineskip}
\noindent
\textbf{人を対象とする研究}、\textbf{研究倫理}、\textbf{インフォームド・コンセント}(説明と同意)、\textbf{プライバシー保護}、\textbf{人権}、\textbf{個人情報}、\textbf{匿名化}、\textbf{同意の撤回}、\textbf{真正性}(AI補足)、\textbf{相互作用性}(AI補足)

\subsection{理解度確認クイズ}
\begin{enumerate}
	\item 研究対象者に対し、研究の目的、方法、リスク等を事前に十分説明し、自由意志に基づいた同意を得るプロセスを何と呼ぶか。
	\item 人を対象とする研究において、研究倫理が守るべき最も重要な大前提は何か。
	\item 研究対象者が未成年者である場合、本人の同意の他に誰の同意が追加で必要か。
	\item 収集したデータから氏名や住所など、特定の個人を識別できる情報を削除し、特定できないように処理することを何と呼ぶか。
	\item 研究対象者は、一度研究への協力を同意した後でも、不利益を受けることなくその同意を取り下げることができる。この権利を何と呼ぶか。
	\item MBA学生が自社の同僚にインタビューを行う際、自身の身元や研究の真の目的を偽りなく説明し、誠実に行動する倫理的態度は、AI補足で言及されたどの概念に該当するか。
	\item 研究対象となった企業に対し、研究成果のサマリーをフィードバックするなど、研究が対象者にとっても有益なものとなるよう配慮する考え方を、AI補足では何と呼んだか。
	\item アンケート調査において、質問紙への回答・提出という行為をもって、研究への同意がなされたものとみなす手続きを何と呼ぶか。
	\item インタビューの文字起こしを外部業者に委託した場合、その業者も遵守すべき義務は何か。
	\item 研究対象者は、研究者が保有する自身に関するデータについて、開示を求める権利を有するか。
	\item 人を対象とする研究において、対象者の身体的・精神的な負担や苦痛を最小限にするよう努めるべきであるか。
	\item 「人を対象とする研究」の定義には、心理学的なアンケート調査や社会学的なインタビューも含まれるか。
	\item 本講義で解説された、研究実施前に大学の委員会から承認を得る制度を何と呼ぶか。
	\item 研究倫理上の問題や、調査対象者とのトラブルが発生(またはその兆候がある)した場合、研究者が取るべき最も適切な行動は何か。
	\item 研究が承認され実施が終了した後、研究者が委員会に対して行うべき手続きは何か。
\end{enumerate}

\subsubsection*{解答一覧}
1. インフォームド・コンセント(説明と同意)、2. 研究対象者の人権を守ること(またはプライバシー保護)、3. 保護者または後見人、4. 匿名化(非識別化)、5. 同意の撤回権(または同意を撤回する権利)、6. 真正性 (Authenticity)、7. 相互作用性 (Reciprocity)、8. みなし同意(または同意の簡略化)、9. 研究倫理の遵守義務(または守秘義務)、10. 有する、11. 努めるべきである、12. 含まれる、13. 事前審査制度、14. 速やかにFD・SD委員会(等の定められた窓口)に連絡・相談し、個人で対応しない、15. 終了報告書を(定められた期限内、例:1ヶ月以内)に提出すること

\end{document}
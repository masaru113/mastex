\documentclass[uplatex,a4j,12pt,dvipdfmx]{jsarticle}
\usepackage{amsmath,amsthm,amssymb,bm,color,enumitem,mathrsfs,url,epic,eepic,ascmac,ulem,here,ascmac}
\usepackage[letterpaper,top=2cm,bottom=2cm,left=3cm,right=3cm,marginparwidth=1.75cm]{geometry}
\usepackage[english]{babel}
\usepackage[dvipdfm]{graphicx}
\usepackage[hypertex]{hyperref}
\title{\hspace{17mm} Research Ethics Lecture 2 Notes: \newline Reference and Citation Methods}
\author{M. O.}
\date{\today}
\begin{document}
\maketitle
\tableofcontents
\section{Why Are Citations and References Necessary?}
\subsection{Introduction}
These lecture notes summarize a lecture on 'Research Ethics' and 'Proper Citation Methods,' which form the cornerstone of academic activities in the MBA program. In producing business reports and master's theses, the reliability of one's claims is secured by objective evidence (data and preceding studies). As emphasized in this lecture, \textbf{research misconduct} such as \textbf{fabrication}, \textbf{falsification}, and \textbf{plagiarism} is a grave issue. It not only leads directly to the loss of academic credits or expulsion but also fundamentally undermines an individual's credibility. The purpose of these notes is to clarify the definition of acts considered 'plagiarism' in particular, and to organize the foundational knowledge required to ensure academic integrity.
\subsection{Key Concepts and Points}
This section organizes the fundamental ethical concepts that all individuals engaged in research activities (including faculty and students) must observe, as presented in this lecture.
\subsubsection{The Responsibility of Research Ethics Guidelines}
As a premise for this lecture, the existence of \textbf{research ethics guidelines} to ensure the reliability and fairness of academic research was established. Researchers bear a responsibility to thoroughly record, preserve, and appropriately handle their own research and survey data, and to prevent the occurrence of misconduct.
\subsubsection{Major Research Misconduct (FFP)}
The guidelines identify the following three points as particularly serious forms of misconduct. These are also known as 'FFP' from their English initials.
\begin{itemize}
	\item \textbf{Fabrication}: Creating non-existent data or research results.
	\item \textbf{Falsification}: Altering research materials, equipment, or processes, thereby processing data or results obtained from research activities into something that is not genuine.
	\item \textbf{Plagiarism}: Appropriating another researcher's ideas, analysis, data, research results, or papers as one's own without proper citation.
\end{itemize}
\subsubsection{The Definition of Plagiarism and Its Grave Risks}
This lecture issued a strong warning, especially regarding the dangers of \textbf{plagiarism}.
\begin{itemize}
	\item \textbf{Definition}: The act of using papers written by others, data they collected, or descriptions from the internet in a report or thesis \textbf{as if one had thought of and written them oneself}.
	\item \textbf{Intent}: The excuse of 'it was an accident' is never accepted. The very \textbf{fact} of using another's work without citation is deemed plagiarism.
	\item \textbf{Specific Penalties}: This can extend beyond academic evaluation (loss of credits) to become subject to disciplinary action, including \textbf{suspension} or \textbf{expulsion}.
	\item \textbf{Legal Risks}: If plagiarism infringes upon another's copyright, it can escalate into a legal issue as \textbf{copyright infringement}, carrying the risk that the publication of the paper or work will be prohibited.
\end{itemize}
\subsection{Application and Case Analysis}
The lecture presented typical examples of the specific situations in which plagiarism occurs.
\subsubsection{Case Study: Plagiarism of Internet Information}
One of the most common pitfalls cited was the act of searching the internet while writing a report, copying and pasting descriptions from another person's report or website, and submitting it as one's own report.
This constitutes classic \textbf{plagiarism}, so long as the source is not cited. From an analytical perspective, this act deviates from the original purpose of a report, which is 'to research, think, and write for oneself'. The research results and data of others are, to the last, to be 'referenced' as 'evidence' to support one's own claims or as 'preceding research'; they are not to be disguised as one's own work.
\subsection{Deeper Context and Lessons}
Organizing the practical context and the instructor's personal views gleaned from the main thrust of the lecture.
\textbf{\paragraph{Related Tangent Topic: The Strictness That Forbids 'Accidents'}}
The lecture used the expression 'it is not excused as an accident'. This suggests that research ethics, especially regarding plagiarism, is a strict norm that does not question whether the act was 'intentional' or 'negligent'. For the professionalism demanded of an MBA holder, 'I didn't know' or 'I forgot' is unacceptable; one's work carries with it complete accountability. The lesson is that ethical norms are judged not by intent, but by the \textbf{objective act}.
\textbf{\paragraph{Related Tangent Topic: Citation as the 'Grounds' for a Thesis}}
While the lecture defined a report as 'something one researches, thinks about, and writes oneself', it also stated that 'it requires something to serve as grounds'. This indicates that citation is not merely a passive duty to avoid plagiarism, but an active academic technique to secure the \textbf{objectivity} and \textbf{reliability} of one's own claims. Proper citation is the essential means of presenting the 'grounds' that enhance a report's persuasiveness.
\textbf{\subsubsection{AI Supplement: Expansion of Key Points}}
This lecture centered on the ethical aspect of 'Why' citation is necessary and the definition of plagiarism ('What'). However, it offered limited mention of the specific methodology of \textbf{'How' to cite}, which is equally crucial in practical MBA work.
\textbf{\paragraph{Specific Citation Methodology (Direct and Indirect Quotation)}}
To avoid plagiarism and properly present 'grounds', one must master citation techniques.
\begin{itemize}
	\item \textbf{Direct Quotation}:
	      The method of using another's text \textbf{exactly as it is, word-for-word}. In Japanese, this is generally enclosed in quotation marks (`「」`), and the source (author name, year, page number, etc.) is specified immediately after.
	\item \textbf{Indirect Quotation / Paraphrasing}:
	      The method of summarizing or rephrasing another's ideas or analysis \textbf{in one's own words}. Paraphrasing is one of the most frequently used techniques in MBA reports. The crucial point is that even when paraphrased, as long as the \textbf{origin of the idea} lies with someone else, the source must be cited. Neglecting this citation is one of the most common forms of 'plagiarism' among students.
\end{itemize}
Furthermore, in-text citations and the \textbf{Reference List} at the end of the paper must correspond. One must follow the citation style stipulated by the academic field or instructor, such as APA style or Harvard style.
\subsection{Conclusion}
This lecture reaffirmed the importance of \textbf{research ethics}, the most fundamental premise for compiling the learning outcomes of an MBA. In particular, \textbf{fabrication}, \textbf{falsification}, and \textbf{plagiarism} (FFP) were defined as serious misconduct, signifying expulsion from the academic community.
The practical lessons drawn from this lecture can be summarized in the following two points. First, \textbf{plagiarism} is determined by the objective fact of using another's work without citation, regardless of intent (the 'accident' excuse is invalid). Second, \textbf{citation} is not just a passive defense against plagiarism, but also an active academic technique to clarify the 'grounds' for one's own claims and enhance the argument's reliability. It is essential for MBA students to master both the maintenance of ethical standards and the specific citation techniques of direct quotation and indirect quotation (paraphrasing).
\subsection{Key Terms List}
People: (None)
\vspace{\baselineskip}
Universal MBA Concepts: Research Ethics, Fabrication, Falsification, Plagiarism, Citation, References, Copyright
\subsection{Comprehension Check Quiz}
\begin{enumerate}
	\item What do the 'FFP' in research ethics stand for?
	\item Define 'Plagiarism' specifically.
	\item Explain the difference between 'Fabrication' and 'Falsification'.
	\item If you summarize (paraphrase) another person's thesis idea in your own words, is it necessary to cite the source?
	\item If you unintentionally ('accidentally') forget to cite a source, is it still considered plagiarism?
	\item What is 'Self-Plagiarism'?
	\item Is citation necessary if you use a graph from a blog post on the internet in your own report?
	\item What is 'Common Knowledge', and how should it be handled?
	\item What is the difference in roles between a 'Citation' (in-text) and a 'Reference List'?
	\item If you use ideas from a classic text whose copyright has expired (e.g., Adam Smith's 'The Wealth of Nations'), is citation necessary in an academic report?
	\item Is the act of using a sentence from another's paper almost verbatim, merely swapping a few words for synonyms (patchwriting), permissible?
	\item Other than avoiding plagiarism (the ethical aspect), state one other main academic purpose of citation.
	\item In what situations is it most effective to use a 'Direct Quotation'?
	\item In academic papers and reports, when citing primary and secondary sources, which is considered a more reliable source of evidence?
	\item Name two serious penalties, other than academic ones (e.g., loss of credits), that could be expected if research misconduct is discovered.
\end{enumerate}
\subsubsection*{Answer Key}
1. Fabrication, Falsification, Plagiarism, 2. Using another's ideas, words, data, etc., as one's own without proper citation, 3. Fabrication is creating 'non-existent data'; Falsification is intentionally manipulating or altering 'existing data', 4. Yes, it is always necessary as the source of the idea, 5. Yes, regardless of intent, the lack of attribution itself is considered plagiarism, 6. Reusing one's own previously published work without proper citation (e.g., duplicate publication), 7. Yes, just like text, figures and tables must also have their source cited, 8. Facts that are so generally known that they do not need a citation (e.g., 'the Earth revolves around the Sun'). However, if in doubt, it is recommended to cite, 9. A citation is the act of specifically indicating the source of an idea within the text; a reference list is a complete list of bibliographic information for the cited works, located at the end of the paper, 10. Yes, regardless of copyright status, indicating the source of another's idea is essential from the perspective of academic integrity, 11. No, this is considered a form of plagiarism (patchwriting), 12. To provide evidence (grounds) for one's own claims and increase the argument's reliability (or, to allow readers to trace the original source), 13. When the author's original expression is particularly important, or when that expression itself is the subject of analysis, 14. Generally, primary sources (direct data or records from the source) are considered more reliable, 15. Suspension, Expulsion (or, revocation of degree)
\section{Credibility of Sources / Quotation, Reproduction, and Reference}
\subsection{Introduction}
The quality of an academic report or thesis is determined by the quality of the 'evidence' supporting its claims. In addition to avoiding 'plagiarism', as learned in the previous lecture, another crucial skill for MBA students to acquire is the ability to correctly discern the \textbf{credibility of information sources}. No matter how logical a report may be, if its evidence is based on information of low reliability (e.g., an anonymous web article), the value of the entire report is compromised.
The purpose of these notes is to clarify the criteria for selecting information sources that can withstand academic scrutiny, and further, to organize the legal distinctions between 'quotation' and 'reproduction' when using others' works, along with the proper usage based on \textbf{copyright law}.
\subsection{Key Concepts and Points}
This lecture presented two central themes: methods for evaluating information sources and methods of use based on copyright law.
\subsubsection{Evaluation Criteria for Source Credibility}
Information used as references for a report demands high reliability.
\begin{itemize}
	\item \textbf{Highly Reliable Sources}:
	      \begin{itemize}
		      \item \textbf{Academic Journal Articles}: Having passed a strict \textbf{peer-review process} by experts in the research field, a certain level of reliability is guaranteed.
		      \item \textbf{Government Databases, Newspapers, Specialized Books}: The publisher of the information (author, affiliated institution, publisher) is clear, and the accountability for the written content is unambiguous.
	      \end{itemize}
	\item \textbf{Unreliable (Inappropriate) Sources}:
	      \begin{itemize}
		      \item \textbf{Anonymous Web Articles}: It is unclear who wrote it and on what basis. As objective verification of accuracy is impossible, these are inappropriate as references for academic reports.
	      \end{itemize}
\end{itemize}
\subsubsection{Five Perspectives for Confirming Credibility}
To guarantee the credibility of a given document, it was stated that information must be checked from the following five perspectives:
\begin{enumerate}
	\item \textbf{Who wrote it?} (Author's name, affiliation)
	\item \textbf{What type of literature is it?} (Book, academic journal, collection of papers, etc.)
	\item \textbf{Where was it published from?} (Publisher, academic society, public institution)
	\item \textbf{When was it created?} (Publication year, date)
	\item \textbf{On what information is it based?} (Citation of references)
\end{enumerate}
\subsubsection{The Legal Distinction Between 'Quotation' and 'Reproduction'}
When using another person's copyrighted work (text, figures, tables, photos, etc.), the method is broadly divided into 'quotation' and 'reproduction', which are treated differently under the law.
\begin{itemize}
	\item \textbf{Proper Quotation}: Can be used \textbf{without obtaining permission} from the copyright holder.
	\item \textbf{Reproduction}: In principle, \textbf{permission from the copyright holder is required}.
\end{itemize}
\subsubsection{Article 32 of the Copyright Act and the Three Requirements for 'Quotation'}
Based on Article 32 of the Copyright Act, to be considered a lawful 'quotation', all three of the following requirements must be met:
\begin{enumerate}
	\item It must be a \textbf{work that has already been made public}.
	\item It must be \textbf{compatible with fair practice}.
	\item It must be within a \textbf{justifiable range} for purposes such as reporting, criticism, or research.
\end{enumerate}
\subsubsection{Specific Criteria for 'Fair Practice' and 'Justifiable Range'}
To judge requirements 2 and 3 specifically, the following three points were presented, based on materials from the Agency for Cultural Affairs:
\begin{itemize}
	\item \textbf{Principal-Subordinate Relationship}: It must be clear, both qualitatively and quantitatively, that one's own work is 'principal' and the quoted part is 'subordinate'. If the quotation makes up the majority of the report, the relationship is reversed and it is not considered a 'quotation'.
	\item \textbf{Clear Distinction}: One's own writing and the quoted part must be clearly distinguished by using quotation marks, block-quoting (indentation), or other means.
	\item \textbf{Necessity}: Quoting that specific part must be indispensable for the development of one's own argument.
\end{itemize}
If these requirements are not met, the use is judged as 'reproduction', and permission from the copyright holder becomes necessary.
\subsubsection{Descriptive Differences Between 'Quotation' and 'Reference'}
\begin{itemize}
	\item \textbf{Quotation (Direct Quotation)}: Using the copyrighted work \textbf{in its original form}. Enclose it in quotation marks (`「」`) and write it faithfully (without changing a single character).
	\item \textbf{Reference (Indirect Quotation)}: Using the content of the copyrighted work by \textbf{summarizing it in one's own words} (paraphrasing). Describe it in a form such as 'According to...'.
\end{itemize}
\subsection{Application and Case Analysis}
Analyzing the concepts presented in the lecture by applying them to specific cases.
\subsubsection{Case: Reliability of Academic Journal Articles}
The reason academic journal articles are considered highly reliable evidence is not only because they are written by experts, but also because they have passed through a quality assurance system known as the 'peer-review process'. In this process, anonymous experts in the same field rigorously examine the content's validity, novelty, and logical structure, thus guaranteeing a certain level of objectivity.
\subsubsection{Inappropriate Case: Use of Anonymous Blog Articles}
Using information from anonymous blog articles or websites with unknown publishers as evidence for a report is strictly forbidden in MBA studies. This is because it is impossible to distinguish whether the information is a personal impression or a verified fact, and its credibility is zero. An analysis based on such information is merely a 'book report' and is not considered an academic 'report'.
\subsubsection{Descriptive Examples: 'Quotation' and 'Reference'}
\begin{itemize}
	\item \textbf{Example of Quotation (Direct Quotation)}:
	      This lecture states, regarding the requirements for quotation, 'A work that has been made public may be used by quoting it' (\textbf{Article 32 of the Copyright Act}).
	\item \textbf{Example of Reference (Indirect Quotation)}:
	      According to materials from the Agency for Cultural Affairs, to be considered a lawful quotation, the \textbf{principal-subordinate relationship} must be clear, with one's own argument as 'principal' and the quoted part as 'subordinate'.
\end{itemize}
\subsection{Deeper Context and Lessons}
Organizing the practical context and the instructor's personal views gleaned from the main thrust of the lecture.
\textbf{\paragraph{Related Tangent Topic: 'Reproduction' as a Gray Area and Risk Avoidance}}
The lecture mentioned that 'the judgment of whether it is a quotation or not is very difficult'. This is especially true for the use of figures and tables, where experts may disagree on whether the requirements for the \textbf{principal-subordinate relationship} and \textbf{necessity} have been met. The advice from the lecture, 'If you are unsure, it is safer to obtain permission', suggests the importance of \textbf{compliance awareness} and risk avoidance as a business professional, even before academic correctness. The lesson is that in practice, maintaining good relations with stakeholders (in this case, the copyright holder) should be prioritized over pushing legal gray areas.
\textbf{\paragraph{Related Tangent Topic: Evaluating the 'Publisher's' Credibility}}
The lecture placed 'Who wrote it?' at the top of the credibility criteria. This demonstrates the importance of \textbf{critical thinking}—evaluating not just the information itself (What), but also the publisher of the information (Who). When encountering diverse information, such as in MBA case studies, it is essential to train oneself to discern the biases (e.g., a specific company, an ideology) upon which that information is being disseminated.
\textbf{\subsubsection{AI Supplement: Expansion of Key Points}}
This lecture focused on the credibility of sources (Who/Where) and the legal requirements for quotation (Why/What). However, the following two points, which are extremely important for the practical application of MBA theses regarding 'specific citation techniques (How)', were lacking.
\textbf{\paragraph{1. The Danger of Secondary Citation (Citing from a Citation)}}
The lecture stressed the importance of checking 'on what information it is based', but it did not mention the danger of 'secondary citation', which is the practical application of this. \textbf{Secondary citation} is citing document B, which was cited in document A, based only on the information from A, without checking B directly. This carries an extremely high risk of perpetuating author A's misinterpretations or citation errors. The principle of academic research is to access the \textbf{primary source (original text)}; secondary citation should be avoided as much as possible, and the habit of checking the original text is required.
\textbf{\paragraph{2. Unification of Citation Style}}
While the lecture provided Japanese descriptive examples like using quotation marks or 'According to...', it did not mention the systematic \textbf{citation styles} (e.g., APA style, Harvard style, Chicago style) required in international business schools. These styles are international rules for systematically linking in-text citations (Author, Year) with the reference list (bibliographic information) at the end, ensuring that readers can accurately trace the necessary information. Whichever style is adopted, unifying it to one style throughout the entire report is an essential requirement for an academic document.
\subsection{Conclusion}
This lecture covered the technical requirements for 'how to show high-quality evidence and use it legally correctly', built upon the ethical foundation of 'avoiding plagiarism' in MBA report writing.
The key points of this report are the clear distinction between reliable sources (peer-reviewed papers, public statistics) and inappropriate sources (anonymous web articles), and the three requirements for lawful 'quotation' based on \textbf{Article 32 of the Copyright Act} (publicly available, fair practice, justifiable range), especially the importance of the \textbf{principal-subordinate relationship} and \textbf{clear distinction}.
The practical lessons drawn from the 'Deeper Context' section are the thorough application of \textbf{critical thinking} (Who disseminated the information?) when encountering information, and the importance of compliance awareness (obtain permission if in doubt). MBA students, being not only 'consumers' of information but also 'producers' who disseminate highly reliable information themselves, must be conscious of this and thoroughly check original sources and adhere to citation rules.
\subsection{Key Terms List}
People: (None)
\vspace{\baselineskip}
Universal MBA Concepts: Credibility of Sources, Peer Review, Primary Source, Secondary Source, Copyright, Quotation, Reproduction, Copyright Act Article 32, Fair Practice, Principal-Subordinate Relationship, Clear Distinction, Secondary Citation, Citation Style (APA, etc.)
\subsection{Comprehension Check Quiz}
\begin{enumerate}
	\item What is the main reason that articles published in academic journals are considered more reliable than general web articles?
	\item Explain why it is inappropriate to use an anonymous blog post as evidence in an academic report.
	\item What is the most important legal distinction between 'quotation' and 'reproduction' when using another's work?
	\item What are the three requirements for a lawful 'quotation' as stipulated by \textbf{Article 32 of the Copyright Act}?
	\item What does the 'principal-subordinate relationship' in quotation refer to, both qualitatively and quantitatively?
	\item To ensure 'clear distinction' in quotation, what is one specific method that should be used in the text?
	\item What is the 'necessity' requirement in quotation?
	\item Is the act of summarizing the conclusion of another's paper in one's own words a 'direct quotation' or an 'indirect quotation'?
	\item In the act described in question 8 (indirect quotation), is it necessary to cite the source?
	\item What is 'secondary citation (citing from a citation)', and what is the biggest reason it is discouraged?
	\item Is data from a press release published on a company's website usable as evidence in an academic report?
	\item If the majority of a report is composed of 'quotations' from others' papers, which requirement of Article 32 of the Copyright Act is it most likely failing to meet?
	\item When taking a figure or graph from another's work, is it generally easy or difficult to meet the requirements for 'quotation'? State the reason as well.
	\item What is 'fair practice' thought to refer to specifically?
	\item What are rules like APA, MLA, and Chicago called in the context of academic report writing?
\end{enumerate}
\subsubsection*{Answer Key}
1. Because they have undergone a peer-review process by experts, 2. Because the publisher is unknown, and the accuracy, objectivity, and basis of the description cannot be verified, 3. Whether permission from the copyright holder is required or not, 4. (1) It must be a work that has been made public, (2) It must be compatible with fair practice, (3) It must be within a justifiable range for purposes such as reporting, criticism, or research, 5. A relationship where one's own argument is 'principal' and the quoted part is 'subordinate' (supplementary or illustrative), both in quality and quantity, 6. Enclose it in quotation marks (`「」`) or use block-quoting (indentation), 7. That quoting the part is indispensable for developing one's own argument, 8. Indirect quotation (or reference), 9. Yes, because the origin of the idea lies with someone else, it is always necessary, 10. Citing a document (B) that another document (A) cited, without checking the original source of B. The reason is the high risk of perpetuating A's citation errors or misinterpretations, 11. It is usable. However, the publisher's (company's) bias should be considered, and if possible, it should be used in conjunction with primary sources (original data) or third-party verification, 12. The principal-subordinate relationship (because one's own argument has become 'subordinate'), 13. Difficult. A figure or table is easily seen as an independent work in itself, making it difficult to render 'subordinate' to the text, and explaining its necessity can also be difficult (in many cases, permission for reproduction is required), 14. Following the rules and etiquette generally accepted within the academic community of that field, such as citing the source, 15. Citation Styles (or Reference Styles)
\section{How to Show Sources}
\subsection{Introduction}
In academic reports and papers, when making claims based on preceding research or data, citing the 'grounds' for those claims is essential from the perspective of avoiding plagiarism (research ethics). While the previous lecture discussed *why* citation is necessary, this session focuses on the specific techniques of *how* to cite sources, namely, \textbf{Citation Styles}.
Describing sources in a unified, appropriate style is not merely a matter of formatting; it is an academic 'etiquette' that enables readers to accurately trace the original sources, thereby guaranteeing the transparency and reliability of the discussion. The purpose of these notes is to understand the characteristics of the two main citation styles introduced in the lecture, the \textbf{Vancouver style} and the \textbf{Harvard style}, and to organize their proper application.
\subsection{Key Concepts and Points}
This lecture first defined the 'source' itself, and then explained two contrasting styles as specific methods for describing it.
\subsubsection{Definition of a Source (Bibliographic Elements)}
The lecture introduced the definition by Setsuko Fujita. According to it, a 'source' (shutten) specifically refers to the \textbf{author's name}, \textbf{book title}, \textbf{publication year}, etc., and the collection of these individual \textbf{bibliographic elements} constitutes the 'source'. In academic reports, it is necessary to describe these bibliographic elements accurately and completely so that the reader can identify the material.
\subsubsection{Vancouver Style}
The first system, introduced based on the classification by the Japan Science and Technology Agency (JST).
\begin{itemize}
	\item \textbf{Alias}: \textbf{Citation-Order System}.
	\item \textbf{Characteristics}: Assigns \textbf{sequential numbers} (e.g., [1], (2), 3)) to the cited locations in the text, in the order the citations appear.
	\item \textbf{Reference List}: Lists the bibliographic information \textbf{in the order of the sequential numbers} assigned in the text.
	\item \textbf{Main Fields of Use}: Widely used primarily in the natural sciences and medical fields. It keeps the text clean and is suitable for concisely referencing many documents.
\end{itemize}
\subsubsection{Harvard Style}
The second system introduced, and the mainstream system in the social sciences, including MBA programs.
\begin{itemize}
	\item \textbf{Alias}: \textbf{Author-Date System}.
	\item \textbf{Characteristics}: Describes the \textbf{author's name and publication year} directly in parentheses (e.g., (Tanaka, 2023)) at the cited location in the text.
	\item \textbf{Reference List}: Lists the bibliographic information in \textbf{alphabetical order} of the author's name (or Japanese syllabary order). The in-text citation (author's name) and the list correspond intuitively.
	\item \textbf{Main Fields of Use}: Widely used primarily in the social sciences and humanities. Its advantage is that the reader can immediately grasp 'who' argued 'when' just by reading the text.
\end{itemize}
\subsection{Application and Case Analysis}
The two systems presented in the lecture produce clear differences in the 'look' and 'information presentation method' of a paper or report.
\subsubsection{Example of Vancouver Style}
\begin{itemize}
	\item \textbf{Example in-text}:
	      'According to recent research [1], market volatility is said to be increasing. This trend is also apparent in the Japanese market [2].'
	\item \textbf{Example reference list}:
	      \begin{enumerate}
		      \item [1] A. Sato, \textit{Global Market Trends}, 2023.
		      \item [2] B. Suzuki, \textit{Japanese Economic Review}, 2022.
	      \end{enumerate}
\end{itemize}
\subsubsection{Example of Harvard Style}
\begin{itemize}
	\item \textbf{Example in-text}:
	      'According to recent research (Sato, 2023), market volatility is said to be increasing. Furthermore, Suzuki (2022) points out that this trend is also apparent in the Japanese market.'
	\item \textbf{Example reference list}: (Alphabetical order)
	      \begin{itemize}
		      \item Sato, A. (2023). \textit{Global Market Trends}.
		      \item Suzuki, B. (2022). \textit{Japanese Economic Review}.
	      \end{itemize}
\end{itemize}
\subsubsection{Consideration for MBA Application}
In MBA reports and theses, it is common to develop one's own arguments based on the theories of specific business scholars (e.g., Porter, Drucker) or specific preceding studies (e.g., (Aaker, 1991)). For this reason, the \textbf{Harvard style (author-date system)}, which immediately shows 'whose' and 'when' the discussion is being referenced in the text, is extremely suitable for clarifying the context of the argument.
\subsection{Deeper Context and Lessons}
Organizing the practical context and the instructor's personal views gleaned from the main thrust of the lecture.
\textbf{\paragraph{Related Tangent Topic: The Biggest Rule is 'Do Not Mix'}}
The lecture 'introduced' the two systems side-by-side, but the most important lesson in the practice of academic writing is to \textbf{never mix} these two (or other styles). Using the Vancouver style (sequential numbers) at the beginning of a report and then switching to the Harvard style (author, year) mid-way is considered the most amateurish mistake. Whichever style is chosen, the discipline to \textbf{strictly unify it to one style} throughout a single report is required.
\textbf{\paragraph{Related Tangent Topic: The Role of the Japan Science and Technology Agency (JST)}}
This lecture mentioned the 'Japan Science and Technology Agency (JST)' as the source for classifying citation styles. This is because JST is the core institution responsible for the distribution of scientific and technological information in Japan and operates academic paper databases (like J-STAGE). This suggests that JST is an authoritative source to refer to not only for collecting papers but also for standardizing the meta-rules of 'how to write' and 'how to show' academic information.
\textbf{\subsubsection{AI Supplement: Expansion of Key Points}}
This lecture introduced citation styles by classifying them into two broad types, 'Vancouver' and 'Harvard'. While this is effective for understanding the concepts, the following important points regarding the specific styles required in MBA practice should be supplemented.
\textbf{\paragraph{'APA Style' as a Specific Example of the 'Harvard Style'}}
The 'Harvard style (author-date system)' introduced in the lecture does not refer to a single, specific rule, but is a 'general term' for styles that use the author's name and publication year.
The most widely adopted representative of this Harvard style in the social sciences, including MBA programs, is the format stipulated by the \textbf{APA style (American Psychological Association)}.
APA style follows the basic concept of the Harvard style (author-name/publication-year in-text, author-name-order list), while also stipulating extremely detailed rules for all cases, including the description order of bibliographic elements, punctuation, use of italics, and notation of DOI (Digital Object Identifier). When an MBA student is asked to 'write in Harvard style', in many cases, it means to follow the strict rules of this \textbf{APA style (e.g., 7th Edition)}.
\subsection{Conclusion}
This lecture focused on the specific technique of 'how to show sources' in academic reports. We learned that two main citation styles exist for citing reliable sources (bibliographic elements): the \textbf{Vancouver style (citation-order system)} and the \textbf{Harvard style (author-date system)}. The former excels in textual brevity (mainly in science/engineering), while the latter excels in clarifying the origin (author) of the discussion (mainly in humanities/social sciences).
The practical lesson from this lecture, in addition to which system to choose, is the importance of the discipline of \textbf{strict unification} of the chosen style throughout the entire report. As MBA students, mastering the specific formatting rules of the Harvard style, which is standard in one's own specialized field, particularly its leading example, the \textbf{APA style}, is an essential practical skill for creating a reliable report.
\subsection{Key Terms List}
People: Setsuko Fujita
\vspace{\baselineskip}
Universal MBA Concepts: Citation Style, Vancouver Style, Harvard Style, Citation-Order System, Author-Date System, Bibliographic Elements, Reference List, APA Style
\subsection{Comprehension Check Quiz}
\begin{enumerate}
	\item Name three specific examples of 'bibliographic elements' that should be described when citing a source.
	\item What is another name (characteristic) for the Vancouver style?
	\item What is another name (characteristic) for the Harvard style?
	\item What citation style assigns numbers in the text, such as [1], [2]..., in the order of citation?
	\item What citation style describes (Author, Publication Year) in the text?
	\item In what order is the reference list for the Vancouver style arranged?
	\item In what order is the reference list for the Harvard style arranged?
	\item Which style is relatively more common in the fields of medicine and natural sciences?
	\item Which style is relatively more common in the social sciences (such as MBA programs)?
	\item When writing a report, is it acceptable to mix the Vancouver and Harvard styles?
	\item What is the 'APA style'? Which of the systems introduced in the lecture is it related to?
	\item Why is the Harvard style (author-date system) considered suitable for discussions in the social sciences?
	\item When referencing multiple different documents at the same point, how is this written in the Vancouver style? (Example)
	\item When referencing multiple different documents at the same point, how is this written in the Harvard style? (Example)
	\item From a perspective other than plagiarism prevention, state two main purposes of citation styles.
\end{enumerate}
\subsubsection*{Answer Key}
1. Author's name, Book title, Publication year (or Journal name, Volume/Issue, Page numbers, etc.), 2. Citation-Order System, 3. Author-Date System, 4. Vancouver style, 5. Harvard style, 6. In the order of citation in the text (sequential order), 7. In alphabetical order of the author's name (or Japanese syllabary order), 8. Vancouver style, 9. Harvard style, 10. No, a single style must be unified throughout the entire report, 11. It is the citation style defined by the American Psychological Association and is a type of Harvard style (author-date system), 12. Because it allows the reader to immediately grasp 'who' argued 'when' within the text, 13. e.g., [1-3, 5] or [1, 2, 5], consolidating the numbers, 14. e.g., (Sato, 2020; Suzuki, 2021; Tanaka, 2019), separating with semicolons, 15. To show the evidence (grounds) for the argument, To allow the reader to trace the original source (primary material) (or, To show respect for preceding research)
\section{How to Write References}
\subsection{Introduction}
The reliability of claims in an academic report depends on whether the reader can accurately trace the \textbf{references} that form their basis. In addition to showing respect for preceding research and avoiding plagiarism (research ethics), describing accurate \textbf{bibliographic elements} in a unified format so that others can find the source material is a crucial 'technical' responsibility of a researcher.
This lecture explains the specific descriptive methods for major information sources such as books, journal articles, and websites, based on the 'How to Write References' of \textbf{SIST 02 (Standards for Information of Science and Technology)}, which is the standard in Japan's science and technology fields. It also covers the 'figure and table' citation and reproduction rules, which require special attention. The purpose of these notes is to organize these specific descriptive rules and to acquire the practical knowledge to ensure academic accuracy.
\subsection{Key Concepts and Points}
This lecture presented the constituent elements of references based on SIST 02, the descriptive rules for major types of media, and the legal distinctions for using figures and tables.
\subsubsection{The Four Essential Bibliographic Element Groups for References}
To show sufficient information as a reference, in principle, the following four groups of bibliographic elements must be described in this order:
\begin{enumerate}
	\item \textbf{Bibliographic elements concerning the author} (Author name, Editor name, Translator name, etc.)
	\item \textbf{Bibliographic elements concerning the title} (Book title, Journal name, Article title, etc.)
	\item \textbf{Bibliographic elements concerning publication and physical characteristics} (Publisher, Place of publication, Year of publication, Edition, Volume/Issue, Page numbers, etc.)
	\item \textbf{Bibliographic elements (notes)} (Medium designator [online], URL, Date accessed [date of reference], etc.)
\end{enumerate}
\subsubsection{Key Points for Description by Medium}
\begin{description}
	\item[Monograph (Book)] The required elements are 'Author name', 'Book title', 'Edition (not required for first edition)', 'Place of publication (not essential)', 'Publisher', 'Year of publication', and 'Total page count'.
	\item[Journal Article] The required elements are 'Author name', 'Article title', 'Journal name', 'Year of publication', 'Volume number', 'Issue number', and 'Page range of the article (start page-end page)'.
	\item[Web Page] The required elements are 'Author name (or organization/institution name)', 'Web page title', 'Website name', 'Date of update (can be omitted if unknown)', 'Location (URL)', and 'Date accessed (date of reference)'.
\end{description}
\subsubsection{Distinction Between Quotation and Reproduction of Figures/Tables}
When using figures or tables published in another person's paper or work, the legal procedures differ fundamentally depending on the form of use.
\begin{description}
	\item[When considered within the scope of 'Quotation']
	      If it is within the scope of justifiable 'quotation' recognized by copyright law (meeting the principal-subordinate relationship, clear distinction, and necessity), \textbf{permission from the copyright holder is not required}. However, one is obligated to \textbf{clearly state the source} in the caption (footnote) of the figure or table.
	\item[When exceeding the scope of 'Quotation' (= Reproduction)]
	      If the figure or table becomes the 'principal' part of the argument, lacks necessity, or otherwise exceeds the scope of quotation in quality or quantity, it is deemed \textbf{'reproduction'}. In principle, \textbf{permission from the copyright holder is required}.
	\item[When using a modified figure or table]
	      The same distinction applies when modifying an original figure or table. If it is within the scope of 'quotation', one may simply cite the source, e.g., 'Created based on (Author, Year)'. However, if it falls under 'reproduction', permission from the copyright holder is required for the modification itself.
\end{description}
\subsection{Application and Case Analysis}
Organizing specific descriptive examples based on the standards presented in the lecture.
\subsubsection{Example Description for a Monograph (Book)}
As a punctuation rule, a \textbf{period} is placed at the end of a major group of bibliographic elements (e.g., author group, title group), and elements within the same group (e.g., place of publication and publisher) are separated by a \textbf{comma}.
\begin{itemize}
	\item \textbf{Example (Japanese book)}: Author Name. \textit{Book Title}. Edition. Publisher, Year of publication, Total pages p.
	\item \textbf{Example (Western book)}: Author, L. F. \textit{Title of Book}. Edition. Publisher, Year, Total pages p.
	      \begin{itemize}
		      \item Western author names are listed as 'Last name, First name', and the first name is often initialed.
		      \item For Western titles, it is common to capitalize the first letter of each word except for articles, conjunctions, and prepositions.
		      \item If there are multiple authors, they may be separated by a semicolon (;) (SIST 02 recommends separation by a comma).
	      \end{itemize}
\end{itemize}
\subsubsection{Rules for Describing Page Numbers}
\begin{itemize}
	\item \textbf{Total page count}: `350 p.` (Total pages in the book)
	\item \textbf{Single page reference}: `p. 350` (Only page 350)
	\item \textbf{Continuous page reference}: `p. 350-360`
	\item \textbf{Discontinuous page reference}: `p. 350, 362`
\end{itemize}
*Note: Although styles using `pp.` also exist, SIST 02 unifies this to `p.`
\subsubsection{Example Description for a Journal Article}
Organizing the example from 'Diamond Harvard Business Review' presented in the lecture, based on SIST 02.
\begin{itemize}
	\item \textbf{Author}: Guhan Subramanian
	\item \textbf{Translator}: Yukinobu Koda (trans.)
	\item \textbf{Article Title}: Three Principles of Corporate Governance as a Source of Competitive Advantage
	\item \textbf{Special Feature Name}: Special Feature: Corporate Governance
	\item \textbf{Journal Name}: Diamond Harvard Business Review
	\item \textbf{Year}: 2016
	\item \textbf{Volume(Issue)}: 41(3)
	\item \textbf{Page Range}: p. 75-87
	      \textbf{Example Description}:
	      Guhan Subramanian. Yukinobu Koda, trans. Three Principles of Corporate Governance as a Source of Competitive Advantage. Special Feature: Corporate Governance. \textit{Diamond Harvard Business Review}. 2016, \textbf{41}(3), p. 75-87.
\end{itemize}
*Note: It is common to italicize the journal name (\textit{Diamond Harvard Business Review}) and bold the volume number (\textbf{41}).
\subsubsection{Example Description for a Web Page}
The most important elements for citing a website are the \textbf{URL}, which allows the reader to access the information, and the \textbf{date of reference (date accessed)}, which proves when that information existed.
\begin{itemize}
	\item \textbf{Example Description}:
	      Author (or Institution). "Web Page Title". \textit{Website Name}. Date of update, \url{https://...}, (accessed 2025-11-04).
\end{itemize}
\subsection{Deeper Context and Lessons}
Organizing the practical context and the instructor's personal views gleaned from the main thrust of the lecture.
\textbf{\paragraph{Related Tangent Topic: The Gray Area of 'Scope of Quotation' and Risk Management}}
The lecture frankly stated that judging whether the use of a figure or table is 'quotation' or 'reproduction' is 'very difficult'. It specifically mentioned that 'one or two figures/tables might be considered within the scope of quotation, but even a single one could be deemed reproduction, qualitatively or quantitatively', indicating this is a gray area with no clear line.
The MBA-related lesson here is the perspective of academic and legal \textbf{risk management}. The advice, 'If you are unsure, it is safer to obtain permission from the copyright holder', demonstrates an extremely important stance for a business professional: prioritizing compliance and preventing future legal troubles.
\textbf{\paragraph{Related Tangent Topic: SIST 02 as a 'Technical Standard'}}
This lecture explains how to write references not as 'etiquette' or 'manners', but based on a specific 'technical standard', \textbf{SIST 02}. This suggests that the description of references is not merely a formatting issue, but an objective and precise system of rules for supporting the 'distribution' of scientific and technological information. The lesson is that in MBA programs, too, strict description based on standardized rules is required, not impressionistic reports.
\textbf{\subsubsection{AI Supplement: Expansion of Key Points}}
This lecture explained in detail the components for 'manually' writing references based on SIST 02. However, in the practice of modern MBA and academic research, the following two points, which are essential for ensuring operational efficiency and accuracy, should be supplemented.
\textbf{\paragraph{1. Utilization of Reference Management Tools}}
Applying strict rules like SIST 02 or APA style manually is very time-consuming and prone to typos and formatting errors (like mixing up periods and commas).
In modern research activities, the use of \textbf{reference management tools} (e.g., \textbf{EndNote}, \textbf{Mendeley}, \textbf{Zotero}, etc.) has become common practice. These tools manage bibliographic information for papers and books as a database and integrate with writing software like Word to \textbf{automatically} insert in-text citations and generate an end-of-paper reference list in a specified style (SIST, APA, Harvard, etc.). Utilizing this technology directly links to improved productivity in an MBA program.
\textbf{\paragraph{2. The Importance of DOI (Digital Object Identifier)}}
While the lecture emphasized URLs for web pages, in the world of academic papers (especially electronic journals), the \textbf{DOI (Digital Object Identifier)} is used as a more permanent identifier than a URL. Unlike URLs, which are prone to link rot, a DOI is a persistent ID tied to the content itself. In recent major citation styles (like APA 7th Edition), it is strongly recommended (or mandatory) to list the DOI instead of the URL if a DOI is available.
\subsection{Conclusion}
This lecture explained the standard descriptive method (SIST 02 compliant) for \textbf{references} as a concrete technique for guaranteeing the reliability of academic reports.
The key points of this report are that a reference is composed of four groups of \textbf{bibliographic elements} (author, title, publication, notes), and that the elements to be described differ by medium (book, journal, web).
As a particularly important point, it was emphasized that a legal distinction exists between 'quotation' and 'reproduction' for the \textbf{use of figures and tables}, and that careless use carries the risk of \textbf{copyright infringement}.
The practical lesson from the 'Deeper Context' section is that a \textbf{risk-avoidant approach}—seeking permission—is wise for the gray area of 'the scope of quotation'. As MBA students, while understanding strict rules like SIST 02, it is also necessary to acquire modern academic skills: improving operational accuracy and efficiency by utilizing \textbf{reference management tools} as indicated in the AI supplement, and securing the permanence of information by using the latest identifiers like \textbf{DOI}.
\subsection{Key Terms List}
People: Setsuko Fujita
\vspace{\baselineskip}
Universal MBA Concepts: References, Citation Style, Bibliographic Elements, SIST 02, Copyright, Quotation, Reproduction, Source, DOI, Reference Management Tools (Zotero, Mendeley, etc.)
\subsection{Comprehension Check Quiz}
\begin{enumerate}
	\item According to SIST 02, the bibliographic elements of a reference are broadly classified into what four groups?
	\item In a book citation, if it is the first edition, is it necessary to describe the edition?
	\item If a book's total page count is 120 pages, how is this written according to SIST 02 conventions?
	\item If a paper is only on a specific page (e.g., page 25), how is this written?
	\item When citing a web page, what 'date' must be written in addition to the URL?
	\item When can a figure or table from another person's paper be used without the copyright holder's permission?
	\item Even in the case of 6, what must always be specified in the figure's caption?
	\item If use is judged to exceed the scope of 'quotation', what is it called, and what is required?
	\item Even when 'modifying' another's figure or table, in what case is the copyright holder's permission required?
	\item In academic papers, what is used as a more permanent and reliable identifier than a URL?
	\item What are software applications like EndNote and Zotero called?
	\item Why is the use of reference management tools recommended? (From a perspective other than efficiency)
	\item In a journal article citation, if it says 'Vol. 10, No. 2', what do '10' and '2' refer to, respectively?
	\item If one uses another's figure or table without obtaining permission for 'reproduction', what legal problem could potentially arise?
	\item Explain the basic distinction in the use of punctuation (periods and commas) when creating a reference list.
\end{enumerate}
\subsubsection*{Answer Key}
1. Author, Title, Publication, Notes (Location, etc.), 2. It is not necessary, 3. 120 p., 4. p. 25, 5. The date accessed (date of reference), 6. When it is deemed to be within the scope of 'quotation', 7. The source (e.g., Author, Year, etc.), 8. It is called 'reproduction', and permission from the copyright holder is required, 9. When it is deemed to exceed the scope of 'quotation' (= reproduction), 10. DOI (Digital Object Identifier), 11. Reference management tools (or bibliographic management software), 12. To ensure the uniformity of the citation style and prevent formatting errors (accuracy), 13. Vol. 10 is the 'volume number', No. 2 is the 'issue number', 14. Copyright infringement, 15. A period separates major groups of bibliographic elements; a comma separates elements within the same group
\section{Practical Examples from Faculty}
\subsection{Introduction}
In the lectures so far, we have learned *why* citation is necessary from the perspective of research ethics, and *what* is considered plagiarism. This lecture focuses on the more practical technique of *how* to write citations and references.
The lecture introduced practical examples from multiple faculty members, showing that citation styles differ greatly by field (e.g., a system that lists everything at the end vs. a system that writes in the footnotes at the bottom of the page). The purpose of these notes is to recognize this diversity of styles, and at the same time, to organize the 'golden rules' of citation common to all fields and styles: the importance of \textbf{consistency} and \textbf{traceability}.
\subsection{Key Concepts and Points}
This lecture explained the practical use of 'reference' and 'quotation', the differences in styles by academic field, and the common principles underlying them.
\subsubsection{Practical Distinction Between 'Reference' and 'Quotation'}
The persuasiveness of a paper is enhanced by showing that its claims are based on preceding research.
\begin{itemize}
	\item \textbf{Reference (Indirect Quotation)}:
	      Describing another's research results (e.g., a survey of 500 entrepreneurs) or claims (e.g., a university president's theory on leadership) by \textbf{summarizing and interpreting} them in one's own words. Used to reinforce one's own claims.
	\item \textbf{Quotation (Direct Quotation)}:
	      Bringing in another's text in its original form, \textbf{without modifying so much as a single character}. Used when it is necessary to convey the author's intent accurately. Modifying the original text is considered a serious act of misconduct, as it \textbf{distorts the author's intent}.
\end{itemize}
\subsubsection{Forms of Direct Quotation}
\begin{itemize}
	\item \textbf{Short Quotation}: Incorporated into the main text and enclosed in quotation marks (`「」`).
	\item \textbf{Long Quotation}: Made independent from the main text and \textbf{indented} (block-quote).
\end{itemize}
*Note: Whether to include the page number or not differs depending on the style of the affiliated academic society or field.
\subsubsection{Diversity of Citation Styles}
The lecture showed, through two practical examples, that mainstream styles differ completely depending on the academic field.
\begin{description}
	\item[Style A: End-of-Paper List System]
	      A system commonly seen in business administration and the social sciences. The author's name and publication year (e.g., (Author, 2015)) are briefly noted in the text, and a complete list of references is placed together at the \textbf{very end} of the paper.
	\item[Style B: Footnote System]
	      A system commonly seen in law and the humanities. A small number (superscript) is placed at the relevant point in the text, and the corresponding bibliographic information for the reference is described in the \textbf{footnote at the bottom of that page}.
\end{description}
\subsubsection{Identifying Works by the Same Author in the Same Year}
If an author has published two books in 2015 and one refers to both, the notation (Author, 2015) alone is insufficient to distinguish them. In this case, the common rule is to add letters, such as `2015a` and `2015b`, to clearly identify both.
\subsubsection{The 'Golden Rules' of Citation}
What was emphasized most in the lecture was not the superiority of one style over another, but adherence to the following two principles:
\begin{enumerate}
	\item \textbf{Consistency}:
	      The biggest error is \textbf{mixing} multiple styles. Starting with a footnote system and then switching to an end-of-paper list system mid-way is not permissible. One is required to \textbf{strictly unify (maintain all the way through) one style} from the beginning to the end of the report.
	\item \textbf{Identifiability \& Traceability}:
	      The purpose of a style is to allow the reader to \textbf{quickly find} the document. Differences in format (punctuation or order) are secondary; what is most important is that the \textbf{necessary and sufficient bibliographic elements} (who, what, when, in which medium) for identifying the document are accurately described.
\end{enumerate}
\subsection{Application and Case Analysis}
A comparative analysis of the two different citation styles presented in the lecture.
\subsubsection{Case 1: Social Science (End-of-Paper List System) Practice}
In the case of one faculty member (business administration), 'reference' and 'quotation' were clearly distinguished in their own paper.
\begin{itemize}
	\item \textbf{Use of 'Reference'}: Two different preceding studies—an overseas entrepreneurship survey and a domestic book (a university president's leadership theory)—were summarized in the author's own words. By combining them, the \textbf{persuasiveness} of their original claim (the importance of vicarious experience as a leader) was enhanced.
	\item \textbf{Use of 'Quotation'}: To accurately present the views of other authors, short quotations (using `「」`, with page numbers) and long quotations (indented) were used selectively.
	\item \textbf{List Format}: All references were consolidated at the \textbf{end} of the paper, with Japanese and foreign-language documents listed separately.
\end{itemize}
\subsubsection{Case 2: Law (Footnote System) Practice}
In the case of another faculty member (law), a completely different style was adopted.
\begin{itemize}
	\item \textbf{Use of 'Footnotes'}: Only small numbers, such as `...is [1]`, were added to the cited locations in the text.
	\item \textbf{List Format}: The complete bibliographic information, such as `[1] Author name...`, was described directly in the footnote section at the \textbf{bottom of that page}, not at the end of the paper.
	\item \textbf{Implication}: This comparison shows that the 'conventions' of citation differ greatly among academic fields.
\end{itemize}
\subsection{Deeper Context and Lessons}
Organizing the peripheral lessons gleaned from the main thrust and practical examples of the lecture.
\textbf{\paragraph{Related Tangent Topic: The Persuasiveness of Citation ('Standing on the Shoulders of Giants')}}
One faculty member remarked, 'Even if it's my opinion, it's not persuasive'. This shows that academic research is not a personal essay, but an endeavor of 'standing on the shoulders of giants'—that is, preceding research. Proper quotation and referencing are essential acts for objectively reinforcing one's own claims and positioning the discussion within an academic context.
\textbf{\paragraph{Related Tangent Topic: 'Copy-Pasting' a Style as a Practical Hack}}
For students who feel 'the rules are annoying', the instructor offered a practical piece of advice: 'copy-paste the reference example from one of the professors, and then type your own stuff into it'. This suggests a life hack: rather than memorizing the details of the rules, appropriating the \textbf{format} of a reliable 'model' is a more realistic and safer way to maintain \textbf{consistency}.
\textbf{\paragraph{Related Tangent Topic: Strict Prohibition on Altering Citations}}
As a rule for direct quotation, the point that 'you must not modify so much as a single character' was strongly emphasized. This is an ethical requirement, and a warning that the act of modifying a quotation is a grave offense equivalent to \textbf{'falsification'}, which arbitrarily distorts the author's intent.
\textbf{\subsubsection{AI Supplement: Expansion of Key Points}}
This lecture emphasized the diversity of styles and the importance of consistency, but it lacked the following two important points regarding modern methodologies for *ensuring* that 'consistency'.
\textbf{\paragraph{1. Utilization of Reference Management Tools}}
The lecture introduced the manual method of 'copy-pasting a format', but this is inefficient and prone to human error (punctuation mistakes, order errors, etc.). In the practice of modern MBA and academic research, the use of \textbf{reference management tools} (e.g., \textbf{Zotero}, \textbf{Mendeley}, \textbf{EndNote}, etc.) is indispensable.
These tools database bibliographic information and integrate with writing software (like Word) to \textbf{automatically} insert in-text citations and generate an end-of-paper reference list in a specified style (APA, Chicago, Harvard, etc.). This ensures that \textbf{consistency} is perfectly maintained, and style changes can be completed in an instant.
\textbf{\paragraph{2. Names of Major Style Guides}}
The lecture pointed out differences by field, such as 'law' or 'business administration', but did not mention the specific 'names' of those styles.
\begin{itemize}
	\item A representative example of the \textbf{footnote system} is the \textbf{Chicago Manual of Style}, which is widely used in the humanities and history.
	\item A representative example of the \textbf{end-of-paper list system} is the \textbf{APA style (American Psychological Association)}, which is the standard in psychology, education, and many social science fields, including business administration.
\end{itemize}
By knowing the names of these specific style guides, students can access the correct rulebook to reference.
\subsection{Conclusion}
This lecture concretely demonstrated the \textbf{diversity of citation styles} (e.g., end-of-paper list system vs. footnote system) in academic writing by comparing and contrasting practical examples from multiple faculty members.
The key point of this report lies in the 'golden rule' that maintaining \textbf{'consistency'} is more important than the format of the style itself. Mixing styles is one of the most serious formal errors in an academic report.
Lessons learned from the 'Deeper Context' include that citation is not merely a means of avoiding plagiarism but an active strategy to enhance the \textbf{persuasiveness} of one's own claims, and that the \textbf{absolute integrity of the original text} (prohibition of alteration) is strictly required in direct quotation.
As MBA students, while basing their work on the 'end-of-paper list system' standard in their field (business administration), such as the \textbf{APA style}, they should also actively utilize \textbf{reference management tools} as noted in the AI supplement. They must acquire the technique to reliably secure the two great principles of citation—'consistency' and 'traceability'—while preventing human error.
\subsection{Key Terms List}
People: (None)
\vspace{\baselineskip}
Universal MBA Concepts: Quotation, Reference, Direct Quotation, Indirect Quotation, Footnote, Reference List, Citation Style, Consistency (Style Unification), Bibliographic Elements, Traceability, APA Style, Chicago Style, Reference Management Tools (Zotero, Mendeley, etc.)
\subsection{Comprehension Check Quiz}
\begin{enumerate}
	\item What is the act of summarizing another person's writing and introducing it in one's own words called?
	\item What is the act of describing another person's writing 'without modifying so much as a single character' called?
	\item When including a short direct quotation in the main text, what punctuation is generally used in Japanese?
	\item What is the general formatting rule when making a long direct quotation?
	\item What is the style that describes (Author, Publication Year) at the point of citation in the text and places an author-name-ordered list at the end called?
	\item What is the style that places a small number at the point of citation in the text and shows the bibliographic information in a footnote at the bottom of the page called?
	\item Which of the above (5 or 6) is relatively common in law and the humanities?
	\item Which of the above (5 or 6) is relatively common in business administration and the social sciences?
	\item What notation is generally used to distinguish between two different works published by the same author in the same year?
	\item Is it permissible to mix Style A and Style B within a single report?
	\item What is the most important 'golden rule' in citation styles?
	\item What is the ultimate purpose common to all citation styles?
	\item Name three examples of 'bibliographic elements' that must be included in a reference list.
	\item What are software applications like Zotero and Mendeley called?
	\item Why is altering the original text strictly forbidden in direct quotation?
\end{enumerate}
\subsubsection*{Answer Key}
1. Reference (or Indirect Quotation), 2. Quotation (or Direct Quotation), 3. Hook brackets `「」`, 4. Indent it and write it as a separate block, 5. Author-Date System (e.g., Harvard Style, APA Style), 6. Footnote System (e.g., Chicago Style), 7. 6 (Footnote System), 8. 5 (Author-Date System), 9. Add a letter to the publication year (e.g., 2015a, 2015b), 10. No, it is not permissible (it is one of the worst mistakes to avoid), 11. Consistency (unifying the style throughout the entire report), 12. To allow the reader to accurately and quickly trace (identify) the original source, 13. Author's name, Publication year, Title (article/book title), (or, Publisher, Journal name, Volume/Issue, Page numbers, URL, etc.), 14. Reference management tools (or bibliographic management software), 15. Because it distorts the author's intent and misrepresents the original meaning (it is equivalent to falsification)
\end{document}
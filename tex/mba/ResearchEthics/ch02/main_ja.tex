\documentclass[uplatex,a4j,12pt,dvipdfmx]{jsarticle}
\usepackage{amsmath,amsthm,amssymb,bm,color,enumitem,mathrsfs,url,epic,eepic,ascmac,ulem,here,ascmac}
\usepackage[letterpaper,top=2cm,bottom=2cm,left=3cm,right=3cm,marginparwidth=1.75cm]{geometry}
\usepackage[english]{babel}
\usepackage[dvipdfm]{graphicx}
\usepackage[hypertex]{hyperref}
\title{研究倫理 第2回 講義ノート: 参考と出典の方法}
\author{M. O.}
\date{\today}

\begin{document}
\maketitle
\tableofcontents

\section{なぜ引用・参考が必要なのか}

\subsection{はじめに}
本講義ノートは、MBAプログラムにおける学術活動の根幹をなす「研究倫理」と「適切な引用方法」に関する講義をまとめたものである。ビジネスレポートや修士論文の作成において、主張の信頼性は客観的な根拠(データや先行研究)によって担保される。本講義で強調されたように、研究成果の\textbf{捏造}、\textbf{改ざん}、\textbf{盗用}といった\textbf{不正行為}は、単位不認定や退学処分に直結するだけでなく、個人の信頼性を根本から損なう重大な問題である。本ノートの目的は、特に「盗用」とみなされる行為の定義を明確にし、学術的な誠実性を担保するための基礎知識を整理することにある。

\subsection{主要な概念と論点}
本講義で提示された、研究活動に従事する者(教員、学生を含む)が遵守すべき基本的な倫理概念を整理する。

\subsubsection{研究倫理ガイドラインの責務}
本講義の前提として、学術研究の信頼と公正を確保するための\textbf{研究倫理ガイドライン}の存在が示された。研究者は、自らの研究・調査データの記録保存や適切な取り扱いを徹底し、不正行為の発生を未然に防止する責務を負う。

\subsubsection{主要な研究不正行為 (FFP)}
ガイドラインにおいて、特に重大な不正行為として以下の3点が挙げられた。これらは英語の頭文字を取り「FFP」とも呼ばれる。
\begin{itemize}
	\item \textbf{捏造 (Fabrication)}: 存在しないデータや研究結果などを作成すること。
	\item \textbf{改ざん (Falsification)}: 研究資料、機器、過程を変更する操作を行い、データや研究活動によって得られた結果などを真正でないものに加工すること。
	\item \textbf{盗用 (Plagiarism)}: 他の研究者のアイデア、分析、データ、研究結果、論文などを、適切な出典表示を行わずに自らのものとして流用すること。
\end{itemize}

\subsubsection{盗用の定義と重大なリスク}
本講義では、特に\textbf{盗用}の危険性について強く警鐘が鳴らされた。
\begin{itemize}
	\item \textbf{定義}: 他の人が書いた論文、収集したデータ、あるいはインターネット上の記述を、あたかも\textbf{自分自身が考えて記述したかのように}レポートや論文に使用する行為を指す。
	\item \textbf{意図の有無}: 「うっかりしていた」という弁解は一切通用しない。出典を明示せずに他者の成果を利用したという\textbf{事実自体}が盗用とみなされる。
	\item \textbf{具体的な罰則}: 学術的な評価(単位不認定)にとどまらず、\textbf{停学}、\textbf{退学}などの懲戒処分の対象となりうる。
	\item \textbf{法的リスク}: 盗用が他者の著作権を侵害する場合、\textbf{著作権侵害}として法的な問題に発展し、論文や成果の公開が差し止められるリスクも存在する。
\end{itemize}

\subsection{応用と事例分析}
講義では、盗用が具体的にどのような状況で発生するかについて、典型的な事例が示された。

\subsubsection{インターネット情報の盗用事例}
最も陥りやすい事例として、レポート作成時にインターネットで検索し、得られた他人のレポートやウェブサイトの記述をそのままコピー&ペーストして、自身のレポートとして提出する行為が挙げられた。

これは、出典を明記していない限り、典型的な\textbf{盗用}に該当する。分析の視点として、この行為は「自分で調べて考えて書く」というレポート本来の目的から逸脱している。他者の研究成果やデータは、あくまで自身の主張を裏付ける「根拠」として、あるいは「先行研究」として「参考」にするものであり、自らの成果として偽装するものではない。

\subsection{深層背景と教訓}
講義の本論から読み取れる、実践的な背景や講師の私見を整理する。

\textbf{\paragraph{本論から逸れた寄り道トピック名:「うっかり」は許されない厳格性}}
講義では「うっかりではすまされません」という表現が用いられた。これは、研究倫理、特に盗用が「故意」であるか「過失」であるかを問わない厳格な規範であることを示唆している。MBAホルダーとして求められるプロフェッショナリズムにおいて、「知らなかった」「忘れていた」は通用せず、成果物に対する完全な説明責任が伴う。倫理規範は、意図ではなく\textbf{客観的な行為}によって判断されるという教訓である。

\textbf{\paragraph{本論から逸れた寄り道トピック名:論文の「根拠」としての引用}}
講義では、レポートを「自分で調べて考えて書くもの」と定義する一方で、「根拠となるものが必要」と述べられた。これは、引用(Citation)が単なる盗用回避のための消極的な義務ではなく、自身の主張の\textbf{客観性}と\textbf{信頼性}を担保するための積極的な学術的技法であることを示している。適切な引用こそが、レポートの説得力を高めるための重要な「根拠」提示の手段となる。

\textbf{\subsubsection{AIによる補足:重要論点の拡張}}
本講義では、なぜ引用が必要か(Why)という倫理的側面と、盗用の定義(What)が中心であった。しかし、MBAの実務において同様に重要な、\textbf{「どのように」引用すべきか(How)}という具体的な方法論については言及が限定的であった。

\textbf{\paragraph{具体的な引用の方法論(直接引用と間接引用)}}
盗用を回避し、適切に「根拠」を示すためには、引用の技術を習得する必要がある。
\begin{itemize}
	\item \textbf{直接引用 (Direct Quotation)}:
	      他者の記述を\textbf{一言一句そのまま}用いる方法。日本語の場合は原則としてカギ括弧(「」)で囲み、出典(著者名、発行年、ページ番号など)を直後に明記する。

	\item \textbf{間接引用 (Indirect Quotation / Paraphrasing)}:
	      他者のアイデアや分析結果を、\textbf{自分自身の言葉で要約・言い換え}て記述する方法。間接引用は、MBAのレポートで最も多用される技術の一つである。重要なのは、言い換えた場合であっても、その\textbf{アイデアの源泉}が他者にある限り、必ず出典を明記しなければならない点である。この出典明記を怠ることが、学生が最も陥りやすい「盗用」の形態の一つである。
\end{itemize}
また、本文中での引用と、巻末の\textbf{参考文献リスト (Reference List)}は連動していなければならず、APAスタイル、ハーバードスタイルなど、学術分野や指導教員によって定められた引用スタイルに従う必要がある。

\subsection{結論}
本講義は、MBAの学習成果をまとめる上で最も基本的な前提である\textbf{研究倫理}の重要性を再確認するものであった。特に、\textbf{捏造}、\textbf{改ざん}、\textbf{盗用}(FFP)は、学術コミュニティからの追放を意味する重大な不正行為であると定義された。

本講義から得られる実践的な教訓は、以下の二点に集約される。第一に、\textbf{盗用}は意図の有無を問わず、他者の成果を出典明記なしに使用したという客観的な事実によって認定されるということ(「うっかり」の弁明不可)。第二に、\textbf{引用}とは、盗用を回避するための消極的な防衛策であると同時に、自らの主張の「根拠」を明示し、議論の信頼性を高めるための積極的な学術技法であるということ。MBA学生は、倫理観の保持と、直接引用・間接引用(パラフレーズ)といった具体的な引用技術の両方を習得することが不可欠である。

\subsection{重要キーワード一覧}
人名:(該当なし)

\vspace{\baselineskip}
普遍的MBA概念:研究倫理、捏造、改ざん、盗用、引用、参考文献、著作権

\subsection{理解度確認クイズ}
\begin{enumerate}
	\item 研究倫理における「FFP」とは、それぞれ何を指すか?
	\item 「盗用(Plagiarism)」を具体的に定義せよ。
	\item 「捏造(Fabrication)」と「改ざん(Falsification)」の違いを説明せよ。
	\item 他者の論文のアイデアを自分自身の言葉で要約(パラフレーズ)して記述する場合、出典の明記は必要か?
	\item 意図せず(うっかり)出典を明記し忘れた場合、それは盗用とみなされるか?
	\item 「自己盗用(Self-Plagiarism)」とは何か?
	\item インターネット上のブログ記事から得たグラフを自身のレポートに使用する場合、引用は必要か?
	\item 「周知の事実(Common Knowledge)」とは何か、またその取り扱いについて述べよ。
	\item 「引用(Citation)」と「参考文献リスト(Reference List)」の役割の違いは何か?
	\item 著作権が切れた古典的な文献(例:アダム・スミスの『国富論』)のアイデアを利用する場合、学術レポートにおいて引用は必要か?
	\item 他者の論文の一文を、単語をいくつか同義語に入れ替えただけでほぼそのまま使用する行為(パッチライティング)は許容されるか?
	\item 引用の学術的な目的のうち、盗用回避(倫理的側面)以外の主要な目的を一つ挙げよ。
	\item 「直接引用」はどのような場合に用いるのが最も効果的か?
	\item 学術論文やレポートにおいて、一次資料(Primary Source)と二次資料(Secondary Source)を引用する際、どちらがより信頼性の高い根拠とされるか?
	\item 研究不正行為が発覚した場合に想定される、学術的な罰則(例:単位)以外の重大な処分を2つ挙げよ。
\end{enumerate}

\subsubsection*{解答一覧}
1. 捏造(Fabrication)、改ざん(Falsification)、盗用(Plagiarism)、2. 他者のアイデア、言葉、データなどを適切な出典明記なしに自分のものとして使用すること、3. 捏造は「存在しないデータ」を作ること、改ざんは「存在するデータ」を意図的に操作・変更すること、4. はい、アイデアの源泉として必ず必要である、5. はい、意図の有無に関わらず、帰属の欠如自体が盗用とみなされる、6. 自身が過去に発表した著作物を、適切に引用せず再利用すること(二重投稿など)、7. はい、テキストと同様に図表も出典明記が必須である、8. 引用元を示す必要がないほど一般的に知られている事実(例:地球は太陽の周りを回っている)。ただし、判断に迷う場合は引用することが推奨される、9. 引用は本文中でアイデアの出所を具体的に示す行為、参考文献リストは引用した文献の完全な書誌情報を巻末に一覧で示すもの、10. はい、著作権の有無に関わらず、他者のアイデアの源泉を示すことは学術的誠実性の観点から必須である、11. いいえ、これは盗用の一形態(パッチライティング)とみなされる、12. 自身の主張の根拠(エビデンス)を示し、議論の信頼性を高めること(または、読者が原典を辿れるようにすること)、13. 著者のオリジナルの表現が特に重要である場合、またはその表現自体を分析対象とする場合、14. 一般に一次資料(当事者による直接のデータや記録)がより信頼性が高いとされる、15. 停学、退学(または、学位の剥奪)

\section{情報源の信憑性/引用・転載・参考}

\subsection{はじめに}
学術レポートや論文の品質は、その主張を支える「根拠」の質によって決定される。前回の講義で学んだ「盗用」の回避に加え、MBA学生が習得すべきもう一つの重要なスキルは、信頼に足る\textbf{情報源の信憑性}を正しく見極める能力である。いかに論理的なレポートであっても、その根拠が信頼性の低い情報(例:匿名のWeb記事)に基づいていた場合、レポート全体の価値が損なわれる。
本ノートの目的は、学術的な議論に耐えうる情報源を選定する基準を明確にし、さらに他者の著作物を利用する際の法的な区分である「引用」と「転載」の違い、および\textbf{著作権法}に基づく適切な利用方法を整理することにある。

\subsection{主要な概念と論点}
本講義では、情報源の評価方法と、著作権法に基づく利用方法の2点が中心的な論点として提示された。

\subsubsection{情報源の信憑性の評価基準}
レポートの参考文献として利用する情報には高い信頼性が求められる。
\begin{itemize}
	\item \textbf{信頼性が高い情報源}:
	      \begin{itemize}
		      \item \textbf{学術雑誌掲載論文}: 研究領域の専門家による厳格な\textbf{査読プロセス}を経ているため、一定の信頼性が担保されている。
		      \item \textbf{政府系データベース、新聞、専門書籍}: 情報の発信者(著者、所属機関、出版社)が明確であり、記述内容に対する責任の所在がはっきりしている。
	      \end{itemize}
	\item \textbf{信頼性が低い(不適切な)情報源}:
	      \begin{itemize}
		      \item \textbf{匿名のWeb記事}: 誰がどのような根拠で記述したかが不明であり、客観性や正確性の検証が不可能なため、学術レポートの参考文献として不適切である。
	      \end{itemize}
\end{itemize}

\subsubsection{信憑性確認の5つの視点}
ある文献の信憑性を担保するためには、以下の5つの視点で情報を確認する必要があるとされた。
\begin{enumerate}
	\item \textbf{誰が書いているか} (著者名、所属機関)
	\item \textbf{文献の種類は何か} (書籍、学術ジャーナル、論文集など)
	\item \textbf{どこから公開されているか} (出版社、学会、公的機関)
	\item \textbf{いつ作られたものか} (発表年度、日付)
	\item \textbf{どんな情報を元に書かれているか} (参考文献の明記)
\end{enumerate}

\subsubsection{「引用」と「転載」の法的区分}
他者の著作物(文章、図表、写真など)を利用する際、その方法は「引用」と「転載」に大別され、法的取り扱いが異なる。
\begin{itemize}
	\item \textbf{適切な引用}: 著作権者の\textbf{許諾を得ずに}利用可能。
	\item \textbf{転載}: 原則として著作権者の\textbf{許諾が必要}。
\end{itemize}

\subsubsection{著作権法第32条と「引用」の3要件}
著作権法第32条に基づき、適法な「引用」とみなされるためには、以下の3つの要件をすべて満たす必要がある。
\begin{enumerate}
	\item \textbf{公表された著作物}であること。
	\item \textbf{公正な慣行}に合致すること。
	\item 報道、批評、研究その他の\textbf{正当な範囲内}で行われること。
\end{enumerate}

\subsubsection{公正な慣行・正当な範囲の具体的判断基準}
上記2. 3.の要件を具体的に判断するために、文化庁の資料に基づき以下の3つのポイントが示された。
\begin{itemize}
	\item \textbf{主従関係}: 自身の著作物が「主」、引用部分が「従」であるという質的・量的な関係が明確であること。引用がレポートの大半を占めるような場合は、主従関係が逆転しており「引用」とはみなされない。
	\item \textbf{明瞭区分性}: 自身の記述と引用部分が、鉤括弧やブロッククオート(段落下げ)などによって明確に区別されていること。
	\item \textbf{必然性}: その部分を引用することが、自らの論を展開する上で不可欠であること。
\end{itemize}
これらの要件を満たさない場合は「転載」と判断され、著作権者の許諾が必要となる。

\subsubsection{「引用」と「参考」の記述上の違い}
\begin{itemize}
	\item \textbf{引用(直接引用)}: 著作物を\textbf{そのままの形}で利用する場合。鉤括弧「」で囲み、原文を忠実に(一字一句違わずに)記述する。
	\item \textbf{参考(間接引用)}: 著作物の内容を\textbf{自分なりに要約して}(パラフレーズして)利用する場合。「~によれば」等の形で記述する。
\end{itemize}

\subsection{応用と事例分析}
講義で示された概念を、具体的な事例に適用して分析する。

\subsubsection{事例:学術雑誌論文の信頼性}
学術雑誌の論文が信頼性の高い根拠とされるのは、単に専門家が書いているからだけではなく、「査読プロセス(ピア・レビュー)」という品質保証システムを経ている点にある。このプロセスにおいて、匿名の同領域の専門家が内容の妥当性、新規性、論理構成を厳しく審査するため、一定の客観性が担保される。

\subsubsection{不適切事例:匿名ブログ記事の利用}
匿名のブログ記事や、発信者不明のWebサイトの情報をレポートの根拠として用いることは、MBAの学習において厳しく禁じられる。これは、その情報が個人の感想なのか、検証された事実なのかを区別できず、信憑性がゼロであるためである。このような情報に基づいた分析は、単なる「感想文」であり、学術的な「レポート」とはみなされない。

\subsubsection{記述例:「引用」と「参考」}
\begin{itemize}
	\item \textbf{引用(直接引用)の例}:
	      本講義では、引用の要件について「公表された著作物は、引用して利用することができる」(\textbf{著作権法第32条}) と述べられている。

	\item \textbf{参考(間接引用)の例}:
	      文化庁の資料によれば、適法な引用とみなされるためには、自身の議論が「主」で引用部分が「従」であるという\textbf{主従関係}が明確でなければならない。
\end{itemize}

\subsection{深層背景と教訓}
講義の本論から読み取れる、実践的な背景や講師の私見を整理する。

\textbf{\paragraph{本論から逸れた寄り道トピック名:グレーゾーンとしての「転載」とリスク回避}}
講義では「引用か引用でないか、ここの判断がとても難しい」と言及された。特に図表の利用などにおいて、\textbf{主従関係}や\textbf{必然性}の要件を満たしているかどうかの判断は、専門家でも見解が分かれる場合がある。本講義の「不安がある場合は、許諾を得ておいた方が安心」という助言は、学術的な正しさ以前に、ビジネスパーソンとしての\textbf{コンプライアンス意識}とリスク回避の重要性を示唆している。実務においては、法的なグレーゾーンを攻めることよりも、ステークホルダー(この場合は著作権者)との良好な関係を維持することが優先されるべきであるという教訓である。

\textbf{\paragraph{本論から逸れた寄り道トピック名:「発信者」の信頼性評価}}
講義では「誰が書いているか」を信憑性のトップに挙げた。これは、情報そのもの(What)だけでなく、情報の発信者(Who)を評価する\textbf{批判的思考(クリティカル・シンキング)}の重要性を示している。MBAのケーススタディなどで多様な情報に触れる際、その情報がどのようなバイアス(例:特定の企業、イデオロギー)に基づいて発信されているかを見抜く訓練が不可欠である。

\textbf{\subsubsection{AIによる補足:重要論点の拡張}}
本講義では、情報源の信憑性(Who/Where)と引用の法的要件(Why/What)に焦点が当てられた。しかし、MBA論文の実践において極めて重要な「具体的な引用の技術(How)」に関して、以下の2つの論点が不足していた。

\textbf{\paragraph{1. 二次引用(孫引き)の危険性}}
講義では「どんな情報を元に書かれているか」の確認が重要とされたが、その実践である「二次引用(孫引き)」の危険性については言及がなかった。\textbf{二次引用}とは、Aという文献が引用しているBという文献を、自身がBを直接確認せずにAからの情報だけで引用することである。これは、Aの著者の解釈ミスや引用ミスをそのまま引き継いでしまうリスクが非常に高い。学術研究の原則は\textbf{一次資料(原典)}へのアクセスであり、二次引用は可能な限り避け、必ず原典を確認する習慣が求められる。

\textbf{\paragraph{2. 引用スタイル(Citation Style)の統一}}
講義では鉤括弧や「~によれば」といった日本語の記述例が示されたが、国際的なビジネススクールで求められる体系的な\textbf{引用スタイル}(例:APAスタイル、ハーバードスタイル、シカゴスタイルなど)についての言及がなかった。これらのスタイルは、本文中の引用(著者名, 発行年)と巻末の参考文献リスト(書誌情報)を体系的に紐付け、読者が必要な情報を正確に辿れるようにするための国際的なルールである。どのスタイルを採用するにせよ、レポート全体で一つのスタイルに統一することが学術文書の必須要件である。

\subsection{結論}
本講義は、MBAのレポート作成において、「盗用の回避」という倫理的な土台の上に、「いかに質の高い根拠を示し、法的に正しく利用するか」という技術的な要請を学ぶものであった。

本レポートの要点は、信頼できる情報源(査読論文、公的統計など)と不適切な情報源(匿名Web記事)の明確な区別、および\textbf{著作権法第32条}に基づく適法な「引用」の3要件(公表、公正な慣行、正当な範囲)、特に\textbf{主従関係}と\textbf{明瞭区分性}の重要性にある。

「深層背景」部分から得られる実践的な教訓は、情報に接する際の\textbf{批判的思考}(誰が発信した情報か)の徹底と、コンプライアンス意識(判断に迷う場合は許諾を得る)の重要性である。MBA学生は、情報の「消費者」であると同時に、自らも信頼性の高い情報を発信する「生産者」であることを自覚し、原典の確認と引用ルールの遵守を徹底する必要がある。

\subsection{重要キーワード一覧}
人名:(該当なし)

\vspace{\baselineskip}
普遍的MBA概念:情報源の信憑性、査読(ピア・レビュー)、一次資料、二次資料、著作権、引用、転載、著作権法第32条、公正な慣行、主従関係、明瞭区分性、二次引用(孫引き)、引用スタイル(APAなど)

\subsection{理解度確認クイズ}
\begin{enumerate}
	\item 学術雑誌に掲載された論文が、一般的なWeb記事よりも信頼性が高いとされる最大の理由は何か?
	\item 匿名のブログ記事を学術レポートの根拠として使用することが不適切とされる理由を説明せよ。
	\item 他者の著作物を利用する際、「引用」と「転載」を法的に区別する最も重要な違いは何か?
	\item \textbf{著作権法第32条}が定める、適法な「引用」の3つの要件とは何か?
	\item 引用における「主従関係」とは、質的・量的にどのような状態を指すか?
	\item 引用における「明瞭区分性」を担保するために、文章中で行うべき具体的な方法を1つ挙げよ。
	\item 引用における「必然性」の要件とは何か?
	\item 他者の論文の結論を自分自身の言葉で要約して紹介する行為は、「直接引用」と「間接引用」のどちらに該当するか?
	\item 上記8の行為(間接引用)において、出典の明記は必要か?
	\item 「二次引用(孫引き)」とは何か、またそれが推奨されない最大の理由は何か?
	\item 企業のWebサイトに掲載されているプレスリリースのデータは、学術レポートの根拠として利用可能か?
	\item レポートの大部分が他者の論文の「引用」で構成されている場合、それは著作権法第32条のどの要件を満たしていない可能性が最も高いか?
	\item 図表やグラフを他者の著作物から持ってくる場合、「引用」の要件を満たすのは一般的に容易か、困難か? その理由も述べよ。
	\item 「公正な慣行」とは、具体的に何を指していると考えられるか?
	\item APA、MLA、シカゴなどは、学術レポート作成において何と呼ばれるルールか?
\end{enumerate}

\subsubsection*{解答一覧}
1. 専門家による査読(ピア・レビュー)プロセスを経ているため、2. 発信者が不明であり、記述の正確性や客観性、根拠が検証できないため、3. 著作権者の許諾が必要か不要か、4. (1)公表された著作物であること、(2)公正な慣行に合致すること、(3)報道、批評、研究などの正当な範囲内であること、5. 自身の議論が「主」であり、引用部分が質的にも量的にも「従」である(補足的・例証的)関係であること、6. 鉤括弧「」で囲む、またはブロッククオート(字下げ)にする、7. 自身の議論を展開する上で、その部分を引用することが不可欠であること、8. 間接引用(または参考)、9. はい、アイデアの源泉が他者にあるため、必ず必要である、10. 他の文献(A)が引用している文献(B)を、Bの原典を確認せずにAから引用すること。理由は、Aの引用ミスや解釈ミスをそのまま引き継ぐリスクがあるため、11. 利用可能。ただし、発信者(企業)のバイアスを考慮し、可能な限り一次資料(元データ)や第三者による検証と併用すべきである、12. 主従関係(自身の議論が「従」になっているため)、13. 困難。図表はそれ自体が独立した著作物とみなされやすく、本文の「従」とすることが難しく、必然性の説明も難しいため(転載許諾が必要なケースが多い)、14. 出典を明記するなど、その分野の学術コミュニティで一般的に受け入れられているルールや作法、15. 引用スタイル(または、参照スタイル、Citation Style)


\section{出典の示し方}
\subsection{はじめに}
学術レポートや論文において、先行研究やデータに基づく主張を行う際、その「根拠」を明示することは盗用回避(研究倫理)の観点から不可欠である。前回の講義では、なぜ引用が必要かが議論されたが、今回は「どのように」出典を明記するかの具体的な技術、すなわち\textbf{引用スタイル(Citation Style)}に焦点を当てる。
適切なスタイルで出典を統一的に記述することは、単なる体裁の問題ではなく、読者が原典を正確に辿ることを可能にし、議論の透明性と信頼性を担保する学術的な「作法」である。本ノートの目的は、講義で紹介された主要な2つの引用スタイル、\textbf{バンクーバー方式}と\textbf{ハーバード方式}の特徴を理解し、その適切な運用方法を整理することにある。

\subsection{主要な概念と論点}
本講義では、まず「出典」そのものの定義が示され、次いでその具体的な記述方法として2つの対照的なスタイルが解説された。

\subsubsection{出典(書誌要素)の定義}
講義では藤田節子氏の定義が紹介された。それによれば、「出展(出典)」とは、具体的には\textbf{著者名}、\textbf{書名}、\textbf{出版年}などを指し、これら一つ一つの\textbf{書誌要素}の集まりが「出典」となる。学術レポートにおいては、読者がその資料を特定できるよう、これらの書誌要素を過不足なく正確に記述する必要がある。

\subsubsection{バンクーバー方式 (Vancouver Style)}
科学技術振興機構(JST)の分類に基づき紹介された第一の方式である。
\begin{itemize}
	\item \textbf{別名}: \textbf{引用順方式}。
	\item \textbf{特徴}: 本文中の引用箇所に、引用が登場した順番で\textbf{連番}(例: [1], (2), 3)など)を付与する。
	\item \textbf{参考文献リスト}: 本文中で付与した\textbf{連番順}に、書誌情報をリストアップする。
	\item \textbf{主な使用分野}: 主に自然科学や医学分野で広く用いられる。本文の記述がスッキリし、多くの文献を簡潔に参照することに適している。
\end{itemize}

\subsubsection{ハーバード方式 (Harvard Style)}
紹介された第二の方式であり、MBAを含む社会科学分野で主流とされる方式である。
\begin{itemize}
	\item \textbf{別名}: \textbf{著者名・発行年方式} (Author-Date system)。
	\item \textbf{特徴}: 本文中の引用箇所に、括弧書きで\textbf{著者名と発行年}(例: (Tanaka, 2023))を直接記述する。
	\item \textbf{参考文献リスト}: 著者名の\textbf{アルファベット順}(または五十音順)に書誌情報をリストアップする。本文中の引用(著者名)とリストが直感的に対応する。
	\item \textbf{主な使用分野}: 主に社会科学、人文科学分野で広く用いられる。「誰が」「いつ」発表した議論なのかが、本文を読むだけですぐに把握できる利点がある。
\end{itemize}

\subsection{応用と事例分析}
講義で示された2つの方式は、論文やレポートの「見た目」と「情報の提示方法」において明確な違いを生む。

\subsubsection{バンクーバー方式の記述例}
\begin{itemize}
	\item \textbf{本文の記述例}:
	      「近年の研究[1]によれば、市場の変動性が高まっているとされる。また、この傾向は日本市場においても同様である[2]。」
	\item \textbf{参考文献リストの例}:
	      \begin{enumerate}
		      \item [1] A. Sato, \textit{Global Market Trends}, 2023.
		      \item [2] B. Suzuki, \textit{Japanese Economic Review}, 2022.
	      \end{enumerate}
\end{itemize}

\subsubsection{ハーバード方式の記述例}
\begin{itemize}
	\item \textbf{本文の記述例}:
	      「近年の研究(Sato, 2023)によれば、市場の変動性が高まっているとされる。また、Suzuki (2022)は、この傾向は日本市場においても同様であると指摘している。」
	\item \textbf{参考文献リストの例}: (アルファベット順)
	      \begin{itemize}
		      \item Sato, A. (2023). \textit{Global Market Trends}.
		      \item Suzuki, B. (2022). \textit{Japanese Economic Review}.
	      \end{itemize}
\end{itemize}

\subsubsection{MBAにおける適用の考察}
MBAのレポートや論文では、特定の経営学者(例:ポーター、ドラッカー)の理論や、特定の先行研究((Aaker, 1991)など)に基づいて自らの議論を展開することが多い。そのため、本文中で即座に「誰の」「いつの」議論を参照しているかがわかる\textbf{ハーバード方式(著者名・発行年方式)}が、議論の文脈を明確にする上で非常に適している。

\subsection{深層背景と教訓}
講義の本論から読み取れる、実践的な背景や講師の私見を整理する。

\textbf{\paragraph{本論から逸れた寄り道トピック名:最大のルールは「混在させない」こと}}
講義では2つの方式が並列に「紹介」されたが、アカデミック・ライティングの実践において最も重要な教訓は、これら2つ(あるいは他のスタイル)を\textbf{絶対に混在させない}ことである。レポートの冒頭でバンクーバー方式(連番)を使い、途中からハーバード方式(著者名, 年)に切り替えるといった記述は、最も稚拙なミスとみなされる。どちらのスタイルを選択するにせよ、一つのレポート内では\textbf{一つのスタイルに厳格に統一する}規律が求められる。

\textbf{\paragraph{本論から逸れた寄り道トピック名:科学技術振興機構(JST)の役割}}
本講義で、引用スタイルの分類元として「科学技術振興機構(JST)」が言及された。これは、JSTが日本の科学技術情報の流通を担う中核機関であり、学術論文データベース(J-STAGEなど)を運営しているためである。JSTは、単に論文を収集するだけでなく、学術情報の「書き方」「示し方」といったメタなルール標準化においても、参照すべき権威ある情報源であることを示唆している。

\textbf{\subsubsection{AIによる補足:重要論点の拡張}}
本講義では、引用スタイルを「バンクーバー」と「ハーバード」という2つの大きな型(Type)に分類して紹介した。これは概念の理解には有効だが、MBAの実務で要求される具体的なスタイルについて、以下の重要な論点が補足されるべきである。

\textbf{\paragraph{「ハーバード方式」の具体例としての「APAスタイル」}}
講義で紹介された「ハーバード方式(著者名・発行年方式)」は、特定の単一のルールを指すのではなく、著者名と発行年を用いるスタイルの「総称」である。
MBAを含む社会科学分野で、このハーバード方式の代表格として世界的に最も広く採用されているのが、\textbf{APAスタイル(American Psychological Association: アメリカ心理学会)}の定める書式である。

APAスタイルは、ハーバード方式の基本的な考え方(本文中に著者名・発行年、リストは著者名順)を踏襲しつつ、書誌要素の記述順序、句読点、イタリックの使用法、DOI(Digital Object Identifier)の記載方法など、あらゆるケースについて極めて詳細なルールを定めている。MBA学生が「ハーバード方式で書く」ことを求められた場合、それは多くの場合、この\textbf{APAスタイル(第7版など)}の厳密なルールに従うことを意味している。

\subsection{結論}
本講義は、学術レポートにおける「出典の示し方」という具体的な技術に焦点を当てたものであった。信頼できる情報源(書誌要素)を明記する上で、主要な2つの引用スタイル、すなわち\textbf{バンクーバー方式(引用順方式)}と\textbf{ハーバード方式(著者名・発行年方式)}が存在することを学んだ。前者は本文の簡潔性に優れ(主に理系)、後者は議論の出所(著者)の明示性に優れる(主に文系・社会科学)。

本講義から得られる実践的な教訓は、どちらの方式を選択するかに加え、選択したスタイルをレポート全体で\textbf{厳格に統一する}という規律の重要性である。MBA学生としては、自身の専門分野で標準とされるハーバード方式、特にその代表格である\textbf{APAスタイル}の具体的な書式ルールを習得することが、信頼性の高いレポートを作成する上で不可欠な実践的スキルとなる。

\subsection{重要キーワード一覧}
人名:藤田節子

\vspace{\baselineskip}
普遍的MBA概念:引用スタイル、バンクーバー方式、ハーバード方式、引用順方式、著者名・発行年方式、書誌要素、参考文献リスト、APAスタイル

\subsection{理解度確認クイズ}
\begin{enumerate}
	\item 出典を明記する際に記述すべき「書誌要素」の具体例を3つ挙げよ。
	\item バンクーバー方式の別名(特徴)は何か?
	\item ハーバード方式の別名(特徴)は何か?
	\item 本文中に引用した順に [1], [2]... と番号を振る引用スタイルは何か?
	\item 本文中に (著者名, 発行年) を記述する引用スタイルは何か?
	\item バンクーバー方式の参考文献リストは、どのような順序で並べられるか?
	\item ハーバード方式の参考文献リストは、どのような順序で並べられるか?
	\item 医学や自然科学分野で比較的多く用いられるスタイルはどちらか?
	\item 社会科学(MBAなど)で比較的多く用いられるスタイルはどちらか?
	\item レポートを作成する際、バンクーバー方式とハーバード方式を混在させて使用してもよいか?
	\item 「APAスタイル」とは何か? 講義で紹介されたどちらの方式の仲間か?
	\item なぜハーバード方式(著者名・発行年方式)は、社会科学の議論に適しているとされるのか?
	\item 複数の異なる文献を同じ箇所でまとめて参照する場合、バンクーバー方式ではどのように記述するか? (例)
	\item 複数の異なる文献を同じ箇所でまとめて参照する場合、ハーバード方式ではどのように記述するか? (例)
	\item 引用スタイルの主な目的を、盗用防止以外の観点から2つ挙げよ。
\end{enumerate}

\subsubsection*{解答一覧}
1. 著者名, 書名, 出版年(または、雑誌名, 巻号, ページなど)、2. 引用順方式、3. 著者名・発行年方式、4. バンクーバー方式、5. ハーバード方式、6. 本文中の引用順(連番順)、7. 著者名のアルファベット順(または五十音順)、8. バンクーバー方式、9. ハーバード方式、10. いいえ、レポート全体で一つのスタイルに統一しなければならない、11. アメリカ心理学会が定める引用スタイルで、ハーバード方式(著者名・発行年方式)の一種である、12. 本文中で「誰が」「いつ」発表した議論なのかが即座にわかるため、13. 例: [1-3, 5] や [1, 2, 5] のように連番をまとめる、14. 例: (Sato, 2020; Suzuki, 2021; Tanaka, 2019) のようにセミコロンで区切る、15. 議論の根拠(エビデンス)を示すこと、読者が原典(一次資料)を辿れるようにすること(または、先行研究への敬意)


\section{参考文献の書き方}

\subsection{はじめに}
学術レポートにおける主張の信頼性は、その根拠となった\textbf{参考文献}を読者が正確に追跡(トレース)できるか否かにかかっている。先行研究への敬意と盗用の回避(研究倫理)に加え、他者がその情報源にたどり着けるよう、正確な\textbf{書誌要素}を統一された書式で記述することは、研究者の重要な「技術的」責務である。
本講義では、日本の科学技術分野で標準とされる\textbf{SIST 02(科学技術情報流通技術基準)}の「参考文献の書き方」をベースに、単行本、雑誌論文、ウェブサイトといった主要な情報源の具体的な記述方法と、特に注意が必要な「図表」の引用・転載ルールについて解説する。本ノートの目的は、これらの具体的な記述ルールを整理し、学術的な正確性を担保するための実践的知識を習得することにある。

\subsection{主要な概念と論点}
本講義では、SIST 02を基準とした参考文献の構成要素と、主要な媒体別の記述ルール、および図表利用の法的区分が示された。

\subsubsection{参考文献の4つの必須書誌要素}
参考文献として十分な情報を示すためには、原則として以下の4つの書誌要素グループを、この順番で記述する必要がある。
\begin{enumerate}
	\item \textbf{著者に関する書誌要素} (著者名, 編集者名, 翻訳者名など)
	\item \textbf{表題に関する書誌要素} (書名, 雑誌名, 論文題名など)
	\item \textbf{出版・物理的特徴に関する書誌要素} (出版社, 出版地, 出版年, 版表示, 巻号, ページなど)
	\item \textbf{中(傍)記的な書誌要素} (媒体表示 [online], 入手先URL, 入手日付 [参照日] など)
\end{enumerate}

\subsubsection{媒体別の主要な記載ポイント}
\begin{description}
	\item[単行本(書籍)] 必要な要素は「著者名」「書名」「版表示(初版は不要)」「出版地(必須ではない)」「出版社」「出版年」「総ページ数」である。
	\item[雑誌論文] 必要な要素は「著者名」「論文名」「掲載雑誌名」「出版年」「巻数」「号数」「論文の掲載ページ範囲(始めのページ-終わりのページ)」である。
	\item[ウェブページ] 必要な要素は「著者名(または団体・機関名)」「ウェブページの題名」「ウェブサイトの名称」「更新日付(不明な場合は省略可)」「入手先(URL)」「入手の日付(参照日)」である。
\end{description}

\subsubsection{図表の引用と転載の区分}
他者の論文や著作物に掲載されている図表を利用する場合、その利用形態によって法的な手続きが根本的に異なる。
\begin{description}
	\item[「引用」の範囲内とみなされる場合]
	      著作権法で認められる正当な「引用」の範囲内(主従関係、明瞭区分性、必然性を満たす)であれば、\textbf{著作権者の許諾は不要}である。ただし、図表のキャプション(脚注)に\textbf{出典元を明記}する義務がある。

	\item[「引用」の範囲を超える(=転載)場合]
	      図表が議論の「主」となる、必然性がない、あるいは質的・量的に引用の範囲を超える場合は\textbf{「転載」}とみなされ、原則として\textbf{著作権者の許諾が必要}となる。

	\item[図表を改変して利用する場合]
	      元の図表に手を加えて利用する場合も同様の区分が適用される。「引用」の範囲内であれば「(著者名, 年)をもとに作成」などと出典を明記すればよいが、「転載」に該当する場合は改変すること自体にも著作権者の許諾が必要となる。
\end{description}

\subsection{応用と事例分析}
講義で示された基準に基づき、具体的な記述例を整理する。

\subsubsection{単行本(書籍)の記述例}
句読点のルールとして、書誌要素の大きなグループの終わり(例:著者名グループ、書名グループ)には\textbf{ピリオド}を打ち、同一グループ内の要素(例:出版地と出版社)は\textbf{カンマ}で区切る。
\begin{itemize}
	\item \textbf{和書の例}: 著者名. 書名. 版表示. 出版社, 出版年, 総ページ数 p.
	\item \textbf{洋書の例}: Author, L. F. \textit{Title of Book}. Edition. Publisher, Year, Total pages p.
	      \begin{itemize}
		      \item 洋書の著者名は「姓, 名」の順で記載し、名(ファーストネーム)はイニシャルにすることが多い。
		      \item 洋書のタイトルは、冠詞・接続詞・前置詞を除く各単語の先頭を大文字で記述するのが一般的である。
		      \item 著者が複数名の場合は、セミコロン(;)で区切ることがある。(SIST 02ではカンマ区切りが推奨される)
	      \end{itemize}
\end{itemize}

\subsubsection{ページ数の記述ルール}
\begin{itemize}
	\item \textbf{総ページ数}: `350 p.` (書籍全体のページ数)
	\item \textbf{単一ページ参照}: `p. 350` (350ページ目のみ)
	\item \textbf{連続ページ参照}: `p. 350-360`
	\item \textbf{非連続ページ参照}: `p. 350, 362`
\end{itemize}
※表記法には `p.` 以外に `pp.` を用いるスタイルも存在するが、SIST 02では `p.` に統一されている。

\subsubsection{雑誌論文の記述例}
講義で示された『Diamond Harvard Business Review』の事例をSIST 02に基づき整理する。
\begin{itemize}
	\item \textbf{著者名}: Guhan Subramanian
	\item \textbf{翻訳者}: 幸田 幸伸 (訳)
	\item \textbf{論文名}: 競争優位の源泉となるコーポレート・ガバナンスの3つの原則
	\item \textbf{特集名}: 特集:コーポレート・ガバナンス
	\item \textbf{雑誌名}: Diamond Harvard Business Review
	\item \textbf{出版年}: 2016
	\item \textbf{巻(号)}: 41(3)
	\item \textbf{ページ範囲}: p. 75-87

	      \textbf{記述例}:
	      Guhan Subramanian. 幸田 幸伸, 訳. 競争優位の源泉となるコーポレート・ガバナンスの3つの原則. 特集:コーポレート・ガバナンス. \textit{Diamond Harvard Business Review}. 2016, \textbf{41}(3), p. 75-87.
\end{itemize}
※雑誌名(\textit{Diamond Harvard Business Review})はイタリック体、巻数(\textbf{41})はボールド体にするのが一般的である。

\subsubsection{ウェブページの記述例}
ウェブサイトの引用で最も重要なのは、読者がその情報にアクセスするための\textbf{URL}と、その情報がいつ存在したかを証明する\textbf{参照日(入手日)}である。
\begin{itemize}
	\item \textbf{記述例}:
	      著者名(または機関名). "ウェブページの題名". \textit{ウェブサイトの名称}. 更新日, \url{https://...}, (参照 2025-11-04).
\end{itemize}

\subsection{深層背景と教訓}
講義の本論から読み取れる、実践的な背景や講師の私見を整理する。

\textbf{\paragraph{本論から逸れた寄り道トピック名:「引用の範囲」というグレーゾーンとリスク管理}}
講義では、図表の利用が「引用」か「転載」かの判断は「とても難しい」と率直に述べられた。特に「1点、2点の図表であれば引用の範囲とみなされるが、質的・量的に1点でも転載とみなされる可能性」に言及しており、これは明確な線引きがないグレーゾーンであることを示している。
ここから得られるMBA的な教訓は、学術的・法的な\textbf{リスク管理}の視点である。「不安がある場合、著作権者に許諾を得た方が安心」という助言は、コンプライアンスを重視し、後の法的トラブルを未然に防ぐという、ビジネスパーソンとして極めて重要な姿勢を示している。

\textbf{\paragraph{本論から逸れた寄り道トピック名:SIST 02という「技術標準」}}
本講義は、参考文献の書き方を「作法」や「マナー」としてではなく、\textbf{SIST 02}という具体的な「技術基準」に基づいて説明している。これは、参考文献の記述が、単なる体裁の問題ではなく、科学技術情報の「流通」を支えるための客観的で精密なルール体系であることを示唆している。MBAにおいても、感覚的なレポートではなく、標準化されたルールに基づく厳密な記述が求められるという教訓である。

\textbf{\subsubsection{AIによる補足:重要論点の拡張}}
本講義では、SIST 02に基づき「手動で」参考文献を記述する際の構成要素が詳細に解説された。しかし、現代のMBAや学術研究の実践において、作業の効率性と正確性を担保するために不可欠な、以下の2つの論点が補足されるべきである。

\textbf{\paragraph{1. 参考文献管理ツールの活用}}
SIST 02やAPAスタイルなど、厳密なルールを手動で適用し続けることは、非常に時間がかかり、タイプミスや書式間違い(ピリオドとカンマの違いなど)を誘発しやすい。
現代の研究活動では、\textbf{参考文献管理ツール}(例:\textbf{EndNote}, \textbf{Mendeley}, \textbf{Zotero}など)の活用が常識となっている。これらのツールは、論文や書籍の書誌情報をデータベースとして管理し、Wordなどの執筆ツールと連携して、本文中の引用挿入と巻末の参考文献リストの生成を、指定したスタイル(SIST, APA, Harvardなど)で\textbf{自動的に}行う。この技術の活用は、MBAにおける生産性向上に直結する。

\textbf{\paragraph{2. DOI (Digital Object Identifier) の重要性}}
講義ではウェブページのURLが重視されたが、学術論文(特に電子ジャーナル)の世界では、\textbf{DOI(デジタルオブジェクト識別子)}がURLよりも恒久的な識別子として標準的に用いられる。DOIは、リンク切れを起こしやすいURLとは異なり、コンテンツ自体に紐付いた永続的なIDである。近年の主要な引用スタイル(APA第7版など)では、DOIが利用可能な場合は、URLの代わりにDOIを記載することが強く推奨(または必須)とされている。

\subsection{結論}
本講義は、学術レポートの信頼性を担保するための具体的な技術として、\textbf{参考文献}の標準的な記述法(SIST 02準拠)を解説したものであった。
本レポートの要点は、参考文献が4つの\textbf{書誌要素}(著者, 表題, 出版, 中記)のグループから構成されること、そして媒体(単行本, 雑誌, ウェブ)ごとに記述すべき要素が異なることである。
特に重要な論点として、\textbf{図表の利用}には「引用」と「転載」の法的な区分が存在し、安易な利用は\textbf{著作権侵害}のリスクを伴うことが強調された。

「深層背景」部分から得られる実践的な教訓は、「引用の範囲」というグレーゾーンに対しては、許諾を得るという\textbf{リスク回避的なアプローチ}が賢明であるという点にある。MBA学生としては、SIST 02のような厳密なルールを理解すると同時に、AI補足で示した\textbf{参考文献管理ツール}を活用して作業の正確性と効率性を高め、\textbf{DOI}のような最新の識別子を用いて情報の恒久性を担保する、現代的な学術スキルを習得することが求められる。

\subsection{重要キーワード一覧}
人名:藤田節子

\vspace{\baselineskip}
普遍的MBA概念:参考文献, 引用スタイル, 書誌要素, SIST 02, 著作権, 引用, 転載, 出典, DOI, 参考文献管理ツール (Zotero, Mendeleyなど)

\subsection{理解度確認クイズ}
\begin{enumerate}
	\item 参考文献の書誌要素は、SIST 02において大きく4つのグループに分類されるが、それは何か?
	\item 書籍の引用において、初版(第1版)の場合、版表示は記述する必要があるか?
	\item 書籍の総ページ数が120ページである場合、SIST 02の慣例ではどのように記述するか?
	\item 論文が特定のページ(例:25ページ)のみの場合、どのように記述するか?
	\item ウェブページを引用する際、URLの他に必ず記述すべき「日付」は何か?
	\item 他者の論文の図表を、著作権者の許諾なしに利用できるのは、それが何とみなされる場合か?
	\item 上記6の場合でも、図表のキャプションに必ず明記しなければならないものは何か?
	\item 「引用」の範囲を超えると判断された場合、それは「何」と呼ばれ、何が必要か?
	\item 他者の図表を「改変」して利用する場合でも、著作権者の許諾が必要となるのはどのようなケースか?
	\item 学術論文において、URLよりも恒久的で信頼性の高い識別子として用いられるものは何か?
	\item EndNoteやZoteroといったソフトウェアは、何と呼ばれるツールか?
	\item なぜ参考文献管理ツールの利用が推奨されるのか?(効率性以外の観点から)
	\item 雑誌論文の引用において「Vol. 10, No. 2」とある場合、「10」と「2」はそれぞれ何を指すか?
	\item 「転載」の許諾を得ずに他者の図表を無断で利用した場合、法的にどのような問題が生じる可能性があるか?
	\item 参考文献リストを作成する際の句読点(ピリオドとカンマ)の基本的な使い分けを説明せよ。
\end{enumerate}

\subsubsection*{解答一覧}
1. 著者, 表題, 出版, 中記(入手先など)、2. 不要である、3. 120 p.、4. p. 25、5. 入手の日付(参照日)、6. 「引用」の範囲内とみなされる場合、7. 出典元(例:著者名, 年など)、8. 「転載」と呼ばれ、著作権者の許諾が必要、9. 「引用」の範囲を超える(=転載)とみなされる場合、10. DOI (デジタルオブジェクト識別子)、11. 参考文献管理ツール(または書誌管理ソフト)、12. 引用スタイルの統一性を担保し、書式の間違いを防ぐため(正確性)、13. Vol. 10は「巻数」、No. 2は「号数」、14. 著作権侵害、15. ピリオドは書誌要素の大きなグループの区切り、カンマは同一グループ内の要素の区切り


% \section{句読点の用法}

\section{教員の実践例}

\subsection{はじめに}
これまでの講義で、研究倫理の観点から「なぜ」引用が必要か、そして「何を」盗用とみなすかについて学んできた。本講義では、さらに実践的な「どのように」引用・参考文献を記述するかという技術に焦点を当てる。
講義では、複数の教員の実践例が紹介され、分野によって引用スタイルが大きく異なる(例:文末にまとめる方式と、ページ下部の脚注に記す方式)ことが示された。本ノートの目的は、これらのスタイルの多様性を認識すると同時に、分野やスタイルを問わず共通する引用の「黄金律」、すなわち\textbf{一貫性}と\textbf{追跡可能性}の重要性を整理することにある。

\subsection{主要な概念と論点}
本講義では、引用の実践的な使い分けと、学術分野によるスタイルの違い、そしてそれに共通する原則が解説された。

\subsubsection{「参考」と「引用」の実践的使い分け}
論文の説得力は、自身の主張が先行研究に基づいていることを示すことで高まる。
\begin{itemize}
	\item \textbf{参考(間接引用)}:
	      他者の研究成果(例:500人の起業家への調査結果)や主張(例:ある学長のリーダーシップ論)を、自身の言葉で\textbf{要約・解釈}して記述すること。自身の主張を補強するために用いられる。
	\item \textbf{引用(直接引用)}:
	      他者の記述を\textbf{一文字足りとも修正せず}、そのままの形で持ってくること。著者の意図を正確に伝える必要がある場合に用いる。原文を改変した場合、\textbf{著者の意図をねじまげた}ことになり、重大な不正行為とみなされる。
\end{itemize}

\subsubsection{直接引用の形式}
\begin{itemize}
	\item \textbf{短い引用}: 本文中に組み込み、鉤括弧`「」`で囲む。
	\item \textbf{長い引用}: 本文から独立させ、\textbf{字下げ(インデント)}を行う(ブロッククオート)。
\end{itemize}
※ページ数を併記するか否かは、所属する学会や分野のスタイルによって異なる。

\subsubsection{引用スタイルの多様性}
講義では、学術分野によって主流なスタイルが全く異なることが、2つの実践例によって示された。
\begin{description}
	\item[スタイルA:文末リスト方式]
	      経営学や社会科学で一般的に見られる方式。本文中には著者名と発行年(例:(Author, 2015))などを簡潔に記し、論文の\textbf{一番最後}に、参考文献の完全なリストをまとめて掲載する。

	\item[スタイルB:脚注方式]
	      法学や人文学で一般的に見られる方式。本文中の該当箇所に小さな番号(上付き文字)を振り、その\textbf{ページの最下部(脚注)}に、対応する参考文献の書誌情報を記述する。
\end{description}

\subsubsection{同一著者・同一年の文献の識別}
仮にある著者が2015年に2冊の本を出版しており、両方を参照する場合、(Author, 2015) という表記だけでは区別がつかない。この場合、`2015a`, `2015b` のようにアルファベットを付加して、両者を明確に識別するルールが一般的である。

\subsubsection{引用における「黄金律」}
講義で最も強調されたのは、スタイルそのものの優劣ではなく、以下の2つの原則を守ることである。
\begin{enumerate}
	\item \textbf{一貫性 (Consistency)}:
	      最大の過ちは、複数のスタイルを\textbf{混在}させることである。脚注方式で始めたにもかかわらず、途中から文末リスト方式に変える、といった行為は許されない。レポートの最初から最後まで、\textbf{一つのスタイルを厳格に統一する(一気通貫)}ことが求められる。

	\item \textbf{識別性と追跡可能性 (Identifiability \& Traceability)}:
	      スタイルの目的は、読者がその文献を\textbf{迅速に辿り着ける}ようにすることである。書式(句読点や順序)の違いは副次的であり、最も重要なのは、文献を特定するための\textbf{必要十分な書誌要素}(誰が, 何を, いつ, どの媒体で)が正確に記載されていることである。
\end{enumerate}

\subsection{応用と事例分析}
講義で示された2つの異なる引用スタイルを比較分析する。

\subsubsection{事例1:社会科学系(文末リスト方式)の実践}
ある教員(経営学)の事例では、自身の論文で「参考」と「引用」を明確に使い分けていた。
\begin{itemize}
	\item \textbf{「参考」の活用}: 海外の起業家調査と、国内の書籍(ある学長によるリーダーシップ論)の2つの異なる先行研究を自身の言葉でまとめ、それらを組み合わせることで、独自の主張(リーダーとしての疑似体験の重要性)の\textbf{説得力}を高めていた。
	\item \textbf{「引用」の活用}: 他の著者の見解を正確に示すため、短い引用(`「」`を使用、ページ番号併記)と長い引用(インデント)を使い分けていた。
	\item \textbf{リスト形式}: 参考文献はすべて論文の\textbf{末尾}に集約し、日本語文献と外国語文献を分けてリスト化していた。
\end{itemize}

\subsubsection{事例2:法学系(脚注方式)の実践}
別の教員(法学)の事例では、全く異なるスタイルが採用されていた。
\begin{itemize}
	\item \textbf{「脚注」の活用}: 本文中の引用箇所には `...である[1]` のように小さな番号のみを付記。
	\item \textbf{リスト形式}: `[1] 著者名...` といった完全な書誌情報が、論文末尾ではなく、その\textbf{ページの下部}にある脚注欄に直接記述されていた。
	\item \textbf{示唆}: この比較から、学術分野ごとに引用の「慣習」が大きく異なることがわかる。
\end{itemize}

\subsection{深層背景と教訓}
講義の本論および実践例から読み取れる、周辺的な教訓を整理する。

\textbf{\paragraph{本論から逸れた寄り道トピック名:引用の説得力(「巨人の肩の上に立つ」)}}
ある教員は、「私の意見だとしても説得力がない」と述べた。これは、学術研究が個人の感想文ではなく、先行研究という「巨人の肩の上に立つ」営みであることを示している。適切な引用と参考は、自身の主張を客観的に補強し、議論を学術的な文脈に位置づけるための必須の行為である。

\textbf{\paragraph{本論から逸れた寄り道トピック名:スタイルの「コピペ」という実務的ハック}}
「ルールがめんどくさい」と感じる学生に対し、教員は実務的なアドバイスとして「どなたかの先生の参考文献の例をコピペして、自分のやつを打ち込んでいく」ことを推奨した。これは、ルールの詳細を暗記するよりも、信頼できる「お手本」の\textbf{フォーマット}を流用する方が、\textbf{一貫性}を保つ上で現実的かつ安全であるという処世術を示している。

\textbf{\paragraph{本論から逸れた寄り道トピック名:引用改変の厳禁}}
直接引用のルールとして「必ず1文字足りとも修正してはいけない」という点が強く強調された。これは倫理的な要請であり、引用を修正する行為は、著者の意図を恣意的にねじまげる\textbf{「改ざん」}に等しい重い不正行為であるという警告である。

\textbf{\subsubsection{AIによる補足:重要論点の拡張}}
本講義では、スタイルの多様性と一貫性の重要性が強調されたが、その「一貫性」を担保するための現代的な方法論について、以下の2点の重要な論点が不足していた。

\textbf{\paragraph{1. 参考文献管理ツールの活用}}
講義では「フォーマットをコピペする」という手動の方法が紹介されたが、これは非効率的であり、ヒューマンエラー(句読点のミス、順序の間違いなど)を誘発しやすい。現代のMBAや学術研究の実践において、\textbf{参考文献管理ツール}(例:\textbf{Zotero}, \textbf{Mendeley}, \textbf{EndNote}など)の活用は不可欠である。
これらのツールは、書誌情報をデータベース化し、執筆ソフト(Wordなど)と連携して、本文中の引用挿入と巻末の参考文献リストの生成を、指定したスタイル(APA, Chicago, Harvardなど)で\textbf{自動的に}行う。これにより、\textbf{一貫性}が完全に担保され、スタイルの変更も一瞬で完了する。

\textbf{\paragraph{2. 主要なスタイルガイドの名称}}
講義では「法学系」「経営学系」といった分野による違いが示されたが、それらのスタイルの具体的な「名称」には言及がなかった。
\begin{itemize}
	\item \textbf{脚注方式}の代表例は、\textbf{シカゴ・マニュアル・オブ・スタイル (Chicago Manual of Style)}であり、人文学や歴史学で広く用いられる。
	\item \textbf{文末リスト方式}の代表例は、\textbf{APAスタイル (American Psychological Association)} であり、心理学、教育学、そして経営学を含む多くの社会科学分野で標準となっている。
\end{itemize}
これらの具体的なスタイルガイドの名称を知ることで、学生は参照すべき正確なルールブックにアクセスすることができる。

\subsection{結論}
本講義は、複数の教員の実践例を比較検討することで、アカデミック・ライティングにおける\textbf{引用スタイルの多様性}(例:文末リスト方式 vs 脚注方式)を具体的に示した。

本レポートの要点は、スタイルの形式そのものよりも、\textbf{「一貫性」}を保つことが最も重要であるという「黄金律」にある。スタイルを混在させることは、学術レポートにおいて最も重大な形式的過誤の一つである。

「深層背景」から得られる教訓として、引用は単なる盗用回避の手段ではなく、自身の主張の\textbf{説得力}を高めるための積極的な戦略であること、そして直接引用における\textbf{原文の完全性}(改変の禁止)が厳格に求められることが挙げられる。
MBA学生としては、自身の分野(経営学)で標準とされる\textbf{APAスタイル}などの「文末リスト方式」を基本としつつ、AI補足で示したような\textbf{参考文献管理ツール}を積極的に活用し、ヒューマンエラーを防ぎながら「一貫性」と「追跡可能性」という引用の二大原則を確実に担保する技術を習得すべきである。

\subsection{重要キーワード一覧}
人名:(該当なし)

\vspace{\baselineskip}
普遍的MBA概念:引用、参考、直接引用、間接引用、脚注、参考文献リスト、引用スタイル、一貫性(スタイル統一)、書誌要素、追跡可能性、APAスタイル、シカゴスタイル、参考文献管理ツール(Zotero, Mendeleyなど)

\subsection{理解度確認クイズ}
\begin{enumerate}
	\item 他者の文章を要約し、自身の言葉で紹介する行為を何と呼ぶか?
	\item 他者の文章を「一文字足りとも修正せず」そのまま記述する行為を何と呼ぶか?
	\item 短い直接引用を本文中に含める際、日本語で一般的に用いられる句読点は何か?
	\item 長い直接引用を行う際の、一般的な書式上のルールは何か?
	\item 本文中の引用箇所に(著者名, 発行年)を記述し、巻末に著者名順のリストを載せるスタイルを一般に何と呼ぶか?
	\item 本文中の引用箇所に小さな番号を振り、ページ下部の脚注に書誌情報を示すスタイルを一般に何と呼ぶか?
	\item 法学や人文学で比較的よく見られるスタイルは、上記5と6のどちらか?
	\item 経営学や社会科学で比較的よく見られるスタイルは、上記5と6のどちらか?
	\item 同一著者が同一年に発表した2つの異なる文献を区別するために、一般的にどのような表記法が用いられるか?
	\item 引用スタイルにおいて、スタイルAとスタイルBを一つのレポート内で混在させることは許容されるか?
	\item 引用スタイルにおける最も重要な「黄金律」は何か?
	\item あらゆる引用スタイルに共通する、最大の目的は何か?
	\item 参考文献リストに必ず含めるべき「書誌要素」の例を3つ挙げよ。
	\item ZoteroやMendeleyといったソフトウェアは、何と呼ばれるツールか?
	\item なぜ直接引用において、原文の改変が厳しく禁じられているのか?
\end{enumerate}

\subsubsection*{解答一覧}
1. 参考(または間接引用)、2. 引用(または直接引用)、3. 鉤括弧「」、4. 字下げ(インデント)を行い、独立したブロックとして記述する、5. 著者名・発行年方式(例:ハーバード方式、APAスタイル)、6. 脚注方式(例:シカゴスタイル)、7. 6(脚注方式)、8. 5(著者名・発行年方式)、9. 発行年にアルファベットを付加する(例:2015a, 2015b)、10. いいえ、許容されない(最も避けるべき過ちの一つである)、11. 一貫性(スタイルをレポート全体で統一すること)、12. 読者が原典を正確かつ迅速に追跡(特定)できるようにすること、13. 著者名、発行年、表題(論文名・書名)、(または、出版社名、雑誌名、巻号、ページ、URLなど)、14. 参考文献管理ツール(または書誌管理ソフト)、15. 著者の意図をねじまげ、原文の意味を歪めてしまう(改ざん)ことになるため

\end{document}
\documentclass[uplatex,a4j,12pt,dvipdfmx]{jsarticle}
\usepackage{amsmath,amsthm,amssymb,bm,color,enumitem,mathrsfs,url,epic,eepic,ascmac,ulem,here,ascmac}
\usepackage[letterpaper,top=2cm,bottom=2cm,left=3cm,right=3cm,marginparwidth=1.75cm]{geometry}
\usepackage[english]{babel}
\usepackage[dvipdfm]{graphicx}
\usepackage[hypertex]{hyperref}
\title{Marketing Lecture Note 4: The Product Life Cycle (PLC)}
\author{M. O.}
\date{\today}
\begin{document}
\maketitle
\tableofcontents
\section{Organizing Lecture Materials: Strategic Implications and Contemporary Issues of the Product Life Cycle (PLC) Stages}
\subsection{Introduction}
The market is not a static entity but a dynamic existence that is constantly changing. It's inevitable that market analyses and strategies that were effective at one point will become obsolete over time. The concept of the \textbf{Product Life Cycle (PLC)} has played a central role in marketing strategy theory as a framework for predicting and systematically responding to these market shifts. This lecture note aims to deepen our understanding of the market environment, competition, consumer characteristics, and the strategic responses companies should adopt across the four stages—'Introduction', 'Growth', 'Maturity', and 'Decline'—that classify the process from a product's market launch to its withdrawal.
\subsection{Key Concepts and Issues}
\subsubsection{Definition of the Product Life Cycle (PLC)}
The \textbf{Product Life Cycle} models the trend of sales and profits from a product category (or brand)'s appearance in the market until its eventual disappearance, analogized to a biological lifespan. This model is often represented by an S-shaped curve, with time on the horizontal axis and sales revenue and profit on the vertical axis.
\begin{itemize}
	\item \textbf{I. Introduction Stage}: The period immediately following the product's market launch. Sales growth is slow, and profits are often negative (a loss) due to low product awareness, high production costs, and substantial marketing investment (especially for market development).
	\item \textbf{II. Growth Stage}: The phase where the product gains market acceptance and sales expand rapidly. Economies of scale and the \textbf{Experience Curve} begin to take effect, and profits also increase. However, the market's increasing attractiveness stimulates active entry by competitors.
	\item \textbf{III. Maturity Stage}: The stage where market growth slows, and sales level off (market saturation). Demand primarily consists of replacement purchases and repeat purchases. Competition is at its most intense, with price wars and market share battles becoming commonplace. Many firms form an \textbf{oligopoly}.
	\item \textbf{IV. Decline Stage}: The phase where sales and profits continuously decrease due to the emergence of substitutes from technological innovation or changes in consumer preferences. The number of companies withdrawing from the market increases.
\end{itemize}
\subsubsection{PLC's Driving Factors: Firm-Side and Consumer-Side}
The formation of the PLC's S-shaped curve involves changes from both the firm's side and the consumer's side.
\paragraph{Firm-Side Changes (Competition and Technology)}
\begin{itemize}
	\item \textbf{Changes in the Number of Firms}: The industry structure changes from Introduction (monopoly) $\to$ Growth (increasing entry) $\to$ Maturity (shakeout and oligopolization) $\to$ Decline (increasing withdrawal).
	\item \textbf{Focus of Technological Innovation}: \textbf{Vertical Differentiation}, which enhances product performance, is central during the Introduction and Growth stages. Competition in technological innovation is particularly active until a '\textbf{Dominant Design}' is established, where diverse early-industry technologies converge into a specific standard design. Conversely, innovation in the Maturity and Decline stages centers on production process innovation (cost reduction) and \textbf{Horizontal Differentiation}, such as design and feature diversification.
\end{itemize}
\paragraph{Consumer-Side Changes (Knowledge and Adopter Segments)}
\begin{itemize}
	\item \textbf{Accumulation of Product Knowledge}: Products have novelty during the Introduction stage, but over time, knowledge is accumulated socially, and consumers come to understand which attributes are important (e.g., cell phone camera performance).
	\item \textbf{Change in Consumer Segments}: As the PLC progresses, the segments of consumers purchasing the product change. This is explained by the Innovator Theory.
\end{itemize}
\subsubsection{Innovator Theory (Timing of New Product Purchase)}
The process by which a new product diffuses throughout the market is classified into five groups based on the attributes of the adopters.
\begin{itemize}
	\item \textbf{Innovators}: Adoption occurs in the very early part of the Introduction stage. They seek newness and are not afraid of risk. They are highly informed and often enthusiasts.
	\item \textbf{Early Adopters}: Adoption occurs in the latter half of the Introduction stage to the early Growth stage. They are local opinion leaders and have significant influence on those around them.
	\item \textbf{Early Majority}: Adoption occurs during the Growth stage. They are relatively cautious but follow the lead of Early Adopters. They act as a 'bridge,' making up the majority of the market.
	\item \textbf{Late Majority}: Adoption occurs in the Maturity stage. They are conservative and only adopt after the majority of people around them have done so.
	\item \textbf{Laggards}: Adoption occurs in the Decline stage (or late Maturity stage). They are the most conservative and value tradition.
\end{itemize}
\subsection{Application and Case Analysis}
The recommended strategies differ at each stage.
\subsubsection{Introduction Stage Strategies}
The market size is small, and growth is slow. Targets are Innovators and Early Adopters.
\begin{itemize}
	\item \textbf{Market Development and First-Mover Advantage}: Prioritizing the creation of awareness and demand for the product category itself is crucial (e.g., the early eco-car market). The goal is to be a technological pioneer and rapidly achieve the \textbf{Experience Curve} (the phenomenon where costs decline as cumulative production increases). Success can lead to becoming a synonym for the product category (e.g., Cup Noodles) and potentially enjoying a \textbf{First Mover Advantage}.
	      \begin{itemize}
		      \item Research by \textbf{Carpenter and Nakamoto (1989)} suggests the durability of this advantage, noting that many brands that were market leaders in the US market in 1923 remained leaders 60 years later.
		      \item However, \textbf{Golder and Tellis (1993)} point out that the condition for long-term advantage is not merely being the 'inventor' or 'product pioneer' but the '\textbf{market pioneer}' who actually shaped the market.
	      \end{itemize}
	\item \textbf{Pricing Strategies}:
	      \begin{enumerate}
		      \item \textbf{Market Skimming Strategy}: Setting a high price to recover development costs quickly from segments like Innovators, who have low price elasticity (e.g., the initial price of the PS3).
		      \item \textbf{Penetration Strategy}: Setting a low price to rapidly gain market share. This is effective when price is a bottleneck or when economies of scale are strongly at play.
	      \end{enumerate}
\end{itemize}
\subsubsection{Growth Stage Strategies}
Sales surge, and market uncertainty decreases, leading to the entry of many late-comer firms. The target shifts to the Early Majority.
\begin{itemize}
	\item \textbf{Market Share Expansion}: Despite competitor entry, sales grow as the overall market expands, but maintaining and improving market share is essential.
	\item \textbf{Emphasis on Product Differentiation}: To clearly distinguish from competitors, \textbf{Vertical Differentiation} (performance improvement) and differentiation through advertising and channels are required.
	\item \textbf{Appeal to Adoption Factors}: Since the Early Majority is more pragmatic than Innovators, the marketing communication needs to emphasize factors that encourage the adoption of new products (as pointed out by \textbf{Kotler, 2003; Shoemaker and Shoaf, 1975}, and others).
	      \begin{itemize}
		      \item \textbf{Relative Advantage}: Being superior to existing products.
		      \item \textbf{Compatibility}: Fitting with existing systems and values.
		      \item \textbf{Simplicity}: Being easy to understand and use.
		      \item \textbf{Observability (Communicability)}: The benefits are obvious to others.
	      \end{itemize}
\end{itemize}
\subsubsection{Maturity Stage Strategies}
Sales growth slows, and demand centers on replacement purchases. Competition intensifies, often leading to price wars. The target is the Late Majority.
\begin{itemize}
	\item \textbf{Responding to Demand Diversification}: Consumer preferences diversify, with a growing emphasis on 'taste' (horizontal attributes) over performance (vertical attributes) (e.g., the varied colors and shapes of Post-it Notes).
	\item \textbf{Market Segment Development}: Strengthening \textbf{Segmentation} and developing new niche markets can extend the PLC's Maturity stage (e.g., Glico's diversified Pocky flavors and the premium 'Baton d'or' line).
	\item \textbf{Competitive Strategy}: Strategies focus on pursuing cost leadership or specializing in a specific niche market.
\end{itemize}
\subsubsection{Decline Stage Strategies}
The market shrinks due to technological innovation (e.g., LCD TVs replacing CRT TVs) or changes in consumer lifestyles.
\begin{itemize}
	\item \textbf{Withdrawal (Harvest)}: Minimizing investment and gradually exiting the market. Care must be taken regarding the presence of \textbf{Exit Barriers} (e.g., relationships with distributors).
	\item \textbf{Focus on Core Users}: Continuing the business by concentrating on users with high \textbf{Brand Loyalty}. Remaining firms may gain residual profits if competitors withdraw.
	\item \textbf{Re-positioning}: Resetting the PLC by finding a new use or market for the product (e.g., the 'Umaibo' snack expanding its sales channels from traditional candy stores to convenience stores, gaining a new customer base).
\end{itemize}
\subsection{Deeper Context and Lessons}
\textbf{\paragraph{The Convergence of Dominant Design}}
The lecture's example of bicycles (Michaux $\to$ Penny Farthing $\to$ Safety Bicycle) symbolizes the concept of the \textbf{Dominant Design}. Early in an industry, diverse design philosophies (technologies) proliferate, but as innovation continues, they converge into a specific design most accepted by the market (e.g., the modern 'safety' bicycle with near-equal-sized front and rear wheels and chain drive). Once this Dominant Design is established, the focus of competition shifts from product innovation itself to production process innovation and horizontal differentiation. This is strongly linked to the transition from the PLC's Growth to Maturity stage.
\textbf{\paragraph{Changes in Consumer Segments and Advertising Strategy}}
The fact that the purchasing segment changes as the PLC advances (Innovators $\to$ Laggards) necessitates a change in marketing communication strategy. The lecture's suggestion regarding the evolution of PC advertising (a detailed listing of product specifications in the early 1990s $\to$ image-based commercials featuring celebrities today) is a prime example. 'Informational advertising' that appeals to technical superiority is effective for the well-informed segment (up to the Early Majority) during the Introduction and Growth stages. However, 'emotional brand-image advertising' that is easier for the segment (Late Majority) generally considered to have lower information processing ability becomes more effective in the Maturity stage.
\textbf{\paragraph{Strategic Implications of the Experience Curve}}
The \textbf{Experience Curve} (the phenomenon where unit costs decrease by a constant rate as cumulative production doubles) is the theoretical pillar of market share expansion strategies in the PLC's Introduction and Growth stages. By achieving mass production early and descending the cost curve faster than competitors, a firm can establish an overwhelming cost advantage. This translates into the message that 'a larger market share enables lower prices and higher quality (or equivalent quality)' and is used to justify the Penetration Strategy.
\textbf{\subsubsection{AI Supplement: Extension of Key Issues}}
The basic framework of the PLC and strategies for each stage were detailed in this lecture. The following supplements provide important cautions and contemporary issues for applying the PLC in practice.
\begin{itemize}
	\item \textbf{The Danger of the Self-Fulfilling Prophecy}:
	      While the PLC is a 'forecasting model,' a management team over-relying on it as a 'normative guide' can fall into a dangerous trap. For example, the moment a product is 'diagnosed' as entering the Maturity stage, management may cut R\&D and marketing investment, thereby '\textbf{self-hastening}' the transition to the Decline stage. The PLC is not fate; its shape and duration can be actively altered by a firm's strategic actions (innovation and re-positioning).
	\item \textbf{Transformation of the PLC in the Digital Age}:
	      Although the lecture mentioned examples like video games skipping the Introduction stage, the traditional PLC is increasingly inapplicable, especially for modern software, SaaS (Software as a Service), and platform businesses.
	      \begin{itemize}
		      \item \textbf{Network Effects}: Products whose value increases with the number of users (e.g., social media) often exhibit explosive growth (hyper-growth) after breaking through the Introduction stage, showing a steeper incline than the traditional S-shaped curve.
		      \item \textbf{Continuous Updates}: Unlike physical products, software and services attempt to avoid 'Maturity' or 'Decline' by frequent updates, aiming to sustain a perpetual Growth stage. The PLC may occur for each 'version' instead of the 'product,' or an entirely new curve may be drawn.
	      \end{itemize}
	\item \textbf{Product Decline vs. Customer Needs Survival}:
	      The PLC is ultimately the lifecycle of a 'product category.' However, the 'fundamental customer needs' that the product satisfied often persist even after the product declines. For instance, the need to 'listen to music' has seen the product (technology) satisfying it—records $\to$ CDs $\to$ digital downloads $\to$ streaming—go through repeated cycles of decline and regeneration. Even for products judged to be in the Decline stage, redefining the underlying customer need can create a new market opportunity.
\end{itemize}
\subsection{Conclusion}
The \textbf{Product Life Cycle (PLC)} is a powerful framework that captures the dynamic changes in the market and indicates how a company's marketing strategy (product, price, channel, promotion) should evolve over time. The challenges and strategic focus differ clearly at each stage, from market development in the Introduction stage to market share expansion in the Growth stage, differentiation and efficiency in the Maturity stage, and withdrawal or re-positioning in the Decline stage.
However, as detailed in this lecture and the 'AI Supplement,' the mechanical application of the PLC carries risks. The PLC is less an 'inevitable fate' and more the 'result' of a company's strategic decisions and innovation. Especially in today's increasingly digital market, cases where the traditional S-shaped curve does not apply are growing.
The practical lesson for MBA students is to use the PLC's stage characteristics as a '\textbf{diagnostic tool}' to objectively analyze their current position but not to be constrained by its 'curse.' The shape of the PLC (especially the Maturity and Decline stages) can be actively modified and extended through strategy. Continuously redefining customer needs and exploring innovation possibilities is the key to sustained growth.
\subsection{List of Key Terms}
Carpenter, Nakamoto, Golder, Tellis, Kotler, Shoemaker, Shoaf
\vspace{\baselineskip}
Product Life Cycle (PLC), Introduction Stage, Growth Stage, Maturity Stage, Decline Stage, Dominant Design, Vertical Differentiation, Horizontal Differentiation, Experience Curve, First Mover Advantage, Innovators, Early Adopters, Early Majority, Late Majority, Laggards, Market Skimming Strategy, Penetration Strategy, Relative Advantage, Compatibility, Simplicity, Observability (Communicability), Brand Loyalty, Market Segment, Private Brand (PB), Exit Barriers, Re-positioning
\subsection{Comprehension Quiz}
\begin{enumerate}
	\item Which stage of the Product Life Cycle is characterized by a rapid expansion of sales and the most active entry of competitors?
	\item Which stage is often characterized by negative profits (a loss) due to large initial investments immediately following the product's market launch?
	\item Which stage of the Product Life Cycle is marked by intense competition, sales growth slowing to a plateau (saturation), and demand primarily centered on replacement purchases?
	\item In Innovator Theory, what is the term for the consumer segment that adopts the new product at the earliest phase, showing high-risk preference?
	\item In Innovator Theory, which segment is considered an opinion leader in the local community and acts as a 'bridge' for diffusion to the subsequent majority?
	\item What is the phenomenon, or the design itself, called when diverse technologies and designs proliferate in the early industry, but eventually converge into a specific design that becomes the market standard?
	\item What is the specific term for differentiation aimed at improving the objective superiority of a product, such as performance or quality?
	\item What is the phenomenon called when the cost per unit of a product decreases by a constant rate as the cumulative production volume increases?
	\item What is the pricing strategy for the Introduction stage that sets a high price to quickly recover investment costs from wealthy or innovator segments who are less price-sensitive?
	\item What is the pricing strategy for the Introduction stage that sets a low price to rapidly gain market share and build a barrier to competitor entry?
	\item What is the fundamental concept that forms the basis of a strategy in the Maturity stage, responding to consumer 'tastes' (horizontal attributes) and deploying a diverse product lineup?
	\item Which stage is characterized by market shrinkage due to technological innovation, such as the replacement of CRT TVs by LCD TVs?
	\item What is the concept that indicates the degree to which a new product or service can be used in conjunction with existing products or systems, acting as a factor promoting adoption?
	\item What is the collective term for factors that make it difficult to withdraw from a specific business in the Decline stage (e.g., relationships with distributors, un-disposable equipment)?
	\item What is the strategy of finding new uses or target markets for a product in the Decline stage to revitalize it?
\end{enumerate}
\subsubsection*{Answer Key}
1. Growth Stage, 2. Introduction Stage, 3. Maturity Stage, 4. Innovators, 5. Early Adopters, 6. Dominant Design, 7. Vertical Differentiation, 8. Experience Curve, 9. Market Skimming Strategy, 10. Penetration Strategy, 11. Horizontal Differentiation (or Market Segmentation), 12. Decline Stage, 13. Compatibility, 14. Exit Barriers, 15. Re-positioning
\section{Marketing in the Introduction and Growth Stages 1}
\subsection{Introduction}
The objective of this lecture is to learn about the \textbf{Product Life Cycle (PLC)}, one of the fundamental models in marketing. Given that markets are constantly changing, understanding the typical pattern a product follows from its launch to its eventual exit is essential for a company to predict changes and devise appropriate strategies. This lecture provides an overview of the four stages of the PLC—\textbf{Introduction}, \textbf{Growth}, \textbf{Maturity}, and \textbf{Decline}—and the characteristics of the market environment and marketing activities that companies should pursue in each.
\subsection{Key Concepts and Issues}
The \textbf{Product Life Cycle (PLC)} is a model that illustrates the trends in sales and profits over time, from a product's market entry until its final withdrawal. It is often likened to the biological process of birth, growth, maturity, and decline.
The PLC typically consists of the following four stages:
\begin{description}
	\item[\textbf{Introduction}] The stage where the product is first launched into the market. Sales growth is slow, and \textbf{substantial marketing investment} (especially promotional costs) is necessary to raise product awareness, often resulting in negative profits.
	\item[\textbf{Growth}] The stage where the product gains market acceptance, and sales rapidly expand. Word-of-mouth and marketing efforts bear fruit, and the number of consumers increases. The entry of \textbf{new firms (competitors)} also begins to increase during this period. Profits are often maximized at this stage.
	\item[\textbf{Maturity}] The stage where sales growth slows and reaches a stable or saturated state. Most market participants have already purchased and used the product at least once, with the focus shifting to \textbf{repurchases} and brand switching. \textbf{Competition is the most intense}, and profits begin to decrease due to ongoing price wars and investment in differentiation.
	\item[\textbf{Decline}] The stage where market demand decreases, and sales and profits continually decline. The main causes are the emergence of substitute technologies or changes in consumer preferences. Firms need to consider \textbf{withdrawal} or \textbf{harvesting}.
\end{description}
A critical implication of the PLC model is that since the market environment differs across stages, firms must \textbf{dynamically adapt their marketing strategies to align with the characteristics of each stage}.
\subsection{Application and Case Analysis}
The lecture cited several examples to understand market changes in each PLC stage.
\subsubsection{Dominant Design (Bicycles)}
The pattern of technological innovation was cited as a factor driving market change. In the Introduction stage of a product, various designs and features are proposed by different companies (e.g., the diverse wheel size ratios in early bicycles). However, as innovation is repeated, a phenomenon occurs where the market (consumers and producers) converges on a specific design. This is called the \textbf{Dominant Design}.
Once the Dominant Design is established (e.g., the form of the modern standard bicycle), the focus of competition shifts from the product's fundamental design to the \textbf{efficiency of production processes} (cost reduction) and more detailed \textbf{Horizontal Differentiation} (e.g., color, design, specialized uses). This phenomenon is often observed during the transition from the Growth to the Maturity stage.
\subsubsection{Convergence on Core Features (Mobile Phones, Home Appliances)}
Changes on the consumer side also drive the market. While consumer knowledge is limited in the Introduction stage, it accumulates as the market matures. As a result, out of products equipped with diverse features (e.g., early multi-function mobile phones), the \textbf{'core features actually utilized'} by consumers (e.g., email, camera, internet speed) become clear. Firms grasp this trend from sales data and begin to focus on developing products that enhance these core features, leading to a convergence of product features.
\subsubsection{Changes in Advertising Strategy (Personal Computers)}
As the PLC progresses, the target consumer segment also changes. For example, in the PC market of the 1990s (Growth stage), advertisements primarily featured detailed explanations of product features and specifications. This was because the main consumer segment at the time possessed a certain level of product knowledge and had high information processing capability.
However, as the market entered the Maturity stage and penetration became very high, the main target shifted to \textbf{conservative} consumer segments with \textbf{not necessarily high information processing capability} (e.g., the elderly, those with little interest in IT). Consequently, modern advertising has shifted its focus from detailed feature explanations to methods that appeal to emotion (or foster a sense of security), such as image-based commercials featuring celebrities.
\subsection{Deeper Context and Lessons}
The core lesson of this lecture is not to memorize the PLC curve itself but to understand the 'market change factors' that generate the curve.
\textbf{\paragraph{Factors Driving Market Change (Seller-Side)}}
As the PLC progresses, the \textbf{industry structure} changes. The Introduction stage is characterized by a \textbf{monopoly} by the developing firm, but the Growth stage sees the entry of \textbf{numerous competitors} attracted by the market's appeal. In the Maturity stage, competition leads to a \textbf{polarization} into a few large firms with a dominant position and smaller firms targeting specific niche markets. In the Decline stage, many firms withdraw from the market.
In terms of technology, \textbf{Vertical Differentiation}, which enhances product performance and features, is central during the Introduction and Growth stages. After the establishment of a Dominant Design, however, the axis of competition shifts to the efficiency of production processes and \textbf{Horizontal Differentiation} (design, brand image, etc.) that caters to diverse consumer preferences.
\textbf{\paragraph{Factors Driving Market Change (Buyer-Side): Innovator Theory}}
The progression of the PLC is also a process where the consumer segment adopting the product changes over time. This is explained by the \textbf{Innovator Theory}.
\begin{itemize}
	\item \textbf{Innovators}: The segment that purchases the product first in the Introduction stage. They prioritize novelty and have a very high-risk tolerance. They include enthusiasts and hobbyists.
	\item \textbf{Early Adopters}: The segment that adopts in the early Growth stage. They are trend-sensitive and act as \textbf{Opinion Leaders}, influencing other consumers.
	\item \textbf{Early Majority}: The segment that adopts during the middle of the Growth stage. They are relatively cautious, referencing the trends of the Early Adopters before deciding to adopt. They are the first large segment that makes up the majority of the market.
	\item \textbf{Late Majority}: The segment that adopts in the Maturity stage. They are conservative, only adopting after confirming that the majority of people around them have adopted (often due to social pressure).
	\item \textbf{Laggards}: The most conservative segment, adopting only as the market enters the Decline stage. They value tradition and dislike change.
\end{itemize}
As pointed out in the lecture, the segments purchasing in the Introduction and Growth stages (Innovators, Early Adopters) have a high-risk tolerance and are proactive in gathering product knowledge. Conversely, the segments purchasing in the Maturity and Decline stages (Late Majority, Laggards) tend to be \textbf{conservative}, strongly risk-averse, and do not prefer complex information processing. Firms must recognize what stage of the PLC their product is in and \textbf{which segment constitutes their current main customers}, adapting their marketing strategy (especially advertising and communication) accordingly.
\textbf{\subsubsection{AI Supplement: Extension of Key Issues}}
This note focused on the overview of the PLC and the factors driving market change. As part of MBA learning, the following two issues are supplemented.
\textbf{1. Typical Marketing Mix (4Ps) Strategies at Each Stage}
It was stated that strategies need to change at each PLC stage, but what specifically should be changed? Let's organize this from the perspective of the Marketing Mix (4Ps: Product, Price, Place, Promotion).
\begin{itemize}
	\item \textbf{Introduction Stage}:
	      \begin{itemize}
		      \item Product: Basic product (avoiding unnecessary features)
		      \item Price: High price for initial investment recovery (Skimming Strategy) or low price for market share acquisition (Penetration Strategy)
		      \item Place: Limited, or focused on cooperative channels
		      \item Promotion: Intensive advertising and sales promotion to increase product awareness and encourage trial
	      \end{itemize}
	\item \textbf{Growth Stage}:
	      \begin{itemize}
		      \item Product: Quality improvement, feature addition, product line (variation) expansion
		      \item Price: Maintain or slightly lower price due to competitor entry
		      \item Place: Rapid expansion of distribution channels
		      \item Promotion: Shift emphasis from awareness to building \textbf{brand preference}
	      \end{itemize}
	\item \textbf{Maturity Stage}:
	      \begin{itemize}
		      \item Product: \textbf{Differentiation} of brands and models, product improvement
		      \item Price: Price competition intensifies due to increased competition. Maintain or lower price
		      \item Place: Build and maintain maximum distribution network
		      \item Promotion: Maintain brand loyalty (reminder advertising), sales promotion to maintain share
	      \end{itemize}
	\item \textbf{Decline Stage}:
	      \begin{itemize}
		      \item Product: Streamlining or reduction of unprofitable products (simplification of the line)
		      \item Price: Price maintenance (for niche segments) or inventory disposal
		      \item Place: Limited to profitable channels
		      \item Promotion: Reduced to a minimum
	      \end{itemize}
\end{itemize}
\textbf{2. Limitations and Problems of the PLC Model}
While the lecture mentioned touching upon 'model problems' at the beginning, specific reference was omitted in the body text. I will supplement this. While the PLC is a useful concept, caution is needed in its use.
\begin{itemize}
	\item \textbf{Limitations as a Forecasting Tool}: The PLC is often suitable for describing the 'result' a product has followed but is unreliable as a tool for accurately predicting future sales or the 'timing' and 'length' of each stage.
	\item \textbf{Danger of Self-Fulfilling Prophecy}: This is the most critical point to note. If management 'diagnoses' their product as having entered the Decline stage based on the PLC, they may withdraw investment (cutting advertising costs, stopping R\&D). This \textbf{strategic decision itself can cause} the product's sales to genuinely decline, leading to its self-fulfilling prophecy.
	\item \textbf{Not Applicable to All Products}: There are cases where the PLC is not applicable (e.g., fashion trends, long-term stable basic products) depending on the product category.
\end{itemize}
\subsection{Conclusion}
The Product Life Cycle (PLC) model is a framework that captures the pattern of a product's sales and profits changing over time across four stages (Introduction, Growth, Maturity, Decline).
The crucial lesson learned in this lecture is that this PLC curve is generated by the \textbf{dynamic interaction} of the seller-side (industry structure, technological competition) and the buyer-side (accumulation of consumer knowledge, change in adopter segments).
In practice, managers should not passively accept the PLC as an 'inevitable fate.' As pointed out in the AI supplement, the PLC is also the result of a firm's strategic actions. Therefore, the key to maximizing a product's life and securing corporate profit lies in objectively analyzing which stage their product is currently in and \textbf{proactively changing and executing the marketing mix (4Ps)} tailored to the consumer segment and competitive environment of that stage.
\subsection{List of Key Terms}
(People mentioned in this lecture)
None
\vspace{\baselineskip}
(Universal concepts mentioned in this lecture)
Product Life Cycle (PLC), Introduction Stage, Growth Stage, Maturity Stage, Decline Stage, Marketing Mix (4Ps), Industry Structure, Dominant Design, Vertical Differentiation, Horizontal Differentiation, Market Segmentation, Innovator Theory, Innovators, Early Adopters, Early Majority, Late Majority, Laggards, Risk Perception
\subsection{Comprehension Quiz}
\begin{enumerate}
	\item What is the most appropriate description of the market situation characteristic of the PLC's 'Introduction Stage'?
	\item What is the highest priority for a firm's main marketing objective in the 'Growth Stage'?
	\item Briefly explain the main reason why price and differentiation competition intensifies in the 'Maturity Stage.'
	\item What strategy, other than immediate withdrawal, was suggested in the lecture (or is generally known) for the 'Decline Stage'?
	\item What trend does the typical S-shaped curve of the Product Life Cycle primarily show?
	\item What is the phenomenon where, after various product designs are attempted, the market converges on a specific design?
	\item After a Dominant Design is established, to what does the axis of competition between firms tend to shift from fundamental product innovation? (Name two points)
	\item What is the term for differentiation based on objective superiority, such as a product's quality or performance?
	\item Which stage is primarily characterized by the shift in importance to differentiation based on consumers' subjective preferences, such as design or brand image, as performance differences narrow?
	\item In Innovator Theory, what is the term for the segment that prioritizes novelty the most and adopts the product regardless of risk?
	\item Which consumer segment functions as an opinion leader and is considered the key to crossing the 'chasm' for overall market penetration?
	\item Name the two consumer segments that account for the majority of the market but are relatively cautious in adopting, referring to the trends of others.
	\item What is the main motivation for 'Laggards' in Innovator Theory to finally adopt a product?
	\item What is the biggest financial risk a firm faces in the Introduction stage?
	\item Briefly explain what danger the 'Self-Fulfilling Prophecy,' one of the limitations of the PLC model, refers to.
\end{enumerate}
\subsubsection*{Answer Key}
1. Low sales and awareness, a deficit due to front-loaded investment, limited competitors, 2. Maximization of market share, 3. Because market saturation and slowed growth lead to competition for taking existing market share, 4. Harvesting Strategy, focus on a specific niche market, 5. Sales revenue, 6. Dominant Design, 7. Production process efficiency (cost reduction), Horizontal Differentiation, 8. Vertical Differentiation, 9. Maturity Stage, 10. Innovators, 11. Early Adopters, 12. Early Majority, Late Majority, 13. Lack of traditional alternatives or social pressure from the surrounding community, 14. The risk of the product not being accepted by the market, leading to withdrawal without recovering large initial investments (R\&D, marketing costs), 15. The danger that the strategic decision itself by management to reduce investment because they judged 'our product is in the Decline stage' causes the product's sales to genuinely decline.
\section{Marketing in the Introduction and Growth Stages 2}
\subsection{Introduction}
This lecture focuses on the first two stages of the Product Life Cycle (PLC): the \textbf{Introduction Stage} and the \textbf{Growth Stage}. The challenges faced and the target consumer segments are fundamentally different between the highly uncertain period immediately after a product is launched and the period of rapid market expansion. The purpose of this note is to understand the market characteristics of these two stages and to clarify the differences in the pricing, product, and promotion strategies that companies should adopt.
\subsection{Key Concepts and Issues}
The \textbf{Product Life Cycle (PLC)} is a model that tracks the trends in product sales and profits over time. This lecture specifically addresses the following two stages:
\begin{description}
	\item[\textbf{Introduction Stage}] The market size is small, and sales growth is slow. Consumers are limited to \textbf{Innovators} and \textbf{Early Adopters}. Market uncertainty is high, and there are few competitors. Profits are often negative due to substantial initial investment not being matched by sales.
	\item[\textbf{Growth Stage}] The market accepts the product, and sales rapidly expand. The \textbf{Early Majority} joins as a new consumer segment. The market's attractiveness increases, leading to the \textbf{entry of numerous competitors}.
\end{description}
During this transition, firms face a choice between two major pricing strategies.
\begin{itemize}
	\item \textbf{Skimming Strategy (Price Skimming)}: Setting a high price initially to secure high profits from segments like Innovators and Early Adopters, who have low price elasticity.
	\item \textbf{Penetration Strategy (Market Penetration)}: Setting a low price initially (sometimes below cost) to achieve rapid market share acquisition and cost reduction through the \textbf{Experience Curve}.
\end{itemize}
\subsection{Application and Case Analysis}
\subsubsection{The Myth of First-Mover Advantage}
The first-mover entering the market in the Introduction stage is said to gain the \textbf{First-Mover Advantage}, which includes the \textbf{Experience Curve} (cost reduction through mastering the production process) and the status of being the 'synonym' for the product category (e.g., Nissin's Cup Noodles).
However, this advantage is not absolute. Research by \textbf{Golder and Tellis (1993)} shows that in many markets, first-movers have been overtaken by late-movers (e.g., \textbf{Lotus 1-2-3} in the spreadsheet market was defeated by the late-comer Microsoft Excel). For a first-mover to maintain its advantage, it must be more than merely 'fast'; it needs to be the 'product pioneer' holding a patent or the 'market pioneer' establishing the distribution channels.
\subsubsection{Practical Example of Pricing Strategy (PlayStation 3)}
The PlayStation 3, launched in 2006, is a classic example of the \textbf{Skimming Strategy}. Its initial price was high (around 60,000 yen), as it targeted Innovators and Early Adopters (avid gamers) who possessed technological preference and low price elasticity. As the market matured, the price was lowered to around 30,000 yen, penetrating a broader consumer base.
\subsubsection{Promotion in the Market Expansion Phase (Hybrid Cars)}
Promotion during the transition from the Introduction to the Growth stage prioritizes the \textbf{demand expansion of the product category itself} over individual brand differentiation. For example, in the early days of hybrid cars (eco-cars), advertising focused on \textbf{empathy for the problem} and the \textbf{category's enlightenment} ('how environmentally friendly hybrid cars are') rather than advocating the functional superiority of 'Toyota's Prius' individually.
\subsection{Deeper Context and Lessons}
\textbf{\paragraph{Qualitative Differences in Consumer Segments: Innovators vs. Early Adopters}}
\textbf{Innovators} and \textbf{Early Adopters}, the main customers in the Introduction stage, are often confused, but their motivations differ. Innovators purchase out of pure personal interest or curiosity, possessing technical knowledge. Early Adopters, on the other hand, are \textbf{Opinion Leaders} within their community and value the act of disseminating information to others. Therefore, engaging Early Adopters is critically important for the transition to the Growth stage (i.e., generating word-of-mouth).
\textbf{\paragraph{Late-Entrant's Logic (Follower Advantage)}}
There is a rational reason for firms to enter in the Growth stage rather than the Introduction stage.
\begin{enumerate}
	\item \textbf{Reduced Uncertainty}: Investment risk decreases because market growth potential and the technical standard (\textbf{Dominant Design}) become clearer compared to the Introduction stage.
	\item \textbf{Easier Investment Scale Calculation}: Analyzing market fluctuations and the first-mover's trends allows for more accurate calculation of production scale and capital investment.
	\item \textbf{Marketing Efficiency}: Late-movers can 'free-ride' on the efforts of first-movers, who have already spent substantial costs educating the market (consumers). They can analyze consumer reactions and devise more efficient marketing plans (e.g., targeting a segment overlooked by the first-mover).
\end{enumerate}
\textbf{\paragraph{The Market Share Paradox: Brand Switching}}
When a late-comer (Brand B) enters the market of an existing monopoly (Brand A), the first-mover's share naturally declines (even if its brand power is strong) due to consumers' \textbf{Variety-Seeking} bias.
As explained in the lecture, even if the switching rate from Brand A to B is low (e.g., p\%) and the switching rate from B to A is high (e.g., 30\%), the absolute number of people leaving Brand A is large because Brand A's base is 100\%. This necessitates first-movers to adopt measures to enhance \textbf{Brand Loyalty} to curb the decline in share.
\textbf{\subsubsection{AI Supplement: Extension of Key Issues}}
This note focused on the market characteristics and corporate strategies of the Introduction and Growth stages. As part of MBA learning, I will supplement the concept of the '\textbf{Chasm},' which is the biggest obstacle in this transition.
\textbf{Chasm Theory (Crossing the Chasm)}:
In Innovator Theory (proposed by Geoffrey Moore), a deep, disconnected gap (the Chasm) exists between the \textbf{Early Adopters} and the \textbf{Early Majority}.
\begin{itemize}
	\item \textbf{Cause of the Gap}: As mentioned in the lecture, Early Adopters (opinion leaders, visionaries) purchase motivated by 'newness, innovation.' However, the Early Majority (pragmatists) prioritize 'proven results, a sense of security, case studies, and surrounding reputation,' and are risk-averse.
	\item \textbf{Strategic Implication}: Success with Early Adopters does not guarantee that the same marketing approach will work with the Early Majority. When transitioning from the Introduction to the Growth stage, firms must fundamentally shift their marketing message and strategy from appealing to 'newness' to appealing to '\textbf{practicality},' '\textbf{reliability},' and '\textbf{proven results}.' Products (especially high-tech ones) that fail this shift cannot cross the Chasm and disappear from the market (i.e., fail to reach the Growth stage).
\end{itemize}
\subsection{Conclusion}
This lecture note analyzed the market environment, consumer characteristics, and corporate response strategies (especially pricing and competitive strategies) for the PLC's Introduction and Growth stages.
The Introduction stage is characterized by high market uncertainty, requiring \textbf{category enlightenment} and \textbf{risk reduction} for the specialized segments of Innovators and Early Adopters. The Growth stage, on the other hand, is defined by clear market appeal and the entry of numerous competitors. The challenge here is to \textbf{establish market share} by effectively engaging the pragmatic \textbf{Early Majority}.
A practical lesson is that firms must not rest on the 'First-Mover Advantage.' As Golder and Tellis pointed out, late-movers can free-ride on the first-mover's market education costs and capture the market with more efficient strategies. First-movers must not neglect the construction of a technical entry barrier through \textbf{Vertical Differentiation} and the enhancement of \textbf{Brand Loyalty} to counter the competitors (especially large firms with superior technology) entering in the Growth stage.
\subsection{List of Key Terms}
(People mentioned in this lecture)
Carpenter, Nakamoto, Golder, Tellis
\vspace{\baselineskip}
(Universal concepts mentioned in this lecture)
Product Life Cycle (PLC), Introduction Stage, Growth Stage, Innovators, Early Adopters, Early Majority, Opinion Leader, First-Mover Advantage, Experience Curve, Skimming Strategy (High-Price Strategy), Penetration Strategy (Market Penetration Strategy), Price Elasticity, Relative Advantage, Compatibility, Simplicity, Observability, Dominant Design, Follower Advantage, Brand Loyalty, Vertical Differentiation, Chasm
\subsection{Comprehension Quiz}
\begin{enumerate}
	\item Name two main reasons why a firm's profit tends to be negative in the 'Introduction Stage' of the Product Life Cycle.
	\item Distinguish between 'Innovators' and 'Early Adopters' in Innovator Theory from the perspective of their purchasing motivation.
	\item What is the phenomenon called when the cost per unit of a product decreases as the cumulative production volume increases?
	\item What is the pricing strategy that sets a high initial price to target high profits during the new product introduction?
	\item The above strategy (Q4) is effective when the targeted initial consumer segment has what characteristic?
	\item What is the pricing strategy that intentionally sets a low price during the new product introduction to achieve rapid market share acquisition and build barriers to competitor entry?
	\item Name two main reasons, other than market appeal (profit opportunity), why numerous competitors enter the market in the 'Growth Stage'.
	\item As pointed out by Golder and Tellis (1993), why can the market 'first-mover' not always maintain its advantage? Explain briefly from the perspective of the late-mover.
	\item Name three of the five characteristics cited by the diffusion of innovation theory as factors that accelerate consumer adoption of a new product or service.
	\item What is the biggest threat the first-mover firm faces in the 'Growth Stage'?
	\item What is the term for a differentiation strategy based on objective indicators like product performance or features?
	\item What psychological asset of consumers should the first-mover firm strengthen to counter the fierce pursuit of late-comers and maintain market share in the 'Growth Stage'?
	\item While the promotion strategy in the Introduction stage often focuses on 'product category enlightenment,' what should the promotion strategy focus on in the Growth stage?
	\item What is the 'Chasm,' proposed by Geoffrey Moore, and between which two consumer segments does this discontinuity exist in Innovator Theory?
	\item Why is overcoming the 'Chasm' difficult? Explain briefly by focusing on the difference in values between the two segments.
\end{enumerate}
\subsubsection*{Answer Key}
1. Occurrence of substantial initial investment (R\&D costs, capital expenditure), large marketing costs to increase product awareness, 2. Innovators are motivated by technical interest or novelty itself, while Early Adopters are motivated by exercising influence (opinion leadership) over others or realizing a vision, 3. Experience Curve, 4. Skimming Strategy (Price Skimming), 5. Low price elasticity (demand does not decline even with a high price), 6. Penetration Strategy (Market Penetration), 7. Reduced market uncertainty, establishment of a technical standard (Dominant Design), 8. Because late-movers can learn from the first-mover's failures (and successes) and enter the market with a superior product or a more efficient marketing strategy (Follower Advantage), 9. Relative Advantage, Compatibility, Simplicity, Trialability, Observability (any three), 10. Rapid decline in market share due to the entry of competitors, 11. Vertical Differentiation, 12. Brand Loyalty, 13. Building the superiority of their own brand (brand preference), 14. Between Early Adopters and Early Majority, 15. Because Early Adopters value 'innovation,' while the Early Majority values 'practicality, security, and proven results,' meaning their values are fundamentally different.
\section{Marketing in the Maturity and Decline Stages 1}
\subsection{Introduction}
This lecture focuses on the latter two stages of the Product Life Cycle (PLC): the \textbf{Maturity Stage} and the \textbf{Decline Stage}. The challenges faced and the strategies adopted are fundamentally different between the stage where market growth slows and competition intensifies and the stage where the market shrinks and withdrawal is a consideration. The purpose of this note is to understand the market characteristics of these two stages (e.g., horizontal differentiation, price competition, exit barriers) and to analyze the strategic options available to firms for securing sustained profit or minimizing losses.
\subsection{Key Concepts and Issues}
The latter half of the \textbf{Product Life Cycle (PLC)} is understood as two distinct phases: 'saturation' and 'contraction' of the market.
\subsubsection{Maturity Stage}
The stage where sales growth slows or halts, and total sales remain high but plateau (or slightly increase/decrease).
\begin{itemize}
	\item \textbf{Consumers}: The main purchasing segments become the \textbf{Late Majority} and \textbf{Laggards}. \textbf{Repeat purchases} by those who bought in the Growth stage also sustain sales.
	\item \textbf{Market Environment}:
	      \begin{itemize}
		      \item \textbf{Overcapacity}: Many firms, having over-invested in facilities based on Growth stage forecasts, easily fall into oversupply.
		      \item \textbf{Intensified Competition}: \textbf{Price competition} intensifies to absorb excess production capacity and to fight for market share.
		      \item \textbf{Rise of Private Brands}: Retailers' \textbf{Private Brands (PB)} emerge as strong competitors, pressuring the profits of national brands.
	      \end{itemize}
	\item \textbf{Market Structure}: Firms with a competitive disadvantage withdraw, and an \textbf{oligopoly market} is formed by a few large firms with strong brand power, cost competitiveness, or technology. Surrounding them are \textbf{niche firms} that cater to specific needs.
\end{itemize}
\subsubsection{Decline Stage}
The stage where sales and profits for the entire product class decrease rapidly and continually.
\begin{itemize}
	\item \textbf{Main Causes}:
	      \begin{itemize}
		      \item \textbf{Technological Substitution}: Superior new technologies or substitutes emerge (e.g., CRT TVs to LCD TVs).
		      \item \textbf{Lifestyle Changes}: Consumer preferences or habits change, causing demand for the product category itself to be lost (e.g., the change in demand for Morishita Jintan).
	      \end{itemize}
	\item \textbf{Corporate Challenges}: Product maintenance becomes a management burden due to declining profitability from reduced sales, rising unit costs from decreased production volume, and increased burden from inventory and price adjustments.
	\item \textbf{Main Strategic Options}:
	      \begin{itemize}
		      \item \textbf{Withdrawal}: Completely exiting the market.
		      \item \textbf{Focus}: Concentrating on a specific niche segment with high Brand Loyalty.
		      \item \textbf{Re-positioning}: Giving the product a new meaning or use to regenerate in a different market.
	      \end{itemize}
\end{itemize}
\subsection{Application and Case Analysis}
\subsubsection{Maturity Stage Extension Strategy (Pocky)}
The strategy in the Maturity stage is to strengthen \textbf{Horizontal Differentiation} and \textbf{Market Segmentation}. This is because there is little room for technological innovation (Vertical Differentiation), but there is a need to respond to diversifying consumer preferences (especially the Late Majority, who do not necessarily possess high information processing ability).
\begin{itemize}
	\item \textbf{Case (Pocky)}: Ezaki Glico's Pocky is a classic mature product, but new lines like 'Deco Pocky' (for decoration) and 'Premium Pocky' (luxury-oriented, department store exclusive) were launched. This is a strategy to intentionally extend the product's Maturity stage by moving beyond its existing use as 'a snack' and developing new niche markets like 'handmade hobby' or 'luxury gift.'
\end{itemize}
\subsubsection{Causes of Decline Stage (CRT TVs, Morishita Jintan)}
\begin{itemize}
	\item \textbf{Case (CRT TVs)}: The demand for watching TV itself persisted, but the product class of CRT TVs rapidly declined due to the emergence of substitute technologies like LCD and plasma TVs. This is a classic example of \textbf{Technological Substitution}.
	\item \textbf{Case (Morishita Jintan)}: (Based on the lecture's reference) The demand as a breath freshener decreased due to the spread of gum and other tablet products and changes in lifestyle, causing the market to shrink.
\end{itemize}
\subsubsection{Decline Stage Re-positioning Strategy (Umaibo)}
Even products in the Decline stage (or late Maturity) can be extended if a new value is assigned to them.
\begin{itemize}
	\item \textbf{Case (Umaibo)}: The main target is children, but in recent years, an increasing number of office workers (former children) seeking 'nostalgia' purchase the snack at convenience stores. This is an example of \textbf{Re-positioning} to a new target segment by appealing to \textbf{emotional value (nostalgia)} rather than the product's functional value (taste). However, such markets tend to be niche.
\end{itemize}
\subsection{Deeper Context and Lessons}
\textbf{\paragraph{Change in the Quality of Competition: From Vertical to Horizontal}}
The key change in the Maturity stage is the shift in the axis of competition from 'performance (\textbf{Vertical Differentiation})' to 'preference (\textbf{Horizontal Differentiation}).' By the Growth stage, technological superiority has largely peaked, and the quality of competitors' products has converged (following the establishment of a Dominant Design). At that point, a firm's continued investment shifts from new technology development to product line expansion (Market Segmentation) to meet diverse (sometimes trivial) consumer preference differences.
\textbf{\paragraph{The 'Invisible Cost' in the Decline Stage}}
The lecture pointed to the increased burden on sales representatives as an 'invisible cost' in the Decline stage. With sales declining and profit margins compressed, firms are forced into frequent price adjustments (discounts) and inventory adjustments (disposal/scrap). This management cost and operational fatigue create internal pressure to hasten the product's withdrawal, often exceeding the explicit red ink on the financial statements.
\textbf{\paragraph{Channel Relationships as an Exit Barrier}}
\textbf{Withdrawal} from a product in the Decline stage may not be a simple management decision. The biggest factor hindering it is \textbf{Exit Barriers}.
Particularly crucial are relationships with distribution channels (retailers). Even if a product is unprofitable for the company, unilaterally stopping supply can severely damage the relationship if that product is a pillar of the retailer's sales (especially small stores). If the company also supplies other main products to that retailer, a strained relationship could escalate into a halt of all transactions. Thus, channel relationships often act as a 'hostage,' making withdrawal from unprofitable products difficult.
\textbf{\subsubsection{AI Supplement: Extension of Key Issues}}
The strategies for the Decline stage mentioned in this note were 'withdrawal,' 'niche focus,' and 're-positioning.' However, one crucial alternative is missing from MBA strategy theory: the \textbf{Harvesting Strategy}.
\textbf{Harvesting Strategy}:
Harvesting is a strategy in the Decline stage that intentionally \textbf{minimizes additional investment} (marketing costs, R\&D, etc.) to maximize short-term cash flow.
\begin{itemize}
	\item \textbf{Objective}: Instead of immediate withdrawal, the goal is to secure as much profit as possible ('harvest') from the remaining customer segment with \textbf{high Brand Loyalty} until the market completely disappears.
	\item \textbf{Relevance to the Lecture}: The strategy mentioned in the lecture—'focusing only on users with high Brand Loyalty'—is part of this Harvesting Strategy or its prerequisite.
	\item \textbf{Application Conditions}: For this strategy to succeed, it is desirable to have a certain number of loyal customers who are price-insensitive and do not switch to substitutes, and the market shrinkage speed is relatively gradual.
\end{itemize}
The strategy for the Decline stage is not a simple choice between 'immediate withdrawal' or 'extension (re-positioning).' It requires multi-faceted consideration, including 'Harvesting' and 'Divestment' (selling the business unit to another company).
\subsection{Conclusion}
This lecture examined the market dynamics and corporate responses in the Maturity and Decline stages of the PLC.
The Maturity stage is defined by market saturation and overcapacity, with price competition and horizontal differentiation as key themes. Here, the wisdom to thoroughly implement \textbf{Market Segmentation} and unearth niche demands (e.g., Pocky) is required to maintain high sales and avoid competition.
The Decline stage is not merely 'the end' but 'the time for management,' where the optimal allocation of management resources is demanded. As emphasized in the 'Deeper Context' of this lecture, the existence of \textbf{Exit Barriers (especially channel relationships)} complicates the withdrawal decision. Therefore, instead of simple withdrawal, firms need to execute a strategy of '\textbf{wise contraction}' appropriate to the situation, including regeneration in a niche market through \textbf{Re-positioning} (e.g., Umaibo) or maximizing cash flow through the \textbf{Harvesting Strategy} shown in the AI supplement.
\subsection{List of Key Terms}
(People mentioned in this lecture)
None
\vspace{\baselineskip}
(Universal concepts mentioned in this lecture)
Product Life Cycle (PLC), Maturity Stage, Decline Stage, Late Majority, Laggards, Repeat Purchase, Price Competition, Private Brand (PB), Niche Strategy, Horizontal Differentiation, Vertical Differentiation, Market Segmentation, Technological Substitution, Lifestyle Changes, Exit Barriers, Re-positioning, Harvesting Strategy
\subsection{Comprehension Quiz}
\begin{enumerate}
	\item Name two main factors contributing to the intensification of price competition in the 'Maturity Stage' of the Product Life Cycle.
	\item Which PLC stage is most threatened by 'Private Brands (PB)' planned and developed by major retailers?
	\item What is the typical differentiation strategy adopted by firms in the Maturity stage when differentiation based on core product features becomes difficult?
	\item Briefly describe the characteristics of the 'Late Majority' consumer compared to the 'Early Adopters.'
	\item Name two typical external environmental factors that trigger the PLC's 'Decline Stage.'
	\item What is the collective term for the factors that prevent a firm from withdrawing from an unprofitable business?
	\item What is the specific nature of the barrier represented by 'relationships with distribution channels' as discussed in the lecture?
	\item What is the strategy called that seeks regeneration by fundamentally re-examining the target market or use of a product in the Decline stage and assigning a new meaning to it?
	\item What is the strategy that aims to maximize short-term cash flow in the Decline stage by minimizing additional investment?
	\item What is the characteristic of the customer base that is a prerequisite for the success of the above strategy (Q9)?
	\item What is the strategy of concentrating management resources on a specific segment even though the overall market is declining?
	\item Explain the fundamental difference between 'Vertical Differentiation' and 'Horizontal Differentiation' from the perspective of consumer evaluation criteria.
	\item Briefly state the reason why the market in the Maturity stage polarizes into a few oligopolistic firms and numerous niche firms.
	\item What is the purpose of 'Market Segmentation' conducted by firms to stop the sales decline in the Maturity stage?
	\item What potential risk, other than those appearing on the P\&L statement, was suggested in the lecture for immediately withdrawing a product in the Decline stage?
\end{enumerate}
\subsubsection*{Answer Key}
1. Slow market growth (competing for existing share), overcapacity, 2. Maturity Stage, 3. Horizontal Differentiation, 4. The Late Majority is conservative and risk-averse (prioritizing practicality and security), while Early Adopters are innovative and risk-tolerant (prioritizing vision and novelty), 5. Technological Substitution (emergence of substitutes), changes in consumer lifestyles or preferences, 6. Exit Barriers, 7. Strategic Exit Barriers (synergy with other business units or brands, risk of damaging channel relationships), 8. Re-positioning, 9. Harvesting Strategy, 10. A customer base with low price sensitivity and very high Brand Loyalty, 11. Niche Strategy (or Concentration Strategy), 12. Vertical Differentiation is based on objective superiority like performance or quality, while Horizontal Differentiation is based on subjective preferences like design or taste, 13. Because large firms pursue cost advantages through economies of scale, and niche firms specialize in small segments with specific needs, 14. To meticulously respond to diversifying consumer needs and stimulate new demand (or repeat purchases), 15. The risk of relationship deterioration with distribution channels (retailers) negatively affecting the sales of other current products.
\section{Marketing in the Maturity and Decline Stages 2}
\subsection{Introduction}
We have so far studied the four stages of the Product Life Cycle (PLC): Introduction, Growth, Maturity, and Decline. However, this model is not omnipotent and has many limitations and problems. The purpose of this lecture is to understand the reality that the PLC model does not always take a typical shape and to critically examine the structural problems of the model itself (ambiguity of stage transitions, the problem of aggregation level). Furthermore, we will consider how to utilize this model in practical marketing strategy while recognizing its limitations.
\subsection{Key Concepts and Issues}
This lecture discusses the theoretical limitations of the PLC model and methods for applying it despite those limitations.
\subsubsection{Limitations Regarding the Shape of the PLC Model}
The sales trend of the PLC does not necessarily follow the typical \textbf{S-shaped curve}.
\begin{itemize}
	\item \textbf{Absence of Introduction and Growth Stages}:
	      In product categories undergoing \textbf{commoditization}, where numerous competitors exist upon market entry, cases are seen where products start immediately in the \textbf{Maturity Stage} without passing through the Introduction or Growth stages.
	\item \textbf{Re-growth from Decline (Revival)}:
	      Even products seemingly in the Decline stage can show re-growth-like sales increases when demand is re-stimulated by effective promotional activities or the proposal of new value (re-positioning).
	\item \textbf{Factors of Variation Other Than Time Passage}:
	      Sales fluctuations can be actively triggered by a firm's strategic marketing activities (e.g., large-scale advertising, new channel development) rather than passively by the passage of time, making the simple application of the model difficult.
\end{itemize}
\subsubsection{Limitations Regarding the Operation of the PLC Model}
\begin{itemize}
	\item \textbf{Ambiguity of Stage Transitions}:
	      While the model posits the existence of four stages, there is no clear criterion for 'when' and 'by what standard' a product is judged to have moved from the Introduction to the Growth stage.
	\item \textbf{The Problem of Aggregation Level}:
	      The shape and stage of the PLC can vary completely depending on the \textbf{Aggregation Level} of the analysis subject.
	      \begin{itemize}
		      \item \textbf{Industry Level} (e.g., the entire automobile industry)
		      \item \textbf{Product Category Level} (e.g., luxury cars, hybrid cars)
		      \item \textbf{Brand Level} (e.g., a specific car model)
	      \end{itemize}
	      In analyzing real-world market competition, capturing the PLC at the '\textbf{Product Category}' level, where the competitive environment is easier to define, is considered more effective for strategic planning than at the individual 'Brand' level.
\end{itemize}
\subsection{Application and Case Analysis}
It is crucial to recognize the limitations of the PLC model and then apply it to practice.
\subsubsection{Example of Skipping the Introduction Stage (Video Games)}
The typical Introduction stage is often absent in the video game software market. This is because firms employ a strategy of maximizing consumer expectations through massive pre-launch promotions via SNS and advertising. Consequently, the product records explosive sales (the peak of the Growth stage) the moment it is released, and sales then gradually decline (Maturity/Decline), depicting a shape that skips the Introduction and early Growth stages.
\subsubsection{New Brand Introduction in a Mature Market (Canned Coffee)}
The canned coffee market provides an example of strategically utilizing the PLC's 'aggregation level.'
\begin{itemize}
	\item \textbf{Analysis of the Product Category (Canned Coffee)}:
	      The 'Canned Coffee' product category itself is a classic \textbf{Maturity Stage} market, characterized by numerous firm entries over many years, market oligopolization, and highly informed consumers.
	\item \textbf{Strategy for an Individual Brand (New Product)}:
	      Consider a firm introducing a 'new brand' of canned coffee into this mature market. While the market (category) is in the Maturity stage, that 'brand' is undeniably in the \textbf{Introduction Stage}.
	\item \textbf{Strategic Implication}:
	      Therefore, the firm should not simply conclude that 'since the market is mature, it's a price war.' Instead, it needs to execute a typical Introduction stage strategy tailored to the stage of its brand, such as increasing product awareness, advocating for the uniqueness of its manufacturing method or taste, and \textbf{reducing risk} through sampling and trial events.
\end{itemize}
\subsection{Deeper Context and Lessons}
\textbf{\paragraph{PLC is a 'Diagnostic' Model, Not a 'Predictive' One}}
As emphasized in the lecture, the PLC is not a tool to 'predict the fate of all products.' Rather, it is a framework for '\textbf{diagnosing}' which typical characteristics of the PLC stages match the current market environment (competitive situation, customer characteristics) in which the firm is placed. The value lies precisely in using this diagnosis to derive appropriate strategic options (e.g., awareness enhancement in Introduction, differentiation in Maturity).
\textbf{\paragraph{The Shape of the PLC Can Be Changed by 'Strategy'}}
The suggestion that re-growth (revival) from the Decline stage is possible is crucial. It indicates that the PLC is not an inescapable fate; its shape and duration can be altered by a firm's active strategic actions. Managers should not passively accept the PLC stage but should attempt to control the PLC into a desirable shape through marketing efforts.
\textbf{\subsubsection{AI Supplement: Extension of Key Issues}}
This note highlighted the non-universality of the PLC's shape and the problem of aggregation level as limitations. However, the most critical danger of the PLC in MBA education, the \textbf{'Self-Fulfilling Prophecy,'} was omitted and is supplemented here.
The biggest operational danger of the PLC model lies in management's misuse of it as a 'deterministic prediction.' For instance, let's assume management diagnoses that 'our product has entered the Decline stage based on the PLC.' Consequently, they execute a reasonable (for the Decline stage) strategy of withdrawing resources and reducing R\&D and advertising expenditure for that product.
However, this \textbf{strategic decision to 'withdraw investment' itself becomes the cause} of an artificial sales drop, truly pushing a product that might still have had demand into decline. This is the self-fulfilling prophecy of the PLC. It is essential to recognize that the PLC is merely a tool for situational diagnosis and does not automatically determine strategy.
\subsection{Conclusion}
The Product Life Cycle (PLC) model has many limitations, including that it does not depict an S-shaped curve applicable to all products, the criteria for stage transitions are ambiguous, and its appearance changes depending on the level of aggregation analyzed.
However, as learned in this lecture, the PLC is not an 'omnipotent predictive model' but a '\textbf{powerful diagnostic and explanatory model}.' The practical lesson is to utilize the model as a framework for diagnosing the stage of one's 'product category' and analyzing the competitive environment and customer characteristics while understanding its limitations.
Especially crucial is the perspective, demonstrated by the 'canned coffee' example, to \textbf{separate the 'market (category) stage' from the 'firm's product (brand) stage}.' This allows for flexible strategic planning, such as adopting an Introduction stage strategy even within a mature market. MBA students are required to view the PLC not as a deterministic fate but as something controllable through strategic action.
\subsection{List of Key Terms}
(People mentioned in this lecture)
None
\vspace{\baselineskip}
(Universal concepts mentioned in this lecture)
Product Life Cycle (PLC), S-Shaped Curve, Commoditization, Maturity Stage, Introduction Stage, Growth Stage, Decline Stage, Aggregation Level, Product Category, Brand, Revival (Re-growth), Self-Fulfilling Prophecy
\subsection{Comprehension Quiz}
\begin{enumerate}
	\item Name two examples of reasons why the Product Life Cycle (PLC) model does not necessarily follow the typical 'S-shaped curve.'
	\item What is the phenomenon called when the sales of a product that has entered the Decline stage start to increase again due to the firm's marketing efforts?
	\item Name two fundamental problems (limitations) of the PLC model cited in the lecture regarding its operation.
	\item What is the 'aggregation level' for analyzing the PLC? Name the three levels cited in the lecture.
	\item Why is the PLC at the 'Product Category' level emphasized more for strategic planning than the PLC at the 'Brand' level?
	\item What is the main reason why the PLC of video game software often skips the typical Introduction and early Growth stages?
	\item In a market where the product category is in the 'Maturity Stage,' like the 'canned coffee' market, if a 'new brand' is introduced, what kind of marketing strategy should that brand adopt?
	\item Briefly explain why the PLC model should be viewed as a 'diagnostic' model rather than a 'predictive' model.
	\item Briefly explain what danger the 'Self-Fulfilling Prophecy' of the PLC model refers to.
	\item When a product is launched into a 'commoditized' market, which PLC stage is it most likely to start from?
	\item Is the shape and duration of the PLC fixed fate, or can it be modified by the firm's strategic actions?
	\item Is it easy or difficult to clearly delineate the boundaries of each PLC stage (Introduction, Growth, Maturity, Decline)? And why?
	\item Is the current stage the same for the PLC at the 'Industry' level (e.g., the automobile industry) and the 'Product Category' level (e.g., electric vehicles)?
	\item What is the strategic intent of a new brand conducting sampling and trial events in a mature market?
	\item Is the practical significance of learning the PLC model in accurately predicting the future, or in something else?
\end{enumerate}
\subsubsection*{Answer Key}
1. Some products start immediately from the Maturity stage, skipping the Introduction stage (commoditization), and products in the Decline stage can experience re-growth (revival), 2. Revival (or re-growth), 3. The criteria for transitioning between stages are ambiguous, and the shape differs depending on the aggregation level (Industry/Category/Brand), 4. Industry Level, Product Category Level, Brand Level, 5. Because the competitive environment and market characteristics of the entire product category are a more stable basis for strategic planning than the sales fluctuations of an individual brand, 6. Pre-launch promotion (advertising, SNS) maximizes demand, leading to maximum sales at the moment of launch, 7. Although the market is in the Maturity stage, the brand is in the Introduction stage, so it should adopt an Introduction stage strategy such as awareness enhancement and trial promotion (sampling, explaining the manufacturing method, etc.), 8. Because it is a tool for diagnosing the current market environment and deriving appropriate strategic options, not for accurately predicting future sales or transition timings, 9. The danger that the strategic decision itself by management to reduce investment because they judged 'it's the Decline stage' causes the product's sales to genuinely decline, 10. Maturity Stage, 11. It can be modified by the firm's strategic actions, 12. It is difficult. There are no clear criteria or boundaries for transition, 13. Different (e.g., the automobile industry is in the Maturity stage, but the electric vehicle category is in the Growth stage), 14. To reduce consumer perceived risk and purchasing risk for the new brand, even though the category itself is mature, 15. In something else (diagnosing the current market environment and deriving appropriate strategic options corresponding to the stage).
\end{document}
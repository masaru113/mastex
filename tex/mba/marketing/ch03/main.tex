\documentclass[uplatex,a4j,12pt,dvipdfmx]{jsarticle}
\usepackage{amsmath,amsthm,amssymb,bm,color,enumitem,mathrsfs,url,epic,eepic,ascmac,ulem,here,ascmac}
\usepackage[letterpaper,top=2cm,bottom=2cm,left=3cm,right=3cm,marginparwidth=1.75cm]{geometry}
\usepackage[english]{babel}
\usepackage[dvipdfm]{graphicx}
\usepackage[hypertex]{hyperref}
\title{Marketing Lecture 3 Lecture Notes}
\author{M. O.}
\date{\today}
\begin{document}
\maketitle
\tableofcontents
\section{The Meaning of Market Segmentation: A Study on the Theory and Practical Application of Segmentation}
\subsection{Introduction}
In today's market, consumer needs and values have diversified significantly, creating a situation where mass marketing aimed at all consumers is becoming dysfunctional. For companies to efficiently allocate limited management resources and establish a competitive advantage, it is essential not to view the market monolithically. Instead, they must identify and focus on customer groups that share specific needs.
The purpose of these lecture notes is to deepen understanding of the fundamental concepts of \textbf{market segmentation}, which forms the bedrock of marketing strategy, along with its strategic objectives and the conditions for its execution. Furthermore, we will examine specific segmentation approaches using product attributes, complemented by case studies.
\subsection{Key Concepts and Issues}
\subsubsection{Definition of Market Segmentation}
\textbf{Market segmentation} refers to the process of dividing a market into subsets of \textbf{consumers} who exhibit similar product perceptions, values, usage purposes, or purchasing behaviors. In other words, it is the activity of creating \textbf{groupings} of consumers who share some commonality.
The resulting homogeneous strata (groups) of consumers identified through this activity are called \textbf{segments}. The premise of this segmentation is the idea that each divided segment possesses homogeneous characteristics, allowing the company to apply optimized marketing activities (product, price, promotion, channel) tailored to those specific traits.
\subsubsection{The Purpose of Market Segmentation}
The primary objectives for companies undertaking market segmentation are to \textbf{establish a competitive advantage} and \textbf{maximize profits}.
By providing products and services that better fit the unique demands (needs) of a specific segment compared to competitors, it becomes possible to elicit high satisfaction from customers within that segment. This \textbf{high customer satisfaction} consequently leads to securing a high \textbf{market share} in that segment and ensuring \textbf{high profitability} by escaping price competition.
\subsubsection{Conditions for Effective Segmentation}
For a company to select strategically meaningful segments, the segmented markets must meet the following conditions.
\begin{enumerate}
	\item \textbf{Substantiality}: The segmented segment must possess a certain sales volume or market potential sufficient for the company to achieve profitability. If it is too niche, the business may not be viable even if high satisfaction is attained.
	\item \textbf{Measurability / Accessibility}: The segment's size, purchasing power, and characteristics must be measurable, and the company must be able to reach (approach) the segment. For example, criteria like `people with a strong future orientation` are ambiguous, difficult to identify, and problematic to set as targets for marketing activities.
	\item \textbf{Stability}: The market must possess stability, maintaining its characteristics over a certain period, rather than being formed by a temporary boom. If \textbf{sustainability} of sales cannot be expected, recouping product development and investment costs becomes difficult.
\end{enumerate}
\subsection{Application and Case Analysis}
\subsubsection{Market Segmentation by Product Attributes}
Market segmentation can be conducted based on differences in consumer preferences for the `attributes` a product possesses. Product attributes are broadly classified into `horizontal attributes` and `vertical attributes`.
\subsubsection{Horizontal Attributes and Segmentation}
\textbf{Horizontal attributes} refer to characteristics where objective superiority or inferiority is not clear-cut, and evaluation depends on consumer `taste` (e.g., strong vs. subtle detergent fragrance, clothing design, color).
Horizontal attributes easily generate diverse consumer segments/groups, making them a very effective basis for market segmentation.
\paragraph{Case Study: Ice Cream `Richness'}
The ice cream example shown in the lecture (e.g., Meiji \textbf{Super Cup} and Morinaga \textbf{Soh}) is a prime example of segmentation based on this horizontal attribute.
\begin{itemize}
	\item A consumer group (let's call it A) that prefers a light, refreshing taste with an icy texture, like `Soh`.
	\item A consumer group (let's call it B) that prefers a rich vanilla flavor, like `Super Cup`.
\end{itemize}
These are differences in preference, not superiority or inferiority of taste. For example, if there is a tendency for men to prefer refreshing tastes (an ideal point close to A) and women to prefer rich tastes (an ideal point close to B), companies can increase satisfaction in each segment by developing and offering products (A or B) tailored to the \textbf{ideal point} (the combination of attributes that would most satisfy consumers in that segment) of each respective segment.
\subsubsection{Vertical Attributes and Segmentation}
\textbf{Vertical attributes} refer to characteristics where objective superiority is clear, and all consumers agree that `higher (or better) is more desirable` (e.g., higher quality at the same price, PC processing speed, car fuel efficiency).
At first glance, it seems that with vertical attributes, only the `best-performing product` would be chosen, and segmentation would not occur. However, this is not the reality.
\paragraph{Case Study: Refrigerator `Capacity' and `Power Consumption'}
Consumers evaluate products by combining multiple vertical attributes (e.g., larger capacity is better, lower power consumption is better). In this process, the \textbf{weighting} consumers place on each attribute differs depending on their lifestyle and the \textbf{life context} in which the product is used.
\begin{itemize}
	\item \textbf{Single-person household segment}: May prioritize low power consumption or quiet operation, even if the capacity is small.
	\item \textbf{Family household segment}: May prioritize large capacity above all, even if power consumption is slightly higher.
\end{itemize}
Thus, when the combinations of vertical attributes or the weighting patterns for each \textbf{attribute value} are diverse, companies can segment the market by offering products with different attribute balances tailored to each preference pattern.
\subsection{Deeper Context and Lessons}
\textbf{\paragraph{The Influence of Existing Resources and Channel Conditions on Segment Selection}}
Market segmentation does not end merely with the results of market analysis (i.e., which segments exist). Which segment a company chooses to target is strongly influenced by its existing marketing activities and resource allocation.
For example, even after launching a product for the mass market, a company might later broadcast advertising messages focused on a specific age group to strengthen impact. Furthermore, if a company's brand image is strong among younger demographics, or if its relationships with \textbf{retailers} (especially convenience stores or \textbf{discount stores}) are skewed toward channels targeting youth, strategic adjustments may be made. This could involve retroactively changing the new product's concept itself to target the youth segment, thereby leveraging these existing resources (channels and brand).
\textbf{\paragraph{The Variability of Consumer `Ideal Points' due to Advertising Communication}}
The consumer `ideal point` (e.g., preferred ice cream flavor) identified in segmentation analysis is not fixed. Corporate marketing activities, especially \textbf{advertising communication}, can change consumer perceptions and values, potentially causing the ideal point itself to shift.
For instance, the proliferation of social messages or advertisements like `Be kinder to yourself` might cause consumers to temporarily move away from health consciousness and seek richer, sweeter ice cream (shifting the ideal point). This suggests that corporate activities not only analyze the market but can also actively shape the market (and its ideal points).
\textbf{\subsubsection{AI Supplement: Extension of Key Points}}
In the text for this lecture, specific `criteria (variables)` for conducting market segmentation, which will likely be detailed in Section 2 onward, were mentioned only in passing. Since the axes used to divide the market (the grouping criteria) are extremely important for practicing segmentation, key variables are supplemented here.
\begin{description}
	\item[Geographic Variables] Criteria for dividing by country, region, city size, climate, population density, etc.
	\item[Demographic Variables] Criteria for dividing by age, gender, family composition, income, occupation, education, etc. This is the most common and easily measurable criterion.
	\item[Psychographic Variables] Criteria for dividing by lifestyle, values, personality, social class, etc. `Future orientation` falls into this category, but as noted in the lecture, it can be accompanied by difficulties in measurement.
	\item[Behavioral Variables] Criteria based on product knowledge, usage occasion (TPO), purchase frequency, loyalty, benefits sought, etc. Preferences regarding the `horizontal attributes` and `vertical attributes` discussed in this lecture are closely related to segmentation based on `benefits` sought.
\end{description}
In practice, using these variables in combination rather than isolation allows for more precise segmentation.
\subsection{Conclusion}
In this lecture, we learned that market segmentation is not merely a task of market classification. It is a strategic activity directly linked to securing competitive advantage and profitability by achieving high customer satisfaction through responding to diverse consumer needs.
In particular, understanding diverse consumer preference patterns for a product's \textbf{horizontal attributes} (taste) and \textbf{vertical attributes} (performance) leads to the discovery of effective segments.
Furthermore, a practical lesson drawn from the `Deeper Context` analysis is the importance of deciding segmentation and targeting not only based on static market analysis. These decisions must also consider alignment with the company's \textbf{internal resources}, such as existing channel networks and brand image. It also highlighted the importance of adopting a perspective of active engagement with the market, recognizing that marketing activities themselves can shift consumer \textbf{ideal points}.
\subsection{List of Key Terms}
Market Segmentation, Segment, Horizontal Attributes, Vertical Attributes, Ideal Point, Grouping, High Customer Satisfaction, Competitive Advantage
\subsection{Comprehension Quiz}
\begin{enumerate}
	\item What is the process of dividing a market into subsets of `consumers who are similar in their perceptions of products, values, or purchasing behaviors` called?
	\item What is the homogeneous group of consumers resulting from segmentation called?
	\item What is the primary goal, besides establishing a competitive advantage, that companies aim to achieve through market segmentation?
	\item What are the two main outcomes that high customer satisfaction brings to a company, besides a high market share?
	\item What is the condition called that requires a segmented segment to be large enough for a company to earn a profit?
	\item Why is a criterion like `people with a strong future orientation` difficult for segmentation? Which condition is it hard to meet?
	\item What is the segmentation condition that requires characteristics to be maintained for a certain period, not just during a temporary boom?
	\item What are product attributes called where objective superiority is unclear and evaluation depends on `taste`?
	\item What are product attributes called where objective superiority is clear, and most people agree `higher (better) is more desirable`?
	\item Are `rich flavor` and `refreshing flavor` in ice cream classified as horizontal or vertical attributes?
	\item Are `large capacity` and `low power consumption` in refrigerators classified as horizontal or vertical attributes?
	\item Why is segmentation possible even with vertical attributes? It is because the (what) that consumers place on each attribute value is diverse.
	\item What is the combination of attributes (point) that would most satisfy consumers in a given segment called?
	\item When a company changes a product concept to leverage its existing convenience store network, this is an example of being influenced by what (resource)?
	\item Consumer `ideal points` are not fixed. What can they be changed by?
\end{enumerate}
\subsubsection*{Answer Key}
1. Market segmentation, 2. Segment, 3. High profitability, 4. High profitability, 5. Substantiality, 6. Measurability / Accessibility, 7. Stability, 8. Horizontal attributes, 9. Vertical attributes, 10. Horizontal attributes, 11. Vertical attributes, 12. Weighting, 13. Ideal point, 14. Existing channels (resources), 15. Advertising communication (or marketing activities)
\section{Criteria for Market Segmentation: A Systematic Review of Key Segmentation Bases}
\subsection{Introduction}
In the previous section, we learned about the significance and objectives of market segmentation. This section delves deeper into the specific `criteria` that companies use when segmenting the market, as a practical next step.
The market is an aggregation of consumers, and the axes used for \textbf{grouping} this aggregation significantly influence the effectiveness of marketing strategy. The purpose of these notes is to organize and deepen understanding of the three main segmentation variables typically used in consumer markets (B2C) (demographic, psychographic, and behavioral), as well as the segmentation criteria specific to industrial markets (B2B), examining their characteristics and application examples.
\subsection{Key Concepts and Issues}
\subsubsection{Research and Analysis as Prerequisites for Segmentation}
Market segmentation is conducted based not only on intuition but also on fundamental research and data analysis. It requires researching consumer characteristics, values, purchasing behaviors, and usage patterns, and then analyzing that data. In this lecture, the following two statistical methods were mentioned as primary techniques for discovering segments.
\begin{description}
	\item[\textbf{Factor Analysis}] A method used to extract common \textbf{latent factors} underlying multiple measurable variables (survey questions). For example, it is used to identify fundamental motivations (factors) like `emphasis on skincare,` `low-price orientation,` or `preference for specialized knowledge` from numerous questions about cosmetic purchasing behavior.
	\item[\textbf{Cluster Analysis}] A method used to classify large amounts of data based on certain common factors (such as the results of factor analysis), creating multiple groups (clusters) with high similarity. This visualizes what types of consumer groups exist within the market.
\end{description}
\subsubsection{Key Segmentation Variables for Consumer Markets (B2C)}
When segmenting the market as an aggregation of consumers, the following three variables are primarily used.
\begin{enumerate}
	\item \textbf{Demographic Variables}
	      \begin{itemize}
		      \item \textbf{Definition}: Objective and antecedent characteristics that define consumers, such as gender, age, \textbf{household size}, family life cycle, \textbf{income}, education level, \textbf{race}, and nationality.
		      \item \textbf{Characteristics}: \textbf{Easier to measure} than other variables (user-friendly), making them the most commonly used in practice. These variables also tend to be closely linked to consumer wants, preferences, and usage rates.
	      \end{itemize}
	\item \textbf{Psychographic Variables}
	      \begin{itemize}
		      \item \textbf{Definition}: Internal, individual characteristics such as lifestyle, personality, thoughts, and values.
		      \item \textbf{Characteristics}: Directly related to the motivations *why* consumers choose a product, such as `outdoor-oriented,` `brand-conscious,` or `eco-conscious.` While more difficult to measure than demographics, they can capture essential differences between consumers.
	      \end{itemize}
	\item \textbf{Behavioral Variables}
	      \begin{itemize}
		      \item \textbf{Definition}: Criteria based on actual behaviors or perceptions related to the product, such as purchase frequency, product knowledge, attitude toward the product, brand \textbf{loyalty}, usage occasion, and benefits sought.
		      \item \textbf{Characteristics}: Because they are based on actual behavior, they are easy to link directly to sales promotion and channel strategies.
	      \end{itemize}
\end{enumerate}
\subsubsection{Segmentation Variables for Industrial Markets (B2B)}
Segmentation of customers (companies) is also conducted in the \textbf{industrial market} (B2B), where the customers are organizations. The variables used in this context differ from those for general consumers.
\begin{enumerate}
	\item \textbf{Demographic Variables (B2B)}
	      \begin{itemize}
		      \item \textbf{Definition}: Objective characteristics of the customer company.
		      \item \textbf{Example}: Company size (large corporation, SME), region (metropolitan, rural), \textbf{industry type}.
	      \end{itemize}
	\item \textbf{Behavioral Characteristic Variables (B2B)}
	      \begin{itemize}
		      \item \textbf{Definition}: Characteristics and processes related to the customer company's purchasing.
		      \item \textbf{Example}: Size of the \textbf{buying center} (departments involved in purchasing), characteristics of the \textbf{decision-making process} (top-down vs. consensus among multiple departments), urgency of purchase, degree of \textbf{demand for customization}.
	      \end{itemize}
\end{enumerate}
\subsection{Application and Case Analysis}
\subsubsection{Golf Clubs (B2C: Intuitive Application of Multiple Variables)}
The thought process of the golf club salesperson, introduced at the beginning of the lecture, is an example of segmentation that intuitively combines multiple variables.
\begin{itemize}
	\item 30s or older (age), male (gender), holds a certain job title (proxy for income) $\to$ \textbf{Demographic variables}
	\item Participates in golf for social obligations (lifestyle), positive attitude toward golf $\to$ \textbf{Psychographic variables}
	\item Has purchased golf goods in the past (purchase history) $\to$ \textbf{Behavioral variable}
\end{itemize}
\subsubsection{Automobiles \& Housing (B2C: Demographic Variables)}
The types of cars that sell well differ between rural and urban areas. In rural areas, perhaps due to larger family sizes, there may be higher demand for larger vehicles or specific-use vehicles (e.g., light trucks). Regarding housing, there is also a tendency for larger homes to be preferred in rural areas. These are examples of deciding product lineups and \textbf{dealership} placement strategies based on demographic variables such as `region of residence` or `household size`.
\subsubsection{Outdoor Watches (B2C: Psychographic Variables)}
A company launched a watch targeting `young, outdoor-oriented men` (demographic + psychographic), but found that actual demand was strong among `female` consumers with an outdoor orientation. As a result, the company removed the `for men` classification and revised the product concept to `outdoor watch`. This indicates that the true segmentation axis was `outdoor orientation` (psychographic), not gender (demographic).
\subsubsection{Canned Coffee \& Vegetable Juice (B2C: Behavioral Variables - Usage Rate)}
The market is classified by purchase frequency into \textbf{heavy users} and \textbf{light users}. If heavy users are the primary target, the company might pursue a strategy of strengthening sales of 6-packs or bulk boxes rather than single items, and increasing resource allocation to \textbf{channels} that handle them, such as \textbf{discount stores}.
\subsubsection{Apparel \& Travel (B2C: Behavioral Variables - Loyalty \& Benefits)}
\begin{itemize}
	\item \textbf{Loyalty}: To retain customers with a strong attachment to a specific brand (loyal customers), differentiated marketing activities are implemented, such as issuing membership cards or holding members-only sales.
	\item \textbf{Benefits Sought}: In the travel industry, segments are divided based on what consumers are seeking (benefits), such as `time with family,` `rest and relaxation,` or `cuisine and knowledge.` Based on this, dedicated package tours are developed and advertising appeals are made.
\end{itemize}
\subsubsection{Industrial Goods Organizational Response (B2B: Demographics \& Behavioral Characteristics)}
An example of a B2B company segmenting customers on two axes: `company size` (demographic) and `demand for customization` (behavioral).
\begin{itemize}
	\item \textbf{Segment A (Large enterprise, high customization demand)}: Key accounts expected to contribute significantly to profit. $\to$ A \textbf{dedicated sales team} is established to provide thorough support.
	\item \textbf{Segment B (SME, low customization demand)}: Can be served with \textbf{standardized products} (commodities). If the customer base is large. $\to$ A \textbf{call center} is established to handle sales efficiently.
\end{itemize}
Thus, B2B segmentation is directly linked to the design of the company's \textbf{organizational structure}.
\subsection{Deeper Context and Lessons}
\textbf{\paragraph{Discovering Segmentation Variables and the Researcher's Skill}}
Market segmentation is not a simple task of just applying existing variables. As mentioned in this lecture, the ability to utilize methods like \textbf{factor analysis} and \textbf{cluster analysis} to discover effective segments (ways of slicing consumers) for one's own product market is, in itself, a crucial skill (or prowess) for marketing researchers and new product developers. Discovering groups with hidden needs provides clues for new product development.
\textbf{\paragraph{The Reality Behind the `Ease of Use' of Demographic Variables}}
The lecture emphasized that demographic variables are frequently used because they are `\textbf{easier to measure}` (user-friendly). In a survey, asking `What is your age?` yields a clear answer (regardless of whether the respondent is hesitant to answer). However, asking `Are you an indoor or outdoor person?` can confuse respondents (`What's the criterion?` `I stayed home this week, but I usually like being outside`), making it difficult to obtain clear data. This practical `ease of measurement` drives the use of demographic variables.
\textbf{\paragraph{The Influence of `Reference Groups' (Psychographic Variables)}}
One psychographic variable mentioned was the influence from groups to which an individual belongs (or aspires to belong). Segments strongly influenced by a \textbf{Reference Group} (e.g., the high school girl example from the lecture) exhibit conformity, wanting something `because everyone else has it.` Conversely, other segments seek differentiation, disliking things `because everyone else has it.` This `thinking in relation to others` can also be a crucial psychographic variable when determining a product concept.
\textbf{\subsubsection{AI Supplement: Extension of Key Points}}
This lecture detailed the `Variables` for segmenting markets in both B2C and B2B contexts. However, Segmentation (S) is only the first stage of the `STP` marketing strategy process. The lecture text was limited in its discussion of what comes `after` segmentation, so the concept of `Targeting` is supplemented here.
\textbf{Targeting Strategies}:
After clarifying the market structure through segmentation, a company must decide which segment(s) to select as its target. This is called targeting. The main strategy patterns are as follows.
\begin{itemize}
	\item \textbf{Undifferentiated Marketing}: Ignores segment differences and targets the entire market with a single product and marketing mix (mass marketing).
	\item \textbf{Differentiated Marketing}: Selects multiple segments and develops distinct products and marketing mixes for each. The B2B example (using sales teams and call centers) and the watch example (separating for men and women) from the lecture are close to this.
	\item \textbf{Concentrated / Niche Marketing}: A strategy of focusing management resources on a single (or a few) specific segments. Also known as a niche market strategy.
\end{itemize}
The `variables` learned in this lecture serve as crucial judgment material for deciding which of these targeting strategies to adopt.
\subsection{Conclusion}
Market segmentation is an essential strategic approach for building a competitive advantage in a diversifying market. This lecture demonstrated that the `criteria (variables)` used in its practice are extremely important.
In consumer markets (B2C), one must utilize easy-to-measure \textbf{demographic variables}, \textbf{psychographic variables} that capture the consumer's inner world, and \textbf{behavioral variables} directly tied to actual purchasing actions. In industrial markets (B2B), meanwhile, the customer company's \textbf{demographics (size or industry)} and \textbf{behavioral characteristics (purchasing process or customization demand)} become the primary variables.
The practical lessons from this lecture's case studies are: first, that simplistic variables (e.g., `gender` for watches) are not necessarily the optimal segmentation axis, and one must identify the true driver (e.g., `outdoor orientation`); and second, that especially in B2B, the results of segmentation are a critical management decision directly linked to the \textbf{sales structure and organizational design itself}.
\subsection{List of Key Terms}
Factor Analysis, Cluster Analysis, Demographic Variables, Psychographic Variables, Behavioral Variables, Reference Group, Heavy User, Light User, Industrial (B2B) Segmentation, Demographic Variables (B2B), Behavioral Characteristic Variables (B2B), Standardized Products (Commodities)
\subsection{Comprehension Quiz}
\begin{enumerate}
	\item What is the statistical method used to extract common latent factors from various measurable variables?
	\item What is the statistical method used to classify data into highly similar groups (clusters) based on common factors?
	\item What are objective, easily measurable consumer characteristic variables such as age, gender, income, and family structure called?
	\item What is the biggest reason demographic variables are the most commonly used in practice?
	\item What are the segmentation variables based on an individual's internal characteristics, such as lifestyle, values, and personality?
	\item When a consumer thinks, `I want it because my friends have it,` what is the group exerting strong influence called?
	\item What are segmentation variables based on actual behavior, such as product knowledge, purchase frequency, and brand loyalty?
	\item What is a customer segment that consumes large quantities of canned coffee or vegetable juice daily called?
	\item A strategy of strengthening bulk sales at discount stores for heavy users is based on which variable?
	\item In travel services, dividing the market by the value customers seek, such as `relaxation` or `knowledge acquisition,` is based on which specific behavioral variable?
	\item What is a market called where the customers are companies, not general consumers?
	\item In B2B segmentation, what are the variables used to classify customers by company size, region, or industry type?
	\item In B2B segmentation, what are the variables used to classify customers by their purchasing decision-making process or degree of customization demand?
	\item What is a typical organizational response a company might take for B2B customers that are large and have high customization demands?
	\item What is an efficient organizational response suitable for numerous B2B customers that are small, have low customization demand, and can be served by standardized (commodity) products?
\end{enumerate}
\subsubsection*{Answer Key}
1. Factor analysis, 2. Cluster analysis, 3. Demographic variables, 4. Because they are easy to measure (user-friendly), 5. Psychographic variables, 6. Reference group, 7. Behavioral variables, 8. Heavy users, 9. Behavioral variable (usage rate), 10. Benefits sought, 11. Industrial market (B2B market), 12. Demographic variables, 13. Behavioral characteristic variables, 14. Establish a dedicated sales team, 15. Establish a call center
\section{Target Market Selection and Multi-Market Strategies: A Comparative Analysis of Segment Evaluation and Differentiated/Concentrated Approaches}
\subsection{Introduction}
In the preceding sections, we learned about the significance of `market segmentation`—classifying the market into meaningful customer groups—and the variables used as criteria. However, strategy is not complete with segmentation alone. Companies must select the market(s) they should enter from among the multiple identified segments.
These lecture notes focus on the next step after segmentation: `targeting` (setting the target market). The aim is to organize the theoretical frameworks and case studies, and deepen understanding, regarding how to evaluate segmented markets and what strategic approaches companies should take toward the selected single or multiple markets.
\subsection{Key Concepts and Issues}
\subsubsection{Evaluating Market Segments}
When selecting a target market from diverse segments, it is first necessary to evaluate each segment. In modern marketing, it is common to establish more precise segments not just by a single criterion (e.g., age), but by combining multiple variables (e.g., age, investment patterns, risk tolerance, etc., as asked when opening a bank account).
However, this involves a \textbf{trade-off}. If segmentation is pursued excessively, the degree of fit to customer needs increases, but the size of the target market shrinks. Therefore, companies must identify an `appropriate market segment`.
\paragraph{Evaluation by Economic Attractiveness}
The most important criterion for evaluating segments is `\textbf{economic attractiveness}`. Economic attractiveness is judged comprehensively based on the following factors.
\begin{enumerate}
	\item \textbf{Market Size (Scale)}: The segment must secure a certain market size sufficient for the company to achieve profitability. This is essential for enjoying \textbf{economies of scale}. \textbf{Economies of scale} refers to the effect where the production cost per unit (especially \textbf{fixed costs} like factory construction) decreases as production volume increases. If the market size is too small, this effect cannot be obtained, and per-unit costs remain high. Marketing expenses, such as advertising and channel development, also become inefficient if the target audience is too small.
	\item \textbf{Market Growth}: Even if the current market size is small, high growth potential, where an increase in users is anticipated, is also an attractive factor. For example, the overseas travel market in the 1980s was initially small, but travel agencies, anticipating rapid subsequent growth, developed and offered diverse products. Early entry into a growing market offers the potential to gain a `\textbf{first-mover advantage},` establishing a strong position akin to being synonymous with that product category.
	\item \textbf{Competition \& Profitability}: An attractive market (large and high-growth) is, by necessity, also attractive to \textbf{competitors}, and many firms are likely to enter. As a result, competition for share intensifies, leading to a risk of falling into a war of attrition (\textbf{Red Ocean}) where only investments in new product development and advertising pile up, making profits elusive. Therefore, it is necessary to analyze the competitive situation and assess whether sustainable profitability can be expected.
\end{enumerate}
\paragraph{Fit with Own Management Resources}
No matter how high the economic attractiveness, entry will not succeed if the company lacks the capability (\textbf{management resources}) to serve that segment. It is essential to coolly evaluate the company's financial strength, R\&D capabilities, production capacity, existing channel network, and brand value, and examine whether it is a segment where a `\textbf{competitive advantage}` can be established based on the company's strengths and weaknesses.
\subsubsection{Factors Determining the Degree of Market Segmentation}
\textbf{Derek F. Abell} discussed the definition of business domains in his 1980 research. Drawing on that discussion, this lecture considers the factors that determine `to what degree` a company should conduct market segmentation.
\begin{itemize}
	\item \textbf{Factors favoring a low degree of segmentation (targeting a broad market)}:
	      \begin{itemize}
		      \item When customer \textbf{price sensitivity} is high, and price is the primary selection criterion.
		      \item When customers are only focused on the product's \textbf{basic functions} (e.g., a mobile phone just needs to make calls).
		      \item When the management resources (channels, advertising, etc.) needed to serve multiple segments are highly similar, allowing for significant \textbf{experience effects} or \textbf{scale effects}.
		      \item When the company possesses abundant available management resources.
	      \end{itemize}
	\item \textbf{Factors favoring a high degree of segmentation (focusing on a narrow market)}:
	      \begin{itemize}
		      \item When the market has matured, and customers emphasize \textbf{additional or secondary functions}.
		      \item When segment characteristics differ greatly, making the cost of serving each one individually high.
		      \item When customers have high product judgment (knowledge) and can appreciate sophisticated product specifications or the manufacturer's efforts (e.g., high-end customers for high-performance computers).
		      \item When customer \textbf{involvement} is high, and they tend to enjoy the process of comparing and selecting product varieties itself.
	      \end{itemize}
\end{itemize}
\subsubsection{Strategies for Addressing Multiple Markets (Targeting Strategies)}
A company's strategies for addressing the market can be broadly divided into `differentiated` and `concentrated` types.
\paragraph{Differentiated Marketing}
A strategy within a single product business of serving multiple market segments (ideally, almost all of them), each with different products or marketing mixes. Also known as a `\textbf{full coverage strategy},` it is often adopted by \textbf{large corporations} with abundant management resources.
\begin{itemize}
	\item \textbf{Problem}: Costs increase significantly due to addressing all needs. Specific examples include R\&D expenses, production process setup costs for each product, marketing research costs for each segment, and the complexity of inventory management.
	\item \textbf{Key Point}: It is crucial to compare the forecasted increase in sales against the forecasted increase in the above costs, and find the balance that maximizes profit.
\end{itemize}
\paragraph{Concentrated Marketing}
A strategy within a single product business of targeting only one specific product segment and focusing management resources on it.
\begin{itemize}
	\item \textbf{Objective}: To achieve an overwhelming \textbf{competitive advantage (position)} in that specific segment.
	\item \textbf{Characteristics}: Allows for the accumulation of extensive knowledge about that segment, suitable for \textbf{SMEs} with limited management resources.
	\item \textbf{Risk}: If the segment loses its attractiveness due to competitor entry or changes in consumer demand (e.g., a `natural look` trend impacting the eye makeup market), the company cannot compensate with other businesses, placing its overall survival in jeopardy.
\end{itemize}
\subsubsection{Targeting within a Multi-Business Portfolio}
When a company has multiple product businesses (e.g., food and cosmetics), targeting patterns are further divided into two.
\begin{enumerate}
	\item \textbf{Common Segment Strategy}:
	      A pattern of pursuing a common product segment across all product businesses.
	      \begin{itemize}
		      \item Example (Concentrated): \textbf{Dior} or \textbf{Gucci} focus on `luxury-oriented` customers across all businesses, including apparel, bags, and accessories.
		      \item Example (Differentiated): A major confectionery manufacturer targets all age groups (full coverage), from children to the middle-aged and elderly.
	      \end{itemize}
	      \textbf{Advantage}: Allows for the use of common channels and advertising concepts, making it easier to achieve \textbf{corporate image consistency}.
	\item \textbf{Individual Segment Strategy}:
	      A pattern of selecting the optimal, different segment for each product business, considering economic attractiveness and resource constraints.
	      \begin{itemize}
		      \item Example: \textbf{Suntory} adopts a \textbf{differentiated} strategy (diverse customer base) in its food and beverage business, while adopting a \textbf{concentrated} strategy in its cosmetics business, targeting the specific segment of `women aged 40 and over.`
	      \end{itemize}
	      \textbf{Characteristic}: At first glance, resources seem dispersed and inefficient. However, it is a strategic choice suited to the growth stage of each business. For example, it may envision the possibility of future diversification (moving to full coverage) in the cosmetics business by leveraging knowledge accumulated in other businesses.
\end{enumerate}
\subsection{Application and Case Analysis}
\begin{description}
	\item[Opening a bank account] Using a \textbf{combination of multiple variables}—such as customer age, income, and risk tolerance for investment—to identify segments and develop/recommend financial products tailored to each.
	\item[Overseas travel in the 1980s] An example where travel agencies developed diverse products in response to high \textbf{market growth}, despite the small market size at the time.
	\item[High-performance computers] Because \textbf{highly knowledgeable, high-involvement customers} who can understand complex product specifications exist, companies \textbf{increase the degree of segmentation} to meet specialized needs.
	\item[Eye makeup market] A company specializing in mascara and eyeliner, focusing on the single segment of low-price-oriented young women, is a typical example of \textbf{concentrated marketing}.
	\item[Dior / Gucci] An example of having multiple businesses (apparel, bags, accessories) but adopting a \textbf{concentrated} strategy toward a \textbf{common segment} (`luxury-oriented`) across all of them.
	\item[Suntory] An example of an \textbf{individual segment strategy}, using different targeting strategies within its business portfolio, such as the food business (differentiated) and the cosmetics business (concentrated on women 40+).
\end{description}
\subsection{Deeper Context and Lessons}
\textbf{\paragraph{The Trade-off of Segmentation}}
As mentioned at the start of the lecture, `narrowing the target` by increasing segmentation criteria carries risk. Pursuing a perfect fit with needs too aggressively can make the target market too small to be economically viable. It is crucial in practice to always be aware of this \textbf{trade-off}.
\textbf{\paragraph{`First-Mover Advantage' and Market Future}}
The strategy of entering early to anticipate \textbf{market growth} and build a `synonymous` status (first-mover advantage) is attractive. However, it requires the premise that `the market will reliably continue to grow,` and misreading the market poses a significant risk.
\textbf{\paragraph{The Balance Between `What Can Be Done' and `What Should Be Done'}}
The importance of judging segment evaluation on the two axes of `economic attractiveness` (what should be done) and `own management resources` (what can be done) was emphasized. Even if a market is attractive, if it does not fit the company's strengths, it is not a market that should be entered. Before engaging in a war of attrition with competitors in a \textbf{Red Ocean}, it is necessary to coolly assess whether one's own \textbf{competitive advantage} can be established.
\textbf{\paragraph{Lecture Conclusion: Application to Daily Life}}
The way of thinking learned in this lecture on market segmentation is applicable not only in business scenes but also in daily interpersonal relationships (such as customer service). The perspective of not viewing others monolithically, but rather identifying their `type` (characteristics and needs) and changing one's approach (interaction) accordingly, is rich in practical implications.
\textbf{\subsubsection{AI Supplement: Extension of Key Points}}
This lecture provided detailed explanations of `differentiated marketing` and `concentrated marketing` as targeting strategies. However, in foundational marketing strategy theory, there is another main strategy often contrasted with these: `\textbf{Undifferentiated Marketing}`.
\begin{description}
	\item[\textbf{Undifferentiated Marketing}]
	      Also called \textbf{mass marketing}. This is a strategy that deliberately ignores the differences between segments identified by market segmentation, approaching the entire market (the mass) with a single product and a single marketing mix (price, promotion, channel).
	      The Ford Model T and Coca-Cola were classic examples, but this strategy has become difficult to adopt in modern markets where consumer needs have diversified. However, it may still be effective for commodities (homogeneous goods) like salt or sugar, or during a market's introductory phase. The `differentiated` and `concentrated` strategies learned in this lecture can be understood more deeply when contrasted with this `undifferentiated` type.
\end{description}
\subsection{Conclusion}
After visualizing the market structure through market segmentation, companies face the critical decision of `targeting` (selecting the target market). In this lecture, we learned that the evaluation criteria consist of two aspects: `\textbf{economic attractiveness}` (market size, growth, competition) and `\textbf{fit with the company's own management resources}`.
We also understood, referencing Abell's discussion, the need to adjust the degree (breadth) of segmentation according to customer price sensitivity and involvement, and that specific response strategies include the large-corporation model `\textbf{differentiated marketing}` (full coverage) and the SME model `\textbf{concentrated marketing}`.
A practical lesson from this lecture is that an `attractive market` is not necessarily the `optimal market for one's company.` A strategic perspective is essential, one that coolly evaluates the risks of a \textbf{Red Ocean} and whether a \textbf{competitive advantage} can be established, to select a market that yields sustainable profits.
\subsection{List of Key Terms}
Derek F. Abell
\vspace{\baselineskip}
Economic Attractiveness, Economies of Scale, Fixed Costs, Variable Costs, Market Growth, First-Mover Advantage, Red Ocean, Management Resources, Competitive Advantage, Involvement, Differentiated Marketing, Full Coverage Strategy, Concentrated Marketing, Image Consistency
\subsection{Comprehension Quiz}
\begin{enumerate}
	\item What is the comprehensive criterion used to evaluate segmented markets, which includes market size, growth, and profitability?
	\item What is the effect called where the fixed cost per product unit decreases as production volume increases?
	\item What is the characteristic of a market that, while currently small, is projected to see an increase in users in the future?
	\item What is the advantage gained by entering a growth market early and becoming synonymous with the category?
	\item What is the state of a market called where competition is intense, investments pile up, and profits are hard to make?
	\item Even if a market has economic attractiveness, the need to assess if one can compete using one's own funds and technology is called (what)?
	\item If customer price sensitivity is high, should the degree of market segmentation (fineness) be increased or decreased?
	\item If customers have extensive product knowledge and high involvement, should the degree of market segmentation be increased or decreased?
	\item What is the strategy of serving almost all segments within a single product business with different products called?
	\item Is differentiated marketing, also known as the `full coverage strategy,` primarily suited for large corporations or SMEs?
	\item What is the strategy of concentrating resources on only one specific segment within a single product business called?
	\item What is the greatest risk faced by concentrated marketing (e.g., due to changes in demand)?
	\item What is the advantage of a strategy like Dior or Gucci's, which consistently targets a common `luxury-oriented` segment across multiple product businesses?
	\item What did the lecture call the pattern used by Suntory, which employs different segment strategies for its food business and its cosmetics business?
	\item (From AI supplement) What is the strategy of ignoring segment differences and approaching the entire market with a single product called?
\end{enumerate}
\subsubsection*{Answer Key}
1. Economic attractiveness, 2. Economies of scale, 3. Market growth, 4. First-mover advantage, 5. Red ocean, 6. Fit with own management resources, 7. Decrease (serve a broad market), 8. Increase (serve finely), 9. Differentiated marketing, 10. Large corporations, 11. Concentrated marketing, 12. The specific segment losing its attractiveness due to changes in demand or competitor entry, 13. Achievement of image consistency, 14. Individual segment strategy (or a pattern of selecting the appropriate segment for each product business), 15. Undifferentiated marketing (Mass marketing)
\end{document}
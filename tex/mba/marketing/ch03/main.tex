\documentclass[uplatex,a4j,12pt,dvipdfmx]{jsarticle}
\usepackage{amsmath,amsthm,amssymb,bm,color,enumitem,mathrsfs,url,epic,eepic,ascmac,ulem,here,ascmac}
\usepackage[letterpaper,top=2cm,bottom=2cm,left=3cm,right=3cm,marginparwidth=1.75cm]{geometry}
\usepackage[english]{babel}
\usepackage[dvipdfm]{graphicx}
\usepackage[hypertex]{hyperref}
\title{Marketing Lecture 3: Lecture Notes}
\author{M. O.}
\date{\today}
\begin{document}
\maketitle
\tableofcontents
\section{The Meaning of Market Segmentation: A Study of its Theory and Practical Application}
\subsection{Introduction}
In today's market, consumer needs and values have become remarkably diversified, making it difficult for mass marketing that targets all consumers to function effectively. For companies to efficiently allocate limited resources and establish a competitive advantage, it is essential to move beyond a uniform view of the entire market. Instead, they must identify and focus on customer groups that share specific needs.
The purpose of these lecture notes is to deepen the understanding of the fundamental concepts, strategic objectives, and practical conditions of \textbf{market segmentation}, which forms the core of marketing strategy. Furthermore, we will examine specific segmentation approaches using product characteristics, along with case studies.
\subsection{Key Concepts and Points}
\subsubsection{Definition of Market Segmentation}
\textbf{Market segmentation} refers to the process of dividing a market into sets of \textbf{consumers} who are similar in their perceptions of products, values, usage purposes, purchasing behavior, and so on. In other words, it is the activity of creating \textbf{groupings} of consumers who share some commonality.
The strata (groups) of consumers with homogeneous characteristics created by this activity are called \textbf{segments}. The premise of this segmentation is that each divided segment possesses homogeneous characteristics, allowing the company to apply optimized marketing activities (product, price, promotion, place/channel) tailored to those characteristics.
\subsubsection{The Purpose of Market Segmentation}
The primary objectives for a company engaging in market segmentation are \textbf{establishing a competitive advantage} and \textbf{maximizing profit}.
By providing products and services that 'fit' the specific demands (needs) of a particular segment better than competitors, it becomes possible to achieve high satisfaction from customers within that segment. This \textbf{high customer satisfaction} consequently leads to \textbf{securing a high market share} in that segment and ensuring \textbf{high profitability} by escaping price competition.
\subsubsection{Conditions for Effective Segmentation}
For a company to select strategically meaningful segments, the segmented market must meet the following conditions.
\begin{enumerate}
	\item \textbf{Substantiality}: The segmented segment must have a certain sales volume or market potential sufficient for the company to turn a profit. If it is too niche, it may not be commercially viable, even if high satisfaction is achieved.
	\item \textbf{Measurability / Accessibility}: The size, purchasing power, and characteristics of the segment must be measurable, and the company must be able to reach (access) that segment. For example, criteria like 'people who are strongly future-oriented' are ambiguous, difficult to identify, and hard to set as a target for marketing activities.
	\item \textbf{Stability}: The market must not be formed by a temporary boom; it needs stability, maintaining its characteristics for a certain period. If \textbf{sustainability} of sales cannot be expected, product development and investment recovery become difficult.
\end{enumerate}
\subsection{Application and Case Analysis}
\subsubsection{Market Segmentation by Product Characteristics}
Market segmentation can be performed based on differences in consumer preferences for product 'characteristics'. Product characteristics are broadly classified into 'horizontal characteristics' and 'vertical characteristics'.
\subsubsection{Horizontal Characteristics and Segmentation}
\textbf{Horizontal characteristics} refer to attributes where objective superiority is not clear, and evaluation depends on consumer 'taste'. (e.g., strong/subtle scent of detergent, clothing design, color).
Horizontal characteristics easily generate diverse consumer segments/groups, making them a very effective basis for market segmentation.
\paragraph{Case Study: 'Richness' in Ice Cream}
The ice cream example shown in the lecture (e.g., Meiji \textbf{Super Cup} and Morinaga \textbf{Soh}) is a prime example of segmentation based on these horizontal characteristics.
\begin{itemize}
	\item A consumer group that prefers a light, refreshing taste with an icy texture, like 'Soh' (let's call this Group A).
	\item A consumer group that prefers a rich vanilla taste, like 'Super Cup' (let's call this Group B).
\end{itemize}
This is not a matter of superior or inferior taste, but a difference in preference. For example, if there is a tendency for men to prefer a refreshing taste (an ideal point close to A) and women to prefer a rich taste (an ideal point close to B), the company can develop and offer products (A or B) tailored to the \textbf{ideal point} of each segment (the combination of characteristics that would most satisfy that segment's consumers), thereby increasing the satisfaction of each group.
\subsubsection{Vertical Characteristics and Segmentation}
\textbf{Vertical characteristics} refer to attributes where objective superiority is clear, and all consumers would agree that 'higher (better) is more desirable'. (e.g., higher quality for the same price, PC processing speed, car fuel efficiency).
At first glance, it might seem that with vertical characteristics, only the 'best-performing product' would be chosen, and segmentation would not occur. However, this is not the case in reality.
\paragraph{Case Study: Refrigerator 'Capacity' and 'Power Consumption'}
Consumers evaluate products by combining multiple vertical characteristics (e.g., larger capacity is better, lower power consumption is better). In this process, which characteristic the consumer emphasizes (\textbf{weighting}) differs depending on their lifestyle and the \textbf{life situations} in which they use the product.
\begin{itemize}
	\item \textbf{Single-person household segment}: May prioritize low power consumption and quiet operation, even if the capacity is small.
	\item \textbf{Family household segment}: May prioritize large capacity above all, even if power consumption is somewhat high.
\end{itemize}
Thus, when the combinations of vertical characteristics and the patterns of weighting given to each \textbf{attribute value} are diverse, companies can segment the market by offering products with different balances of characteristics to segments with different preference patterns.
\subsection{Deeper Context and Lessons}
\textbf{\paragraph{The Influence of Existing Resources and Channel Conditions on Segment Selection}}
Market segmentation does not end merely with the results of market analysis (i.e., which segments exist). Which segment a company selects as its target is strongly influenced by its existing marketing activities and resource allocation.
For example, even after launching a product for the mass market, a company might send out advertising messages specifically targeting a certain age group to strengthen impact. Furthermore, if a company's brand image is strong among the youth, or if its relationships with \textbf{retailers} (especially convenience stores and \textbf{discount stores}) are skewed toward channels for young people, strategic adjustments may be made, such as changing the new product's concept itself to target the youth afterward, in order to leverage those existing resources (channels and brand).
\textbf{\paragraph{The Variability of Consumer 'Ideal Points' due to Advertising Communication}}
The consumer 'ideal points' (e.g., preferred ice cream taste) identified in segmentation analysis are not fixed. There is a possibility that consumer perceptions and values may change due to corporate marketing activities, especially \textbf{advertising communication}, causing the ideal points themselves to shift.
For example, the spread of social messages or advertisements like 'Be kinder to yourself' might cause consumers to temporarily move away from a health-conscious mindset and seek richer, sweeter ice cream (a shift in the ideal point). This suggests that corporate activities do not just analyze the market, but can also actively shape the market (and its ideal points).
\textbf{\subsubsection{AI Supplement: Expansion of Key Points}}
The text for this lecture (Section 1) had limited mention of the specific 'criteria (variables)' used to perform market segmentation, which will presumably be detailed in Section 2 and beyond. Since the axis used to divide the market (the basis for grouping) is extremely important in practicing segmentation, we will supplement with the main variables.
\begin{description}
	\item[Geographic Variables] Criteria for dividing by country, region, city size, climate, population density.
	\item[Demographic Variables] Criteria for dividing by age, gender, family size, income, occupation, education. This is the most common and easily measured criterion.
	\item[Psychographic Variables] Criteria for dividing by lifestyle, values, personality, social class. 'Future-orientation' falls here, but as pointed out in the lecture, it can be difficult to measure.
	\item[Behavioral Variables] Criteria for dividing by product knowledge, usage occasion (TPO), purchase frequency, loyalty, benefits sought. The preferences for 'horizontal' and 'vertical' characteristics discussed in this lecture are closely related to segmentation based on sought 'benefits'.
\end{description}
In practice, using these variables in combination rather than isolation allows for more precise segmentation.
\subsection{Conclusion}
In this lecture, we learned that market segmentation is not merely a task of market classification, but a strategic activity directly linked to securing competitive advantage and profit for the company by achieving high customer satisfaction through responding to diverse consumer needs.
In particular, understanding the diverse consumer preference patterns for a product's \textbf{horizontal characteristics} (taste) and \textbf{vertical characteristics} (performance) leads to the discovery of effective segments.
Furthermore, the 'Deeper Context' discussion provided practical lessons: first, that segmentation and targeting should be decided not only based on static market analysis but also in consideration of alignment with the company's \textbf{internal resources}, such as existing channel networks and brand image. Second, it highlighted the importance of having a perspective of active engagement with the market, recognizing that the company's own marketing activities can shift consumer \textbf{ideal points}.
\subsection{Key Terminology}
Market Segmentation, Segment, Horizontal Characteristics, Vertical Characteristics, Ideal Point, Grouping, High Customer Satisfaction, Competitive Advantage
\subsection{Comprehension Check Quiz}
\begin{enumerate}
	\item Why is market segmentation, rather than mass marketing targeting the entire market, considered essential in modern marketing? What market change is behind this?
	\item When a company targets a segmented 'segment', what kind of marketing activities is it premised that they will apply?
	\item What are the two primary objectives for a company engaging in market segmentation: establishing competitive advantage and what else?
	\item By what logic does market segmentation lead to securing 'high profit'?
	\item Why is 'Substantiality' (a certain sales volume) considered a condition for segmentation? What is the business reason?
	\item Psychological criteria like 'people who are strongly future-oriented' are said to have difficulty meeting the 'Accessibility' condition. Why is this difficult in practice?
	\item 'Stability' is a condition for segmentation because it relates to the feasibility of which corporate activities?
	\item Why are 'horizontal characteristics' often an effective basis for market segmentation? What property do they have?
	\item Even though 'vertical characteristics' have clear objective superiority, market segmentation may still be possible. Why is that?
	\item Why are the 'richness' and 'refreshing feel' of ice cream classified as horizontal, not vertical, characteristics?
	\item Why are refrigerator 'capacity' and 'power consumption' classified as vertical, not horizontal, characteristics?
	\item In vertical characteristics (e.g., PC processing speed), why might different products be sold to (and segments formed for) single-person households and family households? What differs between them?
	\item What is the strategic aim when a company develops a product tailored to a segment's 'ideal point'?
	\item Beyond the results of market analysis (segmentation), what 'internal factors' strongly influence a company's target selection?
	\item What strategic possibility does the fact that advertising communication can shift consumer 'ideal points' suggest for a company?
\end{enumerate}
\subsubsection*{Answer Key}
1. Because consumer needs and values have become remarkably diversified.
2. Marketing activities optimized for the characteristics of that segment.
3. Maximizing profit.
4. Because fitting specific needs enhances customer satisfaction, allowing escape from price competition.
5. Because it is not a viable business if the scale is insufficient to turn a profit (investment recovery).
6. Because it is difficult to measure, and it is hard to set up efficient means to approach that segment.
7. It affects product development and investment recovery (sustainability of sales).
8. Because there is no objective superiority, and evaluation is divided based on consumer 'taste'.
9. Because consumers differ in which characteristics they emphasize (weighting) or in their usage situations.
10. Because it is a characteristic where evaluation is based on individual 'taste', not objective superiority.
11. Because it is a characteristic with clear, objective superiority that most consumers would agree upon.
12. Because their 'weighting' (what they prioritize) for each attribute value is different.
13. To maximize that segment's satisfaction and gain high support (market share).
14. The company's internal resources, such as existing channel networks and brand image.
15. That a company can actively shape market needs (not just analyze the market).
\section{Criteria for Market Segmentation: A Systematic Organization of Key Segmentation Criteria}
\subsection{Introduction}
In the previous section, we learned the significance and purpose of market segmentation. In this section, as a practical step, we will delve deeper into the specific 'criteria' that companies use when segmenting a market.
A market is a collection of consumers, and the axis used for \textbf{grouping} this collection determines the effectiveness of the marketing strategy. The purpose of these lecture notes is to organize and deepen understanding of the three main segmentation variables (demographic, psychographic, behavioral) commonly used in the consumer market (B2C), as well as the unique segmentation criteria for the industrial goods market (B2B), examining their characteristics and application examples.
\subsection{Key Concepts and Points}
\subsubsection{Preliminary Research and Analysis for Segmentation}
Market segmentation is conducted based not just on intuition, but on fundamental research and data analysis. It is necessary to survey and analyze consumer characteristics, values, purchasing behavior, and usage patterns. In this lecture, the following two statistical methods were raised as primary techniques for discovering segments.
\begin{description}
	\item[\textbf{Factor Analysis}] A technique to extract common \textbf{latent factors} underlying various measurable variables (survey items). For example, it is used to identify fundamental motivations (factors) such as 'emphasis on skincare,' 'low-price orientation,' or 'specialized knowledge orientation' from numerous survey items about cosmetics purchasing behavior.
	\item[\textbf{Cluster Analysis}] A technique to classify large amounts of data based on certain common factors (such as results from factor analysis) to create multiple groups (clusters) with high similarity. This helps visualize what types of consumer groups exist within the market.
\end{description}
\subsubsection{Key Segmentation Variables for the Consumer Market (B2C)}
When segmenting the market as a collection of consumers, the following three variables are primarily used.
\begin{enumerate}
	\item \textbf{Demographic Variables}
	      \begin{itemize}
		      \item \textbf{Definition}: Objective and antecedent characteristics that define consumers, such as gender, age, \textbf{household size}, family life cycle, \textbf{income}, education level, \textbf{race}, and nationality.
		      \item \textbf{Characteristics}: Compared to other variables, they are \textbf{easy to measure} (user-friendly) and thus are the most frequently used in practice. These variables also tend to be closely linked to consumer desires, preferences, and usage frequency.
	      \end{itemize}
	\item \textbf{Psychographic Variables}
	      \begin{itemize}
		      \item \textbf{Definition}: An individual's internal characteristics, such as lifestyle, personality, thoughts, and values.
		      \item \textbf{Characteristics}: Directly relates to the motivation why consumers choose a product, such as 'outdoor-oriented,' 'brand-conscious,' or 'eco-conscious.' Although more difficult to measure than demographic variables, they can capture essential differences between consumers.
	      \end{itemize}
	\item \textbf{Behavioral Variables}
	      \begin{itemize}
		      \item \textbf{Definition}: Criteria based on actual behavior and perceptions related to the product, such as purchase frequency, product knowledge, attitude toward the product, brand \textbf{loyalty}, usage occasion, and benefits sought.
		      \item \textbf{Characteristics}: Because they are based on actual behavior, they are easy to link directly to sales promotion and channel strategies.
	      \end{itemize}
\end{enumerate}
\subsubsection{Segmentation Variables for the Industrial Goods Market (B2B)}
In the \textbf{industrial goods market} (B2B), where the customers are companies, segmentation of customers (companies) is also performed. The variables used in this case differ from those for the general consumer market.
\begin{enumerate}
	\item \textbf{Demographic Variables (B2B)}
	      \begin{itemize}
		      \item \textbf{Definition}: Objective characteristics of the customer company.
		      \item \textbf{Examples}: Company size (large corporation, SME), region (metropolitan, rural), \textbf{industry type}.
	      \end{itemize}
	\item \textbf{Operating/Behavioral Variables (B2B)}
	      \begin{itemize}
		      \item \textbf{Definition}: Characteristics and processes related to the customer company's purchasing.
		      \item \textbf{Examples}: Size of the \textbf{buying center} (departments involved in purchasing), characteristics of the \textbf{decision-making process} (top-down or multi-department consensus), urgency of purchase, degree of \textbf{demand for customization}.
	      \end{itemize}
\end{enumerate}
\subsection{Application and Case Analysis}
\subsubsection{Golf Clubs (B2C: Intuitive Application of Multiple Variables)}
The thought process of the golf club sales representative shown at the start of the lecture is an example of segmentation intuitively combining multiple variables.
\begin{itemize}
	\item 30s or older (age), male (gender), certain job title (proxy for income) $\to$ \textbf{Demographic Variables}
	\item Participates in golf for social reasons (lifestyle), positive about golf $\to$ \textbf{Psychographic Variables}
	\item Has purchased golf goods in the past (purchase history) $\to$ \textbf{Behavioral Variables}
\end{itemize}
\subsubsection{Cars and Housing (B2C: Demographic Variables)}
The types of cars that sell differ between rural and urban areas. In rural areas, perhaps because family sizes are larger, there may be higher demand for larger cars or specific-use vehicles (e.g., light trucks). For housing as well, there is a tendency for larger homes to be preferred in rural areas. These are examples of deciding product lineups and \textbf{distributor/agency} placement strategies based on demographic variables like 'residential area' and 'household size'.
\subsubsection{Outdoor Watches (B2C: Psychographic Variables)}
A company launched a watch targeting 'young, outdoor-oriented men' (demographic + psychographic), but found that there was actually significant demand from 'female' consumers with an outdoor orientation. As a result, the company dropped the 'for men' framing and revised the product concept to 'outdoor watch'. This indicates that the true axis of segmentation was not gender (demographic) but 'outdoor orientation' (psychographic).
\subsubsection{Canned Coffee/Vegetable Juice (B2C: Behavioral Variables - Usage Frequency)}
The market is classified by purchase frequency into \textbf{heavy users} and \textbf{light users}. If heavy users are the main target, the company might strengthen sales of 6-packs or boxes rather than single items, and consider strategies to increase resource allocation to \textbf{channels} that handle them, such as \textbf{discount stores}.
\subsubsection{Apparel/Travel (B2C: Behavioral Variables - Loyalty/Benefits)}
\begin{itemize}
	\item \textbf{Loyalty}: To retain customers with strong attachment to a specific brand (loyal customers), differentiated marketing activities are carried out, such as issuing membership cards and holding members-only sales.
	\item \textbf{Benefits Sought}: In travel services, segments are divided based on what the consumer is seeking (benefits), such as 'time with family,' 'rest and relaxation,' or 'cuisine and knowledge.' Based on this, dedicated package tours are developed and advertising appeals are made.
\end{itemize}
\subsubsection{Industrial Goods Organizational Response (B2B: Demographics \& Operating Characteristics)}
An example of a B2B company segmenting customers on the two axes of 'company size' (demographic) and 'customization demand' (operating characteristic).
\begin{itemize}
	\item \textbf{Segment A (Large enterprise, high customization demand)}: Key customers expected to contribute significant profit. $\to$ Establish a \textbf{dedicated sales team} to provide thorough support.
	\item \textbf{Segment B (SME, low customization demand)}: Can be served with \textbf{standardized products}. If customer numbers are large. $\to$ Set up a \textbf{call center} to handle sales efficiently.
\end{itemize}
Thus, B2B segmentation is also directly linked to the design of the company's \textbf{organizational structure}.
\subsection{Deeper Context and Lessons}
\textbf{\paragraph{Discovering Segmentation Variables and the Skill of the Researcher}}
Market segmentation is not a simple task of just applying existing variables. As mentioned in this lecture, using techniques like \textbf{Factor Analysis} and \textbf{Cluster Analysis} to discover effective segments (ways of slicing consumers) in the market for one's own product is itself a crucial skill (craft) for marketing researchers and new product developers. Discovering groups with hidden needs provides clues for new product development.
\textbf{\paragraph{The 'Usability' of Demographic Variables: What's Behind It}}
The lecture emphasized that demographic variables are frequently used because they are '\textbf{easy to measure}' (user-friendly). In a survey, asking 'What is your age?' yields a clear answer (regardless of whether the respondent is comfortable answering). However, asking 'Are you an indoor or outdoor person?' may cause the respondent to hesitate ('What's the standard?' 'I stayed home this week, but I usually like being outside'), making it difficult to obtain clear data. This practical 'ease of measurement' supports the use of demographic variables.
\textbf{\paragraph{The Influence of 'Reference Groups' (Psychographic Variable)}}
One psychographic variable mentioned was the influence from groups to which an individual belongs (or wishes to belong). Especially, segments strongly influenced by a \textbf{Reference Group} (e.g., the high school girl example from the lecture) exhibit conforming behavior: 'I want it because everyone has it.' On the other hand, a segment that seeks differentiation, 'I don't want it because everyone has it,' also exists. This 'thinking based on relationships with others' can also be an important psychographic variable in determining product concepts.
\textbf{\subsubsection{AI Supplement: Expansion of Key Points}}
In this lecture, the 'variables' for segmenting the market were explained in detail in the B2C and B2B contexts. However, segmentation (S) is only the first stage of the 'STP' marketing strategy process. The lecture text had limited mention of what is done 'after' segmentation, so we will supplement here with the concept of 'Targeting'.
\textbf{Targeting Strategy}:
After clarifying the market structure through segmentation, the company must decide which segment(s) to select as its target. This is called targeting. The main strategic patterns are as follows:
\begin{itemize}
	\item \textbf{Undifferentiated Marketing}:
	      Ignoring the differences between segments and approaching the entire market with a single product and marketing mix (mass marketing).
	\item \textbf{Differentiated Marketing}:
	      Selecting multiple segments and developing different products and marketing mixes for each. The B2B example in the lecture (using both a sales team and a call center) and the watch example (dividing by men and women) are close to this.
	\item \textbf{Concentrated / Niche Marketing}:
	      A strategy of concentrating management resources on a specific single (or few) segment(s). Also known as a niche market strategy.
\end{itemize}
The 'variables' learned in this lecture serve as important decision-making material for adopting any of these targeting strategies.
\subsection{Conclusion}
Market segmentation is an essential strategic approach for building a competitive advantage in a diversifying market. This lecture showed that the 'criteria (variables)' used in its practice are extremely important.
In the consumer market (B2C), one uses \textbf{demographic variables} that are easy to measure, \textbf{psychographic variables} that capture the consumer's inner world, and \textbf{behavioral variables} directly linked to actual purchasing behavior. In the industrial goods market (B2B), on the other hand, the customer company's \textbf{demographics (size and industry)} and \textbf{operating characteristics (purchasing process and customization demand)} are the main variables.
The practical lessons from this lecture are: first, that simplistic variables (e.g., 'gender' for watches) are not always the optimal segmentation axis, and it is necessary to identify the true driver (e.g., 'outdoor orientation'). Second, especially in B2B, the results of segmentation are a crucial management decision directly linked to the \textbf{sales structure and the organizational structure itself}.
\subsection{Key Terminology}
Factor Analysis, Cluster Analysis, Demographic Variables, Psychographic Variables, Behavioral Variables, Reference Group, Heavy User, Light User, Industrial Goods (B2B) Segmentation, Demographic Variables (B2B), Operating/Behavioral Variables (B2B), Standardized Product
\subsection{Comprehension Check Quiz}
\begin{enumerate}
	\item What statistical method do marketing researchers use to extract common 'latent factors' (e.t., emphasis on skincare) underlying diverse variables?
	\item What statistical method is used to classify consumers in the market into 'multiple groups (clusters) with high similarity' based on common factors (like those from factor analysis) to visualize them?
	\item In B2C segmentation, what is the main reason demographic variables like 'age' and 'income' are used more often in practice than other variables (like psychographic)?
	\item While demographic variables are 'easy to measure', what variables capture the internal characteristics (e.g., eco-conscious) linked to the 'why' (motivation) of consumer choice, which demographics alone cannot?
	\item What strategic lesson about segmentation does the outdoor watch case (launched for men but sold to women) illustrate?
	\item In the high school girl example from the lecture, when consumers exhibit conforming behavior like 'I want it because everyone has it', what is the group thought to be strongly influencing their purchase decision called?
	\item What is the strategic advantage of behavioral variables like 'purchase frequency' and 'benefits sought' compared to other variables (psychographic/demographic)?
	\item In the canned coffee market, what is the strategic aim of segmenting the market by purchase frequency (a behavioral variable) and concentrating resources on a specific segment (e.g., heavy users)?
	\item A strategy to strengthen sales at discount stores for heavy users is segmentation based on which specific behavioral criterion?
	\item In travel services, what is the purpose of segmenting the market based on the 'benefits' consumers seek (e.g., 'relaxation' vs. 'knowledge acquisition')?
	\item Why is it necessary for 'industrial goods market (B2B)' segmentation, where the customer is a company, to use different variables (e.g., industry type) than the consumer market (B2C)?
	\item In B2B segmentation, classifying customers by demographic variables like 'company size' or 'industry type' leads to what kind of strategic decisions (e.g., sales resource allocation)?
	\item In B2B segmentation, even if large enterprises have the same demographic (size), what is the strategic reason for further segmenting them by operating/behavioral variables like 'degree of customization demand'?
	\item In B2B, why is it a rational response to set up a 'dedicated sales team' for the 'large enterprise, high customization demand' segment?
	\item In B2B, why is it a rational response to handle the 'SME, low customization demand' segment with a 'call center' (selling standardized products)?
\end{enumerate}
\subsubsection*{Answer Key}
1. Factor Analysis.
2. Cluster Analysis.
3. Because they are objective and 'easy to measure' (user-friendly).
4. Psychographic Variables.
5. That simplistic demographic variables (gender) are not always optimal, and it's necessary to identify the true driver (psychographic variable = outdoor orientation).
6. Reference Group.
7. Because they are based on actual behavior, they are easy to link directly to sales promotion and channel strategies.
8. To efficiently execute channel strategies (e.g., strengthening discount store presence) tailored to that segment's characteristics.
9. Purchase frequency (or usage frequency).
10. To develop optimized products (package tours) and advertising appeals for each segment seeking different benefits.
11. Because the purchasing motives and processes of customers (companies) are fundamentally different from those of general consumers.
12. To build a sales structure (e.g., establishing a large enterprise team) appropriate for the segment's size and characteristics.
13. Because the products and sales support processes (organizational structure) need to be varied according to demand (standardized vs. customized).
14. Because they are key customers expected to contribute significant profit and require thorough support (relationship building).
15. Because the customer count is large and standardized products are sufficient, requiring an efficient sales response (cost reduction).
\section{Setting the Target Market and Combining Multiple Markets: Criteria for Evaluating Market Segments and a Comparative Analysis of Multi-Market Strategies (Differentiated vs. Concentrated)}
\subsection{Introduction}
In the previous sections, we learned about 'market segmentation' (dividing the market into meaningful customer groups) and its criteria. However, strategy is not complete just by segmenting. The company must select which market(s) to enter from among the divided segments.
These lecture notes focus on 'targeting' (setting the target market), the step that follows segmentation. The purpose is to organize and deepen understanding of the theoretical frameworks and examples for how to evaluate segmented segments, and what strategic approaches a company should take toward the selected single or multiple markets.
\subsection{Key Concepts and Points}
\subsubsection{Evaluating Market Segments}
When selecting a target market from various segmented segments, each segment must first be evaluated. In modern marketing, it is common to set more accurate segments by combining multiple variables (e.g., age, investment pattern, risk tolerance asked when opening a bank account), not just a single criterion (e.g., age).
However, this involves a \textbf{trade-off}. If segmentation is taken too far, the degree of 'fit' with customer needs increases, but the target market size becomes smaller. Therefore, companies must identify an 'appropriate market segment'.
\paragraph{Evaluation by Economic Attractiveness}
The most important criterion for evaluating a segment is its '\textbf{Economic Attractiveness}'. Economic attractiveness is judged comprehensively from the following factors.
\begin{enumerate}
	\item \textbf{Market Scale (Substantiality)}:
	      The segment must secure a certain market size sufficient for the company to turn a profit. This is essential for enjoying \textbf{economies of scale}.
	      \textbf{Economies of scale} refer to the effect where the production cost per unit (especially \textbf{fixed costs} like factory construction) decreases as production volume increases. If the market size is too small, this effect cannot be obtained, and per-unit costs remain high. Marketing expenses like advertising and channel development also become inefficient if the target audience is too small.
	\item \textbf{Market Growth}:
	      Even if the current market size is small, high growth potential, where users are expected to increase in the future, is also an attractive factor. For example, the overseas travel market in the 1980s was initially small, but travel agencies, anticipating rapid growth, developed various products to meet it. Early entry into a growing market may allow the company to acquire a '\textbf{first-mover advantage},' establishing a strong position synonymous with that product category.
	\item \textbf{Competition \& Profitability}:
	      An attractive market (large scale and high growth) is inevitably attractive to \textbf{competitors} as well, and many companies are likely to enter. As a result, there is a risk of intensifying competition for market share, leading to escalating investments in new product development and advertising, and falling into a war of attrition (\textbf{Red Ocean}) where profits are hard to come by. Therefore, it is necessary to analyze the competitive situation and assess whether sustainable profitability can be expected.
\end{enumerate}
\paragraph{Fit with Own Company Resources}
No matter how high the economic attractiveness, entry will not succeed if the company lacks the capability (\textbf{management resources}) to serve that segment. It is essential to calmly evaluate the company's financial strength, R\&D capabilities, production capacity, existing channel networks, and brand value, and based on the company's strengths and weaknesses, examine whether it is a segment where a '\textbf{competitive advantage}' can be established.
\subsubsection{Factors Determining the Degree of Market Segmentation}
\textbf{Derek F. Abell} discussed the definition of the business domain in his 1980 research. This lecture borrows from that discussion to consider factors that determine 'to what degree' a company should segment the market.
\begin{itemize}
	\item \textbf{Factors favoring a lower degree of segmentation (targeting a broad market)}:
	      \begin{itemize}
		      \item Customer \textbf{price sensitivity} is high, and price is the main selection criterion.
		      \item Customer emphasizes only the \textbf{basic functions} of the product (e.g., a mobile phone just needs to make calls).
		      \item The management resources (channels, advertising, etc.) for serving multiple segments have high similarity, allowing for significant \textbf{experience effects} or \textbf{scale effects}.
		      \item The company possesses abundant available management resources.
	      \end{itemize}
	\item \textbf{Factors favoring a higher degree of segmentation (targeting a narrow market)}:
	      \begin{itemize}
		      \item The market has matured, and customers now emphasize \textbf{additional/secondary functions}.
		      \item Segment characteristics differ greatly, making the cost of serving each one individually high.
		      \item Customer's product judgment (knowledge) is high, and they can understand advanced product specifications and the manufacturer's efforts. (e.g., high-end customers for high-performance computers).
		      \item Customer's \textbf{purchase involvement} is high, and they tend to enjoy the process of comparing and selecting product varieties itself.
	      \end{itemize}
\end{itemize}
\subsubsection{Multi-Market Response Strategies (Targeting Strategies)}
A company's strategies for responding to the market are broadly divided into 'differentiated' and 'concentrated' types.
\paragraph{Differentiated Marketing}
A strategy of serving multiple, ideally all, market segments within a single product business, using different products or marketing mixes for each. Also called a '\textbf{Full Coverage Strategy},' it is often adopted by \textbf{large corporations} with abundant management resources.
\begin{itemize}
	\item \textbf{Drawbacks}: Costs increase significantly to meet all needs. Specific examples include R\&D costs, production process setup costs for each product, marketing research costs for each segment, and the complexity of inventory management.
	\item \textbf{Key Point}: It is crucial to compare projected sales increases with projected cost increases to find the balance that maximizes profit.
\end{itemize}
\paragraph{Concentrated Marketing}
A strategy of focusing management resources on a single, specific product segment within a single product business.
\begin{itemize}
	\item \textbf{Objective}: To gain an overwhelming \textbf{competitive position} in that specific segment.
	\item \textbf{Characteristics}: The company can accumulate extensive knowledge about that segment, making it suitable for \textbf{SMEs} with limited resources.
	\item \textbf{Risks}: If the segment's attractiveness is lost due to competitor entry or changes in consumer demand (e.g., a trend toward natural makeup in the eye makeup market), the company cannot compensate with other businesses, placing the entire firm's survival in jeopardy.
\end{itemize}
\subsubsection{Targeting in a Multi-Business Portfolio}
When a company has multiple product businesses (e.g., food and cosmetics), targeting patterns are further divided into two.
\begin{enumerate}
	\item \textbf{Common Segment Strategy}:
	      A pattern of pursuing a common product segment across all product businesses.
	      \begin{itemize}
		      \item Example (Concentrated): \textbf{Dior} and \textbf{Gucci} focus on 'luxury-oriented' customers across all businesses, including clothing, bags, and accessories.
		      \item Example (Differentiated): A major confectionery company targets all age groups, from children to middle-aged and older adults (full coverage).
	      \end{itemize}
	      \textbf{Advantage}: Allows for use of common channels and advertising concepts, making it easier to achieve overall \textbf{corporate image consistency}.
	\item \textbf{Separate Segment Strategy}:
	      A pattern of selecting different, optimal segments for each product business, considering economic attractiveness and resource constraints.
	      \begin{itemize}
		      \item Example: \textbf{Suntory} adopts a \textbf{differentiated} strategy (diverse customer base) in its food and beverage business, while adopting a \textbf{concentrated} strategy ('women in their 40s and older') in its cosmetics business.
	      \end{itemize}
	      \textbf{Characteristics}: At first glance, resources seem dispersed and inefficient, but it is a strategic choice based on the growth stage of each business. For example, it may be envisioned that knowledge accumulated in other businesses will be leveraged to diversify (move to full coverage) the cosmetics business in the future.
\end{enumerate}
\subsection{Application and Case Analysis}
\begin{description}
	\item[Bank Account Opening] Identifying segments by combining \textbf{multiple variables} such as customer age, income, and risk tolerance for investment, then developing and recommending financial products tailored to each segment.
	\item[Overseas Travel in the 1980s] An example where travel agencies developed various products in response to high \textbf{market growth}, even though the market size at the time was small.
	\item[High-Performance Computers] Because \textbf{high-judgment, high-end customers} (high involvement) who can understand the complex product specifications exist, the company \textbf{increases the degree of segmentation} to meet specialized needs.
	\item[Eye Makeup Market] A company specializing in mascara and eyeliner, focusing on a single segment of low-price-oriented young women, is a classic example of \textbf{concentrated marketing}.
	\item[Dior / Gucci] An example of having multiple businesses (apparel, bags, etc.) but taking a \textbf{concentrated} strategy toward a \textbf{common segment} ('luxury-oriented') for all of them.
	\item[Suntory] An example of a \textbf{separate segment strategy}, using different targeting strategies within its portfolio, such as the food business (differentiated) and the cosmetics business (concentrated on women 40+).
\end{description}
\subsection{Deeper Context and Lessons}
\textbf{\paragraph{The Segmentation Trade-off}}
As mentioned at the beginning of the lecture, 'narrowing the target' by increasing segmentation criteria carries risks. Pursuing 'fit' with needs too aggressively can make the target market too small, rendering it economically unviable. It is crucial in practice to always be aware of this \textbf{trade-off}.
\textbf{\paragraph{'First-Mover Advantage' and Market Future}}
The strategy of entering early in anticipation of \textbf{market growth} to build a 'synonymous' position (first-mover advantage) is attractive. However, this presumes that 'the market will reliably continue to grow,' and misreading the market poses a significant risk.
\textbf{\paragraph{The Balance Between 'What Can Be Done' and 'What Should Be Done'}}
The importance of judging on the two axes of 'economic attractiveness' (what should be done) and 'own company resources' (what can be done) in segment evaluation was emphasized. Even if a market is attractive, it is not the market to enter if it does not fit the company's strengths. Before engaging in a war of attrition with competitors in a \textbf{Red Ocean}, one must calmly assess if one's own \textbf{competitive advantage} can be established.
\textbf{\paragraph{Lecture Conclusion: Application to Daily Life}}
The way of thinking about market segmentation learned in this lecture is applicable not only in business scenes but also in daily interpersonal relationships (such as customer service). The perspective of not seeing others uniformly, but assessing their 'type' (characteristics and needs) and changing one's response (interaction) accordingly, is rich with practical insights.
\textbf{\subsubsection{AI Supplement: Expansion of Key Points}}
In this lecture, 'differentiated marketing' and 'concentrated marketing' were explained in detail as targeting strategies. However, as a foundation of marketing strategy theory, there is one other main strategy that contrasts with these. That is '\textbf{Undifferentiated Marketing}'.
\begin{description}
	\item[\textbf{Undifferentiated Marketing}]
	      Also called \textbf{mass marketing}. This is a strategy that intentionally ignores the differences between segments identified by segmentation, and approaches the entire market (mass) with a single product and a single marketing mix (price, promotion, channel).
	      The former Ford Model T and Coca-Cola were representative examples, but it has become difficult to adopt in modern markets where consumer needs have diversified. However, it may still be effective for commodities (homogeneous goods) like salt or sugar, or in the introductory phase of a market. The 'differentiated' and 'concentrated' types learned in this lecture can be understood more deeply in contrast to this 'undifferentiated' type.
\end{description}
\subsection{Conclusion}
After visualizing the market structure through market segmentation, companies face the crucial decision of 'targeting' (selecting the target market). In this lecture, we learned that its evaluation criteria are composed of two aspects: '\textbf{Economic Attractiveness}' (market scale, growth, competition) and '\textbf{Fit with Own Company Resources}'.
We also understood, with reference to Abell's discussion, the need to adjust the degree (breadth) of segmentation according to customer price sensitivity and purchase involvement, and the existence of the large-corporation-style '\textbf{Differentiated Marketing}' (full coverage) and the SME-style '\textbf{Concentrated Marketing}' as specific response strategies.
The practical lesson from this lecture is that an 'attractive market' is not necessarily the 'optimal market for one's own company'. A strategic perspective is essential to calmly evaluate the risk of a \textbf{Red Ocean} and whether one's own \textbf{competitive advantage} can be established, and to select a market that generates sustainable profit.
\subsection{Key Terminology}
Derek F. Abell
\vspace{\baselineskip}

Economic Attractiveness, Economies of Scale, Fixed Costs, Variable Costs, Market Growth, First-Mover Advantage, Red Ocean, Management Resources, Competitive Advantage, Purchase Involvement, Differentiated Marketing, Full Coverage Strategy, Concentrated Marketing, Image Consistency
\subsection{Comprehension Check Quiz}
\begin{enumerate}
	\item When a company comprehensively evaluates a segment based on market size, growth, and competition, what is this evaluation criterion called?
	\item In segment evaluation, 'market scale' is emphasized because it allows the company to enjoy economies of scale. This is the effect where the per-unit cost of what declines as production increases?
	\item What is the strategic objective for a company entering a high 'market growth' segment early, like the 1980s overseas travel market?
	\item What is the advantage gained by entering a growing market early and becoming the 'synonymous' entity for that category?
	\item When entering a market with high economic attractiveness (large scale and high growth), what is the greatest risk (market state) that a company faces?
	\item Even if a market has high 'economic attractiveness', a company might abandon entry. This happens when what other crucial evaluation criterion is lacking?
	\item When customers only emphasize 'basic functions' or 'price', why is it more rational for a company to use a 'lower degree' of segmentation (target a broad market)?
	\item In a market like high-performance computers, where customer 'purchase involvement' and judgment are high, why is a strategy of using a 'higher degree' of segmentation (targeting a narrow market) effective?
	\item What is the strategy called where a large corporation with abundant resources serves multiple (ideally all) segments with different products and marketing?
	\item What is the biggest problem (cost) a company faces when executing Differentiated Marketing (Full Coverage Strategy)?
	\item What is the strategy called (e.g., eye makeup market) where an SME with limited resources concentrates those resources on a single specific market segment?
	\item While Concentrated Marketing can build high competitive advantage in a specific market, what significant management risk does it carry?
	\item What is the greatest strategic advantage for Dior or Gucci in adopting a common segment strategy ('luxury-oriented') across multiple businesses (clothing, bags, etc.)?
	\item Suntory's 'separate segment strategy', using different targeting for its food (differentiated) and cosmetics (concentrated) businesses, seems inefficient at first. What strategic rationale is it based on?
	\item (From AI Supplement) What is the strategy of intentionally ignoring segment differences and approaching the entire market with a single product (e.g., old Coca-Cola) called?
\end{enumerate}
\subsubsection*{Answer Key}
1. Economic Attractiveness.
2. Fixed Costs.
3. To acquire a 'first-mover advantage' in anticipation of future market expansion.
4. First-mover advantage.
5. 'Red Ocean' saturation, where many competitors enter, leading to a war of attrition.
6. Fit with own company resources (or the establishment of a competitive advantage).
7. Because there is no need for detailed responses, and economies of scale or experience effects can be enjoyed.
8. Because customers value additional functions and advanced specifications, responding to them leads to a competitive advantage.
9. Differentiated Marketing (Full Coverage Strategy).
10. A significant increase in costs, such as R\&D, production, and inventory management.
11. Concentrated Marketing.
12. The risk that the company's
 entire survival is jeopardized if that specific segment loses its attractiveness.
13. It is easy to achieve 'image consistency' for the entire corporation.
14. It is based on selecting the optimal strategy for each business, considering its growth stage and resource constraints.
15. Undifferentiated Marketing (Mass Marketing).
\end{document}
\documentclass[uplatex,a4j,12pt,dvipdfmx]{jsarticle}
\usepackage{amsmath,amsthm,amssymb,bm,color,enumitem,mathrsfs,url,epic,eepic,ascmac,ulem,here,ascmac}
\usepackage[letterpaper,top=2cm,bottom=2cm,left=3cm,right=3cm,marginparwidth=1.75cm]{geometry}
\usepackage[english]{babel}
\usepackage[dvipdfm]{graphicx}
\usepackage[hypertex]{hyperref}
\title{Marketing Chapter 3 Lecture Notes}
\author{M. O.}
\date{\today}
\begin{document}
\maketitle
\tableofcontents
\section{The Meaning of Market Segmentation: An Analysis of the Theory and Practical Application of Segmentation}
\subsection{Introduction}
This note's \textbf{primary objective} is to analyze the \textbf{theoretical framework} and \textbf{practical application} of \textbf{market segmentation}, based on a lecture given as part of the 'MBA \textbf{Marketing Strategy}' \textbf{course}. Market segmentation is a \textbf{marketing activity of paramount importance} for companies to establish a \textbf{competitive advantage} and acquire \textbf{sustainable profits} and \textbf{market share} in the modern competitive environment. This note will scrutinize the lecture transcript, organizing and analyzing its \textbf{key concepts}, \textbf{practical conditions}, and \textbf{types of segmentation based on product characteristics} from a professional perspective.
\subsection{Key Concepts and Points}
The central focus of this lecture is placed on the definition of \textbf{Market Segmentation} and its \textbf{practical significance}.
\subsubsection{Definition and Purpose of Market Segmentation}
\textbf{Market segmentation} refers to dividing a market into groups of consumers who are '\textbf{similar}' in their '\textbf{perception of products}', '\textbf{values}', '\textbf{purpose of use}', and '\textbf{purchasing behavior}'. The resulting collection of consumers from this division is expressed as a \textbf{segment}.
The \textbf{purpose} for which companies perform market segmentation lies in maximizing \textbf{customer satisfaction} by offering products that \textbf{fit} the \textbf{specific demands} within those segments, thereby achieving \textbf{improved market share} and \textbf{high profits}. This activity is positioned as an \textbf{action to gain an advantage in competition with other companies} in the market.
\subsubsection{Necessary Conditions for Market Segmentation}
For a company to succeed in segmentation and gain business benefits, it must establish segments that meet the following \textbf{conditions}.
\begin{itemize}
	\item \textbf{Measurability}: The chosen axis for extraction must not be vague and must be \textbf{quantitatively graspable}.
	\item \textbf{Substantiality}: The segment must have a certain \textbf{sales scale} and show \textbf{promise of bringing profit} to the company.
	\item \textbf{Stability}: It must possess \textbf{stability that does not change over time}, rather than being a temporary boom.
	\item \textbf{Accessibility}: The company must be able to \textbf{effectively access} the \textbf{designated segment} (to deliver products or convey messages).
	\item \textbf{Actionability}: The company must be \textbf{able to execute} \textbf{suitable marketing activities} (product development, advertising, etc.) for that segment.
\end{itemize}
\subsection{Application and Case Analysis}
This lecture provides a detailed explanation of segmentation criteria based on \textbf{product characteristics} (presence or absence of superiority).
\subsubsection{Segmentation Based on Horizontal Differentiation}
\textbf{Horizontal differentiation} refers to characteristics where product superiority is \textbf{not clear} (e.g., strength of scent, richness of flavor), forming diverse segments as a \textbf{matter of consumer preference}.
\begin{itemize}
	\item \textbf{Case (Ice Cream)}: A \textbf{comparison} between '\textbf{light}' flavors (e.g., \textbf{Soh}) and '\textbf{rich}' flavors (e.g., \textbf{Super Cup}) illustrates that the consumer's \textbf{Ideal Point} differs by segment (e.g., male consumers, female consumers).
	\item \textbf{Analysis}: Based on this \textbf{horizontal differentiation}, developing a product concept that is \textbf{closest} to the \textbf{ideal point} of each segment can maximize \textbf{customer satisfaction} within that segment and establish a competitive advantage. This is concluded to be one of the \textbf{optimal segmentation criteria} for creating \textbf{diverse consumer segments}.
\end{itemize}
\subsubsection{Segmentation Based on Vertical Differentiation}
\textbf{Vertical differentiation} refers to characteristics where product superiority is \textbf{clearly determined} (e.g., \textbf{quality}, \textbf{capacity}, \textbf{power consumption}). Normally, consumers will choose the higher quality product if the price is the same.
\begin{itemize}
	\item \textbf{Case (Refrigerators)}: Consumers select products based on a combination of \textbf{vertical characteristics} like \textbf{capacity} (large) and \textbf{power consumption} (low). However, a situation arises where the \textbf{combination of vertical characteristics} (the \textbf{weighting} placed on performance) \textbf{scatters diversely} according to '\textbf{lifestyle}' (usage scenarios or segment attributes), such as '\textbf{single-person or family use}'.
	\item \textbf{Analysis}: Companies are required to introduce \textbf{diverse products} tailored to segments with \textbf{diverse preference patterns} (\textbf{prioritization} of performance), rather than segmenting by a single vertical characteristic. This suggests that \textbf{substantial segments} are formed even with vertical differentiation, based on the \textbf{multi-axial combination of consumer situations and preferences}.
\end{itemize}
\subsection{Deeper Context and Lessons}
This section extracts peripheral information to deepen the understanding of the lecture's main points, as well as episodes that enriched the instructor's discussion.
\paragraph{Ex-post Changes to Segment Concepts due to Corporate Marketing Activities}
The lecture mentioned situations where companies, influenced by \textbf{existing marketing activities} and \textbf{resource allocation}, \textbf{later change their product concepts or target segments}. For example, even if a product was released for all customer demographics, a company might later \textbf{narrow the age range} to attach an impactful advertising message, or change the \textbf{color palette or design} to match the target demographic. Cases were also introduced where \textbf{product concepts} were \textbf{altered to fit the channel}, taking into account close relationships with existing \textbf{distribution channels (e.g., convenience stores, discount stores)} and that channel's \textbf{customer base} (e.g., \textbf{young consumers}). This suggests that \textbf{segment decisions are not static}, but demand \textbf{dynamic alignment} with a company's \textbf{internal resources} and \textbf{external environment (channel conditions)}.
\paragraph{Shifting Ideal Points through Advertising Communication}
In the analysis of \textbf{horizontal differentiation}, it was pointed out that a consumer's \textbf{ideal point} can be \textbf{shifted upward} by \textbf{advertising communication}. Messages like '\textbf{Be a little kinder to yourself}' can influence consumer \textbf{psychology} and \textbf{social trends}, consequently changing the \textbf{ideal point of preference for a product} (e.g., richness, sweetness). This episode illustrates a perspective of \textbf{proactive marketing}, rather than \textbf{passive} marketing, where segment \textbf{demand} and \textbf{preference} can be \textbf{created or guided} by the \textbf{company's efforts (promotional activities)}.
\paragraph{Lifestyle and Weighting of Vertical Characteristics}
As a background for why \textbf{vertical characteristics} (e.g., refrigerator capacity and power consumption) are selected on a \textbf{multi-axial} basis, it was emphasized that \textbf{lifestyle} (e.g., '\textbf{single-person or family use}') influences the \textbf{weighting given to attributes}. This suggests that combining not just \textbf{demographic attributes} or \textbf{geographic attributes}, but also more \textbf{behavioral and psychological} \textbf{psychographic attributes} (e.g., lifestyle, values) as segmentation criteria is \textbf{important} in the \textbf{market segmentation of vertical products}.
\subsubsection{Supplement by AI: Expanding on Key Points}
In discussing the lecture's theme of \textbf{market segmentation}, \textbf{Targeting} and \textbf{Positioning}, which are \textbf{indispensable components} of the \textbf{STP strategy}, were omitted. They are supplemented here.
\paragraph{Integration of Targeting and Positioning}
The \textbf{STP strategy} is a \textbf{basic framework for marketing strategy}, taking the initials of \textbf{Segmentation}, \textbf{Targeting}, and \textbf{Positioning}.
\begin{itemize}
	\item \textbf{Targeting}: This is the process of \textbf{selecting} an \textbf{attractive} segment as a \textbf{target} from the multiple segments extracted through market segmentation, one where the company's \textbf{resources}, \textbf{strengths}, and \textbf{competitive advantage} can be \textbf{best leveraged}. Not all segments that meet the segmentation conditions are targeted. The selection of a target segment considers \textbf{size}, \textbf{growth potential}, \textbf{competitive structure}, and \textbf{fit with corporate objectives}.
	\item \textbf{Positioning}: This is the activity of \textbf{designing} the \textbf{product concept} and \textbf{marketing mix} so that, \textbf{compared to competitors}, the \textbf{company's product} is \textbf{differentiated} with \textbf{unique value} \textbf{in the minds of the target segment's customers}. Whereas segmentation and targeting relate to '\textbf{who}', positioning relates to '\textbf{what value to provide}', and plays the role of \textbf{clearly linking the selected segment's demands} with the \textbf{company's value proposition}.
\end{itemize}
\textbf{Segmentation} is the \textbf{analysis} phase, while \textbf{Targeting} and \textbf{Positioning} are the \textbf{strategy decision} and \textbf{execution} phases. Understanding this \textbf{sequential flow (STP)} in an integrated manner maximizes the \textbf{practical value} of market segmentation.
\subsection{Conclusion}
This note has examined the \textbf{definition}, \textbf{conditions for establishment}, and \textbf{application of segmentation based on product characteristics} (horizontal and vertical) of market segmentation. Market segmentation is a \textbf{foundational strategy} for companies to build a \textbf{competitive advantage} through \textbf{high customer satisfaction} by developing \textbf{products and marketing activities deeply suited} to \textbf{specific consumer segments}.
The \textbf{practical lessons} derived from the analysis, especially from the \textbf{Deeper Context}, are as follows:
\begin{itemize}
	\item \textbf{Strategic Flexibility}: Segmentation and targeting are \textbf{not one-time decisions} but \textbf{dynamic processes} that should be \textbf{flexibly revised} according to the \textbf{company's resource situation} and \textbf{distribution channels}.
	\item \textbf{Proactive Demand Creation}: \textbf{Communication activities} such as advertising not only appeal to existing demand but also have the power to \textbf{actively shape and guide consumer ideal points} and \textbf{preferences}. Therefore, they should be utilized as a \textbf{strategic tool for creating a segment's latent needs}.
	\item \textbf{Integrated Understanding of STP}: To maximize the results of segmentation, it is essential to promote \textbf{all phases} of the \textbf{STP strategy} \textbf{consistently}: not just segmentation (S), but also the \textbf{selection} of target markets (T) and \textbf{differentiation from competitors (P)}. This leads to \textbf{sustainable profit acquisition}.
\end{itemize}
These lessons suggest that MBA learners should understand market segmentation not as a \textbf{static market analysis}, but as \textbf{strategic decision-making that responds} to \textbf{corporate resources} and \textbf{changes in the external environment}.
\section{Criteria for Market Segmentation: A Systematic Organization of Key Segmentation Standards}
\subsection{Introduction}
In modern marketing strategy, accurately grasping diversifying customer needs and identifying the most effective targets for a company's resource investment is a fundamental challenge for establishing a competitive advantage. The basic approach to meeting this challenge is $\textbf{market segmentation}$.
This lecture explained the main variables used to group the consumers who make up a market based on specific criteria. The purpose of this note is to systematically organize the segmentation criteria (variables) for $\textbf{consumer goods markets}$ and $\textbf{industrial goods (BtoB) markets}$ presented in the lecture, and to analyze their characteristics and application examples. Furthermore, it aims to deepen the understanding of the practical significance of market segmentation by considering practical issues derived from the lecture's context and supplementary key concepts not mentioned in the main discussion.
\subsection{Key Concepts and Points}
The lecture described market segmentation as the process of dividing an entire heterogeneous market into homogeneous subgroups (segments) that share specific needs or characteristics. In this process, the selection of appropriate $\textbf{segmentation variables (criteria)}$ is essential.
\subsubsection{Analytical Methods for Market Segmentation}
The lecture mentioned that the following statistical methods are effective for discovering segmentation variables and identifying segments.
\begin{description}
	\item[Factor Analysis] A method for finding common underlying $\textbf{factors}$ behind many measurable variables (question items). For example, it is used to extract key factors such as 'skin care', 'low price', and 'knowledge' in cosmetics purchasing behavior.
	\item[Cluster Analysis] A method for classifying large amounts of data into several $\textbf{clusters (groups)}$ based on similarity according to some common factor. It is used to form consumer groups with high similarity.
\end{description}
\subsubsection{Segmentation Criteria for Consumer Goods Markets}
It was explained that three main variables are used in the segmentation of consumer goods markets (BtoC).
\paragraph{Demographic Variables}
Criteria based on an individual's objective and predetermined characteristics.
\begin{itemize}
	\item \textbf{Main variables:} Gender, age, household size, family life cycle, income, education level, occupation, nationality, etc.
	\item \textbf{Characteristics:}
	      \begin{enumerate}
		      \item Compared to other variables (e.g., lifestyle), they are easy to measure and 'easy to use'. Survey respondents also find them easy to answer, facilitating data collection.
		      \item They tend to be linked with consumer $\textbf{wants (needs)}$ and $\textbf{degree of product use}$ (e.g., differences in car types between rural and urban areas).
	      \end{enumerate}
\end{itemize}
\paragraph{Psychographic Variables}
Criteria based on a consumer's internal characteristics.
\begin{itemize}
	\item \textbf{Main variables:} $\textbf{Lifestyle}$ (e.g., outdoor-oriented, indoor-oriented), $\textbf{personality}$, $\textbf{values}$ (e.g., eco-conscious, brand-conscious), awareness of $\textbf{reference groups}$ (conformity or differentiation), etc.
	\item \textbf{Characteristics:}
	      \begin{enumerate}
		      \item They are useful for understanding the motivation of 'why' a product is chosen, which cannot be captured by demographic variables alone.
		      \item Objective measurement is difficult, and attempts are made to quantify them through surveys on attitudes, etc., using questionnaires.
	      \end{enumerate}
\end{itemize}
\paragraph{Behavioral Variables}
Criteria based on a consumer's knowledge, attitude, usage, or response to a product or service.
\begin{itemize}
	\item \textbf{Main variables:}
	      \begin{itemize}
		      \item \textbf{Purchase frequency/Usage rate:} $\textbf{Heavy users}$, middle users, light users.
		      \item \textbf{Brand loyalty:} Degree of loyalty to a specific brand (e.g., loyal customers, switchers).
		      \item \textbf{Benefits sought:} The primary purpose sought from the product (e.g., 'rest', 'acquiring knowledge', 'time with family' for travel).
		      \item \textbf{Product knowledge}, $\textbf{attitude toward the product}$ (positive, negative, etc.).
	      \end{itemize}
	\item \textbf{Characteristics:} These are the criteria most directly linked to purchasing behavior and are easy to connect to specific marketing measures (e.g., discounts for heavy users).
\end{itemize}
\subsubsection{Segmentation Criteria for Industrial Goods Markets (BtoB)}
Segmentation is also important when the customer is a company (BtoB), and two main groups of variables are used.
\begin{description}
	\item[Demographic variables] Objective attributes of the customer company.
	      \begin{itemize}
		      \item Ex: Company size (large corporation, SME), region (metropolitan, rural), industry type.
	      \end{itemize}
	\item[Behavioral characteristics variables] The customer company's purchasing process and required characteristics.
	      \begin{itemize}
		      \item Ex: Size and structure of the $\textbf{buying center}$, $\textbf{decision-making process}$ (sole decision by a key person or consensus among multiple departments), urgency of purchase, degree of $\textbf{customization demand}$.
	      \end{itemize}
\end{description}
\subsection{Application and Case Analysis}
The following examples were cited in the lecture to understand the segmentation criteria described above.
\begin{description}
	\item[Golf Clubs (Assumption of composite criteria)]
	      When envisioning a target customer, 'university students in their 20s' are excluded, and $\textbf{demographic variables}$ (age, income, gender) and $\textbf{psychographic variables}$ (attitude toward the activity) are considered in combination, such as '30s or older', 'with a certain job title (income)', 'male' (due to the prevalence of 'social' golf), and 'positive' about the sport of golf.
	\item[Automobiles (Demographic variable: Region)]
	      The types of cars that sell (e.g., larger cars, family-oriented cars in rural areas) differ clearly based on the regional variable of $\textbf{rural}$ vs. $\textbf{urban}$ areas. This is also linked to other demographic variables such as household size and housing space.
	\item[Outdoor Watches (Psychographic variable: Lifestyle)]
	      A watch initially developed for 'young men' (demographic variable) was found to also have demand among 'outdoor-oriented' $\textbf{women}$. This shows that the $\textbf{psychographic variable (lifestyle)}$ of 'outdoor orientation' was a more appropriate segmentation criterion for this product than the criterion of gender.
	\item[Canned Coffee (Behavioral variable: Usage rate)]
	      In the canned coffee or vegetable juice markets, if a segment of $\textbf{heavy users}$ is identified, the marketing strategy might involve 'box sales' such as '6-packs' instead of 'single-item sales', and resource concentration on specific sales channels like $\textbf{discount stores}$.
	\item[Apparel (Behavioral variable: Loyalty)]
	      To retain a customer segment with high $\textbf{loyalty}$ to a specific brand, measures such as 'discount rates exclusively for loyal customers' or 'members-only advance sales' are effective.
	\item[Travel (Behavioral variable: Benefits sought)]
	      In travel services, segments are divided based on the $\textbf{benefits sought}$ by consumers (e.g., 'time with family', 'rest', 'acquiring knowledge'), and package tours and advertisements optimized for each are developed.
	\item[Industrial Goods (BtoB: Demographics $\times$ Behavioral characteristics)]
	      An example in BtoB of segmenting customers on the two axes of 'company size' (demographic) and 'customization demand' (behavioral).
	      \begin{itemize}
		      \item \textbf{Large corporation and high customization demand:} A dedicated $\textbf{sales team}$ is established to provide thorough support, as they contribute significantly to profit.
		      \item \textbf{Small-to-medium enterprise and low customization demand (numerous):} A $\textbf{call center}$ is set up for efficiency, as they can be served with standard products.
	      \end{itemize}
	      Thus, a strategy of changing the organizational structure and sales system for each segment is adopted.
\end{description}
\subsection{Deeper Context and Lessons}
While the main part of the lecture focused on systematic theory, the anecdotes and the instructor's personal views shared in between contained important insights for applying theory to practice.
\subsubsection{Practical Perspectives from the Lecture Context}
\textbf{\paragraph{Detour Topic Name: Golf and 'Otsukiai' (Social) Culture}}
In the golf club example, there was a mention that 'many people play golf for social reasons' and 'men respond better than women'. This is not just a matter of gender (demographic variable) but suggests that the $\textbf{social context}$ of Japanese business culture, such as 'entertainment golf' or 'company golf', influences the demand structure for a specific product. Segmentation cannot be done without understanding this cultural background.
\textbf{\paragraph{Detour Topic Name: Golf Course Development and Environmental Awareness}}
The comment that 'some people cannot view the act of building a golf course positively because they value nature' indicates the importance of $\textbf{psychographic variables (values)}$. Attitudes toward the activity of golf itself are not uniform, and for segments with 'eco-conscious' values, an appeal for golf clubs may not resonate, or could even provoke a negative reaction.
\textbf{\paragraph{Detour Topic Name: The Practical Viewpoint of 'Ease of Answering' Surveys}}
As a reason for the frequent use of demographic variables, the $\textbf{practical aspect}$ of 'being easy to answer (in surveys)' was emphasized, in addition to the theoretical aspect of 'tending to link with wants'. The contrast with the ambiguity of the definition in the question 'Are you an indoor-oriented or outdoor-oriented person?' shows that in marketing research, the 'usability' and 'measurability' of data collection are just as important as theoretical precision.
\textbf{\paragraph{Detour Topic Name: High School Girls and Reference Groups}}
'High school girls' were mentioned as an example of psychographic variables, with the observation that 'it's an age where they like styles that everyone likes' and 'a time when they pay close attention to $\textbf{reference groups}$'. On the other hand, it was noted that a $\textbf{desire for differentiation}$, such as 'I don't want it because everyone has it', also exists. This indicates that in specific segments (age/gender), conflicting psychological forces of $\textbf{desire for conformity}$ and $\textbf{desire for differentiation}$ are strongly at play, which is a crucial insight for designing marketing messages.
\subsubsection{Supplement by AI: Expanding on Key Points}
The lecture provided a comprehensive explanation of the 'criteria (variables)' for market segmentation. However, certain points that are important for practical strategy formulation were mentioned only in a limited way, perhaps due to time constraints.
This concerns the overall picture of $\textbf{STP marketing}$, namely the linkage to $\textbf{Targeting}$ (deciding which segment to pursue after segmentation) and $\textbf{Positioning}$ (what kind of unique position to build in that market).
Two points are particularly important:
\begin{enumerate}
	\item \textbf{Cross-Segmentation (Combination of variables):}
	      Although the lecture explained each variable individually, in practice, it is rare to divide a market using a single variable (e.g., age only). For example, the segment 'men in their 30s' (demographic variable) alone is insufficient, as it includes both 'outdoor-oriented' (psychographic variable) and 'indoor-oriented' people, whose needs are completely different.
	      To derive a practical segment, $\textbf{cross-segmentation (combining multiple variables)}$—such as $\textbf{demographic} \times \textbf{psychographic} \times \textbf{behavioral}$ variables, as in the lecture's golf club example—is essential. Without this perspective, the resolution of the segment will be coarse and will not lead to effective measures.
	\item \textbf{Conditions for Effective Market Segmentation:}
	      Even if variables are set and the market is segmented, it is meaningless if that segment is not effective as a target for marketing strategy. Philip Kotler lists the following four (or five) conditions for effective segmentation.
	      \begin{itemize}
		      \item \textbf{Measurability:} The size and purchasing power of the segment can be measured.
		      \item \textbf{Accessibility:} The segment can be effectively reached and served (products and communication can be delivered).
		      \item \textbf{Substantiality:} The segment is large or profitable enough to be worth pursuing (it has sufficient scale or growth potential).
		      \item \textbf{Actionability:} Effective marketing programs can be designed for and implemented in the segment (the company's resources are sufficient).
	      \end{itemize}
	      The 'usability of demographic variables' mentioned in the lecture relates to 'Measurability', 'channel investment for heavy users' to 'Accessibility', and 'setting up sales teams in BtoB' to 'Actionability'. A perspective that systematically evaluates these conditions is an indispensable point for the purpose of segmentation.
\end{enumerate}
\subsection{Conclusion}
This note systematically organized the main segmentation criteria based on the 'Marketing Strategy' MBA lecture on $\textbf{market segmentation}$.
As shown in the lecture, $\textbf{demographic variables}$ (objective attributes), $\textbf{psychographic variables}$ (lifestyle and values), and $\textbf{behavioral variables}$ (usage rate and loyalty) are the main criteria for consumer goods markets. For industrial goods markets (BtoB), $\textbf{demographic variables}$ (company size and industry) and $\textbf{behavioral characteristics variables}$ (purchasing process and customization demand) are emphasized.
The practical lessons derived from the analysis of the main points and the 'Deeper Context and Lessons' section can be summarized as follows:
First, segmentation is not just a classification task but a process of understanding the 'living' context of customers, such as $\textbf{cultural backgrounds}$ (e.g., social golf) and the influence of $\textbf{reference groups}$.
Second, it is important to balance theoretical precision (e.g., psychographic variables) with $\textbf{practical measurability in research}$ (e.g., demographic variables).
And as pointed out in the AI supplement, the most important insight is that a practical target image only emerges by combining these variables in a composite manner through $\textbf{cross-segmentation}$, rather than using them in isolation. Ultimately, rigorously evaluating whether that segment is 'measurable', 'accessible', and 'substantial' for the company (the $\textbf{STP}$ perspective) is the key to successful market segmentation.
\section{Target Market Selection and Multi-Market Combination: An Analysis of Segment Evaluation Criteria and a Comparison of Multi-Market Strategies (Differentiated vs. Concentrated)}
\subsection{Introduction}
In modern marketing strategy, mass marketing, which targets an unspecified large number of consumers, is losing its effectiveness. As consumer needs diversify and individualize, it is essential for companies to establish a sustainable competitive advantage by appropriately segmenting the market and selecting a $\textbf{target market}$ that best fits their resources.
The purpose of this note is to organize and analyze the key points from the lecture concerning 'Target Market Selection and Multi-Market Combination'. Specifically, it will focus on the criteria for evaluating market segments, particularly \textbf{economic attractiveness} and fit with \textbf{management resources}. It will further examine the characteristics and trade-offs of the main strategies ( $\textbf{differentiated marketing}$ and $\textbf{concentrated marketing}$) that companies use when addressing multiple market segments.
\subsection{Key Concepts and Points}
\subsubsection{Evaluation Criteria for Market Segments}
Market segmentation is a process of subdividing the market by combining $\textbf{multiple variables}$ (e.g., age, risk tolerance, lifestyle), not just a single criterion. This allows for the identification of clearer customer groups, but it simultaneously creates a $\textbf{trade-off}$. As segmentation becomes more refined, the fit with customer needs improves, but the target $\textbf{market size}$ tends to shrink.
\subsubsection{Segments with Economic Attractiveness}
The segments a company should target are those with high $\textbf{economic attractiveness}$. This attractiveness is comprehensively evaluated based on the following factors:
\begin{enumerate}
	\item \textbf{Market Size:}
	      For a segment to possess a certain scale is essential for enjoying $\textbf{Economies of Scale}$. Economies of scale refer to the effect where the per-unit production cost (especially the proportion of $\textbf{fixed costs}$) decreases as production volume increases. Since initial investments (factory construction, employment, etc.) as $\textbf{fixed costs}$ are incurred regardless of production volume, profitability will be squeezed by high per-unit costs if sufficient sales volume cannot be expected. A certain market size is also necessary from the perspective of the efficiency of marketing investments, such as advertising expenses and $\textbf{channel development}$ costs.
	\item \textbf{Market Growth:}
	      Even if the current market size is small, a market expected to grow in the future (e.g., the overseas travel market in the 1980s) is attractive. By entering a growth market early, a company can enjoy a $\textbf{First-Mover Advantage}$ and potentially establish itself as the representative brand in that category.
	\item \textbf{Profitability and Risk:}
	      An attractive market is, inevitably, attractive to $\textbf{competitors}$ as well. If numerous companies enter as the market grows, competition will intensify, requiring additional investment (new product development, increased advertising costs) to capture share. As a result, there is a risk that the market will become a $\textbf{red ocean}$, and profitability commensurate with the investment will not be achieved.
\end{enumerate}
\subsubsection{Fit with Own Management Resources}
Even if a segment is economically attractive, it is necessary to rigorously evaluate whether it can be addressed with the company's own $\textbf{management resources}$ (funds, R\&D capabilities, production capacity, brand value, etc.) and whether a $\textbf{competitive advantage}$ can be established in that market. If competitors with strong financial power deploy aggressive marketing within these limited resources, the company is likely to lose the competition. Therefore, it is crucial to objectively assess one's own strengths and weaknesses and select a market where one's own company can win.
\subsection{Application and Case Analysis}
\subsubsection{Degree of Market Segmentation according to Abell}
Derek F. Abell provided guidelines on the extent to which a company should segment the market, based on customer characteristics and the company's resource situation.
\begin{itemize}
	\item \textbf{Customer Focus:} The degree of segmentation should be low (catering to a broad customer base) when customers have a high $\textbf{focus on price}$, medium when they prioritize $\textbf{basic functions}$, and high when they begin to prioritize $\textbf{secondary (additional) functions}$.
	\item \textbf{Product Judgment and Purchase Involvement:} In a market where customers' $\textbf{product judgment ability}$ is low (e.g., the nascent computer market), the need for segmentation is low, but as judgment ability increases, segmentation is required. Customers with high $\textbf{purchase involvement}$ tend to enjoy comparing products, thus requiring a higher level of segmentation.
	\item \textbf{Management Resources and Efficiency:} When the management resources required to serve multiple segments are similar, or when $\textbf{experience effects}$ or scale effects in advertising and channels are large, it is easier to increase the degree (or widen the scope) of segmentation.
\end{itemize}
\subsubsection{Strategies for Addressing Multiple Markets}
How a company responds to multiple market segments can be broadly classified into two strategies.
\paragraph{Differentiated Marketing}
\textbf{Differentiated marketing} is a strategy that, within a single product business (category), targets multiple different market segments, or even the entire market ($\textbf{full coverage strategy}$), and develops products and marketing mixes tailored to their respective needs. A confectionery manufacturer having a diverse product lineup from children to the middle-aged and elderly is one example of this, and it is primarily adopted by large corporations with abundant management resources.
The biggest problem with this strategy is $\textbf{increased costs}$. There is a risk that R\&D costs to support each segment, production process setup costs, marketing research costs for each segment, and the complexity of inventory management will exceed the increase in sales.
\paragraph{Concentrated Marketing}
\textbf{Concentrated marketing} is a strategy that, within a single product business, concentrates management resources on only one specific segment. An example is a company that specializes in a specific age group (young women) and a specific category (eye makeup).
This strategy is suitable for $\textbf{small and medium-sized enterprises (SMEs)}$ with limited management resources, allowing them to accumulate extensive knowledge about a specific segment and potentially build a strong $\textbf{competitive advantage}$ (a strong presence) in that market.
On the other hand, it harbors a high $\textbf{danger (risk)}$ that the entire business will suffer a severe blow if the \textbf{demand} in that segment \textbf{changes} (e.g., the eye makeup market shrinks due to a trend toward natural makeup) or if a powerful competitor enters.
\subsubsection{Segment Combinations across Multiple Businesses}
For companies with multiple product businesses (e.g., beverage business and cosmetics business), two patterns of segment combination are seen.
\begin{enumerate}
	\item \textbf{Pursuit of a common product segment:}
	      A pattern like Dior or Gucci, where the company operates multiple product businesses such as apparel, bags, and accessories, but targets a $\textbf{common product segment}$ ('luxury-oriented', high-price-range, high-brand-value) in all of them. This makes it possible to maintain brand image $\textbf{consistency}$ and increase the efficiency of channel and promotional activities.
	\item \textbf{Selection of different market segments:}
	      A pattern, as in the example of Suntory, where different segment strategies are adopted for each product business. While adopting a $\textbf{differentiated}$ (full coverage) strategy in the food business, they adopt a $\textbf{concentrated}$ strategy in the cosmetics business, targeting a specific segment of 'women in their 40s and older'. At first glance, this seems inefficient, but it is the result of reflecting the growth stage and resource accumulation status of each business. It is also suggested that in the future, they may leverage the customer knowledge of other age groups (10s-30s) accumulated in the food business to expand the cosmetics business into a differentiated strategy.
\end{enumerate}
\subsection{Deeper Context and Lessons}
\subsubsection{Supplementary Topics in the Lecture}
\paragraph{\textbf{Applying Segmentation Thinking in Daily Life}}
At the end of the lecture, the instructor suggested to the attendees that this way of thinking about market segmentation is applicable not only in business scenes but also in 'how to interact with people' in daily life. This refers to the mindset of understanding the type and characteristics of the other party (the target) and changing (optimizing) one's own response and communication methods accordingly, suggesting the versatility of 'customer understanding', which is at the core of marketing.
\subsubsection{\textbf{Supplement by AI: Expanding on Key Points}}
The lecture provided a detailed explanation of Segmentation and Targeting. However, there was a lack of clear reference to the indispensable third stage of the strategy: $\textbf{Positioning}$.
In the $\textbf{STP strategy}$ (Segmentation, Targeting, Positioning), positioning is the activity of clearly defining how one's own product differs from competing products for the selected target market, what unique value (benefit) it offers, and establishing a specific place in the customer's mind.
Even if a target market is selected (T) and resources are concentrated on it (concentrated marketing), if the company cannot make customers recognize how it is 'superior' to or 'different' from other competitors in that market, it will either fall into price competition or simply not be chosen by customers. Therefore, when evaluating the economic attractiveness of a segment, it is simultaneously crucial to consider 'whether the company can establish a feasible positioning with a unique $\textbf{competitive advantage}$ (e.g., cost leadership, differentiation) in that segment'.
\subsection{Conclusion}
This note has analyzed the target market selection process and multi-market response strategies based on the lecture content.
The evaluation of market segments requires a multilateral analysis, not only of 'economic attractiveness' such as $\textbf{market size}$ and $\textbf{growth potential}$, but also of the fit with the company's $\textbf{management resources}$ and competitor trends (risk of becoming a $\textbf{red ocean}$). In particular, the pursuit of $\textbf{economies of scale}$ in production has a trade-off relationship with the degree of segmentation refinement.
The strategies adopted by companies are broadly divided into $\textbf{differentiated marketing}$ (full coverage) and $\textbf{concentrated marketing}$. The former can respond to a wide range of customer needs but risks becoming high-cost, while the latter makes it easier to build an advantage in a specific segment but carries vulnerability to changes in demand. As the examples of Suntory and Gucci show, these strategies are strategically selected and combined within the company's overall business portfolio.
The practical lessons learned from this lecture are, first, that marketing strategy is about deciding 'what not to do' (= concentration), and the importance of facing the limits of one's own resources. Second, as pointed out in the AI supplement, the perspective of $\textbf{Positioning}$—not just selecting an attractive market (T), but how to build unique value in that market and make it recognized by customers (P)—is indispensable. This $\textbf{STP}$ consistent logic is the source of competitive advantage.
\end{document}
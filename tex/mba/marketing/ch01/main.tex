\documentclass[uplatex,a4j,12pt,dvipdfmx]{jsarticle}
\usepackage{amsmath,amsthm,amssymb,bm,color,enumitem,mathrsfs,url,epic,eepic,ascmac,ulem,here,ascmac}
\usepackage[letterpaper,top=2cm,bottom=2cm,left=3cm,right=3cm,marginparwidth=1.75cm]{geometry}
\usepackage[english]{babel}
\usepackage[dvipdfm]{graphicx}
\usepackage[hypertex]{hyperref}
\title{Marketing - Lecture 1: Lecture Notes}
\author{M. O.}
\date{\today}
\begin{document}
\maketitle
\tableofcontents
\section{Basic Definitions of Marketing}
\subsection{Introduction}
This lecture covers the fundamental concepts of marketing. Although 'marketing' is a term used in daily life, its definition is often misunderstood. This report will begin by clarifying the differences between marketing and 'management' and 'selling'. It will then trace the evolution of the American Marketing Association (AMA) definitions, organizing the process by which modern marketing has evolved into a concept emphasizing 'value co-creation' and 'relationship management'. The final aim is to understand the essence of marketing by analyzing the four historical backgrounds (market competition, information technology, distribution, and demand) that spurred its development.
\subsection{Key Concepts and Arguments}
\subsubsection{Definitions and Misconceptions of Marketing}
Marketing, loosely defined, refers to 'all of a company's activities directed toward the market'. This concept is often confused with other business activities, making it necessary to clarify its boundaries.
\textbf{\paragraph{Difference from Management}}
Whereas management primarily focuses on the management of \textbf{internal resources}—such as 'organization, people, money, and materials'—marketing is characterized by its focus on responding to the \textbf{external environment}, namely the '\textbf{market}' and '\textbf{customers}'.
\textbf{\paragraph{Difference from Selling}}
A common misconception is that marketing is synonymous with selling, but they are distinctly different. It was explained in the lecture that the ultimate goal of marketing is 'to make \textbf{selling superfluous}'. This means not relying on the abilities of individual star salespeople, but rather creating an environment where the \textbf{product sells naturally} through comprehensive marketing activities—such as demand research, product development, and advertising—without requiring extraordinary sales efforts.
\textbf{\paragraph{Difference from Rules of Thumb (Collections of Success Stories)}}
Marketing is not merely an accumulation of past success stories or effective rules of thumb. Marketing theory does contain 'laws' (e.g., the \textbf{skimming strategy}). However, what is important is not memorizing the laws themselves, but theoretically considering '\textbf{why that law is effective}' and '\textbf{under what conditions it becomes effective}'.
\subsubsection{Evolution of the American Marketing Association (AMA) Definition}
The definition of marketing and its central role have evolved in accordance with changes in the era and social environment.
\begin{itemize}
	\item \textbf{1960s:} 'Business activities that direct the flow of \textbf{goods} and services from producer to consumer (user)'. The focus was on efficiently \textbf{flowing (distributing)} things.
	\item \textbf{1985s:} 'Comprehensive activities for creating market demand'. The concept of the company \textbf{creating} consumer '\textbf{demand}' was clarified.
	\item \textbf{2004s:} 'An organizational function... for creating, communicating, and delivering \textbf{value} to customers and for managing \textbf{customer relationships}'. Emphasis began to shift to building sustainable relationships rather than one-off sales. This definition also made the marketing concept applicable to \textbf{non-profit organizations} and public institutions.
	\item \textbf{2013s:} 'The process for exchanging, communicating, \textbf{dialoguing}, and \textbf{creating} offerings that have value for customers, partners, and society at large'. Against the backdrop of the proliferation of SNS, the concept was added that firms do not just unilaterally provide value, but 'dialogue' with customers and '\textbf{create value together} (\textbf{value co-creation})'.
\end{itemize}
\subsection{Application and Case Analysis}
\subsubsection{Case: Perception of Potato Chips}
As an introduction in the lecture, two brands of potato chips (e.g., \textbf{Calbee} and an unknown brand) were presented. It is assumed many students would choose the Calbee product. This is the result of the perception that the company 'advertises heavily' and 'is easily available in convenience stores'—in other words, that its marketing activities (promotion and distribution channels) are superior.
\subsubsection{Case: Manufacturer Brand Strategy (Distribution Countermeasures)}
In the past, before the emergence of large retail chains (like supermarkets), \textbf{wholesalers} held significant power over manufacturers and retailers. Wholesalers often treated a manufacturer's products as 'commodities' interchangeable with competing products, making manufacturers susceptible to \textbf{price competition}.
As a countermeasure, manufacturers used famous actors in television commercials to deliver information directly to consumers. Through this, they aimed to instill a specific '\textbf{brand preference}' in consumers, compelling retailers to stock the product by demand ('We must carry that brand'). This strategy attempted to suppress the wholesalers' power and secure stable distribution channels. This is an example of marketing (specifically promotion) being utilized for distribution strategy.
\subsection{Deeper Context and Lessons}
\textbf{\paragraph{The Four Backgrounds of Marketing's Development}}
The growing importance of the marketing concept in business management is closely related to four socio-economic changes.
\begin{enumerate}
	\item \textbf{Market Competition:} In the late 19th-century United States, as the \textbf{mass production system} was established, a situation emerged where 'products wouldn't sell just by being made'. Initially, this was a competition to develop new markets (sales destinations), but large corporations could quickly follow (imitate). Therefore, the axis of competition shifted to '\textbf{new product development}', which takes time and cost to imitate, and consumer needs research and analysis activities grew in importance.
	\item \textbf{Information Transmission Technology:} In the past, '\textbf{information asymmetry}' existed between companies and consumers, which made '\textbf{hard selling}' possible. However, this generated customer dissatisfaction and was not conducive to sustainable sales. With the spread of mass media such as newspapers, radio, television, and the internet, companies became able to transmit information widely through advertising. At the same time, competitors could use the same means, leading to a diversification of marketing methods.
	\item \textbf{Distribution:} As mentioned earlier, the balance of power in distribution has shifted with the times—from an era where wholesalers were strong, to manufacturers gaining power through advertising (brand strategy), and to the modern era where large retail chains and \textbf{online sales} (multi-channel) have increased their sales power. Manufacturers are constantly forced to adapt.
	\item \textbf{Demand:} A market was established due to rising incomes for the general populace accompanying economic growth, and the nationwide expansion of the purchasing class. Furthermore, as demand diversified, the concept of '\textbf{market segmentation}'—dividing the market by gender, age, etc.—became necessary.
\end{enumerate}
\textbf{\subsubsection{AI Supplement: Expansion of Key Arguments}}
This lecture comprehensively explained specific examples of marketing activities, such as 'new product development', 'advertising', and 'distribution channels', along with their developmental backgrounds. However, the term for the most fundamental framework used to systematically organize and execute these activities—the '\textbf{Marketing Mix (4Ps)}'—was conspicuously absent from the text.
The Marketing Mix is the combination of concrete tools used to implement a marketing strategy, composed of the four Ps: \textbf{Product (goods/services)}, \textbf{Price}, \textbf{Place (distribution/channels)}, and \textbf{Promotion (sales promotion/advertising)}. The concepts mentioned in the lecture—'new product development' (Product), 'skimming strategy' (Price), 'wholesalers and online sales' (Place), and 'hard selling or advertising' (Promotion)—correspond precisely to these 4P elements. Awareness of this framework allows for an integrated understanding of the concepts that appeared fragmented in the lecture, viewing them as systematic activities for strategy execution.
\subsection{Conclusion}
In this lecture, we learned that marketing is not merely 'selling' or a 'collection of success stories', but rather a company's strategic activity focused on the market and customers. Its definition, as shown by the evolution of the American Marketing Association (AMA) definitions (from the 'flow of goods' in the 1960s to 'value co-creation' in 2013), continues to evolve with the times.
Behind this evolution, there have always been four environmental changes: 'intensifying market competition' (leading to new product development), 'development of information technology' (from hard selling to advertising), 'shifts in distribution power', and 'diversification of demand' (leading to market segmentation).
The practical lesson drawn from this is that marketing is not a fixed concept. It is a process that requires \textbf{constantly monitoring the environment surrounding the company (competition, technology, distribution, customers) and flexibly continuing to change strategies and execution methods in response}.
\subsection{Key Keyword List}
Names:
None
\vspace{\baselineskip}
Theories/Concepts:
Marketing, Selling, Management, Market, Calbee, Skimming Strategy, American Marketing Association (AMA), Value Creation, Customer Relationships, Non-profit Organizations, Value Co-creation, Mass Production System, Market Competition, New Product Development, Information Asymmetry, Hard Selling, Distribution Channels, Brand Preference, Market Segmentation
\subsection{Comprehension Check Quiz}
\begin{enumerate}[label=\arabic*.]
	\item What is the difference in focus between marketing and 'management'?
	\item How does marketing differ from 'selling'?
	\item What is the ultimate goal of marketing regarding selling?
	\item Why is marketing theory not just an 'accumulation of rules of thumb'? What does it consider?
	\item What is the pricing strategy called where a high price is set at product launch and gradually lowered?
	\item In the 1960s AMA definition, what was the central activity of marketing?
	\item In the 1985s AMA definition, what were companies said to 'create'?
	\item What two important concepts became central to the 2004s AMA definition?
	\item What two new concepts were added in the 2013s AMA definition, reflecting the spread of SNS?
	\item What change in the production system in the late 19th century spurred the spread of marketing?
	\item In market competition, why did companies shift their focus from 'new market development' to 'new product development'?
	\item In an era of undeveloped media, what sales behavior did 'information asymmetry' between firms and consumers cause?
	\item What marketing method did manufacturers once use to counter powerful wholesalers?
	\item As market demand diversified, what marketing method became necessary for companies?
	\item (AI Supplement) What is the basic framework that combines the four elements of product, price, distribution, and promotion?
\end{enumerate}
\subsubsection*{Answer Key}
1. The market and customers (external environment), 2. It is a broader activity that includes demand research, product development, advertising, etc., 3. To make selling unnecessary, 4. To theoretically consider 'why' and 'under what conditions' a law is effective, 5. Skimming strategy, 6. Directing the flow of goods and services from producer to consumer (distribution), 7. (Market) demand, 8. Value (creation) and Customer relationships (management), 9. Dialogue, Value co-creation (or creation), 10. The establishment of the mass production system, 11. New markets could be easily imitated by competitors, whereas new product development was difficult to imitate, 12. Hard selling, 13. Advertising-based brand strategy (establishing consumer brand preference), 14. Market segmentation, 15. The Marketing Mix (4Ps)
\section{On the Purpose of Marketing and the Market}
\subsection{Introduction}
In the previous lecture, we learned that marketing constitutes 'all of a company's activities directed toward the market'. This report will first unpack the dual meaning of the \textbf{Market} itself, the 'stage' for marketing activities. Furthermore, regarding the ultimate purpose of marketing, we will contrast the two perspectives of '\textbf{Profit}' and '\textbf{Customer Satisfaction}'. The objective is to deeply consider why marketing theory emphasizes customer satisfaction as the guiding principle for activity.
\subsection{Key Concepts and Arguments}
\subsubsection{The Dual Meaning of the Market}
In marketing theory, the word 'market' is used in two different senses, depending on the context.
\begin{enumerate}
	\item \textbf{The Market as a 'Place' of Competition} \\
	      This sense is used in phrases like, 'The market environment is harsh'. The market refers to the '\textbf{place}' where multiple sellers (companies) and multiple buyers (customers) confront each other, and where companies engage in fierce competition for transactions with customers. Companies actively work to be chosen by customers in this arena.
	\item \textbf{The Market as an 'Aggregation' of Customers} \\
	      This sense is used in phrases like, 'The market's \textbf{needs} are changing' or 'This product will not be accepted by the market'. The market refers to the '\textbf{aggregation of customers}' itself, which is the target of the company's marketing activities (like new product development or promotion), or the recipient of those activities.
\end{enumerate}
A company must simultaneously recognize the market as a 'place' to formulate competitive strategy, and recognize the market as an 'aggregation of customers' to conduct activities that respond to their needs.
\subsubsection{The Purpose of Marketing: Why 'Customer Satisfaction' and Not 'Profit'}
While the (general) goal of a business is said to be the 'pursuit of \textbf{profit}', the primary goal in marketing theory is '\textbf{Customer Satisfaction}'.
The reason for this lies in the clarity of the process for determining the direction of activities.
\begin{itemize}
	\item \textbf{The Problem with 'Profit Maximization' as the Goal:}
	      There are innumerable ways to increase profit. For example, in the pursuit of short-term profit, 'slightly cutting raw material quality' could also become a viable option. However, this action will lead to a loss of customer trust in the long run and does not lead to customer satisfaction.
	\item \textbf{The Clarity of 'Customer Satisfaction' as the Goal:}
	      When the goal is set to 'customer satisfaction', the process of activities becomes clear. 'First, who are the customers we must satisfy (= defining \textbf{target customers})?', 'What satisfies those target customers (= analyzing needs')', and 'What kind of product or service should we develop to achieve that?'—in this way, a concrete direction for activities emerges.
\end{itemize}
Therefore, in marketing, profit is regarded as the 'result', and customer satisfaction is emphasized as the 'guiding principle (purpose)' that leads the process to that result.
\subsection{Application and Case Analysis}
\subsubsection{Case: The Short-Sighted Trap of 'Profit Maximization'}
The lecture presented the case of 's\textbf{lightly compromising on raw material quality}' as a common trap fallen into when 'profit maximization' is the sole objective.
This operation might reduce product costs and contribute to profit expansion in the short term. However, this decision lacks the perspective of 'customer satisfaction'. If customers notice the decline in quality, trust in the product will be lost, ultimately causing customer defaction and damaging long-term profits.
This is a prime example showing how the operational decisions of a company (in this case, procurement or manufacturing) can be diametrically opposed depending on whether the marketing objective is set to 'profit' or 'customer satisfaction'.
\subsection{Deeper Context and Lessons}
\textbf{\paragraph{Detour Topic Name (Diverged from Main Theory)}}
(This lecture's text was focused on the main theory and did not include any particular detour topics.)
\textbf{\subsubsection{AI Supplement: Expansion of Key Arguments}}
The lecture explained 'profit' and 'customer satisfaction' in contrast. To supplement their relationship, we add the perspective of \textbf{Philip Kotler}, a major figure in marketing theory.
Kotler defines the marketing concept as 'achieving \textbf{profits through customer satisfaction}'. This indicates that customer satisfaction and profit are not in a trade-off relationship; rather, \textbf{customer satisfaction is the very source of long-term profit}. In other words, customer satisfaction is positioned as the 'purpose (guiding principle for activity)', and profit is the 'result (and condition for corporate survival)'.
However, in practice, balancing the two is critical. If, in the pursuit of 'customer satisfaction', a company continues to provide excessive quality or service, costs will increase and pressure profits. A company must make strategic decisions within the constraints of its resources (people, materials, money): '\textbf{Which target customers}' to serve, '\textbf{what level of satisfaction}' to provide, and how to do so '\textbf{more efficiently than competitors}'. This is why the importance of 'defining target customers' was emphasized in the lecture.
\subsection{Conclusion}
In this lecture, we learned that the market, the 'stage' of marketing, holds a dual meaning: as a 'place of competition' and as an 'aggregation of customers'.
We also came to understand why the primary purpose of marketing is placed on 'customer satisfaction' rather than 'profit'. This is because, unlike the ambiguous goal of 'profit', the guiding principle of 'customer satisfaction' clarifies the direction for concrete activities, such as product development and service design based on the \textbf{needs of target customers}.
The practical lesson from this is the importance of always maintaining a marketing perspective when making daily operational decisions (e.g., cost reduction). Even if a decision contributes to short-term profit, one must ask: '\textbf{Does this decision improve (or at least, not harm) the satisfaction of our target customers?}'
\subsection{Key Keyword List}
Names:
Philip Kotler (AI Supplement)
\vspace{\baselineskip}
Theories/Concepts:
Market (dual meaning), Place of competition, Aggregation of customers, Market needs, Profit, Customer Satisfaction, Target Customers
\subsection{Comprehension Check Quiz}
\begin{enumerate}[label=\arabic*.]
	\item What is the 'stage' where marketing activities take place?
	\item What is the first meaning of 'market' presented in the lecture?
	\item When we say, 'The market environment is harsh', how are we viewing the market?
	\item What is the second meaning of 'market' presented in the lecture?
	\item When we say, 'The market's needs are changing', how are we viewing the market?
	\item What is generally said to be the common goal of a business?
	\item What is considered the primary goal in marketing theory?
	\item What is the main reason why 'profit maximization' is not set as the marketing goal?
	\item What is an example of a short-term, profit-seeking measure that runs counter to customer satisfaction?
	\item What is the greatest advantage of setting 'customer satisfaction' as the marketing goal?
	\item By setting 'customer satisfaction' as the goal, what is the first thing that must be clarified in the process?
	\item What is the term for defining the group of customers a company intends to satisfy?
	\item (AI Supplement) How did Philip Kotler define the relationship between customer satisfaction and profit?
	\item (AI Supplement) Under what circumstances can the pursuit of customer satisfaction put pressure on profits?
	\item By using customer satisfaction as a guide, the 'direction' of what becomes clear?
\end{enumerate}
\subsubsection*{Answer Key}
1. The market, 2. A 'place' of competition, 3. As a place of competition, 4. An 'aggregation' of customers, 5. As an aggregation of customers, 6. (The pursuit of) profit, 7. Customer satisfaction, 8. Because 'profit maximization' is ambiguous about the process or direction of activities (and can include short-sighted measures), 9. Slightly cutting raw material quality, 10. It becomes easier to determine (clarifies) the direction of activities, 11. Which customers to satisfy (defining target customers), 12. (Defining) target customers, 13. Achieving profits through customer satisfaction, 14. If excessive quality or service is provided, increasing costs, 15. (Marketing) activities
\section{Market Concepts in Marketing Theory}
\subsection{Introduction}
This lecture addresses the fundamentals of marketing theory for grasping corporate activities within the market. The objective is to understand the three primary characteristics of the market that marketing activities presuppose: the '\textbf{Differentiated Market}', the '\textbf{Segmented Market}', and the '\textbf{Changing Market}'. We will clarify how these characteristics form the bedrock of a company's \textbf{non-price competition} and product strategies.
\subsection{Key Concepts and Arguments}
In marketing theory, the market in which companies operate is presupposed to have the following three characteristics.
\subsubsection{Differentiated Market}
A \textbf{Differentiated Market} refers to a market where companies conduct diverse marketing activities (such as design, functionality, advertising, or limiting sales channels) to make customers perceive their products as different from those of competitors.
In this market, consumers respond to these marketing efforts and form specific \textbf{preferences}. As a result, they are assumed to develop \textbf{non-substitutable preferences} for certain company's products (a state where they cannot be satisfied by anything other than that company's product).
Under such conditions, competition between firms shifts away from price and toward how to increase the \textbf{added value} of the product (e.g., design, functionality, brand image). In short, \textbf{non-price competition} becomes central.
In contrast, in a \textbf{Homogeneous Market} (where consumers perceive all sellers' products as identical), consumers will simply select the lowest-priced product. Competition inevitably becomes \textbf{price competition}, and profits are easily compressed. Marketing is the activity aimed at escaping this homogeneous market and increasing profits in a market where differentiation is possible.
\subsubsection{Segmented Market}
It is self-evident that differences exist in consumer preferences and needs within any given product category. Marketing theory posits that it is possible to identify groups (sub-markets) of consumers who share the same preferences.
This process is called \textbf{Market Segmentation}. Segmentation is conducted by classifying consumers based on criteria such as gender, age, lifestyle, or values.
The sub-markets identified through this segmentation are called \textbf{market segments}. Consumers within a segment are assumed to have similar (or homogeneous) characteristics and preferences.
By developing and providing products that are specialized for the preferences of a specific segment, a company can create non-substitutable demand that is clearly differentiated from other companies, making it possible to engage in non-price competition.
\subsubsection{Changing Market}
The market is understood not as static, but as something that is constantly changing over time. 'Change' here primarily refers to two points: '\textbf{changes in consumer preferences}' and '\textbf{changes in the nature of corporate competition}'.
Companies must predict these changes and formulate marketing plans through market analysis. To avoid unpredictability and enhance the effectiveness of strategic planning, it is necessary to extract the \textbf{stable patterns} underlying market changes.
A representative example of such a pattern is the concept of the '\textbf{Product Life Cycle (PLC)}'. The PLC classifies the market changes a product experiences from its introduction to its withdrawal into stages such as 'Introduction', 'Growth', 'Maturity', and 'Decline'. The market characteristics of each stage (e.g., demand growth, competitive situation) are assumed to be common, allowing companies to deploy marketing strategies (such as product modification, pricing, or promotion) appropriate for each stage.
\subsection{Application and Case Analysis}
The concepts presented in the lecture are applicable to many real-world business activities.
\subsubsection{Case of Differentiation: Apple Inc.}
Apple is a textbook example of creating a 'differentiated market'. In the smartphone market, the company engages not merely in functional (spec) competition, but achieves clear differentiation from other products (Android devices) through its proprietary OS (iOS), refined design, ecosystem (App Store, iCloud), and powerful brand advertising. Consumers are paying not just based on price (as in a homogeneous market), but for the unique \textbf{added value} and experience (a non-substitutable preference) that Apple products provide.
\subsubsection{Case of Segmentation: Diversification by Beverage Manufacturers}
The beverage market is a prime example of a 'segmented market'. For instance, The Coca-Cola Company offers not only the traditional 'Coca-Cola' but also 'Coca-Cola Zero' (targeting health-conscious males), 'Diet Coke' (zero-calorie preference), and even 'Ayataka' (targeting specific preferences in the green tea market). It identifies \textbf{market segments} based on diverse criteria such as age, gender, health-consciousness, and drinking occasion, and provides products optimized for each.
\subsubsection{Case of Change: The Digital Camera Market and PLC}
The concepts of the 'changing market' and the \textbf{Product Life Cycle} can be confirmed in the transition of the digital camera market. Compact digital cameras, once the market mainstream, rapidly entered the 'decline stage' due to the dramatic improvement of smartphone camera functions (a change in consumer preferences and the emergence of a substitute). Conversely, mirrorless interchangeable-lens cameras entered the 'growth stage' through new technological innovations, forcing companies to reallocate resources and alter their strategies.
\subsection{Deeper Context and Lessons}
\subsubsection{Organizing the Core Premises of the Lecture}
\textbf{\paragraph{The Absolute Importance of 'Differentiation'}}
This lecture emphasizes 'differentiation' as the most crucial concept in marketing. For a company to increase its profits, it is essential to avoid the homogeneous markets prone to price competition and to make efforts to create a \textbf{differentiable market} itself.
\textbf{\paragraph{The 'Three Fundamental Axes' of Marketing Strategy}}
As a conclusion of the lecture, all individual product strategies, price strategies, and advertising strategies considered by a company are deployed upon the three fundamental axes (market assumptions): the 'differentiated market', 'segmented market', and 'changing market'. Therefore, when planning any marketing activity, it is extremely important to return to these three fundamental axes to analyze the market environment.
\subsubsection{\textbf{AI Supplement: Expansion of Key Arguments}}
This lecture's transcript focused on the three 'market' characteristics that premise marketing activity. However, it lacked discussion on what specific strategies companies execute *after* understanding these market characteristics. We will supplement two key points that were missing.
\textbf{\paragraph{The Overall Picture of STP Strategy}}
The lecture covered \textbf{Market Segmentation (S)}, but this is only one part of the process known as STP strategy. After segmentation (S), the company:
\begin{enumerate}
	\item \textbf{Targeting (T)}: Evaluates the attractiveness of the identified segments and its own strengths, and selects the target market segment(s) to pursue.
	\item \textbf{Positioning (P)}: Designs and communicates the product and brand image so that it occupies a unique, clear position in the minds of the target customers, relative to competing products.
\end{enumerate}
Only through this entire S$\to$T$\to$P process do 'differentiation' and 'non-price competition' become concretely executable.
\textbf{\paragraph{The Marketing Mix (4Ps)}}
After the strategic direction of 'who to serve and what value to provide' is determined by the STP strategy, the company must combine execution tactics to deliver that value to the market. This is the \textbf{Marketing Mix}, generally composed of four elements called the \textbf{4Ps}.
\begin{itemize}
	\item \textbf{Product}: The product/service that meets the target's needs (quality, design, features).
	\item \textbf{Price}: The pricing commensurate with that value (list price, discounts).
	\item \textbf{Place}: The location/method for the target to acquire the product (channels, location).
	\item \textbf{Promotion}: Activities to communicate the product's value to the target and spur purchase (advertising, PR, personal selling).
\end{itemize}
In each stage of the 'Product Life Cycle' mentioned in the lecture, the company responds to market changes by optimizing the combination (mix) of these 4Ps.
\subsection{Conclusion}
This lecture note analyzed the three essential characteristics of the market that marketing theory presupposes as its premise: \textbf{Differentiation}, \textbf{Segmentation}, and \textbf{Change}. Companies must deeply understand these market characteristics to avoid homogeneous price competition and secure sustainable profits.
As confirmed in the 'Deeper Context', these three characteristics are the 'fundamental axes' that form the foundation for individual strategies (like product or advertising strategy). As a practical lesson, we practitioners must constantly ask whether the market we face is becoming 'homogeneous', whether we are overlooking opportunities for 'segmentation' of customer needs, and whether we are predicting 'change' patterns (such as the Product Life Cycle) and reflecting them in our strategies (STP and 4Ps).
\subsection{Key Keyword List}
\textbf{Names:}
(None)
\vspace{\baselineskip}
\textbf{Theories/Concepts:}
Differentiated Market, Homogeneous Market, Added Value, Price Competition, Non-price Competition, Market Segmentation, Market Segment, Non-substitutable Preference, Changing Market, Consumer Preference, Product Life Cycle (PLC)
\subsection{Comprehension Check Quiz}
\begin{enumerate}
	\item What do you call a market where companies use design and advertising to make customers perceive their products as different from competitors?
	\item What do you call a market where consumers perceive all sellers' products as identical?
	\item What is the main type of competition in a homogeneous market?
	\item What do you call the competition based on value other than price, which companies aim for in a differentiated market?
	\item What is the general term for value added to a product other than price, such as design, function, or brand image?
	\item What is the process of creating groups (sub-markets) of people with similar preferences, based on differences in consumer preferences?
	\item What is the specific term for a sub-market identified through market segmentation, which has homogeneous preferences?
	\item How did the lecture describe the preference where a consumer is only satisfied by one specific product?
	\item Of the three market characteristics presupposed by marketing theory, which one refers to consumer preferences and competition changing over time?
	\item What concept, which divides a product's life from introduction to decline into stages, was given as an example of a stable pattern in market change?
	\item What are the four stages of the Product Life Cycle? Introduction, Growth, Maturity, and...?
	\item (AI Supplement) What is the final process of STP strategy, alongside Segmentation (S) and Targeting (T)?
	\item (AI Supplement) What is the activity of designing a clear position for one's product in the target customer's mind, relative to competitors?
	\item (AI Supplement) What is the general term for the execution tactics that combine Product, Price, Place, and Promotion?
	\item What was emphasized in the lecture as the most important concept in marketing?
\end{enumerate}
\subsubsection*{Answer Key}
1. Differentiated Market, 2. Homogeneous Market, 3. Price Competition, 4. Non-price Competition, 5. Added Value, 6. Market Segmentation, 7. Market Segment, 8. Non-substitutable Preference, 9. Changing Market, 10. Product Life Cycle (PLC), 11. Decline Stage, 12. Positioning, 13. Positioning, 14. Marketing Mix (or 4Ps), 15. Differentiation
\section{The Two Aspects of Marketing}
\subsection{Introduction}
The purpose of this lecture is to understand two fundamentally different aspects that must be considered when formulating marketing strategy: the '\textbf{Market Aspect}' and the '\textbf{Relationship Aspect}'. Depending on the type of customer (whether they are an unspecified large number or a specific few), the focus of the marketing activities required of the company differs radically. This note will contrast these two aspects using cases (Calbee and FUSERASHI) and analyze the strategic approaches that are important in each.
\subsection{Key Concepts and Arguments}
Marketing activities are broadly classified into the following two aspects based on the nature of the target customers.
\subsubsection{Market Aspect}
The \textbf{Market Aspect} mainly corresponds to markets faced by \textbf{consumer goods} (e.g., potato chips) manufacturers, where customers are an '\textbf{unspecified large number}' and the faces of individual customers are difficult to see.
\begin{itemize}
	\item \textbf{Customer Characteristics}: Customers are innumerable, and the company does not know them individually.
	\item \textbf{Important Activities}:
	      \begin{enumerate}
		      \item \textbf{Customer Analysis}: Analyzing the general \textbf{behavior patterns} of countless customers and classifying them into groups with common needs or characteristics (\textbf{Grouping} / \textbf{Market Segmentation}).
		      \item \textbf{Marketing Research}: Investigating what kind of products should be developed and offered to specific segments.
		      \item \textbf{Marketing Mix (4P) Planning/Execution}:
		            \begin{itemize}
			            \item \textbf{Advertising/Promotion}: Designing advertisements that resonate with the target demographic based on analysis, and deploying them through mass media. Price adjustments (discounts or bonuses) are also included.
			            \item \textbf{Distribution/In-store Management}: Distributing products to locations where consumers can easily purchase them (e.g., convenience stores, supermarkets) and conducting negotiations or display management to secure '\textbf{eye-catching}' shelf space (e.g., eye-level shelves).
		            \end{itemize}
	      \end{enumerate}
	\item \textbf{Relevant Fields}: Analysis of this aspect is primarily handled in fields such as '\textbf{Marketing Management}', '\textbf{Strategic Marketing}', and '\textbf{Marketing Research}'.
\end{itemize}
However, even for industrial goods, if they are \textbf{general-purpose components} like '\textbf{screws}' sold broadly, an approach similar to the Market Aspect is necessary.
\subsubsection{Relationship Aspect}
The \textbf{Relationship Aspect} mainly corresponds to markets faced by \textbf{industrial goods} (e.g., automotive parts) manufacturers, where customers are a '\textbf{specific few}' (e.g., Nissan, Toyota, etc., perhaps 10-odd companies) and the 'faces' of the customers are clearly visible.
\begin{itemize}
	\item \textbf{Customer Characteristics}: The number of customers is limited, and the company builds close relationships with each customer's representatives (typical of BtoB marketing).
	\item \textbf{Important Activities}:
	      \begin{enumerate}
		      \item \textbf{Sales Activity (Personal Selling)}: Rather than advertising strategy, individual relationship-building by talented sales representatives becomes crucial. This involves understanding what new products the customer will develop next season, what components they will require, and lobbying for the adoption of one's own products.
		      \item \textbf{Customer Relationship Management (CRM)}: Continuing and developing transactions with a limited number of important customers becomes the top priority. Activities focus on increasing customer satisfaction and maintaining/managing long-term relationships.
		      \item \textbf{Organizational Response (Inter-departmental Cooperation)}: To respond quickly to sudden customer requests (e.g., shortening a \textbf{delivery date}, adjusting production volume), close \textbf{inter-departmental cooperation} between sales, production, and support departments is vital, as is an organizational management structure that enables this (a corporate \textbf{strength}).
	      \end{enumerate}
	\item \textbf{Relevant Fields}: This aspect is handled in-depth in fields such as 'Sales Management', 'Distribution Channel Theory', and 'Customer Relationship Management (CRM)'.
\end{itemize}
\subsection{Application and Case Analysis}
In the lecture, the following two companies were raised as cases to contrast these two aspects.
\subsubsection{Market Aspect Case: Calbee, Inc.}
The potato chips sold by Calbee are a typical consumer good. The customers are 'myself and everyone here', in other words, an '\textbf{unspecified large number}' of people worldwide. What is important for Calbee is not knowing the face of each customer, but analyzing \textbf{general behavior patterns}—such as 'what flavors and sizes are preferred'—through market research. Based on that analysis, they develop products, run TV commercials (\textbf{advertising}), and secure '\textbf{eye-catching}' shelf space in supermarkets and convenience stores. These \textbf{Market Aspect} activities determine sales.
\subsubsection{Relationship Aspect Case: FUSERASHI Co., Ltd.}
FUSERASHI is a company that manufactures and sells automotive components (such as special bolts). Its customers are limited to 16 companies worldwide (as of the lecture), including Toyota, Daihatsu, and Nissan. What is important for FUSERASHI is not advertising to the general public, but \textbf{sales activities} that build close relationships with the representatives of specific customers (the automakers). Co-developing components to match the next car the customer will develop, and possessing an \textbf{organizational management structure} that can handle sudden changes to \textbf{delivery} schedules—this is the source of their competitiveness in the \textbf{Relationship Aspect}.
\subsection{Deeper Context and Lessons}
\subsubsection{Organizing the Lecture's Arguments}
\textbf{\paragraph{The Resolution of the Customer's 'Face'}}
The core of the lecture is that marketing strategy changes fundamentally based on how concretely the customer's 'face' can be seen. The face of the consumer buying potato chips is not visible (=\textbf{Market Aspect}), but the face of the purchasing agent buying auto parts is visible (=\textbf{Relationship Aspect}).
\textbf{\paragraph{The Intermediate Approach in Service Sales}}
The lecture provided a supplementary reference to cosmetics sales. This belongs to the Market Aspect (unspecified large number), but it differs from pure consumer goods in that a \textbf{salesperson} (human) mediates at the point of purchase. In this case, the \textbf{service quality} of the salesperson (product knowledge, customer service attitude) greatly influences sales. Therefore, activities such as \textbf{manualizing} the service and standardizing/managing its quality through employee \textbf{education} become important. This can be described as a Market Aspect approach that incorporates human elements (elements akin to the Relationship Aspect).
\textbf{\paragraph{Connection to this Lecture Course's Overall Structure}}
This entire lecture course is structured around these two aspects as twin pillars. First, students will learn the theories of the '\textbf{Market Aspect}' (consumer behavior research, segmentation, product differentiation). Following that, they will learn the theories of the '\textbf{Relationship Aspect}' (sales activities, inter-departmental cooperation, customer relationship management). Finally, the course is structured to consider how to apply these two aspects in the new environment of international markets (overseas expansion).
\subsubsection{\textbf{AI Supplement: Expansion of Key Arguments}}
The lecture explained the importance of the 'Relationship Aspect', but it lacked a reference to the central theoretical framework that supports this aspect. We will supplement with the following two concepts, which are indispensable for understanding BtoB (business-to-business) marketing.
\textbf{\paragraph{Relationship Marketing}}
The activity of 'maintaining and managing customer relationships' described in the lecture is theorized in business administration as '\textbf{Relationship Marketing}'. Whereas traditional marketing (the Market Aspect) seeks to maximize singular transactions, this approach focuses on building, maintaining, and strengthening long-term, favorable 'relationships' with customers. FUSERASHI's activities aimed at continuing its transactions with Toyota are a prime example of this theory in practice.
\textbf{\paragraph{The Buying Center}}
In the Relationship Aspect (especially BtoB), the 'customer' that a sales representative engages with is not a single 'representative'. In reality, many stakeholders are involved in a corporate purchasing process. This decision-making group is called the '\textbf{Buying Center}'. The Buying Center includes 'Users' (who actually use the product), 'Buyers' (who handle the practical contracting), 'Influencers' (who determine technical specifications), 'Deciders' (who give final approval), and 'Gatekeepers' (who manage the flow of information). Advanced sales activity in the Relationship Aspect means understanding the dynamics of this complex Buying Center and appropriately appealing to each stakeholder.
\subsection{Conclusion}
This lecture note analyzed how marketing activities are broadly divided into two different aspects: the '\textbf{Market Aspect}' (for unspecified, unseen customers) and the '\textbf{Relationship Aspect}' (for specific, visible customers).
In the Market Aspect, strategic \textbf{Marketing Management} (advertising, promotion, channel management) based on \textbf{research} and \textbf{grouping} is the key to success. Conversely, in the Relationship Aspect, the sources of competitive advantage are \textbf{sales activities} (personal selling) for close \textbf{Customer Relationship Management} and the \textbf{inter-departmental cooperation} required to meet customer demands, rather than advertising.
As a practical lesson, we must calmly evaluate which aspect our own business primarily belongs to. Should a consumer goods manufacturer also compete in the Market Aspect like a general-purpose component (screw)? Or should an industrial goods manufacturer specialize in the Relationship Aspect (Relationship Marketing)? Discerning this is the first step toward the optimal allocation of marketing resources and strategy formulation.
\subsection{Key Keyword List}
\textbf{Names:}
(None)
\vspace{\baselineskip}
\textbf{Theories/Concepts:}
Market Aspect, Relationship Aspect, Unspecified Large Number, Specific Few, BtoC (Business-to-Consumer), BtoB (Business-to-Business), Grouping (Market Segmentation), Consumer Behavior Research, Marketing Management, Marketing Research, Sales Activity (Personal Selling), Customer Relationship Management (CRM), Inter-departmental Cooperation, Relationship Marketing, Buying Center
\subsection{Comprehension Check Quiz}
\begin{enumerate}
	\item What are the two primary aspects of marketing presented in the lecture?
	\item What is the aspect called that consumer goods manufacturers like Calbee primarily deal with, where customers are an 'unspecified large number'?
	\item What is the aspect called that automotive parts manufacturers like FUSERASHI primarily deal with, where customers are a 'specific few'?
	\item In the Market Aspect, what is the activity of classifying countless customers by common needs?
	\item In the Relationship Aspect, what activity is considered more important than advertising strategy?
	\item Negotiating for 'eye-catching' shelf space in a supermarket is an activity in which aspect?
	\item 'Inter-departmental cooperation' to respond to a customer's sudden delivery date change is important in which aspect?
	\item Even for industrial goods, if selling 'general-purpose components' like screws, which aspect does the approach resemble?
	\item What is one example of an academic field in marketing that primarily handles analysis of the Market Aspect?
	\item In cosmetics sales, what methods were introduced in the lecture to standardize the service quality of salespeople?
	\item The fact that FUSERASHI's customers are limited to 16 companies strongly shows which characteristic, BtoB or BtoC?
	\item In the structure of this lecture course, which aspect is scheduled to be studied after the Market Aspect?
	\item (AI Supplement) What is the marketing concept that focuses on building long-term relationships with customers, rather than single transactions?
	\item (AI Supplement) In BtoB marketing, what is the term for the group involved in the purchasing decision within a customer's company?
	\item Why are the cases of 'Calbee' and 'FUSERASHI' considered contrasting for explaining the difference in marketing strategy?
\end{enumerate}
\subsubsection*{Answer Key}
1. Market Aspect and Relationship Aspect, 2. Market Aspect, 3. Relationship Aspect, 4. Grouping (or Market Segmentation), 5. Sales Activity (or Personal Selling, Customer Relationship Management), 6. Market Aspect, 7. Relationship Aspect, 8. Market Aspect, 9. Marketing Management (or Strategic Marketing, Marketing Research), 10. Manualization and education, 11. BtoB (Business-to-Business), 12. Relationship Aspect, 13. Relationship Marketing, 14. Buying Center, 15. Because their target customers are at opposite extremes: 'unspecified large number' (Market Aspect) and 'specific few' (Relationship Aspect).
\end{document}
\documentclass[uplatex,a4j,12pt,dvipdfmx]{jsarticle}
\usepackage{amsmath,amsthm,amssymb,bm,color,enumitem,mathrsfs,url,epic,eepic,ascmac,ulem,here,ascmac}
\usepackage[letterpaper,top=2cm,bottom=2cm,left=3cm,right=3cm,marginparwidth=1.75cm]{geometry}
\usepackage[english]{babel}
\usepackage[dvipdfm]{graphicx}
\usepackage[hypertex]{hyperref}
\title{Marketing Lecture 1: Lecture Notes}
\author{M. O.}
\date{\today}
\begin{document}
\maketitle
\tableofcontents
\section{Basic Definitions of Marketing}
\subsection{Introduction}
This lecture covers the fundamental concepts of marketing. While 'marketing' is a commonly used term, its definition is often misunderstood. This report first clarifies the differences between marketing, 'management', and 'selling'. Next, by tracing the evolution of the American Marketing Association (AMA) definition, it organizes the process by which modern marketing has evolved into a concept emphasizing 'value co-creation' and 'relationship management'. Finally, the aim is to understand the essence of marketing by analyzing the four historical backgrounds (market competition, information technology, distribution, and demand) that have driven its development.
\subsection{Key Concepts and Points}
\subsubsection{Definitions and Misconceptions of Marketing}
Marketing, roughly speaking, refers to 'all of a company's activities directed at the market'. This concept is often confused with other business activities, making it necessary to clarify its boundaries.
\textbf{\paragraph{Difference from Management}}
Whereas management primarily focuses on the \textbf{management of internal resources}—such as 'organization, people, money, and materials'—marketing is distinct in its focus on responding to the \textbf{external environment}, namely the '\textbf{market}' and '\textbf{customers}'.
\textbf{\paragraph{Difference from Selling}}
A common misconception is that marketing is synonymous with selling, but the two are distinctly different. It was explained in the lecture that the ultimate goal of marketing is 'to make \textbf{selling superfluous}'. This means creating an \textbf{environment where the product essentially sells itself}—without relying on the abilities of individual star salespeople—through comprehensive marketing activities such as demand research, product development, and advertising.
\textbf{\paragraph{Difference from Rules of Thumb (Success Stories)}}
Marketing is not merely an accumulation of past success stories or effective rules of thumb. 'Laws' exist in marketing theory (e.g., \textbf{skimming strategy}). However, the important thing is not to memorize the laws themselves, but to theoretically consider '\textbf{why that law is effective}' and '\textbf{under what conditions it becomes effective}'.
\subsubsection{Evolution of the American Marketing Association (AMA) Definition}
As times and social environments have changed, the central role of marketing—and its definition—has also shifted.
\begin{itemize}
	\item \textbf{1960s:} 'The business activities that direct the flow of \textbf{goods} and services from producer to consumer (user).' The focus was on efficiently \textbf{moving (distributing)} things.
	\item \textbf{1985:} 'Comprehensive activities for creating market demand.' The concept of the company \textbf{creating} consumer '\textbf{demand}' was clarified.
	\item \textbf{2004:} 'An organizational function for creating, communicating, and delivering \textbf{value} to customers and for managing \textbf{customer relationships}.' The emphasis shifted from one-off sales to building sustainable relationships. This definition also made the concept of marketing applicable to \textbf{non-profit organizations} and public institutions.
	\item \textbf{2013:} 'A process for exchanging, communicating, \textbf{dialoguing}, and \textbf{creating} offerings that have value for customers, partners, and society at large.' Against the backdrop of SNS proliferation, the concept was added that companies do not just unilaterally provide value, but '\textbf{dialogue}' with customers and '\textbf{create value together}' (\textbf{value co-creation}).
\end{itemize}
\subsection{Application and Case Analysis}
\subsubsection{Case: Perception of Potato Chips}
As an introduction to the lecture, potato chips from two companies (e.g., \textbf{Calbee} and an unknown brand) were presented. It is assumed that many students would choose the Calbee product. This is the result of the perception that the company's marketing activities (promotion and distribution channels) are superior—images such as 'they put effort into advertising' and 'they are easily available at convenience stores'.
\subsubsection{Case: Manufacturer Brand Strategy (Distribution Countermeasures)}
In the past, before the emergence of large-scale retail chains (such as supermarkets), \textbf{wholesalers} held significant power over manufacturers and retailers. Wholesalers treated manufacturers' products as 'commodities' identical to other companies' products, making it easy for manufacturers to get caught in \textbf{price competition}.
As a countermeasure to this situation, manufacturers used famous actors in television commercials to deliver information directly to consumers. Through this, they attempted to instill a specific '\textbf{brand preference}' in consumers, inducing retailers to specifically request the product ('we must stock that brand'), thereby curbing the power of wholesalers and securing a stable distribution channel. This is an example of marketing (particularly promotion) being utilized for distribution strategy.
\subsection{Deeper Background and Lessons}
\textbf{\paragraph{Four Backgrounds to Marketing's Development}}
The growing importance of the marketing concept in business management is closely related to changes in four socio-economic conditions.
\begin{enumerate}
	\item \textbf{Market Competition:} In late 19th-century America, the establishment of \textbf{mass production systems} created a situation where 'products wouldn't sell just by being made'. Initially, it was a competition to develop new markets (sales destinations), but large corporations could quickly follow (imitate). Therefore, the axis of competition shifted to '\textbf{new product development}', which requires time and cost to imitate, and correspondingly, activities for researching and analyzing consumer needs became crucial.
	\item \textbf{Information Technology:} In the past, an '\textbf{information asymmetry}' existed between companies and consumers, which allowed companies to engage in '\textbf{hard selling}'. However, this created customer dissatisfaction and did not lead to sustainable sales. With the spread of mass media such as newspapers, radio, television, and the internet, companies became able to transmit information widely through advertising. At the same time, other companies could use the same means, leading to the diversification of marketing methods.
	\item \textbf{Distribution:} As mentioned earlier, the balance of power in distribution has changed over time—from an era where wholesalers were strong, to manufacturers gaining power through advertising (brand strategy), and to the modern era where large-scale retail chains and \textbf{online sales} (multi-channel) have increased their sales power. Manufacturers are constantly forced to adapt.
	\item \textbf{Demand:} Markets were established due to the rise in average personal income accompanying economic growth and the nationwide expansion of the purchasing population. Furthermore, as demand diversified, the concept of '\textbf{market segmentation}'—dividing the market by criteria such as gender or age—became necessary.
\end{enumerate}
\textbf{\subsubsection{AI Supplement: Expanding on Key Points}}
This lecture comprehensively explained specific examples of marketing activities, such as 'new product development', 'advertising', and 'distribution channels', along with their developmental backgrounds. However, the term '\textbf{Marketing Mix (4Ps)}'—the most fundamental framework for systematically organizing and executing these activities—was conspicuously absent from the text.
The Marketing Mix is a combination of specific tools for putting marketing strategy into action, consisting of the four Ps: \textbf{Product}, \textbf{Price}, \textbf{Place} (distribution/channels), and \textbf{Promotion}. 'New product development' (Product), 'skimming strategy' (Price), 'wholesalers and online sales' (Place), and 'hard selling and advertising' (Promotion) mentioned in this lecture correspond precisely to these 4P elements. By being aware of this framework, the concepts that appeared fragmented in the lecture can be understood integrally as systematic activities for strategy execution.
\subsection{Conclusion}
In this lecture, we learned that marketing is not merely 'selling' or a 'collection of success stories', but a strategic corporate activity focused on the market and customers. Its definition, as shown by the evolution of the AMA definition (from the 'flow of goods' in the 1960s to 'value co-creation' in 2013), continues to evolve with the times.
Behind this evolution, four environmental changes have always been present: 'intensifying market competition' (leading to new product development), 'development of information technology' (from hard selling to advertising), 'changes in distribution power', and 'diversification of demand' (leading to market segmentation).
The practical lesson derived from this is that marketing is not a fixed concept, but a process that \textbf{requires constantly monitoring changes in the company's surrounding environment (competition, technology, distribution, customers) and flexibly and continuously adapting strategies and execution methods in response}.
\subsection{Key Terms List}
People:
None
\vspace{\baselineskip}
Theories/Concepts:
Marketing, Selling, Management, Market, Calbee, Skimming Strategy, American Marketing Association (AMA), Value Creation, Customer Relationships, Non-profit Organizations, Value Co-creation, Mass Production System, Market Competition, New Product Development, Information Asymmetry, Hard Selling, Distribution Channels, Brand Preference, Market Segmentation
\subsection{Comprehension Quiz}
\begin{enumerate}[label=\arabic*.]
	\item Whereas management focuses on internal resources, what contrasting domain does marketing focus on?
	\item Marketing activities encompass selling activities, but what is the strategic objective of marketing that differs from selling alone?
	\item What kind of state does the ultimate marketing goal of 'making selling superfluous' mean to create?
	\item Unlike a mere collection of success stories, what is the purpose of theoretically considering 'laws' like the skimming strategy in marketing theory?
	\item What is the name of the strategy mentioned in the lecture as an example of a 'law', which involves setting a high price when launching a new product?
	\item In the evolution of the AMA definition, what two concepts became most important when transitioning from the 1960s 'flow of goods' activity to the 2004 definition?
	\item What change in the production system (from the lecture's 'Deeper Background') lies behind the 1985 AMA definition stating that companies 'create demand'?
	\item The 2004 AMA definition became applicable even to 'non-profit organizations' because the objective of marketing shifted from 'sales' to the management of what?
	\item What change in the company-customer relationship is suggested as the background for the 'value co-creation' concept being added to the 2013 AMA definition?
	\item Why did the establishment of 'mass production systems' in the late 19th century create a 'products won't sell just by being made' situation, increasing the importance of marketing?
	\item What was the strategic reason for the axis of market competition shifting from 'new market development' to 'new product development'?
	\item In the era of 'information asymmetry', what sales behavior was possible but unsustainable because it created customer dissatisfaction?
	\item What change in the balance of power in distribution channels was the strategic objective when manufacturers formerly tried to establish 'brand preference' through advertising?
	\item What concept was introduced as a method for companies to divide the market into groups with common needs in response to the environmental change of demand diversification?
	\item (AI Supplement) What is the general term for the framework (4Ps) that systematically organizes individual activities mentioned in the lecture, such as 'new product development', 'skimming strategy', and 'advertising', for strategy execution?
\end{enumerate}
\subsubsection*{Answer Key}
1. The market and customers (external environment), 2. To create an environment where the product sells naturally through activities like demand research, product development, and advertising., 3. Creating an environment where the product sells naturally without salespeople having to exert great effort., 4. To theoretically consider 'why' and 'under what conditions' the law is effective., 5. Skimming strategy, 6. 'Value (creation)' and 'customer relationships (management)', 7. The establishment of mass production systems (which created a 'products won't sell just by being made' situation), 8. Customer relationships (management), 9. A change to a relationship where companies do not unilaterally provide value, but 'dialogue' with customers and 'create value together'., 10. Because the ability to produce outstripped demand, creating a situation where 'products wouldn't sell just by being made'., 11. Because new market development is easily 'followed (imitated)' by competitors, whereas new product development is difficult to imitate., 12. Hard selling, 13. To curb the power of wholesalers and secure manufacturer-led distribution channels., 14. Market segmentation, 15. Marketing Mix (4Ps)
\section{On the Purpose of Marketing and the Market}
\subsection{Introduction}
In the previous lecture, we learned that marketing constitutes 'all of a company's activities directed at the market'. This report deciphers the dual meaning of the concept of the \textbf{Market}, which is the 'stage' for marketing activities. Furthermore, regarding the ultimate purpose of marketing, the aim is to deeply examine the reason why customer satisfaction is emphasized as the guiding principle in marketing theory, by contrasting the two perspectives of '\textbf{Profit}' and '\textbf{Customer Satisfaction}'.
\subsection{Key Concepts and Points}
\subsubsection{The Dual Meaning of 'Market'}
In marketing theory, the word 'market' is used in two different senses depending on the context.
\begin{enumerate}
	\item \textbf{The Market as a 'Place' of Competition} \\
	      This sense is used when saying things like 'the market environment is harsh'. The market refers to the '\textbf{place}' where multiple sellers (companies) and multiple buyers (customers) confront each other, and companies engage in fierce competition over transactions with customers. Companies actively appeal to customers in this place to ensure they are chosen.
	\item \textbf{The Market as an 'Aggregation' of Customers} \\
	      This sense is used when saying 'market \textbf{needs} are changing' or 'the market will not accept this product'. The market refers to the very '\textbf{aggregation of customers}' that is the target of a company's marketing activities (like new product development or promotion), or the recipient of those activities.
\end{enumerate}
Companies must devise competitive strategies by recognizing the market as a 'place', while simultaneously conducting activities to meet needs by recognizing the market as an 'aggregation of customers'.
\subsubsection{The Purpose of Marketing: Why 'Customer Satisfaction' and Not 'Profit'}
While the (general) goal of a company is the 'pursuit of \textbf{profit}', the primary goal in marketing theory is said to be '\textbf{Customer Satisfaction}'.
The reason for this lies in the clarity of the process for determining the direction of activities.
\begin{itemize}
	\item \textbf{Problems with Aiming for 'Profit Maximization':}
	      Countless methods exist to maximize profit. For example, in pursuit of short-term profit, 'slightly reducing the quality of raw materials' could become an option. However, this action loses customer trust in the long run and does not lead to customer satisfaction.
	\item \textbf{Clarity When Aiming for 'Customer Satisfaction':}
	      Setting the goal as 'customer satisfaction' clarifies the activity process. A specific direction for activities is derived, such as: 'First, who are the customers we should satisfy (= defining the \textbf{target customer})?' 'What will satisfy those target customers (= analyzing needs)?' 'To that end, what kind of product/service should we develop?'
\end{itemize}
Therefore, in marketing, profit is regarded as the 'result', and customer satisfaction is emphasized as the 'guiding principle (objective)' that leads to that result.
\subsection{Application and Case Analysis}
\subsubsection{Case: The Short-Sighted Trap of 'Profit Maximization'}
In the lecture, the case of '\textbf{slightly compromising on raw material quality}' was presented as a common trap when focusing solely on 'profit maximization'.
This operation might lower product costs and contribute to profit maximization in the short term. However, this decision lacks the perspective of 'customer satisfaction'. If customers notice the decline in quality, trust in the product is lost, consequently causing customer churn and damaging long-term profits.
This is a prime example of how a company's specific operational decisions (in this case, procurement or manufacturing) can be diametrically opposed depending on whether the marketing objective is set as 'profit' or 'customer satisfaction'.
\subsection{Deeper Background and Lessons}
\textbf{\paragraph{Detour Topic Name (Divergence from Main Theory)}}
(This lecture's text focused on the main theory and did not include any particular detour topics.)
\textbf{\subsubsection{AI Supplement: Expanding on Key Points}}
In this lecture, 'profit' and 'customer satisfaction' were explained in contrast, but the perspective of \textbf{Philip Kotler}, a giant in marketing theory, will be supplemented regarding the relationship between the two.
Kotler defines the marketing concept as '\textbf{achieving profits through customer satisfaction}'. This indicates that customer satisfaction and profit are not in a trade-off relationship; rather, \textbf{customer satisfaction is the very source of long-term profit}. In other words, customer satisfaction is positioned as the 'objective (guiding principle for activities)', and profit is the 'result indicating that those activities were correct (and the condition for corporate survival)'.
However, in practice, a balance between the two is crucial. If, in the pursuit of 'customer satisfaction', a company continues to provide excessive quality or service, costs will increase and squeeze profits. Companies must make strategic decisions—within the constraints of their own resources (people, materials, money)—about '\textbf{which target customers}' to provide '\textbf{what level of satisfaction}' to, '\textbf{more efficiently than competitors}'. This is why the importance of 'defining the target customer' was pointed out in the lecture.
\subsection{Conclusion}
In this lecture, we learned that the market, the 'stage' of marketing, has a dual meaning: a 'place of competition' and an 'aggregation of customers'.
We also understood why the primary purpose of marketing is placed on 'customer satisfaction' rather than 'profit'. This is because, unlike the ambiguous goal of 'profit', the guideline of 'customer satisfaction' clarifies the direction of specific activities, such as product development and service design based on \textbf{target customer needs}.
The practical lesson derived from this is the importance of always maintaining a marketing perspective when making decisions in daily operations (e.g., cost reduction)—even if it contributes to short-term profit, one must ask: '\textbf{Does this decision improve (or at least not harm) our target customers' satisfaction?}'
\subsection{Key Terms List}
People:
Philip Kotler (AI Supplement)
\vspace{\baselineskip}
Theories/Concepts:
Market (dual meaning), Place of Competition, Aggregation of Customers, Market Needs, Profit, Customer Satisfaction, Target Customer
\subsection{Comprehension Quiz}
\begin{enumerate}[label=\arabic*.]
	\item What concept, described as the 'stage' for marketing activities, was explained in the lecture as having a dual meaning?
	\item When 'market' is used in the context 'the market environment is harsh', what does it mean as a 'place'?
	\item When companies devise competitive strategies, which of the two meanings of market is primarily used?
	\item When 'market' is used in the context 'market needs are changing', whose 'aggregation' does it refer to?
	\item When companies analyze needs for new product development, which of the two meanings of market is primarily used?
	\item In marketing theory, what is the strategic reason for setting 'customer satisfaction', rather than 'profit', as the primary objective?
	\item Why does setting 'profit maximization' as the objective (guideline) tend to make the direction of specific activities ambiguous?
	\item What example of an operation, shown in the lecture, runs counter to 'customer satisfaction' when pursuing short-term 'profit maximization'?
	\item What is the logic behind the concern that the decision in the lecture's case ('slightly compromising on raw material quality') will harm long-term profits?
	\item By setting the marketing objective as 'customer satisfaction', what aspect of the company's activity process is clarified?
	\item When 'customer satisfaction' is the guiding principle, what is the definition that a company must strategically make first ('who to satisfy')?
	\item (AI Supplement) Why was 'defining the target customer' pointed out as important in the lecture? It is because companies have constraints on what (resources)?
	\item (AI Supplement) Based on Kotler's definition, what kind of relationship (trade-off or causal) do customer satisfaction and profit have?
	\item (AI Supplement) Customer satisfaction is the source of long-term profit, but why, in practice, can its pursuit sometimes squeeze short-term profits?
	\item As the lecture's conclusion, why is the 'customer satisfaction' perspective required in daily operations such as cost reduction?
\end{enumerate}
\subsubsection*{Answer Key}
1. The Market, 2. A 'place' where (fierce) competition between companies occurs., 3. A 'place' of competition., 4. An 'aggregation' of customers., 5. An 'aggregation' of customers., 6. Because the 'customer satisfaction' guideline clarifies the direction of specific activities based on target customer needs., 7. Because there are countless means of profit maximization (e.g., cost cutting), and short-term means do not necessarily lead to customer satisfaction (long-term profit)., 8. Slightly compromising on raw material quality., 9. Because customers will notice the quality decline, lose trust, and customer churn will occur., 10. The direction of activities (the process of what to do)., 11. The target customer (definition of)., 12. Resources (people, materials, money)., 13. A causal relationship (customer satisfaction is the source of long-term profit)., 14. Because excessive quality or service can increase costs., 15. To prevent the pursuit of short-term profit (cost reduction) from harming target customer satisfaction and losing long-term profit.
\section{Market Concepts in Marketing Theory}
\subsection{Introduction}
This lecture deals with the fundamentals of marketing theory as an academic discipline for understanding corporate activities in the market. The specific aim is to understand the three main characteristics of the market that marketing activities presuppose: the '\textbf{Differentiated Market}', the '\textbf{Segmented Market}', and the '\textbf{Changing Market}'. It will clarify how these characteristics form the basis of a company's \textbf{non-price competition} and product strategies.
\subsection{Key Concepts and Points}
In marketing theory, the market in which companies operate is presupposed to have the following three characteristics.
\subsubsection{The Differentiated Market}
A \textbf{differentiated market} refers to a market where companies conduct various marketing activities (design, features, advertising, limited sales channels, etc.) to make customers perceive their products as different from competitors' products.
In this market, consumers respond to corporate marketing efforts and form specific \textbf{preferences}. As a result, they are assumed to develop \textbf{non-substitutable preferences} for a specific company's products (a state where they can only be satisfied by that company's product).
Under these conditions, competition between companies shifts away from price and toward how to increase the \textbf{added value} of the product (design, functionality, brand image, etc.). In other words, \textbf{non-price competition} becomes central.
In contrast, in a \textbf{homogeneous market} (a market where consumers perceive all sellers' products as the same), consumers choose the lowest-priced product, so competition inevitably becomes \textbf{price competition}, and profits are easily squeezed. Marketing is the activity of escaping this homogeneous market and increasing profits in a market where differentiation is possible.
\subsubsection{The Segmented Market}
It is self-evident that differences exist in consumer preferences and needs within a given product category. In marketing theory, it is believed that groups of consumers (sub-markets) with the same preferences can be identified based on these preference differences.
This process is called \textbf{market segmentation}. Market segmentation is conducted by classifying consumers based on criteria such as gender, age, lifestyle, and values.
The sub-markets identified by this segmentation are called \textbf{market segments}. Consumers within a segment are assumed to have similar (or homogeneous) characteristics and preferences.
By developing and providing products specialized for the preferences of a specific segment, companies can create non-substitutable demand clearly differentiated from competitors, enabling them to engage in non-price competition.
\subsubsection{The Changing Market}
The market is not seen as static, but as something that constantly changes over time. The 'changes' referred to here mainly point to two things: '\textbf{changes in consumer preferences}' and '\textbf{changes in the nature of corporate competition}'.
Companies must predict these changes and formulate marketing plans through market analysis. To avoid unpredictability and increase the effectiveness of strategy formulation, it is necessary to extract the \textbf{stable patterns} behind market changes.
The representative pattern for this is the concept of the \textbf{Product Life Cycle (PLC)}. The PLC categorizes the market changes from a product's introduction to its decline into stages such as 'Introduction', 'Growth', 'Maturity', and 'Decline'. The market characteristics of each stage (demand growth, competitive situation, etc.) are assumed to be common, and it is thought that companies can deploy marketing strategies (product improvement, pricing, promotion, etc.) appropriate for each stage.
\subsection{Application and Case Analysis}
The concepts presented in the lecture can be applied to many real-world corporate activities.
\subsubsection{Case of Differentiation: Apple Inc.}
Apple is a typical example of creating a 'differentiated market'. In the smartphone market, the company differentiates itself clearly from other products (Android devices) not just through feature (spec) competition, but through its proprietary OS (iOS), sophisticated design, ecosystem (App Store, iCloud), and powerful brand advertising. Consumers are paying not just for the price (homogeneous market), but for the unique \textbf{added value} and experience (non-substitutable preference) that Apple products provide.
\subsubsection{Case of Segmentation: Diversification by Beverage Manufacturers}
The beverage market is a prime example of a 'segmented market'. For example, The Coca-Cola Company identifies \textbf{market segments} based on diverse criteria such as age, gender, health consciousness, and consumption occasion, offering optimized products for each. Examples include not only the traditional 'Coca-Cola' but also 'Coke Zero' (health-conscious, male target), 'Diet Coke' (zero-calorie preference), or 'Ayataka' (specific preference in the green tea market).
\subsubsection{Case of Change: The Digital Camera Market and PLC}
The concepts of the 'changing market' and the \textbf{Product Life Cycle} can be seen in the transition of the digital camera market. Compact digital cameras, once the market mainstream, entered the 'decline' stage as the market rapidly shrank due to the dramatic improvement of smartphone camera functions (a change in consumer preference, emergence of a substitute). Meanwhile, mirrorless interchangeable-lens cameras entered the 'growth' stage thanks to new technological innovations, forcing companies to reallocate resources and change strategies.
\subsection{Deeper Background and Lessons}
\subsubsection{Organizing the Lecture's Core Premises}
\textbf{\paragraph{The Absolute Importance of 'Differentiation'}}
In this lecture, 'differentiation' is emphasized as the most important concept in marketing. For companies to increase profits, it is essential to avoid homogeneous markets prone to price competition and strive to create a \textbf{differentiable market} themselves.
\textbf{\paragraph{The 'Three Fundamental Axes' of Marketing Strategy}}
As the lecture concludes, individual product strategies, price strategies, advertising strategies, etc., that companies consider are all deployed upon the three fundamental axes (market premises): the 'differentiated market', the 'segmented market', and the 'changing market'. Therefore, when planning any marketing activity, it is extremely important to return to these three fundamental axes to analyze the market environment.
\subsubsection{\textbf{AI Supplement: Expanding on Key Points}}
This lecture's transcript focused on the three characteristics of the 'market' that are prerequisites for marketing activities. However, it lacked sufficient mention of what specific strategies companies execute after understanding these market characteristics. The following two particularly important points are supplemented.
\textbf{\paragraph{The Overall Picture of STP Strategy}}
The lecture covered \textbf{Market Segmentation (S)}, but this is only one part of the process known as STP strategy. After conducting Segmentation (S), companies perform:
\begin{enumerate}
	\item \textbf{Targeting (T)}: Evaluating their own strengths and the attractiveness of the segmented markets, and selecting the market segment to target.
	\item \textbf{Positioning (P)}: Designing and communicating the product and brand image so that the company's product occupies a unique and clear position in the minds of the selected target market's customers, relative to competing products.
\end{enumerate}
Only through this entire S->T->P process do 'differentiation' and 'non-price competition' become concretely executable.
\textbf{\paragraph{The Marketing Mix (4Ps)}}
After the strategic direction of 'who to provide what value to' is determined by STP strategy, companies need to combine executional tactics to deliver that value concretely to the market. This is the \textbf{Marketing Mix}, generally composed of four elements called the \textbf{4Ps}.
\begin{itemize}
	\item \textbf{Product}: The product/service that meets the target's needs (quality, design, features).
	\item \textbf{Price}: Pricing appropriate for that value (list price, discounts).
	\item \textbf{Place}: The location/method for the target to easily acquire the product (channels, location).
	\item \textbf{Promotion}: Activities to communicate the product's value to the target and encourage purchase (advertising, PR, personal selling).
\end{itemize}
In each stage of the 'Product Life Cycle' mentioned in the lecture, companies respond to market changes by optimizing this 4P combination (mix).
\subsection{Conclusion}
In these lecture notes, we analyzed the three essential characteristics of the market that marketing theory assumes as premises for activity: \textbf{differentiation}, \textbf{segmentation}, and \textbf{change}. Companies must deeply understand these market characteristics to avoid homogeneous price competition and secure sustainable profits.
As confirmed in the 'Deeper Background', these three characteristics are the 'fundamental axes' that form the basis for individual strategies (product strategy, advertising strategy). As a practical lesson, we practitioners must constantly ask whether the market we face is becoming 'homogeneous', whether we are overlooking opportunities for 'segmentation' of customer needs, and whether we are predicting patterns of 'change' (like the PLC) and reflecting them in our strategies (STP and 4Ps).
\subsection{Key Terms List}
\textbf{People:}
(None)
\vspace{\baselineskip}
\textbf{Theories/Concepts:}
Differentiated Market, Homogeneous Market, Added Value, Price Competition, Non-price Competition, Market Segmentation, Market Segment, Non-substitutable Preference, Changing Market, Consumer Preference, Product Life Cycle (PLC)
\subsection{Comprehension Quiz}
\begin{enumerate}[label=\arabic*.]
	\item What is the market state called where marketing activities prevent product commoditization (homogeneous market) and enable non-price competition?
	\item In a 'homogeneous market', what is the main form of competition companies face, and what tends to happen to profits as a result?
	\item What is the strategic reason marketing aims to escape from a 'homogeneous market'?
	\item As seen in the Apple case, what form of competition becomes possible when a company increases 'added value' (differentiates)?
	\item In a differentiated market, what did the lecture call the state of 'cannot be satisfied unless it is that product', which becomes the reason consumers choose a specific product (and not just price)?
	\item What is the strategic purpose of viewing the market as a 'segmented market' (market segmentation)?
	\item As in the beverage maker case, what is the strategic advantage of providing products specialized for a specific 'market segment'?
	\item What two market characteristics, presupposed by marketing theory, are utilized in common by both the Apple (differentiation) and Coca-Cola (segmentation) cases?
	\item When viewing the market as a 'changing market', what are the two main changes companies must analyze?
	\item Why do companies try to analyze market changes using stable patterns like the 'Product Life Cycle (PLC)'?
	\item As shown in the digital camera case, what strategic response should companies take for a product (compact digital camera) that has entered the PLC 'decline' stage?
	\item (AI Supplement) After market segmentation (S), what do companies specifically try to achieve through 'Targeting (T)' and 'Positioning (P)'?
	\item (AI Supplement) What kind of activity is 'Positioning'? It involves creating what kind of perception of the company's product in the target customer's mind, relative to competitors?
	\item (AI Supplement) After deciding the strategic direction of 'who to provide what to' with STP strategy, what is the combination of the four specific tactical tools (4Ps) used to execute it called?
	\item The 'three fundamental axes' of marketing strategy shown in the lecture (differentiation, segmentation, change) are premises for companies to avoid what, and engage in what kind of competition?
\end{enumerate}
\subsubsection*{Answer Key}
1. Differentiated market, 2. Price competition, and the squeezing of profits., 3. To avoid price competition and increase profits., 4. Non-price competition, 5. Non-substitutable preference, 6. To respond to diverse consumer preferences (needs) and enable non-price competition., 7. By specializing in the preferences of a specific segment, it becomes possible to differentiate from other companies (create non-substitutable demand)., 8. Differentiated market, Segmented market, 9. 'Changes in consumer preferences' and 'changes in the nature of corporate competition'., 10. To avoid the difficulty of predicting market changes and increase the effectiveness of strategy formulation., 11. Reallocation of resources and change in strategy (e.g., focusing on mirrorless cameras)., 12. The concrete execution of differentiation and non-price competition., 13. An activity to occupy (or have them perceive) a unique and clear position., 14. Marketing Mix (4Ps), 15. Premises for avoiding homogeneous price competition and engaging in non-price competition (differentiation, segmentation, responding to change).
\section{The Two Aspects of Marketing}
\subsection{Introduction}
The purpose of this lecture is to understand the two fundamentally different aspects that must be considered when formulating marketing strategy: the '\textbf{Market Aspect}' and the '\textbf{Relationship Aspect}'. Depending on the customer type (an unspecified majority or a specified few), the focus of the marketing activities required of a company differs radically. In these notes, these two aspects will be contrasted using cases (Calbee and Fuserashi), and the strategic approaches that are important in each will be analyzed.
\subsection{Key Concepts and Points}
Marketing activities are broadly classified into the following two aspects, based on the nature of the target customers.
\subsubsection{The Market Aspect}
The \textbf{Market Aspect} mainly corresponds to markets where the customers are an '\textbf{unspecified majority}' and individual customers are hard to 'see', such as in \textbf{consumer goods} (e.g., potato chips) manufacturing.
\begin{itemize}
	\item \textbf{Customer Characteristics}: Customers are innumerable, and the company does not grasp the details of individual customers.
	\item \textbf{Key Activities}:
	      \begin{enumerate}
		      \item \textbf{Customer Analysis}: Analyzing the general \textbf{behavioral patterns} of innumerable customers and classifying ( \textbf{grouping} / \textbf{market segmentation}) them into cohorts with common needs or characteristics.
		      \item \textbf{Marketing Research}: Researching what kind of products should be developed and offered to specific segments.
		      \item \textbf{Planning and Executing the Marketing Mix (4Ps)}:
		            \begin{itemize}
			            \item \textbf{Advertising/Promotion}: Designing advertisements that resonate with the target segment based on analysis results, and deploying them through mass media. Price adjustments (discounts, bonuses) are also included.
			            \item \textbf{Distribution/Sales Floor Management}: Distributing products to locations where consumers can easily purchase them (e.g., convenience stores, supermarkets) and conducting negotiations and display management to secure '\textbf{eye-catching}' shelf space (e.g., eye-level shelves).
		            \end{itemize}
	      \end{enumerate}
	\item \textbf{Relevant Fields}: Analysis of this aspect is mainly handled in fields such as '\textbf{Marketing Management}', '\textbf{Strategic Marketing}', and '\textbf{Marketing Research}'.
\end{itemize}
However, even for industrial goods, an approach close to the market aspect is necessary when widely selling \textbf{general-purpose components} like '\textbf{screws}'.
\subsubsection{The Relationship Aspect}
The \textbf{Relationship Aspect} mainly corresponds to markets where customers are limited to a '\textbf{specified few}' (e.g., a dozen or so companies like Nissan, Toyota) and the customers' 'faces' are clearly visible, such as in \textbf{industrial goods} (e.g., automotive parts) manufacturing.
\begin{itemize}
	\item \textbf{Customer Characteristics}: The number of customers is limited, and the company builds close relationships with the person in charge at each customer (typical of BtoB marketing).
	\item \textbf{Key Activities}:
	      \begin{enumerate}
		      \item \textbf{Sales Activities (Personal Selling)}: Rather than advertising strategy, individual relationship-building by skilled sales representatives becomes crucial. This involves understanding what new products the customer will develop next season, what parts they will need, and lobbying for the adoption of one's own products.
		      \item \textbf{Customer Relationship Management (CRM)}: Continuing and developing transactions with a limited number of important customers becomes the top priority. It involves increasing customer satisfaction and maintaining/managing long-term relationships.
		      \item \textbf{Organizational Response (Interdepartmental Cooperation)}: To respond quickly to sudden customer requests (e.g., shortening \textbf{delivery dates}, adjusting production volume), close \textbf{interdepartmental cooperation}—not just with the sales department, but also with production and support departments—and an organizational management structure that enables this (as a company \textbf{strength}) become critical.
	      \end{enumerate}
	\item \textbf{Relevant Fields}: This aspect is handled with emphasis in 'Sales Management', 'Distribution Channel Theory', and 'Customer Relationship Management (CRM)'.
\end{itemize}
\subsection{Application and Case Analysis}
In the lecture, the following two companies were raised as cases to contrast these two aspects.
\subsubsection{Case of the Market Aspect: Calbee, Inc.}
The potato chips Calbee sells are typical consumer goods. The customers are 'you and I'—that is, an '\textbf{unspecified majority}' worldwide.
What is important for Calbee is not knowing the face of individual customers, but analyzing \textbf{general behavioral patterns} through market research, such as 'what flavors and sizes are preferred'. Then, \textbf{Market Aspect} marketing activities—developing products based on those analysis results, running \textbf{advertisements} like TV commercials, and securing '\textbf{eye-catching}' shelf space in supermarkets and convenience stores—determine sales.
\subsubsection{Case of the Relationship Aspect: Fuserashi Co., Ltd.}
Fuserashi is a company that manufactures and sells automotive parts (like special bolts). Its customers are limited to 16 companies worldwide (at the time of the lecture), such as Toyota, Daihatsu, and Nissan.
What is important for Fuserashi is not advertising aimed at an unspecified majority, but \textbf{sales activities} that build close relationships with the person in charge at specific customers (automakers). Co-developing parts to match the cars the customer will develop next, and having an \textbf{organizational management structure} that can handle sudden \textbf{delivery date} changes, becomes the source of competitiveness in the \textbf{Relationship Aspect}.
\subsection{Deeper Background and Lessons}
\subsubsection{Organizing the Lecture's Points}
\textbf{\paragraph{The Resolution of the Customer's 'Face'}}
The core of the lecture is that marketing strategy changes fundamentally depending on how clearly the target customer's 'face' can be seen. The face of the consumer buying potato chips is not visible (= \textbf{Market Aspect}), but the face of the purchasing manager buying auto parts is visible (= \textbf{Relationship Aspect}).
\textbf{\paragraph{The Intermediate Approach in Service Sales}}
The case of cosmetics sales was supplementarily mentioned in the lecture. This belongs to the Market Aspect (unspecified majority), but it differs from pure consumer goods in that a \textbf{salesperson} is involved at the point of purchase. In this case, the salesperson's \textbf{service quality} (product knowledge, customer service attitude) greatly influences sales, so activities to \textbf{manualize} the service and standardize/manage quality through employee \textbf{education} become important. This can be called an approach within the Market Aspect that incorporates human elements (Relationship Aspect elements).
\textbf{\paragraph{Connection to This Course's Overall Structure}}
This course (as a whole) is structured with these two aspects as its two wheels. First, students learn the theories of the '\textbf{Market Aspect}' (the fundamentals of marketing management), such as consumer behavior research, segmentation, and product differentiation. After that, they learn the theories of the '\textbf{Relationship Aspect}', such as sales activities, interdepartmental cooperation, and customer relationship management. Finally, the structure flows toward considering how to apply these two aspects in the new environment of the international market (overseas expansion).
\subsubsection{\textbf{AI Supplement: Expanding on Key Points}}
The lecture explained the importance of the 'Relationship Aspect', but it lacked mention of the central theoretical framework that supports this aspect. The following two concepts, essential for deepening the understanding of BtoB (Business-to-Business) marketing, are supplemented.
\textbf{\paragraph{Relationship Marketing}}
The activity of 'maintaining and managing customer relationships' mentioned in the lecture is theorized in business administration as '\textbf{Relationship Marketing}'. This is an approach that focuses on building, maintaining, and enhancing long-term, favorable 'relationships' with customers, in contrast to traditional marketing (Market Aspect), which tries to maximize one-off 'transactions'. Fuserashi's activities aimed at continuing transactions with Toyota in the Relationship Aspect are precisely the practice of this theory.
\textbf{\paragraph{The Buying Center}}
In the Relationship Aspect (especially BtoB), the customer a salesperson faces is not a single 'person in charge'. In the actual corporate purchasing process, many stakeholders are involved. This decision-making group is called the '\textbf{Buying Center}'. The Buying Center includes 'Users' (who actually use the product), 'Buyers' (who handle the contracting), 'Influencers' (who determine technical specs), 'Deciders' (who give final approval), and 'Gatekeepers' (who control the flow of information). Advanced sales activity in the Relationship Aspect means understanding the dynamics of this complex Buying Center and appealing appropriately to each stakeholder.
\subsection{Conclusion}
In these lecture notes, we analyzed how marketing activities are broadly divided into two different aspects: the '\textbf{Market Aspect}' (an unspecified majority of 'faceless' customers) and the '\textbf{Relationship Aspect}' (a specified few 'visible' customers).
In the Market Aspect, strategic \textbf{marketing management} (advertising, promotion, channel management) based on \textbf{research} and \textbf{grouping} is the key to success. On the other hand, in the Relationship Aspect, \textbf{sales activities} (personal selling) for close \textbf{customer relationship management} and \textbf{interdepartmental cooperation} to meet customer demands—rather than advertising—become the source of competitive advantage.
As a practical lesson, we must soberly evaluate which aspect our own business primarily belongs to. Discerning whether one should compete in the Market Aspect (like general-purpose screws, even as an industrial goods maker) or specialize in the Relationship Aspect (relationship marketing) is the first step toward optimal allocation of marketing resources and strategy formulation.
\subsection{Key Terms List}
\textbf{People:}
(None)
\vspace{\baselineskip}
\textbf{Theories/Concepts:}
Market Aspect, Relationship Aspect, Unspecified Majority, Specified Few, BtoC (Business-to-Consumer), BtoB (Business-to-Business), Grouping (Market Segmentation), Consumer Behavior Research, Marketing Management, Marketing Research, Sales Activities (Personal Selling), Customer Relationship Management (CRM), Interdepartmental Cooperation, Relationship Marketing, Buying Center
\subsection{Comprehension Quiz}
\begin{enumerate}[label=\arabic*.]
	\item When formulating marketing strategy, what are the two fundamental aspects classified by the type of target customer (whether their 'face' is visible or not)?
	\item In the 'Market Aspect' (e.g., Calbee), because individual customers' faces are not visible (unspecified majority), what analytical activity (approach to customers) becomes strategically important?
	\item In the 'Relationship Aspect' (e.g., Fuserashi), because the number of customers is limited (specified few), what activity becomes more important than mass advertising?
	\item In the Market Aspect, why is 'Grouping (Market Segmentation)' essential? It is due to what characteristic of the target customers?
	\item In the Relationship Aspect, what organizational strength, besides the sales department, is crucial for responding to sudden changes from customers (BtoB), such as delivery date modifications?
	\item In the Calbee case, why is the activity of securing 'eye-catching' shelf space in convenience stores (distribution/sales floor management) emphasized?
	\item In the Fuserashi case (Relationship Aspect), why can 'interdepartmental cooperation' (and the organizational management structure that enables it) become a source of competitiveness?
	\item Even for industrial goods (BtoB), why are general-purpose components like 'screws' thought to require an approach closer to the 'Market Aspect'?
	\item What is the academic field mentioned in the lecture that handles the analysis of the 'Market Aspect', such as marketing research and the planning/execution of the 4Ps?
	\item Why is the case of cosmetics sales considered an intermediate approach between the 'Market Aspect' and the 'Relationship Aspect'?
	\item Derived from the contrast between the Fuserashi (BtoB) and Calbee (BtoC) cases, what is the fundamental factor (the resolution of the customer's 'face') that determines marketing strategy?
	\item Why is this course (as a whole) structured to first learn the theory of the 'Market Aspect' and then the theory of the 'Relationship Aspect'?
	\item (AI Supplement) 'Relationship Marketing', the theoretical pillar of the 'Relationship Aspect', focuses on building what, rather than maximizing what (in contrast to traditional Market Aspect marketing)?
	\item (AI Supplement) In BtoB (Relationship Aspect) sales activities, why is understanding the 'Buying Center' important?
	\item As the lecture's conclusion, why is discerning whether one's own business is in the 'Market Aspect' or 'Relationship Aspect' the first step in strategy formulation?
\end{enumerate}
\subsubsection*{Answer Key}
1. The Market Aspect and the Relationship Aspect, 2. Analysis of general customer behavioral patterns and grouping (market segmentation)., 3. Sales activities (personal selling) and Customer Relationship Management (CRM)., 4. Because customers are an 'unspecified majority' and individual faces are not visible., 5. Interdepartmental cooperation (and the organizational management structure that enables it)., 6. Because customers (unspecified majority) purchase products in 'eye-catching' locations (making distribution channels and promotion crucial)., 7. Because responding organizationally to sudden requests from customers (specified few) directly links to customer satisfaction and relationship maintenance., 8. Because the customers are not a specified few, but an unspecified majority (general-purpose)., 9. Marketing Management (or Strategic Marketing, Marketing Research)., 10. Because the customers are an unspecified majority (Market Aspect), but a salesperson is involved at the point of purchase, and their service quality (a Relationship Aspect element) has an influence., 11. The difference between whether the target customers are an 'unspecified majority' or a 'specified few'., 12. To learn the fundamentals of marketing (Market Aspect) and its application (Relationship Aspect) in a balanced manner, as two wheels., 13. Building long-term relationships, rather than maximizing one-off transactions., 14. Because the purchasing decision is not made by one person in charge, but by a group involving various stakeholders (users, deciders, etc.)., 15. Because the important strategies (advertising vs. sales) and the necessary resource allocation differ fundamentally depending on which aspect one belongs to.
\end{document}
\documentclass[uplatex,a4j,12pt,dvipdfmx]{jsarticle}
\usepackage{amsmath,amsthm,amssymb,bm,color,enumitem,mathrsfs,url,epic,eepic,ascmac,ulem,here,ascmac}
\usepackage[letterpaper,top=2cm,bottom=2cm,left=3cm,right=3cm,marginparwidth=1.75cm]{geometry}
\usepackage[english]{babel}
\usepackage[dvipdfm]{graphicx}
\usepackage[hypertex]{hyperref}
\title{Marketing Chapter 1 Lecture Notes}
\author{M. O.}
\date{\today}
\begin{document}
\maketitle
\tableofcontents
\section{Basic Definitions of Marketing}
\subsection{Introduction}
These notes organize the content of the first lecture in the MBA core course, 'Marketing Theory.' While marketing is central to business administration, its definition is often vaguely understood. This lecture provided an overview of the basic definition of marketing, corrected common misconceptions associated with it, and examined how the concept was historically established and developed. The purpose of these notes is to systematically organize these points and gain a deeper understanding of the essence of marketing activities.
\subsection{Key Concepts and Points}
\subsubsection{The Difference Between Marketing and Management}
As an introduction, the lecture highlighted the difference in focus between \textbf{Marketing} and \textbf{Management}.
\begin{itemize}
	\item \textbf{Management}: Primarily focuses on the internal organization, referring to managers and executives motivating employees and efficiently managing and operating the organization (\textbf{people, goods, money, and information}).
	\item \textbf{Marketing}: Primarily focuses on the exterior of the company, namely \textbf{customers} and the \textbf{market}, referring to all corporate activities directed toward the market.
\end{itemize}
Although the two fields have been merging in recent years, the uniqueness of marketing theory lies in its explicit focus on and thinking about the 'market.'
\subsubsection{Common Misconceptions About Marketing}
The lecture pointed out two major, frequently seen misconceptions regarding marketing.
\paragraph{Misconception 1: Marketing = Selling}
The view that equates marketing with mere \textbf{Selling} or sales promotion activities is incorrect. It was explained in the lecture that the ideal of marketing is 'to \textbf{make selling superfluous}.'
In other words, unlike selling, which is the activity of 'pushing' products relying on the ability of individual salespeople, marketing is a more comprehensive activity that \textbf{creates demand} and builds a system and environment where products sell naturally (e.g., \textbf{demand research}, \textbf{product development}, \textbf{advertising}).
\paragraph{Misconception 2: Marketing = A Collection of Success Stories}
The view that marketing is a 'collection of success stories' or 'an accumulation of effective rules of thumb' is also not accurate. Marketing is a scientific academic discipline.
The lecture used the \textbf{skimming strategy} (a method of setting a high price when launching a new product and gradually lowering it to maximize profit) as an example, stating that the goal is not merely to know the rule, but that the essence of marketing theory is to logically analyze and consider '\textbf{why}' that rule is effective and '\textbf{under what conditions}' it becomes effective.
\subsubsection{Evolution of the American Marketing Association (AMA) Definition}
The definition of marketing has evolved along with changes in the times and social environment. The lecture introduced the evolution of the definition by the \textbf{American Marketing Association (AMA)}.
\begin{itemize}
	\item \textbf{1960s}: The business activities that manage the \textbf{flow of goods and services} from producer to consumer or user. The focus was on 'distribution.'
	\item \textbf{1985s}: Comprehensive activities for \textbf{creating market demand}. The active concept of 'creating demand' was introduced for the first time.
	\item \textbf{2004s}: An organizational function and a set of processes for \textbf{creating, communicating, and delivering customer value} and \textbf{managing customer relationships} for the benefit of the organization and its stakeholders. Value creation and \textbf{CRM (Customer Relationship Management)} became central.
	\item \textbf{2013s}: The activity, set of institutions, and processes for \textbf{creating, communicating, delivering, and exchanging} offerings that have value for customers, clients, partners, and society at large. Reflecting the spread of SNS, the concepts of '\textbf{dialogue}' and '\textbf{value co-creation},' where companies and customers create value interactively, were added.
\end{itemize}
\subsection{Application and Case Analysis}
\subsubsection{Case Study: Calbee's Potato Chips}
At the beginning of the lecture, Calbee's potato chips were raised as an example to grasp the image of a 'company skilled at marketing.'
The fact that many consumers choose Calbee over competing products (or feel that Calbee 'is skilled at marketing') can be analyzed as a reflection of the following marketing activities.
\begin{enumerate}
	\item \textbf{Advertising and promotional activities}: High \textbf{brand recognition} and favorable image formation through frequent TV commercials and campaigns.
	\item \textbf{Strong distribution channels}: Making the product 'readily available' everywhere, such as in convenience stores and supermarkets nationwide (ensuring \textbf{ease of access}).
\end{enumerate}
This case demonstrates that marketing is not merely about product quality but is a comprehensive activity that includes the management of recognition (advertising) and distribution (channels).
\subsection{Deeper Context and Lessons}
This section organizes topics branching off from the main lecture and background information to deepen understanding.
\subsubsection{Lecture Background: Four Factors in the Spread of Marketing}
The lecture cited four major environmental changes as the historical background for the birth and development of the marketing concept in late 19th-century America. These are extremely important for understanding why marketing activities became necessary and how they evolved.
\begin{enumerate}
	\item \textbf{Market Competition (Establishment of Mass Production)}:
	      Once \textbf{mass production systems} by large corporations were established, the era of 'if you make it, it will sell' ended, leading to excess supply. Initially, this was addressed by developing new sales destinations (new markets), but differentiation was difficult as competitors followed quickly. The focus of competition shifted to \textbf{new product development}, which requires time and cost to imitate, making research and analysis of customer needs crucial.
	\item \textbf{Development of Information Transmission Technology}:
	      Initially, a significant \textbf{information gap} existed between sellers and buyers, and companies sold using 'Hard Selling.' However, with the development of mass media such as newspapers, radio, television, and the internet, companies became able to widely disseminate information through \textbf{advertising}, while consumers also gained easy access to information. This led to the diversification and sophistication of marketing methods among companies.
	\item \textbf{Changes in Distribution Structure}:
	      In the past, \textbf{wholesalers} positioned between manufacturers and retailers held strong power, and manufacturers were easily drawn into price competition. To break this situation, manufacturers used advertising to appeal directly to consumers, establishing a specific \textbf{brand preference} (purchase by designation), thereby attempting to increase their bargaining power against wholesalers. Today, the balance of power in distribution is changing again with the rise of large-scale retailers (chain stores) like Aeon and e-commerce.
	\item \textbf{Changes in Demand (Economic Growth and Diversification)}:
	      With economic growth, the \textbf{income improvement} of the general public led to a nationwide expansion of the class with purchasing power. This created the necessity not just to supply goods, but to 'stimulate' demand. Furthermore, as society matured, demand diversified, making \textbf{market segmentation} based on gender, age, lifestyle, etc., and corresponding marketing methods essential.
\end{enumerate}
\subsubsection{Topics Diverging from the Main Thesis}
\textbf{\paragraph{Instructor's Specialization (Self-Introduction)}}
During the introduction, the instructor stated that his specialization is 'management of sales organizations and activities in \textbf{industrial goods companies} (BtoB)' and 'the \textbf{international marketing} of growing companies.'
\subsubsection{AI Supplement: Expansion of Key Points}
\textbf{\paragraph{Marketing Mix (4Ps) and 4Cs}}
While the lecture focused on the evolution of the AMA definition (creating, communicating, and delivering value), as an MBA foundation, there was a lack of reference to the '\textbf{Marketing Mix},' the framework for concretely executing and managing that value.
When putting marketing strategy into practice, the main elements controllable by the company are collectively called the \textbf{4Ps}.
\begin{itemize}
	\item \textbf{Product}: The value provided to the customer (product, service, quality, design)
	\item \textbf{Price}: The consideration the customer pays for that value (pricing, discounts)
	\item \textbf{Place}: The location where the customer can obtain the value (channel, location, logistics)
	\item \textbf{Promotion}: Activities to communicate the value to the customer and encourage purchase (advertising, PR, personal selling)
\end{itemize}
These are the means of embodying the 2004 AMA definition (creating, communicating, and delivering value) mentioned in the lecture.
Furthermore, as a framework reflecting the shift to 'dialogue' and 'customer perspective' emphasized in the lecture's 2013 definition, the \textbf{4Cs} advocated by Robert Lauterborn exist. This reframes the 4Ps from the customer's perspective.
\begin{itemize}
	\item \textbf{Customer Value}: (Replaces Product) Not what the company wants to sell, but the value the customer seeks.
	\item \textbf{Customer Cost}: (Replaces Price) Includes not only the monetary cost the customer pays, but also temporal and psychological costs.
	\item \textbf{Convenience}: (Replaces Place) The ease of access and use for the customer.
	\item \textbf{Communication}: (Replaces Promotion) Not one-way promotion from the company, but two-way dialogue with the customer.
\end{itemize}
This 4C perspective is strongly linked to the concept of '\textbf{value co-creation},' which was introduced in the lecture as the latest trend.
\subsection{Conclusion}
In this lecture, we learned that marketing is not merely selling or imitating success stories, but rather a process of scientifically analyzing '\textbf{why things sell}' targeting the \textbf{market} and \textbf{customers}, creating and communicating \textbf{value} that meets evolving customer needs, and ultimately '\textbf{co-creating}' value through '\textbf{dialogue}' with customers.
The instructor's specialization mentioned as background (\textbf{sales organizations in industrial goods companies}) at first glance seems to be a field different from consumer-oriented marketing. However, the evolution of the marketing definition learned in this lecture (from one-way distribution to relationship management, and then to dialogue and co-creation) and the perspective of the \textbf{4Cs} (especially Communication) raised in the AI supplement precisely suggest the importance of \textbf{continuous relationship building with customers} and solution proposal in BtoB (industrial goods) marketing.
The essential definition of marketing gained in this lecture can be said to be a practical lesson that forms the foundation of all corporate activities, regardless of BtoC or BtoB.
\section{Regarding the Purpose of Marketing and the Market}
\subsection{Introduction}
The previous lecture provided an overview of the basic definition of marketing and its historical development. This report organizes the content of the following lecture. This time, the focus is on the specific definition of the '\textbf{market}' that companies, the protagonists of marketing activities, confront, and on the fundamental '\textbf{purpose}' of those activities. In particular, it deeply analyzes the logical background of why the purpose of marketing is placed on '\textbf{customer satisfaction}' rather than profit (which is the ultimate goal of the company).
\subsection{Key Concepts and Points}
This lecture defined two core elements essential for understanding marketing theory: 'market' and 'purpose.'
\subsubsection{The Dual Meaning of the Market}
In marketing theory, the word '\textbf{market}' has two different meanings depending on the context.
\begin{enumerate}
	\item \textbf{Market as an arena of competition}:
	      This is the market as used in phrases like 'a harsh market environment.' It refers to the aspect of an '\textbf{arena}' (platform) where multiple \textbf{sellers (companies)} and multiple \textbf{buyers (customers)} exist, and companies engage in \textbf{fierce competition} over transactions with customers. Companies present advantageous conditions and actively appeal to customers in this arena.
	\item \textbf{Market as a collection of customers}:
	      This is the market as used in phrases like 'market needs are changing' or 'the market will not accept that product.' It refers to the aspect of a '\textbf{collection of customers}' that is the \textbf{target} of the company's marketing activities (new product development, promotion, etc.) and the \textbf{recipient} of those activities.
\end{enumerate}
\subsubsection{The Purpose of Marketing: Customer Satisfaction vs. Profit}
It goes without saying that the ultimate goal for a company's survival and development is to acquire \textbf{profit}. However, the lecture emphasized that the \textbf{primary goal of marketing activities} should be placed on \textbf{Customer Satisfaction}.
The reason for this is that goal-setting determines the subsequent \textbf{direction of activities}.
\begin{itemize}
	\item \textbf{When 'Profit Maximization' is the goal}:
	      The means are extremely diverse (e.g., new product development, cost reduction, asset sales). As a result, in the pursuit of short-term profit, there is a danger of choosing actions that lose customer trust in the long run, such as '\textbf{compromising slightly on the quality of raw materials}.'
	\item \textbf{When 'Customer Satisfaction' is the goal}:
	      The question 'Which customers are we satisfying?' inevitably arises. This clarifies the specific direction of marketing activities, such as:
	      \begin{enumerate}
		      \item Clarification of the \textbf{target customer}
		      \item Analysis of the \textbf{value} (needs) that will satisfy those customers
		      \item Specification of the \textbf{products/services} to be developed and provided
	      \end{enumerate}
	      This is the logical structure where the pursuit of customer satisfaction consequently leads to long-term profit (the company's survival and development).
\end{itemize}
\subsection{Application and Case Analysis}
\subsubsection{Case Study: The Trap of Short-Term Profit Pursuit}
The lecture exemplified the trap often fallen into when 'profit maximization' is the direct goal with the act of '\textbf{compromising slightly on the quality of raw materials}.'
This accurately illustrates a dilemma faced by many companies.
\begin{itemize}
	\item \textbf{Analysis}: For example, if a food manufacturer, citing rising costs, sets 'profit maximization (or maintenance)' as its top priority, the decision to switch to lower-quality raw materials with cheaper procurement costs may seem rational from an accounting perspective (securing short-term profit).
	\item \textbf{Marketing-based Evaluation}: However, this decision lacks the perspective of \textbf{customer satisfaction}. If the decline in taste or concerns about safety are perceived by customers, customer satisfaction will drop significantly, and customers will defect to competitors. As a result, sales will decrease, and the profit that was supposed to have been secured in the short term will be lost in the \textbf{long term}.
	\item \textbf{Implication}: Setting 'customer satisfaction' as the goal of marketing functions as a management guideline to prevent such short-sighted decision-making and to secure \textbf{sustainable profit}.
\end{itemize}
\subsection{Deeper Context and Lessons}
This section describes background information supplementing the main thread of the lecture and AI-based expansions of the points.
\subsubsection{Topics Diverging from the Main Thesis}
\textbf{\paragraph{The Diverse Means of 'Profit Maximization'}}
During the lecture, the instructor stated, 'When it comes to maximizing profit, there are an awful lot of methods.' The main point was to show the superiority of 'customer satisfaction,' but here the instructor deliberately cited the negative example of 'compromising on raw material quality.' This was a supplementary remark suggesting the reality of management: that a company's \textbf{goal-setting} (profit or customer satisfaction) is not just a slogan but has a direct impact on concrete, field-level actions, and sometimes even on \textbf{ethics} and \textbf{quality standards}.
\subsubsection{AI Supplement: Expansion of Key Points}
\textbf{\paragraph{STP Theory: The Process of Determining Target Customers}}
The lecture explained that setting 'customer satisfaction' as the goal determines the '\textbf{target customer}' and provides a direction for activities. However, it lacked reference to the strategic process of how to specifically determine that 'target customer' and link it to activities. Filling this gap is the \textbf{STP theory}, which forms the core of marketing strategy.
STP is a logical framework for analyzing the market as a 'collection of customers' (as defined in the lecture) and achieving 'customer satisfaction' efficiently and effectively.
\begin{description}
	\item[S (Segmentation)] :
	      The process of dividing the market as a 'collection of customers,' which is heterogeneous, into homogeneous small groups (\textbf{market segments}) based on common needs, attributes, purchasing behavior, etc.
	\item[T (Targeting)] :
	      Corresponds to the process of 'determining the target customer' mentioned in the lecture. From the divided segments, select the segment that best leverages the company's strengths and is most attractive, and decide on it as the \textbf{target market} (target customer group).
	\item[P (Positioning)] :
	      The process of designing and executing the marketing mix (4Ps) to establish a clear and desirable position in the \textbf{perception} of the selected target market (target customers) as having '\textbf{unique value}' compared to competing products/services.
\end{description}
The 'direction aiming for customer satisfaction' mentioned in the lecture means executing this STP process and establishing a clear \textbf{positioning}.
\subsection{Conclusion}
In this lecture, we confirmed that the market, as the '\textbf{stage}' for marketing activities, has a dual meaning as both an arena of competition and a collection of customers. As a more important point, we learned that the purpose of marketing activities, as a '\textbf{compass},' should be set on \textbf{customer satisfaction}, not short-term \textbf{profit}.
The reason is that the process of pursuing customer satisfaction prompts the clarification of '\textbf{target customers}' and the specification of the value to be provided to them, and as a result, prevents the company's \textbf{direction of activities} from going astray (e.g., avoiding easy compromises on quality).
The practical lesson from this lecture and the AI supplement (STP) is that in modern management, 'customer satisfaction' is not merely a slogan or a result indicator, but a \textbf{strategic starting point} for generating \textbf{sustainable profit}. The STP thought process of '\textbf{which customers to satisfy, and how}' is the very foundation of marketing strategy and the key to guiding a company in the right direction.
\section{Market Concepts in Marketing Theory}
\subsection{Introduction}
In the previous lecture, we learned why the purpose of marketing activities should be 'customer satisfaction.' This report analyzes the three essential characteristics of the market, the 'stage' for these activities, as presented in the lecture. In marketing theory, the market is not viewed statically as a simple place of transaction; rather, marketing theory constructs strategies upon three dynamic premises: '\textbf{differentiation},' '\textbf{segmentation},' and '\textbf{change}.' The purpose of this report is to clarify how these three market characteristics contribute to a company's non-price competition and profit generation.
\subsection{Key Concepts and Points}
The lecture cited the following three points as characteristics of the market that marketing theory presupposes.
\subsubsection{Differentiated Market}
The first premise of a market where marketing activities can function effectively is that \textbf{differentiation} is possible.
\begin{itemize}
	\item \textbf{Definition}: A market where companies, through various marketing efforts such as design, functionality, advertising, and limited sales channels, have made customers \textbf{perceive} their products as different from those of other companies.
	\item \textbf{Customer Response}: Consumers react to corporate marketing efforts and form \textbf{preferences} for specific brands or products. This creates a \textbf{non-substitutable preference} that cannot be satisfied by other products.
	\item \textbf{Impact on Competition}: The focus of competition shifts toward the creation of \textbf{added value}, making \textbf{price competition less likely}.
	\item \textbf{Contrast (Homogeneous Market)}: Conversely, in a \textbf{homogeneous market} where consumers perceive all sellers' products as the same, customers will choose the lowest-priced product, and competition will be confined to \textbf{price competition}, making it difficult to secure profits.
\end{itemize}
\subsubsection{Segmented Market}
The second premise is that the market is not uniform and \textbf{segmentation} is possible.
\begin{itemize}
	\item \textbf{Premise}: Differences in consumer \textbf{preferences} naturally exist within a product category.
	\item \textbf{Definition}: A market where companies can identify customer groups (=\textbf{market segments}) with common preferences or characteristics and provide products/services specialized for that segment's needs.
	\item \textbf{Method}: \textbf{Market segmentation}, which classifies customers by gender, age, lifestyle, etc.
	\item \textbf{Impact on Competition}: By focusing on a specific segment, the company can provide an optimal (=\textbf{differentiated}) product for those customers, and as a result, it becomes possible to engage in \textbf{non-price competition}.
\end{itemize}
\subsubsection{Changing Market}
The third premise is that the market is not static but constantly \textbf{changing}.
\begin{itemize}
	\item \textbf{Definition}: A market is not static; it is an arena where \textbf{consumer preferences} and \textbf{corporate competitive methods} change over time.
	\item \textbf{Corporate Response}: Companies must \textbf{predict} this change and formulate marketing plans through market analysis. In doing so, to avoid \textbf{unpredictability}, they are required to extract '\textbf{stable patterns}' that exist within the change.
	\item \textbf{Recognizing Stable Patterns}: A representative framework for this is the \textbf{Product Life Cycle (PLC)}.
\end{itemize}
\subsection{Application and Case Analysis}
\subsubsection{Specific Measures for 'Differentiation' to Avoid Price Competition}
The lecture stated that differentiation is the key to creating a situation where 'price competition is less likely.' The following were raised as specific activities companies undertake to escape a \textbf{homogeneous market}.
\begin{itemize}
	\item \textbf{Design and innovative features}: Establishing clear uniqueness through the physical characteristics of the product (e.g., Apple's product design).
	\item \textbf{Attractive advertising}: Appealing not only to the product's functional value but also to its brand image and emotional value, building the customer's \textbf{non-substitutable preference} (e.g., Nike's 'Just Do It' campaign).
	\item \textbf{Limited points of sale}: Managing distribution channels to maintain brand image and achieve differentiation, such as luxury brands selling only in department stores or direct boutiques.
\end{itemize}
All these activities are attempts to appeal to customer perception and create judgment criteria other than '\textbf{price}.'
\subsubsection{Pattern Recognition of Change: Strategic Use of the 'Product Life Cycle'}
The lecture introduced the \textbf{Product Life Cycle (PLC)} as a 'stable pattern' of market change. This is a theory that a product's sales and profits follow a common sequence and set of characteristics over time: 'Introduction,' 'Growth,' 'Maturity,' and 'Decline.'
Based on this PLC concept, companies can systematically plan strategies to respond to unpredictable market changes.
\begin{itemize}
	\item \textbf{Introduction Stage}: Product recognition is low, and the market needs to be established, so resources are concentrated on \textbf{advertising (Promotion)} and securing \textbf{distribution channels (Place)}.
	\item \textbf{Maturity Stage}: The market becomes saturated, and competition intensifies. Here, strengthening \textbf{differentiation (Product)} (e.g., adding features, new packaging) and pricing strategies (Price) that encourage brand switching become important.
\end{itemize}
In this way, the PLC functions as a framework that indicates the guidelines (marketing mix) for marketing activities that companies should take in response to the premise of a 'changing market.'
\subsection{Deeper Context and Lessons}
This section describes information supplementing the main thread of the lecture and AI-based expansions of the points.
\subsubsection{Topics Diverging from the Main Thesis}
\textbf{\paragraph{Lecture's Conclusion: The Importance of the Three Market Characteristics}}
In concluding the lecture, the instructor strongly emphasized, '(Differentiation, segmentation, change) these three are the basic axes,' and 'It is important to study these three basic axes thoroughly.' This can be interpreted as reflecting the instructor's teaching policy and personal view that establishing the foundational \textbf{view of the market (= what kind of place the market is)} is the most important thing in learning marketing theory, even before learning tactical theories like specific product or advertising strategies.
\subsubsection{AI Supplement: Expansion of Key Points}
\textbf{\paragraph{The Perspective of 'T (Targeting)' in STP Strategy}}
The lecture explained concepts equivalent to the 'S' and 'P' of \textbf{STP strategy}: the '\textbf{segmented market}' (= Segmentation) and the ability to provide a '\textbf{differentiated product}' as a result (= implication for Positioning). However, there was a lack of clear reference to the '\textbf{T}' (\textbf{Targeting}) that lies in the middle.
STP strategy consists of the following three steps:
\begin{enumerate}
	\item \textbf{Segmentation}: As explained in the lecture, dividing the market into segments with common needs.
	\item \textbf{Targeting (Selection of target market)}: This is the process of evaluating the divided segments based on the company's strengths (resources) and market attractiveness (size, growth potential, competitive situation), and \textbf{deciding which segment to pursue}.
	\item \textbf{Positioning}: Activities to establish a clear, desirable, and \textbf{differentiated} position from competing products in the minds of the chosen target customers.
\end{enumerate}
'Segmentation' does not directly lead to 'differentiation' as stated in the lecture; in reality, the decision-making of \textbf{Targeting}—'which segment to choose'—is indispensable. This \textbf{strategic choice} (= selection and concentration) is the key to investing limited management resources most effectively and advancing non-price competition advantageously.
\subsection{Conclusion}
In this lecture, we learned that marketing theory presupposes a market characterized by three features: '\textbf{differentiation},' '\textbf{segmentation},' and '\textbf{change}.' These premises serve as the 'basic axes' for companies when formulating marketing strategies.
The practical lesson from this lecture is that business management is a battle of how to escape from \textbf{price competition} in a \textbf{homogeneous market}, find a customer segment (target) with \textbf{non-substitutable preference}, and have them recognize \textbf{added value} (differentiation). And, because the rules of that battle (= consumer preferences and competitor moves) are \textbf{constantly changing} (PLC), strategies must be continuously reviewed. The three market characteristics presented in this lecture are the starting point for this continuous strategic thinking.
\section{Two Phases in Marketing}
\subsection{Introduction}
Marketing strategy requires fundamentally different approaches depending on the nature of the target customers. This lecture presented two basic 'phases' for understanding marketing activities. These are the '\textbf{Market Phase},' which targets a large, \textbf{non-specific} number of anonymous customers, and the '\textbf{Relationship Phase},' which emphasizes relationships with a \textbf{specific few} identifiable customers. The purpose of this report is to contrast these two phases and to analyze and organize the key marketing activities and management focuses in each, based on concrete examples.
\subsection{Key Concepts and Points}
The lecture explained that marketing activities are broadly divided into two main phases based on differences in customer type.
\subsubsection{Market Phase}
The \textbf{Market Phase} refers to marketing activities primarily targeting a \textbf{non-specific, large number} of 'anonymous' customers, such as in \textbf{consumer goods (BtoC)} companies.
\begin{itemize}
	\item \textbf{Core challenge}: To analyze and respond to the \textbf{general behavioral patterns} of the market as a whole, rather than individual customers.
	\item \textbf{Key activities}:
	      \begin{itemize}
		      \item \textbf{Customer analysis}: Grasping needs by \textbf{grouping} (segmenting) customers through \textbf{consumer behavior surveys} and \textbf{marketing research}.
		      \item \textbf{Strategy formulation}: Planning product development, \textbf{advertising} strategy, pricing, and \textbf{promotions} (sales promotion activities) based on the analysis.
		      \item \textbf{Execution management}: Management of executing the plan at the '\textbf{point of sale}.' When mediated by salespeople (e.g., cosmetics \textbf{PoV} [Point of Visual]), \textbf{manualizing service quality} and \textbf{education} become important. When not mediated by salespeople (e.g., snacks), securing \textbf{in-store displays} is important.
	      \end{itemize}
	\item \textbf{Related fields}: \textbf{Marketing Management Theory}, \textbf{Strategic Marketing}.
\end{itemize}
\subsubsection{Relationship Phase}
The \textbf{Relationship Phase} refers to marketing activities primarily targeting a \textbf{specific, small number} of 'identifiable,' limited customers, such as in \textbf{industrial goods (BtoB)} companies.
\begin{itemize}
	\item \textbf{Core challenge}: To \textbf{manage relationships with individual customers} and maintain/develop \textbf{continuous transactions}.
	\item \textbf{Key activities}:
	      \begin{itemize}
		      \item \textbf{Sales activities}: Relationship building at a deep level by \textbf{excellent sales representatives}, such as understanding the customer's (e.g., automaker's) next-generation product plans.
		      \item \textbf{Pursuit of customer satisfaction}: Responding quickly to individual customer requests (e.g., production adjustments, shortened delivery times) to increase satisfaction.
		      \item \textbf{Organizational management}: Building and managing an organizational response system, including \textbf{inter-departmental collaboration} between sales, manufacturing, and development departments, to handle individual requests.
	      \end{itemize}
	\item \textbf{Related fields}: \textbf{Sales Activities (Management)}, \textbf{Distribution Channel Management}, \textbf{Channel Relationship Management}.
\end{itemize}
\subsection{Application and Case Analysis}
\subsubsection{Case 1: Calbee (Typical Market Phase)}
Calbee's potato chips are a typical example of the \textbf{Market Phase}, as the customer base is a \textbf{non-specific, large number} ('everyone in the world').
\begin{itemize}
	\item \textbf{Analysis}: The faces of individual customers cannot be identified. Therefore, while mass advertising and promotions are important, the lecture particularly emphasized execution 'at the point of sale.'
	\item \textbf{Tactic}: Since there are no specialized salespeople, securing the \textbf{in-store display} (eye-level spots like the second or third shelf), where consumers unconsciously reach, becomes an extremely critical marketing activity that dictates sales. This requires the sales effort and negotiation power of the manufacturer (Calbee).
\end{itemize}
\subsubsection{Case 2: Fusenashi (Typical Relationship Phase)}
'Fusenashi' (mentioned by the instructor), a company that manufactures and sells automotive parts, is a
typical example of the \textbf{Relationship Phase} because its customers are \textbf{limited to 16 companies}, such as Nissan and Toyota.
\begin{itemize}
	\item \textbf{Analysis}: The customers are a specific few, and the representatives on both sides know each other.
	\item \textbf{Tactic}: Advertising strategies for a non-specific large audience are meaningless. Instead, nurturing \textbf{excellent sales representatives}, grasping the customer's next-generation development needs, and having their own parts incorporated is the core activity. Increasing \textbf{customer satisfaction} and ensuring \textbf{continued business} is the most important goal, and organizational support for this (e.g., responding to sudden delivery date changes) is a key marketing challenge.
\end{itemize}
\subsubsection{Case 3: Cosmetics (Execution Management in the Market Phase)}
Cosmetics are consumer goods (Market Phase), but unlike Calbee, \textbf{PoV} (salespeople) handle customers at the point of sale. This is a case showing that the form of 'execution management' differs even within the same market phase.
\begin{itemize}
	\item \textbf{Analysis}: The customer's purchasing decision is heavily influenced by the salesperson's service and \textbf{communication skills}.
	\item \textbf{Tactic}: It becomes a crucial marketing activity for the company (manufacturer) to '\textbf{manage}' a system that provides a consistent level of service at all stores nationwide by enhancing the \textbf{product knowledge} of salespeople, \textbf{manualizing} the \textbf{service quality} of sales talk and customer service, and enforcing \textbf{education}.
\end{itemize}
\subsection{Deeper Context and Lessons}
This section describes information supplementing the main thread of the lecture and AI-based expansions of the points.
\textbf{\paragraph{Divergent Topic: The Exception of the 'Market Phase' in Industrial Goods}}
The lecture supplemented that it is not a simple dichotomy of industrial goods = 'relationship phase.' For example, for \textbf{commodity parts} (general-purpose goods) used in any device, like \textbf{screws} (written as 'deji' in the text), the customers become a non-specific, large number of 'anonymous' companies. In this case, even if the product is an industrial good, the marketing activities are centered on the \textbf{Market Phase} approach (e.g., catalog distribution, web marketing, pricing). In other words, the classification axis is not the product category (consumer/industrial), but the \textbf{nature of the customer base} (non-specific large number / specific small number).
\textbf{\paragraph{Divergent Topic: The Structure of This Course (3 Parts)}}
The instructor explained the structure of the entire course based on this discussion.
\begin{enumerate}
	\item \textbf{Part 1 (Market Phase)}: Strategies targeting a non-specific, large number, such as consumer behavior surveys, market analysis, segmentation, and product differentiation.
	\item \textbf{Part 2 (Relationship Phase)}: Building relationships with specific customers, such as sales activities, sales department management, and \textbf{inter-departmental collaboration} for customer satisfaction.
	\item \textbf{Part 3 (Application)}: Learning how to deploy the above two phases in the context of the \textbf{international market} (entry of foreign companies, overseas expansion of Japanese companies).
\end{enumerate}
\subsubsection{AI Supplement: Expansion of Key Points}
\textbf{\paragraph{The Modern Significance of the 'Market Phase' in BtoB Marketing}}
The lecture contrasted the 'Market Phase' (BtoC) and the 'Relationship Phase' (BtoB). However, an important supplement is that in modern BtoB marketing, while the importance of the 'relationship phase' remains unchanged, the importance of 'market phase'-like approaches is increasing.
The BtoB example shown in the lecture (Fusenashi) was centered on deep relationship building by sales representatives (\textbf{Relationship Phase}). However, digital technology has brought significant changes, especially in acquiring new customers.
\textbf{Content marketing} (providing useful technical information on the web) and \textbf{inbound marketing} (a method of having customers find you via search) are precisely \textbf{market phase} activities conducted toward a 'faceless' potential customer base (= market).
Modern BtoB companies employ a process that links both phases, acquiring and nurturing leads through these 'market phase' activities, and only handing them over to \textbf{sales representatives} (relationship phase) when they reach a promising stage. Therefore, the two phases are not mutually exclusive; rather, a perspective that integrates them is essential, especially in BtoB.
\subsection{Conclusion}
This lecture presented an extremely practical framework for broadly classifying marketing activities into the '\textbf{Market Phase}' and the '\textbf{Relationship Phase}' from the fundamental perspective of the '\textbf{nature of the customer}.'
In the \textbf{Market Phase}, one must statistically analyze a \textbf{non-specific, large number} of customers and implement planned strategies (4Ps) and a system to execute and manage them (e.g., in-store displays, salesperson education). Conversely, in the \textbf{Relationship Phase}, the core is for \textbf{sales representatives} to build deep relationships with a \textbf{specific few} customers, supported by the entire organization (\textbf{inter-departmental collaboration}).
The practical lesson from this lecture is that marketers must first diagnose which phase their business is primarily in and concentrate resources on the activities appropriate for it. And, as indicated in the AI supplement, in modern BtoB marketing, how to strategically link these two phases has become a source of competitive advantage.
\end{document}
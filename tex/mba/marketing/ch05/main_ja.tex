\documentclass[uplatex,a4j,12pt,dvipdfmx]{jsarticle}
\usepackage{amsmath,amsthm,amssymb,bm,color,enumitem,mathrsfs,url,epic,eepic,ascmac,ulem,here,ascmac}
\usepackage[letterpaper,top=2cm,bottom=2cm,left=3cm,right=3cm,marginparwidth=1.75cm]{geometry}
\usepackage[english]{babel}
\usepackage[dvipdfm]{graphicx}
\usepackage[hypertex]{hyperref}

\title{マーケティング 第5回 講義ノート}
\author{Masaru Okada}
\date{\today}

\begin{document}
\maketitle
\tableofcontents

\section{講義資料整理:消費者行動の意思決定プロセスと購買類型}

\subsection{はじめに}

マーケティング戦略の策定において、顧客、すなわち消費者を理解することは不可欠である。消費者がどのようなニーズを持ち、製品やサービスをどのように認知し、最終的に購買に至るのか。そのメカニズムを解明することは、効果的な\textbf{マーケティング・ミックス(4P)}を展開するための基盤となる。
本講義ノートでは、消費者が製品やサービスを「取得し、消費し、処分する」際の一連の活動、すなわち\textbf{消費者行動(Consumer Behavior)}について、その心理的プロセスと行動パターンを体系的に整理し、MBAにおけるマーケティング学習の核心的な論点を明確にすることを目的とする。


\subsection{主要な概念と論点}


\subsubsection{市場の把握と消費者行動の定義}
マーケティング活動の前提として、標的市場を正確に把握する必要がある。経営学者の\textbf{フィリップ・コトラー(Philip Kotler)}は、市場を把握するための視点として\textbf{「7つのO」}を提示した。

\begin{itemize}
	\item \textbf{Occupants(主体)}: 誰が市場を構成しているか
	\item \textbf{Objects(客体)}: 何が購入されているか
	\item \textbf{Objectives(目的)}: なぜ購入するのか
	\item \textbf{Organization(組織)}: 誰が購買に関与しているか
	\item \textbf{Occasions(時期)}: いつ購入するのか
	\item \textbf{Outlet(販路)}: どこで購入するのか
	\item \textbf{Operations(活動)}: どのように購入するのか
\end{itemize}
この中でも特に「なぜ」「誰が」「どのように」といった問いに答えるのが消費者行動論である。

消費者行動の定義について、\textbf{Blackwell, Miniard and Engel (2005)}は「人々が製品やサービスを取得し、消費し、処分する際に従事する諸活動」と定義し、\textbf{Hoyer and MacInnis (2009)}は「意思決定単位によって、財、サービス、時間、アイデアの取得、消費、処分に関して行われる意思決定の総体」と定義している。これらは、購買という一点だけでなく、その前後のプロセス全体を分析対象とすることを示している。

\subsubsection{消費者の購買意思決定プロセス}
消費者の購買行動は、一般的に以下の5つの段階を経る情報処理モデルとして説明される。

\begin{enumerate}
	\item \textbf{問題認識}: 消費者が「理想とする状態」と「現実の状態」の間に乖離(ギャップ)を知覚し、それが一定の閾値を超えたときに、解決すべき問題(ニーズ)として認識される。
	\item \textbf{情報探索}: 問題解決のために情報を集める活動。
	\item \textbf{代案の評価}: 収集した情報に基づき、複数の選択肢(ブランドや製品)を比較評価する。
	\item \textbf{選択(購買)}: 最も評価の高い代案を選択し、実際の購買行動に移す。
	\item \textbf{結果の評価}: 購買後、製品を使用した結果が事前の「期待水準」を上回ったか(満足)、下回ったか(不満足)を評価する。
\end{enumerate}

\subsubsection{情報探索:内部探索と外部探索}
情報探索は、探索源によって二つに大別される。
\begin{itemize}
	\item \textbf{内部探索}: 消費者自身の記憶(\textbf{長期記憶})に蓄積された知識や経験を検索する活動。ブランドに関する情報は「連想のネットワーク」として記憶されている。
	\item \textbf{外部探索}: 内部探索で十分な情報が得られない場合、店舗、広告、インターネット、知人など外部の情報源から能動的に情報を収集する活動。
\end{itemize}
購買経験の有無や関与の程度により、情報探索の範囲は\textbf{「広範囲問題解決」(高関与・新規購入)}、\textbf{「限定的問題解決」(中関与・既知)}、\textbf{「定型的問題解決」(低関与・反復購入)}に分類される。

\subsubsection{代案の評価:情報統合の方略}
消費者は、複数の評価軸(属性)に基づき、製品の総合評価を行う。その情報統合の方法は、大きく二つに分類される。
\begin{itemize}
	\item \textbf{補償型モデル}: ある属性の評価が低くても、他の属性の評価が高ければ、それが補償されて総合評価が形成されるモデル。代表例として\textbf{フィッシュバイン・モデル}がある。これは、各属性の「重要度(評価)」と、その製品が持つ「信念(主観的評価)」の積和によって態度が形成されるとするモデルである。
	\item \textbf{非補償型モデル}: すべての属性を考慮せず、特定の基準に基づいて選択肢を絞り込むモデル。
	      \begin{itemize}
		      \item \textbf{連結型}: 各属性に最低限の必要条件(足切り点)を設定し、一つでも満たさない選択肢を除外する。
		      \item \textbf{分離型}: 一つでも十分条件(卓越した点)を満たせば、その選択肢を採用する。
		      \item \textbf{辞書編纂型}: 最も重視する属性で比較し、差がつかなければ次に重視する属性で比較する。
	      \end{itemize}
\end{itemize}
一般に、選択肢が少ない場合は補償型が、多い場合は非補償型が採用されやすい。

\subsubsection{購買後の評価と認知的不協和}
購買後の評価は、将来の購買行動に大きな影響を与える。特に、期待水準を成果が下回った場合(不満足)は、ネガティブな口コミや再購買の停止につながる。
また、特に高額な購買や、選択肢間に魅力的な差がなかった場合、消費者は「本当にこの選択で良かったのか」という心理的な不安、すなわち\textbf{認知的不協和}を感じることがある。

\subsubsection{製品関与と精緻化見込みモデル(ELM)}
消費者の情報処理の仕方は、その製品に対する\textbf{製品関与(こだわり、重要度やリスクの知覚)}の程度によって異なる。この関係を示したのが、\textbf{Petty and Cacioppo (1986)}が提唱した\textbf{精緻化見込みモデル(ELM)}である。
\begin{itemize}
	\item \textbf{中心的ルート(高関与)}: 関与が高い消費者は、情報のメッセージ内容(品質、機能など)を詳細に吟味・精緻化して態度を形成する。
	\item \textbf{周辺的ルート(低関与)}: 関与が低い消費者は、メッセージ内容よりも、広告タレントやパッケージデザイン、雰囲気といった周辺的な手がかりに影響されて態度を形成する。
\end{itemize}
中心的ルートで形成された態度は強固で行動に結びつきやすいが、周辺的ルートで形成された態度は一時的で変化しやすい。

\subsubsection{購買行動の類型(Assaelモデル)}
\textbf{Assael (1987)}は、「製品関与の程度(高・低)」と「ブランド間の知覚差異(大・小)」という2つの軸を用い、購買行動を4類型に分類した。

\begin{itemize}
	\item \textbf{情報処理型(高関与・差異大)}: 例:自動車、PC。購買意思決定プロセスを詳細に経て、信念→態度→行動という認知的学習プロセスをたどる。
	\item \textbf{不協和解消型(高関与・差異小)}: 例:高級家具、カーペット。ブランド間の差が分かりにくいため、まずは購買し(行動)、その後に自分の選択を正当化するために信念や態度を形成する(認知的不協和の解消)。
	\item \textbf{バラエティ・シーキング型(低関与・差異大)}: 例:缶コーヒー、菓子。不満がなくとも目新しさや多様性を求め、頻繁にブランドスイッチを行う。
	\item \textbf{慣性型(低関与・差異小)}: 例:トイレットペーパー、電池。情報探索をほとんど行わず、最小限の労力で「見かけ上のロイヤリティ」に基づき習慣的に購買する。
\end{itemize}


\subsection{応用と事例分析}


\subsubsection{外部探索の程度とマーケティング戦略}
講義では、外部探索の程度が異なる製品例として自動車(高)とチョコレート(低)が挙げられた。
\begin{itemize}
	\item \textbf{自動車(高外部探索)}: 消費者は価格、品質、燃費、デザインなど多様な情報を能動的に収集する。したがって、企業は販売店員による詳細な説明、パンフレットの充実、比較サイトへの情報提供など、消費者の情報収集を支援する活動が重要となる。
	\item \textbf{チョコレート(低外部探索)}: 消費者は店頭で短時間で決定することが多い。この場合、広告(特にテレビCM)を通じてブランド名を\textbf{内部記憶(長期記憶)}に刷り込み、店頭でのブランド想起(純粋想起・助成想起)を容易にすることが購買に直結する。
\end{itemize}

\subsubsection{非計画購買の実態とインストア・マーケティング}
青木幸弘氏ら(1989年)の研究によれば、スーパーマーケットなどにおける購買の約89\%が「非計画購買」であると示された。これは、事前に計画していた「計画購買」(11.0\%)に対し、店内でブランドを決定したり(ブランド選択)、買うつもりがなかったものを買ったり(想起購買、\textbf{衝動購買})する割合が圧倒的に高いことを示す。
この事実は、消費者が購買意思決定プロセスの多くを店頭(Outlet)で行っていることを意味し、POP広告、陳列方法、試食販売といったインストア・マーケティングの重要性を裏付けている。

\subsubsection{バラエティ・シーキングと製品戦略}
缶コーヒー市場(例:BOSS, TULLY'S, GEORGIA)は、\textbf{バラエティ・シーキング型}購買行動の典型例である。消費者は特定のブランドに不満がなくても、「今日は違う味を試してみよう」と目新しさを求めてブランドをスイッチする。
この市場において、企業は市場シェアを維持・拡大するために、絶えず新フレーバーの投入、パッケージのリニューアル、季節限定商品の開発を行い、消費者の「ブランド新鮮さ」への欲求に応え続ける戦略が求められる。


\subsection{深層背景と教訓}

本講義の主要な論点の周辺にある、実務的な示唆や背景を整理する。

\textbf{\paragraph{仮説検証を通じた消費者心理の解明}}
講義内で「寒くなればなるほど濃厚な味のアイスクリームが売れる」という仮説が提示された。これは、消費者行動論が単なる机上の理論ではなく、「なぜか?(脂肪を蓄えようとする本能か?)」といった心理的・生理的要因まで踏み込み、データに基づいた仮説検証を通じて市場の事実を確認する実証的な学問であることを示している。

\textbf{\paragraph{「見かけ上のロイヤリティ」の罠(慣性型購買)}}
トイレットペーパーや電池などの\textbf{慣性型購買}は、一見すると特定のブランドが継続的に選ばれているため、ロイヤリティが高いように見える(=見かけ上のロイヤリティ)。しかし、これは積極的な選好ではなく、単に「選ぶのが面倒」「どれも同じ」という消極的な理由(慣性)によるものである。このような市場では、競合他社が低価格戦略や目新しい機能(例:香りのするトイレットペーパー)を打ち出すと、消費者は容易にブランドスイッチしてしまうリスクを常に抱えている。

\textbf{\paragraph{10代女性と準拠集団}}
講義で「10代の女の子は準拠集団への意識が強い」として「数量限定商品」が有効である可能性が示唆された。これは、消費者の行動が、その人が所属したり、憧れたりする集団(=準拠集団)の規範や価値観に強く影響されることを意味する。特にファッションや音楽など自己表現性の高い製品において、この傾向は顕著である。

\textbf{\subsubsection{AIによる補足:重要論点の拡張}}
本講義の文脈において、さらに深掘りすべき重要な関連概念を補足する。

\begin{itemize}
	\item \textbf{準拠集団(Reference Group)}: 上記のトピックに関連するが、これは消費者行動論の非常に重要な概念である。消費者の態度形成や購買行動は、家族や友人(一次的集団)、あるいは自身が所属していないが憧れている著名人や専門家(憧憬集団)から強い影響を受ける。特に\textbf{情報処理型購買}や社会的リスク(他者からどう見られるか)の高い製品において、準拠集団内の\textbf{オピニオンリーダー}の口コミや推薦は、企業の広告活動以上に強力な影響力を持つことがある。

	\item \textbf{知覚リスク(Perceived Risk)}: 外部探索の動機として「製品価格の高さ」や「品質のばらつき」が挙げられたが、これらは消費者が購買に際して感じる「知覚リスク」の構成要素である。知覚リスクには、期待した性能が得られない「機能的リスク」、金銭的に損をする「金銭的リスク」、他者からの評価が下がる「社会的リスク」、自己イメージとそぐわない「心理的リスク」などがある。消費者はこの知覚リスクを低減するために情報探索を活発化させ、あるいは\textbf{不協和解消型}の行動(購買後の正当化)をとる。
\end{itemize}


\subsection{結論}

本講義ノートでは、消費者行動の基本的なフレームワークについて学んだ。消費者行動論とは、コトラーの「7つのO」に代表される「誰が、何を、なぜ、どのように買うのか」という問いに対し、心理学的な側面から体系的に解明する学問である。

消費者の購買プロセスは、「問題認識」から「購買後評価」までの一連の流れとして捉えられ、特に「製品関与」の程度と「ブランド間の知覚差異」が、\textbf{ELM}(情報処理の経路)や\textbf{Assaelモデル}(購買行動の類型)を決定づける重要な要因となる。

本講義から得られる実践的な教訓は、マーケターが自社製品・サービスを、消費者がどの行動類型(例:\textbf{情報処理型}か\textbf{慣性型}か)で購買しているかを正確に把握することの重要性である。さらに、青木氏の研究が示すように、消費者の購買決定の多くが店頭で行われる(\textbf{非計画購買})という事実は、チャネル戦略やインストア・マーケティングが、広告戦略と同様に(あるいはそれ以上に)売上を左右する決定的な要因であることを示唆している。


\subsection{重要キーワード一覧}

フィリップ・コトラー、ブラックウェル、ミニアード、エンゲル、ホーヤー、マッキニス、阿部周造、中西正雄、フィッシュバイン、青木幸弘、ペティ、カシオッポ、アサエル
\vspace{\baselineskip}

マーケティング・ミックス、消費者行動、購買意思決定プロセス、問題認識、情報探索(内部・外部)、短期記憶、長期記憶、連想ネットワーク、広範囲問題解決、限定的問題解決、定型的問題解決、補償型モデル、非補償型モデル、フィッシュバイン・モデル、非計画購買、衝動購買、認知的不協和、製品関与、精緻化見込みモデル(ELM)、中心的ルート、周辺的ルート、不協和解消型購買、バラエティ・シーキング、慣性型購買、準拠集団、知覚リスク

\subsection{理解度確認クイズ}
\begin{enumerate}
	\item コトラーが提唱した「7つのO」のうち、購買の「目的」を問うのはどれか。
	\item 消費者の記憶(内部情報)に基づいて情報を検索する行動を何と呼ぶか。
	\item 購買経験がなく、多くの情報を必要とする購買意思決定のパターンを何と呼ぶか。
	\item 製品の属性(例:価格)のマイナス評価を、他の属性(例:デザイン)のプラス評価で補う評価方法を何と呼ぶか。
	\item 最も重視する属性で製品を比較し、優劣がつかない場合に次の属性で比較する評価方法を何と呼ぶか。
	\item フィッシュバイン・モデルが分類されるのは、補償型か非補償型か。
	\item 購買後に消費者が感じる「選択が正しかったか」という不安や後悔の感情を何と呼ぶか。
	\item 消費者が製品に対して感じる重要性やリスクの程度を示す概念は何か。
	\item 精緻化見込みモデル(ELM)において、関与が低い消費者が取る情報処理経路を何と呼ぶか。
	\item ELMにおいて、関与が高い消費者が取る情報処理経路を何と呼ぶか。
	\item Assaelの購買行動類型で、高関与かつブランド間差異が小さい場合の類型は何か。
	\item Assaelの購買行動類型で、低関与かつブランド間差異が大きい場合の類型は何か。
	\item Assaelの購買行動類型で、高関与かつブランド間差異が大きい場合の類型は何か。
	\item Assaelの購買行動類型で、低関与かつブランド間差異が小さい場合の類型は何か。
	\item 既存のブランドに不満がなくとも、目新しさを求めて頻繁にブランドスイッチを行う行動を何と呼ぶか。
\end{enumerate}

\subsubsection*{解答一覧}
1. Objectives、2. 内部探索、3. 広範囲問題解決、4. 補償型モデル(方略)、5. 辞書編纂型、6. 補償型、7. 認知的不協和、8. 製品関与(または購買関与)、9. 周辺的ルート、10. 中心的ルート、11. 不協和解消型、12. バラエティ・シーキング型、13. 情報処理型、14. 慣性型、15. バラエティ・シーキング

\section{マーケティングと消費者行動}

\subsection{はじめに}
本講義(第5回)は、\textbf{消費者行動論}の基本的な枠組みについて解説する。企業がマーケティング活動を行う上で、消費者を理解する必要性と、その分析がもたらす意味について概説する。消費者行動論は、一人の代表的な消費者が、いかに情報を収集・理解・利用し、その結果として購買行動に至るかのプロセスを説明する学問である。本ノートでは、講義で示された主要な概念、分析の意義、およびその応用について整理する。

\subsection{主要な概念と論点}

\subsubsection{マーケティングにおける消費者行動の位置づけ}
企業のマーケティング活動の前提として、市場を把握する必要がある。フィリップ・\textbf{コトラー}は、市場を把握するための視点として「\textbf{7O}」の枠組みを提示している。
\begin{itemize}
	\item \textbf{Occupants (市場主体)}: 誰が市場を構成しているか。
	\item \textbf{Objects (購買対象)}: 何が購入されているか。
	\item \textbf{Objectives (購買目的)}: なぜ購入されているか。
	\item \textbf{Organizations (購買組織)}: 誰が購買に関与しているか。
	\item \textbf{Occasions (購買時期)}: いつ購入されているか。
	\item \textbf{Outlets (購買場所)}: どこで購入されているか。
	\item \textbf{Operations (購買方法)}: どのように購入されているか。
\end{itemize}
本講義で扱う消費者行動論は、特にこの中の「\textbf{Occupants (市場主体)}」を深く理解するための学問領域であり、消費者が「どのように、どこで、いつ、誰の影響を受けて」購買に至るかのプロセスを解明する。

\subsubsection{消費者行動の定義}
\paragraph{消費とは}
講義では、消費を「使い尽くすこと」と定義し、人々が欲望を充足するために\textbf{財・サービス}をなくしていく行為であると説明された。このような行為者が集まる社会を「消費社会」と呼ぶ。

\paragraph{学術的定義}
\begin{itemize}
	\item \textbf{アメリカ・マーケティング協会 (AMA)}: 消費者行動を、製品やサービス市場における消費者(意思決定者)の行為と定義し、それを理解・記述するための\textbf{学際的}な研究領域とした。
	\item \textbf{Blackwellら (2005)}: 人々が製品やサービスを取得・消費・処分する際に従事する全ての活動。
	\item \textbf{Hoyer and MacInnis (2009)}: ある意思決定単位(消費者)によって、財、サービス、時間、アイデアを取得・消費・処分する際に行われる意思決定の総体。
\end{itemize}

\subsubsection{消費者行動分析の対象}
企業が把握しようとする消費者行動は、二つの側面に大別される。
\begin{enumerate}
	\item \textbf{外面的な行動 (結果)}: 「購買する」「購買しない」「情報収集のみ行う」といった、表面に現れる行動。広告などのマーケティング努力に対し、売上などの数値として観測可能である。
	\item \textbf{内面的なプロセス (段階)}: 刺激(例:広告)に対して、製品やブランドにいかなる感情を抱き(\textbf{知覚})、それが次の行動にどう結びつくか、という目に見えない抽象的な心理プロセス。
\end{enumerate}

\subsection{応用と事例分析}

\subsubsection{企業が消費者行動を分析する意義}
企業が消費者行動を分析する最大の目的は、「把握した消費者の特性の傾向に合わせて、効果的なマーケティング戦略を立てるため」である。
\begin{itemize}
	\item \textbf{効果的なマーケティング戦略の立案}: ターゲット層の好み、購買場所、購買\textbf{シーン}に合わせて、\textbf{小売店舗}の設定やプロモーション(\textbf{4P})を設計し、収益向上を図る。
	\item \textbf{市場状況の詳細な説明}: 市場は、\textbf{製品差別化}や\textbf{市場細分化}(セグメンテーション)、\textbf{製品ライフサイクル}といった理論的枠組みだけでなく、その市場に属する消費者の好みや実際の購買行動によっても形成されている。消費者行動分析は、この実態を説明する手段となる。
	\item \textbf{現実的な市場の定量的把握}:
	      経済学的な価格・\textbf{数量}モデルは、価格と数量以外の全条件を捨象(隔離)して理論的なメカニズムを説明しようとするため、現実の市場行動の説明には限界がある。
	      対照的に、消費者行動モデルは、多様な\textbf{変数}(心理的要因など)を考慮に入れ、それらが現実の市場状況(購買行動)にどう影響しているかを「\textbf{測定可能な行動データ}」として分析できる。
\end{itemize}

\subsubsection{分析事例:仮説検証プロセス}
消費者行動分析は、アンケート調査などを用いて「なぜその現象が起こるのか」という仮説を検証するプロセスで活用される。

\paragraph{事例1:濃厚なアイスクリームの購買理由}
\begin{itemize}
	\item \textbf{現象}: 濃厚な味のアイスクリームがよく売れている。
	\item \textbf{仮説}: 季節が寒くなったからではないか。
	\item \textbf{深掘り}: 「なぜ消費者は寒くなると、あっさりした味より濃厚な味を求めるのか?」という心理的要因(例:満足感、温かみの連想など)を探る。
	\item \textbf{分析}: その心理的要因が、実際の購買行動に影響を与えているかを数値的に検証する。
\end{itemize}

\paragraph{事例2:新製品購買者の特性}
\begin{itemize}
	\item \textbf{調査対象}: 新製品を購買する消費者の特性は何か。
	\item \textbf{分析結果}: 「インターネットの利用時間が長い人」と「新製品の購買行動」の間に、正の\textbf{相関}が見られた。
	\item \textbf{戦略への応用}: この分析結果に基づき、企業は「新製品のプロモーションにはインターネットが有効なチャネルである」と予測できる。
	\item \textbf{理由の推定}: なぜなら、「インターネットを通じて新しい情報に頻繁に露出されるから」という理由(仮説)が考えられ、情報露出の重要性という示唆が得られる。
\end{itemize}
このように、Bakerも指摘するように、4P(商品、価格、プロモーション、流通)と消費者の反応との関係を予測し、購買頻度を高める計画を立てることが可能になる。

\subsubsection{製品コンセプトへの応用(事例)}
\begin{itemize}
	\item \textbf{ターゲット設定}: 10代の女性。
	\item \textbf{把握された特性}: ターゲット層は、他世代と比較して\textbf{準拠集団}(友人など)への意識が非常に強い。
	\item \textbf{戦略的検討}: この特性に基づき、「あえて商品の数量を限定し、\textbf{希少性}を強調する」マーケティング計画が有効かどうかを社内で検討する、といった応用が考えられる。
\end{itemize}

\subsection{深層背景と教訓}

\textbf{\paragraph{本論から逸れた寄り道トピック名:消費者行動論における「消費者」の前提条件}}
本講義で扱う消費者行動論では、分析の基本単位として「消費者は、一人で、自分のために購買の意思決定を行う」と仮定する。現実には、家族で使用する冷蔵庫のように共同で意思決定を行うケースも多いが、学問上の前提としては、個人内の心理プロセスと意思決定に焦点を当てる。
これが、組織として合理的な意思決定を行う「\textbf{生産財購買}(組織購買)」との根本的な違いとなる。

\textbf{\paragraph{本論から逸れた寄り道トピック名:消費者の「異質性」と「一般性」}}
消費者行動を分析する際、二つの側面が考慮される。一つは、人間が持つ個性や好みという「\textbf{異質性}」である。もう一つは、ある程度の買い物行動の枠という「\textbf{一般性}」である。
消費者行動論では、この複雑さを捨象し、「一人の代表的な消費者」を想定して分析を進めることが多い。マーケティング戦略上、消費者間の「\textbf{異質性}」は、「市場セグメント(\textbf{市場細分化})間の違い」として扱われる。

\textbf{\subsubsection{AIによる補足:重要論点の拡張}}
\textbf{消費者意思決定プロセスモデルの補完}

講義では、消費者行動が「段階を構成する」と言及されたが、その具体的な中身(内面プロセス)についての詳細なモデルが提示されなかった。MBAの消費者行動論において最も基礎的かつ重要な枠組みは、「\textbf{消費者意思決定プロセスモデル}」である。

このモデルは、消費者が購買に至るまでの内面プロセスを以下の5段階で捉える。
\begin{enumerate}
	\item \textbf{問題認識 (Problem Recognition)}: 消費者が自身の「理想の状態」と「現状」との間にギャップを認識する段階。(例:「のどが渇いた」)
	\item \textbf{情報探索 (Information Search)}: ギャップを埋めるための手段に関する情報を探索する段階。内部探索(記憶)と外部探索(インターネット、友人、店頭)がある。
	\item \textbf{代替品評価 (Alternative Evaluation)}: 収集した情報に基づき、複数の選択肢(ブランドや製品)を特定の評価基準(例:価格、品質、デザイン)で比較・検討する段階。
	\item \textbf{購買決定 (Purchase Decision)}: 最も評価の高い代替案を選択し、実際に購入する(あるいは購入しない)ことを決定する段階。
	\item \textbf{購買後行動 (Post-purchase Behavior)}: 製品を使用した結果、満足または不満足を感じる段階。この評価が次の購買行動(リピートや口コミ)に影響を与える。
\end{enumerate}
講義で触れられた「なぜ濃厚なアイスが売れるか」といった分析は、このプロセスの「代替品評価」や「問題認識」の段階で、消費者がどのような心理的要因を重視しているかを解明しようとする試みである。

\subsection{結論}
本講義ノートでは、マーケティング戦略立案の基盤としての消費者行動論の役割を整理した。コトラーの7Oにおける「Occupants(市場主体)」を解明することが、本学問の中核的な目的である。

講義で示された事例(アイスクリーム、新製品購買)は、企業が単に「何が売れたか」という結果(外面的な行動)を追うだけでなく、消費者の「なぜそれが選ばれたか」という\textbf{内面的なプロセス}をモデル化し、仮説検証を通じて理解することの重要性を示している。

\textbf{実践的な教訓}として、消費者行動論は、複雑で捉えどころのない市場を「測定可能な行動データ」として分析する手段を提供する。これにより、経済学的な理論モデルと現実の市場とのギャップを埋め、より精度の高いマーケティング戦略を策定するための実践的な示唆を得ることができる。

\subsection{重要キーワード一覧}
コトラー、Blackwell、Hoyer、MacInnis、Baker
\vspace{\baselineskip}
マーケティング・ミックス(4P)、市場細分化(セグメンテーション)、製品差別化、製品ライフサイクル、生産財購買、準拠集団、希少性、消費者意思決定プロセス

\subsection{理解度確認クイズ}
\begin{enumerate}
	\item コトラーが提唱した市場把握の枠組み「7O」のうち、「誰が市場を構成しているか」を問う要素は何か。
	\item AMA(アメリカ・マーケティング協会)の定義において、消費者行動論はどのような研究領域として位置づけられているか。
	\item 消費者行動分析の対象のうち、「購買する」「購買しない」といった、客観的に観測可能な行動を何と呼ぶか。
	\item 消費者行動分析の対象のうち、「知覚」や「感情」といった、目に見えない心理的なプロセスを何と呼ぶか。
	\item 経済学的な価格・数量モデルと比較した際、消費者行動モデルが持つ分析上の利点は何か。
	\item ある現象(例:新製品購買)と特定の要素(例:インターネット利用時間)の間に統計的な関連性が見出された場合、この関連性を何と呼ぶか。
	\item マーケティング戦略において、製品(Product)、価格(Price)、流通(Place)、プロモーション(Promotion)の4つの要素の組み合わせを何と呼ぶか。
	\item 消費者が自身の所属集団や、行動の規範としたいと考える集団を何と呼ぶか。
	\item 講義で扱った消費者行動論の分析前提において、家族の意思決定と対比される「生産財購買」の最も大きな特徴は何か。
	\item 市場を構成する消費者のニーズや特性が均一ではなく、多様であることを示す概念は何か。
	\item 消費者が自身の「理想の状態」と「現状」のギャップを認識する、消費者意思決定プロセスの最初の段階は何か。
	\item 消費者が購買決定を行う前に、記憶を辿ったり(内部探索)、インターネットや友人に尋ねたり(外部探索)する段階は何か。
	\item 消費者が複数の選択肢を、価格や品質といった特定の基準で比較検討する段階は何か。
	\item 製品を使用した結果、消費者が感じる満足や不満足が、次の購買行動に影響を与える段階は何か。
	\item あえて供給量を絞ることで、消費者の購買意欲を高めようとする戦略は、何の原則を利用したものか。
\end{enumerate}

\subsubsection*{解答一覧}
1. Occupants (市場主体)、2. 学際的な研究領域、3. 外面的な行動(または結果としての行動)、4. 内面的なプロセス、5. (例:多様な変数を考慮し、測定可能なデータで現実の市場を説明できる点)、6. 相関(または相関関係)、7. マーケティング・ミックス(または4P)、8. 準拠集団、9. (例:組織として合理的な意思決定を行う点)、10. 市場の異質性、11. 問題認識、12. 情報探索、13. 代替品評価、14. 購買後行動、15. 希少性(の原則)

\section{消費者の情報探索}


\subsection{はじめに}
本講義の目的は、消費者がどのようにして「購買」という行動に至るのか、その心理的プロセスを理解することにある。消費者行動論では、一人一人の消費者を主人公とし、各自の頭の中で行われる情報処理モデルとして購買行動を分析する。本ノートでは、その中核となる\textbf{購買意思決定プロセス}の各段階、特に消費者の記憶がどのように関与するか(\textbf{情報探索})に焦点を当てて概説する。

\subsection{主要な概念と論点}

\subsubsection{購買意思決定プロセス}
消費者の購買行動は、一般的に以下の5段階を経時的にたどる情報処理プロセスとしてモデル化される。
\begin{enumerate}
	\item \textbf{問題認識}:ニーズやウォンツが発生する段階。
	\item \textbf{情報探索}:問題を解決するための情報を収集する段階。
	\item \textbf{代案評価}:収集した情報をもとに、複数の選択肢を比較評価する段階。
	\item \textbf{購買決定}:最も望ましいと評価した選択肢を選び、実際に購入する段階。
	\item \textbf{購買後評価}:自らの購買選択が正しかったかを評価し、満足または不満(認知的不協和)を感じる段階。
\end{enumerate}
この各段階において、個人の心理的要因(動機、知覚、学習など)や外部の環境要因(文化、社会階層、家族など)が影響を与える。

\subsubsection{問題認識}
プロセスは、消費者が自身の「\textbf{理想とする状況}」と「\textbf{現実の状況}」との間に、無視できないほどの乖離(ギャップ)を認識することから開始される。例えば、友人が持っている新しいカバンを見て「素敵だ」と感じた瞬間、「あのカバンを持っていない自分(現実)」と「持っている自分(理想)」との間に乖離が生じ、それがカバンに対するニーズ、すなわち「問題」として認識される。

\subsubsection{情報探索}
問題認識後、消費者はその解決策を求めて情報探索を開始する。
\begin{description}
	\item[\textbf{内部探索}] まず、消費者は自身の頭の中にある記憶、すなわち\textbf{内部情報}(過去の経験、知識)を探る。この段階で十分な情報が得られ、問題を解決できると判断すれば、プロセスは迅速に購買決定に進む。
	\item[\textbf{外部探索}] 内部情報が不十分な場合、消費者は\textbf{外部探索}に移行する。インターネット、知人・友人からの口コミ、店舗での比較、パンフレットなど、外部の情報源から能動的に新しい情報を収集する。
\end{description}

\subsubsection{記憶の構造とプロセス}
内部探索と密接に関連するのが、人間の記憶のメカニズムである。記憶は一般に3つの貯蔵庫でモデル化される。
\begin{itemize}
	\item \textbf{感覚記憶(感覚レジスター)}:視覚や聴覚など、感覚器官から入ってきた刺激を瞬時(1秒未満)に保持する。
	\item \textbf{短期記憶}:感覚記憶のうち、注意を向けられた情報が転送され、数十秒程度保持される。容量は限定的である(マジカルナンバー7±2)。
	\item \textbf{長期記憶}:短期記憶内の情報がリハーサル(反復)されたり、既存の知識と関連付けて意味づけ(符号化)されたりすることで、永続的に貯蔵される。容量はほぼ無限と考えられる。
\end{itemize}

情報は、外部からの刺激が「感覚レジスター」を経由し、「短期記憶」で一時的に処理され、それが繰り返されることで「長期記憶」へと流れていく。

\subsubsection{連想ネットワークモデルとブランドイメージ}
長期記憶は、単なる情報の羅列ではなく、関連する概念(ノード)同士がリンクで結ばれた巨大なネットワーク構造(\textbf{連想ネットワークモデル})をしていると考えられる。
例えば、「コカ・コーラ」というノードは、「アメリカ」「赤色」「炭酸」「特殊な形の瓶」といった他のノードと強く結びついている。

\textbf{ブランドイメージ}とは、まさにこの連想ネットワークの表れであり、あるブランドと関連する様々な物事、属性、感情、経験がリンクされた結果として、消費者の頭の中に形成される。企業がブランド構築に注力するのは、この連想ネットワークを通じて、自社ブランドを消費者の長期記憶に好意的に、かつ強固に残すためである。

\subsection{応用と事例分析}

\subsubsection{ブランド戦略と長期記憶:エピソード記憶の活用}
企業は、自社ブランドを単なる短期記憶(広告を見た直後だけ思い出される)ではなく、永続的な長期記憶として定着させることを目指す。特に有効なのが、個人の具体的な体験や出来事に関する記憶、すなわち\textbf{エピソード記憶}とブランドを結びつけることである。
\begin{itemize}
	\item \textbf{事例:チキンラーメン} \\
	      ロングセラーブランドであるチキンラーメンの広告では、「子供の頃、家族と食べた」「受験勉強の夜食だった」といった、個人のエピソードや懐かしさに訴えかける表現が多用される。これは、ブランドを消費者の個人的な「良き思い出」という強力なエピソード記憶と関連付け、長期記憶におけるブランドへの愛着(ロイヤルティ)を醸成する戦略である。
\end{itemize}

\subsubsection{情報探索の程度とマーケティング戦略}
消費者が購買時にどれだけ積極的に情報探索を行うかは、製品カテゴリーによって大きく異なる。
\begin{itemize}
	\item \textbf{事例1:低関与製品(例:明治ミルクチョコレート)} \\
	      チョコレートのような安価な最寄品(低関与製品)では、消費者は失敗のリスクが低いため、積極的な\textbf{外部探索}は行わない。購買は「内部探索」のみ(いつも買っているもの)、あるいは店頭での短期的な刺激(パッケージデザインやPOP)によって決定されることが多い。
	      \textbf{マーケティング示唆}:この場合、企業は反復的な広告出稿により、消費者の\textbf{内部記憶}(短期記憶)を頻繁に刺激し、購買時点で自社ブランドを想起されやすくすること(ブランド再生)や、情緒的な側面(例:美味しそう、楽しそう)に訴えて\textbf{ブランド選好}を形成させることが重要となる。

	\item \textbf{事例2:高関与製品(例:PC、自動車)} \\
	      PCや自動車のような高価格な専門品・買回品(高関与製品)では、失敗のリスクが高いため、消費者は\textbf{外部探索}を積極的に行う。
	      \textbf{マーケティング示唆}:この場合、企業は消費者の外部探索の各接点(Webサイト、パンフレット、販売店)で、充実した比較情報や専門的なアドバイスを提供することが不可欠となる。店頭での販売員の知識や、Webサイトの情報の分かりやすさといった\textbf{チャネル戦略}が、購買決定に重大な影響を与える。
\end{itemize}

\subsection{深層背景と教訓}

\textbf{\paragraph{情報探索の「量」を規定する要因}}
消費者が外部探索をどの程度行うかは、以下の要因に左右される。
\begin{itemize}
	\item \textbf{情報の必要性(探索の便益)}:(1)製品価格が高いほど、(2)製品へのこだわり(重要度)が強いほど、(3)製品の品質にバラつきが大きいと知覚されるほど、失敗を避けるために情報を多く収集しようとする(必要性が高い)。
	\item \textbf{消費者の記憶量(知識)}:製品に関する知識が豊富な消費者(専門家)は、すでに十分な内部情報を保持しているため、外部探索の必要性が低い。逆に、知識が全くない初心者も、何を調べてよいか分からず探索を諦めることがある。
	\item \textbf{情報収集費用(探索のコスト)}:情報収集にかかる時間、労力、金銭的コスト。かつてはタクシーの料金やサービスを乗車前に比較するのが困難だったように、収集費用が高すぎると消費者は探索を断念する。しかし、\textbf{インターネット}の普及は、この情報収集費用を劇的に低下させ、高関与製品における消費者の外部探索行動を大きく変容させた。
\end{itemize}

\textbf{\paragraph{購買行動の類型(3つのパターン)}}
情報探索の程度と消費者の関与度・経験に基づき、購買行動は以下の3つに類型化できる。
\begin{itemize}
	\item \textbf{広範の問題解決}:購買経験がなく、製品への関与が非常に高い場合(例:初めての住宅購入)。ブランドや選択基準も不明確なため、大規模な情報収集と慎重な評価(長い時間)を費やす。
	\item \textbf{限定的な問題解決}:ある程度の購買経験はあるが、明確なブランド選好は確立していない場合。既知の選択基準に基づき、いくつかの選択肢(新製品を含む)を比較検討する。
	\item \textbf{定型的な問題解決}(習慣的購買行動):製品への関与が低く、繰り返し購買している場合(例:前述のチョコレート)。特定のブランドへの選好が確立しており、情報探索はほとんど行われず、習慣的に同じものを購入する。
\end{itemize}

\textbf{\subsubsection{AIによる補足:重要論点の拡張(代案評価モデル)}}
本講義では「情報探索」に多くの時間が割かれたが、その次の段階である「\textbf{代案評価}」の具体的なプロセスは省略されていた。MBA学習において、この評価プロセスはマーケティング戦略(特にポジショニング)に直結するため、極めて重要である。

消費者は収集した情報を、自らが設定した\textbf{評価基準}(例:PCの場合、価格、処理速度、バッテリー持続時間、デザイン)に基づいて評価する。この評価方法には、大きく分けて2つのモデルが存在する。

\begin{itemize}
	\item \textbf{補完的モデル(多属性態度モデルなど)}:
	      ある評価基準における弱点(例:価格が高い)が、他の評価基準における強点(例:デザインが非常に優れている)によって\textbf{補われる}(相殺される)と考えるモデル。消費者は各属性の重要度を考慮して総合点を算出し、最も点数の高い選択肢を選ぶ。
	\item \textbf{非補完的モデル}:
	      ある基準での弱点は、他で補われないとするモデル。
	      \begin{itemize}
		      \item \textbf{辞書編纂型}:最も重要な評価基準(例:価格)で各選択肢を比較し、それが最も優れているものを選択する。
		      \item \textbf{連結型}:すべての評価基準において、最低限の「足切りライン」(例:価格は5万円以下、かつバッテリーは8時間以上)を設定し、それをすべてクリアしたものを選択する。
	      \end{itemize}
	      高関与製品では補完的モデルが、低関与製品では非補完的モデル(特に単純な足切り)が用いられやすい傾向がある。
\end{itemize}

\subsection{結論}
本講義ノートでは、消費者の購買行動を\textbf{購買意思決定プロセス}という情報処理の枠組みで捉え、特に「問題認識」から「情報探索」の段階、そしてその基盤となる「記憶」のメカニズム(内部探索、長期記憶、連想ネットワーク)について整理した。

本講義から得られる\textbf{実践的な教訓}は、マーケターは自社製品が消費者の購買プロセスにおいて、どの類型(広範・限定・定型)に属するかを正確に把握しなければならない、という点にある。
高関与製品であれば、消費者が\textbf{外部探索}を行うあらゆる接点(Web、店舗)で、比較検討に耐えうる質の高い情報を提供し、評価基準で優位に立つ必要がある。
一方で低関与製品であれば、反復広告によって消費者の\textbf{内部記憶}に刷り込み、あるいは\textbf{エピソード記憶}に訴えかけることで、情報探索自体を省略させ、「選ばれるブランド」としての地位を確立することが求められる。ブランド構築とは、消費者の長期記憶における連想ネットワークをデザインする活動に他ならない。

\subsection{重要キーワード一覧}
人名:
(該当なし)

\vspace{\baselineskip}
普遍的な概念:
購買意思決定プロセス、問題認識、情報探索、内部探索、外部探索、代案評価、購買後評価、情報処理モデル、感覚記憶、短期記憶、長期記憶、連想ネットワークモデル、ブランドイメージ、エピソード記憶、情報収集費用、広範の問題解決、限定的な問題解決、定型的な問題解決

\subsection{理解度確認クイズ}
\begin{enumerate}
	\item 理想状態と現実状態の乖離を認識することから始まる、購買意思決定プロセスの第一段階は何か?
	\item 購買意思決定プロセスの第二段階であり、自己の記憶を探る「内部探索」と、外部の情報を探る「外部探索」に大別される活動は何か?
	\item 購買意思決定プロセスの最終段階で、自己の購買選択が正しかったかを評価し、満足や不満(後悔)を感じる段階は何か?
	\item 感覚器官から入った情報が、注意を引くことで数秒から数十秒保持される記憶の貯蔵庫は何か?
	\item 短期記憶内の情報が反復(リハーサル)されたり、意味づけられたりすることで、永続的に貯蔵される記憶の貯蔵庫は何か?
	\item 記憶内の概念(ノード)が、互いの関連性(リンク)によって結びついているとする記憶モデルを何と呼ぶか?
	\item 特定のブランド名(例:コカ・コーラ)を聞いたときに、関連するイメージ(例:アメリカ、赤色、瓶)が次々と思い浮かぶ現象を説明するモデルは何か?
	\item チキンラーメンのCMが子供時代の「懐かしさ」に訴えかけるように、個人の体験や出来事に関する記憶を特に何と呼ぶか?
	\item 製品の価格が高いほど、また品質のバラつきが大きいほど、消費者は情報探索を活発に行う傾向がある。これは探索行動における何の要因(便益)によるものか?
	\item インターネットの普及により、消費者が情報収集にかける時間や労力が劇的に低下した。このようなコストを一般に何と呼ぶか?
	\item 明治ミルクチョコレートのように、消費者が情報探索をほとんど行わず、習慣的に購買する製品の購買決定プロセスを何と呼ぶか?
	\item 自動車や住宅のように、消費者が購買経験がなく、広範な情報収集と慎重な評価を行って購買する決定プロセスを何と呼ぶか?
	\item 消費者が収集した情報を評価する際に用いる基準(例:価格、デザイン、性能)を何と呼ぶか?
	\item ある属性の弱点(例:価格が高い)が、他の属性の強点(例:性能が良い)によって補われる評価方法を何モデルと呼ぶか?
	\item 「価格が1万円以下」といった最低基準を満たさない選択肢を即座に除外する評価方法を何モデルと呼ぶか?
\end{enumerate}

\subsubsection*{解答一覧}
1. 問題認識、 2. 情報探索、 3. 購買後評価、 4. 短期記憶、 5. 長期記憶、 6. 連想ネットワークモデル、 7. 連想ネットワークモデル、 8. エピソード記憶、 9. 情報の必要性(探索の便益)、 10. 情報収集費用(探索のコスト)、 11. 定型的な問題解決(習慣的購買行動)、 12. 広範の問題解決、 13. 評価基準、 14. 補完的モデル、 15. 非補完的モデル


\section{代案の評価と選択}

\subsection{はじめに}
本講義ノートは、購買意思決定プロセスの後半、すなわち消費者が問題を認識し情報を探索した後に行う「対案の評価と選択」および「購買後の評価」に焦点を当てる。消費者は製品を「\textbf{属性}の束」として捉えており、その多数の属性情報をいかに統合し、一つの製品を選択するに至るかを解説する理論モデルを理解することを目的とする。

\subsection{主要な概念と論点}
消費者が製品の属性情報を統合し、最終的な評価を下すプロセスは、大きく二つのモデルに大別される。それは、情報処理の負荷が異なる「\textbf{補償型モデル}」と「\textbf{非補償型モデル}」である。

\subsubsection{補償型モデル (Compensatory Model)}
補償型モデルは、消費者がコンピュータのように合理的であり、対象となる製品の全属性を評価し、その総合評価値に基づいて最適な選択を行うと仮定するモデルである。
\begin{itemize}
	\item \textbf{特徴:} ある属性における評価の低さ(弱点)が、他の属性における評価の高さ(強み)によって\textbf{補償}される。
	\item \textbf{代表的モデル: フィッシュバイン・モデル (Fishbein Model)}
	      \begin{itemize}
		      \item 各属性の「\textbf{重要度}」と、その製品が各属性を有している程度に関する「\textbf{信念}(主観的評価)」を掛け合わせ、それらを全属性について総和したものが総合評価値となると考える。
		      \item 概念式: 総合評価 = $\sum (\text{属性}i\text{の重要度} \times \text{属性}i\text{への信念評価})$
		      \item 消費者は、この総合評価値が最も高いブランドを選択するとされる。
	      \end{itemize}
\end{itemize}

\subsubsection{非補償型モデル (Non-compensatory Model)}
非補償型モデルは、消費者は限定された情報処理能力しか持たない(\textbf{限定合理性})という前提に立ち、評価を簡略化するための\textbf{ヒューリスティクス}(簡便法)を用いると仮定するモデルである。
\begin{itemize}
	\item \textbf{特徴:} 全ての属性を考慮するわけではなく、ある属性が基準を満たさない場合、他の属性がどれだけ優れていてもその弱点は補償されず、選択肢から除外される。
	\item \textbf{多様なモデル:}
	      \begin{itemize}
		      \item \textbf{感情依存型(感情参照型):} 過去の購買経験や使用経験から生じる感情に基づき、最も好むブランドを習慣的に選択する。
		      \item \textbf{連結型 (Conjunctive Model):} 各属性に「最低受容基準(足切りライン)」を設定し、一つでもその基準を下回る属性がある製品は、選択肢から除外する。
		      \item \textbf{分離型 (Disjunctive Model):} 各属性に「十分条件」を設定し、どれか一つの属性でもその基準を(非常に高く)満たす製品があれば、それを選択する。
		      \item \textbf{辞書編纂型 (Lexicographic Model):} 最も重要視する属性で全選択肢を比較し、そこで優劣が決まれば(例:最も価格が安いもの)、その時点で選択を決定する。もし同点の場合は、二番目に重要な属性で比較する、というプロセスを繰り返す。
	      \end{itemize}
\end{itemize}

\subsubsection{モデルの選択}
消費者は、状況に応じてこれらのモデルを使い分ける。ブランド選択肢が少ない場合は、詳細な比較が可能な\textbf{補償型}モデルが採用されやすい。一方、選択肢が多い場合は、情報処理の負荷を軽減するために\textbf{非補償型}モデルが採用されやすい傾向がある。

\subsection{応用と事例分析}
講義では、フィッシュバイン・モデル(補償型)の具体例として、2つのブランド(A、B)の自動車購入が挙げられた。

\begin{itemize}
	\item \textbf{評価軸(属性):} 価格、燃費
	\item \textbf{消費者の設定:}
	      \begin{itemize}
		      \item 属性の重要度: 価格(3), 燃費(1)
	      \end{itemize}
	\item \textbf{各ブランドへの信念(評価点):}
	      \begin{itemize}
		      \item ブランドA: 価格(3), 燃費(-1)
		      \item ブランドB: 価格(1), 燃費(1)
	      \end{itemize}
	\item \textbf{総合評価の算出:}
	      \begin{itemize}
		      \item \textbf{ブランドA:} (価格重要度3 $\times$ 価格評価3) + (燃費重要度1 $\times$ 燃費評価-1) = 9 - 1 = \textbf{8点}
		      \item \textbf{ブランドB:} (価格重要度3 $\times$ 価格評価1) + (燃費重要度1 $\times$ 燃費評価1) = 3 + 1 = \textbf{4点}
	      \end{itemize}
	\item \textbf{結論:} この消費者(の評価軸)においては、総合評価の高いブランドAが選択されることになる。
\end{itemize}

\subsection{深層背景と教訓}

\textbf{\paragraph{計画購買と非計画購買の実態}}
消費者の購買行動は、必ずしも上記のような合理的な情報統合プロセスを経るわけではない。ある調査によれば、実際の購買行動のうち、事前に計画された「計画購買」の比率は約11%に過ぎないとされる。むしろ、多くの購買は「\textbf{非計画購買}」であり、店内の店員による説明、試食、あるいはPOP(店頭広告)といった\textbf{インストア・プロモーション}によって、店舗内で最終的な意思決定がなされている。

\textbf{\paragraph{購買後の評価プロセス}}
購買は完了ではなく、次の意思決定へのインプットとなる。消費者は購買・使用後に「購買後の評価」を行う。
\begin{itemize}
	\item \textbf{満足の決定要因:} 企業の提供価値が、消費者の「\textbf{期待水準}」を上回るか否かによって満足・不満足が決定される。
	\item \textbf{満足した場合:} 期待通り、あるいは期待を上回った場合、消費者は「満足」する。これにより、\textbf{ヒューリスティクスの単純化}が起こる。つまり、「次回もこのブランドを買えば問題ない」と考えるようになり、将来の意思決定プロセスが簡略化される(=再購買)。
	\item \textbf{不満足した場合:} 期待に成果が満たない場合、「不満足」が生じる。これにより、\textbf{ヒューリスティクスの精緻化}が起こる。消費者は「なぜ失敗したのか」と不満要素を洗い出し、次回に向けて他の対案を積極的に情報収集しようとする意欲が高まる。
\end{itemize}

\textbf{\paragraph{不満足の影響と認知的不協和}}
不満足は、怒り、後悔、諦めといったネガティブな感情を生み出し、\textbf{再購買の中止}という企業にとって望ましくない結果をもたらす。
\begin{itemize}
	\item \textbf{口コミの影響:} 近年では、インターネットを通じて不満足な体験が共有されやすい。一般的に「\textbf{負の口コミ}」は「正の口コミ」の2倍以上の拡散力を持つとされ、企業へのダメージが大きい。
	\item \textbf{認知的不協和 (Cognitive Dissonance):} 一方で、購買後の消費者心理には「認知的不協和」と呼ばれる特有の状態も発生する。これは「自分の選択は本当に正しかったのか」という不安や葛藤である。この不協和を解消(=自分の選択を正当化)するため、消費者はあえて自分の選択を支持するような追加情報を収集したり、選択の失敗を(自分の中で)慰めようとしたりする行動をとることがある。これは企業にとって、購買後のコミュニケーション(例:御礼メール、使用方法のサポート)によって不協和を低減させ、ロイヤルティを高める機会とも言える。
\end{itemize}

\textbf{\subsubsection{AIによる補足:重要論点の拡張}}
本講義では補償型・非補償型モデルが紹介されたが、消費者がこれらの情報処理モデルを(意図的または無意識的に)使い分ける際の重要な前提条件として、「\textbf{関与 (Involvement)}」のレベルに関する言及が不足していた。
\begin{itemize}
	\item \textbf{関与とは:} 製品や購買シチュエーションに対する個人の重要度、関心度、またはリスクの知覚度合いを指す。
	\item \textbf{高関与と情報処理:} 自動車や住宅のような\textbf{高関与商材}(購買の失敗リスクが高い、価格が高い、自己表現性が高い)の場合、消費者は情報処理への動機づけが高く、能動的に情報を探索する。この際、講義で紹介された\textbf{補償型モデル}(フィッシュバイン・モデルなど)を用い、属性を詳細に比較検討する傾向が強い。
	\item \textbf{低関与と情報処理:} 日用品やスナック菓子のような\textbf{低関与商材}(失敗リスクが低い)の場合、消費者は情報処理の努力を最小限にしようとする。このため、\textbf{非補償型モデル}(特に習慣的選択や辞書編纂型)や、価格、ブランドの知名度、パッケージデザインといった末梢的な手がかり(ヒューリスティクス)に頼った意思決定を行う。
	\item \textbf{精緻化見込みモデル (ELM):} この関与のレベルが情報処理の深さを決定するという理論が「\textbf{精緻化見込みモデル (Elaboration Likelihood Model, ELM)}」である。高関与時は論理的な情報処理(\textbf{中心的ルート})が、低関与時はヒューリスティクスや感情的な手がかり(\textbf{周辺的ルート})が態度変容に影響を与えやすいとする。本講義の対案評価モデルの選択は、このELMの枠組みによって補完的に理解することができる。
\end{itemize}

\subsection{結論}
消費者の対案評価プロセスは、常に合理的であるとは限らない。コンピュータのように全属性を比較検討する「\textbf{補償型}」のアプローチと、情報処理の負荷を軽減するために簡便法を用いる「\textbf{非補償型}」のアプローチを、状況(選択肢の数や関与の度合い)に応じて使い分けている。
本講義から得られる実践的な教訓は、購買後の評価プロセスの重要性にある。特に「\textbf{不満足}」は、負の口コミを通じて甚大な被害をもたらす可能性があるため、期待水準のマネジメントと、不満足発生時の迅速な対応が不可欠である。同時に、消費者が購買後に抱く「\textbf{認知的不協和}」を理解し、それを低減させるためのコミュニケーション(選択の正当化支援)を行うことは、顧客の満足度とロイヤルティを構築する上で極めて重要である。

\subsection{重要キーワード一覧}
フィッシュバイン

\vspace{\baselineskip}
購買意思決定プロセス、属性、補償型モデル、非補償型モデル、フィッシュバイン・モデル、連結型モデル、分離型モデル、辞書編纂型モデル、ヒューリスティクス、非計画購買、期待水準、認知的不協和、関与、精緻化見込みモデル

\subsection{理解度確認クイズ}
\begin{enumerate}
	\item 製品のある属性における弱点(低い評価)を、他の属性の強み(高い評価)で相殺して総合的に評価するモデルを何と呼ぶか。
	\item 講義で紹介されたフィッシュバイン・モデルは、補償型・非補償型のどちらのモデルに分類されるか。
	\item 「各属性において、最低限この基準は満たしていなければならない」という足切りラインを設定し、一つでも下回れば選択肢から除外するモデルは何か。
	\item 「最も重要視する属性でまず全選択肢を比較し、優劣がつかなければ二番目に重要な属性で比較する」という評価モデルは何か。
	\item 「多数ある属性のうち、どれか一つでも非常に卓越した(十分条件を満たす)評価点があれば、その製品を選択する」という評価モデルは何か。
	\item 消費者が直面する選択肢の数が多い場合、情報処理の負荷を減らすため、補償型・非補償型のどちらのモデルが採用されやすいか。
	\item 人が意思決定を行う際に用いる、経験則に基づいた「簡便な解法や規則」を指す用語は何か。
	\item 消費者が購買後に抱く「この選択で本当に正しかっただろうか」という心理的な不安や葛藤を指す用語は何か。
	\item 消費者の購買後の満足度は、購買前の「何」と、実際の「パフォーマンス(成果)」との比較によって決まるか。
	\item 店舗内でのPOP広告、試食販売、店員による推奨など、店頭での刺激によって購買が決定される行動を何と呼ぶか。
	\item 期待水準に対して実際のパフォーマンスが下回った場合、消費者が抱く感情は何か。
	\item 満足した消費者が、次回の購買時に同じブランドを深く考えずに選択するようになる意思決定プロセスの簡略化を、講義では何と呼んだか。
	\item 一般的に、満足した体験よりも不満足な体験の方が、口コミとして他者により広く伝達されやすい傾向を指す言葉は何か(講義では「何の口コミ」が「何の口コミ」の2倍以上と表現されたか)。
	\item (AI補足より) 自動車や住宅のように、購買の失敗リスクが高く、消費者が情報探索に意欲的になる状態を「何のレベルが高い」と表現するか。
	\item (AI補足より) 精緻化見込みモデル(ELM)において、高関与の消費者が製品の属性情報を論理的に処理する経路を何と呼ぶか。
\end{enumerate}

\subsubsection*{解答一覧}
1. 補償型モデル、2. 補償型モデル、3. 連結型モデル、4. 辞書編纂型モデル、5. 分離型モデル、6. 非補償型モデル、7. ヒューリスティクス、8. 認知的不協和、9. 期待水準(または期待)、10. 非計画購買(またはインストア・マーチャンダイジング)、11. 不満足、12. ヒューリスティクスの単純化、13. 負の口コミ(が正の口コミの2倍以上)、14. 関与(のレベルが高い)、15. 中心的ルート

\section{購買行動状況の諸類型}

\subsection{はじめに}
本講義ノートは、消費者の購買行動が常に同一のプロセスを辿るわけではなく、状況によってそのパターンが異なることを理解する。特に、製品や購買に対する消費者の「\textbf{関与}」が、情報処理の深さと態度形成にどのように影響するかを「\textbf{精緻化見込みモデル(ELM)}」を通じて学ぶ。さらに、関与の程度と「\textbf{ブランド間の知覚差異}」という2つの軸を用いて、購買行動を4類型に分類する「\textbf{アサエルのモデル}」を理解することを目的とする。

\subsection{主要な概念と論点}
購買意思決定プロセスは固定的ではなく、製品の重要性や関心度によって、各段階の比重や情報処理の深さが変動する。

\subsubsection{関与 (Involvement)}
\textbf{関与}とは、消費者がある製品カテゴリーや購買状況について、どれほど\textbf{重要性}や(失敗の)\textbf{リスク}を感じるかという程度を指す。一般に「こだわり」とも表現される。この関与の水準が、情報処理の仕方に大きな影響を与える。

\subsubsection{精緻化見込みモデル (ELM: Elaboration Likelihood Model)}
ELMは、説得的コミュニケーション(広告など)が消費者の態度変容に及ぼす影響を説明する理論であり、関与水準によって情報処理のルートが分岐すると考える。
\begin{itemize}
	\item \textbf{中心的ルート (Central Route)}:
	      \begin{itemize}
		      \item \textbf{条件:} 製品への\textbf{関与水準が高い}(動機付けが高い)、かつ情報を処理する能力(知識)も高い場合。
		      \item \textbf{処理:} 消費者はメッセージ(情報)の内容を\textbf{精緻化}(=深く吟味)し、理論的に評価し、順位付けを行う。
		      \item \textbf{結果:} このルートで形成された態度は強固であり、持続性が高く、実際の購買行動に結びつきやすい。
	      \end{itemize}
	\item \textbf{周辺的ルート (Peripheral Route)}:
	      \begin{itemize}
		      \item \textbf{条件:} 製品への\textbf{関与水準が低い}(動機付けが低い)、または情報を処理する能力が低い場合。
		      \item \textbf{処理:} 消費者はメッセージの本質的な内容ではなく、広告塔(タレントイメージ)や感情、雰囲気といった\textbf{周辺的な手がかり}に注目して態度を決定する。
		      \item \textbf{結果:} このルートで形成された態度は一時的であり、変わりやすく、他の情報によって容易に変化する特徴がある。
	      \end{itemize}
\end{itemize}

\subsubsection{アサエル (Assael) の購買行動4類型}
マーケティング学者のヘンリー・アサエルは、消費者の購買行動を「\textbf{関与の程度}(高・低)」と「\textbf{ブランド間の知覚差異}(大・小)」の2軸でマッピングし、4つのタイプに分類した。

\begin{enumerate}
	\item \textbf{情報処理型購買行動 (Complex Buying Behavior)}: 高関与かつブランド間差異大
	      \begin{itemize}
		      \item \textbf{対象:} 高価な製品、めったに買わない製品、自己表現性の高い(こだわりの強い)製品。
		      \item \textbf{行動:} 消費者は集中的かつ包括的な情報処理を行う。まず製品カテゴリー内の各ブランドへの「\textbf{信念}」を(評価の上で)形成し、「\textbf{態度}」を決定し、その上で「\textbf{購買}」に至る。最も認知的・段階的なプロセスを辿る。
	      \end{itemize}

	\item \textbf{不協和解消型購買行動 (Dissonance-Reducing Buying Behavior)}: 高関与かつブランド間差異小
	      \begin{itemize}
		      \item \textbf{特徴:} 関与は高い(失敗したくない)が、ブランド間に大きな差があるとは認識していない。
		      \item \textbf{行動:} 事前に十分な情報収集を行っているため、購買自体は手早く済ませる傾向がある。重要なのは購買「後」であり、「本当にこの選択で良かったか」という\textbf{購買後の不協和(不安)}を解消するために、外部から追加情報を集める努力が行われる。
	      \end{itemize}

	\item \textbf{バラエティ・シーキング型購買行動 (Variety-Seeking Buying Behavior)}: 低関与かつブランド間差異大
	      \begin{itemize}
		      \item \textbf{特徴:} 関与は低いが、ブランド間の違いは認識している。
		      \item \textbf{行動:} 特定のブランドに不満があるわけではないが、\textbf{目新しさ}や\textbf{多様性}を求める動機から、頻繁に\textbf{ブランドスイッチ}(乗り換え)が発生する。
	      \end{itemize}

	\item \textbf{慣性型購買行動 (Habitual Buying Behavior)}: 低関与かつブランド間差異小
	      \begin{itemize}
		      \item \textbf{特徴:} 関与が低く、どれもこれも似たり寄ったりで、何を買っても大した問題はないと信じている。
		      \item \textbf{行動:} 特定の信念や態度に基づかず、習慣的に同じものを購入する。この行動は、企業側からは高いロイヤルティに見えるが、実際には積極的な選択ではない「\textbf{見かけ上のロイヤルティ}」である可能性が高い。
	      \end{itemize}
\end{enumerate}

\subsection{応用と事例分析}

\subsubsection{バラエティ・シーキング型:缶コーヒー}
講義では、\textbf{缶コーヒー}がバラエティ・シーキング型の典型例として挙げられた。日本市場では多様なメーカーが多種多様な製品を販売している。消費者(講義では講師自身の例)は、特定のブランド(例:ジョージア)を好んでいたとしても、新製品や他メーカーの製品を見ると「なんとなく試してみたくなる」ことがある。これは製品への不満ではなく、新奇性を求める心理に基づいている。

\subsubsection{慣性型購買行動:トイレットペーパー}
\textbf{トイレットペーパー}やティッシュペーパーは、\textbf{慣性型購買行動}の例として挙げられた(もちろん、強いこだわりを持つ消費者も存在する)。多くの消費者にとって、これらは低関与であり、ブランド間の差異も小さいと認識されるため、深く考えずにいつものブランドが選択される傾向がある。

\subsection{深層背景と教訓}

\textbf{\paragraph{バラエティ・シーキングへの企業対応}}
消費者のバラエティ・シーキング行動に対し、企業は「\textbf{ブランドの新鮮さ}」を製品属性として追加する戦略をとる。具体的には、シーズンごとに新製品や限定品を市場に投入することである。これにより、新奇性を求める消費者のブランドスイッチを取り込み、\textbf{一時的に市場シェアを向上}させることが可能になる。しかし、この戦略は諸刃の剣でもある。自社の既存ブランドの利用者も、他社の新製品にスイッチしてしまう可能性があるためである。したがって、企業は新製品投入によるシェア獲得と同時に、既存ブランドの\textbf{ブランド・ロイヤルティ}を高め、顧客の流出を防ぐ努力も行わなければならない。

\textbf{\paragraph{コンビニと「なんとなくの消費」}}
若年層を中心に広がるバラエティ・シーキング行動は、コンビニエンスストアの利用行動とも関連している。明確な用事がなくても、新商品を見るためだけにコンビニに立ち寄る消費者が存在する。企業が季節ごとに新製品を投入し続けるのは、このような消費者の「\textbf{新しさへのニーズ}」に応え、購買の楽しさを充足させるためでもある。

\textbf{\paragraph{「見かけ上のロイヤルティ」の罠}}
慣性型購買行動は、企業にとって重要な示唆を含む。消費者が「いつも同じメーカーの同じもの」を買っている場合、企業はそれを「高いロイヤルティを持つ顧客」と解釈しがちである。しかし、アサエルの分類によれば、それは低関与・低差異認識からくる単なる「\textbf{慣性}」であり、積極的な支持ではない「\textbf{見かけ上のロイヤルティ}(Pseudo-Loyalty)」に過ぎない可能性がある。このような顧客は、競合他社がより安価な代替品や魅力的なプロモーションを行えば、容易にスイッチしてしまうリスクを抱えている。

\textbf{\subsubsection{AIによる補足:重要論点の拡張}}
本講義では関与とブランド間差異の2軸で購買行動を整理したが、実務においてはこの2軸の「性質」について、さらに深い理解が必要である。
\begin{itemize}
	\item \textbf{関与の二面性(状況的関与と継続的関与):}
	      講義では「関与」を一つの尺度として扱ったが、関与には二種類ある。一つは、普段は関心がないが、購買が必要になった時だけ一時的に高まる「\textbf{状況的関与}」(例:引っ越しに伴う家具選び)である。もう一つは、個人の趣味や価値観と結びつき、常にその製品カテゴリーに関心を持ち続ける「\textbf{継続的関与}」(例:熱心なオーディオマニア)である。ELMにおける中心的ルートを辿るのは、主に後者の継続的関与が高い消費者、あるいは前者でも失敗のリスクが極めて高いと認識している消費者である。

	\item \textbf{「ブランド間の知覚差異」は操作可能である:}
	      アサエルのモデルにおいて「ブランド間の差異が小さい」と分類される市場(例:不協和解消型、慣性型)は、企業にとって固定的な所与の条件ではない。マーケティング活動の本質は、まさにこの「\textbf{知覚差異}」を創り出すことにある。\textbf{ブランディング}やイノベーションを通じて、競合他社との間に明確な差別化(機能、デザイン、意味、ストーリー)を打ち出し、消費者の認識を「差異小」から「差異大」へとシフトさせることができれば、慣性型購買(低価格競争)から脱し、情報処理型購買(高付加価値)の市場へと移行させることが可能になる。
\end{itemize}

\subsection{結論}
消費者の購買行動は一様ではなく、その製品や状況に対する「\textbf{関与}」のレベルと、市場における「\textbf{ブランド間の知覚差異}」によって、情報処理の深さ(ELM)と行動パターン(アサエルの4類型)が決定される。
本講義から得られる実践的な教訓は、自社製品が消費者に「どの類型」として購買されているかを正確に見極めることの重要性である。特に、\textbf{慣性型購買}を「高いロイヤルティ」と誤解することは、企業にとって致命的な戦略ミスにつながりかねない。バラエティ・シーキングが主流の市場では、ロイヤルティ維持と新奇性追求のバランス感覚が、マーケティング戦略の鍵となる。

\subsection{重要キーワード一覧}
アサエル(ヘンリー・アサエル)

\vspace{\baselineskip}
関与、精緻化見込みモデル(ELM)、中心的ルート、周辺的ルート、ブランド間の知覚差異、情報処理型購買行動、不協和解消型購買行動、バラエティ・シーキング、ブランドスイッチ、慣性型購買行動、見かけ上のロイヤルティ、ブランド・ロイヤルティ

\subsection{理解度確認クイズ}
\begin{enumerate}[label=\arabic*.]
	\item 消費者がある製品カテゴリーに対して感じる重要性やリスクの認識度合いを何と呼ぶか。
	\item 関与水準が高い時と低い時で、消費者の情報処理プロセスが異なると説明する態度変容モデルを何と呼ぶか。
	\item 精緻化見込みモデル(ELM)において、関与水準が高く、情報の内容を深く吟味・評価する情報処理経路を何と呼ぶか。
	\item 精緻化見込みモデル(ELM)において、関与水準が低く、広告タレントのイメージや雰囲気などの手がかりで態度を決定する経路を何と呼ぶか。
	\item 周辺的ルートで形成された態度は、中心的ルートで形成された態度と比較して、どのような特徴を持つか。
	\item 「関与」と「ブランド間の知覚差異」の2軸で購買行動を4類型に分類した学者(姓)は誰か。
	\item アサエルの4類型モデルで用いられる2つの分類軸は何か。
	\item 高関与で、ブランド間の差異も大きいと認識されている場合の購買行動(例:高価な自動車の購入)を何と呼ぶか。
	\item 高関与だが、ブランド間の差異は小さいと認識されている場合の購買行動(例:高価だが差が分かりにくいカーペット)を何と呼ぶか。
	\item 低関与だが、ブランド間の差異は大きいと認識されている場合の購買行動(例:缶コーヒー)を何と呼ぶか。
	\item 低関与で、ブランド間の差異も小さいと認識されている場合の購買行動(例:トイレットペーパー)を何と呼ぶか。
	\item バラエティ・シーキング型購買行動において、消費者が目新しさを求めて購入ブランドを頻繁に変更することを何と呼ぶか。
	\item バラエティ・シーキング型購買行動は、製品への「不満」が原因で起こるのか、それとも別の要因か。
	\item 慣性型購買行動において、一見するとロイヤルティが高そうに見えるが、実際には積極的な支持ではない状態を何と呼ぶか。
	\item バラエティ・シーキング型の消費者が他社製品に流出するのを防ぐために、企業が既存ブランドに対して行うべき施策は何か。
\end{enumerate}

\subsubsection*{解答一覧}
1. 関与、2. 精緻化見込みモデル(ELM)、3. 中心的ルート、4. 周辺的ルート、5. 態度の持続性が低く、変わりやすい(または、説得に結びつきにくい)、6. アサエル、7. 関与(の程度)、ブランド間の知覚差異、8. 情報処理型購買行動、9. 不協和解消型購買行動、10. バラエティ・シーキング型購買行動、11. 慣性型購買行動、12. ブランドスイッチ、13. 目新しさや多様性を求めるため(不満が原因ではない)、14. 見かけ上のロイヤルティ、15. ブランド・ロイヤルティ(の向上)

\end{document}
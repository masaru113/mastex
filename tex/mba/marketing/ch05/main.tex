\documentclass[uplatex,a4j,12pt,dvipdfmx]{jsarticle}
\usepackage{amsmath,amsthm,amssymb,bm,color,enumitem,mathrsfs,url,epic,eepic,ascmac,ulem,here,ascmac}
\usepackage[letterpaper,top=2cm,bottom=2cm,left=3cm,right=3cm,marginparwidth=1.75cm]{geometry}
\usepackage[english]{babel}
\usepackage[dvipdfm]{graphicx}
\usepackage[hypertex]{hyperref}

\title{Marketing, Lecture 5: Lecture Notes}
\author{Masaru Okada}
\date{\today}

\begin{document}
\maketitle
\tableofcontents

\section{Lecture Material Review: Consumer Behavior Decision-Making Processes and Purchasing Typologies}

\subsection{Introduction}

Understanding the customer—that is, the consumer—is indispensable in formulating marketing strategy. To elucidate the mechanisms by which consumers perceive their needs, recognize products and services, and ultimately arrive at a purchase provides the foundation for developing an effective \textbf{marketing mix (the 4Ps)}.
This lecture note aims to systematically organize the series of activities consumers engage in when 'acquiring, consuming, and disposing' of products and services—namely, \textbf{consumer behavior}. It seeks to clarify the core issues of marketing studies in an MBA context by examining these psychological processes and behavioral patterns.


\subsection{Key Concepts and Issues}


\subsubsection{Understanding the Market and Defining Consumer Behavior}
As a prerequisite for marketing activities, the target market must be accurately understood. The management scholar \textbf{Philip Kotler} proposed the \textbf{'Seven Os'} as a framework for understanding the market.

\begin{itemize}
	\item \textbf{Occupants}: Who constitutes the market?
	\item \textbf{Objects}: What is being bought?
	\item \textbf{Objectives}: Why do they buy?
	\item \textbf{Organization}: Who participates in the buying?
	\item \textbf{Occasions}: When do they buy?
	\item \textbf{Outlet}: Where do they buy?
	\item \textbf{Operations}: How do they buy?
\end{itemize}
The field of consumer behavior theory specifically seeks to answer questions such as 'why,' 'who,' and 'how.'

Regarding the definition of consumer behavior, \textbf{Blackwell, Miniard, and Engel (2005)} define it as 'the activities people undertake when obtaining, consuming, and disposing of products and services.' \textbf{Hoyer and MacInnis (2009)} define it as 'the totality of decisions about the acquisition, consumption, and disposition of goods, services, time, and ideas by human decision-making units.' These definitions indicate that the subject of analysis is not just the single point of purchase, but the entire process preceding and following it.

\subsubsection{The Consumer Purchasing Decision-Making Process}
Consumer purchasing behavior is generally explained as an information-processing model that proceeds through the following five stages.

\begin{enumerate}
	\item \textbf{Problem Recognition}: The consumer perceives a divergence (gap) between their 'ideal state' and 'actual state.' When this gap exceeds a certain threshold, it is recognized as a problem (need) to be solved.
	\item \textbf{Information Search}: Activities undertaken to gather information to solve the problem.
	\item \textbf{Evaluation of Alternatives}: Based on the collected information, multiple options (brands or products) are compared and evaluated.
	\item \textbf{Choice (Purchase)}: The alternative with the highest evaluation is selected, leading to the actual purchasing action.
	\item \textbf{Evaluation of Results}: After the purchase, the consumer evaluates whether the product's performance post-use exceeded (satisfaction) or fell short of (dissatisfaction) their prior 'level of expectation.'
\end{enumerate}

\subsubsection{Information Search: Internal and External Search}
Information search is broadly divided into two categories based on the source.
\begin{itemize}
	\item \textbf{Internal Search}: An activity involving the retrieval of knowledge and experience stored in the consumer's own memory (\textbf{long-term memory}). Information about brands is stored as an 'associative network.'
	\item \textbf{External Search}: When internal search fails to yield sufficient information, this is the activity of actively collecting information from external sources such as stores, advertisements, the internet, or acquaintances.
\end{itemize}
Depending on the presence of purchase experience and the degree of involvement, the scope of information search is classified into \textbf{'Extensive Problem Solving'} (high involvement, new purchase), \textbf{'Limited Problem Solving'} (medium involvement, known), and \textbf{'Routine Problem Solving'} (low involvement, repeat purchase).

\subsubsection{Evaluation of Alternatives: Information Integration Strategies}
Consumers conduct an overall evaluation of a product based on multiple evaluation criteria (attributes). The methods for integrating this information are largely classified into two types.
\begin{itemize}
	\item \textbf{Compensatory Model}: A model where a low evaluation on one attribute can be compensated for by a high evaluation on another attribute, forming an overall assessment. A typical example is the \textbf{Fishbein Model}. This model posits that an attitude is formed by the sum of the products of the 'importance (evaluation)' of each attribute and the 'belief (subjective evaluation)' that the product possesses that attribute.
	\item \textbf{Non-compensatory Model}: A model that does not consider all attributes, but instead narrows down options based on specific criteria.
	      \begin{itemize}
		      \item \textbf{Conjunctive}: Sets minimum necessary conditions (cut-off points) for each attribute and eliminates any option that fails to meet even one.
		      \item \textbf{Disjunctive}: Adopts an option if it meets at least one sufficient condition (an outstanding point).
		      \item \textbf{Lexicographic}: Compares options based on the most important attribute. If no difference is found, the comparison moves to the next most important attribute.
	      \end{itemize}
\end{itemize}
Generally, compensatory models tend to be used when there are few options, while non-compensatory models are favored when there are many.

\subsubsection{Post-Purchase Evaluation and Cognitive Dissonance}
Post-purchase evaluation significantly influences future purchasing behavior. In particular, if the outcome falls below the expectation level (dissatisfaction), it can lead to negative word-of-mouth and the cessation of repeat purchases.
Furthermore, especially in cases of expensive purchases or when there was no compelling difference between the chosen options, consumers may feel a psychological anxiety about whether they made the right choice, known as \textbf{cognitive dissonance}.

\subsubsection{Product Involvement and the Elaboration Likelihood Model (ELM)}
The way consumers process information varies depending on their degree of \textbf{product involvement} (personal relevance, perceived importance, or risk). The \textbf{Elaboration Likelihood Model (ELM)}, proposed by \textbf{Petty and Cacioppo (1986)}, illustrates this relationship.
\begin{itemize}
	\item \textbf{Central Route (High Involvement)}: Highly involved consumers meticulously examine and elaborate on the message content (quality, features, etc.) to form an attitude.
	\item \textbf{Peripheral Route (Low Involvement)}: Consumers with low involvement are less influenced by the message content and more by peripheral cues, such as the celebrity in the ad, package design, or atmosphere, when forming an attitude.
\end{itemize}
Attitudes formed via the central route are strong and likely to lead to behavior, whereas those formed via the peripheral route are temporary and susceptible to change.

\subsubsection{Typology of Buying Behavior (Assael's Model)}
\textbf{Assael (1987)} classified buying behavior into four types using two axes: 'degree of product involvement (high/low)' and 'perceived differences between brands (large/small).'

\begin{itemize}
	\item \textbf{Complex Buying Behavior (High Involvement / Significant Differences)}: E.g., automobiles, PCs. Consumers go through the purchasing decision-making process in detail, following a cognitive learning process of Beliefs -> Attitudes -> Behavior.
	\item \textbf{Dissonance-Reducing Buying Behavior (High Involvement / Few Differences)}: E.g., high-end furniture, carpets. Because brand differences are difficult to discern, consumers first make a purchase (Behavior), and then form beliefs or attitudes to justify their choice (resolving cognitive dissonance).
	\item \textbf{Variety-Seeking Buying Behavior (Low Involvement / Significant Differences)}: E.g., canned coffee, snacks. Consumers frequently switch brands, driven by a desire for novelty or variety, even if not dissatisfied.
	\item \textbf{Habitual Buying Behavior (Low Involvement / Few Differences)}: E.g., toilet paper, batteries. Consumers engage in almost no information search, purchasing habitually based on 'apparent loyalty' with minimal effort.
\end{itemize}


\subsection{Application and Case Analysis}


\subsubsection{Degree of External Search and Marketing Strategy}
In the lecture, automobiles (high) and chocolate (low) were cited as examples of products with differing degrees of external search.
\begin{itemize}
	\item \textbf{Automobiles (High External Search)}: Consumers actively collect diverse information on price, quality, fuel efficiency, design, etc. Therefore, it is crucial for companies to engage in activities that support consumer information gathering, such as detailed explanations by salespeople, comprehensive brochures, and providing information to comparison websites.
	\item \textbf{Chocolate (Low External Search)}: Consumers often make decisions quickly in-store. In this case, embedding the brand name into \textbf{internal memory (long-term memory)} through advertising (especially TV commercials) and facilitating brand recall (unaided or aided) at the point of purchase directly links to sales.
\end{itemize}

\subsubsection{The Reality of Unplanned Purchases and In-Store Marketing}
Research by Yukihiro Aoki et al. (1989) indicated that approximately 89\% of purchases in supermarkets are 'unplanned purchases.' This contrasts with 'planned purchases' (11.0\%), which were planned in advance. This shows an overwhelmingly high proportion of decisions made in-store, such as deciding on a brand (brand selection) or buying something not originally intended (reminder purchase, \textbf{impulse purchase}).
This fact signifies that consumers conduct much of their purchasing decision-making process at the point of sale (Outlet), underscoring the importance of in-store marketing, such as POP (point-of-purchase) advertising, display methods, and product sampling.

\subsubsection{Variety Seeking and Product Strategy}
The canned coffee market (e.g., BOSS, TULLY'S, GEORGIA) is a prime example of \textbf{variety-seeking} buying behavior. Even if consumers are not dissatisfied with a particular brand, they switch brands seeking novelty, thinking, 'I'll try a different flavor today.'
In this market, to maintain and expand market share, companies must adopt strategies that continuously meet the consumer's desire for 'brand freshness.' This requires constantly introducing new flavors, renewing packaging, and developing seasonal limited-edition products.


\subsection{Deeper Context and Lessons}

Here, we organize the practical implications and background context surrounding the main points of this lecture.

\textbf{\paragraph{Elucidating Consumer Psychology through Hypothesis Testing}}
The lecture presented the hypothesis that 'the colder it gets, the better rich-flavored ice cream sells.' This demonstrates that consumer behavior theory is not mere desk theory, but an empirical discipline that delves into psychological and physiological factors—'Why? (Is it an instinct to store fat?)'—and confirms market facts through data-driven hypothesis testing.

\textbf{\paragraph{The Trap of 'Apparent Loyalty' (Habitual Buying)}}
\textbf{Habitual buying} of items like toilet paper or batteries, at first glance, seems to indicate high loyalty, as a specific brand is chosen repeatedly (= apparent loyalty). However, this is not due to active preference but to passive reasons (inertia), such as 'it's bothersome to choose' or 'they are all the same.' In such markets, there is a constant risk that consumers will easily switch brands if a competitor introduces a low-price strategy or a novel feature (e.g., scented toilet paper).

\textbf{\paragraph{Teenage Females and Reference Groups}}
The lecture suggested that 'limited quantity products' might be effective for 'teenage girls, who are highly conscious of their reference groups.' This implies that consumer behavior is strongly influenced by the norms and values of the groups to which they belong or aspire (reference groups). This tendency is particularly pronounced for products with high self-expression, such as fashion and music.

\textbf{\subsubsection{AI Supplement: Extension of Key Issues}}
In the context of this lecture, we supplement with important related concepts that merit further exploration.

\begin{itemize}
	\item \textbf{Reference Group}: Related to the topic above, this is a crucial concept in consumer behavior. A consumer's attitude formation and purchasing behavior are strongly influenced by family and friends (primary groups), or by admired figures or experts to whom they do not belong (aspirational groups). Especially in \textbf{complex buying behavior} or for products with high social risk (how one is seen by others), the word-of-mouth or recommendations from \textbf{opinion leaders} within these reference groups can be more powerful than corporate advertising.

	\item \textbf{Perceived Risk}: While 'high product price' or 'quality variability' were mentioned as motivations for external search, these are components of the 'perceived risk' consumers feel when making a purchase. Perceived risk includes 'functional risk' (not getting the expected performance), 'financial risk' (losing money), 'social risk' (lowering one's evaluation by others), and 'psychological risk' (conflicting with one's self-image). Consumers intensify their information search to reduce this perceived risk, or they engage in \textbf{dissonance-reducing} behavior (post-purchase justification).
\end{itemize}


\subsection{Conclusion}

In this lecture note, we learned the basic framework of consumer behavior. Consumer behavior theory is an academic field that systematically elucidates, from a psychological perspective, the 'who, what, why, and how' of consumer purchasing, as represented by Kotler's 'Seven Os.'

The consumer purchasing process is understood as a series of steps from 'problem recognition' to 'post-purchase evaluation.' Notably, the degree of 'product involvement' and 'perceived differences between brands' are key factors that determine the \textbf{ELM} (information processing route) and \textbf{Assael's Model} (typology of buying behavior).

A practical lesson from this lecture is the importance for marketers to accurately grasp which behavior type (e.g., \textbf{complex} or \textbf{habitual}) consumers use when purchasing their products. Furthermore, the fact that many purchasing decisions are made in-store (\textbf{unplanned purchases}), as shown by Aoki's research, suggests that channel strategy and in-store marketing are decisive factors influencing sales, just as much as (or perhaps even more than) advertising strategy.


\subsection{Key Terms List}

Philip Kotler, Blackwell, Miniard, Engel, Hoyer, MacInnis, Shuzo Abe, Masao Nakanishi, Fishbein, Yukihiro Aoki, Petty, Cacioppo, Assael
\vspace{\baselineskip}

Marketing Mix, Consumer Behavior, Purchasing Decision-Making Process, Problem Recognition, Information Search (Internal/External), Short-Term Memory, Long-Term Memory, Associative Network, Extensive Problem Solving, Limited Problem Solving, Routine Problem Solving, Compensatory Model, Non-compensatory Model, Fishbein Model, Unplanned Purchase, Impulse Purchase, Cognitive Dissonance, Product Involvement, Elaboration Likelihood Model (ELM), Central Route, Peripheral Route, Dissonance-Reducing Buying Behavior, Variety-Seeking, Habitual Buying Behavior, Reference Group, Perceived Risk

\subsection{Comprehension Quiz}
\begin{enumerate}
	\item Which of Kotler's 'Seven Os' asks about the 'purpose' of the purchase?
	\item What is the term for the action of searching for information based on a consumer's own memory (internal information)?
	\item What is the pattern of purchase decision-making called when a consumer has no prior experience and requires a lot of information?
	\item What is the evaluation method called where a negative evaluation on one attribute (e.g., price) is offset by a positive evaluation on another (e.g., design)?
	\item What is the evaluation method called where products are compared on the most important attribute, and if no difference is found, the next most important attribute is used?
	\item Is the Fishbein Model classified as compensatory or non-compensatory?
	\item What is the term for the anxiety or regret a consumer feels after a purchase, wondering if their choice was correct?
	\item What concept indicates the degree of importance or risk a consumer feels towards a product?
	\item In the Elaboration Likelihood Model (ELM), what is the information processing route taken by low-involvement consumers called?
	\item In the ELM, what is the information processing route taken by high-involvement consumers called?
	\item In Assael's buying behavior typology, what is the type for high involvement and small perceived brand differences?
	\item In Assael's buying behavior typology, what is the type for low involvement and large perceived brand differences?
	\item In Assael's buying behavior typology, what is the type for high involvement and large perceived brand differences?
	\item In Assael's buying behavior typology, what is the type for low involvement and small perceived brand differences?
	\item What is the term for frequently switching brands to seek novelty, even without dissatisfaction with the existing brand?
\end{enumerate}

\subsubsection*{Answer Key}
1. Objectives, 2. Internal Search, 3. Extensive Problem Solving, 4. Compensatory Model (Strategy), 5. Lexicographic Model, 6. Compensatory, 7. Cognitive Dissonance, 8. Product Involvement (or Purchasing Involvement), 9. Peripheral Route, 10. Central Route, 11. Dissonance-Reducing Buying Behavior, 12. Variety-Seeking Buying Behavior, 13. Complex Buying Behavior (Information Processing Type), 14. Habitual Buying Behavior, 15. Variety-Seeking

\section{Marketing and Consumer Behavior}

\subsection{Introduction}
This lecture (No. 5) explains the fundamental framework of \textbf{consumer behavior theory}. It outlines the necessity of understanding consumers in corporate marketing activities and the significance of that analysis. Consumer behavior theory is the study that explains the process by which a representative consumer gathers, understands, and utilizes information, ultimately leading to purchasing behavior. This note organizes the main concepts, the significance of analysis, and its applications as presented in the lecture.

\subsection{Key Concepts and Issues}

\subsubsection{The Position of Consumer Behavior in Marketing}
As a prerequisite for corporate marketing activities, it is necessary to understand the market. Philip \textbf{Kotler} presented the \textbf{'7Os'} framework as a perspective for understanding the market.
\begin{itemize}
	\item \textbf{Occupants}: Who constitutes the market?
	\item \textbf{Objects}: What is being bought?
	\item \textbf{Objectives}: Why do they buy?
	\item \textbf{Organizations}: Who participates in the buying?
	\item \textbf{Occasions}: When do they buy?
	\item \textbf{Outlets}: Where do they buy?
	\item \textbf{Operations}: How do they buy?
\end{itemize}
The consumer behavior theory discussed in this lecture is the academic domain dedicated to deeply understanding \textbf{'Occupants'} in particular, and it elucidates the process of 'how, where, when, and under whose influence' consumers make purchases.

\subsubsection{Definition of Consumer Behavior}
\paragraph{What is Consumption?}
The lecture defined consumption as 'using up,' explaining it as the act of eliminating \textbf{goods and services} to satisfy desires. A society where such actors gather is called a 'consumer society.'

\paragraph{Academic Definitions}
\begin{itemize}
	\item \textbf{American Marketing Association (AMA)}: Defined consumer behavior as the actions of consumers (decision-makers) in the product and service market, positioning it as an \textbf{interdisciplinary} research area for understanding and describing these actions.
	\item \textbf{Blackwell et al. (2005)}: All activities people engage in when acquiring, consuming, and disposing of products and services.
	\item \textbf{Hoyer and MacInnis (2009)}: The totality of decisions made by a decision-making unit regarding the acquisition, consumption, and disposition of goods, services, time, and ideas.
\end{itemize}

\subsubsection{The Scope of Consumer Behavior Analysis}
The consumer behavior that companies seek to understand is broadly divided into two aspects.
\begin{enumerate}
	\item \textbf{External Behavior (Result)}: Overt actions such as 'purchase,' 'do not purchase,' or 'only gather information.' These are observable as numerical results, like sales, in response to marketing efforts such as advertising.
	\item \textbf{Internal Process (Stages)}: The abstract, invisible psychological process of how a stimulus (e.g., advertising) leads to feelings about a product or brand (\textbf{perception}) and how that, in turn, connects to the next action.
\end{enumerate}

\subsection{Application and Case Analysis}

\subsubsection{The Significance of Analyzing Consumer Behavior for Companies}
The primary objective for companies analyzing consumer behavior is 'to formulate effective marketing strategies tailored to the identified tendencies of consumer characteristics.'
\begin{itemize}
	\item \textbf{Formulating Effective Marketing Strategies}: Designing \textbf{retail store} setups and promotions (\textbf{4Ps}) aligned with the target segment's preferences, purchasing locations, and purchasing \textbf{scenes} to increase profitability.
	\item \textbf{Detailed Description of Market Conditions}: The market is shaped not only by theoretical frameworks like \textbf{product differentiation}, \textbf{market segmentation}, and the \textbf{product life cycle}, but also by the preferences and actual purchasing behaviors of the consumers within it. Consumer behavior analysis provides the means to describe this reality.
	\item \textbf{Quantitative Grasp of the Real Market}:
	      Economic price/\textbf{quantity} models attempt to explain theoretical mechanisms by abstracting (isolating) all conditions other than price and quantity, thus having limitations in explaining real-world market behavior.
	      In contrast, consumer behavior models can incorporate various \textbf{variables} (such as psychological factors) and analyze how they influence the actual market situation (purchasing behavior) as \textbf{'measurable behavioral data.'}
\end{itemize}

\subsubsection{Analysis Case Study: The Hypothesis Testing Process}
Consumer behavior analysis is utilized in the process of testing hypotheses about 'why a phenomenon occurs' using tools like surveys.

\paragraph{Case 1: Reasons for Purchasing Rich Ice Cream}
\begin{itemize}
	\item \textbf{Phenomenon}: Rich-flavored ice cream is selling well.
	\item \textbf{Hypothesis}: Perhaps it is because the season has turned colder.
	\item \textbf{Deeper Inquiry}: Explore the psychological factors—'Why do consumers prefer rich flavors over light ones when it gets cold?' (e.g., satisfaction, association with warmth).
	\item \textbf{Analysis}: Numerically verify whether those psychological factors are influencing actual purchasing behavior.
\end{itemize}

\paragraph{Case 2: Characteristics of New Product Purchasers}
\begin{itemize}
	\item \textbf{Research Subject}: What are the characteristics of consumers who buy new products?
	\item \textbf{Analysis Result}: A positive \textbf{correlation} was found between 'people who spend a long time using the internet' and 'new product purchasing behavior.'
	\item \textbf{Application to Strategy}: Based on this result, the company can predict that 'the internet is an effective channel for new product promotions.'
	\item \textbf{Reasoning (Hypothesis)}: The conceivable reason (hypothesis) is 'because they are frequently exposed to new information via the internet,' yielding the insight that information exposure is important.
\end{itemize}
Thus, as Baker also points out, it becomes possible to predict the relationship between the 4Ps (Product, Price, Promotion, Place) and consumer response, and to make plans to increase purchase frequency.

\subsubsection{Application to Product Concepts (Case Study)}
\begin{itemize}
	\item \textbf{Target Setting}: Teenage females.
	\item \textbf{Identified Characteristic}: This target segment is extremely conscious of their \textbf{reference groups} (friends, etc.) compared to other generations.
	\item \textbf{Strategic Consideration}: Based on this characteristic, applications such as internally reviewing whether a marketing plan 'that emphasizes \textbf{scarcity} by deliberately limiting product quantity' would be effective can be considered.
\end{itemize}

\subsection{Deeper Context and Lessons}

\textbf{\paragraph{Detour Topic: Preconditions of the 'Consumer' in Consumer Behavior Theory}}
In the consumer behavior theory discussed in this lecture, the basic unit of analysis assumes that 'the consumer makes purchasing decisions alone and for themselves.' While in reality, many cases involve joint decisions, like a refrigerator for family use, the academic premise focuses on the intra-personal psychological process and decision-making.
This is the fundamental difference from \textbf{'industrial buying'} (organizational buying), which involves rational decision-making as an organization.

\textbf{\paragraph{Detour Topic: Consumer 'Heterogeneity' and 'Generality'}}
When analyzing consumer behavior, two aspects are considered. One is \textbf{'heterogeneity,'} such as human individuality and preferences. The other is \textbf{'generality,'} referring to the framework of shopping behavior to some extent.
Consumer behavior theory often proceeds with its analysis by abstracting this complexity and assuming 'one representative consumer.' In marketing strategy, consumer \textbf{'heterogeneity'} is treated as 'differences between market segments (\textbf{market segmentation})'.

\textbf{\subsubsection{AI Supplement: Extension of Key Issues}}
\textbf{Supplementing the Consumer Decision-Making Process Model}

The lecture mentioned that consumer behavior 'constitutes stages,' but a detailed model of its specific content (the internal process) was not presented. In MBA consumer behavior theory, the most fundamental and crucial framework is the \textbf{'Consumer Decision-Making Process Model.'}

This model captures the internal process leading to a purchase in the following five stages.
\begin{enumerate}
	\item \textbf{Problem Recognition}: The stage where the consumer recognizes a gap between their 'ideal state' and 'current state.' (e.g., 'I am thirsty.')
	\item \textbf{Information Search}: The stage of searching for information on ways to fill the gap. This includes internal search (memory) and external search (internet, friends, in-store).
	\item \textbf{Alternative Evaluation}: The stage of comparing and examining multiple options (brands or products) based on specific evaluation criteria (e.g., price, quality, design) using the collected information.
	\item \textbf{Purchase Decision}: The stage of selecting the highest-evaluated alternative and deciding to actually purchase (or not to purchase).
	\item \textbf{Post-purchase Behavior}: The stage of feeling satisfaction or dissatisfaction based on the result of using the product. This evaluation influences the next purchasing behavior (repeat purchase or word-of-mouth).
\end{enumerate}
The analysis mentioned in the lecture, such as 'why rich ice cream sells,' is an attempt to elucidate what psychological factors consumers prioritize at the 'Alternative Evaluation' or 'Problem Recognition' stages of this process.

\subsection{Conclusion}
This lecture note organized the role of consumer behavior theory as the foundation for marketing strategy formulation. Elucidating the 'Occupants' in Kotler's 7Os is the core objective of this academic field.

The cases presented in the lecture (ice cream, new product purchases) show the importance not just of tracking the results (external behavior) of 'what sold,' but also of modeling and understanding the consumer's \textbf{internal process} of 'why it was chosen' through hypothesis testing.

As a \textbf{practical lesson}, consumer behavior theory provides a means to analyze the complex and elusive market as 'measurable behavioral data.' This allows marketers to bridge the gap between economic theoretical models and the actual market, yielding practical insights for formulating more accurate marketing strategies.

\subsection{Key Terms List}
Kotler, Blackwell, Hoyer, MacInNis, Baker
\vspace{\baselineskip}
Marketing Mix (4Ps), Market Segmentation, Product Differentiation, Product Life Cycle, Industrial Buying, Reference Group, Scarcity, Consumer Decision-Making Process

\subsection{Comprehension Quiz}
\begin{enumerate}
	\item In Kotler's '7Os' framework for understanding the market, which element asks 'Who constitutes the market?'
	\item In the AMA's definition, what kind of research area is consumer behavior theory positioned as?
	\item Of the subjects of consumer behavior analysis, what are objectively observable actions, such as 'purchase' or 'do not purchase,' called?
	\item Of the subjects of consumer behavior analysis, what are the invisible psychological processes, such as 'perception' and 'emotion,' called?
	\item Compared to economic price/quantity models, what is the analytical advantage of consumer behavior models?
	\item When a statistical relationship is found between a phenomenon (e.g., new product purchase) and a specific element (e.g., internet usage time), what is this relationship called?
	\item In marketing strategy, what is the combination of the four elements—Product, Price, Place, and Promotion—called?
	\item What is the term for the groups to which consumers belong or that they wish to use as a standard for their behavior?
	\item In the analytical premises of consumer behavior theory discussed in the lecture, what is the biggest characteristic of 'industrial buying' as contrasted with family decision-making?
	\item What concept indicates that the needs and characteristics of consumers composing a market are diverse rather than uniform?
	\item What is the first stage of the consumer decision-making process, where the consumer recognizes a gap between their 'ideal state' and 'current state'?
	\item What is the stage where the consumer searches their memory (internal search) or asks the internet or friends (external search) before making a purchase decision?
	\item What is the stage where the consumer compares multiple options based on specific criteria such as price or quality?
	\item What is the stage where the satisfaction or dissatisfaction felt by the consumer after using a product influences their next purchasing behavior?
	\item A strategy that aims to increase consumer purchase intent by deliberately limiting the supply utilizes what principle?
\end{enumerate}

\subsubsection*{Answer Key}
1. Occupants, 2. An interdisciplinary research area, 3. External behavior (or resultant behavior), 4. Internal process, 5. (e.g., It can explain the real market with measurable data, considering various variables.), 6. Correlation, 7. Marketing Mix (or 4Ps), 8. Reference Group, 9. (e.g., Making rational decisions as an organization), 10. Market heterogeneity, 11. Problem Recognition, 12. Information Search, 13. Alternative Evaluation, 14. Post-purchase Behavior, 15. (The principle of) Scarcity

\section{Consumer Information Search}


\subsection{Introduction}
The objective of this lecture is to understand the psychological process by which consumers arrive at the action of 'purchasing.' Consumer behavior theory analyzes purchasing behavior as an information-processing model that takes place within the mind of each individual consumer as the protagonist. This note focuses on the core \textbf{purchasing decision-making process}, particularly how consumer memory is involved (\textbf{information search}).

\subsection{Key Concepts and Issues}

\subsubsection{The Purchasing Decision-Making Process}
Consumer purchasing behavior is generally modeled as an information-processing process that chronologically follows the five stages below.
\begin{enumerate}
	\item \textbf{Problem Recognition}: The stage where needs or wants arise.
	\item \textbf{Information Search}: The stage of collecting information to solve the problem.
	\item \textbf{Alternative Evaluation}: The stage of comparing and evaluating multiple options based on the collected information.
	\item \textbf{Purchase Decision}: The stage of selecting the option evaluated as most desirable and making the actual purchase.
	\item \textbf{Post-Purchase Evaluation}: The stage of evaluating whether one's purchase choice was correct, leading to feelings of satisfaction or dissatisfaction (cognitive dissonance).
\end{enumerate}
At each of these stages, individual psychological factors (motivation, perception, learning, etc.) and external environmental factors (culture, social class, family, etc.) exert influence.

\subsubsection{Problem Recognition}
The process begins when the consumer recognizes a gap (divergence) between their \textbf{'ideal state'} and \textbf{'actual state'} that is too significant to ignore. For example, the moment a consumer sees a new bag a friend has and feels 'That's nice,' a gap arises between 'myself without that bag (actual)' and 'myself with it (ideal),' which is then recognized as a need for the bag—that is, a 'problem.'

\subsubsection{Information Search}
After recognizing the problem, the consumer begins an information search seeking a solution.
\begin{description}
	\item[\textbf{Internal Search}] First, the consumer searches their own memory—that is, \textbf{internal information} (past experiences, knowledge). If sufficient information is obtained at this stage and the problem is deemed solvable, the process moves quickly to the purchase decision.
	\item[\textbf{External Search}] If internal information is insufficient, the consumer transitions to \textbf{external search}. They actively collect new information from external sources such as the internet, word-of-mouth from acquaintances and friends, in-store comparisons, and brochures.
\end{description}

\subsubsection{The Structure and Process of Memory}
Closely related to internal search is the mechanism of human memory. Memory is generally modeled with three storage components.
\begin{itemize}
	\item \textbf{Sensory Memory (Sensory Register)}: Instantaneously holds stimuli received from sensory organs (vision, hearing, etc.) (for less than a second).
	\item \textbf{Short-Term Memory}: Information from sensory memory that has been attended to is transferred here and held for about tens of seconds. Its capacity is limited (the magic number 7±2).
	\item \textbf{Long-Term Memory}: Information in short-term memory is stored permanently by being rehearsed (repeated) or by being associated with existing knowledge and encoded (given meaning). Its capacity is considered almost limitless.
\end{itemize}

Information flows from external stimuli via the 'sensory register,' is temporarily processed in 'short-term memory,' and through repetition, flows into 'long-term memory.'

\subsubsection{Associative Network Model and Brand Image}
Long-term memory is not just a list of information; it is thought to be a vast network structure (\textbf{Associative Network Model}) where related concepts (nodes) are connected by links.
For example, the 'Coca-Cola' node is strongly linked to other nodes like 'America,' 'red color,' 'carbonated,' and 'distinctive bottle shape.'

\textbf{Brand image} is precisely a manifestation of this associative network. It is formed in the consumer's mind as a result of various things, attributes, emotions, and experiences being linked to a certain brand. Companies focus on brand building to favorably and robustly embed their brand in the consumer's long-term memory through this associative network.

\subsection{Application and Case Analysis}

\subsubsection{Brand Strategy and Long-Term Memory: Utilizing Episodic Memory}
Companies aim to establish their brand not just in short-term memory (recalled only immediately after seeing an ad), but as an enduring long-term memory. Particularly effective is linking the brand to \textbf{episodic memory}—memories of an individual's specific experiences and events.
\begin{itemize}
	\item \textbf{Case: Chicken Ramen} \\
	      In advertisements for the long-selling brand Chicken Ramen, expressions that appeal to personal episodes and nostalgia, such as 'I ate this with my family as a child' or 'It was my late-night snack while studying for exams,' are frequently used. This is a strategy to foster brand attachment (loyalty) in long-term memory by associating the brand with the consumer's powerful, positive personal 'good memories.'
\end{itemize}

\subsubsection{Degree of Information Search and Marketing Strategy}
How actively consumers engage in information search at the time of purchase varies greatly by product category.
\begin{itemize}
	\item \textbf{Case 1: Low-Involvement Products (e.g., Meiji Milk Chocolate)} \\
	      For inexpensive convenience goods (low-involvement products) like chocolate, the risk of failure is low, so consumers do not conduct active \textbf{external searches}. Purchases are often decided based solely on 'internal search' (what I always buy) or short-term stimuli at the point of sale (package design or POP displays).
	      \textbf{Marketing Implication}: In this case, it is crucial for the company to frequently stimulate the consumer's \textbf{internal memory} (short-term memory) through repetitive advertising, making the brand easy to recall (brand retrieval) at the point of purchase, or to form \textbf{brand preference} by appealing to emotional aspects (e.g., looks delicious, seems fun).

	\item \textbf{Case 2: High-Involvement Products (e.g., PCs, Automobiles)} \\
	      For high-priced specialty goods or shopping goods (high-involvement products) like PCs and automobiles, the risk of failure is high, so consumers actively conduct \textbf{external searches}.
	      \textbf{Marketing Implication}: In this case, it is essential for the company to provide comprehensive comparative information and expert advice at every touchpoint of the consumer's external search (websites, brochures, dealerships). \textbf{Channel strategy}, such as the knowledge of salespeople and the clarity of website information, has a critical impact on the purchase decision.
\end{itemize}

\subsection{Deeper Context and Lessons}

\textbf{\paragraph{Factors Determining the 'Amount' of Information Search}}
The extent to which consumers conduct external searches depends on the following factors.
\begin{itemize}
	\item \textbf{Need for Information (Benefits of Search)}: (1) The higher the product price, (2) the stronger the personal relevance (importance) of the product, and (3) the greater the perceived variability in product quality, the more information consumers will try to gather to avoid failure (high necessity).
	\item \textbf{Amount of Consumer Memory (Knowledge)}: Consumers with extensive knowledge about a product (experts) already possess sufficient internal information, reducing the need for external search. Conversely, complete novices may not know what to search for and may abandon the search.
	\item \textbf{Cost of Information Gathering (Costs of Search)}: The time, effort, and monetary costs associated with gathering information. Just as it was once difficult to compare taxi fares and services before riding, consumers will abandon the search if the gathering costs are too high. However, the spread of the \textbf{internet} has dramatically lowered these costs, greatly transforming consumer external search behavior for high-involvement products.
\end{itemize}

\textbf{\paragraph{Typology of Buying Behavior (3 Patterns)}}
Based on the degree of information search and the consumer's involvement and experience, buying behavior can be categorized into the following three types.
\begin{itemize}
	\item \textbf{Extensive Problem Solving}: Used when there is no purchasing experience and product involvement is very high (e.g., buying a home for the first time). As brands and selection criteria are unclear, extensive information gathering and careful evaluation (a long time) are spent.
	\item \textbf{Limited Problem Solving}: Used when there is some purchasing experience, but a clear brand preference has not been established. Based on known selection criteria, several options (including new products) are compared.
	\item \textbf{Routine Problem Solving} (Habitual Buying Behavior): Used when product involvement is low and purchases are repetitive (e.g., the aforementioned chocolate). A preference for a specific brand is established, little to no information search is conducted, and the same item is purchased out of habit.
\end{itemize}

\textbf{\subsubsection{AI Supplement: Extension of Key Issues (Alternative Evaluation Models)}}
While much time in this lecture was devoted to 'Information Search,' the specific process of the next stage, \textbf{'Alternative Evaluation,'} was omitted. In MBA studies, this evaluation process is extremely important as it directly links to marketing strategy (especially positioning).

Consumers evaluate the collected information based on the \textbf{evaluation criteria} they have set (e.g., for a PC: price, processing speed, battery life, design). Two main models exist for this evaluation method.

\begin{itemize}
	\item \textbf{Compensatory Model (e.g., Multi-Attribute Attitude Model)}:
	      A model assuming that a weakness in one evaluation criterion (e.g., high price) can be \textbf{compensated} for (offset) by a strength in another (e.g., excellent design). Consumers calculate a total score considering the importance of each attribute and choose the option with the highest score.
	\item \textbf{Non-Compensatory Model}:
	      A model assuming that a weakness in one criterion cannot be compensated for by others.
	      \begin{itemize}
		      \item \textbf{Lexicographic}: Compares all options on the most important evaluation criterion (e.g., price) and selects the one that performs best.
		      \item \textbf{Conjunctive}: Sets a minimum 'cut-off line' (e.g., price under 50,000 yen AND battery over 8 hours) for all criteria and selects the one that clears all of them.
	      \end{itemize}
	      There is a tendency for compensatory models to be used for high-involvement products, while non-compensatory models (especially simple cut-offs) are used for low-involvement products.
\end{itemize}

\subsection{Conclusion}
This lecture note framed consumer purchasing behavior within the information-processing framework of the \textbf{purchasing decision-making process}. It specifically organized the stages from 'problem recognition' to 'information search' and the underlying 'memory' mechanisms (internal search, long-term memory, associative networks).

The \textbf{practical lesson} from this lecture is that marketers must accurately grasp which category (extensive, limited, or routine) their product falls into within the consumer's purchasing process.
For high-involvement products, it is necessary to provide high-quality information that withstands comparative review at every \textbf{external search} touchpoint (web, store) and to gain an advantage on the evaluation criteria.
For low-involvement products, on the other hand, the goal is to establish a position as the 'brand of choice,' prompting consumers to skip the information search altogether. This is achieved by embedding the brand in the consumer's \textbf{internal memory} through repetitive advertising or by appealing to \textbf{episodic memory}. Brand building is, in essence, the activity of designing the associative network in the consumer's long-term memory.

\subsection{Key Terms List}
Names:
(None)

\vspace{\baselineskip}
Universal Concepts:
Purchasing Decision-Making Process, Problem Recognition, Information Search, Internal Search, External Search, Alternative Evaluation, Post-Purchase Evaluation, Information-Processing Model, Sensory Memory, Short-Term Memory, Long-Term Memory, Associative Network Model, Brand Image, Episodic Memory, Cost of Information Gathering, Extensive Problem Solving, Limited Problem Solving, Routine Problem Solving

\subsection{Comprehension Quiz}
\begin{enumerate}
	\item What is the first stage of the purchasing decision-making process, which begins with recognizing a gap between an ideal state and an actual state?
	\item What is the second stage of the purchasing decision-making process, which is divided into 'internal search' (searching one's own memory) and 'external search' (searching for outside information)?
	\item What is the final stage of the purchasing decision-making process, where one evaluates whether their purchase choice was correct, leading to satisfaction or dissatisfaction (regret)?
	\item What is the memory store where information from sensory organs, if attended to, is held for a few seconds to several tens of seconds?
	\item What is the memory store where information from short-term memory is stored permanently through rehearsal (repetition) or encoding (giving meaning)?
	\item What is the memory model called that posits concepts (nodes) in memory are connected by their mutual relationships (links)?
	\item What model explains the phenomenon where hearing a specific brand name (e.g., Coca-Cola) brings related images (e.g., America, red, bottle) to mind one after another?
	\item As seen in Chicken Ramen commercials appealing to childhood 'nostalgia,' what is the specific term for memories related to personal experiences and events?
	\item Consumers tend to search for more information as the product price gets higher and the variability in quality increases. This is due to what factor (benefit) in search behavior?
	\item The spread of the internet has dramatically reduced the time and effort consumers spend gathering information. What is this type of cost generally called?
	\item What is the decision-making process called for products like Meiji Milk Chocolate, where consumers conduct almost no information search and purchase habitually?
	\item What is the decision-making process called for products like cars or homes, where consumers have no prior purchase experience and conduct extensive information gathering and careful evaluation?
	\item What are the standards (e.g., price, design, performance) that consumers use when evaluating the information they have gathered called?
	\item What is the evaluation model called where a weakness in one attribute (e.g., high price) is offset by a strength in another (e.g., good performance)?
	\item What is the evaluation model called where options that fail to meet a minimum standard (e.g., 'price must be under 10,000 yen') are immediately eliminated?
\end{enumerate}

\subsubsection*{Answer Key}
1. Problem Recognition, 2. Information Search, 3. Post-Purchase Evaluation, 4. Short-Term Memory, 5. Long-Term Memory, 6. Associative Network Model, 7. Associative Network Model, 8. Episodic Memory, 9. Need for information (Benefit of search), 10. Cost of information gathering (Cost of search), 11. Routine Problem Solving (Habitual Buying Behavior), 12. Extensive Problem Solving, 13. Evaluation Criteria, 14. Compensatory Model, 15. Non-compensatory Model


\section{Evaluation and Choice of Alternatives}

\subsection{Introduction}
This lecture note focuses on the latter half of the purchasing decision-making process: 'Evaluation and Choice of Alternatives' and 'Post-Purchase Evaluation,' which occur after the consumer has recognized a problem and searched for information. The objective is to understand the theoretical models explaining how consumers perceive products as a '\textbf{bundle of attributes}' and how they integrate this multi-attribute information to select a single product.

\subsection{Key Concepts and Issues}
The process by which consumers integrate product attribute information to make a final evaluation is broadly divided into two models, which differ in their information-processing load: \textbf{'Compensatory Models'} and \textbf{'Non-compensatory Models'}.

\subsubsection{Compensatory Model}
Compensatory models assume that consumers are rational, like computers, and that they evaluate all attributes of the target products and select the optimum one based on a comprehensive evaluation score.
\begin{itemize}
	\item \textbf{Characteristic:} A low evaluation (weakness) on one attribute is \textbf{compensated} for by a high evaluation (strength) on another attribute.
	\item \textbf{Representative Model: Fishbein Model}
	      \begin{itemize}
		      \item It posits that an overall evaluation is derived by multiplying the \textbf{'importance'} of each attribute by the \textbf{'belief'} (subjective evaluation) of the extent to which the product possesses that attribute, and then summing these products for all attributes.
		      \item Conceptual Formula: Overall Evaluation = $\sum (\text{Importance of attribute }i \times \text{Belief rating for attribute }i)$
		      \item The consumer is assumed to select the brand with the highest overall evaluation score.
	      \end{itemize}
\end{itemize}

\subsubsection{Non-compensatory Model}
Non-compensatory models are based on the premise of \textbf{limited rationality}—that consumers have limited information-processing capacity—and assume they use \textbf{heuristics} (shortcuts) to simplify the evaluation.
\begin{itemize}
	\item \textbf{Characteristic:} Not all attributes are considered. If an attribute fails to meet a standard, that weakness is not compensated for by other attributes, and the option is eliminated.
	\item \textbf{Various Models:}
	      \begin{itemize}
		      \item \textbf{Affect-Referral Model:} Habitually selecting the most-liked brand based on feelings derived from past purchase or usage experience.
		      \item \textbf{Conjunctive Model:} Sets a 'minimum acceptance level (cut-off line)' for each attribute. Products that fall below this standard on even one attribute are eliminated from consideration.
		      \item \textbf{Disjunctive Model:} Sets a 'sufficient condition' for each attribute. If a product meets (at a very high level) this standard on any single attribute, it is selected.
		      \item \textbf{Lexicographic Model:} All options are compared on the most important attribute. If a decision can be made there (e.g., the cheapest one), the choice is finalized. If there is a tie, the process is repeated with the second most important attribute.
	      \end{itemize}
\end{itemize}

\subsubsection{Model Selection}
Consumers switch between these models depending on the situation. \textbf{Compensatory} models, which allow for detailed comparison, are more likely to be used when there are few options. Conversely, \textbf{non-compensatory} models are more likely to be used when there are many options, to reduce the information-processing load.

\subsection{Application and Case Analysis}
The lecture provided a specific example of the Fishbein model (compensatory) for the purchase of two car brands (A and B).

\begin{itemize}
	\item \textbf{Evaluation Criteria (Attributes):} Price, Fuel Efficiency
	\item \textbf{Consumer's Settings:}
	      \begin{itemize}
		      \item Attribute Importance: Price(3), Fuel Efficiency(1)
	      \end{itemize}
	\item \textbf{Beliefs about each brand (Evaluation Score):}
	      \begin{itemize}
		      \item Brand A: Price(3), Fuel Efficiency(-1)
		      \item Brand B: Price(1), Fuel Efficiency(1)
	      \end{itemize}
	\item \textbf{Calculation of Overall Evaluation:}
	      \begin{itemize}
		      \item \textbf{Brand A:} (Price Importance 3 $\times$ Price Rating 3) + (Fuel Eff. Importance 1 $\times$ Fuel Eff. Rating -1) = 9 - 1 = \textbf{8 points}
		      \item \textbf{Brand B:} (Price Importance 3 $\times$ Price Rating 1) + (Fuel Eff. Importance 1 $\times$ Fuel Eff. Rating 1) = 3 + 1 = \textbf{4 points}
	      \end{itemize}
	\item \textbf{Conclusion:} For this consumer (with these evaluation criteria), Brand A, having the higher overall evaluation, will be selected.
\end{itemize}

\subsection{Deeper Context and Lessons}

\textbf{\paragraph{The Reality of Planned vs. Unplanned Purchases}}
Consumer buying behavior does not always follow a rational information integration process as described above. According to one study, 'planned purchases' (planned in advance) account for only about 11\% of actual buying behavior. Rather, many purchases are \textbf{'unplanned purchases,'} where the final decision is made inside the store due to \textbf{in-store promotions} such as explanations from staff, product sampling, or POP (point-of-purchase) displays.

\textbf{\paragraph{The Post-Purchase Evaluation Process}}
A purchase is not an end; it is an input for the next decision. Consumers conduct a 'post-purchase evaluation' after buying and using a product.
\begin{itemize}
	\item \textbf{Determinants of Satisfaction:} Satisfaction or dissatisfaction is determined by whether the value provided by the company exceeds the consumer's \textbf{'expectation level.'}
	\item \textbf{If Satisfied:} If the outcome meets or exceeds expectations, the consumer is 'satisfied.' This leads to a \textbf{simplification of heuristics}. That is, the consumer comes to think, 'Buying this brand next time will be fine,' and the future decision-making process is simplified (= repeat purchase).
	\item \textbf{If Dissatisfied:} If the outcome falls short of expectations, 'dissatisfaction' occurs. This leads to an \textbf{elaboration of heuristics}. The consumer identifies the elements of dissatisfaction, asking 'Why did this fail?' and their motivation to actively gather information on other alternatives for the next time increases.
\end{itemize}

\textbf{\paragraph{The Impact of Dissatisfaction and Cognitive Dissonance}}
Dissatisfaction creates negative emotions such as anger, regret, and resignation, leading to the \textbf{cessation of repeat purchases}—an undesirable outcome for the company.
\begin{itemize}
	\item \textbf{Impact of Word-of-Mouth:} In recent years, dissatisfied experiences are easily shared via the internet. It is generally said that \textbf{'negative word-of-mouth'} has more than twice the diffusion power of 'positive word-of-mouth,' causing significant damage to companies.
	\item \textbf{Cognitive Dissonance:} On the other hand, a unique state called 'cognitive dissonance' also occurs in the consumer's mind post-purchase. This is the anxiety or conflict of 'Was my choice really correct?' To resolve this dissonance (i.e., to justify their choice), consumers may actively seek additional information that supports their choice or try to (internally) console themselves for the failure of their choice. This also represents an opportunity for the company to reduce dissonance and increase loyalty through post-purchase communication (e.g., thank-you emails, usage support).
\end{itemize}

\textbf{\subsubsection{AI Supplement: Extension of Key Issues}}
This lecture introduced compensatory and non-compensatory models, but it lacked mention of the level of \textbf{'Involvement,'} which is a crucial prerequisite determining how consumers (intentionally or unconsciously) select between these information-processing models.
\begin{itemize}
	\item \textbf{What is Involvement:} Refers to the degree of personal importance, interest, or perceived risk an individual associates with a product or purchase situation.
	\item \textbf{High Involvement and Information Processing:} For \textbf{high-involvement products} (high risk of purchase failure, high price, high self-expression) like cars or homes, consumers are highly motivated to process information and actively search for it. In such cases, they tend to use the \textbf{compensatory models} (like the Fishbein model) discussed in the lecture, engaging in detailed attribute comparisons.
	\item \textbf{Low Involvement and Information Processing:} For \textbf{low-involvement products} (low risk of failure) like daily necessities or snacks, consumers try to minimize their information-processing efforts. Thus, they rely on \textbf{non-compensatory models} (especially habitual choice or lexicographic) or peripheral cues (heuristics) such as price, brand familiarity, or package design.
	\item \textbf{Elaboration Likelihood Model (ELM):} The theory that this level of involvement determines the depth of information processing is the \textbf{'Elaboration Likelihood Model (ELM)'}. It holds that when involvement is high, logical information processing (\textbf{central route}) influences attitude change, while when involvement is low, heuristics and emotional cues (\textbf{peripheral route}) are more influential. The choice of alternative evaluation models in this lecture can be complementarily understood within this ELM framework.
\end{itemize}

\subsection{Conclusion}
The consumer's alternative evaluation process is not always rational. They switch between a \textbf{'compensatory'} approach, which compares all attributes like a computer, and a \textbf{'non-compensatory'} approach, which uses shortcuts to reduce the processing load, depending on the situation (number of choices, degree of involvement).
The practical lesson from this lecture lies in the importance of the post-purchase evaluation process. \textbf{'Dissatisfaction,'} in particular, can cause enormous damage through negative word-of-mouth, making expectation level management and swift responses to dissatisfaction essential. Simultaneously, understanding the \textbf{'cognitive dissonance'} consumers feel post-purchase and providing communication to reduce it (supporting the justification of their choice) is extremely important for building customer satisfaction and loyalty.

\subsection{Key Terms List}
Fishbein

\vspace{\baselineskip}
Purchasing Decision-Making Process, Attributes, Compensatory Model, Non-compensatory Model, Fishbein Model, Conjunctive Model, Disjunctive Model, Lexicographic Model, Heuristics, Unplanned Purchase, Expectation Level, Cognitive Dissonance, Involvement, Elaboration Likelihood Model

\subsection{Comprehension Quiz}
\begin{enumerate}
	\item What is the model called where a weakness (low evaluation) in one product attribute is offset by a strength (high evaluation) in another attribute for an overall evaluation?
	\item Is the Fishbein model, introduced in the lecture, classified as a compensatory or non-compensatory model?
	\item What is the model called that sets a cut-off line for 'minimum standards that must be met' for each attribute, eliminating any option that falls short on even one?
	\item What is the evaluation model called where 'all options are first compared on the most important attribute, and if no difference is found, they are compared on the second most important attribute'?
	\item What is the evaluation model called where 'if a product has an outstanding evaluation score (meets a sufficient condition) on any single attribute, that product is chosen'?
	\item When a consumer faces a large number of options, which model (compensatory or non-compensatory) is more likely to be adopted to reduce the information-processing load?
	\item What is the term for the 'simple solutions or rules' based on rules of thumb that people use when making decisions?
	\item What is the term for the psychological anxiety or conflict a consumer feels after a purchase, such as 'Was this choice really the right one?'
	\item A consumer's post-purchase satisfaction is determined by a comparison between their pre-purchase 'what' and the actual 'performance (outcome)'?
	\item What is the term for behavior where the purchase decision is triggered by in-store stimuli, such as POP advertising, product sampling, or recommendations from staff?
	\item What emotion does a consumer feel when the actual performance falls short of their expectation level?
	\item In the lecture, what was the term for the simplification of the decision-making process, where a satisfied consumer chooses the same brand next time without much thought?
	\item What is the term for the general tendency for dissatisfied experiences to be communicated more widely to others as word-of-mouth than satisfied experiences (in the lecture, 'what kind of word-of-mouth' was said to be 'more than twice' as powerful as 'what kind')?
	\item (From AI supplement) For products like cars or homes, where the risk of purchase failure is high and consumers are motivated to search for information, this is expressed as 'a high level of what'?
	\item (From AI supplement) In the Elaboration Likelihood Model (ELM), what is the pathway called where high-involvement consumers logically process product attribute information?
\end{enumerate}

\subsubsection*{Answer Key}
1. Compensatory Model, 2. Compensatory Model, 3. Conjunctive Model, 4. Lexicographic Model, 5. Disjunctive Model, 6. Non-compensatory Model, 7. Heuristics, 8. Cognitive Dissonance, 9. Expectation Level (or Expectation), 10. Unplanned Purchase (or In-store Merchandising), 11. Dissatisfaction, 12. Simplification of heuristics, 13. Negative word-of-mouth (more than twice as much as positive word-of-mouth), 14. (High level of) Involvement, 15. Central Route

\section{Typologies of Buying Behavior Situations}

\subsection{Introduction}
This lecture note aims to understand that consumer purchasing behavior does not always follow the same process, but rather that its patterns differ depending on the situation. Specifically, it explores how consumer \textbf{'involvement'} with a product or purchase affects the depth of information processing and attitude formation, learning through the \textbf{'Elaboration Likelihood Model (ELM)'}. Furthermore, it aims to understand \textbf{'Assael's Model,'} which classifies buying behavior into four typologies using the two axes of involvement level and \textbf{'perceived differences between brands.'}

\subsection{Key Concepts and Issues}
The purchasing decision-making process is not fixed; the emphasis of each stage and the depth of information processing fluctuate depending on the importance of and interest in the product.

\subsubsection{Involvement}
\textbf{Involvement} refers to the degree of \textbf{importance} or (failure) \textbf{risk} that a consumer feels regarding a product category or purchasing situation. It is also generally expressed as 'personal relevance.' This level of involvement significantly influences the method of information processing.

\subsubsection{Elaboration Likelihood Model (ELM)}
The ELM is a theory that explains the influence of persuasive communication (such as advertising) on consumer attitude change, positing that the information processing route branches depending on the level of involvement.
\begin{itemize}
	\item \textbf{Central Route}:
	      \begin{itemize}
		      \item \textbf{Condition:} When \textbf{product involvement is high} (high motivation) AND the ability (knowledge) to process the information is also high.
		      \item \textbf{Processing:} The consumer \textbf{elaborates} (deeply scrutinizes) the message (information) content, evaluates it logically, and ranks it.
		      \item \textbf{Result:} Attitudes formed via this route are robust, persistent, and likely to be linked to actual purchasing behavior.
	      \end{itemize}
	\item \textbf{Peripheral Route}:
	      \begin{itemize}
		      \item \textbf{Condition:} When \textbf{product involvement is low} (low motivation) OR the ability to process the information is low.
		      \item \textbf{Processing:} The consumer does not focus on the essential content of the message, but instead decides their attitude based on \textbf{peripheral cues} such as the celebrity endorser (talent image), emotions, or atmosphere.
		      \item \textbf{Result:} Attitudes formed via this route are temporary, unstable, and easily changed by other information.
	      \end{itemize}
\end{itemize}

\subsubsection{Assael's Four Types of Buying Behavior}
The marketing scholar Henry Assael mapped consumer buying behavior on two axes—\textbf{'degree of involvement'} (high/low) and \textbf{'perceived differences between brands'} (large/small)—classifying it into four types.

\begin{enumerate}
	\item \textbf{Complex Buying Behavior}: High Involvement \& Significant Differences between Brands
	      \begin{itemize}
		      \item \textbf{Target:} Expensive products, infrequently purchased products, products with high self-expression (high personal relevance).
		      \item \textbf{Behavior:} The consumer engages in intensive and comprehensive information processing. They first form \textbf{'beliefs'} about each brand in the product category (through evaluation), then determine an \textbf{'attitude,'} and finally proceed to \textbf{'purchase.'} This follows the most cognitive and step-by-step process.
	      \end{itemize}

	\item \textbf{Dissonance-Reducing Buying Behavior}: High Involvement \& Few Differences between Brands
	      \begin{itemize}
		      \item \textbf{Characteristic:} Involvement is high (they do not want to fail), but they do not perceive significant differences between brands.
		      \item \textbf{Behavior:} Having gathered sufficient information beforehand, the purchase itself tends to be done quickly. What is important is 'after' the purchase; efforts are made to gather additional external information to resolve \textbf{post-purchase dissonance (anxiety)}—'Was this choice really correct?'
	      \end{itemize}

	\item \textbf{Variety-Seeking Buying Behavior}: Low Involvement \& Significant Differences between Brands
	      \begin{itemize}
		      \item \textbf{Characteristic:} Involvement is low, but differences between brands are recognized.
		      \item \textbf{Behavior:} Although not dissatisfied with a specific brand, frequent \textbf{brand switching} (changing) occurs, motivated by a desire for \textbf{novelty} or \textbf{variety}.
	      \end{itemize}

	\item \textbf{Habitual Buying Behavior}: Low Involvement \& Few Differences between Brands
	      \begin{itemize}
		      \item \textbf{Characteristic:} Involvement is low, and the consumer believes that all options are more or less the same, so it does not matter much what they buy.
		            D     \item \textbf{Behavior:} The same item is purchased habitually, not based on specific beliefs or attitudes. While this behavior may look like high loyalty to the company, it is likely just \textbf{'apparent loyalty'} (pseudo-loyalty) and not active selection.
	      \end{itemize}
\end{enumerate}

\subsection{Application and Case Analysis}

\subsubsection{Variety-Seeking: Canned Coffee}
In the lecture, \textbf{canned coffee} was cited as a typical example of variety-seeking behavior. In the Japanese market, various manufacturers sell a wide variety of products. Even if a consumer (the lecturer's own example) prefers a specific brand (e.g., Georgia), they may feel 'I'll just try it' when they see a new product or another manufacturer's product. This is based not on dissatisfaction, but on a psychological desire for novelty.

\subsubsection{Habitual Buying Behavior: Toilet Paper}
\textbf{Toilet paper} and tissue paper were given as examples of \textbf{habitual buying behavior} (although, of course, some consumers do have strong preferences). For many consumers, these are low-involvement items, and perceived differences between brands are small, so they tend to select the usual brand without much thought.

\subsection{Deeper Context and Lessons}

\textbf{\paragraph{Corporate Response to Variety Seeking}}
In response to consumer variety-seeking behavior, companies adopt a strategy of adding \textbf{'brand freshness'} as a product attribute. Specifically, this involves launching new or limited-edition products every season. This allows them to capture the brand switching of consumers seeking novelty and \textbf{temporarily increase market share}. However, this strategy is a double-edged sword. Users of the company's own existing brands may also switch to competitors' new products. Therefore, while gaining share with new products, companies must also strive to increase the \textbf{brand loyalty} of their existing brands to prevent customer churn.

\textbf{\paragraph{Convenience Stores and 'Vague Consumption'}}
Variety-seeking behavior, prevalent among the younger generation, is also related to convenience store usage behavior. Some consumers stop by convenience stores just to see new products, even without a clear errand. The reason companies continue to launch new products seasonally is partly to meet this consumer \textbf{'need for newness'} and to satisfy the enjoyment of shopping.

\textbf{\paragraph{The Trap of 'Apparent Loyalty'}}
Habitual buying behavior holds an important implication for companies. When consumers 'always buy the same thing from the same manufacturer,' companies tend to interpret this as 'a customer with high loyalty.' However, according to Assael's classification, this may be mere \textbf{'inertia'} stemming from low involvement and low perceived differentiation, and thus only \textbf{'apparent loyalty'} (Pseudo-Loyalty), not active support. There is a risk that such customers will easily switch if a competitor offers a cheaper alternative or an attractive promotion.

\textbf{\subsubsection{AI Supplement: Extension of Key Issues}}
While this lecture organized buying behavior on the two axes of involvement and brand differentiation, in practice, a deeper understanding of the 'nature' of these two axes is required.
\begin{itemize}
	\item \textbf{The Duality of Involvement (Situational vs. Enduring Involvement):}
	      The lecture treated 'involvement' as a single scale, but there are two types. One is \textbf{'situational involvement,'} which is low normally but temporarily increases when a purchase becomes necessary (e.g., choosing furniture for a move). The other is \textbf{'enduring involvement,'} which is tied to personal hobbies or values, leading to constant interest in the product category (e.g., a dedicated audiophile). Those who follow the ELM's central route are mainly consumers with high enduring involvement, or those with situational involvement who perceive the risk of failure as extremely high.

	\item \textbf{'Perceived Differences Between Brands' Can Be Manipulated:}
	      Markets classified as having 'few differences between brands' in Assael's model (e.g., dissonance-reducing, habitual) are not fixed conditions for a company. The essence of marketing activity is precisely to create these \textbf{'perceived differences.'} If a company can establish clear differentiation (in function, design, meaning, or story) from competitors through \textbf{branding} and innovation, and shift consumer perception from 'few differences' to 'significant differences,' it becomes possible to escape habitual buying (price competition) and transition to a complex buying (high value-added) market.
\end{itemize}

\subsection{Conclusion}
Consumer buying behavior is not uniform. The depth of information processing (ELM) and the behavioral pattern (Assael's 4 types) are determined by the level of \textbf{'involvement'} with the product and situation, and the \textbf{'perceived differences between brands'} in the market.
The practical lesson from this lecture is the importance of accurately identifying 'which type' of behavior consumers are engaging in when purchasing one's own products. In particular, mistaking \textbf{habitual buying} for 'high loyalty' can lead to critical strategic errors for a company. In markets where variety seeking is mainstream, a balanced sense of maintaining loyalty and pursuing novelty is key to marketing strategy.

\subsection{Key Terms List}
Assael (Henry Assael)

\vspace{\baselineskip}
Involvement, Elaboration Likelihood Model (ELM), Central Route, Peripheral Route, Perceived Differences Between Brands, Complex Buying Behavior, Dissonance-Reducing Buying Behavior, Variety-Seeking, Brand Switching, Habitual Buying Behavior, Apparent Loyalty, Brand Loyalty

\subsection{Comprehension Quiz}
\begin{enumerate}[label=\arabic*.]
	\item What is the term for the degree of importance or risk perception a consumer feels towards a product category?
	\item What is the attitude change model called that explains how a consumer's information processing differs when involvement is high versus low?
	\item In the Elaboration Likelihood Model (ELM), what is the information processing route called where involvement is high, and the information content is deeply scrutinized and evaluated?
	\item In the ELM, what is the information processing route called where involvement is low, and attitudes are decided based on cues like a celebrity endorser's image or the atmosphere?
	\item Compared to attitudes formed via the central route, what is the characteristic of attitudes formed via the peripheral route?
	\item What is the last name of the scholar who classified buying behavior into four types based on the two axes of 'involvement' and 'perceived differences between brands'?
	\item What are the two classification axes used in Assael's 4-type model?
	\item What is the buying behavior called when involvement is high and perceived differences between brands are large (e.g., buying an expensive car)?
	\item What is the buying behavior called when involvement is high, but perceived differences between brands are small (e.g., expensive but hard-to-differentiate carpet)?
	\item What is the buying behavior called when involvement is low, but perceived differences between brands are large (e.g., canned coffee)?
	\item What is the buying behavior called when involvement is low, and perceived differences between brands are also small (e.g., toilet paper)?
	\item In variety-seeking buying behavior, what is the term for consumers frequently changing their purchase brand to seek novelty?
	\item Is variety-seeking buying behavior caused by 'dissatisfaction' with the product, or by another factor?
	\item In habitual buying behavior, what is the term for the state that looks like high loyalty at a glance, but is not actually active support?
	\item To prevent variety-seeking consumers from defecting to competitors' products, what measure should companies take regarding their existing brands?
\end{enumerate}

\subsubsection*{Answer Key}
1. Involvement, 2. Elaboration Likelihood Model (ELM), 3. Central Route, 4. Peripheral Route, 5. Its (the attitude's) persistence is low and it is changeable (or, less likely to lead to persuasion), 6. Assael, 7. Involvement (degree of), Perceived differences between brands, 8. Complex Buying Behavior, 9. Dissonance-Reducing Buying Behavior, 10. Variety-Seeking Buying Behavior, 11. Habitual Buying Behavior, 12. Brand Switching, 13. It is due to a desire for novelty or variety (not dissatisfaction), 14. Apparent loyalty, 15. (Increasing) Brand Loyalty

\end{document}
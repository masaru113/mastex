\documentclass[uplatex,a4j,12pt,dvipdfmx]{jsarticle}
\usepackage{amsmath,amsthm,amssymb,bm,color,enumitem,mathrsfs,url,epic,eepic,ascmac,ulem,here,ascmac}
\usepackage[letterpaper,top=2cm,bottom=2cm,left=3cm,right=3cm,marginparwidth=1.75cm]{geometry}
\usepackage[english]{babel}
\usepackage[dvipdfm]{graphicx}
\usepackage[hypertex]{hyperref}
\title{Operations Management Lecture 0 Course Notes\newline (Supplement: FinTech and Operations Management)}
\author{Masaru Okada}
\date{\today}
\begin{document}
\maketitle
\tableofcontents
\section{FinTech and Operations Management}
\subsection{Introduction}
This lecture is the 0th session, 'FinTech and Operations Management,' of the course 'Operations Management as a Foundation for FinTech.' The objective of this paper is to conduct an integrated analysis of FinTech not as a discrete field of traditional finance theory or technology theory, but from the perspective of \textbf{Technology \& Operations} (T\&O) or \textbf{Management of Technology} (MOT).
The lecture begins by introducing a comment from Kazuhiro Ajima, then-President of Resona Holdings, who stated on NHK's 'Close-up Gendai' in 2016: 'We may have to discard the very business model of being a bank.' This symbolizes the vague yet urgent concern among existing financial institutions that the \textbf{redefinition of industries} and disruption of existing businesses, as caused by the internet in other sectors, might also occur in the financial industry. This report aims to specifically unravel this 'vague concern' from the perspective of a transformation in operations management.
\subsection{Key Concepts and Points}
In this lecture, to understand the transformation brought by FinTech, conventional operations and new operations were explained in contrast.
\subsubsection{Conventional Financial Operations: Cream Skimming}
The operations management of the conventional financial industry is characterized as a \textbf{Cream Skimming Operation}. This is a strategy of concentrating on the most efficient and 'tasty' parts of the business (high-profit customer segments or products), much like eating only the sweetest part of a watermelon.
The theoretical background for this strategy is the \textbf{Pareto Principle} (1897), proposed by the Italian economist \textbf{Vilfredo Pareto}. Also known as the '80/20 rule,' this is a law of imbalance stating that 80\% of outcomes are generated by 20\% of causes, as typified by the empirical rule that '80\% of a society's total income is held by 20\% of high-income earners.' In the financial industry, it has been premised that the top 20\% of customers generate 80\% of the profit; therefore, cream skimming, which concentrates resources on this 20\%, has been considered the most efficient operation.
\subsubsection{FinTech-Era Operations: Bowling Pin Operation}
However, the evolution of the internet and technology is beginning to overturn this premise. What is required in the FinTech era is the \textbf{Bowling Pin Operation}. This refers to a strategy of precisely knocking down a single target, the No. 1 pin in bowling, thereby using the ripple effect to knock down all 10 pins (= taking the entire market).
This operational transformation is being triggered by the rise of \textbf{platform businesses}.
\subsubsection{Theories Underpinning Platform Businesses}
\paragraph{\textbf{Network Effect}}
The greatest characteristic of a platform business is that the \textbf{network effect} is at play. The network effect refers to the phenomenon where, as the number of users of a product or service increases, the utility or value that each user derives from that service also increases. When this effect is active, a specific product or service grows rapidly, making it easy for a \textbf{winner-take-all} situation to emerge in the market.
\paragraph{\textbf{The Long Tail Theory}}
The revenue structure of the Bowling Pin Operation is explained by the \textbf{Long Tail} theory, proposed by \textbf{Chris Anderson}.
In traditional brick-and-mortar business, the \textbf{Pareto Principle} was strong, with 80\% of sales generated from the top 20\% of inventory—the best-selling products (the 'head'). These products have a high \textbf{inventory turnover} and good \textbf{capital efficiency}.
In internet business, conversely, there are no constraints of physical store shelves, so the 80\% of niche products (the 'tail') that were previously ignored also become targets for sale. The core of Chris Anderson's discovery was not merely the existence of this 'tail,' but his demonstration of a structure where the profit generated from this niche product group, previously thought to be unprofitable (90\% of all products in the text's example), could rival (at 33\%) the profit generated by conventional mainstay products (2\% of products, generating 33\% of profit).
\subsubsection{The Inevitability of Operational Transformation}
Internet businesses, including FinTech, are premised on a structure that earns profit from the entirety of the \textbf{Long Tail}—in other words, the \textbf{Bowling Pin Operation}. This signifies a fundamental shift away from the conventional \textbf{Cream Skimming} model. Companies unable to realize this new operation are likely to be eliminated as 'those who cannot take all' of the market, and the financial industry is no exception.
\subsection{Application and Case Analysis}
\subsubsection{Case: Resona Holdings}
The statement 'discard the bank business model' by \textbf{Kazuhiro Ajima}, President of Resona Holdings, introduced at the beginning of the lecture, indicates a sense of crisis that the existing \textbf{Cream Skimming} operation will be fundamentally destroyed by the new operations of FinTech companies, when viewed in light of this lecture's points. The true nature of this 'vague anxiety' lies in the difficulty of adapting to the \textbf{Bowling Pin Operation}.
\subsubsection{Case: LINE}
As a typical example of the \textbf{network effect}, \textbf{LINE} was raised in the lecture. LINE's value depends heavily not just on the quality of the service itself, such as the app's functionality or the appeal of its 'stickers,' but on the scale of the network—the fact that 'one's friends are members and one can communicate with them.' Once the user count surpassed a critical mass, its value increased explosively (an acceleration of the growth speed), resulting in it achieving a \textbf{winner-take-all} position in the communication platform market.
\subsection{Deeper Context and Lessons}
In addition to the main thesis of the lecture, peripheral information to deepen contextual understanding and the instructor's personal views were mentioned.
\textbf{\paragraph{Digression Topic Title: Shigeo Nagashima's Watermelon Episode}}
As a metaphor for \textbf{Cream Skimming}, an anecdote was introduced about \textbf{Shigeo Nagashima} eating the sweetest part of the watermelon first at a training camp. This overlays the image of a star player concentrating on the 'tasty part' (home runs) with the financial industry's concentration on its most efficient business.
\textbf{\paragraph{Digression Topic Title: The Universality of the Pareto Principle}}
It was added that the \textbf{Pareto Principle} is a universal law observed not only in human economic activity (income, sales, taxation) but also in insect societies like ants and bees (where 20\% of ants gather 80\% of the food).
\textbf{\paragraph{Digression Topic Title: Analogies to the Pareto Principle}}
As analogies to the Pareto Principle, the \textbf{Principle of Least Effort} ('resources adjust themselves to minimize labor') and \textbf{Juran's Principle} (in quality improvement, a few factors have a decisive effect on the whole) were introduced.
\textbf{\paragraph{Digression Topic Title: The Instructor's Formative Experience (FinTech Japan 2016)}}
The instructor, Mr. Nomaguchi, left the financial industry over 20 years ago and moved into the consulting industry. FinTech Japan 2016 was a 'reunion' with the financial industry for the instructor (a lecture by \textbf{Yoshitaka Kitao}, CEO of SBI Holdings). The instructor noted that while he intuited FinTech had a high affinity with his own consulting domain, he also felt a strong 'anxiety,' along with hope, about whether the financial industry could break free from its old constitution (in the instructor's words, the 'goso sendan hoshiki' or 'convoy system') and evolve amidst global competition.
\textbf{\subsubsection{AI Supplement: Expansion of Key Points}}
The lecture presented the 'Bowling Pin Operation' and 'platform business' as operations for the FinTech era. However, this chapter lacked references to the lecture's subtitle, 'FinTech as Technology Architecture'—that is, how to technically achieve these operations.
The core technological architecture that enables the FinTech \textbf{Bowling Pin Operation} is the \textbf{API (Application Programming Interface)}.
Conventional financial institutions had a 'vertically integrated,' closed architecture, handling everything from system development to operation and service provision in-house.
In contrast, FinTech—particularly the model known as \textbf{BaaS (Banking as a Service)}—adopts a 'horizontally specialized' architecture where banks modularize their own financial functions (payments, account management, credit, etc.) as \textbf{APIs} and provide them as services to third-party companies (FinTech firms or non-financial enterprises).
This enables the bank itself to become a platformer (function provider), allowing diverse external companies to develop \textbf{Long Tail} niche services on top of it. This architectural transformation makes conventional \textbf{Cream Skimming} (doing only the 'tasty' parts in-house) structurally difficult, forcing a shift to a \textbf{Bowling Pin Operation} that earns revenue from the entire diverse service group created by API partners. 'FinTech as Technology Architecture' is thought to refer precisely to this modularization and recombination of systems via APIs.
\subsection{Conclusion}
This lecture (Session 0) presented the importance of re-evaluating FinTech not merely as a financial product or IT technique, but from the perspective of \textbf{Technology \& Operations} (or MOT).
The traditional financial industry was optimized for \textbf{Cream Skimming} operations based on the \textbf{Pareto Principle}. However, due to the rise of the internet and platform businesses (driven by the \textbf{network effect} and \textbf{Long Tail} theory), a shift to the \textbf{Bowling Pin Operation} (a 'take-all' model) targeting the entire market has become unavoidable.
The instructor's formative experience mentioned in the 'Deeper Context'—the 'anxiety' felt at FinTech Japan 2016—was not simply a concern about industry realignment. It is inferred to have been a concern about a structural and technical challenge: whether the old constitution ('convoy system') of Japanese financial institutions and their closed, 'vertically integrated' system architectures would impede adaptation to the 'horizontally specialized' \textbf{Bowling Pin Operation}, such as the \textbf{API economy} and \textbf{BaaS} discussed in the AI supplement.
In future studies, in addition to financial theory, it is extremely important to maintain an \textbf{MOT} perspective on how the \textbf{technology architecture} (APIs, modularization) that supports FinTech transforms operations and rebuilds competitive advantage.
\subsection{Key Keywords List}
\textbf{Names:} Kazuhiro Ajima (then-President, Resona Holdings), Vilfredo Pareto (Economist), Shigeo Nagashima (Baseball Player), Chris Anderson (Proponent of Long Tail Theory), Yoshitaka Kitao (CEO, SBI Holdings)
\vspace{\baselineskip}
\textbf{Theories/Concepts:} FinTech, Operations Management, Technology \& Operations (T\&O), Management of Technology (MOT), Cream Skimming Operation, Pareto Principle (80/20 Rule), Principle of Least Effort, Juran's Principle, Bowling Pin Operation, Platform Business, Network Effect, Winner-Take-All, Long Tail Theory, Inventory Turnover, Capital Efficiency
\subsection{Comprehension Check Quiz}
\begin{enumerate}
	\item What were the two perspectives mentioned in this lecture for thinking about FinTech?
	\item Who was the then-President of Resona Holdings who commented in 2016 that 'we may have to discard the business model of being a bank'?
	\item What is the name of the operation conventional financial firms conducted, concentrating on the most efficient 'tasty' parts?
	\item What is the empirical rule stating that '80\% of outcomes are generated by 20\% of causes'?
	\item Who is the Italian economist who proposed the Pareto Principle?
	\item What is the operation required in the FinTech era, replacing the Cream Skimming Operation?
	\item What is the business model that executes the Bowling Pin Operation?
	\item What is the effect called where the value obtained by each user increases as more users join?
	\item When the network effect is active, what phenomenon becomes likely to occur in the market?
	\item What smartphone app was given as an example of the network effect in the lecture?
	\item What theory explains the structure in internet business where profits are made not just from the best-selling products (head) but also from the niche products (tail)?
	\item Who proposed the Long Tail theory?
	\item According to Chris Anderson's discovery, did the profit of traditional stores (real business) come mainly from the top 20\% of products or the bottom 80\%?
	\item What characteristic of the internet business revenue structure did Chris Anderson discover, regarding profits being generated from a part that was previously unprofitable?
	\item What does having a high inventory turnover signify?
\end{enumerate}
\subsubsection*{Answer Key}
1. Technology \& Operations (T\&O) and Management of Technology (MOT), 2. Kazuhiro Ajima, 3. Cream Skimming Operation, 4. Pareto Principle, 5. Vilfredo Pareto, 6. Bowling Pin Operation, 7. Platform Business, 8. Network Effect, 9. Winner-take-all, 10. LINE, 11. The Long Tail Theory, 12. Chris Anderson, 13. The top 20\% of products (the part accounting for 80\% of sales), 14. The 'Long Tail' part (90\% of products in the lecture's example), 15. High capital efficiency
\section{FinTech as Technology Architecture}
\subsection{Introduction}
In this lecture, a strong concern was presented that Japan's financial industry might follow the same path as its former digital appliance industry—generating active innovation but failing to create profit, thus becoming 'the second coming of the digital appliance industry.' Behind this lies a polarization in Japan's industrial structure: Japanese companies struggle to generate profit from modularized products, yet maintain high added value in industries requiring close coordination ('suriawase'), such as the automotive industry.
The purpose of this report is to use the framework of '\textbf{architecture}'—a key concept for analyzing this problem—to re-evaluate the essence of FinTech as a technology. It will then consider the impact this will have on the existing financial industry and the potential for structural disruption.
\subsection{Key Concepts and Points}
The architecture theory centrally discussed in this lecture is organized below by its definition and classifications.
\paragraph{What is Architecture?}
\textbf{Architecture} (product architecture) is a concept originally used in manufacturing, but in this lecture, it is treated as a general framework applicable to services and systems as well.
Its definition is: 'A design methodology concerning how to decompose a product as a system into subsystems (components), and how to define the relationships (\textbf{interfaces}) between those subsystems.' Differences in architecture are a critical element that defines the very nature of industrial structure, rules of competition, and corporate strategy.
\paragraph{Modular and Integral Types}
Based on the classification by business scholar \textbf{Takahiro Fujimoto}, architecture is broadly divided into two types.
\paragraph{Modular Type (Combination Type)}
An architecture where the interfaces (rules for combination) between components (modules) are standardized in advance. During development and manufacturing, components made according to these rules are combined like blocks.
\begin{itemize}
	\item \textbf{Characteristic:} A simple, near one-to-one relationship between function and structure (e.g., the 'print' function in a PC is handled by the 'printer' structure).
	\item \textbf{Examples:} PCs, smartphones, digital appliances.
\end{itemize}
\paragraph{Integral Type (Adjustment-based Type)}
An architecture where interfaces are not completely defined in advance. Instead, during the development and manufacturing stages, components are built while performing precise adjustments (\textbf{'suriawase'} or 'close coordination') between them, all while pursuing optimal overall performance.
\begin{itemize}
	\item \textbf{Characteristic:} A complex N-to-N relationship between function and structure (e.g., the 'driving stability' function of a car is determined by the mutual interaction of many components, such as the suspension, body, and engine).
	\item \textbf{Examples:} Automobiles, photocopiers, game consoles.
\end{itemize}
\paragraph{Two-Axis Classification of Architecture}
Architecture is further classified by the following two axes:
\begin{enumerate}
	\item \textbf{Inter-component Characteristics (Interface Characteristics):}
	      The degree of independence between components. Either '\textbf{Modular}' (high independence) or '\textbf{Integral}' (high interdependence).
	\item \textbf{Openness Characteristics:}
	      How standardized the interface is within the industry. Either '\textbf{Open Standard}' (industry standard) or '\textbf{Closed Proprietary}' (company/product specific).
\end{enumerate}
For example, a PC is a typical 'modular' and 'open standard' product, making component replacement and new market entry easy. In contrast, an automobile is a typical 'integral' and 'closed proprietary' product, requiring component coordination ('suriawase') and having low compatibility with other companies' products.
\subsection{Application and Case Analysis}
Applying architecture theory to analyze currency and FinTech.
\paragraph{Analysis: The Architecture of Money (Currency)}
This lecture regards money (currency) as a human invention and a 'type of technology.' While there is a debate in monetary theory, as organized by \textbf{Katsuhito Iwai}, between the \textbf{state theory of money (chartalism)} and the \textbf{commodity theory of money}, we will focus on its functions here.
The economic anthropologist \textbf{Karl Polanyi} listed the following four main functions of money:
\begin{enumerate}
	\item \textbf{Measure of value:} Objectively measuring the exchange value of goods.
	\item \textbf{Payment:} Settling debts (taxes, fines, etc.).
	\item \textbf{Store of value:} Retaining value into the future (e.g., bank deposits, securities).
	\item \textbf{Medium of exchange:} Overcoming the difficulty of \textbf{barter} (the difficulty of a double coincidence of wants) and facilitating transactions.
\end{enumerate}
These functions are handled by relatively independent structures (e.g., 'measure of value' = currency unit, 'payment' = settlement system, 'store of value' = bank deposits), and the relationship between function and structure is close to one-to-one. Therefore, it was reasoned that the architecture of money is likely \textbf{modular} in nature.
\paragraph{Analysis: The Architecture of FinTech}
The services provided by FinTech companies (as classified by NTT Management Research Institute) can also be broken down by function.
\begin{itemize}
	\item \textbf{Payment/Remittance} $\rightarrow$ Mobile payments, P2P transfer apps
	\item \textbf{Lending/Screening} $\rightarrow$ Social lending, crowdfunding
	\item \textbf{Asset Management} $\rightarrow$ Robo-advisors, automated investment
	\item \textbf{Security} $\rightarrow$ Biometric authentication, fraud detection systems
	\item \textbf{Cryptocurrency} $\rightarrow$ Low-cost international remittance
\end{itemize}
These services are applications provided on smartphones (the prime example of a modular product), and each functions as an independent functional module.
This is a move to 'unbundle' (functionally decompose) the 'comprehensive financial services' or 'universal banking' (which are close to the integral type) that traditional financial institutions have provided.
Therefore, the lecture hypothesized that the architecture of FinTech is 'modular' and, furthermore, is positioned as an 'open standard' type where interfaces become standardized.
\subsection{Deeper Context and Lessons}
The following are topics that diverged from the lecture's main thread and an AI-based supplement to the points.
\textbf{\paragraph{Digression Topic Title: Instructor's Concern: The Bifurcation of Japan's Industrial Profit Structure}}
The introduction of the lecture analyzed the current state of Japanese industry as a strong concern of the instructor. While industries with stable technology like automobiles and steel are ableA to create high added value, electrical industries like semiconductors and digital appliances are in an unfortunate state of being 'busy but not at all profitable,' despite active innovation. It was pointed out that the background to this bifurcation lies in the (in)ability to adapt to modularization.
\textbf{\paragraph{Digression Topic Title: The Monetary Debate: State Theory vs. Commodity Theory}}
As a discussion of the nature of money, Katsuhito Iwai's organization of the debate ('state theory of money,' which holds that money derives from the legal system, vs. 'commodity theory of money,' which holds that money itself is a commodity) was introduced. However, from the perspective of this lecture's architectural analysis, it was stated that which position one takes is not critical.
\textbf{\paragraph{Digression Topic Title: Instructor's Personal Memento: FinTech Japan 2016}}
To conclude the lecture, a commemorative photo of the instructor in front of the FinTech Japan 2016 sponsor board was shown. This photo included the logos of the instructor's former employer and certain corporate groups, suggesting that the theme of FinTech is personally memorable for the instructor.
\textbf{\subsubsection{AI Supplement: Expansion of Key Points}}
\textbf{The Modularity Trap and the Smile Curve}
The lecture pointed out the phenomenon of modular products like digital appliances being 'unprofitable,' but a detailed reference to the mechanism was lacking. This phenomenon is also called the '\textbf{modularity trap}' and is explained in management studies by the concept of the '\textbf{Smile Curve}.'
In a modular architecture, when interfaces become \textbf{open standards}, the manufacturing and assembly of each module (component) becomes easier. This leads to a flood of new entrants, and the modules themselves rapidly become commoditized, leading to fierce price competition.
As a result, the industry's added value (profit) flows away from product assembly and manufacturing (e.g., Japanese appliance makers) and concentrates at the two ends:
\begin{enumerate}
	\item \textbf{Upstream (Core components/technology):} The core modules that control the standard (e.g., Intel's CPUs, Qualcomm's chipsets) or the core technology and patents.
	\item \textbf{Downstream (Services/platforms):} The platforms that have the customer interface and provide services and solutions (e.g., Microsoft's OS, Apple's App Store, Google's Android OS).
\end{enumerate}
This distribution of added value forms a U-shape (like a smiling mouth), hence it is called the 'Smile Curve.' Many Japanese digital appliance makers remained at the bottom of this curve—'assembly'—and thus were unable to make a profit despite their high technological capabilities.
If FinTech is modular and open-standard, the financial industry similarly faces the risk that merely providing functional modules (payments, remittances, etc.) will lead to commoditization, with profits being captured by platformers (e.g., GAFA or aggregators).
\subsection{Conclusion}
This lecture applied architecture theory to hypothesize that FinTech is a technology that is inherently '\textbf{modular}' and '\textbf{open standard}'. This has the potential to fundamentally transform the structure of Japan's existing financial industry (which is closer to integral, comprehensive finance) and carries the risk of bringing about the 'unprofitable innovation' experienced by the digital appliance industry.
Taking into account the Smile Curve discussion supplemented in 'Deeper Context and Lessons,' the practical lesson for financial institutions is that they must not stop at simply introducing FinTech and offering individual functional modules (apps). To cope with the wave of modularization while maintaining and creating added value, the following perspectives are necessary:
\begin{enumerate}
	\item \textbf{Defending the Integral (Adjustment-based) Domain:} Strengthening areas where 'suriawase' (close coordination) creates value, such as consulting capabilities for complex client issues (e.g., asset succession for high-net-worth individuals, advanced corporate financial strategy) that cannot be solved by combining individual modules.
	\item \textbf{Pivoting to a Platform Strategy:} Securing added value downstream (in services) by building a platform that becomes the customer interface, incorporating other companies' modules (FinTech services) as well.
\end{enumerate}
Modularization by FinTech is an unavoidable trend. How to avoid the bottom of the Smile Curve and transition to high-value-added domains within that trend will be the key to the future sustainability of the Japanese financial industry.
\subsection{Key Keywords List}
\textbf{Names:}
Karl Polanyi, Katsuhito Iwai, Takahiro Fujimoto
\vspace{\baselineskip}
\textbf{Theories/Concepts:}
Product Architecture, Modular Type (Combination Type), Integral Type (Adjustment-based Type), Interface, Open Standard, Closed Proprietary, State Theory of Money (Chartalism), Commodity Theory of Money, FinTech, Unbundling, Smile Curve
\subsection{Comprehension Check Quiz}
\begin{enumerate}
	\item The instructor is concerned that Japan's financial industry will become a repeat of which other industry?
	\item What is the design methodology called that decomposes a product into subsystems and defines their interfaces?
	\item What is the katakana term for the 'combination type' of architecture?
	\item What is the katakana term for the 'adjustment-based type' ('suriawase-gata') of architecture?
	\item Digital appliances, which Japanese companies struggle to profit from, are classified as which type?
	\item Automobiles, like the Toyota Prius, are classified as which type?
	\item Karl Polanyi defined four functions of money: measure of value, payment, store of value, and what is the fourth?
	\item In the lecture, the architecture of money was considered to be closer to which type?
	\item Which type has a one-to-one correspondence between function and structure, like a PC?
	\item Which type has one function determined by multiple structures, like a car?
	\item Traditional comprehensive finance and universal banking were considered closer to which type?
	\item FinTech unbundles existing financial services by function, so it is classified as which type?
	\item What is the other axis for classifying architecture, which indicates the degree of interface standardization?
	\item PC components are 'open standard' because their interfaces are standardized, but car components are vehicle-specific and thus called what?
	\item The lecture hypothesized that FinTech is 'modular' and also what?
\end{enumerate}
\paragraph*{Answer Key}
1. The digital appliance industry, 2. Architecture, 3. Modular type, 4. Integral type, 5. Modular type, 6. Integral type, 7. Medium of exchange, 8. Modular type, 9. Modular type, 10. Integral type, 11. Integral type, 12. Modular type, 13. Openness characteristics, 14. Closed proprietary, 15. Open standard
\section{The FinTech Value Chain}
\subsection{Introduction}
This report examines the necessity of the transformation that Japan's financial industry will likely face, based on the premise of understanding \textbf{FinTech} as a \textbf{modular architecture} from a technological perspective. Standing on the viewpoint that architectural characteristics are directly linked to industrial structure and a company's competitive advantage, this report aims to analyze the impact of the shift to a modular model—at which Japanese companies are not adept—on the financial industry and what kind of response it demands.
\subsection{Key Concepts and Points}
\subsubsection{Architecture and Organizational Capability}
Product architecture is broadly divided into the \textbf{integral type}, where inter-component coordination and adjustments are complex, and the \textbf{modular type}, where inter-component interfaces are standardized and combination is easy.
Correspondingly, the required organizational capabilities also differ.
\begin{itemize}
	\item \textbf{Integrative Adjustment Capability}: Necessary for integral-type products. An area where Japanese companies (e.g., the auto industry) have traditionally been strong.
	\item \textbf{Selection and Combination Capability}: Necessary for modular-type products. An area where US, Chinese, and Taiwanese companies are strong.
\end{itemize}
The lecture pointed out that FinTech is clearly \textbf{modular} and that this is a domain where Japanese companies have traditionally been weak.
\subsubsection{Reasons Why Japanese Companies are Weak in Modular Models}
Drawing on the case of the Japanese digital appliance industry, which lost its competitiveness by failing to adapt to the wave of modularization, the following three reasons were given for why Japanese companies are weak in modular models:
\begin{enumerate}
	\item \textbf{Cost Issues}: In a business model that procures components on the market and assembles them, Japanese companies cannot win on price against Chinese and Taiwanese companies with lower manufacturing and SG\&A costs.
	\item \textbf{Weakness in creating global systems}: They are not adept at building global supply chains (like Dell) that source the best components from around the world and deliver them quickly to customers.
	\item \textbf{Inability to become a platform leader}: The \textbf{platform leader} (e.g., Intel, Microsoft) who decides the rules for component combination (industry standards) captures most of the added value. Japanese companies, despite having advanced component technology, tend to be unable to establish this position.
\end{enumerate}
This structure suggests that even financial institutions with high technological capabilities risk failing to adapt to the modular architecture of FinTech—in other words, the risk of becoming '\textbf{the next digital appliance makers}.'
\subsubsection{The Value Chain and Modularization}
We will analyze the impact of modularization from the perspective of the \textbf{Value Chain}, as proposed by \textbf{Michael Porter}. Management scholars \textbf{Baldwin} and \textbf{Clark} listed the following three advantages of modularization:
\begin{itemize}
	\item \textbf{Simplification between subsystems}: Interfaces are standardized, reducing the complexity of the overall system and lowering development costs.
	\item \textbf{Standardization between subsystems}: Standardization enables component sharing and diverse combinations, reducing design and manufacturing costs.
	\item \textbf{Independence of subsystems}: Each subsystem can focus independently on technological development, vitalizing innovation.
\end{itemize}
\subsubsection{Two Major Impacts of Modularization on the Value Chain}
The three advantages above bring two major, and potentially contradictory, impacts to the entire value chain.
\paragraph{1. Eased Integration and Cost Reduction}
As component integration becomes easier, the following phenomena occur:
\begin{itemize}
	\item \textbf{Promotion of inter-firm specialization}: Specialization by function or component, such as design-focused (fabless) or manufacturing-focused (EMS), progresses.
	\item \textbf{Ease of new entry}: New players can enter by simply procuring components from the market, intensifying competition. As a result, differentiation becomes difficult, often leading to \textbf{commoditization} and fierce price competition (excessive competition).
\end{itemize}
In the medium to long term, modularization can be a '\textbf{double-edged sword}' that makes it difficult for companies to create added value.
\paragraph{2. Revitalization of Innovation}
Due to the independence of subsystems, innovation activities can be conducted in a distributed manner, both inside and outside the company, as long as the \textbf{design rules} (design standards) are followed. This accelerates the \textbf{Open Innovation} proposed by Professor \textbf{Henry Chesbrough}.
Because FinTech is modular, innovation is active, and it holds the potential to rapidly render the technology of existing financial institutions obsolete.
\subsubsection{The Importance of the Platform Leader}
'Eased integration (cost reduction)' and 'revitalization of innovation' are naturally in a trade-off relationship. As innovation becomes more active, existing design rules become obsolete, requiring their revision and resetting.
The entity that leads this \textbf{setting and revising of design rules} and leads the entire value chain is called the \textbf{platform leader} (e.g., Intel, Cisco). They create the most added value in modular industries with active innovation.
In the FinTech industry as well, the focal point will be who captures this platform leader position.
\subsection{Application and Case Analysis}
\subsubsection{Digital Appliance and PC Industries}
These are representative modular industries where Japanese companies were defeated. In the PC industry, \textbf{Intel} (CPU) and \textbf{Microsoft} (OS) held the industry standards as platform leaders, while \textbf{Dell} succeeded with its global system (selection and combination capability). Meanwhile, Japanese companies, despite their high technology, failed to become platform leaders and were caught in cost competition.
\subsubsection{DVD Players and Digital Cameras}
For DVD players, Chinese companies entered the market simply by procuring and assembling parts, with over 1,000 entrants leading to excessive competition. In the digital camera market, although \textbf{Matsushita} and \textbf{Sony} supplied the core component, the CCD, competitors were also able to purchase it, making differentiation in the final product difficult and hampering profit creation.
\subsubsection{Smartphones (Japanese Mobile Phones)}
Traditional Japanese mobile phones, which were 'over-built' with an integral mindset, rapidly lost the market with the advent of smartphones (iPhone and Android), which combined modular components. This suggests the possibility that Japanese financial institutions, which possess high-quality, scratch-built (integral) systems, may follow a similar path of collapse by hesitating to introduce seemingly simple FinTech (modular) solutions.
\subsection{Deeper Context and Lessons}
\textbf{\paragraph{Reference to Chapter 16 (Cultural Background of Japanese Companies and Countermeasures)}}
During the lecture, a detailed discussion of the true reasons why Japanese companies are weak in modular models (cultural backgrounds beyond just cost) and specific countermeasures for FinTech (such as the path to becoming a platform leader) was intentionally postponed, marked as 'Chapter 16.'
\textbf{\paragraph{Instructor's Personal View: Recollection of a Japanese Mobile Phone Developer}}
An anecdote was shared about a Japanese mobile phone developer the instructor had met in the past. He recounted, 'Our company was able to integrate (in-house) extremely superior component technology, so we hesitated to use the cheap, modular components supplied externally.' As a result, his business collapsed with the arrival of smartphones. This serves as a strong warning of a trap that current Japanese financial institutions are likely to fall into.
\textbf{\paragraph{Invitation to Next Week (Operations Management)}}
The end of this lecture connected to the 'Operations Management' lectures starting next week. Technological evolution and globalization driven by FinTech are forcing unprecedented operational reforms upon financial institutions. It was suggested that the lectures from next week would provide insights into a trinitarian reform of strategy, operations, and organization.
\textbf{\subsubsection{AI Supplement: Expansion of Key Points}}
The lecture emphasized the importance of the platform leader, but the following supplement regarding their profit mechanism and the role of design rules is beneficial for deepening understanding.
\paragraph{Design Rules and Value Distribution}
The core of modular architecture lies in the \textbf{design rules} that govern the entire system. Design rules are divided into 'Visible Rules' and 'Hidden Rules.'
\begin{itemize}
	\item \textbf{Visible Rules}: Information disclosed to module suppliers, such as interface specifications.
	\item \textbf{Hidden Rules}: The core design philosophy and future change plans for the system, known only to the platform leader.
\end{itemize}
The platform leader strategically controls these 'Hidden Rules' and leads changes in the architecture, thereby gaining an advantage over other companies that provide complementary modules (apps, peripherals, etc.) and enabling them to monopolistically capture value (profit).
\paragraph{Network Externalities}
\textbf{Network externalities} (network effects) are a source of the platform leader's strength. This is the phenomenon where 'the value of a product or service increases as more users use it.'
In a FinTech platform (e.g., a payment system or OS), once a large number of users and complementary services (linked apps, etc.) are acquired, it creates a barrier to entry (increasing switching costs), further solidifying the leader's position. Therefore, competition in the FinTech industry is not just about technological capability, but also about the speed at which these network externalities can be built.
\subsection{Conclusion}
This lecture clarified the threat facing Japanese financial institutions by re-framing FinTech as a \textbf{modular architecture}. It showed that possessing high technological capabilities and high-quality existing systems (integral type) could, paradoxically, delay the response to modularization, risking a repeat of the failures of the digital appliance and mobile phone manufacturers.
The practical lesson from this threat is the danger of adhering to existing 'over-built' systems and operations. Japanese financial institutions must recognize that their strengths can be transformed into weaknesses in the age of modularization. They must not stop at mere technology adoption (using FinTech) but must seriously consider transforming their value chain, adopting strategies to become a \textbf{platform leader}, or adopting a \textbf{keystone strategy} (a strategy to function as the core of an ecosystem).
The study of operations management from next week will be essential for gaining concrete insights to put this transformation into practice.
\subsection{Key Keywords List}
\textbf{Names:}
Michael Porter, Baldwin, Clark, Henry Chesbrough
\vspace{\baselineskip}
\textbf{Theories/Concepts:}
Modular Architecture, Integral Architecture, Selection and Combination Capability, Integrative Adjustment Capability, Platform Leader, Value Chain, Open Innovation, EMS (Electronics Manufacturing Services), Fabless, Design Rules, Commoditization, Excessive Competition, Keystone Strategy, Network Externalities
\subsection{Comprehension Check Quiz}
\begin{enumerate}
	\item What are the two types of product architecture?
	\item Which architecture have Japanese companies traditionally excelled at?
	\item What organizational capability is required for modular-type products?
	\item In this lecture, which architecture was FinTech classified as?
	\item What are the three reasons given for why Japanese companies are weak in modular models?
	\item Which management scholar proposed the concept of the Value Chain?
	\item What are the three advantages of modularization proposed by Baldwin and Clark?
	\item What are the two major impacts of modularization on the value chain?
	\item When modularization makes new entry easy, what state is the market likely to fall into?
	\item In a modular product industry, what is the leading company that sets the rules for component combination and industry standards called?
	\item Who is the University of California scholar who proposed the concept of Open Innovation?
	\item What industry did the instructor use as an example of the danger Japanese financial institutions face due to the advance of FinTech?
	\item In a modular industry with active innovation, what is the crucial role led by the platform leader?
	\item In the instructor's example of the Japanese mobile phone developer, what was the cause of failure for the company that had high-quality in-house products?
	\item What is one of the strategic modes of behavior for a platform leader mentioned in the lecture?
\end{enumerate}
\subsubsection*{Answer Key}
1. Integral type and Modular type, 2. Integral type, 3. Selection and combination capability, 4. Modular type, 5. Cost issues, Weakness in creating global systems, Inability to become a platform leader, 6. Michael Porter, 7. Simplification, standardization, and independence of subsystems, 8. Eased integration/cost reduction, and Revitalization of innovation, 9. Excessive competition (or commoditization), 10. Platform leader, 11. Henry Chesbrough, 12. The digital appliance industry, 13. Setting and revising design rules, 14. Hesitating to use cheap, external modular components, 15. Keystone strategy
\end{document}
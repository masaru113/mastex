\documentclass[uplatex,a4j,12pt,dvipdfmx]{jsarticle}
\usepackage{amsmath,amsthm,amssymb,bm,color,enumitem,mathrsfs,url,epic,eepic,ascmac,ulem,here,ascmac}
\usepackage[letterpaper,top=2cm,bottom=2cm,left=3cm,right=3cm,marginparwidth=1.75cm]{geometry}
\usepackage[english]{babel}
\usepackage[dvipdfm]{graphicx}
\usepackage[hypertex]{hyperref}
\title{Operations Management Lecture 0 Lecture Notes\newline (Supplement: FinTech and Operations Management)}
\author{M. O.}
\date{\today}
\begin{document}
\maketitle
\tableofcontents
\section{FinTech and Operations Management}
\subsection{Introduction}
This lecture is the 0th session, 'Operations Management,' of the course 'Operations Management as the Foundation of FinTech.' The objective of this paper is to conduct an integrated analysis of FinTech not as a discrete domain of traditional finance theory or technology theory, but from the perspective of \textbf{Technology \& Operations} (T\&O) or \textbf{Management of Technology} (MOT).

The lecture begins by introducing a comment made by Kazuhiro Ajima, then-President of Resona Holdings, on NHK's 'Close-up Gendai' in 2016: 'We may have to discard the business model of being a bank.' This symbolizes the vague yet urgent sense of crisis among existing financial institutions, wondering if the \textbf{redefinition of industries} and disruption of existing businesses caused by the internet in other sectors might also occur in the financial industry. This report aims to concretely unravel this 'vague sense of crisis' from the viewpoint of transformation in operations management.

\subsection{Key Concepts and Points}
In this lecture, to understand the transformation brought about by FinTech, conventional operations and new operations were explained in contrast.

\subsubsection{Conventional Financial Operations: Cream Skimming}
The operations management of the conventional financial industry is characterized as a \textbf{Cream Skimming Operation}. This is a strategy of concentrating on the most efficient and 'tasty' parts of the business (such as high-profit customer segments or products), much like eating only the sweetest part of a watermelon.

The theoretical background for this strategy is the \textbf{Pareto Principle} (1897), proposed by the Italian economist \textbf{Vilfredo Pareto}. Also known as the '80/20 rule,' this is a law of imbalance stating that 80\% of outcomes result from 20\% of causes, typified by empirical rules like '80\% of society's total income is held by 20\% of high-income earners.' In the financial industry, the premise has long been that the top 20\% of customers generate 80\% of the profits; therefore, cream skimming, which concentrates resources on this 20\%, has been considered the most efficient operation.

\subsubsection{FinTech-Era Operations: The Bowling Pin Operation}
However, the evolution of the internet and technology is overturning this premise. What is required in the FinTech era is the \textbf{Bowling Pin Operation}. This refers to a strategy of capturing the entire market (knocking down all 10 pins) by precisely targeting and knocking down a single target—the number 1 pin in bowling—and leveraging the resulting ripple effect.

This operational transformation is being driven by the rise of \textbf{platform businesses}.

\subsubsection{Theories Supporting Platform Businesses}
\paragraph{\textbf{Network Effect}}
The primary characteristic of a platform business is that it benefits from the \textbf{Network Effect}. The network effect is a phenomenon where, as the number of users of a product or service increases, the utility or value each user derives from that service also increases. When this effect is active, a specific product or service can grow rapidly, making it easy for a \textbf{winner-take-all} scenario to emerge in the market.

\paragraph{\textbf{The Long Tail} Theory}
The revenue structure of the Bowling Pin Operation is explained by \textbf{The Long Tail} theory, proposed by \textbf{Chris Anderson}.

In conventional brick-and-mortar businesses, the \textbf{Pareto Principle} was strongly at play: 80\% of sales were generated from the top 20\% of inventory—the hit products (the 'head'). These products had a high \textbf{inventory turnover} and good \textbf{capital efficiency}.

In internet businesses, conversely, there are no physical shelf-space constraints, allowing the 80\% of niche products (the 'tail') that were previously ignored to become targets for sale. The core of Chris Anderson's discovery was not just the existence of this 'tail,' but his demonstration of a structure where the profits generated from this niche product group (90\% of all products in the textbook's example), previously considered unprofitable, could rival the profits (33\%) generated by conventional hit products (2\% of products, generating 33\% of profit).

\subsubsection{The Inevitability of Operational Transformation}
Internet businesses, including FinTech, are premised on a structure that profits from the entirety of the \textbf{Long Tail}—in other words, the \textbf{Bowling Pin Operation}. This signifies a fundamental shift away from the conventional \textbf{Cream Skimming} model. Companies unable to realize this new operation are likely to be eliminated as entities 'that cannot take all' of the market, and the financial industry is no exception.

\subsection{Application and Case Studies}
\subsubsection{Case Study: Resona Holdings}
The statement 'discard the business model of being a bank,' made by President \textbf{Kazuhiro Ajima} of Resona Holdings and introduced at the beginning of the lecture, indicates, in light of this lecture's points, a sense of crisis that existing \textbf{Cream Skimming} operations will be fundamentally disrupted by the new operations of FinTech companies. The true nature of this 'vague anxiety' lies in the difficulty of adapting to the \textbf{Bowling Pin Operation}.

\subsubsection{Case Study: LINE}
\textbf{LINE} was raised in the lecture as a typical example of the \textbf{Network Effect}. LINE's value depends heavily not only on the quality of the service itself, such as the app's functionality or the appeal of its 'stickers,' but also on the scale of the network—the fact that 'one's friends are members and available for communication.' Once the user base surpassed a critical mass, its value increased exponentially (accelerating the speed of growth), resulting in its \textbf{winner-take-all} position in the communication platform market.

\subsection{Deeper Context and Lessons}
In addition to the main thesis of the lecture, peripheral information to deepen contextual understanding and the instructor's personal views were mentioned.

\textbf{\paragraph{Digression Topic: Shigeo Nagashima's Watermelon Anecdote}}
As a metaphor for \textbf{Cream Skimming}, an anecdote was shared about \textbf{Shigeo Nagashima} eating the sweetest part of a watermelon first at a training camp. This draws a parallel between a star player focusing on the 'tasty part' (hitting home runs) and the financial industry concentrating on its most efficient business.

\textbf{\paragraph{Digression Topic: The Universality of the Pareto Principle}}
It was added that the \textbf{Pareto Principle} is a universal law observed not only in human economic activities (income, sales, taxation) but also in insect societies, such as ants and bees (where 20\% of ants gather 80\% of the food).

\textbf{\paragraph{Digression Topic: Analogies to the Pareto Principle}}
As analogies to the Pareto Principle, the \textbf{Principle of Least Effort} ('resources adjust themselves to minimize labor') and \textbf{Juran's Principle} (in quality improvement, a few factors have a decisive effect on the whole) were introduced.

\textbf{\paragraph{Digression Topic: The Instructor's Formative Experience (FinTech Japan 2016)}}
The instructor, Mr. Nomaguchi, left the financial industry over 20 years ago and moved into consulting. 'FinTech Japan 2016' (featuring a lecture by SBI Holdings President \textbf{Yoshitaka Kitao}) was an occasion for the instructor to 're-encounter' the financial industry. While intuiting that FinTech had a high affinity with his own consulting domain, the instructor stated that he felt not only anticipation but also strong 'anxiety' about whether the financial industry could break away from its old constitution (termed 'Goso Sendan Hoshiki' [convoy system] by the instructor) and evolve amid global competition.

\textbf{\subsubsection{AI Supplement: Expansion of Key Points}}
The lecture presented the 'Bowling Pin Operation' and 'platform businesses' as operations for the FinTech era. However, this chapter lacked sufficient mention of the perspective from the lecture's subtitle, 'FinTech as a Technology Architecture'—that is, how these are technologically realized.

The core technological architecture that enables the \textbf{Bowling Pin Operation} realized by FinTech is the \textbf{API (Application Programming Interface)}.

Conventional financial institutions possessed a closed, 'vertically integrated' architecture, handling everything from system development to operation and service provision in-house.

In contrast, FinTech, particularly the model known as \textbf{BaaS (Banking as a Service)}, adopts a 'horizontally specialized' architecture. In this model, banks modularize their own financial functions (payments, account management, credit, etc.) as \textbf{APIs} and provide them as services to third-party companies (FinTech firms or non-financial enterprises).

This allows the banks themselves to become platforms (function providers), enabling diverse external companies to develop niche \textbf{Long Tail} services on top of them. This architectural transformation makes conventional \textbf{Cream Skimming} (doing only the 'tasty' parts in-house) structurally difficult, forcing a shift to a \textbf{Bowling Pin Operation} that generates revenue from the entire diverse group of services created by API partners. 'FinTech as a Technology Architecture' is thought to refer precisely to this modularization and recombination of systems via APIs.

\subsection{Conclusion}
This lecture (Session 0) presented the importance of re-evaluating FinTech not merely as a financial product or IT technique, but from the perspective of \textbf{Technology \& Operations} (MOT).

The traditional financial industry was optimized for \textbf{Cream Skimming} operations based on the \textbf{Pareto Principle}. However, the rise of the internet and platform businesses (driven by the \textbf{Network Effect} and \textbf{Long Tail} theory) has made the shift to a \textbf{Bowling Pin Operation} (a winner-take-all model) targeting the entire market inevitable.

The instructor's formative experience discussed in the 'Deeper Context'—the 'anxiety' felt at FinTech Japan 2016—was not merely apprehension about industry restructuring. It is inferred to have been a concern regarding structural and technological challenges: whether the old constitution ('Goso Sendan Hoshiki') of Japanese financial institutions and their closed, 'vertically integrated' system architectures would impede adaptation to the 'horizontally specialized' \textbf{Bowling Pin Operation}, such as the \textbf{API Economy} or \textbf{BaaS} discussed in the AI supplement.

In future studies, it is extremely important to maintain an \textbf{MOT} perspective, focusing not only on financial theory but also on how the \textbf{technology architecture} (APIs, modularization) that supports FinTech transforms operations and rebuilds competitive advantage.

\subsection{List of Important Keywords}
\textbf{Names:} Kazuhiro Ajima (President, Resona Holdings, then), Vilfredo Pareto (Economist), Shigeo Nagashima (Baseball Player), Chris Anderson (Proponent of Long Tail Theory), Yoshitaka Kitao (President, SBI Holdings)
\vspace{\baselineskip}

\textbf{Theories/Concepts:} FinTech, Operations Management, Technology \& Operations (T\&O), Management of Technology (MOT), Cream Skimming Operation, Pareto Principle (80/20 Rule), Principle of Least Effort, Juran's Principle, Bowling Pin Operation, Platform Business, Network Effect, Winner-Take-All, The Long Tail Theory, Inventory Turnover, Capital Efficiency

\subsection{Comprehension Quiz}
\begin{enumerate}
	\item What two perspectives did this lecture raise for considering FinTech?
	\item Who was the (then) president of Resona Holdings who commented in 2016 that 'we may have to discard the business model of being a bank'?
	\item What is the name of the operation conventional financial institutions conducted, focusing on the most efficient, 'tasty' parts?
	\item What is the empirical rule stating that '80\% of outcomes result from 20\% of causes'?
	\item Who is the Italian economist who proposed the Pareto Principle?
	\item What operation is required in the FinTech era, replacing the Cream Skimming Operation?
	\item What business model executes the Bowling Pin Operation?
	\item What is the effect called where the value obtained by each user increases as the number of users grows?
	\item When the network effect is active, what phenomenon tends to occur in the market?
	\item What smartphone app was given in the lecture as an example of the network effect?
	\item What theory explains the structure in internet business where profits are generated not only from the hit products (head) but also from the niche products (tail)?
	\item Who proposed the Long Tail theory?
	\item According to Chris Anderson's discovery, in conventional stores (brick-and-mortar businesses), were profits generated mainly from the top 20\% of products or the bottom 80\%?
	\item What characteristic of the internet business revenue structure did Chris Anderson discover, regarding profits being generated from a part previously considered unprofitable?
	\item What does high inventory turnover imply is also high?
\end{enumerate}

\subsubsection*{Answer Key}
1. Technology \& Operations (T\&O) and Management of Technology (MOT), 2. Kazuhiro Ajima, 3. Cream Skimming Operation, 4. Pareto Principle, 5. Vilfredo Pareto, 6. Bowling Pin Operation, 7. Platform Business, 8. Network Effect, 9. Winner-Take-All, 10. LINE, 11. The Long Tail Theory, 12. Chris Anderson, 13. The top 20\% of products (the part accounting for 80\% of sales), 14. The 'Long Tail' part (90\% of products in the lecture's example), 15. Capital Efficiency

\section{FinTech as a Technology Architecture}
\subsection{Introduction}
This lecture presented a strong concern that Japan's financial industry might follow the same path as its digital home appliance industry, becoming a 'second digital appliance sector'—unable to generate profits despite active innovation. Behind this lies a polarization of Japan's industrial structure: Japanese firms struggle to create profit from modularized products, yet maintain high added value in industries requiring integration and coordination ('Suriawase'), such as the automotive industry.

The purpose of this report is to analyze this problem using the key conceptual framework of '\textbf{Architecture}.' It aims to re-evaluate the essence of FinTech as a technology and consider the impact and potential for structural disruption it poses to the existing financial industry.

\subsection{Key Concepts and Points}
The definitions and classifications of the architecture theory, which was central to this lecture, are summarized here.

\paragraph{What is Architecture?}
\textbf{Architecture} (product architecture) is a concept originally used in manufacturing, but in this lecture, it is treated as a general framework applicable to services and systems as well.

Its definition is 'a design methodology concerning how a product as a system is decomposed into subsystems (components) and how the relationships (\textbf{interfaces}) between those subsystems are defined.' Differences in architecture become critical factors that dictate the very nature of industrial structure, rules of competition, and corporate strategy.

\paragraph{Modular and Integral Types}
Based on the classification by business scholar \textbf{Takahiro Fujimoto}, architecture is broadly divided into two types.

\paragraph{Modular Type (Combinatorial)}
This is an architecture where the interfaces (rules for combination) between components (modules) are standardized in advance. During development and manufacturing, components made according to these rules are combined like building blocks.
\begin{itemize}
	\item \textbf{Characteristics:} Features a simple, close-to-one-to-one relationship between function and structure (e.g., the 'print' function of a PC is handled by the 'printer' structure).
	\item \textbf{Examples:} PCs, smartphones, digital home appliances.
\end{itemize}

\paragraph{Integral Type (Integrative/Suriawase)}
This is an architecture where interfaces are not completely defined in advance. Instead, it is built up through meticulous adjustment (\textbf{'Suriawase' or integration}) between components during the development and manufacturing stages, all while pursuing optimization of the whole.
\begin{itemize}
	\item \textbf{Characteristics:} Features a complex N-to-N relationship between function and structure (e.g., a car's 'driving stability' function is determined by the mutual interaction of many components, such as the suspension, body, and engine).
	\item \textbf{Examples:} Automobiles, photocopiers, game consoles.
\end{itemize}

\paragraph{Two-Axis Classification of Architecture}
Architecture is further classified in detail by the following two axes:
\begin{enumerate}
	\item \textbf{Inter-component characteristics (Interface characteristics):} The degree of independence between components. '\textbf{Modular}' (high independence) or '\textbf{Integral}' (high interdependence).
	\item \textbf{Openness characteristics:} How standardized the interface is within the industry. '\textbf{Open Standard}' (industry standard) or '\textbf{Closed Proprietary}' (company/product specific).
\end{enumerate}
For example, a PC is a typical example of 'Modular' and 'Open Standard,' where component replacement and new entry are easy. In contrast, an automobile is a typical example of 'Integral' and 'Closed Proprietary,' where component integration is essential and compatibility with other companies' products is low.

\subsection{Application and Case Studies}
Architecture theory is applied to analyze money (currency) and FinTech.

\paragraph{Analysis: The Architecture of Money (Currency)}
In this lecture, money (currency) is also regarded as an artificial invention and 'a type of technology.' In monetary theory, there is a debate between \textbf{Chartalism (the state theory of money)} and \textbf{Metallism (the commodity theory of money)}, as summarized by \textbf{Katsuhito Iwai}, but here we focus on its functions.

The economic anthropologist \textbf{Karl Polanyi} listed the following four main functions of money:
\begin{enumerate}
	\item \textbf{Measure of Value:} Objectively measuring the exchange value of goods.
	\item \textbf{Payment:} Settling debts (taxes, compensation, etc.).
	\item \textbf{Store of Value:} Holding value over time (e.g., bank deposits, securities).
	\item \textbf{Medium of Exchange:} Overcoming the difficulty of \textbf{barter} (the difficulty of the double coincidence of wants) and facilitating transactions.
\end{enumerate}
These functions are each handled by relatively independent structures (e.g., 'Measure of Value' = currency unit, 'Payment' = settlement system, 'Store of Value' = bank deposit), and the relationship between function and structure is close to one-to-one. Therefore, it was considered that the architecture of money is likely inherently \textbf{Modular}.

\paragraph{Analysis: The Architecture of FinTech}
The services provided by FinTech companies (according to NTT Management Research Institute's classification) are also decomposable by function.
\begin{itemize}
	\item \textbf{Payments \& Remittances} $\rightarrow$ Mobile payments, P2P transfer apps
	\item \textbf{Lending \& Credit Scoring} $\rightarrow$ Social lending, Crowdfunding
	\item \textbf{Asset Management} $\rightarrow$ Robo-advisors, Automated investing
	\item \textbf{Security} $\rightarrow$ Biometric authentication, Fraud detection systems
	\item \textbf{Virtual Currency} $\rightarrow$ Low-cost international remittances
\end{itemize}
These services are applications provided on smartphones (a prime example of a modular product), and each functions as an independent functional module.

This is a movement that \textbf{unbundles} (functionally decomposes) the '\textbf{comprehensive finance}' or '\textbf{universal banking}' (which are close to the Integral type) traditionally provided by financial institutions.

Therefore, the lecture hypothesized and concluded that FinTech's architecture is positioned as '\textbf{Modular}' and, furthermore, as '\textbf{Open Standard},' where interfaces are progressively standardized.

\subsection{Deeper Context and Lessons}
Topics derived from the main lecture and supplementary points by AI are noted below.

\textbf{\paragraph{Digression Topic: Instructor's Concern: Polarization of Japan's Industrial Profit Structure}}
In the lecture's introduction, the current state of Japanese industry was analyzed as a strong concern of the instructor. While industries with stable technology like automobiles and steel are able to create high added value, the electronics industry, including semiconductors and digital appliances, is in an unfortunate state of being 'busy but not profitable at all,' despite active innovation. It was pointed out that the background to this polarization lies in the varying skill in responding to modularization.

\textbf{\paragraph{Digression Topic: The Monetary Debate: State vs. Commodity Theories}}
As a debate over the nature of money, Katsuhito Iwai's summary (the 'state theory of money,' which holds that money derives from the legal system, and the 'commodity theory of money,' which holds that money is itself a commodity) was introduced. However, it was noted that from the perspective of this lecture's architectural analysis, which stance one takes is not important.

\textbf{\paragraph{Digression Topic: The Instructor's Personal Memento: FinTech Japan 2016}}
To conclude the lecture, a commemorative photo was shown of the instructor taken in front of the sponsor board at FinTech Japan 2016. This photo included logos of the instructor's former company and specific corporate groups, suggesting that the theme of FinTech is also personally memorable for the instructor.

\textbf{\paragraph{AI Supplement: Expansion of Key Points}}
\textbf{The Modularity Trap and the Smile Curve}

The lecture pointed out the phenomenon that modular products like digital appliances are 'unprofitable,' but a detailed explanation of the mechanism was lacking. This phenomenon is also called the '\textbf{Modularity Trap}' and is explained in business administration by the concept of the '\textbf{Smile Curve}.'

In a modular architecture, when interfaces become \textbf{Open Standard}, the manufacturing and assembly of each module (component) become easy. This leads to a flood of new entrants, and the modules themselves rapidly commoditize, resulting in fierce price competition.

As a result, the industry's added value (profit) flows away from product assembly and manufacturing (e.g., Japanese appliance makers) and concentrates at the following two ends:
\begin{enumerate}
	\item \textbf{Upstream (Key Components/Technology):} Core modules that control the standards (e.g., Intel's CPU, Qualcomm's chipsets) and core technologies/patents.
	\item \textbf{Downstream (Services/Platforms):} Platforms that own the customer interface and provide services or solutions (e.g., Microsoft's OS, Apple's App Store, Google's Android OS).
\end{enumerate}
This U-shaped distribution of added value (like the shape of a smile) is called the 'Smile Curve.' Many Japanese digital appliance makers remained at the 'assembly' part at the bottom of this curve, and thus were unable to raise profits despite their high technological capabilities.

If FinTech is indeed 'Modular' and 'Open Standard,' then the financial industry likewise risks commoditization if it only provides simple functional modules (like payments or remittances), with profits being captured by platformers (e.g., GAFA or aggregators).

\subsection{Conclusion}
This lecture applied architecture theory to hypothesize that FinTech is a technology with inherently '\textbf{Modular}' and '\textbf{Open Standard}' characteristics. This carries the risk of fundamentally transforming the structure of Japan's existing financial industry (which is close to 'Integral' comprehensive finance) and bringing about the kind of 'unprofitable innovation' experienced by the digital appliance industry.

Based on the Smile Curve discussion supplemented in the 'Deeper Context and Lessons,' the practical lesson for financial institutions is that they must not stop at merely introducing FinTech technology and offering individual functional modules (apps). To maintain and create added value while responding to the wave of modularization, the following perspectives are necessary:
\begin{enumerate}
	\item \textbf{Defending the Integral (Suriawase) Domain:} Strengthening areas where 'integration' creates value, such as consulting capabilities for complex client issues that cannot be solved by combining individual modules (e.g., asset succession for high-net-worth individuals, advanced corporate financial strategies).
	\item \textbf{Shifting to a Platform Strategy:} Securing added value downstream (in services) by becoming the customer interface themselves and building a platform that incorporates other companies' modules (FinTech services).
\end{enumerate}
Modularization by FinTech is an unavoidable trend. How the Japanese financial industry can avoid the bottom of the Smile Curve and shift to high-added-value domains within this trend will be the key to its future sustainability.

\subsection{List of Important Keywords}
\textbf{Names:} Karl Polanyi, Katsuhito Iwai, Takahiro Fujimoto
\vspace{\baselineskip}

\textbf{Theories/Concepts:} Product Architecture, Modular Type (Combinatorial), Integral Type (Integrative/Suriawase), Interface, Open Standard, Closed Proprietary, Chartalism (State Theory of Money), Metallism (Commodity Theory of Money), FinTech, Unbundling, Smile Curve

\subsection{Comprehension Quiz}
\begin{enumerate}
	\item The instructor holds concerns that the Japanese financial industry might repeat the failure of which other industry?
	\item What is the design methodology called that involves decomposing a product into subsystems and defining their interfaces?
	\item What is the technical term for the 'combinatorial' (組み合わせ型) type of architecture?
	\item What is the technical term for the 'integrative' (すり合わせ型) type of architecture?
	\item Digital home appliances, which Japanese companies struggle to profit from, are classified as which type?
	\item Automobiles, such as the Toyota Prius, are classified as which type?
	\item What are the four functions of money defined by Karl Polanyi: Measure of Value, Payment, Store of Value, and what else?
	\item In the lecture, which type was the architecture of money considered to be closer to?
	\item Which type, like a PC, tends to have a one-to-one correspondence between function and structure?
	\item Which type, like an automobile, has a single function determined by multiple structures?
	\item Traditional comprehensive finance or universal banking were considered closer to which type?
	\item FinTech unbundles existing financial services by function; therefore, which type is it classified as?
	\item What is the other axis for classifying architecture, which indicates the degree of interface standardization?
	\item PC components are 'Open Standard' because their interfaces are standardized, but what are automobile components called, as they are specific to the car model?
	\item As the lecture's hypothesis, FinTech was positioned as 'Modular' and what else?
\end{enumerate}

\paragraph*{Answer Key}
1. Digital home appliances, 2. Architecture, 3. Modular type, 4. Integral type, 5. Modular type, 6. Integral type, 7. Medium of Exchange, 8. Modular type, 9. Modular type, 10. Integral type, 11. Integral type, 12. Modular type, 13. Openness characteristics, 14. Closed Proprietary, 15. Open Standard

\section{The FinTech Value Chain}
\subsection{Introduction}
This report, based on the premise of FinTech as a \textbf{Modular Architecture} from a technological perspective, considers the necessity of the transformation that the Japanese financial industry will likely face. Standing on the viewpoint presented in the lecture that architectural characteristics are directly linked to industrial structure and a firm's competitive advantage, the objective is to analyze what kind of impact the shift to a modular type—an area where Japanese firms are particularly weak—will have on the financial industry and what responses it demands.

\subsection{Key Concepts and Points}
\subsubsection{Architecture and Organizational Capability}
Product architecture is broadly divided into the \textbf{Integral type}, where coordination and integration ('suriawase') between components are complex, and the \textbf{Modular type}, where inter-component interfaces are standardized and combination is easy.

Correspondingly, the required organizational capabilities also differ.
\begin{itemize}
	\item \textbf{Integrative Capability (Suriawase):} Necessary for integral-type products. An area where Japanese firms (e.g., the auto industry) have traditionally been strong.
	\item \textbf{Selection \& Combination Capability:} Necessary for modular-type products. An area where US, Chinese, and Taiwanese firms are strong.
\end{itemize}
It was pointed out in the lecture that FinTech is clearly \textbf{Modular}, a domain where Japanese companies have traditionally been weak.

\subsubsection{Reasons Why Japanese Firms are Weak in Modular Types}
Drawing on the case of the Japanese digital appliance industry, which lost its competitiveness by failing to adapt to the wave of modularization, the following three reasons were given for why Japanese firms are weak in modular types:
\begin{enumerate}
	\item \textbf{Cost Issues:} In a business model based on procuring and combining components from the market, they cannot compete on price with Chinese and Taiwanese firms that have lower manufacturing and SG\&A costs.
	\item \textbf{Weakness in Global System Building:} They are poor at constructing global supply chains (like Dell) that source optimal components from around the world and deliver them quickly to customers.
	\item \textbf{Inability to Become Platform Leaders:} The \textbf{Platform Leaders} (e.g., Intel, Microsoft) who set the rules for component combination (industry standards) capture most of the added value. Japanese firms, however, tend to be unable to establish this position, even while possessing advanced component technology.
\end{enumerate}
This structure suggests the risk that even financial institutions with high technological capabilities may fail to adapt to the modular architecture of FinTech, in other words, becoming a '\textbf{second digital appliance maker}.'

\subsubsection{Value Chains and Modularization}
The impact of modularization is analyzed from the perspective of the \textbf{Value Chain}, as proposed by \textbf{Michael Porter}. Business scholars \textbf{Baldwin} and \textbf{Clark} list the following three advantages of modularization:
\begin{itemize}
	\item \textbf{Simplification between subsystems:} Interfaces are standardized, reducing the complexity of the overall system and lowering development costs.
	\item \textbf{Standardization between subsystems:} Standardization allows for component commonality and diverse combinations, reducing design and manufacturing costs.
	\item \textbf{Independence of subsystems:} Each subsystem can independently focus on technological development, stimulating innovation.
\end{itemize}

\subsubsection{Two Major Impacts of Modularization on the Value Chain}
The three advantages above bring two major, potentially contradictory, impacts to the entire value chain.

\paragraph{1. Ease of Integration and Cost Reduction}
As component integration becomes easier, the following phenomena occur:
\begin{itemize}
	\item \textbf{Promotion of specialization among firms:} Specialization by function or component, such as design-specialization (Fabless) or manufacturing-specialization (EMS), progresses.
	\item \textbf{Ease of new entry:} Entry is possible just by procuring components from the market, leading to intensified competition. As a result, differentiation becomes difficult, often leading to \textbf{commoditization} and fierce price wars (excessive competition).
\end{itemize}
In the medium-to-long term, modularization can be a '\textbf{double-edged sword}' that makes it difficult for firms to create added value.

\paragraph{2. Stimulation of Innovation}
The independence of subsystems enables decentralized innovation activities, both inside and outside the firm, as long as the \textbf{Design Rules} (design standards) are observed. This accelerates the \textbf{Open Innovation} proposed by Professor \textbf{Henry Chesbrough}.

Because FinTech is modular, innovation is active, holding the potential to rapidly render the technology of existing financial institutions obsolete.

\subsubsection{The Importance of the Platform Leader}
'Ease of integration (cost reduction)' and 'stimulation of innovation' are inherently in a trade-off relationship. As innovation accelerates, existing design rules become obsolete, requiring their revision and resetting.

The entity that leads this \textbf{setting and revision of design rules} and steers the entire value chain is called the \textbf{Platform Leader} (e.g., Intel, Cisco). They create the most added value in modular industries where innovation is active.

In the FinTech industry as well, the focal point is who will capture this position of platform leader.

\subsection{Application and Case Studies}
\subsubsection{Digital Appliance and PC Industries}
These are representative modular industries where Japanese firms were defeated. In the PC industry, \textbf{Intel} (CPU) and \textbf{Microsoft} (OS) grasped industry standards as platform leaders, while \textbf{Dell} succeeded with its global system (selection and combination capability). Meanwhile, Japanese firms, unable to become platform leaders despite possessing high technology, were engulfed in cost competition.

\subsubsection{DVD Players and Digital Cameras}
For DVD players, over 1000 Chinese companies entered the market merely by procuring and assembling components, leading to excessive competition. In the digital camera market, although \textbf{Matsushita} and \textbf{Sony} supplied the core component (CCD), their competitors could also purchase it, making differentiation in the final product difficult and hindering profit creation.

\subsubsection{Smartphones (Japanese Mobile Phones)}
Japan's conventional mobile phones, 'over-engineered' with an integral-type philosophy, rapidly lost the market with the advent of smartphones (iPhone and Android), which combined modular components. This suggests the possibility that Japanese financial institutions, possessing high-quality, built-from-scratch systems, might follow a similar collapse by hesitating to adopt seemingly simple FinTech (modules).

\subsection{Deeper Context and Lessons}
\textbf{\paragraph{Reference to Chapter 16 (Japanese Firms' Cultural Background and Countermeasures)}}
During the lecture, detailed discussion on the true reasons for Japanese firms' weakness in modular types (cultural backgrounds beyond just cost issues) and specific countermeasures for FinTech (such as the path to becoming a platform leader) was intentionally postponed to 'Chapter 16'.

\textbf{\paragraph{Instructor's Personal View: Recollection of a Japanese Mobile Phone Developer}}
An episode was shared about a Japanese mobile phone developer the instructor once met. He recounted, 'Our company was ableto internalize (integrate) extremely superior component technology, so we hesitated to use cheap, modular components supplied from outside.' As a result, he said, the business itself collapsed with the arrival of smartphones. This serves as a strong warning of a trap that current Japanese financial institutions are susceptible to falling into.

\textbf{\paragraph{Invitation to Next Week (Operations Management)}}
The end of this lecture connected to the 'Operations Management' lectures starting next week. Technological evolution and globalization driven by FinTech are forcing unprecedented operational transformation upon financial institutions. It was suggested that the lectures from next week would offer insights into the trinitarian reform of strategy, operations, and organization.

\textbf{\subsubsection{AI Supplement: Expansion of Key Points}}
The lecture emphasized the importance of the platform leader, but the following supplement regarding their profit mechanism and the role of design rules is beneficial for deepening understanding.

\paragraph{Design Rules and Value Distribution}
The core of modular architecture lies in the \textbf{Design Rules} that govern the entire system. Design rules are divided into 'Visible Rules' and 'Hidden Rules.'
\begin{itemize}
	\item \textbf{Visible Rules:} Information disclosed to module suppliers, such as interface specifications.
	\item \textbf{Hidden Rules:} The core design philosophy of the system and future change plans, known only to the platform leader.
\end{itemize}
The platform leader strategically controls these 'Hidden Rules' and leads architectural changes, thereby gaining an advantage over other companies that provide complementary modules (like apps or peripherals) and enabling the monopolistic capture of value (profit).

\paragraph{Network Externalities}
\textbf{Network Externalities} (Network Effects) are cited as a source of the platform leader's strength. This is the phenomenon where 'the value of a product or service increases as more people use it.'

In a FinTech platform (e.g., a payment system or OS), once a large number of users and complementary services (linked apps, etc.) are acquired, this becomes a barrier to entry (increasing switching costs), further solidifying the leader's position. Therefore, competition in the FinTech industry is not just about technological capability but also has an aspect of a race for speed—how quickly this network externality can be built.

\subsection{Conclusion}
This lecture clarified the threat facing Japanese financial institutions by re-evaluating FinTech as a \textbf{Modular Architecture}. It was shown that possessing high technological capability or high-quality existing systems (Integral-type) could, conversely, delay the response to modularization, risking a repeat of the same failures as the former digital appliance and mobile phone manufacturers.

The practical lesson drawn from this threat is the danger of adhering to existing, 'highly integrated' systems and operations. Japanese financial institutions must recognize that their strengths can transform into weaknesses in the era of modularization. They must seriously consider not just adopting technology (using FinTech), but transforming their value chain, and even adopting strategies to become a \textbf{Platform Leader} or a \textbf{Keystone} (a strategy to function as the linchpin of an ecosystem).

The study of operations management starting next week will be essential for gaining concrete insights into putting this transformation into practice.

\subsection{List of Important Keywords}
\textbf{Names:} Michael Porter, Baldwin, Clark, Henry Chesbrough
\vspace{\baselineskip}

\textbf{Theories/Concepts:} Modular Architecture, Integral Architecture, Selection \& Combination Capability, Integrative Capability (Suriawase), Platform Leader, Value Chain, Open Innovation, EMS (Electronics Manufacturing Services), Fabless, Design Rules, Commoditization, Excessive Competition, Keystone Strategy, Network Externalities

\subsection{Comprehension Quiz}
\begin{enumerate}
	\item What are the two main types of product architecture?
	\item Which architecture have Japanese firms traditionally excelled at?
	\item What organizational capability is necessary for modular-type products?
	\item In this lecture, which architecture was FinTech classified as?
	\item What are the three reasons given for why Japanese firms are weak in modular-type products?
	\item Which business scholar proposed the concept of the Value Chain?
	\item What are the three advantages of modularization proposed by Baldwin and Clark?
	\item What are the two major impacts of modularization on the value chain?
	\item When modularization makes new entry easy, what state does the market tend to fall into?
	\item In a modular-type industry, what is the leading company called that determines the component combination rules or industry standards?
	\item Who is the University of California scholar who proposed the concept of Open Innovation?
	\item What industry did the instructor cite as an example of the danger Japanese financial institutions face with the advance of FinTech?
	\item In a modular industry with active innovation, what is the important role led by the platform leader?
	\item According to the instructor's example of the Japanese mobile phone developer, what causes a company with high-quality internalized products to fail in modularization?
	\item What is one of the strategic behaviors of a platform leader mentioned in the lecture?
\end{enumerate}

\subsubsection*{Answer Key}
1. Integral type and Modular type, 2. Integral type, 3. Selection \& Combination Capability, 4. Modular type, 5. Cost issues, weakness in global system building, and inability to become platform leaders, 6. Michael Porter, 7. Simplification, standardization, and independence of subsystems, 8. Ease of integration/cost reduction, and stimulation of innovation, 9. Excessive competition (or commoditization), 10. Platform Leader, 11. Henry Chesbrough, 12. The digital home appliance industry, 13. Setting and revising design rules, 14. Hesitating to use cheap, modular components from outside, 15. Keystone Strategy
\end{document}
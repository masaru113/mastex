\documentclass[uplatex,a4j,12pt,dvipdfmx]{jsarticle}
\usepackage{amsmath,amsthm,amssymb,bm,color,enumitem,mathrsfs,url,epic,eepic,ascmac,ulem,here,ascmac}
\usepackage[letterpaper,top=2cm,bottom=2cm,left=3cm,right=3cm,marginparwidth=1.75cm]{geometry}
\usepackage[english]{babel}
\usepackage[dvipdfm]{graphicx}
\usepackage[hypertex]{hyperref}
\usepackage{longtable}
\usepackage{booktabs}

\title{オペレーションマネジメント 第14回 講義ノート: \\ ナレッジマネジメントとIT}
\author{Masaru Okada}
\date{\today}

\begin{document}
\maketitle
\tableofcontents

\section{講義資料整理}

\subsection{はじめに}


\subsection{はじめに:知識社会におけるオペレーションの再定義}

\subsubsection{講義の哲学的背景と目的}
ポスト資本主義社会において、企業の競争優位の源泉は、有形資産(Tangible Assets)から無形資産(Intangible Assets)、とりわけ「知識(Knowledge)」へと不可逆的にシフトしている。ピーター・ドラッカーが予見した「知識社会」の到来は、経営管理に対し、従来の科学的管理法(テイラー主義)からの脱却を迫るものである。

本講義の核心的な目的は、ナレッジマネジメント(KM)を単なる「情報の整理整頓」や「ITツールの導入」といった表層的なレベルで捉えるのではなく、\textbf{「組織がいかにして集団的に認識を形成し、新たな意味を創出するか」という認識論的(Epistemological)なプロセス}として再定義することにある。さらに、そのプロセスを加速させる触媒としてのIT(Information Technology)の役割を、社会技術システム(Sociotechnical Systems)の観点から体系化する。

\subsubsection{オペレーション・マネジメントへの統合}
伝統的なオペレーション・マネジメントは、効率性(Efficiency)と品質(Quality)の追求を主眼としてきた。しかし、現代のオペレーションは、環境の変化に適応し、自律的に進化する「学習する組織(Learning Organization)」の基盤でなければならない。

本講義では、ERP、CRM、SFAといった基幹業務システムを、単なるトランザクション処理装置としてではなく、組織の記憶装置であり、かつ知識創造のプラットフォームとして再解釈する。これは、マイケル・ポランニーの「暗黙知」理論や、レイヴとウェンガーの「状況論的学習」を、具体的な業務プロセスにいかに実装するかという実践的な問いへの応答である。



\subsection{主要な概念と論点:理論的枠組みの深化と拡張}

\subsubsection{ナレッジガバナンスとリーダーシップの構造}

知識創造を促進するガバナンス体制は、中央集権的な管理と自律分散的な創発のバランスの上に成り立つ。

\paragraph{CKO (Chief Knowledge Officer) の機能的変遷}
1990年代、多くの企業が\textbf{CKO(最高ナレッジ責任者)}を設置し、知的資本の最大化を試みた。CKOの本質的な役割は、各部門にサイロ化された知識を横串で繋ぎ、組織全体の「知の流動性」を高めることにあった。
しかし現在、CKOの機能は\textbf{CIO (Chief Information Officer)} へと統合されつつある。これは、知識の伝達経路(チャネル)がデジタルインフラと不可分になったことに起因する。現代のCIOには、技術的なアーキテクトとしての能力に加え、組織文化を変革する「チェンジ・エージェント」としての資質が求められる。

\paragraph{専門組織による「知の触媒」機能:エーザイの事例}
製薬大手エーザイの「知創部」は、野中郁次郎の理論を経営の中枢に実装した稀有な事例である。ここでの専門組織の役割は、知識を管理(Manage)することではなく、創発(Create)される環境をデザインすることにある。MR(医薬情報担当者)へのモバイル環境提供は、単なるリモートワーク支援ではなく、顧客接点という「最前線の場」で得られた暗黙知を、即座に組織知へと昇華させるための戦略的インターフェースの整備と解釈すべきである。

\paragraph{トップマネジメントの知識ビジョン}
組織的知識創造は、混沌(Chaos)を伴うプロセスである。トップマネジメントの役割は、この混沌に方向性を与えるための\textbf{「知識ビジョン」}を提示することである。「我々は何者であり、どのような世界を実現したいのか」という存在論的定義こそが、断片的な情報に意味(Semantic)を与え、知識へと結晶化させる核となる。

\subsubsection{「場(Ba)」の現象学とSECIモデルの動的プロセス}

野中理論における「場」は、物理的空間のみならず、関係性が織りなす実存的な空間を指す。これは西田幾多郎の「場所の論理」や現象学的な相互主観性(Intersubjectivity)の概念を経営学に応用したものである。

\paragraph{SECIモデルにおける4つの場のダイナミクス}
知識変換プロセス(SECI)に対応する各「場」は、静的なカテゴリではなく、動的なエネルギーの循環装置として機能する。

\begin{table}[h]
	\centering
	\caption{SECIモデルと対応する「場」の現象学的分析}
	\label{tab:ba_matrix_advanced}
	\begin{tabular}{|l|l|l|l|}
		\hline
		\textbf{プロセス}     & \textbf{場 (Ba)} & \textbf{相互作用の様相} & \textbf{認識論的メカニズムと具体的事象}                \\ \hline
		\textbf{共同化}      & \textbf{創発場}    & 個 vs 個           & \textbf{身体化された共感 (Empathic Embodiment)} \\
		(Socialization)   & (Originating)   & (対面・同期)          & 言語化以前のクオリア(質感)や文脈を共有する。                 \\
		                  &                 &                  & 主客未分の状態で、相手の視点を追体験する。                   \\
		                  &                 &                  & 例:師弟関係における「呼吸」の伝承、MBWA。                 \\ \hline
		\textbf{表出化}      & \textbf{対話場}    & 個 vs 集団          & \textbf{対話による概念化 (Conceptualization)}   \\
		(Externalization) & (Dialoguing)    & (対面・同期)          & 暗黙知をメタファーやアナロジーを用いて切断し、                 \\
		                  &                 &                  & 論理的な形式知へと結晶化させる創造的対立。                   \\
		                  &                 &                  & 例:開発合宿における「創発的カオス」の発生。                  \\ \hline
		\textbf{連結化}      & \textbf{システム場}  & 集団 vs 集団         & \textbf{体系化と編集 (Systematization)}       \\
		(Combination)     & (Systemizing)   & (バーチャル)          & 異質な形式知同士を結合し、新たな意味体系を構築。                \\
		                  &                 &                  & 情報処理的な側面が強く、ITの寄与度が最大化する。               \\
		                  &                 &                  & 例:データウェアハウス統合、経営ダッシュボード。                \\ \hline
		\textbf{内面化}      & \textbf{実践場}    & 個 vs 書物/IT       & \textbf{身体知への還流 (Embodiment)}           \\
		(Internalization) & (Exercising)    & (ハイブリッド)         & 形式知を行動(Action)を通じて自己の身体に                \\
		                  &                 &                  & 再び埋め込み、無意識的なスキルへと沈殿させる。                 \\
		                  &                 &                  & 例:シミュレーション訓練、現場での試行錯誤。                  \\ \hline
	\end{tabular}
\end{table}

\subsubsection{実践共同体(CoP)と状況論的学習:獲得から参加へ}

レイヴとウェンガー(Lave \& Wenger)の「状況論的学習(Situated Learning)」は、学習を「個人の脳内への知識の蓄積(獲得メタファー)」から、「共同体への参加の深化(参加メタファー)」へとパラダイムシフトさせた。

\paragraph{正統的周辺参加(LPP)の権力論的解釈}
\textbf{正統的周辺参加(Legitimate Peripheral Participation)}は、単なる新人教育モデルではない。
\begin{itemize}
	\item \textbf{正統性(Legitimacy):} 周辺的な役割(下働き等)であっても、共同体の再生産に不可欠な要素として承認されていること。これが学習者のアイデンティティを支える。
	\item \textbf{周辺性(Peripherality):} リスクの少ない場所で、熟練者のパフォーマンス全体を観察できる特権的な位置。
	\item \textbf{権力と葛藤:} 参加の軌道(Trajectory)は、古参メンバーとの権力関係や世代間の葛藤を孕みながら、共同体の文化的遺伝子を継承・変異させていくプロセスである。
\end{itemize}

\subsubsection{ITインフラストラクチャの社会技術的進化}

ITは単なる導管(Pipe)ではなく、知識創造の在り方を規定するアーキテクチャである。

\paragraph{ネットワーク・トポロジーの変遷と含意}
\begin{enumerate}
	\item \textbf{閉鎖系(LAN/WAN):} 階層型組織(ヒエラルキー)を強化する中央集権的構造。
	\item \textbf{開放系(Internet/VPN):} 組織境界を透過的にし、「拡張された企業(Extended Enterprise)」を実現。顧客やサプライヤーを知識創造のパートナーとして取り込む。
	\item \textbf{分散系(P2P/Blockchain):} 中央管理者を排した自律分散型ネットワーク。ナレッジワーカー間の直接的な価値交換を可能にし、組織のフラット化を極限まで推し進める。
\end{enumerate}

\paragraph{CSCWとグループウェアの認知的支援}
CSCW (Computer Supported Cooperative Work) は、個人の認知限界を集団的知性で拡張する試みである。ダグラス・エンゲルバートのNLSに端を発するグループウェアは、以下の機能を通じて「集団のIQ」を高める。
\begin{itemize}
	\item \textbf{相互認識(Awareness)の支援:} 「今、誰が何をしているか」という文脈情報を可視化し、阿吽の呼吸をデジタル空間で再現する。
	\item \textbf{組織記憶(Organizational Memory)の外部化:} フロー情報(会話)をストック情報(知識)として蓄積し、忘却を防ぐ。
\end{itemize}

\subsubsection{データ・情報・知識の階層性と存在論的差異}

\paragraph{DIKWモデルに対する批判的検討}
一般に、データ(Data)$\to$ 情報(Information)$\to$ 知識(Knowledge)という階層モデル(DIKWピラミッド)が参照されるが、これには「データが集まれば自動的に知識になる」という誤謬が含まれがちである。

\paragraph{構成主義的アプローチ:文脈の優位性}
グレゴリー・ベイトソンは情報を「差異を生み出す差異(Difference which makes a difference)」と定義した。この定義に従えば、知識とは、無数の物理的信号(データ)の中から「意味のある差異」を識別するための\textbf{「解釈の枠組み(Schema)」}そのものである。
したがって、ナレッジマネジメントにおいては、データを集めること以上に、そのデータを解釈するための「文脈(Context)」や「物語(Narrative)」を組織内でいかに共有するかが本質的な課題となる。



\subsection{応用と事例分析:理論の実践的展開}

\subsubsection{アサヒビールにおける「ワーキング・ナレッジ」の実装}

アサヒビールの事例は、形式知化されたシステムがいかにして現場の暗黙知を活性化させたかを示す好例である。

\paragraph{パラドックスの解消}
「業務効率化(標準化)」と「提案力強化(差別化)」は本来トレードオフの関係にある。しかし、同社は「営業情報玉手箱」を通じて成功事例(Best Practice)を共有することで、標準的な業務レベルを底上げしつつ、余剰リソースを創造的な提案活動(差別化)へ振り向けることに成功した。

\paragraph{「技術部門知恵袋」による形式知の動態化}
技術伝承においては、マニュアル(静的な形式知)だけでは不十分である。同社は、トラブル事例や失敗の教訓を文脈付きでデータベース化することで、形式知に「経験の厚み」を持たせた。これは、過去の知識を現在の課題解決に適用可能な「ワーキング・ナレッジ」へと転換する試みである。

\subsubsection{Linux開発モデル:バザール方式による集合知}

エリック・レイモンドが提唱した「伽藍(Cathedral)とバザール(Bazaar)」の対比は、知識生産様式の転換を示唆している。
\begin{itemize}
	\item \textbf{伽藍方式:} 少数のエリートによる閉鎖的・計画的な知識構築(従来のR\&D)。
	\item \textbf{バザール方式:} 不特定多数の自律的なエージェントによる、並列分散的かつ試行錯誤的な知識進化。
\end{itemize}
このモデルは、イノベーションの源泉が組織内部から外部のネットワーク(エコシステム)へと移行していることを示しており、オープン・イノベーションの先駆的形態と言える。



\subsection{深層背景と教訓:AIによる論点拡張}

\subsubsection{AIによる補足:形式知と暗黙知の相補性と「知の探索・深化」}
講義テキストの行間を埋める重要な視点として、ジェームズ・マーチの「知の探索(Exploration)」と「知の深化(Exploitation)」の概念を導入する。

ITシステムによるナレッジマネジメントは、既存の知識を効率的に再利用・改善するという意味で「知の深化」には極めて有効である。しかし、全く異質な知の結合によるイノベーション(知の探索)には、効率性を度外視した「無駄(Slack)」や「遊び」、そして身体的な相互作用が不可欠である。
現代の企業が直面する課題は、ITによる効率化(深化)を推し進めつつ、同時にいかにして偶発的な出会いやカオス(探索)を組織内に許容するかという「両利きの経営(Ambidexterity)」の実践にある。

\paragraph{精神分析的内面化の功罪}
内面化(Internalization)を防衛機制として捉える視点は、組織文化の強固さと硬直性が表裏一体であることを示唆する。強力なカルチャーを持つ企業(Strong Culture)は、メンバーの同質性が高く統制コストが低い反面、異質な知識に対する免疫反応(拒絶)が強く、環境変化への適応不全(Core Rigidity)に陥るリスクを常にはらんでいる。



\subsection{結論:人間中心主義的ナレッジマネジメントへの回帰}

本講義を通じて明らかになったのは、IT技術が進歩すればするほど、逆説的に「人間」の役割が重要になるという事実である。

\begin{enumerate}
	\item \textbf{技術の限界と人間の領域:} XMLやセマンティックWeb技術がいかに進化しても、最終的な「意味づけ(Sense-making)」を行うのは、歴史的・身体的文脈を持つ人間である。
	\item \textbf{関係性のデザイン:} 優れたKMとは、データベースの構築ではなく、信頼(Trust)と互恵性(Reciprocity)に基づく「場」のデザインである。
	\item \textbf{実践的知恵(Phronesis):} アリストテレスが説いたフロネシス(賢慮)のように、個別具体的な状況において最適な判断を下す能力こそが、AI時代における人間の最終的な防衛線であり、ナレッジマネジメントが目指すべき到達点である。
\end{enumerate}



\subsection{重要キーワード一覧}
\textbf{理論家:} 野中郁次郎、マイケル・ポランニー、ジーン・レイヴ、エティエンヌ・ウェンガー、グレゴリー・ベイトソン、ジェームズ・マーチ、ダグラス・エンゲルバート、アラン・バートン=ジョーンズ、トーマス・ダベンポート

\textbf{概念:} ナレッジマネジメント、SECIモデル(共同化・表出化・連結化・内面化)、場(Ba)、暗黙知/形式知、実践共同体(CoP)、正統的周辺参加(LPP)、状況論的学習、社会技術システム、DIKWモデル、ワーキング・ナレッジ、CSCW、グループウェア、データマイニング/テキストマイニング、セマンティックWeb、両利きの経営、伽藍とバザール

\vspace{\baselineskip}

\subsection{理解度確認クイズ}
以下の問いは、知識の再生ではなく、概念の操作能力と本質的理解を問うものである。

\begin{enumerate}
	\item SECIモデルにおいて、主客未分の身体的相互作用を通じて、言語化不可能なクオリアや文脈を共有するプロセスに対応する「場」の名称を答えよ。
	\item 形式知化された戦略や理論を、実践(Praxis)を通じて自己の身体知として再統合し、無意識的なコンピテンシーへと昇華させるプロセスは何か。
	\item レイヴとウェンガーが提示した、学習者が共同体の周縁から関与し、権力関係やアイデンティティの変容を伴いながら中核的メンバーへと成長する軌道を指す概念は何か。
	\item 認知プロセスを個人の脳内に閉じたものではなく、環境や道具、他者との相互作用の中に分散・偏在するものとして捉える学習理論の名称は何か。
	\item 1960年代にエンゲルバートが提唱し、個人の知能を集団的に拡張するための「コンピュータ支援による協調作業」を指す学際的領域の略称は何か。
	\item 「データは文脈を持たない記号であり、知識は行動を導くための構造化された意味体系である」とする、情報学における標準的な階層モデルの名称は何か。
	\item ダベンポートらが提唱した、アカデミックな真理ではなく、企業のバリューチェーンに埋め込まれ、日々の意思決定において実効性を持つ知識を指す用語は何か。
	\item 大規模な構造化データ(数値等)から、統計的アルゴリズムを用いて未知の相関関係やクラスターを発見する、帰納的な知識発見プロセスを何と呼ぶか。
	\item 自然言語処理技術を用い、非構造化テキストデータから感情、文脈、トピックモデルなどを抽出する、定性情報の定量化技術を何と呼ぶか。
	\item エリック・レイモンドが提示した、Linuxに代表される自律分散的かつ創発的なオープンソース開発モデルを、中央集権的な「伽藍」に対して何と呼ぶか。
	\item 組織内の人的ネットワークを可視化し、「誰が(Who)何を知っているか(Know)」というメタ知識へのアクセスを提供するディレクトリシステムを何と呼ぶか。
	\item 知識資本の最大化をミッションとし、組織文化の変革とITインフラの整備を統合的に指揮する経営幹部職(近年はCIOが兼任傾向)の略称は何か。
	\item 外部の規範や価値体系を自己の自我構造に取り込み、監視がなくとも自律的に規範を遵守するようになる精神分析的・社会化プロセスを何と呼ぶか。
	\item インターネット標準技術(TCP/IP等)を企業内ネットワークに応用し、情報のオープン化と標準化を低コストで実現した技術アーキテクチャは何か。
	\item 公衆網上に暗号化されたトンネルを構築することで、物理的な専用線を用いずにセキュアな拡張ネットワークを実現する技術は何か。
\end{enumerate}

\subsubsection*{解答一覧}
1. 創発場, 2. 内面化, 3. 正統的周辺参加, 4. 状況論的学習, 5. CSCW, 6. DIKWモデル(階層モデル), 7. ワーキング・ナレッジ, 8. データマイニング, 9. テキストマイニング, 10. バザール方式, 11. ナレッジマップ(Know-Whoディレクトリ), 12. CKO, 13. 内面化(Internalization), 14. イントラネット, 15. VPN














\section{序論}


\subsection{はじめに:知識経済社会におけるオペレーションの変容}

本講義では、現代企業の競争力の源泉である「知識(Knowledge)」を、いかにして管理・創造・活用するかという「ナレッジマネジメント(KM)」について論じる。特に、IT(情報技術)が知識創造プロセスにどのように寄与するのか、また、ITだけでは解決できない人間的・社会的側面(場、共同体)がいかに重要であるかを詳説する。

\subsubsection{講義の背景と目的}
かつての産業社会において、経営の三要素は「人・モノ・カネ」であった。しかし、ドラッカーらが予見した通り、現代は「知識」が唯一にして最大の意味を持つ経営資源となる「知識経済社会」である。
オペレーションマネジメントの文脈においても、単なる製造プロセスの効率化(モノの管理)から、現場の知恵や顧客からのフィードバックを製品開発やサービス改善に活かす「知識の循環プロセス(知の管理)」へと焦点がシフトしている。

本講義では、以下の視点を提供することを目的とする。
\begin{enumerate}
	\item \textbf{「場(Ba)」の理論と実践共同体:} 知識が生まれる文脈としての環境設計。
	\item \textbf{ITと人間系の融合:} ERPやCRMなどのシステムを、単なるデータ処理ツールではなく知識創造ツールとして再定義する。
	\item \textbf{オープンネットワーク:} 企業内にとどまらない、Linuxに代表される開放系の知識創造モデルへの展望。
\end{enumerate}

\subsubsection{ITシステムとナレッジマネジメントの関係性}
ナレッジマネジメントの概念は広範であり、企業が導入している多くの基幹系システムも、広義のKMツールとして位置づけることができる。

\begin{itemize}
	\item \textbf{ERP (Enterprise Resource Planning):}
	      企業の「ヒト・モノ・カネ」の情報を統合管理するシステム。KMの観点からは、組織内の「形式知(数値、在庫、財務データ)」を一元化し、経営判断の基盤を提供する役割を果たす。
	\item \textbf{CRM (Customer Relationship Management):}
	      顧客関係管理システム。顧客の購買履歴や属性データだけでなく、コールセンターへの問い合わせ内容やクレームといった「定性的な顧客の声」を蓄積し、組織知化するためのツールである。
	\item \textbf{SFA (Sales Force Automation):}
	      営業支援システム。個々の営業担当者の頭の中に留まりがちな「商談の進捗」「顧客の反応」「成功・失敗事例」を共有し、属人的な営業スキルを組織の標準スキルへと引き上げる役割を担う。
\end{itemize}

重要なのは、「システムを導入すればナレッジマネジメントができる」という技術決定論的な誤解を避けることである。知識という視点から見れば、これらのシステムはあくまで「器」であり、そこにどのような文脈(Context)を持たせるかがマネジメントの課題となる。

\subsection{主要な概念と論点}

\subsubsection{ナレッジマネジメントの推進体制とCKO}
企業がKMを組織的に推進するためには、誰が責任を持つのか、どのような体制を敷くかが初期の重要な論点となる。

\paragraph{CKO(Chief Knowledge Officer:最高知識責任者)の役割}
1990年代後半から2000年代初頭にかけて、KMの導入期に多くの欧米企業で「CKO」という役職が設置された。
\begin{itemize}
	\item \textbf{定義:} 企業における知識資源の価値最大化、知識共有文化の醸成、知的資本の活用に責任を持つ最高責任者。
	\item \textbf{CIOとの兼任問題:} 現代のKMはITインフラと切り離せないため、情報システム部門のトップであるCIO(Chief Information Officer)がCKOを兼任するケースが多い。
	\item \textbf{専任の是非:} CKOの役割は「触媒」である。組織全体に知識共有の文化が定着し、全員がナレッジワーカーとして自律的に動くようになれば、CKOという特定の職位は不要となる。逆に、形だけのCKO設置は、KMを「特定部門の仕事」と矮小化させ、全社運動への発展を阻害するリスクがある。
\end{itemize}

\paragraph{専門部門(機能部門)の設置}
CKOを支える実働部隊として、ナレッジマネジメント推進室や、後述するエーザイの「知創部」のような専門組織が設置されることがある。これは、新たなオペレーション機能としての「知識流通」を担う部門である。

\subsubsection{「場(Ba)」の理論:知識創造のプラットフォーム}
野中郁次郎氏らが提唱した「場」の概念は、知識創造理論の中核を成すものである。

\paragraph{場の定義と本質}
「場」とは、単なる物理的な空間(オフィスや会議室)を指すのではない。\textbf{「人々が文脈(コンテキスト)や意味を共有し、相互作用を行う動的な時空間」}と定義される。
私たちは日常的に「場を読む」「場違い」という言葉を使う。これは、物理的な場所に加えて、その場に流れる空気、人間関係、前提知識といった「見えない文脈」を直感的に理解している証拠である。知識は文脈依存的(Context-specific)であり、適切な「場」がなければ、情報は知識へと昇華しない。

\paragraph{4つの場のタイプとSECIモデル}
知識創造プロセス「SECIモデル」の4段階に対応して、4つの「場」が定義される。これらは独立して存在するのではなく、相互に重なり合い、循環するものである。

\begin{table}[h]
	\centering
	\caption{SECIモデルと4つの場の対応表}
	\begin{tabular}{|l|l|l|l|}
		\hline
		\textbf{SECIプロセス} & \textbf{対応する場}   & \textbf{相互作用の形態} & \textbf{主な特徴と活動例}      \\
		\hline
		\textbf{共同化}      & \textbf{創発場}     & 個人 to 個人         & ・五感を共有する物理的接触          \\
		(Socialization)   & (Originating Ba) & (Face-to-Face)   & ・OJT、顧客訪問、飲みニケーション、合宿  \\
		                  &                  &                  & ・暗黙知の共有、共感、信頼関係の構築     \\
		\hline
		\textbf{表出化}      & \textbf{対話場}     & 個人 to 集団         & ・暗黙知を言語化・概念化する場        \\
		(Externalization) & (Dialoguing Ba)  & (Face-to-Face)   & ・ブレインストーミング、会議、対話      \\
		                  &                  &                  & ・メタファーやアナロジーによるコンセプト創出 \\
		\hline
		\textbf{連結化}      & \textbf{システム場}   & 集団 to 組織         & ・バーチャル空間での形式知の結合       \\
		(Combination)     & (System Ba)      & (Virtual)        & ・データベース、グループウェア、SNS    \\
		                  &                  &                  & ・既存知識の並べ替え、加工、体系化      \\
		\hline
		\textbf{内面化}      & \textbf{実践場}     & 組織 to 個人         & ・形式知を身体知として体得する場       \\
		(Internalization) & (Exercising Ba)  & (On-the-Site)    & ・シミュレーション、ロールプレイング、実務  \\
		                  &                  &                  & ・マニュアルの実践を通じた熟達        \\
		\hline
	\end{tabular}
\end{table}

\begin{itemize}
	\item \textbf{創発場(Originating Ba):}
	      知識創造の出発点。論理よりも感情や経験を共有する場である。トップが現場を歩き回る(MBWA)、あるいは営業担当者が顧客と雑談をする中で、言葉にならないニーズやシーズを発見する。
	\item \textbf{対話場(Dialoguing Ba):}
	      創発場で得たモヤモヤした暗黙知を、集団での対話を通じて「コンセプト」や「言葉」に変換する。ここでは、あえて対立意見を出したり、常識を疑ったりする建設的な葛藤が重要となる。
	\item \textbf{システム場(System Ba):}
	      ITが最も威力を発揮する領域。文書化された知識(形式知)を全社規模で共有し、検索・再利用可能な状態にする。ここでは効率性が重視される。
	\item \textbf{実践場(Exercising Ba):}
	      マニュアルや理論(形式知)を現場で実際に使い、試行錯誤することで、再び個人のスキル(暗黙知)として定着させる。知識が「頭での理解」から「身体での納得」に変わるプロセスである。
\end{itemize}

\subsubsection{実践共同体(Community of Practice: CoP)}
「場」の理論と並んで重要なのが、ジーン・レイヴとエティエンヌ・ウェンガーによる「状況論的学習(Situated Learning)」と「実践共同体」の概念である。

\paragraph{組織図にない学習の場}
企業組織図上の「課」や「部」とは異なり、CoPは特定のテーマや関心事について、実践を通じて学び合う自律的なグループを指す(例:社内のLinux愛好会、若手エンジニアの勉強会など)。
従来の学習観(教室で先生から生徒へ知識を伝達するモデル)に対し、状況論的学習では「社会的な実践活動への参加プロセスそのものが学習である」と捉える。

\paragraph{正統的周辺参加(Legitimate Peripheral Participation: LPP)}
新人がいかにして熟達者(エキスパート)になっていくかのプロセスモデルである。
\begin{enumerate}
	\item \textbf{周辺的参加:} 新人はまず、コピー取り、掃除、会議の議事録作成といった、一見すると本質的ではない「雑用(周辺的業務)」から参加する。しかし、これは共同体の全体像、言葉遣い、暗黙のルールを観察するために必要な「正統的」なステップである。
	\item \textbf{関与の深化:} 先輩の模倣や手伝いを通じて、徐々に中心的な業務へと役割を広げていく。
	\item \textbf{アイデンティティの変容:} 最終的に中核メンバーとなり、次世代を指導する立場になる。
\end{enumerate}
このプロセスにおいて重要なのは、単なるスキル習得だけでなく、「自分はこの共同体の一員である」というアイデンティティの形成が学習の本質であるという点である。

\subsection{応用と事例分析}

\subsubsection{エーザイ株式会社:「知創部」とヒューマン・ヘルスケア(hhc)}
製薬大手のエーザイは、野中理論を全社経営に実装した最も有名な事例の一つである。

\paragraph{戦略的背景:hhc理念の具現化}
エーザイは企業理念として「hhc (human health care)」を掲げている。これは、患者様と生活者の満足度向上を第一義とする考え方である。この理念を実現するため、社員一人ひとりが患者の想いに共感し、新たな知識(新薬やサービス)を創造する必要があった。

\paragraph{具体的な施策とIT活用}
\begin{itemize}
	\item \textbf{知創部の設置:} ナレッジマネジメントを専門に推進する部署を設置し、形式的なIT導入ではなく、知識創造の文化醸成を主導した。
	\item \textbf{MR(医薬情報担当者)のモバイルワーク武装:}
	      MRに対し、早期からノートPC、Wi-Fiルーター、iPadなどのモバイル機器を支給した。
	      \begin{itemize}
		      \item \textbf{目的:} 単なる直行直帰による効率化ではない。「社外にいても社内の知識ベース(システム場)にアクセスできる」こと、そして「現場で得た医師や患者の生の情報を即座に共有する(創発場とシステム場の連結)」ことが狙いである。
		      \item \textbf{機能:} 「顧客価値情報センター」へのアクセス、副作用情報の即時確認などを可能にし、MR個人の知識を組織全体の知能と直結させた。
	      \end{itemize}
	\item \textbf{全社員による共同化の実践:} 研究職や事務職も含め、全社員が就業時間の1\%を患者様とともに過ごす(老人ホームでのボランティア等)というプログラムを実施。これは「創発場」を制度的に強制し、患者の痛みを身体感覚として共有させる(暗黙知の獲得)ための極めてユニークな施策である。
\end{itemize}

\paragraph{成功の要因分析}
エーザイの事例が成功したのは、ITツールを導入したからではない。「患者様の憂慮を知識に変える」という明確な\textbf{「知識ビジョン」}があり、その実現手段として組織(知創部)とIT(モバイル・イントラネット)を有機的に結合させた点にある。

\subsection{深層背景と教訓}

\subsubsection{内面化のメカニズムと影の側面}
講義内で触れられた「内面化(Internalization)」は、単なるスキルの習得以上の心理的な意味合いを持つ。

\paragraph{防衛機制としての内面化}
精神分析理論において、内面化は「防衛機制(Defense Mechanism)」の一種とも解釈される。
子供が親のしつけ(外部からの強制)を守るのは、最初は「怒られるのが怖いから」である。しかし、成長するにつれ、親の価値観を自分の中に取り込み(内面化)、自発的に守るようになる。これにより、外部との葛藤や処罰の恐怖から解放され、精神的な安定を得る。
組織においても同様で、社員が企業文化や規範を内面化することで、監視や罰則がなくても自律的に行動できるようになる。これにより、組織運営のコスト(監視コスト)は劇的に低下する。

\paragraph{内面化の「影」:カルト化のリスク}
一方で、内面化には危険な側面もある。特定の価値観を深く内面化した集団は、その価値観を共有しない外部者に対して「嫌悪」や「排他性」を抱く傾向がある。
\begin{itemize}
	\item \textbf{同質性の罠:} 強すぎる企業文化は、異質な意見や新しいアイデアを「異物」として排除し、イノベーションを阻害する可能性がある。
	\item \textbf{強制の連鎖:} 自分が苦労して内面化した規範を、後輩や他者にも強要しようとする心理(「俺もやったんだからお前もやれ」)が働き、組織が硬直化するリスクがある。
\end{itemize}

\paragraph{山本五十六の言葉とKM}
「やってみせ、言って聞かせて、させてみせ、ほめてやらねば、人は動かず」。この有名な言葉は、SECIモデルとLPPのプロセスを見事に言い当てている。
\begin{itemize}
	\item \textbf{やってみせ:} 共同化(モデリングによる暗黙知の移転)。
	\item \textbf{言って聞かせて:} 表出化(言葉による説明)。
	\item \textbf{させてみせ:} 内面化(実践を通じた体得)。
	\item \textbf{ほめてやらねば:} 承認によるアイデンティティの確立とモチベーション維持。
\end{itemize}
知識移転には、論理的な説明だけでなく、感情的な承認と「やってみせる」という身体的な共有が不可欠であることを示唆している。

\subsubsection{オープンネットワーク型知識創造:Linuxの衝撃}
従来のナレッジマネジメントは、あくまで「企業内」での知識共有を前提としていた。しかし、インターネットの普及は、組織の枠を超えた知識創造を可能にした。

\paragraph{伽藍とバザール(The Cathedral and the Bazaar)}
エリック・レイモンドが提唱したオープンソース開発のモデルである。
\begin{itemize}
	\item \textbf{伽藍(Microsoft Windows等):} 閉鎖的な組織内で、綿密な計画に基づき、選ばれた専門家だけで開発するスタイル。中央集権的で統制されているが、進化の速度は組織のリソースに制限される。
	\item \textbf{バザール(Linux等):} インターネットというオープンな広場で、世界中の雑多な開発者(ボランティア)が自律分散的に開発に参加するスタイル。「バグ出し」や「機能改善」が並列で行われるため、進化の速度が圧倒的に速い。
\end{itemize}

\paragraph{AIによる補足:重要論点の拡張(Web 2.0からWeb 3.0へ)}
講義時点での「オープンネットワーク」の議論を現代的に拡張すると、以下の視点が重要となる。
Linuxの開発プロセスは、現代の「集合知(Collective Intelligence)」や「クラウドソーシング」の先駆けである。
企業はもはや、社内の知識だけで戦うことはできない。P\&Gの「コネクト&デベロップ(C\&D)」戦略のように、社外の技術やアイデアを積極的に取り込む「オープン・イノベーション」こそが、現代のナレッジマネジメントの最前線である。オペレーションマネージャーは、社内の効率化だけでなく、社外のコミュニティといかに接続し、知識を還流させるかを設計する必要がある。

\subsection{結論:知識主導型オペレーションへの転換}

本講義全体を通じて得られる結論は、以下の3点に集約される。

\begin{enumerate}
	\item \textbf{ITは「手段」、ビジョンが「目的」:}
	      高機能なERPやCRMを導入しても、そこに「どのような知識を育てたいか」という意思(知識ビジョン)がなければ、単なるデータ保管庫に終わる。トップのコミットメントと明確な目的設定がすべての出発点である。

	\item \textbf{「場」の多重性を理解する:}
	      リアルな対話(アナログ)とITシステム(デジタル)は対立するものではなく、補完関係にある。エーザイの事例のように、現場での身体的な接触(創発場)と、ITによる広範な共有(システム場)を往復させるハイブリッドな環境設計が求められる。

	\item \textbf{学習の社会的側面を重視する:}
	      人材育成において、マニュアルを読ませるだけの教育は機能しない。新人を「実践共同体」の周辺に参加させ、役割を与え、模倣と実践を通じて徐々に中心へと導く「正統的周辺参加」のプロセスを、意図的にオペレーションに組み込む必要がある。
\end{enumerate}

これからのオペレーションマネジメントは、効率性の追求だけでなく、「組織がいかに賢くなるか」という学習のプロセスを管理する学問へと進化していくのである。

\subsection{重要キーワード一覧}

\textbf{人物名:}
野中郁次郎、マイケル・ポランニー、ジーン・レイヴ、エティエンヌ・ウェンガー、エリック・レイモンド、山本五十六

\textbf{概念・用語:}
ナレッジマネジメント、知識経営、暗黙知、形式知、SECIモデル、場(Ba)、創発場、対話場、システム場、実践場、実践共同体(CoP)、正統的周辺参加(LPP)、状況論的学習、CKO、CIO、ERP、CRM、SFA、知識ビジョン、内面化、防衛機制、オープンソース、Linux、伽藍とバザール、集合知

\vspace{\baselineskip}

\subsection{理解度確認クイズ}

\begin{enumerate}
	\item 組織におけるナレッジマネジメントの最高責任者を指し、しばしばCIOと兼任される役職は何か。
	\item 顧客の属性データだけでなく、コンタクト履歴や定性的な声を管理し、知識として活用するシステムを何と呼ぶか。
	\item 知識創造理論において、主観的で言語化が困難な知識(経験、勘、コツなど)を何と呼ぶか。
	\item 上記の知識に対し、言語や数式、図表などで客観的に表現可能な知識を何と呼ぶか。
	\item 野中郁次郎氏が提唱した、暗黙知と形式知の相互変換による知識創造プロセスモデルの名称は何か。
	\item SECIモデルにおいて、暗黙知から暗黙知へ直接的に知識を移転するプロセス(例:OJT、師弟関係)を何と呼ぶか。
	\item SECIモデルにおいて、暗黙知をメタファーやアナロジーを用いて形式知へと変換するプロセスを何と呼ぶか。
	\item SECIモデルにおいて、形式知と形式知を組み合わせて新たな知識体系を作るプロセス(例:マニュアル作成)を何と呼ぶか。
	\item SECIモデルにおいて、形式知を行動を通じて体得し、再び暗黙知化するプロセスを何と呼ぶか。
	\item 野中理論において、知識が共有・創造されるための「共有された文脈を持つ時空間」を指す日本語の概念は何か。
	\item 物理的な対面接触を重視し、SECIモデルの「共同化」に対応する場のタイプは何か。
	\item ITネットワークなどを活用し、SECIモデルの「連結化」に対応する場のタイプは何か。
	\item ジーン・レイヴらが提唱した、学習を「個人の脳内処理」ではなく「共同体への参加プロセス」と捉える理論は何か。
	\item 新人が共同体の周辺的な雑用から始め、徐々に中心的役割を担うようになるプロセスを指す用語は何か。
	\item 精神分析用語であり、外部の規範を自己の内部に取り込むことで、監視がなくても自律的に行動できるようになる心理メカニズムは何か。
\end{enumerate}

\subsubsection*{解答一覧}
1. CKO(Chief Knowledge Officer)、2. CRM(Customer Relationship Management)、3. 暗黙知、4. 形式知、5. SECIモデル、6. 共同化(Socialization)、7. 表出化(Externalization)、8. 連結化(Combination)、9. 内面化(Internalization)、10. 場(Ba)、11. 創発場、12. システム場、13. 状況論的学習(状況的学習)、14. 正統的周辺参加、15. 内面化(または防衛機制)













\section{ナレッジマネジメントとIT}

\subsection{はじめに:ナレッジマネジメントとITの共進化における技術的決定論の超克}

\subsubsection{講義の背景と認識論的転回}
本講義の核心的テーマは、現代企業経営における「ナレッジマネジメント(Knowledge Management: KM)」と、その不可欠な媒介項である「情報技術(IT)」の相互規定的な関係性の解明にある。

かつて知識経営論は、哲学的・組織論的な抽象概念として語られる傾向にあった。しかし、1990年代以降、KMが単なる認識論的モデルにとどまらず、企業のバリューチェーンに組み込まれた実践的オペレーションとして具現化した背景には、ITの爆発的な進展、すなわち「デジタル・レボリューション」が存在する。
ここで留意すべきは、ITは単なる「道具」ではなく、組織構造や知識のあり方そのものを変容させる「環境」であるという視点である。

\subsubsection{情報インフラストラクチャの確立とネットワーク外部性}
ITを活用したKMが、組織的な知識創造(Knowledge Creation)のための持続可能なエコシステムとして機能するためには、企業内における強固な\textbf{情報インフラストラクチャ(Information Infrastructure)}の確立が必須条件となる。

ここでいうインフラとは、物理的なハードウェアの配備にとどまらない。それは、組織内部のあらゆる部門、さらには組織外部のステークホルダーを有機的に結合し、情報の非対称性を解消する「ネットワーク化」の進展を指す。ネットワーク理論における「メトカーフの法則(ネットワークの価値はユーザー数の二乗に比例する)」が示唆するように、接続性の拡大は、知の結合機会を指数関数的に増大させ、イノベーションの土壌を形成するのである。

\subsection{ネットワーク技術の進化と知識共有圏の拡張}

企業の神経系を構成するネットワーク技術の変遷は、KMが対象としうる「知識の境界線(Knowledge Boundary)」を物理的・地理的制約から解放する歴史であった。

\subsubsection{フェーズ1:物理的制約下の「閉じた」接続(1990年代前半まで)}
この時代のネットワークは、物理的な専用線に依存しており、情報のサイロ化(孤立)が課題であった。

\paragraph{LAN (Local Area Network) と閉鎖的知の共有}
本社や工場など、単一の敷地内(構内)を結ぶネットワーク。高速通信が可能だが、共有範囲は物理的に同居するメンバーに限定され、部門間の壁(Sectionalism)を技術的に超えることは困難であった。

\paragraph{WAN (Wide Area Network) と高コスト構造}
地理的に離れた拠点を結ぶ広域ネットワークだが、当時は通信事業者の高価な専用回線に依存していた。結果として、接続は主要拠点間に限定され、末端の営業所や移動中のナレッジワーカーは情報共有のループから排除されていた。

\subsubsection{フェーズ2:インターネット技術の企業内応用と標準化(1990年代後半)}
インターネット・プロトコル(TCP/IP, HTTP等)の汎用性と低コスト性は、企業ネットワークのパラダイムシフトを引き起こした。

\paragraph{イントラネット (Intranet) による情報の民主化}
インターネット技術を社内ネットワークに応用することで、専用ソフトを排除し、ブラウザという標準インターフェースでの情報アクセスを実現した。これは情報の「閲覧コスト」を劇的に下げ、社内情報のオープン化を加速させた。

\paragraph{エクストラネット (Extranet) とバリューチェーンの結合}
イントラネットを特定の取引先に開放する形態。これにより、SCM(サプライチェーン・マネジメント)における在庫情報の共有など、組織間(Inter-organizational)の知識移転が可能となった。

\subsubsection{フェーズ3:仮想化による「拡張された企業」の実現(2000年代以降)}
ブロードバンドの普及と暗号化技術の進化は、ネットワークの物理的制約を完全に無効化した。

\paragraph{VPN (Virtual Private Network) の戦略的意義}
公衆網(インターネット)上に、暗号化とトンネリング技術を用いて仮想的な専用線を構築する技術。
\begin{itemize}
	\item \textbf{コスト革命と遍在性}: 物理専用線からの脱却により、通信コストが劇的に低下。小規模拠点や海外拠点も安価に統合可能となった。
	\item \textbf{Extended Enterprise(拡張された企業)}: セキュアな接続が容易になったことで、顧客(Customer)やサプライヤー(Supplier)を自社の知識創造プロセスに巻き込むことが可能となった。これは、企業の境界線が曖昧になり、ステークホルダー全体がひとつの知識共同体となることを意味する。
\end{itemize}

\subsection{システムアーキテクチャの変遷:集中から分散、そして自律協調へ}

情報の処理(Processing)と保存(Storage)の配置構造の変化は、ナレッジワーカーの自律性と組織の権限構造に直接的な影響を与える。

\subsubsection{メインフレームからクライアント・サーバ型(C/S型)への移行}

\paragraph{メインフレーム(集中処理)の限界}
すべてのデータと処理能力を中央のホストコンピュータが独占する形態。これは「知の中央集権」であり、現場レベルでの柔軟な分析や創造的活動を阻害する要因となっていた。

\paragraph{クライアント・サーバ型(分散処理)によるエンパワーメント}
データ管理を行う「サーバ」と、ユーザーインターフェースや個別処理を行う「クライアント(PC)」に機能を分散するアーキテクチャ。
\begin{itemize}
	\item \textbf{情報の非中央集権化}: エンドユーザーは手元の高性能PCを用いて、サーバ上のデータに対し自由にアクセス・加工・分析を行う権限を得た。
	\item \textbf{モバイルコンピューティングとSFA}: 無線LANとノートPCの普及は、物理的な場所の制約を撤廃した。営業支援システム(SFA: Sales Force Automation)は、移動時間を知識入力・活用の時間へと転換し、フィールドワークの質を変革した。
\end{itemize}

\subsubsection{P2P (Peer to Peer) :自律分散型知識交換モデル}

サーバを介さず、端末同士が直接通信するP2Pモデルは、組織のヒエラルキーを介さないフラットな知識交換を技術的に具現化するものである。

\begin{table}[H]
	\centering
	\caption{P2Pモデルの分類と組織論的含意}
	\label{tab:p2p}
	\begin{tabular}{|l|p{6cm}|p{6cm}|}
		\hline
		\textbf{分類}         & \textbf{技術的メカニズム}                            & \textbf{組織論的・KM的含意}                       \\
		\hline
		\textbf{ハイブリッドP2P型} & ノードの所在管理(インデックス)のみを中央サーバが行い、データ転送はピア同士で直接行う。 & 中央による「ガバナンス」と現場の「自律性」を両立させる現実的な解。         \\
		\hline
		\textbf{純粋P2P型}     & 検索も含め、サーバを一切介さずにバケツリレー式に探索・通信を行う完全分散モデル。     & 完全な自律分散組織(DAO的構造)。耐障害性は高いが、検索効率や統制に課題が残る。 \\
		\hline
	\end{tabular}
\end{table}

\subsection{CSCWとグループウェア:協働をエンジニアリングする}

ナレッジマネジメント・システムの中核である「グループウェア」は、単なる事務効率化ツールではない。その思想的源流は、コンピュータを介した人間関係の再構築にある。

\subsubsection{CSCW (Computer Supported Cooperative Work) の学際的起源}
CSCWは、「分散環境下におけるコンピュータ支援による協働」と定義される。これは、技術的側面(Computer Supported)と社会的側面(Cooperative Work)を不可分なものとして扱う研究領域である。

\paragraph{D.エンゲルバートと「人間の知性の増幅」}
1968年、ダグラス・エンゲルバート(Douglas Engelbart)は\textbf{NLS (oNLine System)}を発表した。「すべてのデモの母」と称されるこのプレゼンテーションで、彼はマウス、ウィンドウ、ハイパーテキスト、ビデオ会議を披露したが、その真の目的は技術の誇示ではなく、複雑化する社会問題に対処するための「集団的IQの向上(Augmenting Human Intellect)」にあった。現代のKMツールは、彼のビジョンの延長線上にある。

\paragraph{概念の確立}
1980年代、意思決定支援システム(GDSS)の研究と合流し、1984年にアイリーン・グリーフ(I. Greif)とポール・キャッシュマン(P. Cashman)により「CSCW」という用語が定義された。

\subsubsection{グループウェアの統合的機能と知識創造プロセス}
グループウェア(Groupware)は、メール、カレンダー、掲示板等を「統合環境」として提供する。その機能は、野中郁次郎のSECIモデルにおける各プロセスを加速させる。

\paragraph{コンテキストの共有(共同化・表出化支援)}
電子会議室での事前議論は、対面会議の前に「文脈(Context)」と「問題意識」を同期させる。これにより、物理的な会議(Ba)の生産性が飛躍的に向上する。

\paragraph{知の再利用と結合(連結化支援)}
掲示板でのQ\&Aやベストプラクティス・データベースは、個人の経験知を組織の形式知として蓄積・検索可能にする。これは「車輪の再発明」を防ぐだけでなく、異質な知の結合(Combination)を促す。

\paragraph{ワークフローによるプロセスの形式知化}
稟議書等の電子化は、意思決定プロセスそのものを可視化・標準化する。これは業務知識の構造化に寄与する。

\paragraph{Know-Whoディレクトリとトランザクティブ・メモリー}
「誰が何を知っているか」を可視化するナレッジマップは、組織の「トランザクティブ・メモリー(交換記憶)」システムとして機能する。これにより、組織は個人の記憶容量を超えた知識を活用可能となる。

\subsection{知識の蓄積と構造化:リポジトリと意味論}

データが知識へと昇華されるためには、適切な蓄積(Storage)と意味付け(Semantics)が必要となる。

\subsubsection{ナレッジリポジトリ (Knowledge Repository) の構築}
KMの視点から再定義されたデータベース。
\begin{itemize}
	\item \textbf{構造化データ}: 基幹システム上の売上・在庫等の数値データ。
	\item \textbf{非構造化データ}: マニュアル、日報、メール、画像、音声など。企業知識の8割はここに埋蔵されていると言われる。
\end{itemize}
導入初期の課題は「クリティカルマスの確保」にある。コンテンツが空虚なリポジトリは利用されず、利用されないリポジトリには知識が集まらないという負のループを断ち切る必要がある。

\subsubsection{検索技術の高度化とセマンティックWeb}
情報爆発(Information Overload)の中で、「検索(Retrievability)」は死活問題となる。

\paragraph{自然言語検索と文脈解析}
「昨年のA社のトラブル対応資料」といった自然言語による問いかけに対し、構文解析や意味解析を用いて「意図」を理解する技術。これはユーザーのITリテラシー依存度を下げる。

\paragraph{XML (eXtensible Markup Language) によるメタデータ革命}
HTMLが「表示形式(Design)」を記述するのに対し、XMLは「データの意味(Semantics)と構造」を記述するメタ言語である。

\begin{itemize}
	\item \textbf{タグの自由定義}: ユーザーは`<product\_code>`, `<claim\_content>`といった独自のタグを定義できる。
	\item \textbf{機械可読性の向上}: これによりコンピュータはデータを単なる文字列としてではなく、「意味を持った情報」として処理可能となり、検索精度やデータマイニングの質が飛躍的に向上する。
\end{itemize}

\subsubsection{マイニング技術:暗黙のパターンを発見する}
大量のデータから、人間が認知できない法則性を帰納的に発見する技術。

\begin{table}[H]
	\centering
	\caption{データマイニングとテキストマイニングの比較分析}
	\label{tab:mining_detailed}
	\begin{tabular}{|l|p{6cm}|p{6cm}|}
		\hline
		\textbf{比較軸}  & \textbf{データマイニング (Data Mining)}                 & \textbf{テキストマイニング (Text Mining)}            \\
		\hline
		\textbf{対象}   & データウェアハウス(DWH)に格納された「構造化データ」(数値、カテゴリ)。          & コールセンター記録、日報、SNSなどの「非構造化テキストデータ」。           \\
		\hline
		\textbf{手法}   & 統計解析、相関ルール抽出(バスケット分析)、クラスタリング。                  & 自然言語処理(NLP)、形態素解析、センチメント分析、共起ネットワーク。        \\
		\hline
		\textbf{洞察の質} & 「おむつとビール」のような\textbf{定量的相関関係}の発見。仮説検証型および仮説発見型。 & 顧客の感情、潜在的ニーズ、ブランドイメージなどの\textbf{定性的文脈}の可視化。 \\
		\hline
	\end{tabular}
\end{table}

\subsection{深層背景と教訓:技術と組織の相互作用}

\subsubsection{AIによる補足:SECIモデルと「場(Ba)」のデジタル的拡張}

本講義で列挙された各技術は、単独で機能するものではない。これらは野中郁次郎らが提唱した知識創造理論「SECIモデル」の各フェーズを駆動するエンジンとして体系的に理解する必要がある。

\begin{itemize}
	\item \textbf{共同化 (Socialization)}: 物理的共体験が理想だが、Web会議やVRは「仮想的な場」を提供し、暗黙知の移転を支援する。
	\item \textbf{表出化 (Externalization)}: チャットや掲示板での対話、テキストマイニングによる概念抽出は、暗黙知をメタファーや類推を用いて形式知化するプロセスを触発する。
	\item \textbf{連結化 (Combination)}: \textbf{本講義で扱った技術の主戦場}である。DWH、XML、グループウェアのワークフローは、断片的な形式知を体系化・結合し、新たな知識体系を構築する。
	\item \textbf{内面化 (Internalization)}: eラーニングやシミュレーションによる追体験は、形式知を個人の暗黙知として体化する。
\end{itemize}

\paragraph{教訓:メンテナンスという人間的営為}
最後に強調すべきは、いかに高度なXMLやマイニング技術を用いても、知識ベースの「メンテナンス(庭の手入れ)」を怠ればシステムは腐敗するという点である。データのクレンジング、カテゴリの刷新、文脈の補足といった人間による継続的な関与こそが、技術を「知恵」に変える触媒となる。

\subsection{結論}

本講義の結論として、以下の3点を提示する。

\begin{enumerate}
	\item \textbf{インフラの拡張性と戦略的適合性}: VPNやクラウドによるネットワークの遍在化は、企業戦略を「内部資源の最適化」から「外部エコシステムとの共創」へとシフトさせた。
	\item \textbf{協働技術の本質}: グループウェアやCSCWは、業務の自動化(Automation)以上に、人間の相互作用(Interaction)の質と量を増幅させるために設計・運用されるべきである。
	\item \textbf{セマンティクスの重要性}: ビッグデータ時代において競争優位をもたらすのは、データの「量」ではなく、自然言語処理やXML等を通じてデータから抽出された「意味(Context)」の深さである。
\end{enumerate}

\subsection{重要キーワード一覧}

D. エンゲルバート、I. グリーフ、P. キャッシュマン、野中郁次郎

ナレッジマネジメント、情報インフラストラクチャ、LAN/WAN、イントラネット/エクストラネット、VPN(仮想プライベートネットワーク)、クライアント・サーバシステム、SFA、P2P(ハイブリッド型/純粋型)、CSCW、NLS(oNLine System)、GDSS、グループウェア、SECIモデル、ナレッジリポジトリ、自然言語検索、データウェアハウス(DWH)、データマイニング、テキストマイニング、XML(拡張可能なマークアップ言語)、セマンティクス、トランザクティブ・メモリー、ナレッジ・マーケットプレイス

\vspace{\baselineskip}

\subsection{理解度確認クイズ}

\begin{enumerate}
	\item 2000年代以降標準化した、公衆網上にトンネリングと暗号化を用いて仮想的な専用通信路を構築し、企業の境界を超えた知識共有(Extended Enterprise)を可能にする技術は何か。
	\item 「コンピュータ支援による協働」と定義され、技術的側面(CS)と社会的側面(CW)の融合を目指す、グループウェアの理論的・学際的基盤は何と呼ばれるか。
	\item 1968年の「すべてのデモの母」において、マウス、ハイパーテキスト、ビデオ会議などを披露し、「人間の知性の増幅」を提唱したビジョナリーは誰か。
	\item クライアント・サーバ型アーキテクチャの応用例であり、営業担当者がモバイル環境から顧客情報や商談プロセスにアクセスすることで、フィールドワークの知能化を図るシステムは何か。
	\item 中央サーバによる統制を排し(または最小化し)、個々のノードが対等な立場でリソースや知識を直接交換する、自律分散型のネットワークモデルは何か。
	\item グループウェアの機能のうち、稟議書等のドキュメントを電子化し、組織の承認ルール(形式知)に基づいて自動回覧させることで、意思決定プロセスを整流化する機能は何か。
	\item 「何を知っているか(Know-How)」ではなく、「誰がその知識を持っているか(Know-Who)」を可視化し、組織内の専門知へのアクセス性を高めるディレクトリや地図システムは何か。
	\item 組織内に散在する定性的・定量的データを、ナレッジマネジメントの目的のために集約・体系化したデータベースを、「知識の貯蔵庫」という観点から何と呼ぶか。
	\item 検索クエリにおいて、単なるキーワードマッチングを超え、文脈、同義語、ユーザーの意図を解析して結果を返す高度な検索技術を何と呼ぶか。
	\item データマイニングがDWH上の構造化データを扱うのに対し、コールセンターのログや日報などの非構造化テキストデータから、隠れた課題や感情を抽出する分析技術は何か。
	\item HTMLの限界(表示形式の記述)を超え、ユーザーが独自のタグを定義することでデータに「意味(Semantics)」と「構造」を付与し、機械処理の精度を高めるマークアップ言語は何か。
	\item 野中郁次郎のSECIモデルにおいて、IT(特にDWHやワークフロー)が最も寄与するとされる、バラバラな形式知を組み合わせて新たな知識体系を構築するフェーズは何か。
	\item 知識ベースが陳腐化するのを防ぐため、データの正確性維持、不要な情報の削除、分類の更新などを継続的に行う人間的・管理的プロセスを何と呼ぶか。
	\item 電子メール、スケジュール、掲示板などの多様なコミュニケーションツールを、個別のソフトウェアとしてではなく、相互に連携した単一のプラットフォームとして提供する製品カテゴリは何か。
	\item 膨大なデータ集合の中から、統計的手法やAIを用いて、「おむつとビール」のような人間が直感的には気づきにくい相関関係やパターンを発見する知識発見プロセスは何か。
\end{enumerate}

\subsubsection*{解答一覧}
1. VPN, 2. CSCW, 3. D. エンゲルバート, 4. SFA, 5. P2P(ピア・ツー・ピア), 6. ワークフロー, 7. ナレッジマップ(Know-Whoディレクトリ), 8. ナレッジリポジトリ, 9. 自然言語検索, 10. テキストマイニング, 11. XML, 12. 連結化 (Combination), 13. メンテナンス, 14. グループウェア(統合型), 15. データマイニング
















\section{情報・知識}

\subsection{はじめに}

現代の企業経営において、「知識(Knowledge)」は、ヒト・モノ・カネに続く、あるいはそれらを凌駕する第4の経営資源として確固たる地位を築いている。ピーター・ドラッカーが「知識社会」の到来を予見して以来、企業競争力の源泉は物理的な資産から、組織が保有する知的資産へとシフトした。

しかし、多くの組織においてナレッジマネジメント(KM)は誤解され続けている。最も一般的な誤謬は、「データを大量に蓄積すれば、自動的に賢い組織になる」という技術決定論的な思い込みである。本講義では、ナレッジマネジメントの基礎となる「データ」「情報」「知識」という3つの概念の厳密な定義と相互関係を解き明かすことから始める。これらは日常用語としては混同されがちであるが、経営学的には明確に異なる層(レイヤー)として理解する必要がある。

さらに、本講義では理論的な定義にとどまらず、企業活動の現場(オペレーション)において知識がどのように機能すべきかという「ワーキング・ナレッジ(実践知)」の概念を導入する。そして、1990年代後半のアサヒビールにおける大規模なナレッジ共有システムの導入事例を通じて、システムの実装以上に重要となる「組織文化の変革」と「人間中心のアプローチ」について詳細に分析を行う。

\subsection{主要な概念と論点}

\subsubsection{DIKモデル:データ・情報・知識の定義と相互作用}

ナレッジマネジメントを理解するための最も基本的なフレームワークが、Data(データ)、Information(情報)、Knowledge(知識)の3層構造、いわゆる\textbf{DIKモデル}である。これらは静的な積み重ねではなく、動的な相互作用のプロセスとして理解しなければならない。

\paragraph{1. プロセスとしての生成サイクル}
知識が生成されるプロセスは、以下の段階的な変換活動として定義される。

\begin{enumerate}
	\item \textbf{事象(Event / Reality)}:
	      自然界や企業活動の中で発生する、ありのままの事実や現象。この段階ではまだ記録されておらず、混沌とした現実そのものである。

	\item \textbf{データ(Data)への変換}:
	      事象から特定の側面を「抽出(Extraction)」し、記録可能な形式(数値、文字、画像など)で「表現(Representation)」したもの。
	\item \textbf{情報(Information)への昇華}:
	      蓄積されたデータ群から、規則性、傾向、意味のある「パターン(Pattern)」を見つけ出したもの。データ間の関係性が明らかになった状態。
	\item \textbf{知識(Knowledge)への結晶化}:
	      情報を体系化(Systematization)し、構造化することで、新たな状況に適用可能な普遍性を持たせたもの。「なぜそうなるのか」という因果律や論理が含まれる。
\end{enumerate}

\paragraph{2. 知識によるフィードバックループ(知識の先行性)}
講義において強調された重要な視点は、このプロセスが「事象 $\to$ 知識」への一方通行ではないという点である。既存の知識は、下位のレイヤーに対して強力な影響(フィードバック)を与えている。これを\textbf{「観察の理論負荷性」}とも呼ぶことができる。

\begin{itemize}
	\item \textbf{知識 $\to$ データ(埋め込み)}:
	      「何をデータとして記録すべきか」という判断は、観察者の知識に依存する。例えば、熟練の職人は初心者が気づかない微細な音や振動を「異常データ」として認識するが、知識のない者にはそれは単なるノイズ(事象)に過ぎず、データとして記録されない。
	\item \textbf{知識 $\to$ 情報(解釈の枠組み)}:
	      同じデータを見ても、それを受け取る側の知識レベルによって、読み取れる情報の質と量は劇的に変化する。知識は情報を解釈するためのレンズとして機能する。
\end{itemize}

\paragraph{3. バートン=ジョーンズ(Burton-Jones)による定義}
『知識資本主義(Knowledge Capitalism)』の著者であるアラン・バートン=ジョーンズは、経済的価値の観点からこれらを以下のように定義している。

\begin{description}
	\item[データ] \textbf{「信号と合図」}:人間や機械がやり取りする生の記号。これ自体には文脈がなく、意思決定に直接使用することはできない。
	\item[情報] \textbf{「理解可能なデータ」}:データのうち、受け手が解読し理解できるもの。ここで重要なのは、情報の価値は発信者ではなく、\textbf{「受け手の知識レベル」}によって決定されるという相対的な性質を持つ点である。
	\item[知識] \textbf{「二次的な情報・技能」}:情報の受け手が、情報を利用・咀嚼することによって得られた、行動に直結する能力や知見の集合体。
\end{description}



\subsubsection{知識に対する2つのアプローチ:客観主義 vs 構成主義}

知識の本質をどのように捉えるかについて、認識論的に対立する、あるいは補完し合う2つの主要なアプローチが存在する。

\paragraph{【第1の考え方】階層的アプローチ(客観主義的視点)}
データ、情報、知識がピラミッド状の階層構造を成しており、下層から上層へと積み上げられるとする伝統的な情報科学的アプローチである。

\begin{itemize}
	\item \textbf{基本命題}: 知識はデータや情報に対して最上位に位置し、より高度な価値を持つ。
	\item \textbf{データの定義}: 事物や現象を観察・測定した結果(数値、記号)。情報の構成要素であるが、単体では無意味。
	\item \textbf{情報の定義}: データの集合を意味のあるパターンに構成したもの。
	\item \textbf{知識の定義}: 体系化された情報であり、物事の本質的理解。人々の思考やオペレーションを方向づける\textbf{「秩序」や「ルール」}である。
\end{itemize}

\textbf{<第1の考え方の具体例>}
\begin{table}[h]
	\centering
	\begin{tabular}{|l|l|l|l|}
		\hline
		\textbf{分野} & \textbf{データ(素材)} & \textbf{情報(パターン)} & \textbf{知識(体系・ルール)}   \\ \hline
		\textbf{化学} & 元素記号(H, He...)   & 周期表(配列規則)         & 化学理論(反応予測、物質合成の原理)    \\ \hline
		\textbf{音楽} & 音符(ド、レ...)       & メロディ、楽譜           & 音楽理論、演奏家の表現技術         \\ \hline
		\textbf{数学} & 数値(1, 100...)    & 計算結果、数式           & 数学・統計学の定理、証明プロセス      \\ \hline
		\textbf{小売} & POS売上データ         & 売上推移グラフ、他店比較      & 「気温25度超でアイスが売れる」という法則 \\ \hline
	\end{tabular}
\end{table}

このアプローチでは、知識とは「最高気温が25度を超えたら発注を2割増やす」というような、具体的な\textbf{行動ルール(アルゴリズム)}として表現されることが多い。

\paragraph{【第2の考え方】別次元アプローチ(構成主義・文脈主義的視点)}
知識をデータや情報の単なる延長線上にある上位概念としてではなく、人間や社会が介在する\textbf{「全く異なる次元(Dimension)」}のものとして捉えるアプローチである。

\begin{itemize}
	\item \textbf{基本命題}: データや情報は状況から切り離して客観的に流通させることができるが、知識は特定の\textbf{「文脈(コンテキスト)」}に深く依存・埋め込まれており、分離不可能である。
	\item \textbf{主観性と身体性}: 知識は、個人の経験、歴史的経緯、その場の空気感、参加者の思いなどを含んだ、極めて主観的かつ全体的なものである。これは「洞察」や「勘(Intuition)」に近い。
	\item \textbf{事前存在性}: 知識はデータ分析の結果として事後的に生まれるだけでなく、データを解釈する前提として\textbf{「事前に存在」}していなければならない。
\end{itemize}

\textbf{<第2の考え方の具体例:コンビニ発注業務>}
第1の考え方では「過去の売上データ分析 $\to$ 発注」となるが、第2の考え方では以下のようになる。
\begin{quote}
	店長は、空の色や肌に感じる湿度から「今日は蒸し暑くなりそうだ、冷たいものが食べたくなるはずだ」という\textbf{身体的感覚(文脈依存的な知識)}を既に持っている。この「仮説」という知識が先にあり、それを裏付けるために「去年の今日の売上データ」を確認しに行く。
\end{quote}
つまり、データを見るための「枠組み」や「動機」を与えるのが、文脈に埋め込まれた知識である。

\paragraph{統合的理解:ナレッジマネジメントの本質}
講義では、どちらのアプローチが正しいかという二元論を否定している。
\begin{itemize}
	\item 第1の考え方だけでは、データを機械的に処理するだけの自動化システムに過ぎない。
	\item 第2の考え方だけでは、個人の勘に頼った属人化から脱却できない。
\end{itemize}
真のナレッジマネジメントとは、\textbf{「仮説(第2の考え方による主観的知識)」を持って「データ・情報(第1の考え方による客観的検証)」にあたる}という、主観と客観の往復運動の中に存在する。



\subsubsection{ワーキング・ナレッジ(Working Knowledge)}

トーマス・ダベンポート(Thomas H. Davenport)とローレンス・プルサック(Laurence Prusak)は、名著『ワーキング・ナレッジ』において、企業組織における知識の特異性を定義した。

\paragraph{アカデミック・ナレッジとワーキング・ナレッジの断絶}
大学などの教育機関で扱われる知識と、企業経営で求められる知識には、根本的な「目的」と「評価基準」の違いがある。

\begin{table}[h]
	\centering
	\caption{学校知と企業知の比較}
	\begin{tabular}{|l|l|l|}
		\hline
		\textbf{比較軸}  & \textbf{学校教育における知識} & \textbf{企業経営における知識(ワーキング・ナレッジ)}  \\ \hline
		\textbf{目的}   & 知識の獲得・保有そのものが目的     & \textbf{価値創造(Value Creation)}の手段 \\ \hline
		\textbf{評価単位} & 個人の理解度・成績           & 組織全体での成果・業績                      \\ \hline
		\textbf{形態}   & 教科書的、一般的、抽象的        & 実践的、具体的、文脈依存的                    \\ \hline
		\textbf{共有性}  & 個人の頭脳に蓄積されれば良い      & \textbf{共有(Sharing)}されなければ無価値    \\ \hline
	\end{tabular}
\end{table}

\paragraph{オペレーションへの埋め込み(Embeddedness)}
ダベンポートらは、ナレッジマネジメントシステムを業務プロセスから遊離した「図書館」のようなものにしてはならないと警告した。
\begin{itemize}
	\item \textbf{定義}: ワーキング・ナレッジとは、日常の業務オペレーションの中で実際に役に立ち、意思決定の質を高める知識のこと。
	\item \textbf{コンシェルジュの例}: ホテルのコンシェルジュに必要なのは、複雑な数学理論ではなく、「この顧客はどのようなサービスを好むか」「今の表情から何を求めているか」という、顧客対応プロセス(CRM)に即した知識である。
	\item \textbf{一体化と同期化}: ナレッジマネジメントは、企業のバリューチェーン(研究開発、調達、製造、販売、サービス)の各プロセスの中に\textbf{「埋め込まれ(Embedded)」}、業務の流れと同期して自然に活用される状態でなければならない。
\end{itemize}

\subsection{応用と事例分析}

\subsubsection{事例研究:アサヒビールにおけるKMの実践}

本講義では、ワーキング・ナレッジの実装成功例として、1990年代後半から2000年代初頭にかけてのアサヒビールの取り組み(「営業情報玉手箱」と「技術部門知恵袋」)を詳細に分析する。

\paragraph{1. 時代背景と構造的課題}
1987年の「スーパードライ」発売以降、アサヒビールは奇跡的な復活と急成長を遂げた(ドライ戦争)。しかし、1990年代後半にはその急成長自体が新たな経営課題を生み出していた。

\begin{itemize}
	\item \textbf{業務量の爆発的増加}: 売上の急増に伴い、営業現場の負荷が限界に達していた。
	\item \textbf{組織の希薄化とノウハウ断絶}: 業務対応のために大量の新卒・中途採用を行った結果、ベテラン社員が若手を教育する時間的余裕がなくなり、OJT(On-the-Job Training)が機能不全に陥った。これにより、アサヒビール独自の営業ノウハウ(暗黙知)が継承されない危機が生じた。
	\item \textbf{市場環境の複雑化}: かつては酒屋(一般酒販店)へのルートセールスが中心だったが、規制緩和によりコンビニエンスストアや量販店(スーパー)が台頭。単なる御用聞きではなく、棚割り提案や販売促進策を提示する高度な「提案型営業」が求められるようになった。
\end{itemize}

\paragraph{2. 解決策:営業情報玉手箱(1999年導入)}
この課題に対し、アサヒビールはイントラネット上に「営業情報玉手箱」というナレッジ共有システムを構築した。

\begin{itemize}
	\item \textbf{システムの概要}: 全営業担当者が、日々の営業活動で得た成功事例、失敗談、競合情報、提案資料などを自由に登録・検索できるデータベース。
	\item \textbf{モバイルワークの先駆け}: 当時としては先進的であったモバイル端末を活用し、営業担当者が出先からリアルタイムでアクセス可能にした。
	\item \textbf{集合知の活用}: 例えば、ある担当者が担当エリアのスーパーで成功した「夏祭り向けディスプレイ提案」を登録すると、即座に全国の担当者がその資料をダウンロードし、自分の担当店向けにカスタマイズして提案できるようになった。これにより、個人の知恵が瞬時に組織全体の知恵へと拡張された。
\end{itemize}

\paragraph{3. 水平展開:技術部門知恵袋(2000年導入)}
営業部門での成功(=ワーキング・ナレッジの共有による業績向上)を受け、このコンセプトは生産・研究開発部門へと水平展開された。

\begin{itemize}
	\item \textbf{目的のシフト}: 営業部門が「迅速な情報の共有と活用」を重視したのに対し、技術部門では「熟練技術の伝承」と「課題解決力の向上」に重点が置かれた。
	\item \textbf{機能}: 過去の製造トラブルの事例、その原因と対策、実験データ、原理原則に関する知見が蓄積され、新たな技術課題に直面した際、過去の知恵(History)を参照することで解決までのリードタイムを短縮した。
\end{itemize}

\paragraph{4. 成功要因の分析:ITではなく「人」}
本事例の核心は、システム導入そのものではなく、それを支えるマネジメントの哲学にある。当時の経営層(奈良篤氏、福地茂雄氏)のコメントから、以下の成功要因が抽出できる。

\begin{itemize}
	\item \textbf{Give \& Takeの文化醸成}: 「情報は隠して自分だけの武器にする」という旧来の競争意識を排除し、「他者のために情報を提供することが評価される」という組織風土を作り上げた。「自分のために活用し、組織のために提供する」というサイクルの確立である。
	\item \textbf{人間中心主義(Human-Centric)}: 「ITを使いこなすのは人次第」という認識の下、システム構築と同等以上のリソースを「人作り・組織作り」に投じた。システムはあくまで道具であり、主役は現場の社員であるというメッセージを一貫して発信した。
	\item \textbf{業務プロセスへの統合}: 営業日報や提案書作成という日常業務の中にシステム利用を組み込み、わざわざ別システムに入力させる負担感を軽減した(ワーキング・ナレッジ化)。
\end{itemize}

\subsection{深層背景と教訓}

\paragraph{【寄り道トピック】コンテキスト(文脈)というフィルタ}
講義内で触れられた「第2の考え方」における「コンテキスト(文脈)」の重要性は、現代のAI時代においてさらに高まっている。データは事実を表すが、真実を表すとは限らない。
例えば、「顧客満足度が前月比で横ばい」というデータがあるとする。
\begin{itemize}
	\item 文脈A:競合他社が大幅な値下げ攻勢をかけている中での横ばい $\to$ 「顧客ロイヤリティは強固であり、大健闘」と評価できる。
	\item 文脈B:自社が大規模なキャンペーンを行い、市場全体も好況である中での横ばい $\to$ 「施策は失敗であり、危機的状況」と評価される。
\end{itemize}
このように、同じデータでも置かれた文脈(コンテキスト)によって意味が180度変わる。高度なナレッジマネジメントとは、データを集めることではなく、この「文脈を読み解く力(Literacy)」を組織的に養うことである。

\paragraph{【寄り道トピック】ITの「箱」と「魂」}
多くの企業がナレッジマネジメントに失敗する理由は、「仏作って魂入れず」の状態になるからである。「玉手箱」のような立派なデータベース(箱)を作っても、そこに質の高い情報を入力しようとするモチベーション(魂)がなければ、システムはゴミ捨て場と化す。アサヒビールの事例は、人事評価や称賛の文化によって、社員のモチベーションデザインに成功した点が本質的な勝因である。

\subsubsection{AIによる補足:重要論点の拡張(SECIモデル)}
本講義の理解を深めるために、ナレッジマネジメントの世界的権威である野中郁次郎氏の\textbf{「SECI(セキ)モデル」}による補完が不可欠である。講義中の「第1の考え方」と「第2の考え方」は、SECIモデルにおける「形式知」と「暗黙知」に対応する。

\begin{itemize}
	\item \textbf{共同化 (Socialization)}: 暗黙知 $\to$ 暗黙知(共体験)。OJTや飲みニケーションで、言葉にならない勘やコツを移転する。アサヒビールの「OJT不足」はこのプロセスの機能不全であった。
	\item \textbf{表出化 (Externalization)}: 暗黙知 $\to$ 形式知(概念化)。個人のノウハウを「玉手箱」にマニュアルや事例として入力するプロセス。第2の考え方を第1の考え方に変換するフェーズ。
	\item \textbf{連結化 (Combination)}: 形式知 $\to$ 形式知(体系化)。集まったマニュアルを編集・分類し、新たな提案書としてまとめるプロセス。
	\item \textbf{内面化 (Internalization)}: 形式知 $\to$ 暗黙知(体得)。共有されたマニュアルを読んで実践し、自分のスキルとして身につけるプロセス。
\end{itemize}
アサヒビールの事例は、ITシステムによって「表出化」と「連結化」を加速させ、組織的な知識創造スパイラルを回した好例として位置づけられる。

\subsection{結論}

本講義を通じて、ナレッジマネジメント(KM)は単なるITシステムの導入プロジェクトではないことが明らかになった。それは以下の要素を統合した、経営変革の取り組みである。

\begin{enumerate}
	\item \textbf{概念の理解}: データ・情報・知識の階層性と、それらが相互に作用するループ構造を理解すること。特に、知識がデータ解釈の「前提」となる文脈依存性を認識すること。
	\item \textbf{目的の明確化}: アカデミックな知識の蓄積ではなく、企業の価値創造プロセス(SCM、CRMなど)に貢献する「ワーキング・ナレッジ」の創出を目的とすること。
	\item \textbf{人間中心の実装}: システムはあくまで支援ツールであり、知識を生み出し、活用するのは「人」であるという原点に立つこと。組織文化、モチベーション管理、人材育成とセットで設計されなければ、KMは機能しない。
\end{enumerate}

これからのリーダーに求められるのは、客観的なデータ分析能力(第1の考え方)と、文脈を読み解く主観的な洞察力(第2の考え方)の両方を兼ね備え、組織全体の知恵をダイナミックに循環させるファシリテーション能力である。

\subsection{重要キーワード一覧}

\noindent
\textbf{人名}:\\
アラン・バートン=ジョーンズ、トーマス・H・ダベンポート、ローレンス・プルサック、ピーター・ドラッカー、野中郁次郎、奈良篤、福地茂雄

\vspace{\baselineskip}

\noindent
\textbf{理論・概念}:\\
ナレッジマネジメント(KM)、DIKモデル(データ・情報・知識)、ワーキング・ナレッジ、コンテキスト(文脈)、知識資本主義、イントラネット、サプライチェーン・マネジメント(SCM)、バリューチェーン、カスタマー・リレーションシップ・マネジメント(CRM)、SECIモデル、暗黙知、形式知、観察の理論負荷性、ベストプラクティス、OJT(On-the-Job Training)

\subsection{理解度確認クイズ}

\begin{enumerate}
	\item バートン=ジョーンズは、著書『知識資本主義』において「情報」の価値は何によって決定されると定義したか?
	\item データ、情報、知識をピラミッド状の階層構造として捉え、知識を最上位に置くアプローチを一般に何と呼ぶか?
	\item 元素記号を「データ」とした場合、それらを規則的に配列した「周期表」は、DIKモデルにおいて何に分類されるか?
	\item データ単体(例:数値の羅列)が、意味を持つ「情報」になるために必要な、規則性を見出すプロセスを何と呼ぶか?
	\item 「第1の考え方」において、知識とは人々の考えやオペレーションを方向づける何であると定義されるか?
	\item 「第2の考え方」において、知識は客観的に切り離せるものではなく、何に深く依存・埋め込まれているとされるか?
	\item 「第2の考え方」では、知識はデータを抽出・解釈する際に、どのような状態で存在していると考えるか?
	\item ダベンポートとプルサックが提唱した、企業の日常業務や意思決定にとって真に役立つ知識の名称は何か?
	\item 学校教育における知識のゴールが「個人の獲得」であるのに対し、企業経営における知識のゴールは何の創造か?
	\item アサヒビールの事例において、ナレッジマネジメント導入のきっかけとなった、組織急拡大に伴う人事・教育面での課題は何か?
	\item アサヒビールが導入した「営業情報玉手箱」は、社員がどこからでもアクセスできるよう、どのようなネットワーク技術上に構築されたか?
	\item アサヒビールの事例で、営業部門の成功を受けて生産・研究開発部門に水平展開されたシステムの名称は何か?
	\item アサヒビールの経営層(奈良氏、福地氏)が強調した、どんなに優れたシステムがあっても、それを使いこなすために最も重要な要素は何か?
	\item 知識がデータや情報から生成されるだけでなく、逆に知識が事象からデータを取り出す際のフィルターとして機能する関係性を何と呼ぶか?
	\item (応用)野中郁次郎氏が提唱した、個人の暗黙知と組織の形式知が相互に変換されるプロセスモデルの名称は何か?
\end{enumerate}

\subsubsection*{解答一覧}
1. 受け手の知識レベル、2. 第1の考え方(階層的アプローチ/DIKモデル)、3. 情報、4. パターン化(または体系化/構造化)、5. 秩序(またはルール)、6. 文脈(コンテキスト/状況)、7. 事前に存在している(前提となっている)、8. ワーキング・ナレッジ、9. 価値(Value)の創造、10. ノウハウ伝承の希薄化(OJT機能不全)、11. イントラネット(社内ネットワーク/インターネット技術)、12. 技術部門知恵袋、13. 人間と組織(人づくり/組織風土)、14. ループ(相互作用/フィードバック)、15. SECI(セキ)モデル


\end{document}
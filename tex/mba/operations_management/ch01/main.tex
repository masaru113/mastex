\documentclass[uplatex,a4j,12pt,dvipdfmx]{jsarticle}
\usepackage{amsmath,amsthm,amssymb,bm,color,enumitem,mathrsfs,url,epic,eepic,ascmac,ulem,here,ascmac}
\usepackage[letterpaper,top=2cm,bottom=2cm,left=3cm,right=3cm,marginparwidth=1.75cm]{geometry}
\usepackage[english]{babel}
\usepackage[dvipdfm]{graphicx}
\usepackage[hypertex]{hyperref}
\title{\hspace{7mm}Operations Management Lecture 1: Lecture Notes \newline Background and Problem Awareness of this Course}
\author{M. O.}
\date{\today}
\begin{document}
\maketitle
\tableofcontents
\section{Guidance}
\subsection{Introduction}
This course aims to understand the continuous innovation in the field of operations work, driven by the rapid development of information technology and management techniques. The goal is to cultivate the ability to execute 'operations reform', which goes beyond mere operational improvement to build innovative business models by \textbf{strategically utilizing} IT and management methods. This report organizes the basic definition of operations management, the importance of a strategic perspective, and the overall structure of the course as presented in the first lecture.
\subsection{Key Concepts and Arguments}
\subsubsection{Definition of Operations Management in this Course}
In this course, \textbf{Operations Management} is defined as 'the process of a company utilizing information and systems to formulate, execute, evaluate, and improve its business strategy'.
The 'systems' referred to here are not just IT systems, but mechanisms in a broad sense: 'anything that can consistently reproduce reproducible inputs and outputs'.
Whereas traditional operations management tended to focus on the 'How' (the \textbf{means}), this course expands its scope to include the 'What' (the \textbf{purpose}), namely \textbf{strategy} itself. For this reason, the content of this course could be called '\textbf{Strategy Operation Management}', and it strongly emphasizes the inseparable nature of strategy and operations.
\subsubsection{Learning Goal: Perspective of a CXO Candidate}
The learning goal of this course is to acquire the skills to examine directions for solutions and derive the optimal answer when faced with an operational problem. This is the capability required of \textbf{CXO}s (Chief X Officers), the executive leaders who oversee everything from reform planning to execution and results evaluation.
Specifically, students are expected to adopt the perspectives of the following roles:
\begin{itemize}
	\item \textbf{COO (Chief Operating Officer)}: Overseeing execution
	\item \textbf{CSO (Chief Strategy Officer)}: Overseeing strategy
	\item \textbf{CIO (Chief Information Officer)}: Overseeing information strategy
\end{itemize}
From these viewpoints, students will acquire the foundational knowledge necessary for corporate operational reform projects and strategy execution management.
\subsubsection{Central Themes of the Course}
Over the 15 lectures, from within the broad field of operations management, the learning will center on the areas of highest importance in modern business, specifically \textbf{Supply Chain Management (SCM)} and \textbf{Customer Relationship Management (CRM)}.
\subsection{Application and Case Analysis}
\subsubsection{Case: Tokyo Skytree's Strategic Mismatch}
The lecture featured \textbf{Tokyo Skytree} as a symbolic case for understanding the relationship between operations and strategy.
\begin{itemize}
	\item \textbf{Original Purpose (Strategy):}
	      The original purpose of constructing the Tokyo Skytree was to serve as the 'world's tallest radio tower', delivering terrestrial digital broadcast signals throughout the entire Tokyo metropolitan area. This was based on Japan's technological ambition to showcase the high-quality television (hardware) technology of companies like Sony and Panasonic to the world and to build the infrastructure for next-generation broadcast technology. The \textbf{technical operation} of construction and transmission was achieved at the world's highest standard.
	\item \textbf{Current Situation and 'Strategic Miscalculation':}
	      However, while the project was underway, the mainstream 'delivery operation' for viewing television and videos rapidly shifted from broadcast waves to internet streaming. As a result, the Tokyo Skytree ended up with a \textbf{strategic mismatch}: 'the world's tallest radio tower, built in the age of internet streaming'.
	\item \textbf{Analysis:}
	      Today, the Skytree's function as a tourist destination has become its main role, more so than its function as a radio tower. This is a case where, although the construction operation itself was a success, its original strategic value was compromised by misjudging the changes in the market environment (changes in video delivery operations). For this reason, it is sometimes derided as a 'symbol of Japan's decline'. This case strongly suggests how operations must be linked to strategy (purpose).
\end{itemize}
\subsection{Deeper Context and Lessons}
\textbf{\paragraph{Stance as a 'Practitioner-Faculty'}}
The instructor for this course is a '\textbf{practitioner-faculty}' who worked for many years as a practitioner at a consulting firm while simultaneously undergoing academic research training in a doctoral program. This background informs the course's approach, which is to provide a mix of both practical, experience-based insights and academic, theoretical knowledge behind business phenomena.
\textbf{\paragraph{Attitude for the Course}}
The instructor asked students to approach the lectures '\textbf{tensely yet relaxed}'. This is intended to encourage students to maintain a healthy tension regarding the importance of the learning content, while also keeping their minds flexible to actively participate in and contribute to discussions.
\textbf{\paragraph{Reference (Instructor's Book)}}
No specific textbook is assigned for this course, but the instructor's own book (published in 2013, related to 'Theory and Practice of Transformation Management') was mentioned as a reference. This book reflects the fundamental ideas and philosophy on transformation and management that form the basis of this course, and reading it concurrently is recommended.
\textbf{\subsubsection{AI Supplement: Expansion of Key Arguments}}
While the importance of 'strategic operations' was emphasized in this lecture, the note was missing an explicit mention of \textbf{'QCD'}, a traditional and core concept of operations management.
Traditionally, the foundation of operations management has been to optimize the three elements of \textbf{Q (Quality)}, \textbf{C (Cost)}, and \textbf{D (Delivery)} to improve productivity.
Modern strategic operations like SCM and CRM are extensions and developments of this QCD concept. For example, SCM aims to shorten lead times (D) and reduce inventory costs (C) across the entire supply chain, while CRM can be seen as an activity to improve the quality (Q) of the customer experience.
The Tokyo Skytree case can also be analyzed this way: in the project operation of 'constructing a radio tower', the Q (world's tallest height/quality), C (budget), and D (construction period) were likely achieved. The problem, however, was with the \textbf{business operation strategy} of 'delivering radio waves' itself. To understand the 'strategic operations' of this course, it is essential to grasp this traditional QCD perspective and how it connects to strategy in the modern era.
\subsection{Conclusion}
This first lecture provided a perspective that reframes operations management not as mere 'means (How)' execution, but as a management activity inseparable from strategy, the 'purpose (What)'.
The Tokyo Skytree case offers a practical lesson: no matter how high a company's technical operational capabilities may be, if it misjudges the market environment or the business strategy itself, that value will be significantly diminished. We MBA learners must adopt the perspective of consultants and practitioner-faculty (moving back and forth between practice and theory), constantly ask the strategic question 'Operations for what purpose?', and train ourselves to derive optimal solutions from a CXO's viewpoint.
\subsection{Key Keyword List}
Names:
None
\vspace{\baselineskip}
Theories/Concepts:
Operations Management, Practitioner-Faculty, Strategic Utilization, CXO, COO (Chief Operating Officer), CSO (Chief Strategy Officer), CIO (Chief Information Officer), Supply Chain Management (SCM), Customer Relationship Management (CRM), Means (How), Purpose (What), Strategy Operation Management, System (Reproducibility), Tokyo Skytree, Terrestrial Digital Broadcasting, Strategic Mismatch
\subsection{Comprehension Check Quiz}
\begin{enumerate}[label=\arabic*.]
	\item What is the definition of 'Operations Management' in this course?
	\item What is the definition of 'system' presented in the lecture?
	\item Does traditional operations management tend to fall into the trap of focusing on 'How' or 'What'?
	\item What is the reason this course can also be called 'Strategy Operation Management'?
	\item What is the general term for the executive leaders responsible for reform, which is the learning goal of this course?
	\item What is the full title of 'COO' and its Japanese-described role (in parentheses)?
	\item What is the full title of 'CSO' and its Japanese-described role (in parentheses)?
	\item What is the full title of 'CIO' and its Japanese-described role (in parentheses)?
	\item What are the two central management fields for this entire 15-week course?
	\item What are the two contradictory words that describe the attitude the instructor requested from students?
	\item The instructor is a 'practitioner-faculty'; what industry did they work in for a long time?
	\item What was the original construction purpose (strategy) of the 'Tokyo Skytree' case?
	\item Why is the Skytree described as a 'strategic mismatch'? What 'operation' changed?
	\item What Japanese industry (product), exemplified by Sony and Panasonic, was the Skytree meant to symbolize?
	\item (AI Supplement) What are the three letters representing the traditional optimization elements of operations management?
\end{enumerate}
\subsubsection*{Answer Key}
1. A company utilizing information and systems to formulate, execute, evaluate, and improve its business strategy, 2. Something that can consistently reproduce reproducible inputs and outputs, 3. Means (How), 4. Because it includes strategy (What) within the scope of operations management, 5. CXO, 6. Chief Operating Officer (Overseeing execution), 7. Chief Strategy Officer (Overseeing strategy), 8. Chief Information Officer (Overseeing information strategy), 9. Supply Chain Management (SCM) and Customer Relationship Management (CRM), 10. Tense and relaxed, 11. Consulting firm (consulting services), 12. As the world's tallest radio tower, to deliver terrestrial digital broadcast signals everywhere, 13. The delivery (streaming) operation for TV and video changed (from broadcast to internet), 14. Televisions, 15. QCD (Quality, Cost, Delivery)
\section{Background}
\subsection{Introduction}
This report outlines the modern value of Operations Management (OM). Starting from the case of 'Gara-Kei' (Japanese feature phones), where Japanese companies lost their competitiveness by focusing too much on 'features' and 'neglecting operations', we will consider how successful modern companies are innovating operations and creating strategic value by leveraging IT. Furthermore, the objective is to organize the modern leadership theories that are indispensable for achieving such operations reform.
\subsection{Key Concepts and Arguments}
\subsubsection{The Value of Operations: The 'Gara-Kei' Lesson}
In the past, the Japanese mobile phone market (commonly known as '\textbf{Gara-Kei}') pursued multi-functionality to such an excess that it neglected operations (usability) at a level that users could intuitively handle. As a result, with the advent of the \textbf{iPhone} (smartphone), which possessed superior usability, major companies like \textbf{NEC} and \textbf{Panasonic} were forced to withdraw from the market. This fact demonstrates that there is decisive value not only in a product's features, but in the operation (usability, experience) of using it.
\subsubsection{Innovation in Operations via IT}
Modern operations management is inseparable from information technology (IT).
\begin{itemize}
	\item \textbf{Deep Learning:} \textbf{Deep Learning}, utilized by companies like \textbf{Google}, automates the operation of analyzing vast amounts of images and videos on the web to identify the same person or animal. The insights gained from this operation create value directly linked to corporate marketing activities.
	\item \textbf{Next-Generation Vending Machines:} The next-generation vending machines deployed by \textbf{JR East Water Business} determine customer \textbf{demographic attributes} such as gender and age, and recommend products in conjunction with weather information. This is not only a sales operation but also a sophisticated information operation that collects and analyzes marketing information, linking it back to product development.
\end{itemize}
\subsubsection{The Strategic Value of Operating Systems (OS)}
Just as \textbf{Google} came to dominate the \textbf{Operating System (OS)} for smartphones and tablets, the next strategic target is considered to be the 'automotive OS'. Automating the driving operation of a car (autonomous driving) through IT and mastering that OS could mean seizing hegemony in the automotive industry.
\subsubsection{Operations Reform and Leadership}
The ability to conceive and realize such innovative operational introductions and reforms is a core element of modern \textbf{leadership}.
\subsection{Application and Case Analysis}
\subsubsection{Case: Tesla}
\textbf{Tesla}'s sports car is a prime example of automotive operation innovation through IT.
\begin{itemize}
	\item \textbf{Technical Feature:} It adopted a structure, unthought-of by traditional automakers, of lining the chassis with commercially available \textbf{lithium-ion batteries} (a multitude of cells).
	\item \textbf{Core Capability:} Tesla's strength stems from its origins as a software company. Its operational technology, which uses IT (artificial intelligence) to sophisticatedly control and optimize the output of innumerable batteries, achieved both sports car performance and rapid product development.
\end{itemize}
\subsubsection{Case: Toyota Prius}
The hybrid technology represented by the \textbf{Toyota Prius} is another example of operational innovation. The operation of optimally switching between battery (electric) drive and gasoline engine drive could not have been realized without advanced IT control technology. This achieved the value of 'low fuel consumption', which was difficult to attain with conventional cars.
\subsubsection{Case: GE (General Electric)}
\textbf{Jack Welch} is the leader who built \textbf{GE} into one of the world's most powerful corporations. Among the values he introduced at GE (\textbf{The GE Way}), 'Boundaryless' was a particularly emphasized concept.
\begin{itemize}
	\item \textbf{Problem Awareness:} When 'walls (\textbf{Boundary})' exist between executives and the field (vertical) or between departments (horizontal), internal communication deteriorates, information is blocked, and correct decisions cannot be made. Furthermore, when individuals define their job as 'only up to this point', their scope of action narrows, and innovation is stifled.
	\item \textbf{Leadership:} Welch intensely promoted the removal of these 'walls' and collaboration across organizational boundaries. This became the foundation for operations-wide optimization and innovation.
\end{itemize}
\subsection{Deeper Context and Lessons}
\textbf{\paragraph{Leadership and MBA Education}}
The renowned management scholar \textbf{Henry Mintzberg} criticized American MBA education (for its over-emphasis on theory), arguing that 'leaders are not trained in classrooms'. According to him, leadership, like swimming or cycling, cannot be learned without practice (the acquisition of \textbf{tacit knowledge}). From this viewpoint, the instructor notes that the MBA student's style of learning theory while having practical experience answers Mintzberg's criticism and is therefore extremely valuable.
\textbf{\paragraph{The Power to Envision the Future}}
Leaders are required to have the power to imagine future industrial competition. The management scholar \textbf{Gary Hamel} described this as a competition 'to build the best hypotheses about the future and to foresee the way the industry will develop'. \textbf{Steve Jobs}, who disrupted the industry structure with the iPhone, embodied this power.
\textbf{\paragraph{Importance of Core Competence}}
'\textbf{Core Competence}', a concept proposed by Gary Hamel, is also crucial in operations management. It refers to 'the (organizational) capability to sustainably generate competitive advantage'.
\textbf{\paragraph{Leadership that Crosses Boundaries (Jack Welch and Steve Jobs)}}
The 'Boundaryless' concept advocated by Jack Welch shows the importance of leadership that finds solutions by expanding (or removing) the framework itself, rather than continuing to perceive problems within the existing framework.
This philosophy also resonates with the words \textbf{Steve Jobs} spoke at the 2005 Stanford University commencement: '\textbf{Stay Hungry, Stay Foolish.}' It is a view of leadership unique to him, a man who was never satisfied with existing frameworks, doubted common sense, and continuously removed boundaries.
\textbf{\subsubsection{AI Supplement: Expansion of Key Arguments}}
This lecture focused on 'strategic/innovative operations' like Tesla and Google, and organizational leadership like Jack Welch's 'Boundaryless'. However, another crucial point essential to discussing operations at GE during the Welch era is missing from the note: '\textbf{Six Sigma}'.
Simultaneously with the 'Boundaryless' cultural reform, Welch introduced Six Sigma company-wide as a management method to thoroughly pursue operational 'quality' and 'efficiency'. This is a rigorous quality control operation that aims to reduce process defects to approximately $3.4$ per million opportunities, using statistical methods.
The value of operations management is built on both wheels: 'creating new value (innovation)' as seen at Tesla, and 'thoroughly efficient existing processes' as pursued by GE with Six Sigma. While discussing strategic operations (What), the perspective of managing the quality of the execution process (How well) is also indispensable.
\subsection{Conclusion}
This lecture showed that modern operations are not just 'execution' but have merged with IT and are a critical management issue that dictates a company's strategic value itself (e.g., the cause of Gara-Kei's failure, the reason for Tesla's success).
Creating and executing such innovative operations requires powerful leadership that removes existing frameworks. Gary Hamel's 'power to envision the future', Jack Welch's 'Boundaryless' (removing walls), and Steve Jobs' spirit of 'Stay Hungry, Stay Foolish.' all point to the importance of this leadership.
As Henry Mintzberg pointed out, MBA learners must not stop at theoretical learning in the classroom, but must apply these theories in the 'practice' of their work to acquire tacit knowledge.
\subsection{Key Keyword List}
Names:
Steve Jobs, Henry Mintzberg, Gary Hamel, Jack Welch
\vspace{\baselineskip}
Theories/Concepts:
Gara-Kei, iPhone, Deep Learning, Demographic Attributes, Tesla, Lithium-ion Batteries, Operating System (OS), Autonomous Driving, Prius (Hybrid), Leadership, Tacit Knowledge, Core Competence, GE (General Electric), Boundaryless, The GE Way, Stay Hungry, Stay Foolish.
\subsection{Comprehension Check Quiz}
\begin{enumerate}[label=\arabic*.]
	\item What point, neglected by Japan's 'Gara-Kei', was cited as the reason for its defeat by the iPhone?
	\item What technology, modeled on the human brain's workings, does Google use for web image and video analysis?
	\item What is the marketing term for information like gender and age, collected by JR East Water Business's next-gen vending machines?
	\item What did Tesla line the chassis of its sports car with as a power source?
	\item What technological field, in which Tesla was originally strong, underlies its success in rapid car development?
	\item Following smartphones, what 'automotive OS' is Google aiming for hegemony in, and what does it specifically automate?
	\item What is the core value that Toyota's Prius achieved through IT control?
	\item Why did Henry Mintzberg claim 'leaders are not trained in classrooms', and what did he say is essential for learning leadership?
	\item What concept, proposed by Gary Hamel, refers to 'the capability to sustainably generate competitive advantage'?
	\item According to Gary Hamel, what 'of the industry' should a leader have the power to imagine about the future?
	\item What is the English (katakana) term for the organizational 'walls' that Jack Welch detested and tried to eliminate at GE?
	\item What value (slogan) did Jack Welch advocate for to eliminate these 'walls'?
	\item What was the biggest reason Jack Welch saw organizational walls as a problem? What did he think they made impossible?
	\item What are the famous words Steve Jobs spoke at the Stanford University commencement?
	\item (AI Supplement) What statistical quality control method did Jack Welch introduce at GE to improve operational quality and minimize defects?
\end{enumerate}
\subsubsection*{Answer Key}
1. Operations (Usability), 2. Deep Learning, 3. Demographic attributes, 4. Lithium-ion batteries, 5. IT (Software) / AI control technology, 6. Driving (Autonomous driving), 7. Low fuel consumption (efficient power switching), 8. Practice (or tacit knowledge), 9. Core Competence, 10. Competition, 11. Boundary, 12. Boundaryless, 13. Making correct decisions, 14. Stay Hungry, Stay Foolish., 15. Six Sigma
\section{Problem Awareness}
\subsection{Introduction}
This lecture considers the fundamental operational challenges facing Japanese companies, based on the perspective of Harvard Business School's Professor \textbf{Michael Porter}. As early as 1996, Porter pointed out that Japanese companies had reached the limit of growth (the '\textbf{Productivity Frontier}') through operational improvements by 'imitation' (cost leadership), and that a lack of true 'strategy' was the cause of their deteriorating performance. This report uses this 'imitation without strategy' problem as a starting point to organize the changes in global operations, the dilemma of technology orientation, and Japan's unique structural issues, and to argue for the necessity of corporate transformation through operations management.
\subsection{Key Concepts and Arguments}
\subsubsection{Porter's Strategy Critique and the Productivity Frontier}
Michael Porter analyzed that Japanese companies were able to grow simply by imitating other companies (Japanese and American) in an environment of economic growth and global market penetration. However, he pointed out that today, as the economy has matured and operational efficiency has reached its limit—the '\textbf{Productivity Frontier}'—Japanese companies are performing poorly due to an absence of strategy.
The symbolic example of Japanese companies mutually imitating each other to the point that they became indistinguishable is the '\textbf{Galapagos mobile phone}'. This can be interpreted as the dead-end of a '\textbf{Cost Leadership}' strategy unaccompanied by differentiation, as per Porter's theory.
\subsubsection{Changes in Global Operations and Total Cost}
The optimization of a company's production base is a critical strategic decision in operations.
\begin{enumerate}
	\item \textbf{1990s (Overseas Expansion):} The main keyword was '\textbf{labor cost}'. Production bases moved to China and Vietnam in search of overwhelmingly cheap labor.
	\item \textbf{Modern Era (Return to Domestic):} As the labor cost differential shrinks, other costs become apparent. The keyword for competition has shifted to '\textbf{total cost}'. Specifically, this includes \textbf{logistics costs} (e.g., underdeveloped transport infrastructure within China) and \textbf{inventory costs} to compensate for long \textbf{lead times}.
\end{enumerate}
The use of IT is essential in this strategic decision-making. If inventory levels across the entire \textbf{supply chain} can be grasped in real-time, it becomes possible to optimize facility placement and share inventory, which greatly impacts operational results.
\subsubsection{Dilemmas Peculiar to Japanese Companies}
Japanese companies, especially in manufacturing, suffer from structural dilemmas stemming from their very technological prowess.
\begin{itemize}
	\item \textbf{'Galapagos' Phenomenon:} The phenomenon of having advanced technology, but becoming isolated from global standards as a result of over-optimizing for the domestic market (e.g., mobile phones, digital broadcasting).
	\item \textbf{NIH Syndrome (Not Invented Here):} Engineers' obsession with technology developed in-house (or by themselves), showing a tendency to not use superior external technology (which they should use, even if it means paying patent fees), even if it results in high costs.
\end{itemize}
These are the results of a technology-orientation obscuring the view of customers and the market, and can be seen as a form of the '\textbf{Innovator's Dilemma}' proposed by \textbf{Clayton Christensen}.
\subsubsection{Supply and Demand Imbalance}
As the domestic market enters an era of zero growth and 'product surplus' becomes the norm, companies rush to develop new products, but consumers quickly tire of them. The problem is that manufacturers have not fully grasped the true demand. The \textbf{supply chain} has many layers intervening before the consumer; thus, demand information at the end (consumer) becomes amplified as it goes upstream (to the manufacturer), diverging from actual demand. This misreading is the cause of massive dead inventory.
\subsection{Application and Case Analysis}
\subsubsection{Case: Lenovo's Return to Domestic Production}
China's \textbf{Lenovo}, which acquired IBM's PC business, has begun PC production in Japan. It had been conventional wisdom that '\textbf{commoditized}' products like PCs should be produced in Asia. However, this move to return to domestic production shows that the evaluation axis for operations strategy has shifted from 'labor cost' alone to '\textbf{total cost}' (including logistics, inventory, and quality control).
\subsubsection{Case: Toyota's Kanban System and QC Circles}
The lecture mentioned 'improvement' activities like the '\textbf{Toyota Kanban System}' and '\textbf{QC Circles}' as historical strengths that built the superiority of Japanese companies. This was world-class operational excellence that was central to Japanese manufacturing reaching the 'Productivity Frontier'. However, Porter's point is that this operational efficiency (How) is no substitute for strategy (What).
\subsection{Deeper Context and Lessons}
\textbf{\paragraph{Background of the 'Lost 10/20 Years'}}
Underlying the discussion in this lecture is the long-term stagnation of the Japanese economy, known as the '\textbf{Lost 10 or 20 Years}'. Japanese companies, once lauded in 'Japan as Number One', have been overtaken by other Asian companies, and the national debt is more than double that of Italy, a country with default concerns. The problem awareness of how to recover from this serious situation forms the context of this lecture.
\textbf{\paragraph{Crisis in Technology Transfer: The 2007 Problem}}
The '\textbf{Dankai no Sedai}' (baby boomer generation) that supported Japan's technological strength all began retiring around 2007, leading to a serious decline in shop-floor technical skill (the '\textbf{2007 Problem}'). This generation, which had honed its skills in intense competition and supported high economic growth, often 'kept their skills secret to survive' and retired without actively disclosing or passing them on to younger colleagues. This caused a break in technology transfer and chaos on many production floors.
\textbf{\paragraph{Mismatch of Global and Local}}
In modern corporate management, a tension exists between \textbf{Globalization} and \textbf{Localization}. Companies (management) demand that employees adapt to globalization, such as by making English the official company language. Meanwhile, employees and local communities demand that companies contribute to the region (localization). Responding to this mismatch is also a challenge for operations strategy.
\textbf{\subsubsection{AI Supplement: Expansion of Key Arguments}}
The lecture mentioned the phenomenon in the supply chain where demand information becomes more distorted as it goes upstream (to the manufacturer), causing inventory problems. This refers to an extremely important concept in operations management known as the '\textbf{Bullwhip Effect}'.
The Bullwhip Effect is an analogy to the movement of a whip, where a small flick of the wrist (consumer side) results in a large snap at the end (manufacturer side). The traditional solution to this problem is operational coordination, namely \textbf{information sharing across the entire supply chain}. Using IT systems to allow each layer to directly reference actual demand (consumer purchasing information) is the key to suppressing inventory misalignment and wasteful production. The 'strategic use of IT' emphasized in the lecture is precisely what is most effective in controlling this Bullwhip Effect.
\subsection{Conclusion}
This lecture clarified that the root cause of the stagnation facing Japanese companies is not a problem of operational 'efficiency', but an absence of 'strategy' and structural dilemmas. As Michael Porter pointed out, imitation (cost leadership competition) hit a dead end at the 'Productivity Frontier', while overconfidence in technology, as seen in the 'Galapagos Phenomenon' and 'NIH Syndrome', created a divergence from the market.
The '\textbf{2007 Problem}', mentioned in the 'Deeper Context', offers a practical lesson: operational excellence does not consist of 'superior management systems (IT)' alone, but also depends on the extremely human operation of 'technology transfer', including tacit knowledge.
Ultimately, operations management is a tripartite transformation activity: formulating \textbf{business strategy}, creating the optimal \textbf{business process} to realize it, and supporting that process with \textbf{information systems (IT)}. The execution of this transformation is the key that will separate the 'winners and losers' among Japanese companies.
\subsection{Key Keyword List}
Names:
Michael Porter, Clayton Christensen
\vspace{\baselineskip}
Theories/Concepts:
Productivity Frontier, Cost Leadership, Galapagos Phenomenon, Galapagos Mobile Phone, Return to Domestic Production, Total Cost, Supply Chain, Lead Time, NIH Syndrome (Not Invented Here), Innovator's Dilemma, Toyota Kanban System, QC Circles, Supply and Demand Imbalance, Bullwhip Effect (AI Supplement), Globalization, Localization, 2007 Problem, Dankai no Sedai (Baby Boomer Generation)
\subsection{Comprehension Check Quiz}
\begin{enumerate}[label=\arabic*.]
	\item In 1996, Michael Porter criticized Japanese companies for being good at 'imitation' but lacking what?
	\item What is the term Porter used to describe the state where operational efficiency has reached its limit?
	\item What product example was given as symbolic of Porter's critique, where Japanese companies imitated each other until there was no difference?
	\item What was the main motive (keyword) for Japanese companies moving production bases to Asia in the 1990s?
	\item In the recent 'return to domestic production', what is the general term for the costs that have become more important than just labor cost?
	\item What technological element is indispensable for the operation of optimizing global logistics and inventory costs?
	\item What is the phenomenon called where Japanese technology became isolated from global standards by being over-optimized for the domestic market?
	\item What is the syndrome that refers to the tendency to stick to in-house technology and reject superior external technology?
	\item Clayton Christensen's famous theory, which was pointed out as being related to NIH syndrome, is what?
	\item What is the Toyota production method, once symbolic of Japanese corporate 'improvement'?
	\item (AI Supplement) In a supply chain, what is the phenomenon called where demand information is amplified as it goes upstream (to the manufacturer)?
	\item What is the problem, where the mass retirement of the baby boomer generation led to concerns about a decline in shop-floor technical skills?
	\item What is the background reason why the 2007 Problem became serious (i.e., why the baby boomers did not pass on their skills)?
	\item What did the lecture call the conflict between the 'globalization' demanded of employees by companies and the 'localization' demanded of companies by employees?
	\item What is the tripartite corporate transformation activity defined by this course, consisting of strategy, business processes, and information systems?
\end{enumerate}
\subsubsection*{Answer Key}
1. Strategy, 2. Productivity Frontier, 3. Galapagos mobile phone, 4. Labor cost, 5. Total cost, 6. IT (Information Technology), 7. Galapagos Phenomenon, 8. NIH Syndrome (Not Invented Here), 9. The Innovator's Dilemma, 10. Toyota Kanban System, 11. Bullwhip Effect, 12. The 2007 Problem, 13. Because they kept their skills secret (to survive), 14. Mismatch of Global and Local, 15. Operations Management
\end{document}
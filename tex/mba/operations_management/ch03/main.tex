\documentclass[a4paper,12pt]{article}
\usepackage{amsmath,amsthm,amssymb,bm,color,enumitem,mathrsfs,url,epic,eepic,ascmac,ulem,here,ascmac}
\usepackage[letterpaper,top=2cm,bottom=2cm,left=3cm,right=3cm,marginparwidth=1.75cm]{geometry}
\usepackage[english]{babel}
\usepackage[dvipdfmx]{graphicx}
\usepackage[hypertex]{hyperref}
\title{Operations Management Chapter 3 Lecture Notes}
\author{M. O.}
\date{\today}
\begin{document}
\maketitle
\tableofcontents
\section{The Birth and Historical Evolution of Supply Chain Management (SCM)}
\subsection{Introduction}
This lecture aims to provide an overview of the fundamental concepts and historical evolution of \textbf{Supply Chain Management (SCM)}. SCM is a central element of contemporary \textbf{Operations Management}, and understanding its roots leads to a deeper comprehension of modern business strategy. It is a management methodology that has evolved alongside technological advancements in information systems.
\subsection{Definition of Supply Chain Management (SCM)}
\textbf{SCM} is a management methodology for efficiently and effectively supplying products to end consumers by integrally managing the stock and flow of \textbf{goods, services, money, and information} both within and outside the company. This means effectively operating the supply chain---the entire process chain from raw materials to the delivery of products or services to the consumer---by utilizing IT. The implementation of SCM enables reductions in excess inventory and the establishment of appropriate production systems, which is why it is frequently adopted, particularly for products with \textbf{high demand volatility}.
\subsection{Historical Origins of the SCM Concept}
Although the SCM concept has no single definitive origin, the lecture introduced the following prominent theories.
\subsubsection{Michael Porter's Value Chain}
One theory suggests it originates from the \textbf{Value Chain} concept advocated by Harvard University professor Michael Porter. This theory, which theoretically clarified that \textbf{competitive advantage} could exist beyond simple cost reduction, significantly influenced the formation of the SCM concept.
\subsubsection{Toyota's Just-in-Time (JIT) System}
In the 1980s, as the American manufacturing industry struggled, \textbf{Toyota Motor Corporation's} \textbf{Just-in-Time (JIT) system} gained attention and is considered one of the origins of SCM. JIT is based on the principle of 'producing and delivering \textbf{only what is needed, when it is needed, and in the amount needed}' based on information, contributing to reductions in inventory investment and storage space.
\subsection{Industry Trends That Influenced SCM}
Before the name SCM was used, similar concepts had been introduced in specific industries.
\subsubsection{Quick Response (QR): The Apparel Industry}
This concept was introduced in the American apparel industry during a time when it was losing competitiveness due to rapid catch-up from Japanese textile companies. It aimed to shorten the \textbf{time from production to consumer} and enabled supply chain members to collaborate in responding quickly to customer demands.
\subsubsection{Efficient Consumer Response (ECR): The Grocery and Daily Goods Industry}
A concept seen in the grocery and daily goods industry, it is \textbf{almost the same concept} as QR. It aims for the supply chain to cooperate closely to streamline the flow of products and information from production to retail stores. As the word 'response' in both names suggests, a \textbf{rapid response to market changes} is at their core.
\subsection{Potential and Challenges of SCM}
One of the major problems in modern corporate society is the \textbf{mismatch between supply and demand}, which stems from the ineffective functioning of supply and demand planning. At the root of supply and demand planning lies the issue of \textbf{demand forecast accuracy}, and its resolution involves not only technical challenges, such as implementing tools, but also managerial challenges.
However, SCM today involves advanced initiatives in demand forecasting and planning, having already surpassed the initial concept of SCM. It holds the potential to evolve into a management methodology that enables \textbf{proactive management}, even encompassing the \textbf{realization of demand creation}.
\section{The Historical Origins of SCM and A Structural Analysis of the US-Japan Automotive Industries}
\subsection{Introduction}
The clear impetus for \textbf{SCM's} rise in American industry began in January 1992, when President Bush (Sr.), accompanied by former Chrysler president \textbf{Lee Iacocca} and others, visited Japan and identified the Japanese automotive industry's \textbf{'keiretsu' transactions} as a structural impediment. The purpose of this report is to analyze the differences in the US and Japanese production environments that formed the background to this friction, and to extract the origins and lessons of SCM.
\subsection{Key Concepts and Points of Discussion}
\subsubsection{Contrast Between US and Japanese Production Environments and Competitive Principles}
\begin{itemize}
	\item \textbf{Japanese Production Environment}: A \textbf{stable production environment}, characterized by \textbf{stable procurement} over many years and \textbf{levelled production} (heijunka). This was supported by a behavioral pattern of \textbf{avoiding inventory of parts and people} (the source of the \textbf{Just-in-Time} philosophy).
	\item \textbf{American Production Environment}: A \textbf{competitive production environment}, where \textbf{price and quality} were prioritized over stable supply, leading to a combination of \textbf{competitive and dynamic procurement}.
\end{itemize}
\subsubsection{Stagnation of Technological Innovation in a Competitive Environment}
In America, parts manufacturers with only short-term contracts became \textbf{cautious about new investments} due to the risk of orders being terminated at any time. As a result, the \textbf{quality of American automotive parts stagnated}, eventually leading to a situation where they could only produce \textbf{standardized parts}. It has been pointed out that this was an issue where \textbf{switching costs}, as defined in Michael Porter's \textbf{Value Chain}, should have been taken into consideration.
\subsubsection{Establishment of Japanese Behavioral Patterns and Corporate Groups}
Japanese automakers, constrained by post-war \textbf{capital shortages} and the \textbf{1946 labor disputes}, elevated the behavioral pattern of \textbf{avoiding inventory of parts and people} to the point where it dominated management decision-making.
* \textbf{Production System}: They promoted \textbf{high-mix, low-volume production} and thoroughly implemented \textbf{work studies and 'kaizen'} (Operations Management) to achieve both quality and production efficiency.
* \textbf{Supplier Relations}: They built relationships of trust with partner companies over many years, forming a \textbf{pyramid-shaped corporate group} through \textbf{moderate capital participation} and close communication. Within this corporate group, \textbf{information sharing} was conducted extensively, guided by the philosophy of 'viewing employees as an asset to be retained'.
\subsubsection{The Clear Beginning of SCM}
America initially criticized Japan's \textbf{pyramid-shaped corporate community} as a '\textbf{structural impediment}' that violated the principles of a market economy. However, the \textbf{advantages of long-term transactional relationships} (waste reduction, lower investment risk) were later recognized. Based on the understanding that 'a good supply chain should be actively developed', \textbf{SCP (Supply Chain Planning) activities}---marking the clear beginning of \textbf{SCM (Supply Chain Management)}---emerged in American industry.
\subsection{Application and Case Analysis}
\subsubsection{Response of US Automakers}
US automakers, aiming to prevent technology leakage, \textbf{in-sourced} many parts, pushing their in-house production rate as high as 70%. However, within this trend, the \textbf{spin-off} of \textbf{ACG} (later Delphi), a parts group established within General Motors (GM), as a fully public company in 1999 was mentioned as a case suggesting the limits of in-sourcing.
\subsubsection{Iacocca's Rebuilding of Chrysler and Platform Strategy}
The \textbf{platform strategy} developed by Iacocca, sharing common platforms across many models, contributed significantly to \textbf{reducing development costs}. Through this strategy, the \textbf{Chrysler Corporation}, said to be on the brink of bankruptcy, recovered, protecting the jobs of hundreds of thousands of Americans (e.g., the hit minivan 'Caravan').
\subsection{Deeper Context and Lessons}
\subsubsection{Digression: The Dramatic Life of Lee Iacocca}
\textbf{Lee Iacocca} had a \textbf{highly dramatic career}, having been president of Ford Motor Company and later serving as chairman of its rival, Chrysler. At Ford, he led the development of the popular \textbf{Mustang}, but he was dismissed due to conflicts with Henry Ford II and a management style seen as 'blurring public and private affairs'. This suggests a characteristic of American industry where executives are evaluated beyond the scale or position of their company.
\subsubsection{Digression: Organizational Pathology and the Chain of Incompetence}
When Iacocca moved to Chrysler, he described the disastrous state of the organization at the time in his writings. He criticized a situation where 35 vice presidents \textbf{strongly asserted their own turf} and all worked independently without coordination. Furthermore, he offered a sharp insight into \textbf{organizational pathology}, stating, '\textbf{incompetent managers hire incompetent subordinates}', allowing them to 'hide everyone's incompetence in the shadow of the organization's overall weakness'. This provides the lesson that \textbf{organizational management} itself was a major obstacle, even before the introduction of a system like SCM.
\subsubsection{Digression: The Lecturer's Personal View on Competition and Technological Progress}
The lecturer stated a personal view that while the idea that 'competition breeds progress' is simple, this competition-based American system has ceased to function in the \textbf{modern era, where technological progress cycles have become extremely short}. This view, acknowledging that the basis of modern corporate competition is \textbf{QCDF} (Quality, Cost, Delivery, Flexibility), suggests the importance of long-term investment and cooperation.
\subsection{Conclusion}
\textbf{SCM} is a management methodology that inevitably emerged in American industry to resolve a complex set of challenges: the fundamental differences in industrial structure and management philosophy between the US and Japan, and the \textbf{structural organizational dysfunction} pointed out by Iacocca. The lesson from this is that achieving QCDF, the basics of corporate competition, requires not merely relying on market principles, but also \textbf{long-term relationships of trust}, \textbf{appropriate incentive design}, and organizational reform driven by \textbf{strong leadership}.
\section{Analysis of the Evolution of Supply Chain Management (SCM)}
\subsection{Introduction}
This report aims to analyze the primary concepts, generational characteristics, and specific application examples of Supply Chain Management (\textbf{SCM}), based on lecture content regarding its development and evolution. \textbf{SCM} is a management methodology for optimizing the entire process from product planning and development to customer service delivery, and its importance has grown, especially in response to recent globalization and market changes. This paper discusses the historical background and modern lessons of SCM, focusing on the \textbf{three-generation theory of SCM} presented in the lecture.
\subsection{Key Concepts and Points of Discussion}
This lecture explains the evolution of \textbf{SCM} by dividing it into three generations. This illustrates the history of its management scope expanding from the flow of goods to inter-firm networks.
\begin{enumerate}[label=(\Roman*)]
	\item \textbf{First Generation: Management of Goods Stock and Flow}
	      \begin{itemize}
		      \item \textbf{Management Scope}: Primarily \textbf{inventory} and \textbf{logistics}.
		      \item \textbf{Concept}: Utilizing information to manage and share the \textbf{stock and flow of goods} (logistics) between companies.
		      \item \textbf{Characteristics}: Centered on \textbf{information integration} between existing operational functions, assuming the processes themselves remain constant. Generally referred to as \textbf{Supply Chain Logistics}.
	      \end{itemize}
	\item \textbf{Second Generation: Business Process Re-engineering (BPR)}
	      \begin{itemize}
		      \item \textbf{Management Scope}: The \textbf{business process} as a series of activities that provide value to the customer.
		      \item \textbf{Concept}: Reviewing and \textbf{reforming} the existing processes themselves (\textbf{BPR}: Business Process Re-engineering).
		      \item \textbf{Characteristics}: Aims to improve the efficiency of internal company processes and strengthen market responsiveness.
	      \end{itemize}
	\item \textbf{Third Generation: Management of Inter-firm Networks}
	      \begin{itemize}
		      \item \textbf{Management Scope}: The \textbf{inter-firm network} itself.
		      \item \textbf{Concept}: Concentrating on \textbf{core business processes} and \textbf{dynamically reconfiguring} the external inter-firm network for other functions through \textbf{outsourcing}.
		      \item \textbf{Characteristics}: While based on long-term continuous relationships, partners are flexibly reviewed as needed.
	      \end{itemize}
\end{enumerate}
---
\subsection{Application and Case Analysis}
\subsubsection{Benetton's 'Post-Dyeing' Strategy (Second Generation)}
The Italian apparel company \textbf{Benetton} was introduced as a representative example of \textbf{Second Generation SCM}.
\begin{itemize}
	\item \textbf{Traditional Process}: Dyeing the yarn, then knitting/sewing. This involved long lead times (six months for typical apparel, formerly over a year).
	\item \textbf{Benetton's Reform (Post-Dyeing)}: They developed a technique to dye \textbf{after knitting (sewing)}. This allowed them to stock \textbf{undyed fabric (greige goods)} and dye them in the required colors in \textbf{small lots} according to retail sales trends.
	\item \textbf{Result}: \textbf{Lead times} were drastically reduced (shortened to one week), enabling a rapid response to \textbf{demand fluctuations by color}. This suppressed excess inventory and opportunity losses, realizing an SCM based on a \textbf{marketing-oriented mindset} that meticulously meets market needs.
\end{itemize}
\subsubsection{EMS in the Electronics Industry (Third Generation)}
\textbf{Outsourcing} to \textbf{EMS} (\textbf{Electronics Manufacturing Service}) providers in the electronics industry is an example of \textbf{Third Generation SCM}.
\begin{itemize}
	\item \textbf{Strategy}: Companies concentrate their management resources on \textbf{high-value-added} core functions such as \textbf{product development} and \textbf{marketing}, while entrusting \textbf{manufacturing functions} to \textbf{EMS} firms.
	\item \textbf{Underlying Theory}: The \textbf{Smile Curve theory} lies behind this. It indicates that processes at both ends of the supply chain, such as product development and sales/service, have \textbf{high added value}, while processes like \textbf{equipment manufacturing and assembly} have \textbf{low profit margins} (low added value).
	\item \textbf{Application}: Outsourcing to \textbf{EMS} is a strategy to concentrate the company's profit structure on both ends of the \textbf{Smile Curve} by entrusting these low-value-added manufacturing processes to external parties.
\end{itemize}
\subsubsection{The Evolution of Fast Fashion (Generational Adaptation)}
The \textbf{Fast Fashion} industry is considered an example that has responded most progressively to the evolution of \textbf{SCM}, with each company building a supply chain linked to its unique \textbf{marketing strategy}.
\begin{itemize}
	\item \textbf{ZARA}: Their strength lies in a supply chain that quickly catches and responds with \textbf{trendy designs} for standard fabrics, such as \textbf{monotones}.
	\item \textbf{H\&M}: It was suggested that they employ a strategy of introducing \textbf{printed patterns} to the market and deciding on \textbf{mass production only after gauging sales performance}.
\end{itemize}
---
\subsection{Deeper Context and Lessons}
\paragraph{Origins and Implications of the Smile Curve Theory}
The \textbf{Smile Curve} theory was advocated by Stan Shih, the founder of Taiwan's \textbf{Acer Inc.}, while describing the value-added characteristics of the personal computer manufacturing process. This theory is often used to describe the profit structure of industries like \textbf{electronics}, illustrating the low profitability of manufacturing processes. This theory is also linked to the history of the American auto industry's struggles with \textbf{in-sourcing}, strongly suggesting the importance of \textbf{focus and selection} in \textbf{Operations Management}.
\paragraph{The Historical Risk Structure of the Apparel Industry}
The apparel industry is characterized by being heavily influenced by \textbf{fashion trends} and having \textbf{short product life cycles}. Despite this, the traditional supply chain was structured with \textbf{long lead times} (over six months) and was forced to produce in \textbf{large lots} for \textbf{cost reduction}. This combination of \textbf{large lot sizes and long lead times} was the biggest factor causing \textbf{excess inventory} and \textbf{opportunity losses}, and it was argued that this structure made the apparel industry an \textbf{extremely high-risk industry}. Benetton's \textbf{post-dyeing} can be understood as a challenge to this structural risk.
\paragraph{Digression: The Origins and History of SCM}
In the lecture's summary, it was suggested that the birth and history of \textbf{SCM} (the nominal SCM) arose from the conflict stemming from the \textbf{trade friction between the American and Japanese automotive industries}. Furthermore, the connection was shown that \textbf{Michael Porter's} \textbf{Value Chain} discussion formed part of this background. This indicates that \textbf{SCM} is not merely a discussion of logistics efficiency, but is closely related to higher-level \textbf{management strategy theory}, born from international competition and corporate strategy.
\subsubsection{AI Supplement: Extension of Key Arguments}
The lecture notes mention the 'trade friction between the American and Japanese automotive industries' as background to the birth and development of \textbf{SCM}. However, this context lacks mention of the extremely important role that the \textbf{Toyota Production System (TPS)}, or its core concept \textbf{Just-in-Time (JIT)}, played in the transition to \textbf{First Generation SCM} and, by extension, in the development of \textbf{SCM} philosophy.
\paragraph{The Influence of Just-in-Time (JIT) on SCM}
\textbf{JIT} is a philosophy of producing and supplying only what is needed, when it is needed, and in the amount needed. It thoroughly pursues a departure from traditional \textbf{large-lot production}, namely the \textbf{minimization of goods stock (inventory)} and the \textbf{optimization of flow (logistics)}.
In light of \textbf{First Generation SCM} being defined as 'management of goods stock and flow', it can be said that \textbf{JIT} provided the operational foundation and \textbf{philosophical root} for achieving the initial objective of \textbf{SCM}: the \textbf{inter-firm integration of logistics}. Specifically, realizing \textbf{JIT} requires close \textbf{information linkage} and \textbf{synchronized production} with suppliers, including parts manufacturers. This is the very essence of \textbf{Supply Chain Logistics}, the initial form of \textbf{SCM}.
---
\subsection{Conclusion}
This report has summarized the evolutionary process of \textbf{SCM}, from the integration of inventory and logistics (First Generation), to the reform of business processes (Second Generation), and finally to the dynamic reconfiguration of inter-firm networks (Third Generation).
\textbf{Benetton's post-dyeing} is a successful example of \textbf{marketing-driven SCM} that meets market \textbf{lead time} demands through \textbf{BPR}. \textbf{EMS} functions as a strategic example of \textbf{Third Generation SCM}, concentrating on high-value-added core functions based on the \textbf{Smile Curve theory}.
The \textbf{practical lesson} derived from these cases and the lecture content converges on the point that \textbf{SCM} is not merely a technique for operational efficiency, but is management strategy itself, designed to respond to market demands and changes in the competitive environment. In particular, the evolution of \textbf{SCM} culminates in the ideology of strategic outsourcing (Third Generation): redefining a company's \textbf{core competencies}, concentrating on them, and entrusting low-value-added processes to the external network. As an implication for learning, when considering \textbf{SCM}, a perspective that integrates \textbf{customer needs (marketing)}, the \textbf{value-added structure the company provides (Smile Curve)}, and the \textbf{global competitive environment} is essential.
\end{document}
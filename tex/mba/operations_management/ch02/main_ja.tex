\documentclass[uplatex,a4j,12pt,dvipdfmx]{jsarticle}
\usepackage{amsmath,amsthm,amssymb,bm,color,enumitem,mathrsfs,url,epic,eepic,ascmac,ulem,here,ascmac}
\usepackage[letterpaper,top=2cm,bottom=2cm,left=3cm,right=3cm,marginparwidth=1.75cm]{geometry}
\usepackage[english]{babel}
\usepackage[dvipdfm]{graphicx}
\usepackage[hypertex]{hyperref}
\title{オペレーションズ・マネジメント第2回 講義ノート\newline IT概念と業務改革}
\author{M. O.}
\date{\today}

\begin{document}
\maketitle
\tableofcontents

\section{IT}

\subsection{はじめに}
本講義ノートの目的は、情報システムの歴史的な進化の軌跡を概観し、企業経営における$\textbf{IT (情報技術)}$活用の歴史的変遷を振り返ることにある。
これにより、現代のオペレーション改革の概念が、ITの進化と不可分に融合している実態を理解する基盤を構築する。

\subsection{主要な概念と論点}
本講義では、オペレーションを支えるITの基本的な概念と、その進化を駆動した原理について解説がなされた。

\subsubsection{IT (情報技術) の定義}
$\textbf{情報}$は、「ヒト・モノ・カネ」と並ぶ第4の経営資源として位置づけられる。この情報を管理・活用するための基盤ツールがコンピュータおよび情報システムである。
\begin{itemize}
	\item \textbf{IT (Information Technology)}: 情報に関わる技術の総称。経営にとっては戦略実現の武器であり、重要な投資管理対象となる。
	\item \textbf{ICT (Information and Communication Technology)}: 情報技術に「通信 (Communication)」の側面を含めた概念。本講義では、これらを総称して「IT」と呼ぶ。
	\item \textbf{ITの本質}: ITは、人間の能力の限界、特に「記憶」と「計算」の限界を補うものである。人間の記憶の曖昧さやバイアス、計算の速度限界やミスといった弱点を補完し、オペレーションの再現性や効率性を担保する役割を持つ。
\end{itemize}

\subsubsection{コンピュータの原理と基本機能}
コンピュータは、以下の5つの機能的構成要素から成り立つ。
\begin{enumerate}
	\item 中央演算装置 (CPU)
	\item \textbf{主記憶装置}
	\item 外部記憶装置
	\item 入力装置
	\item 出力装置
\end{enumerate}
データはすべて「0」または「1」の$\textbf{ビット (bit)}$で表現され、8ビットを1単位として$\textbf{バイト (byte)}$と呼ぶ。日本語の漢字などは2バイト文字(16ビット)で表現されることが多い。

また、コンピュータの基本機能は以下の3つに大別される。
\begin{description}
	\item[情報処理] プログラムに基づき、高速かつ正確な計算・演算処理を実行する機能。オペレーションにおける「高性能なソロバンや電卓」に相当する。
	\item[情報管理] 記憶装置により、情報や知識を大量に蓄積し、必要に応じて引き出す機能。「効率的な台帳やファイル」に相当する。
	\item[情報交換] 通信技術と組み合わさり、地理的・時間的制約を超えて情報交換を行う機能。「瞬時に送受信できる伝票や手紙」に相当する。
\end{description}

\subsubsection{コンピュータの進化とムーアの法則}
コンピュータの進化は、構成部品の技術革新によって牽引されてきた。$\textbf{ENIAC}$(後述)で用いられた真空管から、トランジスタ、$\textbf{IC (集積回路)}$、$\textbf{LSI (大規模集積回路)}$、$\textbf{VLSI (超LSI)}$へと高密度化が進んだ。

この進化を象徴するのが、インテル創業者$\textbf{ムーア}$が提唱した$\textbf{ムーアの法則 (Moore's Law)}$である。これは「半導体の集積密度は18ヶ月から24ヶ月で倍増する」という経験則であり、コンピュータ性能の指数関数的な向上を表現するものとして今日まで参照されている。

\subsection{応用と事例分析}
講義では、ITがオペレーションに適用されてきた歴史的な経緯が、具体的な計算機の事例と共に示された。

\subsubsection{計算機械の起源}
コンピュータの起源は17世紀の$\textbf{パスカル}$や$\textbf{ライプニッツ}$による機械式計算機の研究に遡る。19世紀には$\textbf{バベッジ}$が$\textbf{解析エンジン}$を考案したが、当時の技術的制約で実用化には至らなかった。
実用化された最初の計算機は、19世紀末に登場した$\textbf{パンチカードシステム (PCS)}$である。これはアメリカの国勢調査や企業の経理処理に利用され、この時点で既にオペレーションに活用されていた。

\subsubsection{電子コンピュータの登場と汎用コンピュータ時代}
第二次世界大戦中、電子技術の応用によりデジタルコンピュータの原型が誕生した。
\begin{itemize}
	\item \textbf{ENIAC (エニアック)}: 1946年にペンシルバニア大学で開発された最初の電子コンピュータ。真空管が用いられ、非常に大規模なものであった。戦後、科学技術用および事務用として普及が始まった。
	\item \textbf{汎用コンピュータ (メインフレーム)}: 企業が大型のコンピュータを導入し、共有利用する時代が到来した。
	      \begin{itemize}
		      \item \textbf{バッチ処理}: 当初の利用形態。プログラムやデータを磁気テープやパンチカードで投入し、一定時間ごとに一括で処理を実行した。
		      \item \textbf{タイムシェアリングシステム (TSS)}: コンピュータ資源を時間分割し、複数の端末から同時利用を可能にする方式。
		      \item \textbf{IBM S/360 (System/360)}: 1964年にIBMが発表。現在のコンピュータ技術形態の基盤を確立したとされる重要なシステムである。
	      \end{itemize}
	\item \textbf{ミニコンピュータ・スーパーコンピュータ}: その後、技術計算用のミニコンピュータ($\textbf{DEC}$など)や、高性能計算用のスーパーコンピュータ($\textbf{Cray Research}$など。近年の例では日本の「京コンピュータ」が挙げられる)が登場したが、TSSという利用形態の点では大きな差異はなかった。
\end{itemize}

\subsection{深層背景と教訓}
講義の本論に加え、文脈を深めるための周辺情報やエピソードが紹介された。

\textbf{\paragraph{本論から逸れた寄り道トピック名:猫型ロボットによるITの比喩}}
講義では、ITの本質を理解する比喩として、著名な猫型ロボットが挙げられた。能力が十分でない小学生(人間)のオペレーション(夢の実現や日常生活)を、未来の技術(IT)が補助し、記憶や計算の弱点を補う存在として描かれている点が、ITの役割と共通するとの解説があった。

\textbf{\paragraph{本論から逸れた寄り道トピック名:バベッジの解析エンジンの後日談}}
19世紀のバベッジの解析エンジンは、当時は実用化されなかったが、設計図は残されていた。後日、その設計図通りに機械を製作したところ、意図した通りに動作したことが確認された。これは、彼の設計思想自体は正しかったものの、当時の製造技術(機械上の制約)が追いついていなかったことを示すエピソードである。

\textbf{\paragraph{本論から逸れた寄り道トピック名:工業社会から知識社会へ(ドラッカーの見解)}}
経営学者の$\textbf{ドラッカー}$は、コンピュータの商用化を、農業革命(農具の発明)、産業革命(蒸気機関の発明)に匹敵する$\textbf{情報革命}$の原動力であると指摘した。この革命により、社会は「工業社会」から、情報や知識を有効に使うことで価値を生み出す「$\textbf{知識社会}$(情報社会)」へと急速に移行していると論じた。

\textbf{\paragraph{本論から逸れた寄り道トピック名:ムーアの法則の自己評価}}
ムーアの法則を提唱した$\textbf{ムーア}$自身は、後年、この法則が「思った以上に正確だった」と振り返っている。一方で、トランジスタの部品が原子数個分のサイズに達しつつある現代において、集積化のスピードは減速せざるを得ないという物理的限界についても言及している。ただし、技術革新はシリコンベースで継続すると予測している。

\textbf{\subsubsection{AIによる補足:重要論点の拡張}}
本講義では、コンピュータの歴史が1960年代の汎用コンピュータ(メインフレーム)までを中心に解説された。しかし、「IT概念と業務改革」の融合を理解するためには、その後の決定的な変化、特に1980年代以降のトレンドを補足する必要がある。

\begin{description}
	\item[ダウンサイジングとクライアント・サーバシステム (1980年代~)]
	      高価なメインフレームによる集中処理から、安価な$\textbf{パーソナルコンピュータ (PC)}$やワークステーションをネットワーク(LAN)で接続する$\textbf{クライアント・サーバシステム}$(分散処理)への移行が進んだ。これは「ダウンサイジング」と呼ばれ、ITの導入コストを劇的に下げ、部門や個人レベルでのIT活用(エンドユーザー・コンピューティング)を促進し、オペレーションの柔軟性を高めた。

	\item[BPRとERP (1990年代~)]
	      1990年代に入ると、ITの進化を前提として、既存の業務プロセスを抜本的に見直す$\textbf{BPR (ビジネスプロセス・リエンジニアリング)}$が提唱された。これは、ITを単なる既存業務の自動化ツールとして使うのではなく、ITの能力(例:情報共有の容易さ)を前提に、非効率な業務プロセス自体を再設計(リエンジニアリング)する考え方である。このBPRを実現するための経営基盤として、企業の基幹業務(会計、人事、生産、販売など)を統合的に管理する$\textbf{ERP (Enterprise Resource Planning)}$パッケージが急速に普及した。

	\item[インターネット革命とSCM (1990年代後半~)]
	      インターネットの商用化は、企業「間」の情報交換コストを劇的に低下させた。これにより、それまでの企業内オペレーションの最適化(ERPなど)から、企業間のオペレーション最適化、すなわち$\textbf{サプライチェーン・マネジメント (SCM)}$へと関心が移った。ITネットワークを通じて、サプライヤーから顧客までの需要情報や在庫情報をリアルタイムで共有し、サプライチェーン全体の効率化を図ることが可能となった。
\end{description}
これらの変革(ダウンサイジング、BPR、SCM)こそが、ITが単なる「計算機」から「業務改革の武器」へと変貌したプロセスそのものである。

\subsection{結論}
本講義では、ITが人間の記憶・計算能力の限界を補うツールとして誕生し、パンチカードシステムによる業務利用を経て、ENIACに代表される電子コンピュータ、そしてIBM S/360を頂点とする汎用コンピュータ時代へと至る歴史的変遷を概観した。この進化は、$\textbf{ムーアの法則}$に象徴される半導体技術の指数関数的な発展によって支えられてきた。

「深層背景」で触れられた$\textbf{ドラッカー}$の指摘の通り、ITは単なる計算機ではなく、社会構造を「工業社会」から「$\textbf{知識社会}$」へと変革させる$\textbf{情報革命}$の原動力であった。この革命的な変化は、ムーアの法則が示すように(たとえ減速したとしても)継続的なものである。

本学習から得られる実践的な教訓は、オペレーションマネジメントにおいてITを単なる効率化ツール(ソロバンや電卓の延長)としてのみ捉えることの危険性である。ITの進化が、どのような新しい業務プロセス(例:$\textbf{BPR}$)や企業間連携(例:$\textbf{SCM}$)を可能にし、競争優位の源泉となり得るのかを常に問い続ける$\textbf{戦略的視点}$が不可欠である。

\subsection{重要キーワード一覧}
パスカル、ライプニッツ、バベッジ、ドラッカー、ムーア
\vspace{\baselineskip}
IT (情報技術)、ICT (情報通信技術)、ビット、バイト、情報処理、情報管理、情報交換、ムーアの法則、パンチカードシステム (PCS)、ENIAC、バッチ処理、タイムシェアリングシステム (TSS)

\subsection{理解度確認クイズ}
\begin{enumerate}
	\item 経営資源として「ヒト・モノ・カネ」に加えて重視されるようになった第4の資源は何か。
	\item コンピュータが人間の能力の限界を補う主要な2つの側面とは何か。
	\item コンピュータ内部ですべてのデータ(数値、文字、画像など)を表現するために使用される基本的な単位(0または1)を何と呼ぶか。
	\item 8ビットを1単位として扱うデータ量を何と呼ぶか。
	\item プログラムにより高速に計算・演算処理を行うコンピュータの基本機能を何と呼ぶか。
	\item データベースなどを用いて情報を蓄積・検索可能にするコンピュータの基本機能を何と呼ぶか。
	\item ネットワークを介して地理的・時間的制約なく情報をやり取りするコンピュータの基本機能を何と呼ぶか。
	\item 「半導体の集積密度は約2年で倍増する」という経験則を提唱者の名前をとって何と呼ぶか。
	\item アメリカの国勢調査などで初期に実用化された、紙カードに穴を開けてデータを記録・処理するシステムを何と呼ぶか。
	\item 1946年に開発された、真空管を使用した世界最初の電子コンピュータとされるものは何か。
	\item 一定期間のデータを集め、一括して処理する初期のコンピュータ処理方式を何と呼ぶか。
	\item 1960年代の汎用コンピュータで普及した、CPUなどの資源を時間分割して複数の端末が同時に利用できるようにしたシステムを何と呼ぶか。
	\item 1964年にIBMが発表し、その後のコンピュータアーキテクチャの基盤となった汎用コンピュータのシリーズ名は何か。
	\item ドラッカーが提唱した、農業社会、工業社会に次ぐ、コンピュータ革命によってもたらされた社会の形態を何と呼ぶか。
	\item (AI補足論点より) 1990年代に提唱された、ITの能力を前提として既存の業務プロセスを抜本的に再設計する経営改革手法を何と呼ぶか。
\end{enumerate}

\subsubsection*{解答一覧}
1. 情報、2. 記憶と計算、3. ビット (bit)、4. バイト (byte)、5. 情報処理、6. 情報管理、7. 情報交換、8. ムーアの法則、9. パンチカードシステム (PCS)、10. ENIAC (エニアック)、11. バッチ処理、12. タイムシェアリングシステム (TSS)、13. S/360 (System/360)、14. 知識社会 (または情報社会)、15. BPR (ビジネスプロセス・リエンジニアリング)

\section{PCと情報通信}

\subsection{はじめに}
汎用コンピュータが主流であった時代、ITの技術的基盤は一時的な停滞を見せていた。しかし、1980年代の$\textbf{パーソナルコンピュータ (PC)}$の登場は、この状況を一変させ、企業オペレーションにおける情報システムのあり方を根本から変革した。
本講義ノートの目的は、PCの誕生から$\textbf{クライアントサーバシステム}$の普及、そして$\textbf{インターネット}$がIT基盤の標準となるまでの技術的・構造的変遷を整理することにある。特に、現代のOS市場の形成に至った経緯と、ネットワーク技術の進化がオペレーションに与えた影響を深く理解する。

\subsection{主要な概念と論点}
本講義では、汎用コンピュータ時代後のIT基盤を形成した主要な概念について解説がなされた。

\subsubsection{パーソナルコンピュータ (PC) の登場}
1980年代、$\textbf{マイクロプロセッサ}$の性能向上により、個人レベルで利用可能な$\textbf{パーソナルコンピュータ (PC)}$が誕生した。PCは、それまでの汎用コンピュータとは異なり、単独での計算処理、データ分析、レポート作成を可能にし、さらにサーバコンピュータと接続することによる$\textbf{データ共有}$を実現した。

\subsubsection{クライアントサーバシステムと分散処理}
PCの普及に伴い、企業の情報システムは$\textbf{クライアントサーバシステム}$へと移行した。
\begin{itemize}
	\item \textbf{定義}: サービスを要求する$\textbf{クライアント}$(顧客役)としてのPCと、サービスを提供する$\textbf{サーバ}$の組み合わせによる処理方式。
	\item \textbf{分散処理}: 複数のサーバと多数のPC(クライアント)がネットワークを介して接続され、負荷を分散して処理を行う形態。
	\item \textbf{対比}: この方式は、大型の汎用コンピュータがすべての処理を担う$\textbf{集中処理}$と対比される。
\end{itemize}

\subsubsection{ネットワーク技術の進化 (LANとTCP/IP)}
処理方式の変化は、通信技術の進化と不可分であった。
\begin{itemize}
	\item \textbf{汎用コンピュータ時代}: コンピュータメーカー(IBM、$\textbf{UNIVAC}$など)が独自に設定した通信方式であり、中心的な通信制御装置によって管理されていた。
	\item \textbf{クライアントサーバ時代}: 施設内の複数のPCとサーバを接続する$\textbf{LAN (ローカルエリアネットワーク)}$が普及。LAN上では、PCとサーバは通信において同等な位置づけとなり、この接続の標準的な通信規約(プロトコル)として$\textbf{TCP/IP}$が広く採用された。
\end{itemize}

\subsubsection{インターネットとWWWの普及}
企業内LANで$\textbf{TCP/IP}$が標準となったことは、外部の$\textbf{インターネット}$との接続を容易にした。
\begin{itemize}
	\item \textbf{起源}: インターネットの技術的基盤は、米国の国防総省高等研究計画局 (ARPA) が構築した$\textbf{ARPANET (アーパネット)}$にあり、元々は軍事技術であった。その後、大学や研究機関がTCP/IPを採用し、ネットワークが相互接続されていった。
	\item \textbf{商用化の起爆剤 (WWW)}: 1990年代に$\textbf{WWW (ワールドワイドウェブ)}$が普及したことにより、インターネットの商用利用が爆発的に進んだ。
	\item \textbf{HTML}: WWWは、ドキュメントを$\textbf{HTML (HyperText Markup Language)}$と呼ばれる言語で記述することにより、インターネット上でのドキュメントの共有と閲覧を可能にする仕組みである。
\end{itemize}
現在では、汎用コンピュータやクライアントサーバシステムも依然として存在しているが、企業内外の通信ネットワークの基盤としてインターネットが不可欠な「$\textbf{インターネットの時代}$」となっている。

\subsection{応用と事例分析}
PCの登場期におけるOS(オペレーティングシステム)の覇権争いは、現代のIT市場の構造を決定づけた重要な事例である。

\subsubsection{OS覇権を巡る攻防 (Apple vs IBM/Microsoft)}
講義では、現在のPC市場を二分するOS(MacOSとWindows)の成立過程が解説された。
\begin{itemize}
	\item \textbf{Appleの挑戦}: 1984年、$\textbf{スティーブ・ジョブズ}$が率いるApple社は、革新的なグラフィカルユーザインタフェース (GUI) を搭載した$\textbf{Macintosh}$を発売した。
	\item \textbf{IBMのPC参入とOS開発の遅れ}: 巨大企業IBMもPC開発に着手したが、主力である産業用コンピュータ市場に比べ、PC市場を魅力的なものと捉えなかった。結果として、最も重要なソフトウェアである$\textbf{OS (オペレーティングシステム)}$の開発に十分なリソースが割かれず、自社開発を断念した。
	\item \textbf{Microsoftの台頭}: IBMは、当時新興企業であった$\textbf{ビル・ゲイツ}$の$\textbf{Microsoft}$社からOS($\textbf{DOS}$)の提供を受けることを決定した。さらに、Macintoshに対抗するGUIソフトウェア(後の$\textbf{Windows}$)の開発もMicrosoftに依頼した。
	\item \textbf{市場の決定}: この結果、IBMのPC(およびその互換機)をプラットフォームとしてMicrosoftのOS(DOS/Windows)がデファクトスタンダードとなり、Appleの$\textbf{MacOS}$と市場を二分する構図が確立した。
	\item \textbf{IBMの結末}: OSという基幹部分を外部に依存したIBMは、PC市場での主導権を失い、最終的にPC事業を$\textbf{レノボ (Lenovo)}$社に売却することとなった。
\end{itemize}

\subsection{深層背景と教訓}
講義では、本論の理解を深めるための象徴的なエピソードが紹介された。

\textbf{\paragraph{本論から逸れた寄り道トピック名:Appleの1984年CMの衝撃}}
1984年のMacintosh発売時に放映されたコマーシャルは、IT業界における世代交代と反逆の象徴として語り継がれている。
\begin{itemize}
	\item \textbf{IBMの象徴}: CMでは、青白い堅苦しい老人がスクリーンから群衆を支配している。これは、当時「$\textbf{ビッグブルー}$」と呼ばれ、東海岸の伝統的産業(金融、鉄鋼など)を支える、ダークスーツに$\textbf{レジメンタルタイ}$という保守的な企業文化を持つ$\textbf{IBM}$を象徴していた。
	\item \textbf{Appleの象徴}: そこへ、西海岸の自由な文化を象徴するような、赤いホットパンツを履いた女性がハンマーを持って走り込み、そのスクリーンを破壊する。
	\item \textbf{メッセージ}: この過激な描写は、既存の支配者(IBM)による情報の独占を、Macintoshという個人のためのツールが打ち破るという、$\textbf{スティーブ・ジョブズ}$の強烈なメッセージであった。
\end{itemize}

\textbf{\paragraph{本論から逸れた寄り道トピック名:IBMの失策:「軒先を貸しておもやを取られる」}}
IBMは、PCの基盤技術(ハードウェア)を提供しながら、市場の利益の源泉(OS、ソフトウェア)をMicrosoftに奪われる結果となった。これは、経営戦略上の教訓として「$\textbf{軒先を貸しておもやを取られる}$」と形容される。
IBMはPC市場の重要性を過小評価し、中核となるOSを外部(Microsoft)に依存した。Microsoftは、IBMという「軒先」(プラットフォーム)を借りて自社のOSを普及させ、最終的にはPCエコシステム全体(おもや)の支配権を握るに至った。

\textbf{\subsubsection{AIによる補足:重要論点の拡張}}
本講義では、PCからインターネットへのIT基盤の移行が解説された。この変革がオペレーションマネジメントに与えた直接的な影響について、以下の3つの重要論点を補足する。

\begin{description}
	\item[ダウンサイジング (Downsizing)]
	      講義では「分散処理」として言及されたが、この移行は経営・IT用語として「$\textbf{ダウンサイジング}$」と呼ばれる。高価で閉鎖的な汎用コンピュータから、安価で標準化された(オープンな)PCやサーバで構成されるクライアントサーバシステムへ移行することにより、IT導入・維持コストが劇的に低下した。これにより、全社レベルだけでなく部門レベルでの柔軟なシステム導入が可能となり、オペレーションの現場ニーズに即応しやすくなった。

	\item[BPR (ビジネスプロセス・リエンジニアリング)]
	      クライアントサーバシステムとLANの普及は、部門間でのリアルタイムな情報共有を技術的に可能にした。この技術的基盤は、1990年代に提唱された$\textbf{BPR (ビジネスプロセス・リエンジニアリング)}$を強力に後押しした。BPRは、ITの能力を前提として、従来の分断された業務プロセスを抜本的に再設計(リエンジニアリング)し、リードタイム短縮やコスト削減といった劇的なオペレーション改善を目指す経営改革手法である。

	\item[イントラネットとエクストラネット]
	      WWWとTCP/IPの技術は、パブリックなインターネット利用に留まらず、企業「内」の閉じたネットワーク($\textbf{イントラネット}$)や、特定の取引先企業「間」を結ぶネットワーク($\textbf{エクストラネット}$)の構築にも応用された。イントラネットは社内の情報共有や業務システム(例:ワークフロー)の基盤となり、エクストラネットは受発注システムや情報共有など、後の$\textbf{SCM (サプライチェーン・マネジメント)}$や$\textbf{CRM (顧客関係管理)}$の先駆けとなる企業間連携オペレーションの基盤となった。
\end{description}

\subsection{結論}
本講義では、1980年代のPCの登場が、ITの活用形態を汎用コンピュータによる$\textbf{集中処理}$から、クライアントサーバシステムによる$\textbf{分散処理}$へと移行させたプロセスを学んだ。この変革期において、Appleの挑戦的な製品投入と、IBMの戦略的失策(OSの外部依存)が、Microsoftの台頭を許し、現代のOS市場の構造を決定づけた。

さらに、軍事技術から発展した$\textbf{TCP/IP}$と$\textbf{インターネット}$、そしてその起爆剤となった$\textbf{WWW}$の普及が、IT基盤を企業内LANからグローバルなネットワークへと拡張し、現代のオペレーション環境を形成した。

本学習から得られる実践的な教訓は、「軒先を貸しておもやを取られる」というIBMの事例が示す、$\textbf{プラットフォーム戦略}$の重要性である。自社のオペレーションやビジネスモデルの中核(コア)となる技術や基盤(本件ではOS)を外部に依存することは、たとえ短期的には効率的に見えても、長期的には市場の主導権そのものを失う戦略的リスクを孕んでいる。イノベーションの波において、何を守り、何を外部に委ねるかの見極めが経営の死活を分けることを強く示唆している。

\subsection{重要キーワード一覧}
スティーブ・ジョブズ、ビル・ゲイツ
\vspace{\baselineskip}
パーソナルコンピュータ (PC)、オペレーティングシステム (OS)、クライアントサーバシステム、分散処理、集中処理、LAN (ローカルエリアネットワーク)、TCP/IP、インターネット、WWW (ワールドワイドウェブ)、HTML、ダウンサイジング、BPR (ビジネスプロセス・リエンジニアリング)、イントラネット、エクストラネット

\subsection{理解度確認クイズ}
\begin{enumerate}
	\item 1980年代のパーソナルコンピュータ誕生を技術的に可能にした、高性能な集積回路を何と呼ぶか。
	\item PCが登場する以前、企業の情報処理の中心だった大型コンピュータを何と呼ぶか。
	\item 汎用コンピュータが1台の大型機ですべての処理を行う方式を何と呼ぶか。
	\item PCとサーバがネットワークで接続され、処理を分担する方式を何と呼ぶか。
	\item 分散処理システムにおいて、サービスを要求する側のコンピュータ(PCなど)を何と呼ぶか。
	\item 分散処理システムにおいて、ファイル共有やデータ処理などのサービスを提供する側のコンピュータを何と呼ぶか。
	\item 企業や大学のキャンパスなど、比較的狭い範囲のコンピュータを接続するネットワークを何と呼ぶか。
	\item 現在のインターネットや多くのLANで標準的に使用されている通信プロトコル(通信規約)群をアルファベット4文字で何と呼ぶか。
	\item インターネットの起源となった、米国防総省が構築したネットワークの名称は何か。
	\item 1990年代に普及し、インターネットの商用利用を爆発的に広げた、ドキュメント共有・閲覧システムをアルファベット3文字で何と呼ぶか。
	\item WWWにおいて、ウェブページを記述するために使用されるマークアップ言語をアルファベット4文字で何と呼ぶか。
	\item IBMが自社のPCに搭載するため、OSの提供を依頼した企業はどこか。
	\item Apple社が1984年に発売した、GUI(グラフィカルユーザインタフェース)を特徴とする画期的なPCの名称は何か。
	\item (AI補足論点より) 高価な汎用コンピュータから、安価なPCやサーバで構成されるオープンシステムへ移行することを指す経営・IT用語は何か。
	\item (AI補足論点より) WWWの技術を応用し、企業「内」の閉じたネットワークで情報共有や業務システムに利用する形態を何と呼ぶか。
\end{enumerate}

\subsubsection*{解答一覧}
1. マイクロプロセッサ、2. 汎用コンピュータ (またはメインフレーム)、3. 集中処理、4. 分散処理、5. クライアント、6. サーバ、7. LAN (またはローカルエリアネットワーク)、8. TCP/IP、9. ARPANET (アーパネット)、10. WWW (ワールドワイドウェブ)、11. HTML、12. Microsoft (マイクロソフト)、13. Macintosh (マッキントッシュ)、14. ダウンサイジング、15. イントラネット

\section{IT活用の発展と業務支援}

\subsection{はじめに}
コンピュータ技術の進化に伴い、企業における情報システムの活用方法もまた、単純な業務効率化から経営戦略の武器へと高度化・多様化してきた。
本講義ノートの目的は、このIT活用の歴史的発展を体系的に整理することにある。$\textbf{ノーランの発展段階仮説}$のような理論的枠組みから、$\textbf{MIS}$、$\textbf{DSS}$、$\textbf{SIS}$といった各段階の中心的システム概念、さらには$\textbf{知識管理}$に至るまでの変遷を概観する。最終的に、これらのIT概念が現代のオペレーション改革(SCM、CRMなど)といかにして融合するに至ったかを理解する。

\subsection{主要な概念と論点}
本講義では、IT活用の発展を分類する複数のフレームワークと、各時代を象徴するシステム概念について解説された。

\subsubsection{ノーランの発展段階仮説}
$\textbf{リチャード・ノーラン (Richard Nolan)}$は、企業内の情報システム活用が進化するプロセスを6段階の仮説として提示した(当初4段階、1979年に6段階に改定)。
\begin{enumerate}
	\item 初期 (Initiation)
	\item 普及 (Contagion)
	\item 統制 (Control)
	\item 統合 (Integration)
	\item データ管理 (Data Administration)
	\item 成熟 (Maturity)
\end{enumerate}
特に、第3段階「統制」から第4段階「統合」への移行が技術的な転換点であり、対象業務が限定的な$\textbf{DP (データプロセッシング)}$から、企業全体で活用する統合された$\textbf{IT (情報技術)}$へと移行する点が重要であると指摘された。

\subsubsection{情報システム機能の発展分類}
歴史的変遷は、その主要な機能・目的に応じて以下の5つの段階に大別される。
\begin{enumerate}
	\item 業務効率化のための情報システム (DP)
	\item 管理のための情報システム (MIS)
	\item 意思決定支援のための情報システム (DSS)
	\item 戦略実現のための情報システム (SIS)
	\item 知識管理とコミュニケーションのための情報システム
\end{enumerate}

\subsubsection{MIS (経営情報システム) とアンソニーの階層}
1960年代半ば、$\textbf{MIS (Management Information System)}$の概念が登場した。これは、業務効率化の次の段階として、経営管理層へのレポーティング機能を提供するシステムである。
その理論的背景には、$\textbf{ロバート・アンソニー (Robert Anthony)}$による経営管理機能の3階層モデルがある。
\begin{description}
	\item[戦略的計画 (Strategic Planning)] 組織目的、資源配分、基本方針の決定(トップマネジメント)。
	\item[マネジメント・コントロール (Management Control)] 目標達成のための資源の効率的・能率的な取得と使用の確保(ミドルマネジメント)。
	\item[オペレーショナル・コントロール (Operational Control)] 特定業務の効率的・能率的な遂行の確保(ロワーマネジメント)。
\end{description}
MISは、これら経営管理活動全体をシステムとして支援することを目指した。

\subsubsection{DSS (意思決定支援システム)}
1970年代後半、対話型UIやAI技術の進化を背景に、$\textbf{DSS (Decision Support System)}$が登場した。これは、MISが定型的な情報提供を主としたのに対し、$\textbf{非定型的な業務}$を支援するものである。
\begin{itemize}
	\item $\textbf{ゴーリ&スコット・モートン (Gorry \& Scott Morton)}$は、DSSを「$\textbf{反構造的}$および$\textbf{非構造的}$な意思決定を支援する情報システム」と定義した。
	\item $\textbf{スプレーグ&カールソン (Sprague \& Carlson)}$は、DSSの構成要素として、データベース管理、対話管理機能に加え、$\textbf{モデルベース管理機能}$を提示した。モデルベースには、財務モデル、統計モデル、数理科学モデルなどが含まれ、これがDSSの大きな特徴となった。
\end{itemize}

\subsubsection{SIS (戦略的情報システム) とバリューチェーン}
1980年代、ITと通信技術の統合が進み、システムは企業「間」にも拡大した。これにより、ITを競争戦略の武器として捉える$\textbf{SIS (Strategic Information System)}$の概念が生まれた。
\begin{itemize}
	\item $\textbf{マイケル・ポーター (Michael Porter)}$の$\textbf{バリューチェーン (価値連鎖)}$理論は、この現象を説明するフレームワークを提供した。ITによる強固な企業間連携は、顧客の$\textbf{スイッチング・コスト}$(他社への乗り換えコスト)を高め、$\textbf{顧客の囲い込み}$を可能にするとした。
	\item $\textbf{チャールズ・ワイズマン (Charles Wiseman)}$は、SISを「自社の競争優位の獲得や維持、あるいは他社の優位の削減のためのプランニングを支援・形成する情報システム」と定義した。
\end{itemize}
当初は顧客の囲い込みが注目されたが、後にコスト低減、製品革新など、オペレーションにおける競争戦略実現の武器として広く認識されるようになった。

\subsubsection{知識管理とコミュニケーション}
1990年代、$\textbf{WWW (ワールドワイドウェブ)}$の普及と技術革新により、従来扱えなかった文書、静止画、動画、音声といった非構造化データの大量処理が可能となった。
これにより、インターネットサイトを通じたステークホルダーとのコミュニケーションや、$\textbf{グループウェア}$による組織内の$\textbf{知識共有}$(ナレッジマネジメント)がITの新たな活用領域となり、$\textbf{知識経済}$時代を支える基盤となった。

\subsection{応用と事例分析}
各時代のIT活用は、具体的なシステムや社会現象として現れた。

\subsubsection{業務効率化 (DP) の段階}
1950〜60年代のシステム化は、手作業の$\textbf{定型業務}$をコンピュータで代替すること(オペレーションの効率化)が目的であった。
\begin{itemize}
	\item \textbf{会計・経理}: 帳簿のデータベース化、集計の自動計算、帳票印刷。
	\item \textbf{受発注}: 受注から納品・請求・入金までの一連のフロー管理、出荷指示書など関連伝票の自動作成。
	\item \textbf{関連概念}: $\textbf{DP}$ (Data Processing), $\textbf{ADP}$ (Automatic DP), $\textbf{EDP}$ (Electronic DP)。
\end{itemize}

\subsubsection{MISのブームと挫折}
1960年代、米国でMISブームが発生。1967年には日本からも「MIS視察団」が派遣されるなど、国家的な危機感を持って導入が目指された。
しかし、「経営者が求める情報を適時に提供する」という高い理想に対し、当時の技術は未熟であり、MISは「$\textbf{ミス}$(失敗、Myth: 神話)」と揶揄される結果となった。
\begin{itemize}
	\item \textbf{失敗の理由}: (1) データ蓄積のためのコード体系の不統一、(2) データベース構築・管理技術の未整備、(3) IT利用を前提とした経営方式の未確立。
	\item \textbf{ディアデンの批判}: 経営学者の$\textbf{ディアデン (Dearden)}$は、「トータル・インフォメーション・システムは永遠に不可能だ」と述べ、機能別の着実なシステム開発しか道はないと批判した。
\end{itemize}
一方で、MISの試みは、業務データ(DP)を統合して経営管理情報(レポーティング)を提供するという方向性を示し、後のIT発展の礎となったと評価できる。

\subsubsection{エキスパートシステムの実用化}
DSSの時代には、$\textbf{AI (人工知能)}$技術の応用である$\textbf{エキスパートシステム}$が実用化された。これは、専門家の知識を「$\textbf{知識ベース}$」(IF-THENルールなど)として蓄積し、「$\textbf{推論エンジン}$」が専門家のように判断を行うシステムである。
\begin{itemize}
	\item \textbf{応用分野}: $\textbf{パターン認識}$(音声や画像の意味を認識)、$\textbf{データマイニング}$(大量データからの知識抽出)、生産スケジューリング、設備異常診断など。
	      * \textbf{基盤技術}: 判断基準が変化する曖昧な処理を実現するため、人間の脳をモデル化した$\textbf{ニューラルネットワーク}$などの技術が実用化された。
\end{itemize}

\subsection{深層背景と教訓}
講義の本論に加え、文脈を深めるための周辺情報やエピソードが紹介された。

\textbf{\paragraph{本論から逸れた寄り道トピック名:日本のMIS導入への危機感}}
1967年、日本生産性本部が中心となりMIS視察団が訪米した背景には、「コンピュータ・ギャップ」に対する強烈な危機感があった。当時の佐藤首相への報告書では、「原子力や宇宙産業における立ち遅れよりも、コンピュータ・ギャップが国際競争に及ぼす影響の方がはるかに重大である」と結論付けられた。この危機感が、直後の日本情報処理開発センター(JIPDEC)設立などに繋がった。

\textbf{\paragraph{本論から逸れた寄り道トピック名:MIS失敗の予言とERPによる反証}}
1970年代、ディアデンは「トータルシステムは不可能」と予言したが、本講義の講師は、この予言は現実には当たらなかったと指摘する。なぜなら、その後の技術発展(データベース、ネットワーク)により、1990年代に登場した$\textbf{ERP (Enterprise Resource Planning)}$が、まさに企業全体の業務プロセスを(ある程度)統合し、MISが目指した世界の多くを実現したためである。

\textbf{\paragraph{本論から逸れた寄り道トピック名:プリクラにみるパターン認識}}
エキスパートシステムの応用例であるパターン認識の身近な例として、「プリクラ」(プリント倶楽部)が挙げられた。プリクラ機が自動で人の顔を画像から認識し、目を正確に捉えて拡大したり、まつげを追加したりする処理は、まさにコンピュータが曖昧な(定式化が難しい)画像情報から意味のある特徴を抽出し、オペレーション(加工)を実行している例である。

\textbf{\subsubsection{AIによる補足:重要論点の拡張}}
本講義では、1980年代の戦略的IT活用としてSISとバリューチェーン(主に企業間・対顧客戦略)が解説された。しかし、同時期にITの能力向上を背景に議論され、1990年代に爆発的に普及したもう一つの重要なオペレーション改革概念が$\textbf{BPR (ビジネスプロセス・リエンジニアリング)}$である。

\begin{description}
	\item[BPR (Business Process Reengineering)]
	      $\textbf{BPR}$は、マイケル・ハマーやジェイムズ・チャンピーによって提唱された経営改革手法である。これは、ITを単に既存業務の自動化(Efficiency: 効率化)に使うのではなく、ITの能力(例:共有データベース、通信ネットワーク)を前提として、既存の業務プロセス(Business Process)を抜本的に、劇的に$\textbf{再設計 (Reengineering)}$する考え方である。
	      「自動化するな、廃棄せよ (Don't automate, obliterate.)」というスローガンのもと、従来の分業化されたプロセスをITで繋ぎ直すのではなく、プロセス自体をITありきでゼロから見直し、コスト、品質、サービス、スピードの劇的な向上を目指した。SISが対外的な競争優位に焦点を当てたのに対し、BPRは対内的なオペレーションの根本的改革に焦点を当てた概念であり、ITとオペレーション改革の融合を語る上で不可欠な論点である。
\end{description}

\subsection{結論}
本講義では、IT活用が「業務効率化(DP)」から「管理(MIS)」、「意思決定支援(DSS)」、「戦略(SIS)」、そして「知識管理」へと発展・拡大してきた歴史的プロセスを概観した。
MISの失敗とディアデンの批判にもかかわらず、その理想はERPによって後に実現され、SISはITをオペレーション改革の武器として明確に位置づけた。

この変遷から得られる実践的な教訓は、ITが単なる「効率化のツール」や「管理の道具」ではなく、オペレーションのあり方そのものを定義し、企業の競争優位を左右する$\textbf{戦略的基盤}$であるということだ。
講義の最後に示されたように、現代の主要なオペレーション改革概念である$\textbf{SCM (サプライチェーン・マネジメント)}$、$\textbf{CRM (顧客関係管理)}$、$\textbf{ナレッジマネジメント}$は、それぞれERP、インターネット、グループウェアといったIT基盤と不可分に融合している。IT戦略とオペレーション戦略は、もはや別個に論じることはできない。

\subsection{重要キーワード一覧}
リチャード・ノーラン、ロバート・アンソニー、ディアデン、ゴーリ&スコット・モートン、スプレーグ&カールソン、マイケル・ポーター、チャールズ・ワイズマン
\vspace{\baselineskip}
データプロセッシング (DP)、経営情報システム (MIS)、意思決定支援システム (DSS)、エキスパートシステム、ニューラルネットワーク、パターン認識、データマイニング、戦略的情報システム (SIS)、バリューチェーン (価値連鎖)、スイッチング・コスト、ERP (企業資源計画)、グループウェア、知識管理

\subsection{理解度確認クイズ}
\begin{enumerate}
	\item 企業内のIT活用が6つの段階(初期、普及、統制、統合など)を経て発展するとした仮説を提唱したのは誰か。
	\item ノーランの発展段階仮説において、技術的な転換点とされ、DPからITへの移行が起こる段階は、「統制」からどの段階への移行か。
	\item 1950〜60年代、会計処理や受発注業務など、主に手作業の定型業務を自動化・効率化することを目的としたシステム概念を何と呼ぶか。
	\item 経営管理機能を「戦略的計画」「マネジメント・コントロール」「オペレーショナル・コントロール」の3階層に分類した経営学者は誰か。
	\item 1960年代にブームとなったが、技術的な未熟さなどから「失敗(ミス)」とも揶揄された経営管理システム概念は何か。
	\item 財務モデルや統計モデルなどを「モデルベース」として持ち、非定型的な意思決定を支援するシステムを何と呼ぶか。
	\item DSS (意思決定支援システム) を「反構造的・非構造的意思決定を支援する」と定義した中心人物は誰か。
	\item AI技術を応用し、専門家の知識を「知識ベース」として蓄積し、推論を行うことで専門家のような判断を試みるシステムを何と呼ぶか。
	\item 1980年代に登場した、ITを競争戦略の武器として捉え、競争優位の獲得・維持を目指すシステム概念をアルファベット3文字で何と呼ぶか。
	\item ITの活用により、顧客が他社製品・サービスへ乗り換える際のコスト(金銭的・時間的・心理的コスト)を高める戦略を何と呼ぶか。
	\item 企業の活動を「主活動」と「支援活動」に分類し、ITがどの部分で価値(マージン)を生み出すかを分析するポーターの戦略フレームワークを何と呼ぶか。
	\item 1990年代以降、WWWの普及により、文書・画像・動画などの非構造化データを共有・活用する経営手法が重視されるようになった。これを何と呼ぶか。
	\item ディアデンが「不可能」と予言したトータルシステムを、1990年代に技術的進歩によって実現したとされる統合業務パッケージを何と呼ぶか。
	\item (AI補足論点より) 1990年代に提唱された、ITの能力を前提に、既存の業務プロセスを抜本的に再設計する経営改革手法を何と呼ぶか。
	\item 講義の結論として、現代のIT概念はオペレーション改革概念と「統合されてきた」とされた。サプライチェーン・マネジメント(SCM)と強く結びつくITシステムは何か。
\end{enumerate}

\subsubsection*{解答一覧}
1. リチャード・ノーラン、2. 統合、3. DP (データプロセッシング)、4. ロバート・アンソニー、5. MIS (経営情報システム)、6. DSS (意思決定支援システム)、7. ゴーリ&スコット・モートン、8. エキスパートシステム、9. SIS (戦略的情報システム)、10. スイッチング・コスト(の引き上げ)、11. バリューチェーン (価値連鎖)、12. 知識管理 (ナレッジマネジメント)、13. ERP (企業資源計画)、14. BPR (ビジネスプロセス・リエンジニアリング)、15. ERP (企業資源計画)

\end{document}
\documentclass[uplatex,a4j,12pt,dvipdfmx]{jsarticle}
\usepackage{amsmath,amsthm,amssymb,bm,color,enumitem,mathrsfs,url,epic,eepic,ascmac,ulem,here,ascmac}
\usepackage[letterpaper,top=2cm,bottom=2cm,left=3cm,right=3cm,marginparwidth=1.75cm]{geometry}
\usepackage[english]{babel}
\usepackage[dvipdfm]{graphicx}
\usepackage[hypertex]{hyperref}
\title{Operations Management Lecture 2 Course Notes\newline IT Concepts and Business Reform}
\author{M. O.}
\date{\today}
\begin{document}
\maketitle
\tableofcontents
\section{IT}
\subsection{Introduction}
The purpose of these lecture notes is to overview the historical evolutionary trajectory of information systems and review the historical transition of \textbf{IT (Information Technology)} utilization in corporate management.
This will build a foundation for understanding the reality that modern concepts of operations reform are inseparably fused with the evolution of IT.
\subsection{Key Concepts and Points}
This lecture explained the fundamental concepts of IT that support operations and the principles that have driven its evolution.
\subsubsection{Definition of IT (Information Technology)}
\textbf{Information} is positioned as the fourth management resource, alongside 'people, goods, and money'. The foundational tools for managing and utilizing this information are computers and information systems.
\begin{itemize}
	\item \textbf{IT (Information Technology)}: A general term for technologies related to information. For management, it is a weapon for realizing strategy and an important target for investment management.
	\item \textbf{ICT (Information and Communication Technology)}: A concept that adds the 'Communication' aspect to information technology. In this lecture, these are collectively referred to as 'IT'.
	\item \textbf{The Essence of IT}: IT compensates for the limits of human capabilities, particularly the limits of 'memory' and 'calculation'. It plays a role in supplementing human weaknesses such as the ambiguity and bias of memory, and the speed limitations and errors of calculation, thereby ensuring the reproducibility and efficiency of operations.
\end{itemize}
\subsubsection{Computer Principles and Basic Functions}
A computer consists of the following five functional components.
\begin{enumerate}
	\item Central Processing Unit (CPU)
	\item \textbf{Main Memory}
	\item Auxiliary Storage
	\item Input Devices
	\item Output Devices
\end{enumerate}
All data is represented by '0' or '1' \textbf{bits}, and a unit of 8 bits is called a \textbf{byte}. Japanese Kanji and other characters are often represented by 2-byte characters (16 bits).
Furthermore, the basic functions of a computer are broadly divided into the following three:
\begin{description}
	\item[Information Processing] The function of executing high-speed and accurate calculations and operations based on programs. Corresponds to a 'high-performance abacus or calculator' in operations.
	\item[Information Management] The function of storing large amounts of information and knowledge using storage devices and retrieving it as needed. Corresponds to 'efficient ledgers or files'.
	\item[Information Exchange] The function of exchanging information beyond geographical and temporal constraints, combined with communication technology. Corresponds to 'instantaneously transmittable slips or letters'.
\end{description}
\subsubsection{The Evolution of Computers and Moore's Law}
The evolution of computers has been driven by technological innovation in their constituent parts. This progressed from the vacuum tubes used in \textbf{ENIAC} (discussed later), to transistors, \textbf{IC (Integrated Circuits)}, \textbf{LSI (Large-Scale Integration)}, and \textbf{VLSI (Very Large-Scale Integration)}, leading to higher densities.
Symbolizing this evolution is \textbf{Moore's Law}, proposed by Intel founder \textbf{Moore}. This is an empirical rule stating that 'the integration density of semiconductors doubles every 18 to 24 months', and it is still referenced today as an expression of the exponential improvement in computer performance.
\subsection{Application and Case Analysis}
The lecture showed the historical timeline of IT application in operations, along with specific examples of computing machines.
\subsubsection{The Origin of Calculating Machines}
The origin of computers dates back to the 17th-century research on mechanical calculators by \textbf{Pascal} and \textbf{Leibniz}. In the 19th century, \textbf{Babbage} conceived the \textbf{Analytical Engine}, but it was not realized due to the technological constraints of the time.
The first calculating machine to be put into practical use was the \textbf{Punched Card System (PCS)}, which appeared at the end of the 19th century. It was used for the US census and corporate accounting processes, demonstrating its application in operations even at that early stage.
\subsubsection{The Advent of Electronic Computers and the Mainframe Era}
During World War II, the prototype of the digital computer was born through the application of electronic technology.
\begin{itemize}
	\item \textbf{ENIAC}: The first electronic computer, developed at the University of Pennsylvania in 1946. It used vacuum tubes and was extremely large. After the war, its adoption began for scientific, technological, and administrative purposes.
	\item \textbf{Mainframe Computers (General-Purpose Computers)}: The era arrived when companies introduced large computers for shared use.
	      \begin{itemize}
		      \item \textbf{Batch Processing}: The initial mode of use. Programs and data were input via magnetic tapes or punched cards, and processing was executed in batches at fixed intervals.
		      \item \textbf{Time-Sharing System (TSS)}: A method that enabled simultaneous use from multiple terminals by time-slicing computer resources.
		      \item \textbf{IBM S/360 (System/360)}: Announced by IBM in 1964. This is considered a crucial system that established the foundation of modern computer technology architecture.
	      \end{itemize}
	\item \textbf{Minicomputers and Supercomputers}: Subsequently, minicomputers for technical computing (like \textbf{DEC}) and supercomputers for high-performance computing (like \textbf{Cray Research}; a recent example is Japan's 'K computer') emerged, but there was no significant difference in the TSS mode of use.
\end{itemize}
\subsection{Deeper Context and Lessons}
In addition to the main lecture content, related information and anecdotes were introduced to deepen the context.
\textbf{\paragraph{Side Topic Name: The Cat-Type Robot as a Metaphor for IT}}
In the lecture, a famous cat-type robot was raised as a metaphor for understanding the essence of IT. It was explained that the way this character assists an under-performing elementary school student (human) with his operations (realizing dreams and daily life) using future technology (IT), supplementing his weaknesses in memory and calculation, is analogous to the role of IT.
\textbf{\paragraph{Side Topic Name: Epilogue to Babbage's Analytical Engine}}
Babbage's 19th-century Analytical Engine was not practical at the time, but its blueprints survived. Later, when the machine was constructed according to those blueprints, it was confirmed to operate as intended. This anecdote shows that while his design philosophy itself was correct, the manufacturing technology of the time (mechanical constraints) had not caught up.
\textbf{\paragraph{Side Topic Name: From Industrial Society to Knowledge Society (Drucker's View)}}
The management scholar \textbf{Drucker} pointed out that the commercialization of the computer was the driving force of an \textbf{Information Revolution}, comparable to the Agricultural Revolution (invention of farm tools) and the Industrial Revolution (invention of the steam engine). He argued that this revolution is rapidly transitioning society from an 'industrial society' to a '\textbf{knowledge society}' (information society), where value is created by effectively using information and knowledge.
\textbf{\paragraph{Side Topic Name: Moore's Self-Assessment of Moore's Law}}
\textbf{Moore} himself, who proposed Moore's Law, later reflected that the law was 'more accurate than I expected'. On the other hand, he also mentioned the physical limits, stating that as transistor components reach the size of just a few atoms, the speed of integration must inevitably slow down. However, he predicted that technological innovation would continue on a silicon basis.
\textbf{\subsubsection{AI Supplement: Expansion of Important Points}}
This lecture primarily explained the history of computers up to the mainframe era of the 1960s. However, to understand the fusion of 'IT Concepts and Business Reform', it is necessary to supplement with the decisive changes that followed, especially the trends from the 1980s onward.
\begin{description}
	\item[Downsizing and Client-Server Systems (1980s–)]
	      The shift progressed from centralized processing on expensive mainframes to \textbf{client-server systems} (distributed processing) connecting inexpensive \textbf{personal computers (PCs)} or workstations via networks (LANs). This was called 'downsizing'. It dramatically lowered IT implementation costs, promoted IT utilization at the department and individual level (end-user computing), and increased operational flexibility.
	\item[BPR and ERP (1990s–)]
	      In the 1990s, \textbf{BPR (Business Process Re-engineering)} was proposed, which fundamentally re-examined existing business processes based on the premise of IT's evolution. This was the idea of not just using IT as a tool to automate existing tasks, but to redesign (re-engineer) inefficient business processes themselves, assuming IT's capabilities (e.g., ease of information sharing). \textbf{ERP (Enterprise Resource Planning)} packages, which integrally manage a company's core operations (accounting, HR, production, sales, etc.), rapidly spread as the management foundation to realize this BPR.
	\item[The Internet Revolution and SCM (Late 1990s–)]
	      The commercialization of the internet dramatically lowered the cost of information exchange *between* companies. As a result, interest shifted from the optimization of internal company operations (like ERP) to the optimization of inter-company operations, namely \textbf{Supply Chain Management (SCM)}. It became possible to share demand and inventory information in real-time from suppliers to customers via IT networks, enabling efficiency improvements across the entire supply chain.
\end{description}
These transformations (downsizing, BPR, SCM) are the very process by which IT metamorphosed from a mere 'calculator' into a 'weapon for business reform'.
\subsection{Conclusion}
In this lecture, we overviewed the historical transition of IT: from its birth as a tool to supplement the limits of human memory and calculation, through its business use via punched card systems, to the era of electronic computers represented by ENIAC, and finally to the mainframe era culminating in the IBM S/360. This evolution has been supported by the exponential development of semiconductor technology, as symbolized by \textbf{Moore's Law}.
As \textbf{Drucker's} point, mentioned in the 'Deeper Context', indicates, IT was not merely a calculator but the driving force of an \textbf{Information Revolution} that transformed the social structure from an 'industrial society' to a '\textbf{knowledge society}'. This revolutionary change, as Moore's Law suggests, is continuous (even if it slows down).
The practical lesson to be learned from this study is the danger of viewing IT in operations management solely as an efficiency tool (an extension of the abacus or calculator). A \textbf{strategic perspective} is essential, one that constantly questions how IT evolution can enable new business processes (e.g., \textbf{BPR}) or inter-company collaboration (e.g., \textbf{SCM}) and become a source of competitive advantage.
\subsection{List of Key Keywords}
Pascal, Leibniz, Babbage, Drucker, Moore
\vspace{\baselineskip}
IT (Information Technology), ICT (Information and Communication Technology), Bit, Byte, Information Processing, Information Management, Information Exchange, Moore's Law, Punched Card System (PCS), ENIAC, Batch Processing, Time-Sharing System (TSS)
\subsection{Comprehension Check Quiz}
\begin{enumerate}
	\item What is the fourth resource, which has come to be emphasized alongside 'people, goods, and money', as a management resource?
	\item What are the two main aspects of human capability limitations that computers supplement?
	\item What is the basic unit (0 or 1) used to represent all data (numbers, text, images, etc.) inside a computer called?
	\item What is a unit of data consisting of 8 bits called?
	\item What is the basic function of a computer that executes high-speed calculations and operations based on a program?
	\item What is the basic function of a computer that allows for the storage and retrieval of information using databases, etc.?
	\item What is the basic function of a computer that facilitates the exchange of information across geographical and temporal constraints via a network?
	\item What is the empirical rule, named after its proponent, which states that 'the integration density of semiconductors doubles approximately every two years'?
	\item What is the system, initially put into practical use for the US census, that recorded and processed data by punching holes in paper cards?
	\item What is the name of the world's first electronic computer, developed in 1946 using vacuum tubes?
	\item What was the early computer processing method that involved collecting data over a period and processing it all at once?
	\item What is the system, common in 1960s mainframes, that allowed multiple terminals to use CPU and other resources simultaneously by time-slicing?
	\item What is the series name of the mainframe computer announced by IBM in 1964 that became the foundation for subsequent computer architecture?
	\item What did Drucker call the societal form, following agricultural and industrial societies, that was brought about by the computer revolution?
	\item (From AI Supplement) What is the management reform method, proposed in the 1990s, that involves fundamentally redesigning existing business processes based on the premise of IT's capabilities?
\end{enumerate}
\subsubsection*{Answer Key}
1. Information, 2. Memory and calculation, 3. Bit, 4. Byte, 5. Information processing, 6. Information management, 7. Information exchange, 8. Moore's Law, 9. Punched Card System (PCS), 10. ENIAC, 11. Batch processing, 12. Time-Sharing System (TSS), 13. S/360 (System/360), 14. Knowledge society (or Information society), 15. BPR (Business Process Re-engineering)
\section{PCs and Information Communication}
\subsection{Introduction}
In the era when mainframes were dominant, the technological foundation of IT showed temporary stagnation. However, the arrival of the \textbf{personal computer (PC)} in the 1980s completely changed this situation, fundamentally transforming the nature of information systems in corporate operations.
The purpose of these lecture notes is to organize the technological and structural transition from the birth of the PC, through the spread of \textbf{client-server systems}, to the \textbf{internet} becoming the standard IT infrastructure. In particular, we will deeply understand the circumstances that led to the formation of the modern OS market and the impact that network technology evolution had on operations.
\subsection{Key Concepts and Points}
This lecture explained the main concepts that formed the IT infrastructure after the mainframe era.
\subsubsection{The Advent of the Personal Computer (PC)}
In the 1980s, performance improvements in \textbf{microprocessors} gave birth to the \textbf{personal computer (PC)}, usable at the individual level. Unlike mainframes, PCs enabled standalone calculation, data analysis, and report creation, and further realized \textbf{data sharing} by connecting to server computers.
\subsubsection{Client-Server Systems and Distributed Processing}
With the spread of PCs, corporate information systems shifted to \textbf{client-server systems}.
\begin{itemize}
	\item \textbf{Definition}: A processing method based on a combination of a \textbf{client} (customer role), such as a PC, that requests services, and a \textbf{server} that provides them.
	\item \textbf{Distributed Processing}: A form where multiple servers and many PCs (clients) are connected via a network, distributing the load to perform processing.
	\item \textbf{Contrast}: This method is contrasted with \textbf{centralized processing}, where a large mainframe handles all processing.
\end{itemize}
\subsubsection{Evolution of Network Technology (LAN and TCP/IP)}
The change in processing methods was inseparable from the evolution of communication technology.
\begin{itemize}
	\item \textbf{Mainframe Era}: Communication methods were proprietary, set by computer manufacturers (IBM, \textbf{UNIVAC}, etc.), and managed by a central communication controller.
	\item \textbf{Client-Server Era}: \textbf{LAN (Local Area Network)} spread, connecting multiple PCs and servers within a facility. On the LAN, PCs and servers held an equivalent status in communication, and \textbf{TCP/IP} was widely adopted as the standard communication protocol for this connection.
\end{itemize}
\subsubsection{The Spread of the Internet and WWW}
The standardization of \textbf{TCP/IP} on corporate LANs facilitated connection to the external \textbf{internet}.
\begin{itemize}
	\item \textbf{Origin}: The technological foundation of the internet lies in the \textbf{ARPANET}, built by the US Department of Defense's Advanced Research Projects Agency (ARPA); it was originally military technology. Later, universities and research institutions adopted TCP/IP, and networks became interconnected.
	\item \textbf{Ignition for Commercialization (WWW)}: The spread of the \textbf{WWW (World Wide Web)} in the 1990s led to an explosive advance in the commercial use of the internet.
	\item \textbf{HTML}: The WWW is a mechanism that enables the sharing and browsing of documents on the internet by describing them in a language called \textbf{HTML (HyperText Markup Language)}.
\end{itemize}
Today, while mainframes and client-server systems still exist, we are in the '\textbf{Age of the Internet}', where the internet is an indispensable foundation for internal and external corporate communication networks.
\subsection{Application and Case Analysis}
The battle for hegemony over the OS (Operating System) during the dawn of the PC era was a crucial case that determined the structure of today's IT market.
\subsubsection{The Battle for OS Hegemony (Apple vs. IBM/Microsoft)}
The lecture explained the formation process of the two OSs (MacOS and Windows) that currently divide the PC market.
\begin{itemize}
	\item \textbf{Apple's Challenge}: In 1984, Apple, led by \textbf{Steve Jobs}, released the \textbf{Macintosh}, which featured an innovative Graphical User Interface (GUI).
	\item \textbf{IBM's Entry into PCs and OS Development Delay}: The corporate giant IBM also began PC development, but it did not view the PC market as attractive compared to its main industrial computer market. As a result, insufficient resources were allocated to the development of the most crucial software, the \textbf{OS (Operating System)}, and IBM abandoned in-house development.
	\item \textbf{The Rise of Microsoft}: IBM decided to source its OS (\textbf{DOS}) from \textbf{Microsoft}, a fledgling company at the time led by \textbf{Bill Gates}. Furthermore, it also commissioned Microsoft to develop GUI software (later \textbf{Windows}) to counter the Macintosh.
	\item \textbf{The Market's Decision}: As a result, Microsoft's OS (DOS/Windows) became the de facto standard on the IBM PC (and its compatibles) platform, establishing the structure that divides the market with Apple's \textbf{MacOS}.
	\item \textbf{IBM's Fate}: IBM, having relied on an external party for the core component (OS), lost its leadership in the PC market and eventually sold its PC business to \textbf{Lenovo}.
\end{itemize}
\subsection{Deeper Context and Lessons}
The lecture introduced symbolic anecdotes to deepen the understanding of the main topic.
\textbf{\paragraph{Side Topic Name: The Shock of Apple's 1984 Commercial}}
The commercial aired at the launch of the Macintosh in 1984 is remembered as a symbol of generational change and rebellion in the IT industry.
\begin{itemize}
	\item \textbf{Symbol of IBM}: In the CM, a pale, rigid old man dominates a crowd from a screen. This symbolized \textbf{IBM}, which was called '\textbf{Big Blue}' at the time, supported traditional East Coast industries (finance, steel, etc.), and had a conservative corporate culture of dark suits and \textbf{regimental ties}.
	\item \textbf{Symbol of Apple}: Into this scene runs a woman in red hot pants, symbolizing the free culture of the West Coast, holding a hammer, and she smashes the screen.
	\item \textbf{The Message}: This radical depiction was \textbf{Steve Jobs's} powerful message that the monopoly on information by the existing ruler (IBM) would be shattered by a tool for individuals, the Macintosh.
\end{itemize}
\textbf{\paragraph{Side Topic Name: IBM's Misstep: 'Lending the Eaves, Losing the Main House'}}
IBM provided the foundational technology (hardware) for the PC, but the result was that it allowed Microsoft to seize the source of market profit (OS, software). This is described as a lesson in management strategy: '\textbf{Lending the eaves and having the main house taken' (losing the core business)}.
IBM underestimated the importance of the PC market and depended on an external party (Microsoft) for the core OS. Microsoft borrowed IBM's 'eaves' (platform) to spread its own OS, and ultimately seized control of the entire PC ecosystem ('the main house').
\textbf{\subsubsection{AI Supplement: Expansion of Important Points}}
This lecture explained the transition of the IT infrastructure from PCs to the internet. The following three important points supplement the direct impact this transformation had on operations management.
\begin{description}
	\item[Downsizing]
	      Although mentioned in the lecture as 'distributed processing', this transition is referred to in management and IT terminology as '\textbf{downsizing}'. By shifting from expensive, closed mainframes to client-server systems composed of inexpensive, standardized (open) PCs and servers, IT implementation and maintenance costs dropped dramatically. This enabled flexible system implementation not just at the enterprise level but also at the department level, making it easier to respond promptly to the needs of operations on the ground.
	\item[BPR (Business Process Re-engineering)]
	      The spread of client-server systems and LANs made real-time information sharing between departments technologically possible. This technological foundation strongly supported \textbf{BPR (Business Process Re-engineering)}, which was proposed in the 1990s. BPR is a management reform method that, based on the premise of IT's capabilities, aims for dramatic operational improvements such as lead-time reduction and cost cutting by fundamentally redesigning (re-engineering) conventional, fragmented business processes.
	\item[Intranet and Extranet]
	      The technologies of WWW and TCP/IP were not limited to public internet use; they were also applied to the construction of closed networks *within* companies (\textbf{Intranet}) and networks connecting specific *between* trading partner companies (\textbf{Extranet}). The intranet became the foundation for internal information sharing and business systems (e.g., workflow), while the extranet became the foundation for inter-company operational collaboration, such as order processing systems and information sharing, serving as a precursor to later \textbf{SCM (Supply Chain Management)} and \textbf{CRM (Customer Relationship Management)}.
\end{description}
\subsection{Conclusion}
In this lecture, we learned about the process by which the arrival of the PC in the 1980s shifted the form of IT utilization from \textbf{centralized processing} on mainframes to \textbf{distributed processing} via client-server systems. During this period of transformation, Apple's challenging product launch and IBM's strategic misstep (relying externally for the OS) allowed Microsoft to rise, determining the structure of the modern OS market.
Furthermore, the spread of \textbf{TCP/IP} and the \textbf{internet}, which developed from military technology, and the \textbf{WWW} that ignited it, expanded the IT infrastructure from internal corporate LANs to a global network, forming the modern operational environment.
The practical lesson derived from this study, as shown by IBM's 'lending the eaves, losing the main house' example, is the importance of \textbf{platform strategy}. Relying on external parties for the technology or infrastructure (in this case, the OS) that forms the core of one's own operations or business model, even if it seems efficient in the short term, carries the strategic risk of losing market leadership itself in the long term. It strongly suggests that in a wave of innovation, distinguishing what to protect and what to entrust to others divides the life and death of a business.
\subsection{List of Key Keywords}
Steve Jobs, Bill Gates
\vspace{\baselineskip}
Personal Computer (PC), Operating System (OS), Client-Server System, Distributed Processing, Centralized Processing, LAN (Local Area Network), TCP/IP, Internet, WWW (World Wide Web), HTML, Downsizing, BPR (Business Process Re-engineering), Intranet, Extranet
\subsection{Comprehension Check Quiz}
\begin{enumerate}
	\item What is the high-performance integrated circuit that technologically enabled the birth of the personal computer in the 1980s?
	\item What were the large computers that were the center of corporate information processing before the PC appeared?
	\item What is the processing method where one large mainframe computer handles all processing?
	\item What is the processing method where PCs and servers are connected by a network and share processing tasks?
	\item In a distributed processing system, what is the computer (like a PC) that requests services called?
	\item In a distributed processing system, what is the computer that provides services such as file sharing and data processing called?
	\item What is a network that connects computers within a relatively small area, such as a company or university campus, called?
	\item What is the set of communication protocols (rules) used as the standard for the current internet and many LANs, known by its 4-letter acronym?
	\item What is the name of the network, built by the US Department of Defense, that was the origin of the internet?
	\item What is the document sharing and browsing system, known by its 3-letter acronym, that spread in the 1990s and explosively expanded the commercial use of the internet?
	\item What is the markup language, known by its 4-letter acronym, used to describe web pages on the WWW?
	\item What is the company that IBM requested to provide the OS for its PCs?
	\item What is the name of the groundbreaking PC released by Apple in 1984, characterized by its GUI (Graphical User Interface)?
	\item (From AI Supplement) What is the management/IT term for the shift from expensive mainframes to open systems composed of inexpensive PCs and servers?
	\item (From AI Supplement) What is the form of applying WWW technology for information sharing and business systems on a closed *internal* corporate network called?
\end{enumerate}
\subsubsection*{Answer Key}
1. Microprocessor, 2. Mainframe (or General-purpose computer), 3. Centralized processing, 4. Distributed processing, 5. Client, 6. Server, 7. LAN (or Local Area Network), 8. TCP/IP, 9. ARPANET, 10. WWW (World Wide Web), 11. HTML, 12. Microsoft, 13. Macintosh, 14. Downsizing, 15. Intranet
\section{Development of IT Utilization and Business Support}
\subsection{Introduction}
Along with the evolution of computer technology, the ways information systems are utilized in corporations have also advanced and diversified, from simple operational efficiency to weapons of management strategy.
The purpose of these lecture notes is to systematically organize this historical development of IT utilization. We will overview the transition from theoretical frameworks like \textbf{Nolan's Stage Hypothesis}, to the central system concepts of each stage such as \textbf{MIS}, \textbf{DSS}, and \textbf{SIS}, and further, to \textbf{knowledge management}. Ultimately, we will understand how these IT concepts came to merge with modern operations reform (SCM, CRM, etc.).
\subsection{Key Concepts and Points}
This lecture explained several frameworks for classifying the development of IT utilization and the system concepts that symbolize each era.
\subsubsection{Nolan's Stage Hypothesis}
\textbf{Richard Nolan} proposed a 6-stage hypothesis (initially 4 stages, revised to 6 in 1979) for the process of evolution in information system utilization within a company.
\begin{enumerate}
	\item Initiation
	\item Contagion
	\item Control
	\item Integration
	\item Data Administration
	\item Maturity
\end{enumerate}
In particular, it was pointed out that the transition from Stage 3 'Control' to Stage 4 'Integration' is a technological turning point, and it is important that the target tasks shift from limited \textbf{DP (Data Processing)} to integrated \textbf{IT (Information Technology)} utilized across the entire company.
\subsubsection{Classification of Information System Function Development}
The historical transition is broadly divided into the following five stages based on main function/purpose.
\begin{enumerate}
	\item Information systems for operational efficiency (DP)
	\item Information systems for management (MIS)
	\item Information systems for decision support (DSS)
	\item Information systems for strategy realization (SIS)
	\item Information systems for knowledge management and communication
\end{enumerate}
\subsubsection{MIS (Management Information System) and Anthony's Hierarchy}
In the mid-1960s, the concept of \textbf{MIS (Management Information System)} emerged. This was the next stage after operational efficiency, a system providing reporting functions to the management layer.
Its theoretical background is the three-tier model of management control functions by \textbf{Robert Anthony}.
\begin{description}
	\item[Strategic Planning] Determining organizational objectives, resource allocation, and basic policies (Top Management).
	\item[Management Control] Ensuring the efficient and effective acquisition and use of resources to achieve objectives (Middle Management).
	\item[Operational Control] Ensuring the efficient and effective execution of specific tasks (Lower Management).
\end{description}
MIS aimed to support these entire management control activities as a system.
\subsubsection{DSS (Decision Support System)}
In the late 1970s, against the backdrop of evolving interactive UIs and AI technology, the \textbf{DSS (Decision Support System)} emerged. While MIS mainly provided routine information, DSS supports \textbf{non-routine tasks}.
\begin{itemize}
	\item \textbf{Gorry \& Scott Morton} defined DSS as 'an information system that supports \textbf{semi-structured} and \textbf{unstructured} decision-making'.
	\item \textbf{Sprague \& Carlson} proposed, as components of DSS, a database management function, a dialogue management function, and additionally a \textbf{model-base management function}. The model base includes financial models, statistical models, and management science models, which became a major feature of DSS.
\end{itemize}
\subsubsection{SIS (Strategic Information System) and the Value Chain}
In the 1980s, the integration of IT and communication technology progressed, and systems expanded *between* companies. This gave rise to the concept of \textbf{SIS (Strategic Information System)}, which treats IT as a weapon for competitive strategy.
\begin{itemize}
	\item \textbf{Michael Porter's} \textbf{Value Chain} theory provided a framework to explain this phenomenon. It held that strong inter-company collaboration through IT increases customer \textbf{switching costs} (the cost of switching to another company), enabling \textbf{customer lock-in}.
	\item \textbf{Charles Wiseman} defined SIS as 'an information system that supports or shapes planning for acquiring or maintaining one's own competitive advantage, or reducing the advantage of others'.
\end{itemize}
Initially, customer lock-in drew attention, but it later became widely recognized as a weapon for realizing competitive strategy in operations, such as cost reduction and product innovation.
\subsubsection{Knowledge Management and Communication}
In the 1990s, the spread of the \textbf{WWW (World Wide Web)} and technological innovation made it possible to process large volumes of unstructured data that were previously unmanageable, such as documents, static images, videos, and audio.
As a result, communication with stakeholders via internet sites and \textbf{knowledge sharing} (knowledge management) within organizations using \textbf{groupware} became new areas of IT utilization, forming the foundation supporting the \textbf{knowledge economy} era.
\subsection{Application and Case Analysis}
IT utilization in each era manifested as specific systems and social phenomena.
\subsubsection{The Operational Efficiency (DP) Stage}
Systematization in the 1950s and 60s aimed to replace manual \textbf{routine tasks} with computers (operational efficiency).
\begin{itemize}
	\item \textbf{Accounting and Finance}: Turning ledgers into databases, automatic calculation of totals, printing forms.
	\item \textbf{Order Processing}: Managing the series of flows from order receipt to delivery, billing, and payment; automatic creation of related slips such as shipping instructions.
	\item \textbf{Related Concepts}: \textbf{DP} (Data Processing), \textbf{ADP} (Automatic DP), \textbf{EDP} (Electronic DP).
\end{itemize}
\subsubsection{The MIS Boom and Setback}
In the 1960s, an MIS boom occurred in the US. In 1967, an 'MIS Inspection Team' was even dispatched from Japan, and its introduction was pursued with a sense of national crisis.
However, against the high ideal of 'providing the information managers want in a timely manner', the technology of the time was immature, and MIS was derided as '\textbf{M-I-S} (Mistake, Myth)'.
\begin{itemize}
	\item \textbf{Reasons for failure}: (1) Lack of a unified code system for data accumulation, (2) Undeveloped database construction and management technology, (3) Lack of an established management style that presumed IT utilization.
	\item \textbf{Dearden's criticism}: The management scholar \textbf{Dearden} stated, 'A total information system is forever impossible', criticizing that the only path was steady, function-specific system development.
\end{itemize}
On the other hand, the MIS attempt can be evaluated as having shown the direction of integrating operational data (DP) to provide management control information (reporting), laying the groundwork for later IT development.
\subsubsection{Practical Application of Expert Systems}
In the DSS era, \textbf{Expert Systems}, an application of \textbf{AI (Artificial Intelligence)} technology, were put into practical use. These are systems that store expert knowledge as a '\textbf{knowledge base}' (e.g., IF-THEN rules), and an '\textbf{inference engine}' makes judgments like an expert.
\begin{itemize}
	\item \textbf{Application fields}: \textbf{Pattern recognition} (recognizing the meaning of audio or images), \textbf{data mining} (extracting knowledge from large data sets), production scheduling, equipment abnormality diagnosis, etc.
	      * \textbf{Foundational technology}: To realize ambiguous processing where judgment criteria change, technologies such as \textbf{neural networks}, modeled on the human brain, were put to practical use.
\end{itemize}
\subsection{Deeper Context and Lessons}
In addition to the main lecture content, related information and anecdotes were introduced to deepen the context.
\textbf{\paragraph{Side Topic Name: Japan's Sense of Crisis Over MIS Adoption}}
Behind the 1967 visit to the US by the MIS inspection team, centered on the Japan Productivity Center, was a strong sense of crisis regarding the 'computer gap'. The report to Prime Minister Sato at the time concluded, 'The impact of the computer gap on international competitiveness is far more serious than being behind in the nuclear power or space industries'. This sense of crisis led directly to the establishment of the Japan Information Processing Development Center (JIPDEC).
\textbf{\paragraph{Side Topic Name: The Prophecy of MIS Failure and its Refutation by ERP}}
In the 1970s, Dearden predicted that a 'total system' was impossible, but the lecturer for this course points out that this prediction did not hold true in reality. This is because subsequent technological developments (databases, networks) led to the emergence of \textbf{ERP (Enterprise Resource Planning)} in the 1990s, which (to a certain extent) did exactly that: integrate the entire company's business processes and realize much of the world MIS had aimed for.
\textbf{\paragraph{Side Topic Name: Pattern Recognition as Seen in 'Purikura'}}
As a familiar example of pattern recognition, an application of expert systems, 'Purikura' (print club photo booths) was mentioned. The process where a purikura machine automatically recognizes a person's face from an image, accurately identifies the eyes to enlarge them, or adds eyelashes, is a prime example of a computer extracting meaningful features from ambiguous (difficult to formalize) image information and executing an operation (processing).
\textbf{\subsubsection{AI Supplement: Expansion of Important Points}}
This lecture explained SIS and the value chain (mainly inter-company and customer-facing strategies) as strategic IT utilizations of the 1980s. However, another important operations reform concept, discussed against the backdrop of improving IT capabilities in the same period and which spread explosively in the 1990s, is \textbf{BPR (Business Process Re-engineering)}.
\begin{description}
	\item[BPR (Business Process Reengineering)]
	      \textbf{BPR} is a management reform method advocated by Michael Hammer and James Champy. This is the idea of not just using IT for the automation of existing tasks (Efficiency), but of fundamentally and dramatically \textbf{Re-engineering} existing Business Processes, based on the premise of IT's capabilities (e.g., shared databases, communication networks).
	      Under the slogan, 'Don't automate, obliterate', it aimed for dramatic improvements in cost, quality, service, and speed, not by simply connecting conventionally specialized processes with IT, but by reviewing the processes themselves from scratch with IT in mind. While SIS focused on external competitive advantage, BPR is a concept that focused on the fundamental reform of internal operations, and it is an indispensable point in discussing the fusion of IT and operations reform.
\end{description}
\subsection{Conclusion}
In this lecture, we overviewed the historical process by which IT utilization developed and expanded from 'operational efficiency (DP)' to 'management (MIS)', 'decision support (DSS)', 'strategy (SIS)', and finally 'knowledge management'.
Despite the failure of MIS and Dearden's criticism, its ideal was later realized by ERP, and SIS clearly positioned IT as a weapon for operations reform.
The practical lesson derived from this transition is that IT is not merely a 'tool for efficiency' or a 'tool for management', but a \textbf{strategic foundation} that defines the very nature of operations and influences a company's competitive advantage.
As shown at the end of the lecture, the main modern operations reform concepts—\textbf{SCM (Supply Chain Management)}, \textbf{CRM (Customer Relationship Management)}, and \textbf{Knowledge Management}—are inseparably fused with IT infrastructures like ERP, the internet, and groupware, respectively. IT strategy and operations strategy can no longer be discussed separately.
\subsection{List of Key Keywords}
Richard Nolan, Robert Anthony, Dearden, Gorry \& Scott Morton, Sprague \& Carlson, Michael Porter, Charles Wiseman
\vspace{\baselineskip}
Data Processing (DP), Management Information System (MIS), Decision Support System (DSS), Expert System, Neural Network, Pattern Recognition, Data Mining, Strategic Information System (SIS), Value Chain, Switching Cost, ERP (Enterprise Resource Planning), Groupware, Knowledge Management
\subsection{Comprehension Check Quiz}
\begin{enumerate}
	\item Who proposed the hypothesis that IT utilization within a company evolves through six stages (Initiation, Contagion, Control, Integration, etc.)?
	\item In Nolan's Stage Hypothesis, the transition from 'Control' to which stage is considered the technological turning point, where a shift from DP to IT occurs?
	\item What is the system concept from the 1950s-60s that primarily aimed to automate and improve the efficiency of manual, routine tasks such as accounting and order processing?
	\item Which management scholar classified management control functions into three tiers: 'Strategic Planning', 'Management Control', and 'Operational Control'?
	\item What is the management control system concept that boomed in the 1960s but was also derided as a 'failure (Mistake/Myth)' due to technical immaturity?
	\item What is the system that supports non-routine decision-making by incorporating a 'model base' containing financial models, statistical models, etc.?
	\item Who were the central figures who defined DSS (Decision Support System) as supporting 'semi-structured and unstructured decision-making'?
	\item What is the system, applying AI technology, that attempts to make expert-like judgments by storing expert knowledge in a 'knowledge base' and using an 'inference engine'?
	\item What is the system concept, known by its 3-letter acronym, that emerged in the 1980s, viewing IT as a weapon for competitive strategy to gain or maintain a competitive advantage?
	\item What is the strategy of using IT to increase the costs (monetary, time, psychological) for a customer to switch to another company's products or services?
	\item What is Porter's strategic framework that analyzes where IT creates value (margin) by classifying a company's activities into 'primary activities' and 'support activities'?
	\item Following the spread of the WWW in the 1990s, what management method, focusing on sharing and utilizing unstructured data like documents, images, and videos, became important?
	\item What is the integrated business software package, known by its 3-letter acronym, that emerged in the 1990s and is said to have realized the 'total system' Dearden predicted was 'impossible'?
	\item (From AI Supplement) What is the management reform method, proposed in the 1990s, that involves fundamentally redesigning existing business processes based on the premise of IT's capabilities?
	\item As a conclusion of the lecture, modern IT concepts were said to be 'fused' with operations reform concepts. What IT system is strongly linked with Supply Chain Management (SCM)?
\end{enumerate}
\subsubsection*{Answer Key}
1. Richard Nolan, 2. Integration, 3. DP (Data Processing), 4. Robert Anthony, 5. MIS (Management Information System), 6. DSS (Decision Support System), 7. Gorry \& Scott Morton, 8. Expert System, 9. SIS (Strategic Information System), 10. (Raising) Switching Costs, 11. Value Chain, 12. Knowledge Management, 13. ERP (Enterprise Resource Planning), 14. BPR (Business Process Re-engineering), 15. ERP (Enterprise Resource Planning)
\end{document}
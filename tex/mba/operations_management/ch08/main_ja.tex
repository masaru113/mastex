\documentclass[uplatex,a4j,12pt,dvipdfmx]{jsarticle}
\usepackage{amsmath,amsthm,amssymb,bm,color,enumitem,mathrsfs,url,epic,eepic,ascmac,ulem,here,ascmac}
\usepackage[letterpaper,top=2cm,bottom=2cm,left=3cm,right=3cm,marginparwidth=1.75cm]{geometry}
\usepackage{booktabs}
\usepackage[english]{babel}
\usepackage[dvipdfm]{graphicx}
\usepackage[hypertex]{hyperref}

\title{オペレーションマネジメント 第8回 講義ノート: \\ カスタマーリレーションシップマネジメント(1):コンセプトと目的}
\author{Masaru Okada}
\date{\today}

\begin{document}
\maketitle
\tableofcontents

\section{講義資料整理}

\subsection{はじめに}

本講義「オペレーションマネジメント 第8章」では、現代企業の競争優位の源泉となる\textbf{CRM(Customer Relationship Management:顧客関係管理)}のコンセプトとその目的について、体系的に学習する。

講義の構成は、まずマーケティングオペレーションの世界への導入として「ロジカルマーケティング」の演習を行い、顧客ニーズの本質的理解について考える。続いて、CRMの定義とマーケティング情報システムの中核であるPOS(販売時点管理)システムのメカニズム、およびPOSデータの分析手法(DIKWモデル等)を詳細に検討する。最後に、CRMの核心部分である「リレーションシップマーケティング」について、従来のパラダイムとの対比を通じてその本質を明らかにする。

これらは単なるシステムの解説にとどまらず、市場環境の変化(マクロ環境)と技術の進化(テクノロジー)がいかに企業の経営戦略を変容させてきたかという、経営学的な文脈で理解されるべきテーマである。

---

\subsection{主要な概念と論点}

\subsubsection{ロジカルシンキングによる本質的ニーズの探索}

CRMの出発点は「顧客理解」にあるが、顧客の言葉をそのまま受け取るだけでは不十分である。講義では水着メーカーの商品開発担当者の視点に立ち、競泳選手のニーズをロジックツリーを用いて構造化するプロセスが示された。

\paragraph{【事例モデル】競泳用水着開発におけるニーズの階層構造}
顧客(競泳選手)へのインタビューから得られた「着心地が悪いのは嫌」「生地が小さい方がいい」という声を、ロジカルシンキングを用いて以下のように分解・構造化する。

\begin{description}
	\item[1. 表面的なニーズ(VOC)]
	      「着心地が良い水着が欲しい」「生地部分が小さく、伸縮自在な水着が良い」
	      \begin{itemize}
		      \item これらは顧客が認識している顕在的な要望である。しかし、これに応えるだけでは競合他社との差別化は難しい。
	      \end{itemize}
	\item[2. 根本的な悩み(Why?)]
	      なぜ生地を小さくしたいのか?その背後には「動きを妨げられたくない」「水の抵抗を減らしたい」という機能的な悩みや欲求が存在する。
	\item[3. 本質的なニーズ(Goal)]
	      さらに「なぜ?」を繰り返すと、競泳選手としての究極の目的、すなわち\textbf{「どうすれば速く泳げるのか(タイム短縮)」}という一点に集約される。
	\item[4. 実際に最も有効な解決策(Solution)]
	      「速く泳ぐ」という本質的ニーズを解決するためには、顧客が当初要望していた「生地を小さくする(露出を増やす)」こととは真逆の、\textbf{「表面素材を変更し、体を流線型にする(全身を覆う)」}というアプローチが正解となる場合がある。
\end{description}

この事例は、CRMにおいてデータの向こう側にある「顧客の解決すべき課題(Job to be done)」を見抜く洞察力が不可欠であることを示唆している。

\subsubsection{CRMとマーケティング情報システムの構造}

伝統的なマーケティング情報システムは、POSデータに代表される「取引データ」を中心に発展し、品揃えや価格政策に活用されてきた。しかし、CRMはこれを超え、企業が顧客の問題解決を通じて長期的・継続的な便益(ベネフィット)を共有することを狙いとする。

CRMのマネジメント・プロセスは、大きく以下の2つの戦略的側面に分類される。

\begin{description}
	\item[1. 顧客インターフェース戦略]
	      先端的な情報技術を用いて、顧客との接点(インターフェース)から「顧客の声」を収集し、電子化するプロセス。
	      \begin{itemize}
		      \item データは一元管理され、企業内で共有される必要がある。
		      \item 近年ではインターネット上のコミュニティサイトなどを通じ、顧客同士のインタラクション(相互作用)や、製品開発プロセスへの主体的参加も含まれる。
	      \end{itemize}
	\item[2. 顧客ナレッジ分析戦略]
	      収集・蓄積されたデータを分析し、マーケティングの意思決定に資する「知識(ナレッジ)」へと変換するプロセス。
\end{description}

重要なのは、これらのプロセスを通じて効果を生み出すには、単なるツール導入ではなく\textbf{「組織革新」}を伴う必要があるという点である。

\subsubsection{POSシステムのメカニズムと普及}

POS(Point Of Sales:販売時点管理)システムは、マーケティングデータの収集源として現代小売業に不可欠なインフラである。

\paragraph{POSシステムの構成要素}
POSシステムは、POSレジ端末、ストアコントローラー(店舗用コンピュータ)、およびそれらを結ぶネットワークで構成される。
\begin{itemize}
	\item \textbf{ストアコントローラーの定義}: 店舗ごとに設置され、POSレジや在庫検品用のハンディターミナルなどの周辺機器と連携し、店内の様々な業務を集中的に管理するコンピュータ。
\end{itemize}

\paragraph{価格決定のメカニズム:PLUとNON-PLU}
バーコード処理には2つの方式が存在する。

\begin{enumerate}
	\item \textbf{PLU(プライス・ルック・アップ:Price Look Up)}:
	      \begin{itemize}
		      \item 商品に印刷されたJANコードなどのバーコードには「価格データ」は含まれていない。
		      \item スキャナでコードを読み取ると、ストアコントローラーに問い合わせを行い、商品マスターから価格データを検索(Look Up)して表示する。
		      \item \textbf{メリット}: 店舗ごとに異なる価格設定が可能であり、特売時の価格変更もシステム上の設定変更のみで瞬時に反映できる。
	      \end{itemize}
	\item \textbf{NON-PLU(ノン・プライス・ルック・アップ)}:
	      \begin{itemize}
		      \item 生鮮食料品(肉、魚、野菜)など、個体ごとに重量により価格が異なる商品に用いられる。
		      \item 小売店側で発行するバーコードの中に直接価格情報を埋め込む方式である。
	      \end{itemize}
\end{enumerate}

\paragraph{日本におけるPOSの普及背景}
日本においてPOSシステムが爆発的に普及した契機は、\textbf{1989年の消費税導入}である。複雑化するレジ処理の効率化を目的として導入が進み、現在ではコンビニエンスストアやスーパーマーケットで95\%以上の導入率を誇るほか、外食産業(ファミレス、カフェ)など小売店以外でも標準装備となっている。

\subsubsection{DIKWモデル:データから知恵への昇華}

POSデータ活用において、ラッセル・アコフ(Ackoff, 1989)が提唱した「データ、情報、知識、知恵」の4階層モデルは、分析の深度を理解する上で極めて重要である。講義では、アコフの定義に加え、講師独自の解釈による実践的な定義も示された。

\begin{table}[h]
	\centering
	\caption{DIKW階層モデルの詳細比較}
	\label{tab:dikw}
	\begin{tabular}{|p{3cm}|p{6cm}|p{6cm}|}
		\hline
		\textbf{階層}                       & \textbf{Ackoffによる定義}      & \textbf{本講義(講師)の解釈}                            \\
		\hline
		\textbf{知恵\newline (Wisdom)}      & 判断すること、またそれによって新しい理解を得ること & \textbf{意思決定や判断により起こした行動の結果の正しさ}と、その正しさを証明する情報 \\
		\hline
		\textbf{知識\newline (Knowledge)}   & 役に立つ情報の集まり                & 経験から得られた情報に関して、体系化から\textbf{行動意思決定・判断の役に立つ情報}  \\
		\hline
		\textbf{情報\newline (Information)} & 意味が与えられたデータ               & 誰でも同一の理解ができるように、\textbf{分析により意味付けされたデータ}       \\
		\hline
		\textbf{データ\newline (Data)}       & 存在しているものを示す記号             & 存在しているものの中で、特に\textbf{文章や図で表現された記号}            \\
		\hline
	\end{tabular}
\end{table}

このモデルにおいて重要なのは、単なる「数値の羅列(データ)」を、関連付けによって「意味(情報)」に変え、そこから一般性・普遍性を導き出して「仮説(知識)」とし、最終的に「実行(知恵)」に移すというプロセスの循環である。

---

\subsection{応用と事例分析}

\subsubsection{POSデータの具体的活用事例}

POSデータ分析は、単に「何が売れたか」を知るだけでなく、以下のような高度なマーケティング施策に直結している。

\begin{itemize}
	\item \textbf{クラスター分析による売り場づくり}:
	      顧客の購買履歴からライフスタイルを分析し、「健康重視層」「子育て家族層」「価格重視層」などのグループ(クラスター)を抽出。各層に響く棚割りや品揃えを行う。
	\item \textbf{併売分析(バスケット分析)}:
	      一緒に購入される商品の組み合わせを発見する。
	      \begin{itemize}
		      \item \textbf{事例}: 魚介類と「キムチ鍋の素」「寄せ鍋の素」の同時購入が多いことを発見し、鮮魚コーナーの脇に鍋の素を陳列することで、「ついで買い」を誘発し客単価を向上させる。
	      \end{itemize}
	\item \textbf{POSデータ分析から4Pへの展開}:
	      POSデータは4P(Product, Price, Place, Promotion)のうち、特に\textbf{Price(価格政策)}と\textbf{Promotion(販売促進)}において威力を発揮する。
	      \begin{itemize}
		      \item スーパーのチラシ作成において、「いつ」「どの商品を」「いくらで」掲載すれば最大の集客効果が得られるか、過去のデータを基に科学的に決定される。
	      \end{itemize}
\end{itemize}

\subsubsection{事例1:JR東日本「次世代自動販売機」}

JR東日本ウォータービジネスが展開した「次世代自販機」は、CRMとデジタルサイネージを融合させた革新的な事例である。

\paragraph{システムの概要と特徴}
\begin{itemize}
	\item \textbf{インターフェース}: 従来の物理ボタンや商品見本を排除し、全面に47インチの大型タッチパネルディスプレイを採用。
	\item \textbf{デジタルサイネージ機能}: 人がいない時は広告媒体として機能し、人が前に立つと人感センサーが感知して商品棚を表示する。
	\item \textbf{属性識別}: 上部のカメラセンサーにより、購入者の「性別」と「年齢層」を即座に識別する。
	\item \textbf{オペレーション効率化}: 在庫データと連動しており、売り切れ商品は画面に表示されないため、購入時の失望を防ぐ。また、ネットワーク経由で在庫管理を行うため、補充業務の手間が簡略化される。
	\item \textbf{災害時対応}: 災害発生時には、遠隔操作により無料で飲料を提供するインフラとしての機能も備えている。
\end{itemize}

\paragraph{CRM的意義:コンテキスト・マーケティング}
本機は「季節・時間帯」と「属性(性別・年代)」のデータを組み合わせ、最大3つの\textbf{「オススメ商品」}を提案する機能を持つ。
これにより、従来は取得不可能だった「属性別の購買データ」が逐次サーバーに蓄積され、マーケティングデータとしての価値を生み出すと同時に、その場のコンテキスト(文脈)に合わせた販売促進を実現している。

\subsubsection{事例2:コンビニ「冷やし中華」の仮説検証サイクル}

\paragraph{背景と課題}
「冷やし中華は夏に売れる」という通説があるが、実際には春や秋でも売れる日があり、逆に夏でも売れない日がある。この変動を予測できないと、品切れによる機会損失や、廃棄ロスが発生する。

\paragraph{分析とアクション}
\begin{enumerate}
	\item \textbf{データの関連付け(Data $\rightarrow$ Information)}:
	      日別のPOSデータと気象データを突き合わせた結果、「前日よりも気温が上がった日(暖かくなった日)」に売上が伸びる傾向を発見した。
	\item \textbf{一般性の導出(Information $\rightarrow$ Knowledge)}:
	      「絶対的な気温」ではなく、「気温の変化(前日差)」が消費者の購買意欲を刺激するという法則(知識)を導き出した。
	\item \textbf{仮説の実行(Knowledge $\rightarrow$ Wisdom/Action)}:
	      「明日は気温が急上昇するから、通常より多く発注しよう」という仮説に基づき発注量を調整。
	\item \textbf{結果}:
	      売上の最大化と廃棄ロスの削減を同時に達成。この結果自体が新たなデータとなり、次の分析へ還流される。
\end{enumerate}

---

\subsection{深層背景と教訓}

\subsubsection{マーケティング・パラダイムの転換}

本講義の核心は、単なるIT活用論ではなく、マーケティングの基本思想(パラダイム)が「刺激・反応パラダイム」から「関係性パラダイム」へと劇的にシフトしている点の理解にある。

\paragraph{1. 刺激・反応パラダイム(伝統的マーケティング)}
\begin{itemize}
	\item \textbf{顧客観}: 企業が与える刺激(広告・販促)に反応する「受動的な主体」。
	\item \textbf{関係性}: 企業対顧客全体という「1対N」の関係。顧客を一様な集団(マス)として捉える。
	\item \textbf{アプローチ}: 需要は潜在的に存在するという前提で、企業主導で製品を提供する「プロダクト・アウト(プッシュ型)」。
	\item \textbf{焦点}: その時々の商品と貨幣の交換という「短期的取引」の最大化。
\end{itemize}

\paragraph{2. 関係性パラダイム(リレーションシップマーケティング)}
\begin{itemize}
	\item \textbf{顧客観}: 製品開発プロセス等に積極的に関与する「能動的な主体」。
	\item \textbf{関係性}: 属性や履歴に応じて識別された「1対1(One to One)」の双方向的な関係。
	\item \textbf{アプローチ}: 企業と顧客のインタラクションの中で需要を共創する「マーケット・イン(プル型)」。
	\item \textbf{焦点}: 協調的な関係構築による「長期的・継続的な便益」の共有。
\end{itemize}

この転換の背景には、市場の成熟により新規顧客獲得が困難になったこと、および多品種少量生産時代への移行により顧客ニーズが多様化したことがある。

\subsubsection{【寄り道トピック】街の電器屋さんに学ぶ「究極のアナログCRM」}
講義内では、CRMの本質を理解するための好例として「地域の電器屋さん」のエピソードが語られた。
\begin{itemize}
	\item 彼らはPOSデータを持たないが、顧客(特に高齢者)の家庭状況を熟知している。
	\item 「テレビの配線が分からない」「ビデオの使い方が難しい」といった、直接の利益にならない小さな困りごとに即座に駆けつけ対応する。
	\item このような日常的な接触と問題解決の積み重ねが、強固な\textbf{「信頼関係」}を構築する。
	\item 結果として、冷蔵庫や洗濯機といった高額商品の買い替え需要が発生した際、価格比較されることなく「あなたから買いたい」と指名される関係性が成立する。
	\item また、訪問時に「電球が切れかかっている」等の新たな需要を発見したり、新製品がいかに生活を便利にするかを直接提案する機会も得られる。
\end{itemize}
これは、CRMの目的が「データの収集」ではなく、\textbf{「顧客の抱える問題を解決し、長期的な関係を築くこと」}にあることを如実に示している。ITやPOSシステムは、この「街の電器屋さん」的な対応を、規模を拡大して効率的に行うための手段に過ぎないのである。

\subsubsection{AIによる補足:重要論点の拡張}

講義テキストの文脈を補完するため、以下のMBA重要概念を補足する。

\begin{description}
	\item[LTV(Life Time Value:顧客生涯価値)]
	      関係性パラダイムにおける最重要KPI。一回の取引での利益ではなく、一人の顧客が生涯にわたって企業にもたらす利益の総額。CRMの経済的な目的は、顧客維持率(リテンションレート)を高め、アップセル・クロスセルを通じてLTVを最大化することにある。

	\item[RFM分析]
	      顧客データを実務で活用する際の基本フレームワーク。
	      \begin{itemize}
		      \item \textbf{R (Recency)}: 最終購入日(最近いつ買ったか)
		      \item \textbf{F (Frequency)}: 購入頻度(どのくらいの頻度で買うか)
		      \item \textbf{M (Monetary)}: 購入金額(いくら使ったか)
	      \end{itemize}
	      これら3指標で顧客をランク付けし、優良顧客の定義や、離反しそうな顧客へのフォロー施策を決定する。

	\item[カスタマーエクイティ(Customer Equity)]
	      企業の全顧客のLTVの総和。従来の「ブランドエクイティ(ブランド資産価値)」に対し、顧客そのものを資産として捉える考え方であり、CRM経営の究極的な企業価値指標となる。
\end{description}

---

\subsection{結論}

本講義の結論として、CRMとは単なるITシステムの導入やデータの蓄積ではないことが明確になった。

1.  \textbf{目的の明確化}: POSデータは「数値の羅列」に過ぎない。これを「情報」「知識」「知恵」へと昇華させ、仮説検証サイクルを回す分析能力こそが重要である。
2.  \textbf{パラダイムの変革}: 企業は「売るためのマーケティング」から「関係を築くためのマーケティング」へと発想を転換しなければならない。
3.  \textbf{実践的教訓}: JRの自販機や街の電器屋さんの事例が示す通り、テクノロジーの有無にかかわらず、\textbf{「顧客の文脈(コンテキスト)を理解し、その問題を解決することで信頼を得る」}という基本姿勢こそが、長期的利益を生み出す唯一の道である。

次回以降は、これらのコンセプトを前提として、より具体的なCRMの手法論へと学習を進める。

---

\subsection{重要キーワード一覧}

\textbf{人物}:
ラッセル・アコフ

\vspace{\baselineskip}

\textbf{理論・概念}:
CRM(カスタマーリレーションシップマネジメント)、ロジカルシンキング、VOC(顧客の声)、DIKWモデル、POSシステム、ストアコントローラー、PLU(プライス・ルック・アップ)、NON-PLU、クラスター分析、併売分析(バスケット分析)、マーケティング・パラダイム、刺激・反応パラダイム、関係性パラダイム、プロダクト・アウト、マーケット・イン、4P(製品・価格・流通・販促)、LTV(顧客生涯価値)、RFM分析、デジタルサイネージ、JANコード、ワン・トゥ・ワン・マーケティング

---

\subsection{理解度確認クイズ}

以下の問に答え、本講義の主要概念の理解度を確認してください。

\begin{enumerate}
	\item ロジカルシンキングを用いて顧客の声を深掘りする際、表面的な要望の奥にある「根本的な悩み」を解決するための最終的な狙い(Goal)を何と呼ぶか(水着の例では「速く泳ぐこと」)。
	\item CRMにおいて、顧客との接点でデータを収集する戦略を「顧客インターフェース戦略」と呼ぶのに対し、収集したデータを分析・活用する戦略を何と呼ぶか。
	\item アコフのDIKWモデルにおいて、データに意味が与えられ、誰でも同一の理解ができるようになった状態を何と呼ぶか。
	\item 本講義における「知識(Knowledge)」の定義として、情報の体系化から導き出される、意思決定や判断の役に立つ法則性や何を指すか。
	\item POSシステムにおいて、バーコードに価格情報を持たせず、ストアコントローラーのマスターデータを参照して価格を表示する方式をアルファベット3文字で何というか。
	\item 日本においてPOSシステムが急速に普及する契機となった、1989年の出来事は何か。
	\item 生鮮食品などで用いられる、店内で発行するバーコードの中に直接価格情報を埋め込むPOSの方式を何というか。
	\item POSデータ分析の一種で、魚介類と鍋の素のように「一緒に購入される商品」の組み合わせを発見する分析手法を何というか。
	\item 「刺激・反応パラダイム」において、企業と顧客の関係性は「1対1」ではなく、どのように表現されるか。
	\item 「関係性パラダイム」において、需要は企業が一方的に作るものではなく、顧客とのインタラクションを通じてどうするものと考えられるか。
	\item JR東日本の次世代自販機において、購入者が前に立った際にカメラセンサーが識別する2つの属性情報は何か。
	\item コンビニの冷やし中華の事例で、売上増加のトリガーとして発見された気象条件の特性は「絶対的な気温の高さ」ではなく何か。
	\item リレーションシップマーケティングが重視する、一回の取引での利益ではなく、長期間にわたって便益を共有するという視点を何的視点と呼ぶか。
	\item 街の電器屋さんの事例で示された、顧客の抱える問題を解決し続けることで構築される、CRMの基盤となるものは何か。
	\item (AI補足論点)顧客を「Recency」「Frequency」「Monetary」の3つの指標で分類・分析する手法を何と呼ぶか。
\end{enumerate}

\subsubsection*{解答一覧}
1. 本質的なニーズ, 2. 顧客ナレッジ分析戦略, 3. 情報(Information), 4. 仮説, 5. PLU, 6. 消費税の導入, 7. NON-PLU, 8. 併売分析(またはバスケット分析), 9. 1対N(または1対多), 10. 生み出していく(共創する), 11. 性別と年齢層(年代), 12. 前日との気温差(暖かくなった感覚), 13. 長期的視点, 14. 信頼関係, 15. RFM分析

\section{コンセプトと目的}

\subsection{はじめに:講義の全体像と狙い}
本講義は、オペレーション・マネジメントの一環として、企業の持続的競争優位の源泉となる\textbf{「カスタマー・リレーションシップ・マネジメント(CRM)」}のコンセプトと目的を体系的に学習するものである。

現代の経営環境において、企業は単に製品を市場に投入するだけでは生存できない。顧客が抱える本質的な課題(ペインポイント)を的確に把握する\textbf{マーケティング・リサーチ}の能力と、それを継続的な関係性へと昇華させる\textbf{オペレーション・システム}の構築が不可欠である。

本講義では、ロジカル・シンキングを用いたニーズ分析の基礎から入り、マーケティングパラダイムの歴史的転換、そして最新のデジタル技術(IoT、デジタルサイネージ)を駆使した具体的なCRMの実践事例までを網羅する。特に、JR東日本ウォータービジネスの「次世代自動販売機」の事例を通じて、フロントエンド(顧客接点)の革新がバックエンド(物流・在庫管理)の効率化にどう直結するか、そのメカニズムを深く解剖する。

\subsection{主要な概念と論点}

\subsubsection{1. マーケティング・リサーチとロジカル・シンキング}
講義の冒頭では、マーケティングの出発点となる「顧客ニーズの把握」について、競泳用水着の開発を題材にしたケーススタディを通じて考察する。ここでは、顧客の言葉を鵜呑みにすることの危険性と、論理的思考による「真因」の特定プロセスが重要となる。

\paragraph{【ケーススタディ】競泳用水着の商品開発}
\textbf{背景}: あなたは水着メーカーの開発担当者として、競泳選手へのインタビュー(定性調査)を実施した。
\begin{itemize}
	\item \textbf{顧客の声(VOC: Voice of Customer)}:
	      \begin{enumerate}
		      \item 「着心地が悪い水着は泳ぎにくいので、あまり着たくない」
		      \item 「体が引っ張られると、泳ぎにくくなる気がする」
		      \item 「ちょっと恥ずかしいけれど、生地部分が小さい水着が良いのではないか」
	      \end{enumerate}
\end{itemize}

\paragraph{思考の分岐点:表層的対応 vs 本質的解決}
このVOCに対し、開発者がどの深度で思考するかによって、アウトプット(製品)の価値は劇的に変化する。

\begin{description}
	\item[\textbf{アプローチA:表層的ニーズ(Wants)への対応}] \mbox{}\\
	      顧客の言葉を「正解」としてそのまま受け取るアプローチ。
	      \begin{itemize}
		      \item \textbf{解釈}: 「生地が小さいのが良い」と言われた $\rightarrow$ その通りにする。
		      \item \textbf{製品}: 露出度が高く、伸縮性が高く、生地面積を極限まで小さくした水着。
		      \item \textbf{評価}: 顧客の「意見」は満たしているが、競泳選手としての「目的」を最大化しているとは限らない。
	      \end{itemize}

	\item[\textbf{アプローチB:本質的ニーズ(Needs)の深掘り}] \mbox{}\\
	      「なぜ(Why)」を繰り返し、顧客自身も言語化できていない根本課題に到達するアプローチ。
	      \begin{itemize}
		      \item \textbf{論理の深化プロセス}:
		            \begin{enumerate}
			            \item \textbf{Why?} なぜ「体が引っ張られる」「着心地が悪い」と困るのか? $\rightarrow$ タイムが落ちるから。
			            \item \textbf{Goal Definition}: 競泳選手の究極の目的は「\textbf{速く泳ぐこと}」である。
			            \item \textbf{Cause Analysis}: 「引っ張られる感覚」の物理的正体は何か? $\rightarrow$ それは\textbf{「水の抵抗(Drag)」}である。
			            \item \textbf{Solution Hypothesis}: 抵抗を減らすことが真の解決策である。生地を減らすことは一つの手段に過ぎず、逆に表面素材を変える、あるいは身体を締め付けて流線型にする方が抵抗は減るかもしれない。
		            \end{enumerate}
		      \item \textbf{製品}: 表面素材をサメ肌のように加工した水着や、全身を覆い筋肉のブレを抑えて流線型を保つ高機能水着。
		      \item \textbf{結論}: これは当初の「着心地が良い」「生地が小さい」という要望とは逆行する可能性があるが、「速く泳ぎたい」という\textbf{真のニーズ(Insignt)}に対する正解である。
	      \end{itemize}
\end{description}

\subsubsection{2. マーケティングパラダイムの歴史的転換}
企業と顧客の関係性は、経済環境の変化に伴い「取引(トランザクション)」中心から「関係性(リレーションシップ)」中心へと大きくシフトしている。

\paragraph{パラダイムI:取引マーケティング(Traditional Transaction Marketing)}
高度経済成長期のような「作れば売れる」時代に発展したモデル。
\begin{itemize}
	\item \textbf{焦点}: 一回ごとの「取引」の成立と効率化。
	\item \textbf{データ活用}:
	      \begin{itemize}
		      \item \textbf{POSデータ(Point of Sales)}: 「何が、いつ、いくらで売れたか」という結果データのみを重視。
		      \item \textbf{活用領域}: 在庫補充、棚割(スペースマネジメント)、短期的な価格政策(値引き等)、販売促進。
	      \end{itemize}
	\item \textbf{限界}: 誰が買ったかが見えないため、顧客の離反を防ぐ手立てがなく、常に新規顧客を獲得し続けなければならない(穴の空いたバケツ状態)。
\end{itemize}

\paragraph{パラダイムII:リレーションシップ・マーケティング(Relationship Marketing)}
市場の成熟化に伴い登場した、顧客との長期的な絆を資産とするモデル。
\begin{itemize}
	\item \textbf{定義}: 企業が顧客の抱える問題を解決することを通じて、双方が長期的・継続的に便益(ベネフィット)を共有する関係。
	\item \textbf{CRMの登場}: この概念を実現するための具体的な経営手法・オペレーションが\textbf{CRM(Customer Relationship Management)}である。
	\item \textbf{目的}:
	      \begin{itemize}
		      \item 単発の売上ではなく、\textbf{LTV(顧客生涯価値)}の最大化。
		      \item 顧客満足(CS)を超えた、顧客ロイヤルティ(忠誠心)の獲得。
	      \end{itemize}
\end{itemize}

\subsubsection{3. CRMを構成する2つの戦略軸}
CRMは漠然としたスローガンではなく、「インターフェース(入口)」と「ナレッジ(出口)」という2つの戦略機能によって実装される。

\paragraph{A. 顧客インターフェース戦略(Input Strategy)}
顧客との接点を管理し、情報を吸い上げる仕組み。
\begin{itemize}
	\item \textbf{技術的側面}: インターネット、コールセンター、アプリ、IoT端末などの先端技術を用いて、顧客の声や行動データをデジタル化・収集する。
	\item \textbf{組織的側面}: 収集されたデータは一部署(例:営業部だけ)で独占せず、全社的に\textbf{一元管理・共有}される必要がある。これが「組織の壁(サイロ)」を打破する組織革新を要求する。
	\item \textbf{進化}: 近年では、コミュニティサイトなどの「場」を提供することで、顧客同士のインタラクション(相互作用)そのものをインターフェースとして取り込んでいる。
\end{itemize}

\paragraph{B. 顧客ナレッジ分析戦略(Output Strategy)}
蓄積されたデータを分析し、価値ある情報(ナレッジ)へ変換する仕組み。
\begin{itemize}
	\item \textbf{プロセス}: データウェアハウス等に蓄積された膨大な履歴データを、データマイニング等の手法で解析する。
	\item \textbf{活用}:
	      \begin{itemize}
		      \item マーケティングの意思決定(ターゲット選定、商品開発)。
		      \item 個別の顧客に対する最適化されたオファー(One-to-Oneマーケティング)。
	      \end{itemize}
\end{itemize}

\subsection{応用と事例分析:JR東日本「次世代自動販売機」}

講義後半では、CRMとSCM(サプライチェーン・マネジメント)が高度に融合した事例として、JR東日本ウォータービジネスが展開する「次世代自動販売機」を詳細に分析する。これは単なる「ハイテク自販機」ではなく、リテール・オペレーションの革命である。

\subsubsection{1. フロントエンドの革新(顧客体験とデータ収集)}
従来の自販機とは一線を画すインターフェースにより、これまでにない質の高いデータを収集している。

\begin{itemize}
	\item \textbf{デジタルサイネージ(電子看板)化}:
	      \begin{itemize}
		      \item 物理的な商品見本を廃止し、47インチの大型タッチパネルディスプレイを採用。
		      \item \textbf{動的なコンテンツ配信}: 人がいない時は広告媒体として機能し、人が立つと店舗(商品棚)に切り替わる。気温や時間帯に合わせて表示コンテンツを変えることで、購買意欲を刺激する。
	      \end{itemize}
	\item \textbf{センシングと属性推定}:
	      \begin{itemize}
		      \item \textbf{機能}: 上部のカメラセンサーが購入者の「性別」と「年齢層」を識別する。
		      \item \textbf{精度}: フィールドテストでは約75\%の精度で合致。複数人の場合は1名を判定する。マーケティングデータとしては十分な統計的有意性を持つ。
	      \end{itemize}
	\item \textbf{レコメンデーション(推奨販売)}:
	      \begin{itemize}
		      \item 「季節・時間・気温」 $\times$ 「性別・年代(推定)」の組み合わせロジックにより、その瞬間にその顧客が欲しそうな商品を「おすすめ」マーク付きで強調表示する。
		      \item これにより、受動的な販売から能動的な提案型販売へと進化している。
	      \end{itemize}
\end{itemize}

\subsubsection{2. バックエンドの革新(オペレーション効率化)}
この事例の真の凄みは、集めたデータを即座に物流・在庫管理の最適化に繋げている点にある。

\begin{itemize}
	\item \textbf{通信モジュールによるリアルタイム管理}:
	      \begin{itemize}
		      \item 全端末がネットワーク化されており、売上データと在庫状況が逐次サーバーへ送信される。
		      \item 大容量データの送信に適した仕様(WiMAX等)を採用し、リッチコンテンツの配信とログデータの収集を両立。
	      \end{itemize}
	\item \textbf{属性付きPOSデータの活用}:
	      \begin{itemize}
		      \item 従来の自販機POS: 「何が売れたか」のみ。
		      \item 次世代自販機POS: 「\textbf{誰が(30代男性が)}、いつ、何を買ったか」を把握可能。これにより、商品開発や棚割りの精度が飛躍的に向上する。
	      \end{itemize}
	\item \textbf{補充業務(オペレーション)の劇的改善}:
	      \begin{itemize}
		      \item \textbf{Before}: ルート配送員が自販機の前に行って初めて「売り切れ」に気づく。あるいは、売れていないのに補充に行ってしまう無駄が発生。
		      \item \textbf{After}: メインシステム側で在庫を完全把握。必要な商品を、必要な自販機にだけ持って行くことが可能になり、物流コストと作業時間を大幅に削減。
	      \end{itemize}
	\item \textbf{機会損失の最小化と顧客満足}:
	      \begin{itemize}
		      \item 在庫がある商品のみをディスプレイに表示するため、顧客は「ボタンを押したら売り切れだった」という不快な体験(ガッカリ感)から解放される。
	      \end{itemize}
\end{itemize}

\subsubsection{3. 社会的価値(CSV)}
\begin{itemize}
	\item 災害時にはネットワーク経由で指令を出し、即座に飲料を無料で提供するインフラとしての機能を備える。これは企業の社会的責任(CSR)を果たすと同時に、ブランドへの信頼を高める。
\end{itemize}

\subsection{深層背景と教訓}

\paragraph{\textbf{寄り道トピック:プロシューマーとコミュニティの力}}
講義内で触れられた「顧客インターフェースとしてのコミュニティサイト」は、トフラーが提唱した「プロシューマー(生産する消費者)」の概念とリンクする。現代のCRMでは、企業が一方的にサービスを提供するだけでなく、顧客がコミュニティ内で相互に問題を解決したり(サポートコストの削減)、製品開発に参加したりする(R\&Dの効率化)動きが加速している。企業は「管理」するのではなく、この顧客の主体的なエネルギーを「支援(ファシリテート)」する立ち位置が求められる。

\paragraph{\textbf{寄り道トピック:精度75\%のビジネス的妥当性}}
次世代自販機の属性判定精度は「75\%」とされた。技術者視点では「100\%でないと不完全」と考えがちだが、ビジネス(マーケティング)視点では、数万・数十万のサンプルが集まるビッグデータにおいて75\%の精度があれば、傾向分析には十分すぎるほど有用である。完璧な技術を待つのではなく、実用レベルで投入しデータを稼ぐという判断スピードも重要な教訓である。

\subsubsection{\textbf{AIによる補足:重要論点の拡張(CRMの収益構造)}}
本講義では触れられなかったが、CRMを推進する経済的合理性として、以下のマーケティング法則を理解しておくことが重要である。

\begin{enumerate}
	\item \textbf{1:5の法則}:
	      \begin{itemize}
		      \item 新規顧客を獲得するコストは、既存顧客を維持するコストの\textbf{5倍}かかる。
		      \item したがって、リレーションシップ・マーケティングにより既存顧客の離反を防ぐことは、利益率改善に直結する。
	      \end{itemize}
	\item \textbf{5:25の法則}:
	      \begin{itemize}
		      \item 顧客離れを5\%改善すれば、利益は最低でも25\%改善される。
		      \item LTV(生涯価値)の高い優良顧客を繋ぎ止めることのレバレッジ効果は極めて大きい。
	      \end{itemize}
	\item \textbf{デマンドチェーン・マネジメント}:
	      \begin{itemize}
		      \item 次世代自販機の事例は、供給側(サプライ)の都合ではなく、需要側(デマンド)の情報を起点に生産・物流をコントロールしている点で、SCMから「デマンドチェーン」への進化形と言える。
	      \end{itemize}
\end{enumerate}

\subsection{結論}
本日の講義の要諦は以下の3点に集約される。

\begin{enumerate}
	\item \textbf{真のニーズへの到達}:
	      顧客の言葉(Wants)の背後にある「なぜ(Why)」を突き詰め、本質的な課題(Needs)を解決することが商品開発の核心である。水着の事例が示すように、顧客自身も正解を知らないことが多い。

	\item \textbf{関係性資産への投資}:
	      マーケティングは「狩猟型(取引重視)」から「農耕型(関係性重視)」へ移行した。POSデータという「結果」だけでなく、顧客属性や文脈という「プロセス」データを資産化することが、CRMの本質である。

	\item \textbf{デジタルとリアルの融合によるオペレーション革新}:
	      次世代自販機の事例は、フロントエンド(UX向上・レコメンド)とバックエンド(在庫適正化・物流効率化)がデジタル技術によって完全に同期された理想的なモデルである。ここでは、マーケティング戦略とオペレーション戦略が不可分なものとして統合されている。
\end{enumerate}

\subsection{重要キーワード一覧}
\textbf{人名(関連学者・実務家)}:
フィリップ・コトラー、ピーター・ドラッカー、クリステンセン、アルビン・トフラー、ドン・ペパーズ、マーサ・ロジャース

\vspace{\baselineskip}

\textbf{理論・コンセプト}:
CRM(Customer Relationship Management)、リレーションシップ・マーケティング、LTV(顧客生涯価値)、POSデータ、FSP(フリークエント・ショッパーズ・プログラム)、デジタルサイネージ、IoT、ビッグデータ、レコメンデーション、サプライチェーン・マネジメント(SCM)、デマンドチェーン、プロシューマー、CSV(共通価値の創造)、インサイト、潜在ニーズ

\subsection{理解度確認クイズ}
\begin{enumerate}
	\item 顧客が自覚しており、言葉として表現できる表面的な欲求を何と呼ぶか。
	\item 顧客自身も自覚していない、あるいは言葉にできていない本質的な課題や欲求を何と呼ぶか。
	\item 従来型のマーケティングにおいて重視された、販売時点での売上記録データを指す用語は何か。
	\item 企業と顧客が長期的・継続的な関係を築くことで利益を最大化しようとするマーケティング概念は何か。
	\item CRMにおいて、WebやIoTなどを通じて顧客情報を収集・統合する戦略的側面を何と呼ぶか。
	\item 収集された顧客データを分析し、経営判断や個別提案に活かす戦略的側面を何と呼ぶか。
	\item JR東日本の次世代自販機に搭載された、電子的な表示機器を用いた看板システムを何と呼ぶか。
	\item 次世代自販機が取得可能にした、POSデータに紐づく「誰が買ったか」という情報を指す用語は何か。
	\item 顧客の過去の購買履歴や属性に基づいて、その顧客に最適な商品を提案する機能を何と呼ぶか。
	\item 次世代自販機によって解消された、販売員が現地に行くまで在庫切れが分からないという非効率性を解決する管理手法は何か(広義の用語で可)。
	\item 新規顧客獲得コストは既存顧客維持コストの5倍かかるという経験則を何と呼ぶか。
	\item 生産活動に参加し、企業と価値を共創する消費者を指す、トフラーが提唱した造語は何か。
	\item マーケティング活動において、企業内部の部門間の壁(情報の断絶)を指す比喩表現は何か。
	\item 災害時などに自販機の商品を無償提供することは、企業の何(アルファベット3文字)としての活動に分類されるか。
	\item 一人の顧客と企業との取引期間全体からもたらされる利益の総額を指す指標は何か。
\end{enumerate}

\subsubsection*{解答一覧}
1. Wants(ウォンツ)、2. Needs(ニーズ)/インサイト、3. POSデータ、4. リレーションシップ・マーケティング、5. 顧客インターフェース戦略、6. 顧客ナレッジ分析戦略、7. デジタルサイネージ、8. 属性データ(または顧客属性)、9. レコメンデーション、10. 在庫管理(またはSCM)、11. 1:5の法則、12. プロシューマー、13. サイロ(サイロ化)、14. CSR(またはCSV)、15. LTV(顧客生涯価値)

\section{マーケティング情報システム}

\subsection{はじめに}

本講義では、現代の小売流通業において「神経系」とも呼ぶべき中核的な役割を担う\textbf{POS(Point of Sales:販売時点情報管理)システム}について詳解する。
かつて、小売業における顧客との取引は「現金と商品の交換」という物理的な現象のみで完結していた。しかし、IT技術の進展により、この取引の瞬間(Moment of Truth)がデジタルデータとして捕捉可能となった。
本講義の核心は、単なるハードウェアとしてのレジスターの仕組みを学ぶことではなく、収集された膨大なトランザクションデータがいかにして「情報」「知識」「知恵」へと昇華され、企業の意思決定(4P戦略)を劇的に変革しているか、そのロジックを理解することにある。

\subsection{POSシステムの技術的構造とメカニズム}

POSシステムは、店舗オペレーションの効率化とマーケティングデータの収集という二つの側面を持つ。その技術的基盤は、以下の要素によって構成されている。

\subsubsection{ハードウェアとネットワーク構成}

\begin{enumerate}
	\item \textbf{POSレジスター(ターミナル)}:
	      \begin{itemize}
		      \item 消費者との接点となる端末。商品のバーコードをスキャンし、金銭授受を行う。
		      \item 単なる計算機ではなく、データの「入力端末(インプットデバイス)」としての機能を果たす。
	      \end{itemize}
	\item \textbf{ストアコントローラー(Store Controller)}:
	      \begin{itemize}
		      \item 店舗ごとのバックヤードに設置されるサーバーコンピュータ。
		      \item \textbf{機能}: 複数のPOSレジや、検品・発注用のハンディターミナル等の周辺機器を統括制御する。
		      \item \textbf{役割}: 本部(セントラル)から配信される商品マスタ(価格、名称)を受け取り、レジへ配信すると同時に、レジで発生した売上データを集約して本部へ送信するゲートウェイの役割を担う。
	      \end{itemize}
	\item \textbf{セントラルホスト(本部サーバー)}:
	      \begin{itemize}
		      \item 全店舗のデータを集中管理・保管するデータベース。
	      \end{itemize}
\end{enumerate}

\subsubsection{商品識別と価格決定のロジック:PLUとNon-PLU}

POSシステムにおける最も基本的な処理は「価格の特定」である。これには商品の特性に合わせて二つの方式が使い分けられている。

\paragraph{1. PLU(Price Look Up)方式}
\begin{itemize}
	\item \textbf{定義}: 「価格(Price)を参照(Look Up)する」方式。
	\item \textbf{仕組み}: 商品に印刷されたバーコード(JANコード等)自体には「国コード+メーカーコード+商品アイテムコード」のみが含まれており、\textbf{価格情報は含まれていない}。スキャナで読み取ったコードをキーとして、ストアコントローラー内の価格マスター・ファイルを検索し、該当する価格と商品名を呼び出す。
	\item \textbf{採用理由}: 加工食品や雑貨などは、同一商品であれば全国どこでも同じ規格であるため。また、価格変更(特売など)の際、商品一つ一つの値札を張り替える必要がなく、マスターデータの数値を変更するだけで瞬時に全レジに反映できるため、オペレーションコスト削減効果が極めて高い。
\end{itemize}

\paragraph{2. Non-PLU(Non-Price Look Up)方式}
\begin{itemize}
	\item \textbf{定義}: 「価格を参照しない」方式。
	\item \textbf{仕組み}: バーコード自体の中に価格情報が埋め込まれている。スキャナはその数値を直接読み取って売上登録を行う。
	\item \textbf{適用対象}: 肉、魚、野菜、惣菜などの\textbf{生鮮食品(計量商品)}。
	\item \textbf{採用理由}: これらは個体ごとに「重量」が異なり、それに応じて価格も変動する(例:100gあたり200円の肉で、パックごとに200g、215gと異なる場合)。マスタで一律の価格を定義できないため、計量時に発行するインストアラベル(店独自の値札)に価格情報をエンコードする必要がある。
\end{itemize}

\subsection{主要な概念と論点:データから知恵への昇華}

収集されたPOSデータは、そのままでは単なる数値の羅列に過ぎない。これを意思決定に使える形にするための概念フレームワークとして、システム理論家ラッセル・アコフ(Russell L. Ackoff)の「DIKWモデル」が紹介された。

\subsubsection{DIKW階層モデルの定義と適用}

講義では、アコフの定義をベースに、POSデータの文脈で以下のように再解釈されている。

\begin{table}[hbtp]
	\centering
	\caption{DIKWモデルの階層構造とPOSデータ}
	\begin{tabular}{p{2.5cm} p{4.5cm} p{7.5cm}}
		\toprule
		\textbf{階層}                      & \textbf{定義・概念}                             & \textbf{POSデータにおける具体例}                                                        \\
		\midrule
		\textbf{Data\newline(データ)}       & 事実を示す記号の羅列。未処理の状態。                         & レシート明細(トランザクションログ)。「202X年8月1日 14:00 商品コード9999 1個」という記録そのもの。                   \\
		\midrule
		\textbf{Information\newline(情報)} & 分析により意味付けされたデータ。「誰が、いつ、何を」という記述的な問いへの答え。   & 集計レポート。「8月1日は冷やし中華が全店で1,000個売れた」「A店よりB店の方が売上が高い」といった、整理された事実。                 \\
		\midrule
		\textbf{Knowledge\newline(知識)}   & 情報を体系化し、法則性や傾向を見出したもの。「どのように(How)」への答えを含む。 & 傾向分析。「気温が28度を超え、かつ前日差+3度以上の日に冷やし中華の売上が急増する」という経験則やパターンの発見。                    \\
		\midrule
		\textbf{Wisdom\newline(知恵)}      & 知識に基づき、正しい判断・意思決定を行う能力。未来予測とアクションの結果。      & 意思決定と検証。「明日は気温上昇の予報なので、発注を通常の3倍にする」という判断を行い、実際に廃棄ロスを出さずに売上を最大化したという成功体験・ノウハウ。 \\
		\bottomrule
	\end{tabular}
\end{table}

講師は、「知恵」について\textbf{「意思決定や判断により起こした行動の結果の正しさと、それを証明する情報」}と独自に定義し、プロセス(やってみる)と結果(うまくいった)の統合が重要であると説いている。

\subsection{応用と事例分析}

本講義では、POSデータがいかに具体的なマーケティング・アクションに変換されているか、3つの視点から事例分析が行われた。

\subsubsection{1. クラスター分析と商品開発(JR東日本ウォータービジネス)}
\begin{itemize}
	\item \textbf{事例}: JR東日本の駅構内にある次世代自販機(acure)。
	\item \textbf{手法}: 自販機に搭載されたセンサー等から得られる購入データに基づき、顧客を属性や購買傾向でグルーピング(クラスター化)する。
	\item \textbf{発見}:
	      \begin{itemize}
		      \item 「ナツミカンゼリー」や「梨ソーダ」といった特定の商品は、特定属性の顧客層に支持されている。
		      \item 「水(AQUA)」のようなベーシックな商品も、時間帯や場所によって求められ方が異なる。
	      \end{itemize}
	\item \textbf{アクション}: 従来のような「全台一律の商品ラインナップ」ではなく、設置場所の顧客属性(サラリーマンが多い、学生が多い等)に合わせた最適な品揃え(プラノグラム)の構築や、POSデータ起点の新商品開発(Product)へと繋げている。
\end{itemize}

\subsubsection{2. 併買分析(バスケット分析)による売場革新}
POSデータの強力な分析手法として、「何と何が一緒に買われているか」を明らかにする併買分析が挙げられる。これにより、直感に反するような消費者の行動パターンが明らかになる。

\paragraph{具体的な併買事例と解釈}
\begin{itemize}
	\item \textbf{魚介類 $\times$ 鍋の素(キムチ鍋・寄せ鍋)}:
	      \begin{itemize}
		      \item \textbf{現象}: 鮮魚コーナーの魚と、調味料コーナーの鍋つゆが同時に買われる頻度が高い。
		      \item \textbf{アクション}: 鮮魚売場のすぐ横に「鍋の素」を陳列する(クロスマーチャンダイジング)。顧客の買い忘れを防ぎ、動線を短縮させることで客単価を向上させる。
	      \end{itemize}
	\item \textbf{お酒 $\times$ ウコンドリンク}:
	      \begin{itemize}
		      \item \textbf{現象}: アルコール飲料と、酔い止め・肝機能強化ドリンクが同時に買われる。
		      \item \textbf{解釈}: 「酔っ払いたくない」「明日に残したくない」というニーズが、飲酒(購入)の瞬間に顕在化している。
	      \end{itemize}
	\item \textbf{女性用ストッキング $\times$ 水虫薬}:
	      \begin{itemize}
		      \item \textbf{現象}: 一見無関係に見えるが、併買率が高い。
		      \item \textbf{背景(深層心理)}: 仕事でストッキングやブーツを長時間着用する女性は、足の蒸れや皮膚トラブル(水虫)に悩んでいるが、薬局で堂々と買いにくい。
		      \item \textbf{アクション}: コンビニという匿名性が高く、かつ日常品のついでに買える環境が、潜在的なニーズを掘り起こしている。
	      \end{itemize}
	\item \textbf{惣菜 $\times$ お茶}:
	      \begin{itemize}
		      \item \textbf{現象}: 昼食需要として、弁当・惣菜とペットボトル茶がセットで買われる。
		      \item \textbf{アクション}: 惣菜コーナーに常温のお茶を配置するなどのレイアウト変更。
	      \end{itemize}
\end{itemize}

\subsubsection{3. 仮説検証サイクルの実践(冷やし中華の事例)}
コンビニエンスストアにおける発注業務は、まさに「データから知恵へ」のプロセスを体現している。

\begin{enumerate}
	\item \textbf{データ収集}: 過去の販売データと気象データを蓄積。
	\item \textbf{分析(情報化)}: 「冷やし中華」は夏に売れるが、春や秋でも売れる日と売れない日の差が激しいことが判明。
	\item \textbf{法則化(知識化)}: 単に気温が高い日ではなく、\textbf{「前日よりも急激に暖かくなった日(気温差)」}に、消費者は「暑い」と感じ、冷たい麺を欲するという法則を見出す。
	\item \textbf{仮説立案}: 「明日の天気予報は、今日より5度高い。ならば、まだ春だが冷やし中華への需要が急増するはずだ」。
	\item \textbf{実行(アクション)}: 発注端末で、通常より多めの発注数を入力し、目立つ位置に陳列する。
	\item \textbf{検証}: 翌日、実際に完売したか、あるいは売れ残って廃棄ロス(損失)が出たかをPOSデータで確認する。
\end{enumerate}

このサイクルを回すことで、「品切れによる機会損失(チャンスロス)」と「売れ残りによる廃棄ロス」の双方を最小化することが、流通業の利益最大化の鍵となる。

\subsection{深層背景と教訓}

\paragraph{消費税導入(1989年)という転換点}
講義内で触れられた「1989年の変化」は極めて重要な歴史的背景である。日本ではこの年に消費税(3\%)が初めて導入された。
\begin{itemize}
	\item \textbf{導入前の課題}: それまで商品は「内税」や「正札」で管理されており、レジは単純な加算器でよかった。しかし、外税方式の導入により、1円単位の複雑な計算が必須となり、手打ちレジでは対応しきれなくなった。
	\item \textbf{イノベーションの偶発性}: 小売店は当初、「レジ業務の効率化・計算ミス防止」という\textbf{業務改善(省力化)の目的}でPOSシステムを導入した。しかし、結果としてバーコードスキャンによって詳細な販売データが蓄積されるようになり、それが後の「マーケティング革命」の土台となった。
\end{itemize}

\paragraph{POSデータ分析からマーケティングの4Pへ}
POSデータは、マーケティング・ミックス(4P)の全領域における意思決定の精度(確度)を高める根拠となっている。講義では以下の要素が列挙された。

\begin{itemize}
	\item \textbf{Product(製品)}:
	      \begin{itemize}
		      \item 多様性、品質、デザイン、機能、ブランド名、パッケージ、サイズ、サービス、保証、返品対応。
		      \item \textit{POS活用}: 売れ筋・死に筋の判定による新陳代謝、顧客属性に合わせた新商品開発。
	      \end{itemize}
	\item \textbf{Price(価格)}:
	      \begin{itemize}
		      \item リスト価格(定価)、ディスカウント(値引き)、支払期間、支払い条件(前払い・後払い)、固定・変動料金。
		      \item \textit{POS活用}: 値引きした時の売上弾力性の分析、最適な売価設定。
	      \end{itemize}
	\item \textbf{Place(流通・チャネル)}:
	      \begin{itemize}
		      \item チャネル選定、カバレッジ(網羅率)、品揃え(アソートメント)、立地、在庫管理、輸送。
		      \item \textit{POS活用}: 店舗在庫の適正化、物流頻度の調整。
	      \end{itemize}
	\item \textbf{Promotion(プロモーション)}:
	      \begin{itemize}
		      \item 販売促進(SP)、広告、セールスフォース(人的販売)、PR(広報)、ダイレクトマーケティング。
		      \item \textit{POS活用}: \textbf{チラシ配布の最適化}。バイヤーは過去のデータに基づき、「いつ(週末か平日か)」「どの商品を(目玉商品)」「いくらで」掲載すれば集客が最大化するかを予測して紙面を構成する。
	      \end{itemize}
\end{itemize}

\subsubsection{AIによる補足:重要論点の拡張}

\paragraph{ID-POSデータの台頭}
従来のPOSデータは「何が売れたか」はわかるが、「誰が買ったか」までは特定できなかった。近年ではポイントカードやアプリと紐付いた\textbf{ID-POS}が普及しており、講義で触れられた「顧客のライフスタイル」の分析は、より個人の購買履歴(縦断的データ)に基づいたLTV(顧客生涯価値)分析へと進化している。

\paragraph{O2Oとオムニチャネル}
POSデータはリアル店舗のデータであるが、現在はWeb上の行動履歴とPOSデータを統合するO2O(Online to Offline)やオムニチャネル戦略が主流である。店舗で見てECで買う、あるいはその逆といった行動を統合的に分析する必要性が高まっている。

\subsection{結論}

本講義を通じて、以下の点が明らかになった。
\begin{enumerate}
	\item POSシステムは、消費税導入という環境変化を契機に普及したが、その本質的価値は「業務効率化」から「戦略的情報活用」へとシフトした。
	\item データの価値は、単に集めること(Data)ではなく、文脈を与え(Information)、法則を見出し(Knowledge)、意思決定につなげる(Wisdom)というプロセスを経て初めて発揮される。
	\item コンビニエンスストアの事例に見られるように、気象条件などの外部変数とPOSデータを組み合わせた「仮説検証サイクル」を現場レベルで回せるかどうかが、小売業の競争優位を左右する。
\end{enumerate}

\vspace{1\baselineskip}

\subsection{重要キーワード一覧}

ラッセル・アコフ, ジェローム・マッカーシー(4P提唱者)

POSシステム, ストアコントローラー, PLU / Non-PLU, JANコード, DIKWモデル(データ・情報・知識・知恵), クラスター分析, 併買分析(バスケット分析), クロスマーチャンダイジング, 仮説検証サイクル, 機会損失(チャンスロス), 廃棄ロス, 単品管理, マーケティング・ミックス(4P), 消費税導入(1989年), トランザクションデータ

\vspace{2\baselineskip}

\subsection{理解度確認クイズ}

以下の問題は、本講義で扱われた内容に基づき、MBAレベルの理解度を確認するために設計されています。

\subsubsection*{問題}
\begin{enumerate}
	\item \textbf{【技術】} JANコード等のバーコード自体には価格情報を持たせず、スキャン時にサーバー上のマスタを参照して価格を特定する方式を、アルファベット3文字で何と呼ぶか。
	\item \textbf{【技術】} 生鮮食品などの計量商品において用いられる、バーコード自体の中に価格情報がエンコードされている方式を何と呼ぶか。
	\item \textbf{【理論】} ラッセル・アコフのDIKWモデルにおいて、意味付けや文脈がなく、単なる事実や記号の羅列である状態(例:レシートのログ)を何と呼ぶか。
	\item \textbf{【理論】} DIKWモデルにおいて、データを分析・整理し、「誰が」「何を」といった記述的な問いに答えられる状態になったものを何と呼ぶか。
	\item \textbf{【理論】} DIKWモデルにおいて、情報から法則性や傾向を抽出し、「どのように(How)」行動すべきかの指針となる体系化された状態を何と呼ぶか。
	\item \textbf{【理論】} DIKWモデルの最上位に位置し、知識に基づいて正しい判断・意思決定を行い、結果として価値を生み出した状態を何と呼ぶか。
	\item \textbf{【分析】} 顧客の購買データや属性データに基づき、似たような特性を持つ集団(例:健康志向層、価格重視層)に分類する統計的手法は何か。
	\item \textbf{【分析】} 「紙おむつとビール」や「鮮魚と鍋の素」のように、同時に購入される確率が高い商品の組み合わせを発見する分析手法は何か。
	\item \textbf{【戦略】} 併買分析の結果に基づき、関連性の高いカテゴリー(例:お酒コーナーにウコンドリンク)を隣接させて陳列し、ついで買いを誘発する販売手法を何と呼ぶか。
	\item \textbf{【歴史】} 日本の小売業において、レジ業務の複雑化を招き、結果としてPOSシステムの導入を爆発的に加速させる契機となった1989年の出来事は何か。
	\item \textbf{【管理】} 需要予測を誤り、商品を品切れさせてしまったために、本来得られたはずの売上を逃すことを専門用語で何と呼ぶか。
	\item \textbf{【管理】} 逆に、需要予測を超えて発注しすぎた結果、消費期限切れなどで商品を捨てなければならなくなる損失を何と呼ぶか。
	\item \textbf{【応用】} コンビニの事例において、冷やし中華の売上が伸びるのは「単に気温が高い日」ではなく、どのような気象条件の日であると分析されたか。
	\item \textbf{【4P】} POSデータを分析し、特売チラシに掲載する「目玉商品」やその価格を決定する活動は、マーケティング・ミックスの4Pのうち、主にどれに該当するか。
	\item \textbf{【4P】} POSデータに基づき、売れない商品をカットし、新商品を導入して棚割を最適化する活動は、4Pのうち主にどれに該当するか。
\end{enumerate}

\subsubsection*{解答一覧}
1. PLU(Price Look Up), 2. Non-PLU(Non-Price Look Up), 3. データ(Data), 4. 情報(Information), 5. 知識(Knowledge), 6. 知恵(Wisdom), 7. クラスター分析, 8. 併買分析(バスケット分析), 9. クロスマーチャンダイジング, 10. 消費税導入, 11. 機会損失(チャンスロス), 12. 廃棄ロス, 13. 前日より気温が急上昇した日(気温差が大きい日), 14. Promotion(プロモーション), 15. Product(製品)

\section{リレーションシップマーケティング}

\subsection{はじめに:データ活用の背後にある構造変化}

\subsubsection{講義の背景と目的}
本講義の主題は、現代マーケティングの核心概念である\textbf{「リレーションシップ・マーケティング(Relationship Marketing)」}の体系的理解である。

今日、小売業をはじめとする多くの企業が、POS(Point of Sales:販売時点情報管理)データや顧客IDに紐づいた購買履歴を日常的に収集・分析している。表層的に見れば、これは「IT技術の進展によって、電子化されたデータの種類と量が増大したから、それを利用している」という技術主導の現象として片付けられがちである。

しかし、本講義ではそのような技術的決定論を否定する。データ活用の本質的理由は、技術があるからではなく、マーケティングの「パラダイム(支配的な規範・枠組み)」そのものが劇的に転換したことにある。従来のマス・マーケティングが通用しなくなった市場環境において、企業が生存するために必然的に導き出された解こそが、データに基づく個客関係の強化である。本講義では、このパラダイムシフトのメカニズムと、具体的な実践論を詳説する。

\subsection{主要な概念と論点:二つのパラダイム}

マーケティングのアプローチは、歴史的変遷を経て「刺激−反応パラダイム」から「関係性パラダイム」へと移行した。この二項対立の理解は、現代企業の戦略を評価する上で不可欠なフレームワークである。

\subsubsection{1. 刺激−反応パラダイム(Stimulus-Response Paradigm)}

\paragraph{定義とメカニズム}
これは伝統的なマーケティング手法を指す。行動心理学的なモデルに基づき、企業と顧客の関係を以下のように捉える。
\begin{itemize}
	\item \textbf{構造:} 企業(主体) $\xrightarrow{\text{刺激}}$ 顧客(客体) $\xrightarrow{\text{反応}}$ 購買行動
	\item \textbf{プロセス:} 企業が「商品」「広告」「販促」といった情報を\textbf{刺激(Stimulus)}として市場に投下する。顧客はその刺激を受け取り、「買うか・買わないか」という\textbf{反応(Response)}を選択的行動として返すのみである。
\end{itemize}

\paragraph{前提条件と限界}
このパラダイムは、以下の前提に基づいているが、1990年代以降の環境変化により機能不全に陥った。
\begin{enumerate}
	\item \textbf{顧客の受動性(Passive):} 顧客を「企業からの働きかけを待つ受け身な存在」と定義する。
	\item \textbf{均質性の仮定:} 多くの顧客は共通の欲望(Needs)を持っていると仮定し、市場全体を「1対N(マス)」の関係で捉える。
	\item \textbf{プロダクト・アウト(Push型):} 需要は潜在的に市場に存在しており、企業主導で製品を開発・供給すれば売れるという「作れば売れる」時代の発想である。
	\item \textbf{短期的取引志向:} 関心の対象は「その時々の商品と現金の交換(Transaction)」に限定される。
\end{enumerate}

\textbf{没落の背景:}
大量生産・大量消費時代の終焉、多品種少量生産への移行、顧客ニーズの多様化と変化スピードの加速により、企業が一方的に送る「刺激」の効果は著しく低下した。単発の競争優位性を維持することが困難な時代となったのである。

\subsubsection{2. 関係性パラダイム(Relationship Paradigm)}

\paragraph{定義とメカニズム}
リレーションシップ・マーケティングの基盤となる概念である。ここでは顧客と企業の境界線が曖昧になり、相互依存的な関係が構築される。
\begin{itemize}
	\item \textbf{構造:} 企業 $\longleftrightarrow$ 顧客 (相互作用・双方向ループ)
	\item \textbf{プロセス:} 企業は情報を一方的に与えるのではなく、顧客からのフィードバック(情報)を受け取る。顧客は「もっとこういう物が欲しい」「自分はこうしたい」という意思表示を行う。
\end{itemize}

\paragraph{戦略的特質}
\begin{table}[H]
	\centering
	\begin{tabular}{|p{4cm}|p{11cm}|}
		\hline
		\textbf{構成要素}            & \textbf{詳細解説}                                                                   \\
		\hline
		\textbf{顧客の能動性(Active)}  & 顧客を単なる消費者ではなく、企業の製品開発や業務プロセスに積極的に関与する\textbf{「共創パートナー」}として位置づける。                \\
		\hline
		\textbf{個体識別(Identity)}  & 顧客を「マス」としてではなく、属性や過去の取引実績を持つ固有の「個人」として識別し、\textbf{1対1(One-to-One)}の関係を構築する。     \\
		\hline
		\textbf{需要の創発性}          & 需要は最初からそこに「ある」ものではなく、企業と顧客の情報のやり取り(関係性)の中から\textbf{「生み出される(Created)」}ものであると考える。 \\
		\hline
		\textbf{マーケット・イン(Pull型)} & 顧客の真のニーズ(悩みや課題)を出発点とし、そこから製品やサービスを設計する問題解決型のアプローチを取る。                           \\
		\hline
		\textbf{長期的視点}           & 1回の取引での利益最大化を放棄し、顧客満足(CS)を通じて関係を継続させ、\textbf{長期間にわたる総利益(LTV)}を最大化することを目的とする。    \\
		\hline
	\end{tabular}
\end{table}

\subsection{応用と事例分析:アナログとデジタルの融合}

本講義では、リレーションシップ・マーケティングの本質を理解するために、極めて対照的ながら本質を共有する2つの事例(「街の電器屋さん」と「最新のPOSシステム」)を分析する。

\subsubsection{事例研究1:地域家電店における「人間的リレーションシップ」}
\textbf{ケース概要:}
価格競争力で量販店に劣る地域の小規模家電店(いわゆる「街の電器屋さん」)が、なぜ特定の顧客層(特に高齢者)から絶大な支持を得て存続できるのか。

\textbf{分析:機能的価値から情緒的・解決的価値へ}
\begin{enumerate}
	\item \textbf{「代理代行」による障壁除去:}
	      高齢者にとって、最新家電の操作や設置は大きなストレス(苦痛)である。
	      \begin{itemize}
		      \item テレビやビデオデッキの配線・設置を行う。
		      \item 難解な取扱説明書を代わりに読み、必要な操作だけを教える。
		      \item メーカーへの修理依頼や交渉を代行する。
	      \end{itemize}
	      これらの行為は、単なるサービスではなく\textbf{「顧客の無能感や不安の解消」}という価値提供である。

	\item \textbf{信頼関係(Trust)の経済的価値:}
	      「調子が悪くなったらすぐに駆けつけてくれる」という安心感は、強力な信頼関係を構築する。この信頼は、顧客にとって他店へ乗り換える際のリスク(スイッチング・コスト)として機能し、価格差を埋める正当な理由となる。

	\item \textbf{需要の発見と創出(Demand Creation):}
	      頻繁な訪問と対話を通じて、店主は顧客の家庭内の状況を詳細に把握できる。
	      \begin{itemize}
		      \item 「電球が切れかかっている」「電池がない」といった消耗品ニーズをタイムリーに発見する。
		      \item 古くなった家電の買い替え時期を察知する。
		      \item 新製品のカタログを持参し、「これがあれば、今のあなたの生活のこの不便が解消されますよ」と、顧客の生活文脈に合わせた提案(コンテキスト・マーケティング)を行う。
	      \end{itemize}
	      これは、パラダイム論で述べた\textbf{「需要は関係性の中から生み出される」}という理論の実践そのものである。
\end{enumerate}

\subsubsection{事例研究2:POSデータとオペレーションの効率化}
\textbf{ケース概要:}
リレーションシップ・マーケティングにおいて、POSデータや顧客データベースはどのように機能するか。単なる「売上集計」を超えた役割について考察する。

\textbf{分析:オペレーション効率化がもたらす顧客価値}
データ活用には「マーケティング(攻め)」と「オペレーション(守り)」の両面があるが、本講義では後者の重要性が強調されている。

\begin{enumerate}
	\item \textbf{即時性と正確性の担保:}
	      店舗のホストコンピュータに商品情報を登録しておくことで、バーコードスキャンによる瞬時の価格照会が可能となる。これはレジ待ち時間を短縮し、店員の入力ミスによる不快な体験を防ぐ。
	\item \textbf{記憶の外部化と継承:}
	      正確な顧客データを蓄積することで、過去の購入履歴や提案内容を瞬時に参照できる。
	\item \textbf{アフターサービスの高度化:}
	      「いつ、どの部品を交換したか」という履歴データがあれば、次回の修理時に診断が早くなる。これは前述の「街の電器屋さん」が持つ\textbf{「個客に関する記憶」をシステムで再現・拡張したもの}と言える。
\end{enumerate}

\subsection{深層背景と教訓}

\paragraph{【歴史的文脈】1980年代以前からの源流}
リレーションシップ・マーケティングが学術的に体系化されたのは1980年代以降であるが、その実践的起源はさらに古い。日本の伝統的な「御用聞き」や地域密着型の商売は、まさにこの概念を体現していた。
現代におけるCRM(Customer Relationship Management)やデータベース・マーケティングは、かつて人間の記憶力と対話力に依存していた「関係性管理」を、IT技術を用いて大規模かつ組織的に再現しようとする試みであると解釈できる。技術は「目的」ではなく、古来からある商売の基本をスケールさせるための「手段」に過ぎない。

\paragraph{【核心概念】「便利の共有」というWin-Winモデル}
講義内で語られた最も重要なキーワードの一つが「企業と顧客による『便利』の長期的共有」である。
\begin{itemize}
	\item \textbf{顧客側のメリット:} 企業に自分の好みや状況を理解してもらうことで、説明の手間が省け、生活が豊かになる(問題解決)。
	\item \textbf{企業側のメリット:} その対価として、安定した収益と高いロイヤルティを獲得する。
\end{itemize}
伝統的マーケティングが「企業が顧客から利益を奪う(ゼロサムゲーム)」側面を持っていたのに対し、関係性パラダイムは「共に価値を享受する(プラスサムゲーム)」構造を持つ点が決定的違いである。

\paragraph{【未言及の重要視点】顧客満足と購入継続の非対称性}
講義では「商品を購入した顧客が、実は満足していないかもしれない」という点に触れている。これは極めて重要な洞察である。伝統的パラダイムでは「売れた=満足した」と短絡的に見なしがちだが、実際には「他に選択肢がないから仕方なく買った」「二度と買わないと決意して店を出た」というケースも含まれる。
リレーションシップ・マーケティングでは、「購入」はゴールの瞬間ではなく、評価のスタート地点である。使用プロセスにおける満足度をモニタリングし続けなければ、サイレント・クレーマー(物言わぬ離反客)を見逃すことになる。

\paragraph{AIによる補足:重要論点の拡張}
講義内容をMBAレベルで体系化するため、以下の概念を補足する。

\textbf{1. サービス・ドミナント・ロジック(S-Dロジック):}
本講義の「街の電器屋さん」の例は、近年のマーケティング理論であるS-Dロジックに通じる。「モノ(グッズ)」自体に価値があるのではなく、モノを使用した結果得られる「サービス(便益)」にこそ価値があるとする考え方である。電器屋はテレビというモノではなく、「快適な視聴体験」というサービスを提供していると解釈できる。

\textbf{2. スイッチング・コストの多層性:}
リレーションシップ・マーケティングが構築するのは、金銭的なスイッチング・コスト(解約金など)ではなく、\textbf{心理的・手続き的スイッチング・コスト}である。「新しい店でまた一から説明するのが面倒」「いつもの店員さんを裏切れない」という心理的障壁こそが、最強の顧客維持装置となる。

\textbf{3. シェア・オブ・ウォレット(顧客内シェア):}
市場全体のシェア(マーケットシェア)を追うのが伝統的マーケティングだとすれば、リレーションシップ・マーケティングは一人の顧客の生涯支出における自社の割合(シェア・オブ・ウォレット)の最大化を目指す。商圏人口が減少しても、このシェアを高めれば企業は成長できる。


\subsection{結論}

本講義から得られる結論は以下の通りである。

\begin{enumerate}
	\item \textbf{パラダイムシフトの不可逆性:}
	      市場の成熟とニーズの多様化により、企業は「刺激−反応型」から「関係性型」への転換を余儀なくされている。これは一時的な流行ではなく、不可逆的な構造変化である。
	\item \textbf{テクノロジーの本質的役割:}
	      POSやデータベースは、オペレーションを効率化することで、人間が「高付加価値な対話(問題解決)」に注力する時間を生み出すために存在する。デジタル化の目的は、逆説的だが「より人間的なサービス」の実現にある。
	\item \textbf{LTV最大化への道筋:}
	      目先の売上よりも、顧客の抱える問題を解決し「便利を共有」することで信頼を蓄積する戦略こそが、長期的には最も合理的な収益モデルとなる。
\end{enumerate}

\subsection{重要キーワード一覧}

\textbf{人物(概念提唱者等):}
(本講義テキスト内には特定の著名な理論家名の言及なし。※一般的文脈ではレジス・マッケンナやバーバラ・ブンダ・ジャクソン等が関連するが、講義準拠のため割愛)

\vspace{\baselineskip}

\textbf{理論・コンセプト:}
リレーションシップ・マーケティング、刺激−反応パラダイム、関係性パラダイム、マス・マーケティング、ワン・トゥ・ワン・マーケティング、プロダクト・アウト(プッシュ型)、マーケット・イン(プル型)、潜在需要、需要の創発、POSシステム、データベース・マーケティング、顧客生涯価値(LTV)、顧客満足(CS)、スイッチング・コスト、オペレーション効率化、問題解決型アプローチ。

\subsection{理解度確認クイズ}

以下の問題は、講義内容の深い理解およびMBA的視点の獲得を確認するために設計されています。

\begin{enumerate}
	\item \textbf{パラダイム比較:} 刺激−反応パラダイムにおいて、顧客は「受動的」とみなされるが、関係性パラダイムではどのような主体として再定義されるか。
	\item \textbf{需要の性質:} 伝統的マーケティングでは需要は「潜在的に存在するもの」とされるが、リレーションシップ・マーケティングでは需要はどこから生まれるとされるか。
	\item \textbf{アプローチの違い:} 企業主導で製品を提供する「プロダクト・アウト」に対し、顧客の悩みや課題を出発点とするアプローチを何と呼ぶか。
	\item \textbf{関係性の構造:} マス・マーケティングが「1対N」の関係であるのに対し、リレーションシップ・マーケティングが目指す関係構造は何か。
	\item \textbf{時間軸の視点:} リレーションシップ・マーケティングが最大化を目指すのは、単発の取引利益ではなく、どのような時間軸での利益か。
	\item \textbf{事例分析:} 地域の家電店が高齢者顧客に提供している価値において、製品機能以外の「配線」「説明」「修理代行」などを総称して、どのような価値提供と言えるか。
	\item \textbf{顧客心理:} 家電店の事例において、顧客が他店に乗り換えない理由となる、信頼関係に基づく心理的な障壁を専門用語で何と呼ぶか(AI補足より)。
	\item \textbf{データ活用:} POSシステムにおいて、商品情報を事前登録しておくことで得られる、マーケティング以外の実務上のメリットは何か。
	\item \textbf{顧客の捉え方:} リレーションシップ・マーケティングでは、顧客を一様な集団としてではなく、どのように扱うことが求められるか。
	\item \textbf{市場環境:} リレーションシップ・マーケティングが注目される背景となった、新規顧客獲得が困難になる市場の状態を何と呼ぶか。
	\item \textbf{需要創出の具体例:} 「街の電器屋さん」が顧客宅を訪問することで発見できる、消耗品や買い替え以外の需要喚起のきっかけとなるものは何か。
	\item \textbf{満足度の把握:} 顧客との関係構築において、購入時点の情報だけでは不十分であり、特にどのフェーズの情報を把握すべきか。
	\item \textbf{システムの効果:} 正確な顧客データを持つことで、過去の提案や修理履歴を参照できることは、企業の何の効率化に繋がるか。
	\item \textbf{情報の流れ:} 刺激−反応パラダイムが一方向的であるのに対し、関係性パラダイムにおける情報の流れはどのような特性を持つか。
	\item \textbf{核心概念:} 講義の結論として述べられた、企業と顧客が長期的に関係を維持するための、双方にとって利益となる状態を指すフレーズは何か。
\end{enumerate}

\subsubsection*{解答一覧}
1. 能動的(な主体/共創パートナー), 2. 企業と顧客の関係性(相互作用/情報のやり取り)の中から生み出される, 3. マーケット・イン(またはプル型/問題解決型), 4. 1対1(ワン・トゥ・ワン), 5. 長期的視点(LTV/生涯価値), 6. 問題解決(または安心・信頼/サービス価値), 7. スイッチング・コスト, 8. オペレーションの効率化(レジ通過時間の短縮/入力ミスの防止), 9. 個別の属性を持つ個人(個客)として識別する, 10. 市場の成熟化, 11. 顧客の生活上の悩みや不便(生活文脈), 12. 商品の使用状況や満足度(購入後のプロセス), 13. オペレーション(またはアフターサービス), 14. 双方向的(インタラクティブ), 15. 便利の共有(またはWin-Winの関係).

\end{document}
\documentclass[uplatex]{jsarticle}
\usepackage[english]{babel}
\usepackage[letterpaper,top=2cm,bottom=2cm,left=3cm,right=3cm,marginparwidth=1.75cm]{geometry}
\usepackage{amsmath, amssymb}
\usepackage{graphicx}
\usepackage{here}

\title{
Misc.
}

\author{
M. O.
}

\date{Oct. 2025}

\begin{document}
\maketitle

\section{2025年10月1日(水)}

妻に今月のお小遣いを渡したが、家計がぎりぎり。



\section{2025年10月2日(木)}

昼は友人とケニア料理を食べに行く予定だったが、
ケニア料理のお店が閉まっていた。

仕方なく、いつもの五反田のペルー料理を食べに行った。
いつも通り美味しかったので結果的にとても良かった。

食後にカフェでコーヒーを飲みながら友人とお話をした。

学生時代の体育会系の部活ががいかに社会適応に役に立つかという話が出た。

他にも、予備校で教えている数学の苦手な生徒は参考書に書かれている以前に、数式をきちんと写せていないだとか、という話も聞いた。

友人は無事に医学部専門予備校に受かって、受かっただけでなく早速生徒を3名指導しているらしい。

カフェの後に解散して、それから新宿の会社に出社した。

友人と話せたり、チームメンバーと話せたりして良かった。

夜に新宿の古本屋に立ち寄ってから帰宅したら20時すぎになってしまった。



\section{2025年10月3日(金)}

1日中、ひたすら副業の開発をしていた。

Yomi Sheetの分析バッチがおおよそできてきた。

CAの稼働分析についてリーダーの2人、別件でCAのパフォーマンスについての分析システムを開発している人と話した。

どうも内定承諾数が1桁合わない。

おそらく30日や90日の区切りの刈り取りを先に行うか、最も進んだプロセスのフラグを先に立てるか、その順番によるものではないかと疑っている。

夜は友人と北千住にご飯を食べに出かけた。

北千住はビリヤニを食べられる可能性のあるお店は多いものの、不定休が多く、なかなか食べられない。

今日もラーメン二郎系の豚山を食べた。

ハンナ・アーレントの本、因果推論の本などを友人に渡せた。

帰り際にいつものビールの立ち飲み屋でゆっくり話した。

博士課程に入り直したい旨、具体的な研究室まで決まってきている話をした。

友人はコーポレートファイナンス理論に興味が出ているそう。

返ってからKeldysh Green Functionのノートをアップロードした。

\section{2025年10月4日(土)}

ほとんど1日中寝ていた日。

リチャード・ローティの『偶然性・アイロニー・連帯』が届いた。読みたい。

日付が変わるタイミング、つまり年齢が変わるタイミングは、グロダンディーク構成と等しくなるコンマ圏が、要素の圏として圏の積分記号を用いて表されるところを勉強していた。
圏論にも積分が出てきて、フビニの定理まであるとは驚いた。


\section{2025年10月5日(日)}

朝は少し睡眠不足気味。

フェルミ液体のノートを少し修正した。

圏論の極限の勉強をしている。

今日は誕生日なのでお寿司を食べに行く予定だったが、娘が家の中で暴れまわっていて、妻が疲れてしまっていて、お弁当屋さんでお惣菜を買って食べた。

AmmonとErdmengerのゲージ重力対応の本がたまたま中古で出ているがあまり安くない。
以前から買おうか悩んでいたが、誕生日なので買おうかと思っていた。

ひょんなことから入手できた。

ついでにAltrandとSimonsの教科書、中原のトポロジーの教科書、Xiao Gang Wenの教科書も入手できた。
ラッキー。

古本屋ではオードリー・タンの『プルラリティ』、
ミルトン・フリードマンの『資本主義と自由』
も手に入れた。

このMisc.のノートを作ってみた。

\end{document}
\documentclass[uplatex,a4j,12pt,dvipdfmx]{jsarticle}
\usepackage{amsmath,amsthm,amssymb,bm,color,enumitem,mathrsfs,url,epic,eepic,ascmac,ulem,here,ascmac}
\usepackage[letterpaper,top=2cm,bottom=2cm,left=3cm,right=3cm,marginparwidth=1.75cm]{geometry}
\usepackage[english]{babel}
\usepackage[dvipdfm]{graphicx}
\usepackage[hypertex]{hyperref}
\usepackage{tikz-cd}
\title{
随伴
}
\author{岡田 大(Okada Masaru)}

\date{\today}

\begin{document}

\maketitle

\tableofcontents

\ \\

\section{定義}

\subsection{随伴の定義}

圏 $C,D$ に対して、
関手 $F: C \to D$ 、$G: D \to C$
がある。
「このとき $F$ と $G$ は随伴である」とは、
記号では $F \dashv G : C \to D$ と書き、

\begin{itembox}[l]{随伴}
	圏 $C$ の任意の対象 $c$ 、圏 $D$ の任意の対象 $d$ について、自然な全単射
	$$
		\phi_{cd} : \mathrm{Hom}_{D}(Fc,d) \to \mathrm{Hom}_{C}(c,Gd)
	$$
	が存在する。
\end{itembox}

\subsection{「自然な」の意味}

ここでいう「自然な」の意味は、
\begin{enumerate}
	\item $d$ を固定したときに $c$ について自然であり、
	\item $c$ を固定したときに $d$ について自然である
\end{enumerate}
ということを表す。

つまり2つの条件がある。

${}$


まず「 $c$ について自然」という意味については、

\begin{itembox}[l]{$c$ について自然}
	$$
		\theta_{c} : \mathrm{Hom}_{D}(F-,d) \to \mathrm{Hom}_{C}(-,Gd)
	$$

	という自然変換 $\theta_{c}$ が存在する。
\end{itembox}

ということを表す。

${}$

自然変換なのでドメインとコドメインのそれぞれは関手である。

Homは関手になることを思い出そう。

一般に、
ある圏 $C$ の対象 $a \in C$ について、
関手 $\mathrm{Hom}_{C}(-,a)$ は $C^{op} \to \mathbf{Set}$ の関手になる。

同様に、
$\mathrm{Hom}_{C}(-,Gd)$
は
$Gd \in C$ という対象について、
関手 $C^{op} \to \mathbf{Set}$ になっている。

${}$

もう一つの関手
$\mathrm{Hom}_{D}(F-,d)$
の意味を考える。

まず、$F$ がかかっていない
$\mathrm{Hom}_{D}(-,d)$
は関手
$D^{op} \to \mathbf{Set}$
である。

さらに $C \to D$ への関手 $F$ があって、

$$
	C^{op} \xrightarrow{ \ \ F \ \ } D^{op} \xrightarrow{ \ \ \mathrm{Hom}_{D}(-,d) \ \ } \mathbf{Set}
$$

以上の2つの関手が合成されたものが
$\mathrm{Hom}_{D}(F-,d)$
である。

この関手の間の自然変換 $\theta_{c}$ が存在することが、「 $c$ について自然」という意味である。

\ \\

「 $d$ について自然」も同様で、
$$
	\theta_{d} : \mathrm{Hom}_{D}(Fc,-) \to \mathrm{Hom}_{C}(c,G-)
$$

という自然変換 $\theta_{d}$ が存在するという意味である。


\subsection{左と右}

$F \dashv G : C \to D$ すなわち、

圏 $C$ の任意の対象 $c$ 、圏 $D$ の任意の対象 $d$ について、自然な全単射
$$
	\phi_{cd} : \mathrm{Hom}_{D}(Fc,d) \to \mathrm{Hom}_{C}(c,Gd)
$$
が存在するとき、

$F$ を $G$ の左随伴関手、
$G$ を $F$ の右随伴関手という。

左の対象に関手がかかっているのが左随伴関手であり、
右の対象に関手がかかっているのが右随伴関手である。
なぜか文献によって左右が逆転していることがあるらしい。

${}$

左右盲にはつらいが、
$F \dashv G : C \to D$
と書いたときには、
「 $F$ が左随伴である」とも言うし、同じ意味で「 $F$ は右随伴を持つ」とも言う。

\section{例}

\subsection{ベクトル空間の圏と集合の圏の随伴}

ベクトル空間の圏 $\mathbf{Vect}$ から対象 $V$ を取ってくる。

$V$ にはスカラー倍や線形性を保つ足し算などの公理が入っているが、
それらの公理を条件だと思って、その条件を忘れる関手(忘却関手) $U$ を取れる。

\[
	\begin{array}{cccc}
		U : & \textbf{Vect}           & \to     & \textbf{Set}            \\
		    & \rotatebox{90}{${\in}$} &         & \rotatebox{90}{${\in}$} \\
		    & V                       & \mapsto & U(V)
	\end{array}
\]

$\textbf{Vect}$ の射は線形写像$f:V \to W$ であるが、
線形構造を保つことを忘れて、単なる写像と思ったとき、
$\textbf{Set}$ の射である写像$U(f):U(V) \to U(W)$
となる。

${}$

この $U$ の随伴を考える。
つまり、任意の集合 $X$ を取ってきたとき、$F(X)$ が線型空間になるような $F$ を考える。

これは $X$ を基底とする線型空間を取ってくることで得られる。

このとき $F$ は関手になっていて、$F : \mathbf{Set} \to \mathbf{Vect}$ である。

しかも $F$ は $U$ の随伴になっている。
これは自然な全単射
$$
	\phi_{cd} : \mathrm{Hom}_{\mathbf{Vect}}(Fc,d) \to \mathrm{Hom}_{\mathbf{Set}}(c,Ud)
$$
を具体的に構成できることから示すことができる。


\subsection{アーベル群の圏と集合の圏の随伴}

アーベル群の圏 $\mathbf{Ab}$ から構造を忘れて集合の圏 $\mathbf{Set}$ とするような忘却関手 $U$ と、
$F(X)$ として、集合 $X$ で生成される自由アーベル群を成す関手 $F$ の間には随伴がある。

\[
	% https://tikzcd.yichuanshen.de/#N4Igdg9gJgpgziAXAbVABwnAlgFyxMJZABgBpiBdUkANwEMAbAVxiRAB12BbOnACwBGAM2ABBAQF8QE0uky58hFACZyVWoxZtOPfsOABlGDiky52PASIBGNdXrNWiDuwEQc09TCgBzeEVAhACcILiRbEBwIJDIQOD4sIQ9EAGZqARgwKCQAWhTYhy1nADEQagY6DIYABXlLJRAgrB8+DzMQYNCY6ijw6njE5LSQDKzc-Oo+GDpsxDAmBgZ7TScQAFUykAqq2otFNiaWtooJIA
	\begin{tikzcd}
		\mathbf{Ab} \arrow[rr, "U"', bend right, shift right=3] & \bot & \mathbf{Set} \arrow[ll, "F"', bend right, shift right=3]
	\end{tikzcd}
\]



\subsection{位相空間の圏と集合の圏の随伴}

位相空間の圏 $\mathbf{Top}$ から構造を忘れて集合の圏 $\mathbf{Set}$ とするような忘却関手 $U$ と、
$F(X)$ として、集合 $X$ に離散位相を入れる関手 $F$ の間には随伴がある。

密着位相を入れる関手 $G$ も$U$ の随伴になっていて、$U$ の右随伴になっている。

$$
	F \dashv U \dashv G
$$

\[
	% https://tikzcd.yichuanshen.de/#N4Igdg9gJgpgziAXAbVABwnAlgFyxMJZABgBpiBdUkANwEMAbAVxiRAB12BbOnACwBGAM2AAVCGgC+ISaXSZc+QigBM5KrUYs2nHv2HAAyjBzTJGmFADm8IqCEAnCFyRkQOCEgCM1BnQEwDAAKCngEbA5YVnw4INT0zKyIIACqMnIgjs7e1B6u1HB8WEKxiADM1AFgUEgAtGVuCdrJAGJxIH4BwaFKEVEx6fZOLog+7p6IboXFpbUALJUw1UgN8VpJIADiMhSSQA
	\begin{tikzcd}
		\mathbf{Top} \arrow[rr, "U"'] &  & \mathbf{Set} \arrow[ll, "F"', bend right, shift right=3] \arrow[ll, "G", bend left, shift left=4]
	\end{tikzcd}
\]

\end{document}
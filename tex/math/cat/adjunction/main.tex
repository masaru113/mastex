\documentclass[uplatex,a4j,12pt,dvipdfmx]{jsarticle}
\usepackage{amsmath,amsthm,amssymb,bm,color,enumitem,mathrsfs,url,epic,eepic,ascmac,ulem,here,ascmac}
\usepackage[letterpaper,top=2cm,bottom=2cm,left=3cm,right=3cm,marginparwidth=1.75cm]{geometry}
\usepackage[english]{babel}
\usepackage[dvipdfm]{graphicx}
\usepackage[hypertex]{hyperref}
\usepackage{tikz-cd}
\title{
Adjunction
}
\author{Masaru Okada}

\date{\today}

\begin{document}

\maketitle

\tableofcontents

\ \\

\section{Definition}

\subsection{Definition of Adjunction}

Given categories $C$ and $D$,
we have functors $F: C \to D$ and $G: D \to C$.
We say that '$F$ and $G$ are adjoint',
denoted $F \dashv G : C \to D$,

\begin{itembox}[l]{Adjunction}
	if for every object $c$ in $C$ and every object $d$ in $D$, there exists a natural bijection
	$$
		\phi_{cd} : \mathrm{Hom}_{D}(Fc,d) \to \mathrm{Hom}_{C}(c,Gd)
	$$
\end{itembox}

\subsection{Meaning of 'Natural'}

The meaning of 'natural' here is that it is:
\begin{enumerate}
	\item natural in $c$ when $d$ is fixed, and
	\item natural in $d$ when $c$ is fixed.
\end{enumerate}
In other words, there are two conditions.

${}$


First, 'natural in $c$' means:

\begin{itembox}[l]{Natural in $c$}
	$$
		\theta_{c} : \mathrm{Hom}_{D}(F-,d) \to \mathrm{Hom}_{C}(-,Gd)
	$$

	that there exists a natural transformation $\theta_{c}$.
\end{itembox}

${}$

Since this is a natural transformation, both its domain and codomain must be functors.

Recall that Hom is a functor.

In general, for an object $a \in C$ in a category $C$,
the functor $\mathrm{Hom}_{C}(-,a)$ is a functor from $C^{op}$ to $\mathbf{Set}$.

Similarly,
$\mathrm{Hom}_{C}(-,Gd)$
, for the object $Gd \in C$,
is a functor $C^{op} \to \mathbf{Set}$.

${}$

Let's consider the meaning of the other functor,
$\mathrm{Hom}_{D}(F-,d)$.

First, without $F$,
$\mathrm{Hom}_{D}(-,d)$
is a functor
$D^{op} \to \mathbf{Set}$.

Furthermore, we have the functor $F: C \to D$,
$$
	C^{op} \xrightarrow{ \ \ F \ \ } D^{op} \xrightarrow{ \ \ \mathrm{Hom}_{D}(-,d) \ \ } \mathbf{Set}
$$

The composition of these two functors is
$\mathrm{Hom}_{D}(F-,d)$.

The existence of a natural transformation $\theta_{c}$ between these functors is the meaning of 'natural in $c$'.

\ \\

'Natural in $d$' is similar,
$$
	\theta_{d} : \mathrm{Hom}_{D}(Fc,-) \to \mathrm{Hom}_{C}(c,G-)
$$
meaning that there exists a natural transformation $\theta_{d}$.


\subsection{Left and Right}

When $F \dashv G : C \to D$, that is,

for every object $c$ in $C$ and every object $d$ in $D$, there exists a natural bijection
$$
	\phi_{cd} : \mathrm{Hom}_{D}(Fc,d) \to \mathrm{Hom}_{C}(c,Gd)
$$

$F$ is called the \textbf{left adjoint functor} of $G$,
and $G$ is called the \textbf{right adjoint functor} of $F$.

The functor applied to the object on the left (in $\mathrm{Hom}_{D}(Fc,d)$) is the left adjoint,
and the functor applied to the object on the right (in $\mathrm{Hom}_{C}(c,Gd)$) is the right adjoint.
Apparently, the left/right convention is sometimes reversed depending on the literature.

${}$

This is tough for those with left-right confusion, but
when writing
$F \dashv G : C \to D$,
we say '$F$ is left adjoint' and, meaning the same thing, '$F$ has a right adjoint'.

\section{Examples}

\subsection{Adjunction between the Category of Vector Spaces and the Category of Sets}

Let $V$ be an object from the category of vector spaces, $\mathbf{Vect}$.

$V$ is equipped with axioms, such as scalar multiplication and addition preserving linearity.
We can take a functor $U$ that 'forgets' these conditions (a \textbf{forgetful functor}).

\[
	\begin{array}{cccc}
		U : & \textbf{Vect}           & \to     & \textbf{Set}            \\
		    & \rotatebox{90}{${\in}$} &         & \rotatebox{90}{${\in}$} \\
		    & V                       & \mapsto & U(V)
	\end{array}
\]

A morphism in $\textbf{Vect}$ is a linear map $f:V \to W$.
Forgetting that it preserves the linear structure and regarding it as merely a map,
we get $U(f):U(V) \to U(W)$, which is a morphism in $\textbf{Set}$.

${}$

Consider an adjoint for this $U$.
That is, we seek a functor $F$ such that for any set $X$, $F(X)$ becomes a vector space.

This is obtained by taking the vector space with $X$ as its basis.

In this case, $F$ is a functor $F : \mathbf{Set} \to \mathbf{Vect}$.

Moreover, $F$ is adjoint to $U$.
This can be shown by explicitly constructing the natural bijection
$$
	\phi_{cd} : \mathrm{Hom}_{\mathbf{Vect}}(Fc,d) \to \mathrm{Hom}_{\mathbf{Set}}(c,Ud)
$$


\subsection{Adjunction between the Category of Abelian Groups and the Category of Sets}

There is an adjunction between the forgetful functor $U$ from the category of abelian groups $\mathbf{Ab}$ to the category of sets $\mathbf{Set}$ (forgetting the structure),
and the functor $F$, where $F(X)$ forms the \textbf{free abelian group} generated by the set $X$.

\[
	% https://tikzcd.yichuanshen.de/#N4Igdg9gJgpgziAXAbVABwnAlgFyxMJZABgBpiBdUkANwEMAbAVxiRAB12BbOnACwBGAM2ABBAQF8QE0uky58hFACZyVWoxZtOPfsOABlGDiky52PASIBGNdXrNWiDuwEQc09TCgBzeEVAhACcILiRbEBwIJDIQOD4sIQ9EAGZqARgwKCQAWhTYhy1nADEQagY6DIYABXlLJRAgrB8+DzMQYNCY6ijw6njE5LSQDKzc-Oo+GDpsxDAmBgZ7TScQAFUykAqq2otFNiaWtooJIA
	\begin{tikzcd}
		\mathbf{Ab} \arrow[rr, "U"', bend right, shift right=3] & \bot & \mathbf{Set} \arrow[ll, "F"', bend right, shift right=3]
	\end{tikzcd}
\]



\subsection{Adjunction between the Category of Topological Spaces and the Category of Sets}

Between the forgetful functor $U$ from the category of topological spaces $\mathbf{Top}$ to the category of sets $\mathbf{Set}$ (forgetting the structure),
and the functor $F$, where $F(X)$ equips the set $X$ with the \textbf{discrete topology}, there is an adjunction.

The functor $G$ that equips $X$ with the \textbf{indiscrete topology} is also an adjoint of $U$; it is the right adjoint of $U$.

$$
	F \dashv U \dashv G
$$

\[
	% https://tikzcd.yichuanshen.de/#N4Igdg9gJgpgziAXAbVABwnAlgFyxMJZABgBpiBdUkANwEMAbAVxiRAB12BbOnACwBGAM2AAVCGgC+ISaXSZc+QigBM5KrUYs2nHv2HAAyjBzTJGmFADm8IqCEAnCFyRkQOCEgCM1BnQEwDAAKCngEbA5YVnw4INT0zKyIIACqMnIgjs7e1B6u1HB8WEKxiADM1AFgUEgAtGVuCdrJAGJxIH4BwaFKEVEx6fZOLog+7p6IboXFpbUALJUw1UgN8VpJIADiMhSSQA
	\begin{tikzcd}
		\mathbf{Top} \arrow[rr, "U"'] &  & \mathbf{Set} \arrow[ll, "F"', bend right, shift right=3] \arrow[ll, "G", bend left, shift left=4]
	\end{tikzcd}
\]

\end{document}
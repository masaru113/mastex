\documentclass[uplatex,a4j,12pt,dvipdfmx]{jsarticle}
\usepackage{amsmath,amsthm,amssymb,bm,color,enumitem,mathrsfs,url,epic,eepic,ascmac,ulem,here,ascmac}
\usepackage[letterpaper,top=2cm,bottom=2cm,left=3cm,right=3cm,marginparwidth=1.75cm]{geometry}
\usepackage[english]{babel}
\usepackage[dvipdfm]{graphicx}
\usepackage[hypertex]{hyperref}
\usepackage{tikz-cd}
\title{
An Introduction to Adjunctions according to Steve Awodey \\
\normalsize Reading from \texttt{Steve Awodey}, Chapter 9.1
}
\author{Masaru Okada}

\date{\today}

\begin{document}

\maketitle

\begin{abstract}
	This note summarizes the introduction to the definition of adjunctions, following Chapter 9, Section 1 of Steve Awodey's 2nd Edition \cite{Awodey}.
\end{abstract}

\tableofcontents

\ \\

\section{Preparatory Definitions}

\subsection{Kleene Closure constructing words}

Here is an example of one method for 'constructing a free monoid from an arbitrary set'.

Consider the set of alphabetic characters $A = \{ a,b,c,...,y,z \}$.

A finite string of characters (regardless of whether the string is meaningful) is called a \textbf{word} on $A$.
For example,
$$
	word, thisword, categoriesarefun, asdfasdaf, ...
$$

The empty string will be denoted by a hyphen '-'.

The \textbf{Kleene Closure} is then the operator $( \ \cdot \ )^{\rm Kleene}$ defined by,
$$
	A^{\rm Kleene} = \{ -, word, thisword, categoriesarefun, asdfasdaf, ... \}
$$

\ \\

We introduce a string concatenation operation $++$ for the elements (words) in the set $A^{\rm Kleene}$.
This defines $++ : A^{\rm Kleene} \times A^{\rm Kleene} \to A^{\rm Kleene}$ such that:
\[
	\begin{array}{rcl}
		word \ ++ \ - \                       & =                  & \ word \\
		this \ ++ \ word \& =                 & \ thisword                  \\
		categories \ ++ \ are \ ++ \ fun \& = & \ categoriesarefun
	\end{array}
\]

The empty string $-$ serves as the identity element.

With this operation, $(A^{\rm Kleene}, ++)$ becomes a monoid.

${}$

Furthermore, $A^{\rm Kleene}$ satisfies the following conditions, making it a \textbf{free monoid}:
\begin{enumerate}
	\item \textbf{no junk} (All words can be expressed as a product of elements of $A$.)
	\item \textbf{no noise} (For every word, the way it is written as a concatenation of elements from $A$ is unique (apart from the monoid axioms). For example, if $a \neq b$, then $ab \neq ba$.)
\end{enumerate}

\subsection{Universal Property of the Free Monoid}

The two conditions (no junk, no noise) that make a monoid free can be expressed very neatly using a categorical definition.

First, any monoids $M, N$ have \textbf{underlying sets} $U(M), U(N)$.

And any homomorphism $f: N \to M$ has an \textbf{underlying map} $U(f) : U(N) \to U(M)$.

This $U$ is a functor, known as the \textbf{forgetful functor}.

\ \\

The free monoid $M(A)$ constructed from a set $A$ has the following \textbf{universal property}.

\begin{itembox}[l]{Universal Property of the Free Monoid $M(A)$}
	There is a map $i: A \to U(M(A))$, such that for any monoid $N$ and any map $f: A \to U(N)$,
	there exists a \textbf{unique} monoid homomorphism $g: M(A) \to N$ satisfying $U(g) \circ i = f$.
\end{itembox}

${}$

This can be summarized neatly in categories.

${}$

\begin{itembox}[l]{Diagram for the Universal Property of $M(A)$}

	Diagram in $\mathbf{Mon}$:
	\[
		\begin{tikzcd}
			M(A) \arrow[rr, "\exists \ \! ! \ g"] && N
		\end{tikzcd}
	\]

	Diagram in $\mathbf{Set}$:
	\[
		% https://tikzcd.yichuanshen.de/#N4Igdg9gJgpgziAXAbVABwnAlgFyxMJZABgBpiBdUkANwEMAbAVxiRAFUAKAWU4EEAlAJABfUuky58hFACZyVWoxZsuAOWFiJ2PASJlZi+s1aIQfUYphQA5vCKgAZgCcIAWyRkQOCEgCM1MYqZlw2muIgLu5I8t6+iF5BpiBYohFRHoixPv6BysmOINQMdABGMAwACpK6MiDOWDYAFjiWIkA
		\begin{tikzcd}
			U(M(A)) \arrow[rr, "U(g)"] && U(N) \\
			&&\\
			A \arrow[uu, "i"] \arrow[rruu, "f"'] &&
		\end{tikzcd}
	\]

\end{itembox}

\subsection{A Simple Example of a Free-Forgetful Adjunction}

Any monoid $M$ has an underlying set $U(M)$.

Also, as constructed in the previous section, every set $X$ has a \textbf{free monoid} $F(X)$.

Consider the map $\phi$ that sends $g$ to $U(g) \circ i = f$.

\[
	\begin{array}{cccc}
		\phi : & \mathrm{Hom}_{\mathbf{Mon}}(F(X), M) & \to     & \mathrm{Hom}_{\mathbf{Set}}(X, U(M)) \\
		       & \rotatebox{90}{${\in}$}              &         & \rotatebox{90}{${\in}$}              \\
		       & g                                    & \mapsto & U(g) \circ i
	\end{array}
\]

From the universal property of the free monoid, this map is an isomorphism.

$$
	\mathrm{Hom}_{\mathbf{Mon}}(F(X), M) \cong \mathrm{Hom}_{\mathbf{Set}}(X, U(M))
$$

${}$

A mnemonic for this is: '\textbf{Free is left adjoint to Forgetful}'.

\subsection{A Simple Definition of Adjunction}

We define an adjunction by generalizing this flow to categories $\mathbf{C}$ and $\mathbf{D}$.

\begin{itembox}[l]{Adjunction between Categories $\mathbf{C}$ and $\mathbf{D}$}

	An adjunction between categories $\mathbf{C}$ and $\mathbf{D}$ consists of functors $F, G$
	$$
		F : \mathbf{C} \rightleftharpoons \mathbf{D}: G
	$$
	and a natural transformation
	$\eta: 1_{\mathbf{C}} \to G \circ F$.

	They have the following property.

	For any
	$C \in \mathbf{C} \ , \ \ D \in \mathbf{D}$
	and
	$f: C \to G(D)$,
	there exists a \textbf{unique} $g$ such that
	$f = G(g) \circ \eta_{C}$
	holds as follows.

	\[
		% https://tikzcd.yichuanshen.de/#N4Igdg9gJgpgziAXAbVABwnAlgFyxMJZABgBpiBdUkANwEMAbAVxiRADEAKAYQEoQAvqXSZc+QigBM5KrUYs2AEUHCQGbHgJEyk2fWatEIAOKcuffkJEbxRabur6FR04sur1YrSjIAWPfKGINyCsjBQAObwRKAAZgBOEAC2SGQgOBBIAIyOgWwAhBEqcYkpiNLpmYgAzLkGbKYR7iXJSL7UGUi1cvVGsSDUDHQARjAMAAqimhIg8VgRABY4xSAJrYjtlUgVTkEAOnswOHQA+sDcAqECQA
		\begin{tikzcd}
			F(C) \arrow[rr, "!g"] && D\\
			&&\\
			G(F(C)) \arrow[rr, "G(g)"]&& G(D) \\
			&&\\
			C \arrow[rruu, "f"'] \arrow[uu, "\eta_{C}"] &&
		\end{tikzcd}
	\]

\end{itembox}

In this case, $F$ is called the \textbf{left adjoint} to $G$, and $G$ is the \textbf{right adjoint} to $F$, written $F \dashv G$.

$\eta$ is called the \textbf{unit} of the adjunction.

\section{Example: The Diagonal Functor}

\subsection{The Right Adjoint to the Diagonal Functor is the Product Functor}

As an example, consider the \textbf{diagonal functor} $\Delta : \mathbf{C} \to \mathbf{C} \times \mathbf{C}$.

Objects and morphisms are mapped respectively:
\[
	\begin{array}{rclr}
		\Delta(C)          & = & (C, C)                    & \text{for $C \in$ Obj}(\mathbf{C}) \\
		\Delta(f:C \to C') & = & (f,f) : (C,C) \to (C',C') & \text{for $f \in$ Mor}(\mathbf{C})
	\end{array}
\]

${}$

Let's consider the right adjoint $R$ to the diagonal functor.

Since it goes in the opposite direction of $\Delta : \mathbf{C} \to \mathbf{C} \times \mathbf{C}$,
it will be a functor $R : \mathbf{C} \times \mathbf{C} \to \mathbf{C}$.
Let's denote its action on objects as
$$
	R : \mathbf{C} \times \mathbf{C} \ni (X,Y) \mapsto R(X,Y) \in \mathbf{C}
$$

${}$

Recall the construction of the adjunction.

Recalling the free-forgetful adjunction
$$
	\mathrm{Hom}_{\mathbf{Mon}}(F(X), M) \cong \mathrm{Hom}_{\mathbf{Set}}(X, U(M))
$$
and substituting into this correspondence, we get:
$$
	\mathrm{Hom}_{\mathbf{C} \times \mathbf{C}}(\Delta(C), (X,Y)) \cong \mathrm{Hom}_{\mathbf{C}}(C, R(X,Y))
$$

The left-hand side (LHS) of this is:
\[
	\begin{array}{rcl}
		\mathrm{Hom}_{\mathbf{C} \times \mathbf{C}}(\Delta(C), (X,Y))
		 & \cong &
		\mathrm{Hom}_{\mathbf{C} \times \mathbf{C}}((C,C), (X,Y))
		\\ &\cong&
		\mathrm{Hom}_{\mathbf{C}}(C, X)
		\times
		\mathrm{Hom}_{\mathbf{C}}(C, Y)
		\\ &\cong&
		\mathrm{Hom}_{\mathbf{C}}(C, X \times Y)
	\end{array}
\]

The first isomorphism uses the definition of $\Delta(C)$.

The second isomorphism uses the definition of morphisms in the product category $\mathbf{C} \times \mathbf{C}$.

The third isomorphism uses the universal property of the product $X \times Y$ in $\mathbf{C}$:
$\mathrm{Hom}_{\mathbf{C}}(C, X \times Y) \cong \mathrm{Hom}_{\mathbf{C}}(C, X) \times \mathrm{Hom}_{\mathbf{C}}(C, Y)$.

${}$

Comparing the LHS and RHS when substituted into the adjunction definition:
$$
	\mathrm{Hom}_{\mathbf{C}}(C, R(X,Y))
	\cong
	\mathrm{Hom}_{\mathbf{C}}(C, X \times Y)
$$

We want to apply a corollary of the Yoneda Lemma here:
$$
	\mathrm{Hom}_{\mathbf{C}}(C, F)
	\cong
	\mathrm{Hom}_{\mathbf{C}}(C, G)
	\ \Rightarrow
	F \cong G
$$
To use this corollary of the Yoneda Lemma, the isomorphism must be natural in $C$.
In this case, by the definition of adjunction,
there is a natural isomorphism between
$$
	\mathrm{Hom}(-, R(X,Y))
	\cong
	\mathrm{Hom}(-, X \times Y)
$$

From the above, we can conclude:
$$
	R(X,Y)
	\cong
	X \times Y
$$

It has been shown that the right adjoint to the diagonal functor $\Delta$ is the product functor $\times$, i.e., $\Delta \dashv \times$.

\subsection{The Unit of the Adjunction}

Let's consider the unit of the adjunction.
By the definition of the adjunction $\Delta \dashv \times$ (i.e., $L=\Delta, R= \times$), the
\textbf{unit} $\eta$ is a natural transformation
$\eta : 1_{\mathbf{C}} \to R \circ L = \times \circ \Delta$.

Its component $\eta_{C}$, for each object $C$ in $\mathbf{C}$,
is a morphism to $(\times \circ \Delta)(C) = \times (\Delta (C)) = \times (C,C) = C \times C$,
thus having the form
$\eta_{C} : C \to C \times C$.

${}$

This $\eta_{C}$ is defined as the morphism on the RHS corresponding to the
identity morphism $1_{\Delta(C)} : \Delta(C) \to \Delta(C)$ on the LHS
of the adjunction isomorphism
$$
	\mathrm{Hom}_{\mathbf{C} \times \mathbf{C}}(\Delta(C), (X,Y)) \cong \mathrm{Hom}_{\mathbf{C}}(C,\times (X,Y))
$$
by specifically choosing $(X,Y) = \Delta (C) = (C,C)$.

Here, $1_{\Delta(C)}$
is, by the definition of the product category, the pair of morphisms
$(1_{C}, 1_{C})$.
$$
	1_{\Delta(C)} = (1_{C}, 1_{C}) : (C,C) \to (C,C)
$$

${}$

On the other hand, by the universal property of the product $C \times C$
$$
	\mathrm{Hom}_{\mathbf{C}}(C, C \times C)
	\cong
	\mathrm{Hom}_{\mathbf{C}}(C, C)
	\times
	\mathrm{Hom}_{\mathbf{C}}(C, C)
$$
the morphism in $\mathrm{Hom}_{\mathbf{C}}(C, C \times C)$
corresponding to the pair of morphisms $(1_{C}, 1_{C})$
is the unique morphism $f: C \to C \times C$ satisfying
$$
	p_{1} \circ f = 1_{C} \ \ \text{and} \ \ p_{2} \circ f = 1_{C}
$$

This is none other than the definition of the so-called \textbf{diagonal morphism} $\delta_{C}$.
Therefore, the unit of the adjunction is the diagonal morphism $\eta_{C} = \delta_{C}$.

\ \\

Let's consider the universal property of the unit $\eta$.

The universal property of the unit $\eta$ is expressed in this context as follows.

Any morphism
$f: C \to X \times Y \ \ (\in \mathbf{C})$
can be factored through $\eta_{C}$ and
the unique morphism
$g: \Delta(C) \to (X,Y) \ \ (\in \mathbf{C} \times \mathbf{C})$
that corresponds to $f$ via the adjunction.

${}$

If we write the pair of morphisms $g_{1}: C \to X$ and $g_{2}: C \to Y$ as
$g=(g_{1},g_{2})$,
the action of the functor $R=\times$ on morphisms is
$$
	R(g) = g_{1} \times g_{2} : C \times C \to X \times Y
$$

In this case, from the definition of the adjunction
$$
	f =R(g) \circ \eta_{C}
$$
we have
$$
	f = (g_{1} \times g_{2}) \circ \delta_{C}
$$

Expressing this relationship as a commutative diagram gives the following.

\[
	\begin{tikzcd}
		{(C,C)} \arrow[rr, "{\exists ! \ \! (g_{1},g_{2})}"] && {(X,Y)}\\
		&&\\
		C \times C \arrow[rr, "R(g)=g_{1} \times g_{2}"]&& X \times Y \\
		&&\\
		C \arrow[rruu, "f"'] \arrow[uu, "\eta_{C}=\delta_{C}"]&&
	\end{tikzcd}
\]

Here, $f : C \to X \times Y$
and $g=(g_{1},g_{2}): (C,C) \to (X,Y)$
correspond one-to-one via the adjunction.

\begin{thebibliography}{9}
	\bibitem{Awodey}
	Category Theory 2nd Edition - Steve Awodey
\end{thebibliography}

\end{document}
\documentclass[uplatex,a4j,12pt,dvipdfmx]{jsarticle}
\usepackage[english]{babel}
\usepackage[letterpaper,top=2cm,bottom=2cm,left=3cm,right=3cm,marginparwidth=1.75cm]{geometry}
\usepackage{amsmath, amssymb}
\usepackage{graphicx}
\usepackage[colorlinks=true, allcolors=blue]{hyperref}
\usepackage{fancybox}
\usepackage{tikz-cd}
\title{
A First Step into Category Theory
}

\author{
Masaru Okada
}

\begin{document}
\maketitle

\begin{abstract}
Memos on what the author, who does not specialize in mathematics, thought about when getting started with category theory.
\end{abstract}

\section{What is a Category?}

\subsection{Definition of a Category}

A category is a structure consisting of three axioms. Roughly speaking, if you have "objects," "morphisms," and "composition of morphisms," you have a category:

\begin{enumerate}
    \item There are objects.
    \item There are morphisms between objects.
    \item Morphisms can be composed.
\end{enumerate}

More specifically, a category ${\mathcal C}$ is defined as follows:

\begin{enumerate}
    \item Let $X,Y$ be objects of the category ${\mathcal C}$. There is a morphism $f: X \to Y$ between them.
    The set of these morphisms is written as ${\mathcal C}(X,Y)$ or ${\rm Hom}_{\mathcal C}(X,Y)$, etc., and is called a hom-set.
    \item For morphisms $f: X \to Y, g: Y \to Z, h: Z \to W$, the associative law $(hg)f = h(gf): X \to W$ holds.
    \item An identity morphism ${\rm id}_{X}: X \to X$ exists.
\end{enumerate}

Categories can be represented by diagrams.
A diagram is said to be commutative if any paths with the same starting and ending points result in the same morphism.
For example, in a diagram like this:
\[
\begin{tikzcd}
X \arrow[r,"f"]\arrow[rd,"h"'] & Y \arrow[d,"g"]\\
& Z
\end{tikzcd}
\]
the diagram is commutative when $gf=h$.

One might feel a slight sense of unease when first learning category theory here, but the elements that have appeared, such as
$X,Y,Z$
are not elements of a set; they are objects of a category. Therefore, concrete examples would be mathematical structures themselves, such as vector spaces, topological spaces, groups, or rings.

This means we are focusing on a meta-level one step higher than elementary mathematics, for example, set theory.

\subsection{Examples of Categories}

Frequently encountered categories have established notations.
For example:

\begin{enumerate}
    \item When the objects are sets, ${\mathcal C} = {\bf Set}$ is used. In this case, the morphisms are functions.
    \item When the objects are vector spaces over a field $k$, ${\mathcal C} = {\bf Vekt}_{k}$ is used. The morphisms are linear maps.
    \item When the objects are topological spaces, ${\mathcal C} = {\bf Top}$ is used. The morphisms are continuous maps.
    \item When the objects are groups, ${\mathcal C} = {\bf Grp}$ is used. The morphisms are group homomorphisms.
    \item When the objects are modules over a ring $R$, ${\mathcal C} = R \ {\bf Mod}$ is used. The morphisms are $R$-module homomorphisms.
\end{enumerate}

As a technical note, it is known that the "collection of all sets" does not form a set.
Therefore, we restrict ourselves to those "collections that allow for normal discussions in category theory," which are called locally small categories.

\subsection{Opposite Category}

For a category ${\mathcal C}$, the opposite category ${\mathcal C}^{\rm op}$ is one that has the same objects as ${\mathcal C}$ but with the direction of the morphisms reversed.

That is, a category where ${\mathcal C}^{\rm op}(X,Y) = {\mathcal C}(Y,X)$.

Opposite categories also appear in the definition of presheaves and in Yoneda's lemma.

\subsection{Inverse Morphism and Isomorphism}

For a morphism $f: X \to Y$ in a category ${\mathcal C}$, we can consider a reverse morphism $f^{-1}: Y \to X$.

When multiplying from the left gives $f^{-1} f = {\rm id}_{X}$, $f^{-1}$ is said to be left-invertible.

When multiplying from the right gives $ff^{-1} = {\rm id}_{Y}$, $f^{-1}$ is said to be right-invertible.

And when it is both left- and right-invertible, it is said to be invertible.
In this case, $f$ is called an isomorphism, and the objects $X, Y$ are also called isomorphic, written as $X \simeq Y$.

The terminology is a bit tricky, but an invertible morphism is called an isomorphism, and "isomorphism" is sometimes used as a shorthand for "isomorphic morphism." There is also the concept of isomorphism between objects, distinct from that of a morphism.

Isomorphism between objects is an equivalence relation, satisfying the reflexive, symmetric, and transitive properties.

In some categories, a special name is given to isomorphisms.
For example:

\begin{enumerate}
    \item In the category of sets ${\bf Set}$, an isomorphic morphism is called a bijection, and isomorphic objects have the same cardinality.
    \item In the category of vector spaces over a field $k$, ${\mathcal C} = {\bf Vekt}_{k}$, an isomorphic morphism is called a linear isomorphism, and isomorphic objects have the same dimension.
    \item In the category of topological spaces ${\mathcal C} = {\bf Top}$, an isomorphic morphism is called a homeomorphism, and isomorphic objects are homeomorphic.
    \item In the category of groups ${\mathcal C} = {\bf Grp}$, an isomorphic morphism is called a group isomorphism, and isomorphic objects are group isomorphic.
    \item In the category of modules over a ring $R$, ${\mathcal C} = R \ {\bf Mod}$, an isomorphic morphism is called a module isomorphism, and isomorphic objects are module isomorphic.
\end{enumerate}

\subsection{Pushforward and Pullback}

For a morphism $f: X \to Y$ and an object $Z$ in a category, the pushforward by $f$, $f_{*} : \mathcal{C}(Z,X) \to \mathcal{C}(Z,Y)$, is defined as follows:
\[
\begin{tikzcd}
\mathcal{C}(Z,X) \arrow[r,"f_{*}"] & \mathcal{C}(Z,Y) \\
X \arrow[r,"f"] & Y \\
Z \arrow[u,"g"] \arrow[ur,"f_{*}"'] &
\end{tikzcd}
\]
The pushforward composes $f$ from the front: $f_{*} : g \mapsto fg$.

A concept similar to pushforward is the pullback $f^{*} :\mathcal{C}(Y,Z) \to \mathcal{C}(X,Z)$, which is defined as follows:
\[
\begin{tikzcd}
\mathcal{C}(X,Z) & \mathcal{C}(Y,Z) \arrow[l,"f^{*}"'] \\
X \arrow[r,"f"] \arrow[dr,"f^{*}"'] & Y \arrow[d,"g"] \\
& Z
\end{tikzcd}
\]
The pullback composes $f$ from the back: $f^{*} : g \mapsto gf$.

Intuitively, the pushforward is a morphism pushed out from $Z$, while the pullback is one pulled back to $Z$.

\subsubsection{Theorem on Pushforward, Pullback, and Isomorphism}

Regarding pushforward and pullback, the following are equivalent:

\begin{enumerate}
    \item $f : X \to Y$ is an isomorphism.
    \item For any object $Z$, the pushforward $f_{*} : \mathcal{C}(Z,X) \to \mathcal{C}(Z,Y)$ is an isomorphism of sets.
    \item For any object $Z$, the pullback $f^{*} :\mathcal{C}(Y,Z) \to \mathcal{C}(X,Z)$ is an isomorphism of sets.
\end{enumerate}

This theorem states an important viewpoint in category theory:

\begin{center}
\fbox{Objects are completely determined by their relationships with other objects}
\end{center}

Pushforward and pullback describe how a morphism $f: X \to Y$ transforms the relationships (Hom-sets) with other objects $Z$.

If this transformation is "completely isomorphic in its relationship with all objects (i.e., no information is lost)," then there is no longer any distinguishing difference between $X$ and $Y$.

Therefore, an object is not determined by "itself," but its existence is characterized by the "entirety of its relationships with other objects."

This is the meaning of the categorical adage, "Objects are completely determined by their relationships with other objects."

${}$

To summarize the logical flow of this claim from the theorem:

${}$

{\bf 1. Morphisms represent relationships}

In category theory, you cannot directly see the "contents" of an object $X$.

Instead, $X$ is characterized by the morphisms (relationships) it has with other objects.

${}$

{\bf 2. Pushforward and Pullback}

The pushforward $f_{*} : \mathcal{C}(Z,X) \to \mathcal{C}(Z,Y)$ is an operation that pushes a path from $Z$ to $X$ into a path to $Y$.

The pullback $f^{*} :\mathcal{C}(Y,Z) \to \mathcal{C}(X,Z)$ is an operation that pulls back a path to $Z$ into a path from $X$.

Both show how $f$ changes the relationships with other objects.

${}$

{\bf 3. Detecting Isomorphism}

If $f$ is an isomorphism, the set of relationships is completely preserved for every $Z$.

Conversely, if the pushforward and pullback are isomorphic for all objects $Z$, then $f$ loses no information and is, in fact, an isomorphism.

${}$

{\bf 4. Conclusion of the Claim: Relationships define objects}

When asking what an object is, category theory defines the object itself by how it is positioned within a network of relationships with others.

Therefore, the conclusion of the claim is that objects are completely determined by their relationships with other objects.

${}$

It might sound a bit poetic, but something similar could be said about people and their relationships.
When you can't directly measure a person's character, one way to understand them is through who they have relationships with and what those relationships are like.
This is similar to how categorical objects are characterized by their network of relationships.



\end{document}
\documentclass[uplatex,a4j,12pt,dvipdfmx]{jsarticle}
\usepackage[english]{babel}
\usepackage[letterpaper,top=2cm,bottom=2cm,left=3cm,right=3cm,marginparwidth=1.75cm]{geometry}
\usepackage{amsmath, amssymb}
\usepackage{graphicx}
\usepackage[colorlinks=true, allcolors=blue]{hyperref}
\usepackage{fancybox}
\usepackage{tikz-cd}
\title{
圏のはじめの第一歩
}

\author{
mastex
}

\begin{document}
\maketitle

\begin{abstract}
数学を専門に学んでいない著者が圏論について入門していくときに思ったことのメモ。
\end{abstract}

\section{圏とは}

\subsection{圏の定義}

圏は3つの公理から成る構造である。ラフに言えば、「対象」、「射」、「射の結合」があれば圏になる:

\begin{enumerate}
    \item 対象があること
    \item 対象の間に射があること
    \item 射は結合できること
\end{enumerate}

具体的に書くと、圏 ${\mathcal C}$ とは

\begin{enumerate}
    \item 圏 ${\mathcal C}$ の対象を $X,Y$ とする。この間に射 $f: X \to Y$ がある。
    この射の集合は ${\mathcal C}(X,Y)$ または ${\rm Hom}_{\mathcal C}(X,Y)$ 等と書かれて、ホム集合と呼ばれる。
    \item 射 $f: X \to Y, g: Y \to Z, h: Z \to W$ に対し、結合律 $(hg)f = h(gf): X \to W$ を満たす。
    \item 恒等射 ${\rm id}_{X}: X \to X$ が存在する。
\end{enumerate}

図式で表現することができる。
図式が可換であるとは、図の始点と終点が同じ経路であれば同じ射になることである。
例えば、
\[
\begin{tikzcd}
X \arrow[r,"f"]\arrow[rd,"h"'] & Y \arrow[d,"g"]\\
& Z
\end{tikzcd}
\]
のように書かれたとき、$gf=h$ となるとき図式は可換である。

ここで初めて圏論を学ぶときに少し違和感を覚えるかもしれないが、ここに出てきた要素、
例えば
$X,Y,Z$
等は集合の要素ではなく、圏の対象であるので、具体例としてはベクトル空間や位相空間、群や環といった数学的構造そのものが入る。

初等的な数学で取り扱うような、例えば集合論よりも一つメタな階層に注目していることになっている。

\subsection{圏の例}

頻出される圏には表記が定められていることがある。
例えば、

\begin{enumerate}
    \item 対象が集合のとき、${\mathcal C} = {\bf Set}$ 等と書かれる。このとき射は写像である。
    \item 対象が体 $k$ 上のベクトル空間のとき、${\mathcal C} = {\bf Vekt}_{k}$ 等と書かれる。射は線形写像である。
    \item 対象が位相空間のとき、${\mathcal C} = {\bf Top}$ 等と書かれる。射は連続写像である。
    \item 対象が群のとき、${\mathcal C} = {\bf Grp}$ 等と書かれる。射は群準同型である。
    \item 対象が環 $R$ 上の加群のとき、${\mathcal C} = R \ {\bf Mod}$ 等と書かれる。射は $R$ 加群の準同型である。
\end{enumerate}

技術的な注意として、「すべての集合を集めたもの」というのは集合にならないことが知られている。
そこで、「集めても圏論の普通の議論ができるもの」に限って集合とする。
こういったものは局所小圏と呼ばれるらしい。

\subsection{反対圏}

圏 ${\mathcal C}$ に対し、反対圏 ${\mathcal C}^{\rm op}$ とは、
対象は ${\mathcal C}$ と同じで射の向きが反対のものである。

すなわち ${\mathcal C}^{\rm op}(X,Y) = {\mathcal C}(Y,X)$ となっているような圏である。

前層の定義や米田の補題でも反対圏が出てくる。

\subsection{逆射と同型}

圏 ${\mathcal C}$ の射 $f: X \to Y$ に対して、$f^{-1}: Y \to X$ という逆向きの射を考える。

逆向きの射を左から掛けて $f^{-1} f = {\rm id}_{X}$ となるとき、$f^{-1}$ は左可逆という。

逆向きの射を右から掛けて $ff^{-1} = {\rm id}_{Y}$ となるとき、$f^{-1}$ は右可逆という。

そして左可逆かつ右可逆のとき可逆であるといい、
このとき $f$ は同型であると呼ばれ、対象$X, Y$ もまた、同型であると呼ばれ$X \simeq Y$ と書かれる。

言葉遣いがややこしいが、射が可逆であれば同型射と呼ばれ、同型射を省略して同型と呼ばれている。
射の同型とは別に対象の間の同型という概念もある。

対象の間の同型は同値関係であり、反射律、対象律、推移律を満たす。

圏の中には同型に特別な名前が決められていることがある。
例えば、

\begin{enumerate}
    \item 集合の圏 ${\bf Set}$ のとき、同型射は全単射と呼ばれ、同型な対象同士の濃度は等しくなる。
    \item 体 $k$ 上のベクトル空間 ${\mathcal C} = {\bf Vekt}_{k}$ のとき、同型射は線形同型写像と呼ばれ、同型な対象同士は同じ次元になる。
    \item 位相空間の圏 ${\mathcal C} = {\bf Top}$ のとき、同型射は同相写像と呼ばれ、同型な対象同士は同相になる。
    \item 群の圏 ${\mathcal C} = {\bf Grp}$ のとき、同型射は群同型写像と呼ばれ、同型な対象同士は群同型になる。
    \item 環 $R$ 上の加群の圏 ${\mathcal C} = R \ {\bf Mod}$ のとき、同型射は加群同型写像と呼ばれ、同型な対象同士は加群同型になる。
\end{enumerate}

\subsection{押し出しと引き戻し}

圏の射$f : X \to Y$ と 対象 $Z$ に対して、$f$ による押し出し $f_{*} : \mathcal{C}(Z,X) \to \mathcal{C}(Z,Y)$ は次で定義される。
\[
\begin{tikzcd}
\mathcal{C}(Z,X) \arrow[r,"f_{*}"] & \mathcal{C}(Z,Y) \\
X \arrow[r,"f"] & Y \\
Z \arrow[u,"g"] \arrow[ur,"f_{*}"'] &
\end{tikzcd}
\]
押し出しによって $f$ は後ろから合成される: $f_{*} : g \mapsto fg$ 。

押し出しと似た概念で引き戻し $f^{*} :\mathcal{C}(Y,Z) \to \mathcal{C}(X,Z)$ は次のように定義される。
\[
\begin{tikzcd}
\mathcal{C}(X,Z) & \mathcal{C}(Y,Z) \arrow[l,"f^{*}"'] \\
X \arrow[r,"f"] \arrow[dr,"f^{*}"'] & Y \arrow[d,"g"] \\
& Z
\end{tikzcd}
\]
引き戻しによって $f$ は前から合成される: $f_{*} : g \mapsto gf$ 。

感覚的には、押し出しは $Z$ から押し出されて、引き戻しは $Z$ に引き戻されるような射になっている。



\subsubsection{押し出しと引き戻しと同型についての定理}

押し出しと引き戻しについて、以下はそれぞれ同値である。

\begin{enumerate}
    \item $f : X \to Y$ は同型。
    \item 任意の対象 $Z$ に対して、押し出し $f_{*} : \mathcal{C}(Z,X) \to \mathcal{C}(Z,Y)$ は集合の間の同型。
    \item 任意の対象 $Z$ に対して、引き戻し $f^{*} :\mathcal{C}(Y,Z) \to \mathcal{C}(X,Z)$ は集合の間の同型。
\end{enumerate}

この定理は

\begin{center}
\fbox{対象は他の対象との関係で完全に決まる}
\end{center}

という圏論における重要な観点を述べている。

押し出し・引き戻しは、ある射 $f: X \to Y$ が他の対象 $Z$ との関係(Hom集合)をどう変換するかを記述している。

もしこの変換が「すべての対象との関係において完全に同型(=情報を失っていない)」ならば、もはや $X$ と $Y$ の間には区別すべき差が残っていない。

したがって、対象は「自分だけ」で決まるのではなく、「他の対象との関係全体」によってその存在が特徴付けられる。

これが「対象は他の対象との関係で完全に決まる」という圏論的金言の意味である。

${}$

定理から言えるこの主張の論理の流れを整理すると、

${}$

{\bf 1. 射は関係を表す}

圏論において、対象 $X$ の中身を直接覗くことはできない。

代わりに、他の対象との射(関係)によって $X$ を特徴づける。

${}$

{\bf 2.押し出しと引き戻し}

押し出し $f_{*} : \mathcal{C}(Z,X) \to \mathcal{C}(Z,Y)$ は
$Z$ から $X$ へ行く道を、$Y$ に行く道に押し出す操作。

引き戻し $f^{*} :\mathcal{C}(Y,Z) \to \mathcal{C}(X,Z)$ は
$Z$ への道を $X$ からの道に引き戻す操作。

いずれも $f$ が他の対象との関係をどう変更するかを示している。

${}$

{\bf 3. 同型性の検出}

もし $f$ が同型なら、どの $Z$ に対しても関係の集合が完全に対応付けられる。

逆に、もしすべての対象 $Z$ について押し出し・引き戻しが同型なら、 $f$ が失う情報はなく、実際に同型射である。

${}$

{\bf 4. 主張の結論:関係が対象を決める}

ある対象が何者であるかを問うとき、圏論は、他との関係網の中でどう位置づけられるか、で対象それ自体を定める。

したがって、対象は他の対象との関係で完全に決まるというのが主張の結論である。

${}$

ちょっとポエム的になるが、人物と人間関係についても似たようなことが言えそう。
人物の性格を直接測れないとき、その人が誰とどういう関係を持つかで理解するという観点もある。
圏論的対象もこういった関係の網で対象そのものを特徴付けられるという話でした。



\end{document}
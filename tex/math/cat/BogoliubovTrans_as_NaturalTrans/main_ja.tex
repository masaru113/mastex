\documentclass[uplatex,a4j,12pt,dvipdfmx]{jsarticle}
\usepackage[english]{babel}
\usepackage[letterpaper,top=2cm,bottom=2cm,left=3cm,right=3cm,marginparwidth=1.75cm]{geometry}
\usepackage{amsmath, amssymb}
\usepackage{graphicx}
\usepackage[colorlinks=true, allcolors=blue]{hyperref}
\usepackage{tikz-cd}

\title{
自然変換としてのボゴリューボフ変換
}

\author{
岡田 大 (Okada Masaru)
}

\begin{document}
\maketitle

\begin{abstract}
	ボゴリューボフ変換は、場の量子論において特定の代数構造を保存する線形変換であり、単なる行列の対角化操作を超えた深い数学的構造を持つ。本稿では、この変換が圏論の自然変換、およびリー代数コホモロジーの概念とどのように結びつくかを、数学的に厳密な視点から解説する。
\end{abstract}


\section{リー代数とC*環}

ボゴリューボフ変換が持つ代数的構造をより深く理解するためには、リー代数とC*環という二つの重要な数学的対象の関連性を知ることが不可欠である。これらの概念は、量子力学における物理量の数学的記述と深く結びついている。

\subsection{リー代数の定義}

リー代数とは、交換子積という演算を持つベクトル空間である。これは、量子力学における演算子の交換関係を数学的に抽象化したものである。

\begin{itemize}
	\item \textbf{リー代数の定義}:
	      体 $k$ 上のベクトル空間 $\mathfrak{g}$ に、以下の性質を満たす双線形写像 $[\cdot, \cdot]: \mathfrak{g} \times \mathfrak{g} \to \mathfrak{g}$ が定義されているとき、$(\mathfrak{g}, [\cdot, \cdot])$ をリー代数と呼ぶ。
	      \begin{enumerate}
		      \item \textbf{交代性 (Alternating)}: 任意の $X \in \mathfrak{g}$ に対して、$[X, X] = 0$ が成り立つ。
		      \item \textbf{ヤコビ恒等式 (Jacobi Identity)}: 任意の $X, Y, Z \in \mathfrak{g}$ に対して、
		            \[
			            [X, [Y, Z]] + [Y, [Z, X]] + [Z, [X, Y]] = 0
		            \]
		            が成り立つ。
	      \end{enumerate}
	\item \textbf{物理との関連}:
	      量子力学において、物理量(オブザーバブル)は線形演算子で表され、その交換関係 $[A, B] = AB - BA$ が物理的な性質(例:不確定性原理)を決定する。この交換関係は、リー代数のリー括弧の公理を満たす。したがって、物理量の集合は、交換子積をリー括弧とするリー代数を形成する。
\end{itemize}

\subsection{C*環の定義}

C*環は、複素数上の代数であり、解析学的な構造(ノルムと完備性)と代数的な構造(乗法、加法、随伴演算)を併せ持つ。

\begin{itemize}
	\item \textbf{C*環の定義}:
	      複素数上の環 $A$ が、以下の性質を満たすとき、C*環と呼ぶ。
	      \begin{enumerate}
		      \item $A$ は複素ベクトル空間であり、乗法と加法が結合法則、分配法則を満たす。
		      \item ノルム $\|\cdot\|$ が定義され、$A$ はこのノルムに関してバナッハ空間(完備なノルム空間)である。
		      \item 対合(Involution)$*: A \to A$ が定義され、以下の性質を満たす。
		            \begin{itemize}
			            \item $(x+y)^* = x^*+y^*$
			            \item $(\alpha x)^* = \bar{\alpha} x^*$
			            \item $(xy)^* = y^*x^*$
			            \item $(x^*)^* = x$
		            \end{itemize}
		      \item C*条件(C*-identity): 任意の $x \in A$ に対して、
		            \[
			            \|x^*x\| = \|x\|^2
		            \]
		            が成り立つ。
	      \end{enumerate}
	\item \textbf{物理との関連}:
	      量子力学において、ヒルベルト空間上の有界な線形作用素全体はC*環を形成する。このC*環は、量子系のすべての観測可能な量を包含する代数的な枠組みを提供する。自己共役元($A=A^*$)は物理的なオブザーバブルに対応する。
\end{itemize}

\subsection{リー代数とC*環の関係}

リー代数とC*環は、量子力学の記述において相補的な役割を果たす。

\begin{itemize}
	\item \textbf{リー代数}: 物理系の連続的な対称性(回転、並進など)の\textbf{局所的な性質}を、交換関係という線形的な構造で捉える。
	\item \textbf{C*環}: 物理系の観測可能な量全体を、解析的・代数的な構造を持つ\textbf{大域的な枠組み}で記述する。
\end{itemize}
リー代数からC*環への移行は、群の表現論を通じて行われる。リー群の表現(物理系の対称性の具体的な実現)を微分することで、対応するリー代数の表現が得られる。このリー代数の表現が張る作用素の集合から、C*環が生成される。この関係は、量子力学のハミルトニアンや運動量演算子といったリー代数の元が、C*環というより広範な代数構造に埋め込まれていることを示している。


\section{自然変換としてのボゴリューボフ変換}

ボゴリューボフ変換は、超伝導のBCS理論やボース・アインシュタイン凝縮において、元の粒子の生成・消滅演算子から新しい準粒子を定義する重要な操作である。この変換の「自然さ」は、圏論の\textbf{自然変換}として厳密に定式化できる。

\subsection{関手の設定}

まず、ボゴリューボフ変換を記述する適切な圏 $\mathcal{C}$ と、その上の関手 $F, G: \mathcal{C} \to \mathcal{D}$ を定義する。

\begin{itemize}
	\item \textbf{圏 $\mathcal{C}$}: 対象を、物理系の状態を記述する運動量 $k \in \mathbb{R}^3$ とし、射を、運動量の間の連続写像 $f: k \to k'$ とする。
	\item \textbf{関手 $F$(元の粒子系)}: 各運動量 $k \in \mathcal{C}$ を、その運動量に対応する生成・消滅演算子のペア $\{a_{k}, a_{k}^{\dagger}\}$ が張る、複素ベクトル空間 $V_{k} = \mathrm{span}_\mathbb{C}\{a_{k}, a_{k}^{\dagger}\}$ に写す。射 $f: k \to k'$ は、線形変換 $F(f): V_{k} \to V_{k'}$ を誘導する。
	\item \textbf{関手 $G$(準粒子系)}: 各運動量 $k \in \mathcal{C}$ を、ボゴリューボフ変換によって定義される準粒子の演算子ペア $\{\alpha_{k}, \alpha_{k}^{\dagger}\}$ が張るベクトル空間 $W_{k} = \mathrm{span}_\mathbb{C}\{\alpha_{k}, \alpha_{k}^{\dagger}\}$ に写す。射 $f: k \to k'$ は、線形変換 $G(f): W_{k} \to W_{k'}$ を誘導する。
\end{itemize}

\subsection{自然変換の定式化}

ボゴリューボフ変換は、関手 $F$ から $G$ への\textbf{自然変換} $\text{Bog}: F \Rightarrow G$ として定義される。この変換は、各運動量 $k$ ごとに、線形同型写像 $\text{Bog}_{k}: F(k) \to G(k)$ の族 $\{\text{Bog}_{k}\}_{k \in \mathcal{C}}$ として与えられる。

自然変換の定義により、任意の射 $f: k \to k'$ に対して、以下の図式が可換でなければならない。

\[
	\begin{tikzcd}
		F(k) \arrow[r, "F(f)"] \arrow[d, "\text{Bog}_{k}"'] & F(k') \arrow[d, "\text{Bog}_{k'}"] \\
		G(k) \arrow[r, "G(f)"'] & G(k')
	\end{tikzcd}
\]

この可換性は、ボゴリューボフ変換が、物理系の座標変換(運動量変換)と整合的であることを意味する。つまり、元の粒子を変換してからボゴリューボフ変換で準粒子に変換する操作と、先にボゴリューボフ変換で準粒子に変換してから変換された準粒子を変換する操作が等価である。

\subsection{可換性の証明}

この図式が可換であることを、成分計算によって示す。
ボゴリューボフ変換は、係数 $u_{k}, v_{k}$ を用いて以下のように定義される。
\[
	\begin{cases}
		\alpha_{k} = u_{k} a_{k} - v_{k} a_{-k}^{\dagger} \\
		\alpha_{k}^{\dagger} = u_{k} a_{k}^{\dagger} - v_{k} a_{-k}
	\end{cases}
\]
運動量変換 $f: k \to k'$ が、演算子レベルで $a_{k} \mapsto a_{k'}$ を誘導する。このとき、 $a_{-k} \mapsto a_{-k'}$ も同様に成り立つ。

\paragraph{左辺の計算: $\text{Bog}_{k'} \circ F(f)$}
演算子 $a_{k}$ に作用させる。
\[
	(\text{Bog}_{k'} \circ F(f))(a_{k}) = \text{Bog}_{k'}(a_{k'}) = u_{k'} \alpha_{k'} - v_{k'} \alpha_{-k'}^{\dagger}
\]

\paragraph{右辺の計算: $G(f) \circ \text{Bog}_{k}$}
演算子 $a_{k}$ に作用させる。
\[
	(G(f) \circ \text{Bog}_{k})(a_{k}) = G(f)(\text{Bog}_{k}(a_{k})) = G(f)(u_{k} \alpha_{k} - v_{k} \alpha_{-k}^{\dagger})
\]
$G(f)$ の線形性により、
\[
	= u_{k} G(f)(\alpha_{k}) - v_{k} G(f)(\alpha_{-k}^{\dagger})
\]
ここで、$G(f)(\alpha_{k}) = \alpha_{k'}$ と仮定すると、
\[
	= u_{k} \alpha_{k'} - v_{k} \alpha_{-k'}^{\dagger}
\]

物理的な対称性(例えば並進対称性)を持つ系では、ボゴリューボフ変換の係数 $u_{k}, v_{k}$ は運動量の大きさにのみ依存し、方向には依存しないため、 $u_{k'} = u_{k}, v_{k'} = v_{k}$ が成り立つ。この条件下で、左辺と右辺は一致する。

\section{ボゴリューボフ変換とリー代数コホモロジー}

ボゴリューボフ変換は、単なる線形変換ではなく、\textbf{交換関係という代数構造を保存する}。この性質は、リー代数コホモロジーの概念を用いて、より深く理解できる。

\subsection{リー代数とコバウンダリー作用素}

\paragraph{リー代数 $\mathfrak{g}$}:
ボソン演算子 $\{a_{k}, a_{k}^{\dagger}\}$ は、交換子積 $[A, B] = AB-BA$ をリー括弧として、\textbf{Heisenbergリー代数}を生成する。この代数に、ハミルトニアンなどの物理量を表現する行列の空間 $\mathrm{End}(V_{k})$ を加えたものも、リー代数を形成する。

\paragraph{コバウンダリー作用素 $d_{\text{H}}$}:
ある物理量 $H$(ハミルトニアン)に対して、\textbf{コバウンダリー作用素} $d_{\text{H}}: \mathfrak{g} \to \mathfrak{g}$ を、交換子積として定義する。
\[
	d_{\text{H}}(X) = [H, X]
\]
この作用素は、物理量 $H$ と可換でない(非保存的な)作用素を、可換な量から生成する役割を持つ。

\subsection{非対角項のコホモロジー的解釈}

超伝導のハミルトニアン $H_{\text{0}}$ は、対角項 $H_{\text{diag}}$ と非対角項 $H_{\text{off-diag}}$ の和として表される。
\[
	H_{\text{0}} = H_{\text{diag}} + H_{\text{off-diag}}
\]
ここで、ボゴリューボフ変換は、この非対角項を消去する操作である。この操作は、非対角項が\textbf{コバウンダリー}であることを示唆する。

\textbf{主張}: 非対角項 $H_{\text{off-diag}}$ は、ある作用素 $B \in \mathfrak{g}$ を用いて、対角化されたハミルトニアン $H_{\text{diag}}$ のコバウンダリーとして表現できる。
\[
	H_{\text{off-diag}} = d_{H_{\text{diag}}}(B) = [H_{\text{diag}}, B]
\]
この $B$ が、ボゴリューボフ変換を生成する作用素に対応する。この関係式は、ボゴリューボフ変換によって、非対角項が「取り除かれるべき量」(コバウンダリー)として識別されることを示している。

\subsection{物理的同値性とコホモロジー群}

ボゴリューボフ変換で結びつけられるハミルトニアン $H_{\text{0}}$ と $H_{\text{diag}}$ は、物理的に同じ系を記述している。この同値性は、コホモロジーの観点から以下のように解釈される。

\begin{itemize}
	\item コホモロジー群 $H^1$: あるリー代数 $\mathfrak{g}$ 上の1次コホモロジー群 $H^1(\mathfrak{g}, \mathfrak{g})$ は、コサイクル(保存量)の空間をコバウンダリーの空間で割った商空間である。
	      \[
		      H^1(\mathfrak{g}) = \frac{\ker d_{\text{H}}}{\mathrm{im} d_{\text{H}}} = \frac{\{X \mid [H, X] = 0\}}{\{[H, Y] \mid Y \in \mathfrak{g}\}}
	      \]
	\item 物理的意義: ボゴリューボフ変換は、ハミルトニアンを、その交換子積を通じて「コホモロジー的に同値」な記述に変換する。これは、非対角項が、ゲージ変換と同様に、記述の形式に依存する「見かけの量」であり、系の本質的な物理(コホモロジー類)は不変であることを示唆している。
\end{itemize}

\section{普遍性と物理的意義}

ボゴリューボフ変換は、超伝導理論における準粒子を定義する\textbf{最も自然で普遍的な方法}である。この普遍性は、以下の点で物理的意義を持つ。

\begin{itemize}
	\item \textbf{準粒子の概念}: 準粒子は、相互作用する多体系において、系の本質的な物理を単純な粒子として記述するための\textbf{普遍的な「数学的ツール」}である。
	\item \textbf{ゲージ不変性}: ボゴリューボフ変換は、物理的な対称性(例えば運動量保存)と整合的であり、その変換によって、系のエネルギー固有値といった本質的な物理量は不変に保たれる。
\end{itemize}
ボゴリューボフ変換は、単なる数学的操作ではなく、量子多体系の複雑な現象を、\textbf{圏論の自然変換}や\textbf{リー代数コホモロジー}といった体系的な数学的概念によって、より深く理解するための強力なツールである。
\end{document}
\documentclass[uplatex,a4j,12pt,dvipdfmx]{jsarticle}
\usepackage[english]{babel}
\usepackage[letterpaper,top=2cm,bottom=2cm,left=3cm,right=3cm,marginparwidth=1.75cm]{geometry}
\usepackage{amsmath, amssymb}
\usepackage{graphicx}
\usepackage[colorlinks=true, allcolors=blue]{hyperref}
\usepackage{tikz-cd}
\title{
Yoneda's Lemma, Analogies and Proof
}

\author{
Masaru Okada
}

\begin{document}
\maketitle

\begin{abstract}
	On Yoneda's Lemma. Analogies and proof for a deeper understanding.
\end{abstract}

\section{Statement of Yoneda's Lemma}

Let $C$ be a category and $a \in C$ be an object.

We define $y(a)$ as $y(a) = \text{Hom}_{C}(-,a)$
\footnote{
	In other words, we define $y(a) : C^{op} \to \textbf{Set}$ such that $y(a)(x) = \text{Hom}_{C}(x,a)$ and for any morphism $f: x \to x'$, we define $y(a)(f) = \text{Hom}_{C}(f,a)$.

	Here, $\text{Hom}_{C}(f,a): \text{Hom}_{C}(x',a) \to \text{Hom}_{C}(x,a)$ is defined by mapping $\phi \in \text{Hom}_{C}(x',a)$ to $\phi \circ f$.
}
:

\[
	\begin{array}{cccc}
		y(a) : & C^{op}                  & \to     & \textbf{Set}            \\
		       & \rotatebox{90}{${\in}$} &         & \rotatebox{90}{${\in}$} \\
		       & x                       & \mapsto & \text{Hom}_{C}(x,a)
	\end{array}
\]

For a functor $F: C^{op} \to \textbf{Set}$,

\[
	\text{Hom}_{C}(y(a),F) \cong Fa
\]
holds.

The left-hand side,
$\text{Hom}_{C}(y(a),F)$,
is the "set" of all natural transformations from $y(a)$ to $F$.

The right-hand side, $Fa$, is a "set" because the functor $F$ maps it to a set.

The fact that two sets are isomorphic means that there is a bijection between the left-hand side and the right-hand side.



\subsection{Simple Analogy: The Example of Linear Algebra}

Let $V$ be a real vector space.

It is fine to take $V=\mathbb{R}^{3}$.

In this case, the set of all homomorphisms (i.e., linear maps here) from real numbers to the real vector space is isomorphic to the real vector space.

\[
	\text{Hom}(\mathbb{R},V) \cong V
\]

This isomorphism is a basic one in linear algebra, but it serves as a good analogy for Yoneda's Lemma.

In other words, it is the case where we take the category $C$ to be the category of real vector spaces and $a$ to be the real vector space $\mathbb{R}$.


Looking more closely,

\[
	\begin{array}{ccc}
		\text{Hom}(\mathbb{R},V) & \cong   & V                       \\
		\rotatebox{90}{${\in}$}  &         & \rotatebox{90}{${\in}$} \\
		f                        & \mapsto & f(1)
	\end{array}
\]

Using the identity element 1 of $\mathbb{R}$, we map $f$ to $f(1)$ (i.e., we substitute 1).

Since $f$ is a linear map, for any real number $\lambda$, $f(\lambda) = f(\lambda \cdot 1) = \lambda f(1)$ holds.

This means that the value of $f(\lambda)$ for any $\lambda$ is uniquely determined by the constant $\lambda$ and the value of $f(1)$.

In short, the values of $f$ are uniquely determined once we fix a representative value, $f(1)$.

Due to this, there is a one-to-one correspondence, a bijection, between the set of all linear maps $f$ and the set of all possible values of $f(1)$ (i.e., the elements of the vector space $V$).





\subsection{Simple Analogy: The Example of Set Theory}

Yoneda's Lemma can also be seen as an analogy that reinterprets elements in set theory as maps.

In the usual way of thinking, to identify an element $a$ of a set $A$, we simply pick that element from $A$.

However, from the perspective of Yoneda's Lemma, to identify an element $a$ of a set $A$, it is essential to consider maps from other sets to $A$.

As a concrete example, we consider the category of sets, \textbf{Set}, and a set $A$ and a set with only a single element, $\{*\}$.

\[
	\begin{array}{ccc} \text{Hom}(\{*\}, A)& \cong & A \\ \rotatebox{90}{${\in}$} && \rotatebox{90}{${\in}$} \\ f& \mapsto & f(*) \end{array}
\]
This shows that a map $f: \{*\} \to A$ is completely determined by which element of $A$ the single element $*$ is mapped to.

In other words, the map $f$ corresponds one-to-one with the element $f(*)$ of $A$.

\begin{enumerate}
	\item \textbf{Identification of an element}: An element $a$ of a set $A$ corresponds one-to-one with a map $f: \{*\} \to A$ such that $f(*) = a$.
	\item \textbf{Identification of a map}: The set of all maps from any set $X$ to $A$, denoted as $\text{Hom}(\{*\}, A)$, corresponds one-to-one with $A$.
\end{enumerate}

This analogy shows the essential connection between "the object itself" and "the maps to that object," which is the essence of Yoneda's Lemma.





\subsection{Simple Analogy: The Example of Group Theory}

Yoneda's Lemma can be viewed as an analogy that reinterprets "elements of a group" in terms of "homomorphisms."

In the usual way of thinking, to identify an element $g$ of a group $G$, we simply pick that element from $G$.

However, from the perspective of Yoneda's Lemma, to identify an element $g$ of a group $G$, it is essential to consider a homomorphism from another group to $G$.

As a concrete example, we consider a homomorphism $\phi: \mathbb{Z} \to G$ from the infinite cyclic group $\mathbb{Z}$ to $G$.

In the category of groups, $\mathbf{Grp}$, we consider an object $G$. The important thing to consider for the analogy of Yoneda's Lemma is the infinite cyclic group $\mathbb{Z}$.
\[
	\begin{array}{ccc} \text{Hom}_{\mathbf{Grp}}(\mathbb{Z}, G)& \cong & G \\ \rotatebox{90}{${\in}$} && \rotatebox{90}{${\in}$} \\ \phi& \mapsto & \phi(1) \end{array}
\]
This shows that a group homomorphism $\phi: \mathbb{Z} \to G$ is completely determined by the value of $\phi(1)$.

Since any element $n$ of $\mathbb{Z}$ can be expressed as the sum of $n$ ones, the property of group homomorphisms means that $\phi(n) = n\phi(1)$. Thus, if $\phi(1)$ is determined, the entire map is determined. This creates a one-to-one correspondence between the homomorphism $\phi$ and the element $\phi(1)$ of the group $G$.

Why does this serve as an analogy?

\begin{enumerate}
	\item \textbf{Identification of an element}: An element $g$ of a group $G$ corresponds one-to-one with a homomorphism $\phi: \mathbb{Z} \to G$ such that $\phi(1) = g$.
	\item \textbf{Identification of a map}: The set of all homomorphisms from any group $H$ to $G$, denoted as $\text{Hom}_{\mathbf{Grp}}(\mathbb{Z}, G)$, corresponds one-to-one with $G$. This is because any homomorphism $\phi$ is completely determined by the value of $\phi(1)$.
\end{enumerate}

This analogy shows the essential connection between "the object itself" and "the maps to that object" within the framework of group theory, which is the essence of Yoneda's Lemma.









\section{Proof of Yoneda's Lemma}

Let $\sigma$ be the morphism that maps the set $\text{Hom}_{C}(y(a),F)$ to the set $Fa$:
\[
	\begin{array}{cccc}
		\sigma : & \text{Hom}_{C}(y(a),F)  & \to     & Fa                                     \\
		         & \rotatebox{90}{${\in}$} &         & \rotatebox{90}{${\in}$}                \\
		         & \phi                    & \mapsto & \sigma(\phi) = \phi_{a}(\text{id}_{a})
	\end{array}
\]

Here, $\phi: y(a) \to F$ is a natural transformation.

For an object $x \in C$, the $x$-component is a homomorphism $\phi_{x} : y(a)(x) \to Fx$.
The same is true for $y(a)(x)=\text{Hom}_{C}(x,a)$, so $\phi_{x} : \text{Hom}_{C}(x,a) \to Fx$.

In particular,
\[
	\begin{array}{cccc}
		\phi_{a} : & \text{Hom}_{C}(a,a)     & \to     & Fa                      \\
		           & \rotatebox{90}{${\in}$} &         & \rotatebox{90}{${\in}$} \\
		           & \text{id}_{a}           & \mapsto & \phi_{a}(\text{id}_{a})
	\end{array}
\]

If we can show that this defined $\sigma$ is a bijection, Yoneda's Lemma will be proven.

\subsection{Injectivity}

Here, for any object $x$ in $C$, we take a morphism $p: x \to a$ from $\text{Hom}_{C}(x,a)$.

In this case, we can consider $\phi_{x}(p)$.

In the commutative diagram,
\[
	% https://tikzcd.yichuanshen.de/#N4Igdg9gJgpgziAXAbVABwnAlgFyxMJZAJgBoBGAXVJADcBDAGwFcYkQBPACnoEouAHrxABfUuky58hFABYK1Ok1bsAYgNHiQGbHgJF5AZkUMWbRCFX1NE3dKJljNUyovc+PYWNtT9KMgAMJsrmnJ422pJ6MsjyQc4hahE6vjEBCglm7BrekXZ+yOlOSlkW1iKKMFAA5vBEoABmAE4QALZI6SA4EEiGNIz0AEYwjAAKUfYWTVjVABY4IJmuYR5oXlrNbR003UjkS6EAOodos1gA+sACIhGb7Yh9XT2IZCXLx6cXwPQ3uXd7O2erxcoVUXDWtxa93kTyQAFYDuwPmdIVtEAA2QFIADsiIsaEWIAGwzGEz8IGmcwWFREQA
	\begin{tikzcd}
		&  & y(a) \arrow[rr, "\phi"]              &  & F                  \\
		x \arrow[dd, "p"'] &  & y(a)(x) \arrow[dd, "y(a)(p)"'] \arrow[rr, "\phi_{x}"] &  & Fx \arrow[dd, "F(p)"] \\
		&  &                                  &  &                    \\
		a                  &  & y(a)(a) \arrow[rr, "\phi_{a}"]     &  & Fa
	\end{tikzcd}
\]
This diagram shows that $F(p) \circ \phi_{a} = \phi_{x} \circ y(a)(p)$.

Here, $y(a)(p)$ is a map defined by the composition of morphisms, and specifically,
$y(a)(p): y(a)(a) \to y(a)(x)$ is the map that sends any $\varphi \in y(a)(a) = \text{Hom}_{C}(a,a)$ to $\varphi \circ p$.

Applying the element $\text{id}_{a}$ of $y(a)(a)$ to the above commutative diagram, we get
$F(p)(\phi_{a}(\text{id}_{a})) = \phi_{x}(y(a)(p)(\text{id}_{a}))$

Since $y(a)(p)(\text{id}_{a}) = \text{id}_{a} \circ p = p$,
$F(p)(\phi_{a}(\text{id}_{a})) = \phi_{x}(p)$

Writing it out in a more organized way,
\[
	\phi_{x}(p) = \phi_{x}( \text{id}_{a} \circ p) = \phi_{x}( \text{id}_{a} ) * p
\]

Here, $\phi_{x}( \text{id}_{a} ) \in Fa$. This $F$ is a functor and a $C$-set.
$-*p$ means the map $F(p)(-)$ induced by the functor $F$.

Since we defined $\phi_{x}( \text{id}_{a} ) = \sigma(\phi)$,

\[
	\phi_{x}(p) = F(p)(\sigma(\phi))
\]

This proves injectivity
\[
	\sigma(\phi) = \sigma(\phi') \ \ \Rightarrow \ \ \phi = \phi'
\]
because if $\sigma(\phi)$ is determined, $\phi_{x}(p)$ is also determined.

Thus, $\sigma$ is injective.



\subsection{Surjectivity}

To show that $\sigma$ is surjective, we must show that for any element $u$ of the set $Fa$, there exists a natural transformation $\phi: y(a) \to F$ such that $\sigma(\phi) = u$.

Let $u$ be any element of $Fa$. Using this $u$, we define a map $\phi_x : y(a)(x) \to Fx$ for each object $x \in C$ as follows.

For any morphism $p: x \to a$,
$$ \phi_x(p) = F(p)(u) $$
Here, $F(p)$ is the map $F(p): Fa \to Fx$ obtained by the action of the functor $F$ on the morphism $p$.

Next, we confirm that this defined family of maps $\phi = \{\phi_x\}_{x \in C}$ is indeed a natural transformation from $y(a)$ to $F$. By the definition of a natural transformation, the following commutative diagram must hold for any morphism $q: x' \to x$.

\[
	% https://tikzcd.yichuanshen.de/#N4Igdg9gJgpgziAXAbVABwnAlgFyxMJZAJgBoBGAXVJADcBDAGwFcYkQBPACnoEouAHgHJeIAL6l0mXPkIoALBWp0mrdgDFh4ySAzY8BIooDMyhizaIQm7VP2yiZUzXNqr3PoNES7MwyjIABjNVS04ebx09PzlkRWCXUI1bXWkDWMClRIt2LR9U+39kTOcVHKsBcWUYKABzeCJQADMAJwgAWyRMkBwIJGMaRnoAIxhGAAU0hysWrFqACxwQbLdwzwBHSOa2zsRu3qRyFbCAHRO0eawAfWBhMRTWjv6aA8QyMtWzi+vb+-zH3ZHHp9N7HDRcTYPHZIRTApAAVjBVi+lyhT0QADYXiCAOxIkDrZYgIajCZTfwgWYLJZiShiIA
	\begin{tikzcd}
		&  & y(a) \arrow[rr, "\phi"]              &  & F                  \\
		x' \arrow[dd, "q"'] &  & y(a)(x') \arrow[dd, "y(a)(q)"'] \arrow[rr, "\phi_{x'}"] &  & Fx' \arrow[dd, "F(q)"] \\
		&  &                                  &  &                    \\
		x                  &  & y(a)(x) \arrow[rr, "\phi_{x}"]     &  & Fx
	\end{tikzcd}
\]

We check that this diagram commutes, i.e., $F(q) \circ \phi_{x'} = \phi_x \circ y(a)(q)$.

Left-hand side: $F(q) \circ \phi_{x'} = F(q)(\phi_{x'}(p'))$
Here, $p'$ is any element of $y(a)(x')$ (i.e., a morphism $p': x' \to a$).
By definition, $\phi_{x'}(p') = F(p')(u)$, so
$$ F(q)(F(p')(u)) = F(q \circ p')(u) \quad (\text{property of functor } F) $$

Right-hand side: $\phi_x \circ y(a)(q)$
$y(a)(q)$ is the map that sends the morphism $p'$ to $p' \circ q$.
Thus,
$$ \phi_x(y(a)(q)(p')) = \phi_x(p' \circ q) $$
By definition, $\phi_x(p' \circ q) = F(p' \circ q)(u)$.

Since both sides are equal, we have confirmed that $\phi$ is a natural transformation.

Finally, we show that this natural transformation $\phi$ satisfies $\sigma(\phi) = u$.
By the definition of $\sigma(\phi)$,
$$ \sigma(\phi) = \phi_a(\text{id}_a) $$
According to the definition of $\phi_a$, substituting $\text{id}_a$, we get
$$ \phi_a(\text{id}_a) = F(\text{id}_a)(u) = \text{id}_{Fa}(u) = u $$
Thus, $\sigma$ is surjective.

\ \\

Therefore, $\sigma$ is a bijection, and
Yoneda's Lemma
\[
	\text{Hom}_{C}(y(a),F) \cong Fa
\]
is proven.

\end{document}
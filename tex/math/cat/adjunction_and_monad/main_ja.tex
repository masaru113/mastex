\documentclass[uplatex,a4j,12pt,dvipdfmx]{jsarticle}
\usepackage{amsmath,amsthm,amssymb,bm,color,enumitem,mathrsfs,url,epic,eepic,ascmac,ulem,here}
\usepackage[letterpaper,top=2cm,bottom=2cm,left=3cm,right=3cm,marginparwidth=1.75cm]{geometry}
\usepackage[english]{babel}
\usepackage{graphicx}
\usepackage[hypertex]{hyperref}

\usepackage{tikz-cd}
\usetikzlibrary{cd}

\title{
モナドと随伴
}
\author{岡田 大(Okada Masaru)}

\date{\today}

\begin{document}

\maketitle

\tableofcontents

\ \\

\section{定義}

\subsection{随伴の定義}

圏 $C,D$ に対して、
関手 $F: C \to D$ 、$G: D \to C$
がある。
このとき「 $F$ と $G$ は随伴である」とは、
記号では $F \dashv G : C \to D$ と書き、
随伴関係
\[
	\mathrm{Hom}_{D}(Fc,d) \cong \mathrm{Hom}_{C}(c,Gd)
\]
が存在すること。
ここで
$c \in C, \ d \in D$
。

これらから自然変換
$\eta : 1_{C} \to GF$(単位)、
$\varepsilon : FG \to 1_{D}$(余単位)、
が構成できて、単位と余単位は以下の三角等式を満たす。

\begin{figure}[H]
	\centering
	\begin{tikzcd}[row sep=large, column sep=large]
		G \arrow[r, "\eta G"] \arrow[dr, "1_G"'] & GFG \arrow[d, "G\varepsilon"] & F \arrow[r, "F\eta"] \arrow[dr, "1_F"'] & FGF \arrow[d, "\varepsilon F"] \\
		& G & & F
	\end{tikzcd}
	% \caption{随伴の三角等式}
\end{figure}

数式で書けば以下の通りである。
$$
	G \varepsilon \circ \eta G = 1_{G}
$$
$$
	\varepsilon F \circ F \eta = 1_{F}
$$

\subsection{モナドの定義}

$C$ 上のモナド $(T, \eta, \mu)$ は以下で定義される。

\subsubsection{自己関手}

$$
	T = G \circ F = GF : C \to C
$$

\subsubsection{単位}

随伴の単位をそのまま使う:
$$
	\eta : 1_{C} \to GF = T
$$

\subsubsection{乗法}

乗法は随伴の余単位 $\varepsilon$ を $G$ と $F$ で挟んで構成する。

$$
	\mu = G \varepsilon F : GFGF \to GF
$$
すなわち、
$$
	\mu : T^{2} \to T
$$

\section{モナドになることの確認}

\subsection{ 結合律: $(T, \eta, \mu)$ がモナドになることの確認1}

$(T, \eta, \mu)$ がモナドであるためには、以下の図式が可換である必要がある。

\begin{center}
	\begin{tikzcd}[row sep=large, column sep=large]
		T^3 \arrow[r, "T\mu"] \arrow[d, "\mu T"'] & T^2 \arrow[d, "\mu"] \\
		T^2 \arrow[r, "\mu"'] & T
	\end{tikzcd}
\end{center}
すなわち、
$$
	\mu \circ T \mu = \mu \circ \mu T
$$

これを$G,F,\varepsilon$で書くと、
\[
	T \mu = GF(G \varepsilon F) = GFG \varepsilon F
\]
\[
	\mu T = (G \varepsilon F)GF = G \varepsilon FGF
\]
なので、示すべき式は
\[
	G \varepsilon F \circ GFG \varepsilon F = G \varepsilon F \circ G \varepsilon FGF
\]
すなわち、
\[
	G (\varepsilon \circ FG \varepsilon) F = G ( \varepsilon \circ \varepsilon F G)F
\]
である。ここで両端の$G,F$を外した核となる部分は、自然変換 $\varepsilon$ の $FG \xrightarrow{\varepsilon} 1_D$ に対する自然性(Naturality)の可換図式そのものである。

\begin{center}
	\begin{tikzcd}[column sep=huge, row sep=large]
		FGFG \arrow[r, "FG \varepsilon"] \arrow[d, "\varepsilon_{FG}"'] & FG \arrow[d, "\varepsilon"] \\
		FG \arrow[r, "\varepsilon"'] & 1_D
	\end{tikzcd}
\end{center}

この可換性 $\varepsilon \circ FG \varepsilon = \varepsilon \circ \varepsilon F G$ より、結合律は直ちに従う。

\subsection{ 単位律: $(T, \eta, \mu)$ がモナドになることの確認2}

$(T, \eta, \mu)$ がモナドであるためには、以下の図式が可換である必要がある。

\begin{center}
	\begin{tikzcd}[row sep=large, column sep=large]
		T \arrow[r, "\eta T"] \arrow[dr, "1_T"'] & T^2 \arrow[d, "\mu"] & T \arrow[l, "T \eta"'] \arrow[dl, "1_T"] \\
		& T &
	\end{tikzcd}
\end{center}

すなわち、
右単位律 $\mu \circ T \eta = 1_{T}$ と、
左単位律 $\mu \circ \eta T = 1_{T}$
を満たす必要がある。

\subsubsection{右単位律の確認}

定義より
\[
	\mu \circ T \eta = (G \varepsilon F) \circ (GF \eta) = G (\varepsilon F \circ F \eta)
\]
ここで随伴の三角等式 $\varepsilon F \circ F \eta = 1_{F}$ を用いると、
\[
	\mu \circ T \eta = G(1_{F}) = 1_{GF} = 1_{T}
\]
となり成立する。

\subsubsection{左単位律の確認}

同様に、
\[
	\mu \circ \eta T = (G \varepsilon F) \circ (\eta GF) = (G \varepsilon \circ \eta G)F
\]
ここでも随伴の三角等式 $G \varepsilon \circ \eta G = 1_{G}$ を用いると、
\[
	= (1_{G})F = 1_{GF} = 1_{T}
\]
となり成立する。

\section{結論}

以上から、
随伴 $F \dashv G$
から構成される
$C$ 上のトリプル $(T=GF, \eta, \mu=G \varepsilon F)$
は結合律と左右の単位律を満たすのでモナドになる。

\end{document}
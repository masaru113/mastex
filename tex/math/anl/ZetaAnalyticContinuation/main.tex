\documentclass[uplatex,a4j,12pt,dvipdfmx]{jsarticle}
\usepackage[english]{babel}
\usepackage[letterpaper,top=2cm,bottom=2cm,left=3cm,right=3cm,marginparwidth=1.75cm]{geometry}
\usepackage{amsmath,amsthm,amssymb,bm,color,mathrsfs}
\usepackage{epic,eepic,here}
\usepackage[dvipdfm]{graphicx}
\title{Values of the Riemann Zeta Function for Negative Arguments}
\author{Masaru Okada}
\date{\today}
\begin{document}
\allowdisplaybreaks
\maketitle

\begin{abstract}
	This memo considers the values of the Riemann zeta function $\zeta(s)$ at negative integers.
	The series definition $\zeta(s) = \sum_{n=1}^{\infty} n^{-s}$, valid for ${\rm Re}(s) > 1$, formally yields divergent series such as $1+2+3+\cdots$ when applied to negative integers like $s=-1$.
	It is first shown in this memo that the finite value $\zeta(-1) = -1/12$ is derived by analytic continuation using the functional equation.
	The discussion then turns to the Bernoulli numbers $B_n$, which underlie this value, detailing their definition, properties, and relation to the series expansions of functions like $\coth$ and $\cot$.
	Finally, an overview is provided of how the values of the zeta function at negative integers are generally linked to Bernoulli numbers through the relation $\zeta(-m) = -B_{m+1}/(m+1)$.
\end{abstract}

\tableofcontents

\section{Problem}

For a complex number $s$ with ${\rm Re}(s) > 1$, the
Riemann zeta function (hereafter simply 'the zeta function') is defined as follows:
\begin{eqnarray}
	\zeta(s)
	&=&
	1
	+
	\dfrac{1}{2^{s}}
	+
	\dfrac{1}{3^{s}}
	+
	\cdots
	+
	\dfrac{1}{n^{s}}
	+
	\cdots
	\ = \
	\sum_{n=1}^{\infty}
	\dfrac{1}{n^{s}}
\end{eqnarray}

First, it clearly converges for ${\rm Re}(s) > 1$.
At $s=1$, it becomes the harmonic series and diverges.
However, it can be analytically continued to ${\rm Re}(s) < 0$ using the following functional equation:
\begin{eqnarray}
	\zeta(s)
	&=&
	2^{s}
	\pi^{s-1}
	{\rm sin}
	\left(
	\dfrac{\pi s}{2}
	\right)
	\ \!
	\Gamma(1-s)
	\zeta(1-s)
	\label{eqn:1302112125}
\end{eqnarray}

\subsection{At $s=-1$}

So, what happens to $\zeta(s)$ when $s$ takes a negative value? For example, at $s=-1$,
\begin{eqnarray}
	\zeta(-1)
	&=&
	1
	+
	\dfrac{1}{2^{-1}}
	+
	\dfrac{1}{3^{-1}}
	+
	\cdots
	+
	\dfrac{1}{n^{-1}}
	+
	\cdots
	\nonumber \\[3mm] &=&
	1+2+3+\cdots+n+\cdots
\end{eqnarray}
it can be expressed as above. The series on the right-hand side, of course, appears to diverge.
On the other hand, using the functional equation (\ref{eqn:1302112125}) used for analytic continuation,
\begin{eqnarray}
	\zeta(-1)
	&=&
	2^{-1}
	\pi^{-2}
	{\rm sin}
	\left(
	\frac{- \pi}{2}
	\right)
	\ \!
	\Gamma(2)
	\ \!
	\zeta(2)
	\nonumber \\ &=&
	\frac{1}{2 \pi^{2}}
	\times
	(-1)
	\times
	(1!)
	\times
	\frac{\pi^{2}}{6}
	\nonumber \\[3mm] &=&
	- \frac{1}{12}
\end{eqnarray}
we obtain a finite value. These two equations seem to suggest a contradiction.

\section{Bernoulli number}

As preparation, we briefly summarize the Bernoulli numbers $B_{n}$.
$B_{n}$ are defined as the coefficients in the series expansion of the following function:
\begin{eqnarray}
	\dfrac{x}{e^{x}-1}
	&=&
	\sum_{n=0}^{\infty}
	B_{n}
	\dfrac{x^{n}}{n!}
\end{eqnarray}
The general term is also known and is given by the following:
\begin{eqnarray}
	B_{n}
	&=&
	\sum_{k=0}^{n}
	(-1)^{k}
	k^{n}
	\sum_{m=k}^{n}
	\dfrac{{_{m}}{\rm C}_{r}}{m+1}
\end{eqnarray}




Here, the binomial coefficient
\begin{eqnarray}
	{_{n}}{\rm C}_{k}
	&=&
	\dfrac{n!}{(n-k)!k!}
\end{eqnarray}
is denoted as above.
However, since this is a double summation, using this formula directly is computationally intensive.
Therefore, to actually compute $B_{n}$, the following recurrence relation is used:
\begin{eqnarray}
	\left\{
	\begin{array}{l}
		\\
		B_{0} = 1 \\
		B_{0} = -\dfrac{1}{n+1} \displaystyle \sum_{k=0}^{n-1} {_{n+1}}{\rm C}_{k} B_{k}
	\end{array}
	\right.
\end{eqnarray}

All $B_{n}$ are rational numbers. The first few terms are:
$B_{0}=1$, $B_{1}=-\dfrac{1}{2}$, and $B_{2}=\dfrac{1}{6}$.
However, as $n$ increases, the numerators and denominators of $B_{n}$ grow large, making them unsuitable for floating-point arithmetic.
For example,
$B_{24}=-\dfrac{236364091}{2730}$,
$B_{28}=-\dfrac{1869628555}{58}$.

It is also known that $B_{n}=0$ for all odd $n \geq 3$ (i.e., odd numbers other than $n=1$).
This can be proven as follows.
\begin{eqnarray}
	\dfrac{x}{e^{x}-1}
	-
	B_{1}
	\dfrac{x^{1}}{1!}
	&=&
	\dfrac{x}{e^{x}-1}
	+
	\dfrac{1}{2}x
	\nonumber \\ &=&
	\dfrac{2x + x(e^{x} -1)}{2(e^{x}-1)}
	\nonumber \\ &=&
	\dfrac{x}{2}
	\dfrac{(e^{x} +1) \times e^{-x/2}}{(e^{x}-1)\times e^{-x/2}}
	\nonumber \\ &=&
	\dfrac{x}{2}
	{\rm coth}\dfrac{x}{2}
\end{eqnarray}
The $\coth$ function is an odd function, and thus the expression above is an even function.
Consequently, when this expression is expanded as a series, only even-power terms remain.
In other words,
$B_{n}=0$ for all odd $n \geq 3$ (odd numbers other than $n=1$).

\subsection{Series expansion of coth}

Solving this in reverse,
\begin{eqnarray}
	{\rm coth}x
	&=&
	\dfrac{2}{2x}
	\left(
	-
	B_{1}
	\dfrac{(2x)^{1}}{1!}
	+
	\dfrac{2x}{e^{2x}-1}
	\right)
	\nonumber \\ &=&
	\dfrac{1}{x}
	\left(
	-B_{1}(2x)
	+
	\sum_{n=0}^{\infty}
	B_{n}
	\dfrac{(2x)^{n}}{n!}
	\right)
	\nonumber \\ &=&
	\dfrac{1}{x}
	\left(
	-B_{1}(2x)
	+
	B_{0}
	+
	(2x)
	B_{1}
	+
	\dfrac{1}{2!}
	(2x)^{2}
	B_{2}
	+
	\dfrac{1}{3!}
	(2x)^{3}
	B_{3}
	+
	\cdots
	+
	\dfrac{1}{n!}
	(2x)^{n}
	B_{n}
	+
	\cdots
	\right)
	\nonumber \\ &=&
	\dfrac{1}{x}
	\left(
	B_{0}
	+
	\dfrac{1}{2!}
	(2x)^{2}
	B_{2}
	+
	\dfrac{1}{3!}
	(2x)^{3}
	B_{3}
	+
	\cdots
	+
	\dfrac{1}{n!}
	(2x)^{n}
	B_{n}
	+
	\cdots
	\right)
	\label{eqn:1302100637nemui}
\end{eqnarray}
$B_{0}=1$. Furthermore, since $B_{n}=0$ for odd $n \geq 3$,
\begin{eqnarray}
	{\rm eqn.}(\ref{eqn:1302100637nemui})
	&=&
	\dfrac{1}{x}
	\left(
	1
	+
	\dfrac{1}{2!}
	(2x)^{2}
	B_{2}
	+
	\dfrac{1}{4!}
	(2x)^{4}
	B_{4}
	+
	\cdots
	+
	\dfrac{1}{(2n)!}
	(2x)^{2n}
	B_{2n}
	+
	\cdots
	\right)
	\nonumber \\ &=&
	\dfrac{1}{x}
	+
	\sum_{n=1}^{\infty}
	\dfrac{
		2^{2n}
		B_{2n}
	}{(2n)!}
	x^{2n-1}
\end{eqnarray}




\subsection{Series expansion of cot}
\begin{eqnarray}
	{\rm coth} \ \! x
	\ = \
	\dfrac{{\rm cosh} \ \! x}{{\rm sinh} \ \! x}
	\ = \
	\dfrac{{\rm cos}(ix)}{-i{\rm sin}(ix)}
	\ = \
	i {\rm cot} (ix)
\end{eqnarray}
Therefore, substituting $ix \to y$,
\begin{eqnarray}
	{\rm cot} \ \! y
	&=&
	\dfrac{1}{i} {\rm coth} \dfrac{y}{i}
	\ = \
	-i {\rm coth} (-iy)
	\nonumber \\ &=&
	i {\rm coth} (iy)
	\nonumber \\ &=&
	i
	\left\{
	\dfrac{1}{iy}
	+
	\left(
	\dfrac{1}{2!}
	2^{2} (iy)^{1}
	B_{2}
	+
	\dfrac{1}{4!}
	2^{4} (iy)^{3}
	B_{4}
	+
	\cdots
	+
	\dfrac{1}{(2n)!}
	2^{2n} (iy)^{2n-1}
	B_{2n}
	+
	\cdots
	\right)
	\right\}
	\nonumber \\ &=&
	\dfrac{1}{y}
	+
	\sum_{n=1}^{\infty}
	\dfrac{(-1)^{n} 2^{2n} B_{2n} }{(2n)!}
	y^{2n-1}
\end{eqnarray}

\subsection{Series expansion of tan}

Using the double-angle formula
\begin{eqnarray}
	{\rm tan} (2x)
	&=&
	\dfrac{ 2 {\rm tan} \ \! x }{ 1 - {\rm tan}^{2} \ \! x }
\end{eqnarray}
and taking the reciprocal of both sides,
\begin{eqnarray}
	{\rm cot} (2x)
	&=&
	\dfrac{ 1 - {\rm tan}^{2} \ \! x }{2 {\rm tan} \ \! x }
	\nonumber \\ &=&
	\dfrac{1}{2 {\rm tan} \ \! x }
	-
	\dfrac{{\rm tan} \ \! x}{2}
\end{eqnarray}
In other words,
\begin{eqnarray}
	{\rm tan} \ \! x
	&=&
	{ \rm cot } \ \! x
	-
	2 { \rm cot } (2x)
	\nonumber \\[2mm] &=&
	\dfrac{1}{x}
	+
	\sum_{n=1}^{\infty}
	\dfrac{(-1)^{n} 2^{2n} B_{2n} }{(2n)!}
	x^{2n-1}
	-
	2
	\left(
	\dfrac{1}{2x}
	+
	\sum_{n=1}^{\infty}
	\dfrac{(-1)^{n} 2^{2n} B_{2n} }{(2n)!}
		(2x)^{2n-1}
	\right)
	\nonumber \\ &=&
	\sum_{n=1}^{\infty}
	\dfrac{(-1)^{n} 2^{2n} B_{2n} }{(2n)!}
	\Big\{
	x^{2n-1}
	-
	2
	(2x)^{2n-1}
	\Big\}
	\nonumber \\ &=&
	\sum_{n=1}^{\infty}
	\dfrac{(-1)^{n} 2^{2n} ( 1 - 2^{2n} ) B_{2n} }{(2n)!}
	x^{2n-1}
\end{eqnarray}
we get.
The Bernoulli numbers $B_{n}$ are fundamentally involved in the series expansion coefficients of trigonometric functions.
For the $\sin$ and $\cos$ functions, however,
\begin{eqnarray}
	{\rm sin} \ \! x
	&=&
	\dfrac{e^{ix} - e^{-ix}}{2i}
	\\
	{\rm cos} \ \! x
	&=&
	\dfrac{e^{ix} + e^{-ix}}{2}
\end{eqnarray}
because the exponential function does not appear in the denominator and only appears in the numerator, $B_{n}$ do not appear.




\section{Some special cases}

The values of $\zeta(s)$ for general $s$ are computed numerically.
However, for special values of $s$, analytic values have been found, and the following are known.

For any positive even integer $2m$:
\begin{eqnarray}
	\zeta(2m)
	&=&
	\dfrac{(-1)^{2m} B_{2m} (2 \pi)^{2m} }{(2m)! 2}
\end{eqnarray}

For any negative integer $-m$:
\begin{eqnarray}
	\zeta(-m)
	&=&
	-
	\dfrac{B_{m+1}}{m+1}
\end{eqnarray}

No elementary expression is known for the value of $\zeta(s)$ when $s$ is an odd integer.
However, a series representation was given by Ramanujan.
For any odd integer $2m-1$:
\begin{eqnarray}
	\zeta(2m-1)
	&=&
	-
	2^{2m} \pi^{2m-1}
	\sum_{k=0}^{m}
	(-1)^{k+1}
	\dfrac{B_{2k}}{(2k)!2}
	\dfrac{B_{2m-2k}}{(2m-2k)!}
	-
	2
	\sum_{k=1}^{\infty}
	\dfrac{ k^{-2m+1} }{ e^{2 \pi k} - 1 }
\end{eqnarray}

${}$

${}$

${}$

\section{Analytic continuation}

In the Gamma function
\begin{eqnarray}
	\Gamma(s)
	&=&
	\int^{\infty}_{0}
	\!\! dt
	\ \! t^{s-1} e^{-t}
\end{eqnarray}
making the substitution $t=nx$ gives,
\begin{eqnarray}
	\Gamma(s)
	&=&
	\int^{\infty}_{0}
	\!\! ndx
	\ \! (nx)^{s-1} e^{-nx}
	\label{eqn:1302120225hayonena}
\end{eqnarray}
Furthermore, dividing both sides by $n^{s}$ and summing over all natural numbers $n$:
\begin{eqnarray}
	\sum_{n=1}^{\infty}
	\dfrac{ \Gamma(s) }{ n^{s} }
	&=&
	\sum_{n=1}^{\infty}
	\int^{\infty}_{0}
	\!\! dx
	\ \! x^{s-1} e^{-nx}
	\nonumber \\
	\Gamma(s)
	\sum_{n=1}^{\infty}
	\dfrac{ 1 }{ n^{s} }
	&=&
	\int^{\infty}_{0}
	\!\! dx
	\ \! x^{s-1}
	\sum_{n=1}^{\infty}
	e^{-nx}
\end{eqnarray}
Thus, the expression
\begin{eqnarray}
	\Gamma(s)
	\zeta(s)
	&=&
	\int^{\infty}_{0}
	\!\! dx
	\ \!
	\dfrac{x^{s-1}}{e^{x} - 1}
	\label{eqn:1302120248nerana}
\end{eqnarray}
is obtained.




By the way, using the following identity for alternating series
\begin{eqnarray}
	1
	-
	\dfrac{1}{2^{s}}
	+
	\dfrac{1}{3^{s}}
	-
	\cdots
	+
	\dfrac{(-1)^{n+1}}{n^{s}}
	+
	\cdots
	&=&
	\left\{
	1
	+
	\dfrac{1}{2^{s}}
	+
	\dfrac{1}{3^{s}}
	+
	\cdots
	+
	\dfrac{1}{n^{s}}
	+
	\cdots
	\right\}
	-
	2
	\left\{
	\dfrac{1}{2^{s}}
	+
	\dfrac{1}{4^{s}}
	+
	\cdots
	+
	\dfrac{1}{(2n)^{s}}
	+
	\cdots
	\right\}
	\nonumber \\
	\sum_{n=1}^{\infty}
	\dfrac{(-1)^{n+1}}{n^{s}}
	&=&
	\zeta(s)
	-
	2
	\sum_{n=1}^{\infty}
	\dfrac{1}{(2n)^{s}}
	\nonumber \\ &=&
	\zeta(s)
	-
	\dfrac{2}{2^{s}}
	\zeta(s)
	\nonumber \\ &=&
	\zeta(s)
	\left(
	1
	-
	\dfrac{1}{2^{s-1}}
	\right)
\end{eqnarray}
and equation (\ref{eqn:1302120225hayonena}), if we compute the right and left sides of the following independently,
\begin{eqnarray}
	\sum_{n=1}^{\infty}
	\Gamma(s)
	\dfrac{(-1)^{n+1}}{n^{s}}
	&=&
	\sum_{n=1}^{\infty}
	\left\{
	\int^{\infty}_{0}
	\!\! ndx
	\ \! (nx)^{s-1} e^{-nx}
	\right\}
	\dfrac{(-1)^{n+1}}{n^{s}}
	\nonumber \\
	\Gamma(s)
	\sum_{n=1}^{\infty}
	\dfrac{(-1)^{n+1}}{n^{s}}
	&=&
	\int^{\infty}_{0}
	\!\! dx
	\ \! x^{s-1}
	\sum_{n=1}^{\infty}
	(-1)^{n+1}
	e^{-nx}
	\nonumber \\
	\Gamma(s)
	\zeta(s)
	\left(
	1
	-
	\dfrac{1}{2^{s-1}}
	\right)
	&=&
	\int^{\infty}_{0}
	\!\! dx
	\ \! x^{s-1}
	\frac{1}{e^{x} + 1}
\end{eqnarray}
we newly obtain the above.

${}$

From equation (\ref{eqn:1302120225hayonena}),
\begin{eqnarray}
	\dfrac{1}{n^{s}}
	&=&
	\dfrac{1}{\Gamma(s)}
	\int^{\infty}_{0}
	dt \ \! t^{s-1} e^{-nt}
\end{eqnarray}
Using this fact,
\begin{eqnarray}
	\pi^{-s/2}
	\zeta(s)
	&=&
	\sum_{n=1}^{\infty}
	\dfrac{1}{(\pi n )^{s/2}}
	\nonumber \\ &=&
	\sum_{n=1}^{\infty}
	\dfrac{1}{\Gamma(s/2)}
	\int^{\infty}_{0}
	dt \ \! t^{s/2 \ \! -1} e^{-(\pi n^{2})t}
	\nonumber \\ &=&
	\dfrac{1}{\Gamma(s/2)}
	\int^{\infty}_{0}
	dt \ \! t^{s/2 \ \! -1}
	\sum_{n=1}^{\infty}
	e^{- \pi n^{2} t}
\end{eqnarray}

Here, we define $\psi(x)$ as a special case of the Jacobi theta function:
\begin{eqnarray}
	\psi(x)
	&=&
	\sum_{n=1}^{\infty}
	e^{- \pi n^{2} x}
\end{eqnarray}
From the Poisson summation formula,
$\psi(x)$ satisfies the following:
\begin{eqnarray}
	2 \psi ( 1/x )
	+
	1
	&=&
	\sqrt{x}
	\Big( 2 \psi(x) + 1 \Big)
\end{eqnarray}




\section{Conclusion}

In this memo, we have considered the Riemann zeta function $\zeta(s)$, focusing particularly on its values at negative integers.

Applying the definition $\zeta(s) = \sum_{n=1}^{\infty} n^{-s}$ (valid for ${\rm Re}(s) > 1$) formally to $s=-1$ yields
\begin{eqnarray}
	\zeta(-1)
	&=&
	1+2+3+\cdots
\end{eqnarray}
an obviously divergent series.

However, it was shown that $\zeta(-1)$ has the finite value $-1/12$ through analytic continuation using the functional equation (\ref{eqn:1302112125}).

This value can also be derived from the relationship between Bernoulli numbers $B_n$ and the values of the zeta function at negative integers.

${}$

In general, for any negative integer $-m$ ($m \ge 0$), the value of $\zeta(-m)$ is expressed using Bernoulli numbers as follows:
\begin{eqnarray}
	\zeta(-m)
	&=&
	-
	\dfrac{B_{m+1}}{m+1}
\end{eqnarray}
Using the values of the Bernoulli numbers calculated in the main text, $B_1 = -1/2$, $B_2 = 1/6$, and $B_n = 0$ for odd $n \ge 3$, we can confirm the following specific values.

\begin{itemize}
	\item Case $m=0$ ($s=0$):
	      \begin{eqnarray}
		      \zeta(0)
		      &=&
		      -
		      \dfrac{B_{1}}{1}
		      =
		      - \left( - \frac{1}{2} \right)
		      =
		      \frac{1}{2}
	      \end{eqnarray}

	\item Case $m=1$ ($s=-1$):
	      \begin{eqnarray}
		      \zeta(-1)
		      &=&
		      -
		      \dfrac{B_{2}}{2}
		      =
		      - \dfrac{1/6}{2}
		      =
		      - \frac{1}{12}
	      \end{eqnarray}
	      This matches the value derived from the functional equation and corresponds to the famous result $1+2+3+\cdots = -1/12$.

	\item Case $m=2k$ ($k \ge 1$, negative even integer $s=-2k$):
	      \begin{eqnarray}
		      \zeta(-2k)
		      &=&
		      -
		      \dfrac{B_{2k+1}}{2k+1}
		      =
		      0
	      \end{eqnarray}
	      This is derived from the fact that $B_n=0$ (for odd $n \ge 3$) and these are known as the trivial zeros of the zeta function.
\end{itemize}

Thus, the concept of analytic continuation and the properties of Bernoulli numbers provide the theoretical background for assigning finite values to series that appear divergent at first glance.

Attempts to assign finite values to divergent series are studied not only through analytic continuation but also through various summation methods, starting with Borel summation.

The value $\zeta(-1)$ treated in this memo is obtained in a different context (analytic continuation) than these summation methods, but it suggests the importance of a framework for handling divergent series with mathematical rigor.

\ \\

Furthermore, the handling of such divergent series is deeply related to the field of physics.

For example, asymptotic expansions that appear in the WKB approximation or perturbation theory in quantum mechanics often become divergent series.

By interpreting or processing these series appropriately, such as by optimal truncation, they can provide physically meaningful approximate values and insights.

In particular, the value $\zeta(-1)=-1/12$ is actually used in calculations of the Casimir effect in quantum field theory as a method of regularizing divergences (zeta function regularization), serving as an interesting example of a mathematical concept connecting to physical reality.
\end{document}
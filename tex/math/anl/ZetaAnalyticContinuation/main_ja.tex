\documentclass[uplatex,a4j,12pt,dvipdfmx]{jsarticle}
\usepackage[english]{babel}
\usepackage[letterpaper,top=2cm,bottom=2cm,left=3cm,right=3cm,marginparwidth=1.75cm]{geometry}
\usepackage{amsmath,amsthm,amssymb,bm,color,mathrsfs}
\usepackage{epic,eepic,here}
\usepackage[dvipdfm]{graphicx}
\title{負数の変数を持つRiemannのゼータ関数の値}
\author{岡田 大 (Okada Masaru)}
\date{\today}
\begin{document}
\allowdisplaybreaks
\maketitle

\begin{abstract}
	このメモでは、Riemannのゼータ関数 $\zeta(s)$ の負の整数における値について考察する。
	${\rm Re}(s) > 1$ における級数定義 $\zeta(s) = \sum_{n=1}^{\infty} n^{-s}$ は、$s=-1$ のような負の整数に対して $1+2+3+\cdots$ といった発散級数を形式的にもたらす。
	このメモでは、まず関数等式を用いた解析接続によって、$\zeta(-1) = -1/12$ という有限値が導出されることを示す。
	次に、この値の背景にあるベルヌーイ数 $B_n$ について、その定義、性質、および $\coth$ や $\cot$ などの級数展開との関連を述べる。
	最終的に、ゼータ関数の負の整数における値が、$\zeta(-m) = -B_{m+1}/(m+1)$ という関係式によってベルヌーイ数と一般的に結びついていることを概観する。
\end{abstract}

\tableofcontents

\section{Problem}

複素数$s$に対し、${\rm Re}(s) > 1$で定義される
Riemann zeta function(以降、簡単にゼータ関数と呼ぶ。)は次で定義される。
\begin{eqnarray}
	\zeta(s)
	&=&
	1
	+
	\dfrac{1}{2^{s}}
	+
	\dfrac{1}{3^{s}}
	+
	\cdots
	+
	\dfrac{1}{n^{s}}
	+
	\cdots
	\ = \
	\sum_{n=1}^{\infty}
	\dfrac{1}{n^{s}}
\end{eqnarray}

まず、明らかに${\rm Re}(s) >1$では収束する。
$s=1$では調和級数になり、発散する。
しかし${\rm Re}(s) < 0$に対しても次の関数等式を用いることで解析接続することができる。
\begin{eqnarray}
	\zeta(s)
	&=&
	2^{s}
	\pi^{s-1}
	{\rm sin}
	\left(
	\dfrac{\pi s}{2}
	\right)
	\ \!
	\Gamma(1-s)
	\zeta(1-s)
	\label{eqn:1302112125}
\end{eqnarray}

\subsection{At $s=-1$}

では、$s$が負の値を持つときの$\zeta(s)$はどうなるであろうか、例えば$s=-1$のとき、
\begin{eqnarray}
	\zeta(-1)
	&=&
	1
	+
	\dfrac{1}{2^{-1}}
	+
	\dfrac{1}{3^{-1}}
	+
	\cdots
	+
	\dfrac{1}{n^{-1}}
	+
	\cdots
	\nonumber \\[3mm] &=&
	1+2+3+\cdots+n+\cdots
\end{eqnarray}
と表現できる。最後の辺の級数は当然、発散するように見える。
一方、解析接続に用いられた関数等式(\ref{eqn:1302112125})を用いると、
\begin{eqnarray}
	\zeta(-1)
	&=&
	2^{-1}
	\pi^{-2}
	{\rm sin}
	\left(
	\frac{- \pi}{2}
	\right)
	\ \!
	\Gamma(2)
	\ \!
	\zeta(2)
	\nonumber \\ &=&
	\frac{1}{2 \pi^{2}}
	\times
	(-1)
	\times
	(1!)
	\times
	\frac{\pi^{2}}{6}
	\nonumber \\[3mm] &=&
	- \frac{1}{12}
\end{eqnarray}
となり、有限の値に残る。それぞれの等式は、まるで矛盾を示唆するように見える。

\section{Bernoulli number}

準備としてベルヌーイ数$B_{n}$について簡単にまとめておく。
$B_{n}$は次の関数の級数展開の係数として定義される。
\begin{eqnarray}
	\dfrac{x}{e^{x}-1}
	&=&
	\sum_{n=0}^{\infty}
	B_{n}
	\dfrac{x^{n}}{n!}
\end{eqnarray}
一般項も知られていて、次で与えられる。
\begin{eqnarray}
	B_{n}
	&=&
	\sum_{k=0}^{n}
	(-1)^{k}
	k^{n}
	\sum_{m=k}^{n}
	\dfrac{{_{m}}{\rm C}_{r}}{m+1}
\end{eqnarray}




ここで、二項係数
\begin{eqnarray}
	{_{n}}{\rm C}_{k}
	&=&
	\dfrac{n!}{(n-k)!k!}
\end{eqnarray}
と表記した。
しかし二重級数になっているので直接この公式を用いると計算が重くなる。
そこで実際に$B_{n}$を求めるには次の漸化式が用いられる。
\begin{eqnarray}
	\left\{
	\begin{array}{l}
		\\
		B_{0} = 1 \\
		B_{0} = -\dfrac{1}{n+1} \displaystyle \sum_{k=0}^{n-1} {_{n+1}}{\rm C}_{k} B_{k}
	\end{array}
	\right.
\end{eqnarray}

$B_{n}$は全て有理数である。最初の数項を示すと、
$B_{0}=1$、$B_{1}=-\dfrac{1}{2}$、$B_{2}=\dfrac{1}{6}$となる。
しかし、$n$が大きくなると$B_{n}$の分子分母が大きくなっていくので、浮動小数点演算には不向きである。
例えば、
$B_{24}=-\dfrac{236364091}{2730}$、
$B_{28}=-\dfrac{1869628555}{58}$。

また、$n \geq 3$以降の奇数($n=1$以外の奇数)に対して$B_{n}=0$であることが知られている。
これは次のように証明できる。
\begin{eqnarray}
	\dfrac{x}{e^{x}-1}
	-
	B_{1}
	\dfrac{x^{1}}{1!}
	&=&
	\dfrac{x}{e^{x}-1}
	+
	\dfrac{1}{2}x
	\nonumber \\ &=&
	\dfrac{2x + x(e^{x} -1)}{2(e^{x}-1)}
	\nonumber \\ &=&
	\dfrac{x}{2}
	\dfrac{(e^{x} +1) \times e^{-x/2}}{(e^{x}-1)\times e^{-x/2}}
	\nonumber \\ &=&
	\dfrac{x}{2}
	{\rm coth}\dfrac{x}{2}
\end{eqnarray}
{\rm coth}関数は奇関数であるので、上の式は偶関数になる。
従って、上の式を級数展開したときに残る項は偶数次だけである。
すなわち、
$n \geq 3$以降の奇数($n=1$以外の奇数)に対して$B_{n}=0$である。

\subsection{cothの級数展開}

逆に解くと、
\begin{eqnarray}
	{\rm coth}x
	&=&
	\dfrac{2}{2x}
	\left(
	-
	B_{1}
	\dfrac{(2x)^{1}}{1!}
	+
	\dfrac{2x}{e^{2x}-1}
	\right)
	\nonumber \\ &=&
	\dfrac{1}{x}
	\left(
	-B_{1}(2x)
	+
	\sum_{n=0}^{\infty}
	B_{n}
	\dfrac{(2x)^{n}}{n!}
	\right)
	\nonumber \\ &=&
	\dfrac{1}{x}
	\left(
	-B_{1}(2x)
	+
	B_{0}
	+
	(2x)
	B_{1}
	+
	\dfrac{1}{2!}
	(2x)^{2}
	B_{2}
	+
	\dfrac{1}{3!}
	(2x)^{3}
	B_{3}
	+
	\cdots
	+
	\dfrac{1}{n!}
	(2x)^{n}
	B_{n}
	+
	\cdots
	\right)
	\nonumber \\ &=&
	\dfrac{1}{x}
	\left(
	B_{0}
	+
	\dfrac{1}{2!}
	(2x)^{2}
	B_{2}
	+
	\dfrac{1}{3!}
	(2x)^{3}
	B_{3}
	+
	\cdots
	+
	\dfrac{1}{n!}
	(2x)^{n}
	B_{n}
	+
	\cdots
	\right)
	\label{eqn:1302100637nemui}
\end{eqnarray}
$B_{0}=1$である。さらに、$n \geq 3$以降の奇数$n$に対して$B_{n}=0$であるので、
\begin{eqnarray}
	{\rm eqn.}(\ref{eqn:1302100637nemui})
	&=&
	\dfrac{1}{x}
	\left(
	1
	+
	\dfrac{1}{2!}
	(2x)^{2}
	B_{2}
	+
	\dfrac{1}{4!}
	(2x)^{4}
	B_{4}
	+
	\cdots
	+
	\dfrac{1}{(2n)!}
	(2x)^{2n}
	B_{2n}
	+
	\cdots
	\right)
	\nonumber \\ &=&
	\dfrac{1}{x}
	+
	\sum_{n=1}^{\infty}
	\dfrac{
		2^{2n}
		B_{2n}
	}{(2n)!}
	x^{2n-1}
\end{eqnarray}




\subsection{cotの級数展開}
\begin{eqnarray}
	{\rm coth} \ \! x
	\ = \
	\dfrac{{\rm cosh} \ \! x}{{\rm sinh} \ \! x}
	\ = \
	\dfrac{{\rm cos}(ix)}{-i{\rm sin}(ix)}
	\ = \
	i {\rm cot} (ix)
\end{eqnarray}
なので、$ix \to y$と置換して、
\begin{eqnarray}
	{\rm cot} \ \! y
	&=&
	\dfrac{1}{i} {\rm coth} \dfrac{y}{i}
	\ = \
	-i {\rm coth} (-iy)
	\nonumber \\ &=&
	i {\rm coth} (iy)
	\nonumber \\ &=&
	i
	\left\{
	\dfrac{1}{iy}
	+
	\left(
	\dfrac{1}{2!}
	2^{2} (iy)^{1}
	B_{2}
	+
	\dfrac{1}{4!}
	2^{4} (iy)^{3}
	B_{4}
	+
	\cdots
	+
	\dfrac{1}{(2n)!}
	2^{2n} (iy)^{2n-1}
	B_{2n}
	+
	\cdots
	\right)
	\right\}
	\nonumber \\ &=&
	\dfrac{1}{y}
	+
	\sum_{n=1}^{\infty}
	\dfrac{(-1)^{n} 2^{2n} B_{2n} }{(2n)!}
	y^{2n-1}
\end{eqnarray}

\subsection{tanの級数展開}

倍角公式
\begin{eqnarray}
	{\rm tan} (2x)
	&=&
	\dfrac{ 2 {\rm tan} \ \! x }{ 1 - {\rm tan}^{2} \ \! x }
\end{eqnarray}
の両辺で逆数を取って、
\begin{eqnarray}
	{\rm cot} (2x)
	&=&
	\dfrac{ 1 - {\rm tan}^{2} \ \! x }{2 {\rm tan} \ \! x }
	\nonumber \\ &=&
	\dfrac{1}{2 {\rm tan} \ \! x }
	-
	\dfrac{{\rm tan} \ \! x}{2}
\end{eqnarray}
すなわち、
\begin{eqnarray}
	{\rm tan} \ \! x
	&=&
	{ \rm cot } \ \! x
	-
	2 { \rm cot } (2x)
	\nonumber \\[2mm] &=&
	\dfrac{1}{x}
	+
	\sum_{n=1}^{\infty}
	\dfrac{(-1)^{n} 2^{2n} B_{2n} }{(2n)!}
	x^{2n-1}
	-
	2
	\left(
	\dfrac{1}{2x}
	+
	\sum_{n=1}^{\infty}
	\dfrac{(-1)^{n} 2^{2n} B_{2n} }{(2n)!}
		(2x)^{2n-1}
	\right)
	\nonumber \\ &=&
	\sum_{n=1}^{\infty}
	\dfrac{(-1)^{n} 2^{2n} B_{2n} }{(2n)!}
	\Big\{
	x^{2n-1}
	-
	2
	(2x)^{2n-1}
	\Big\}
	\nonumber \\ &=&
	\sum_{n=1}^{\infty}
	\dfrac{(-1)^{n} 2^{2n} ( 1 - 2^{2n} ) B_{2n} }{(2n)!}
	x^{2n-1}
\end{eqnarray}
となる。
三角関数の級数展開係数には基本的に$B_{n}$が関わってくる。
sin、cos関数に限っては
\begin{eqnarray}
	{\rm sin} \ \! x
	&=&
	\dfrac{e^{ix} - e^{-ix}}{2i}
	\\
	{\rm cos} \ \! x
	&=&
	\dfrac{e^{ix} + e^{-ix}}{2}
\end{eqnarray}
指数関数が分母に入らず、分子にしか指数関数が入っていないので$B_{n}$が現れない。




\section{Some special cases}

一般のsで$\zeta(s)$がどのような値を取るかは数値的に計算される。
しかし、特別な$s$に関しては解析的な値が求められており、次のようになることが知られている。

任意の正の偶数$2m$に対して、
\begin{eqnarray}
	\zeta(2m)
	&=&
	\dfrac{(-1)^{2m} B_{2m} (2 \pi)^{2m} }{(2m)! 2}
\end{eqnarray}

任意の負の整数$-m$に対して、
\begin{eqnarray}
	\zeta(-m)
	&=&
	-
	\dfrac{B_{m+1}}{m+1}
\end{eqnarray}

$s$が奇数の場合の$\zeta(s)$の値の初等的な表示は知られていない。
しかし級数表示はラマヌジャンによって与えられている。
任意の奇数$2m-1$に対して、
\begin{eqnarray}
	\zeta(2m-1)
	&=&
	-
	2^{2m} \pi^{2m-1}
	\sum_{k=0}^{m}
	(-1)^{k+1}
	\dfrac{B_{2k}}{(2k)!2}
	\dfrac{B_{2m-2k}}{(2m-2k)!}
	-
	2
	\sum_{k=1}^{\infty}
	\dfrac{ k^{-2m+1} }{ e^{2 \pi k} - 1 }
\end{eqnarray}

${}$

${}$

${}$

\section{Analytic continuation}

ガンマ関数
\begin{eqnarray}
	\Gamma(s)
	&=&
	\int^{\infty}_{0}
	\!\! dt
	\ \! t^{s-1} e^{-t}
\end{eqnarray}
において、$t=nx$として変数変換すると、
\begin{eqnarray}
	\Gamma(s)
	&=&
	\int^{\infty}_{0}
	\!\! ndx
	\ \! (nx)^{s-1} e^{-nx}
	\label{eqn:1302120225hayonena}
\end{eqnarray}
さらに、両辺を$n^{s}$で割って、全ての自然数$n$に関して和を取る。
\begin{eqnarray}
	\sum_{n=1}^{\infty}
	\dfrac{ \Gamma(s) }{ n^{s} }
	&=&
	\sum_{n=1}^{\infty}
	\int^{\infty}_{0}
	\!\! dx
	\ \! x^{s-1} e^{-nx}
	\nonumber \\
	\Gamma(s)
	\sum_{n=1}^{\infty}
	\dfrac{ 1 }{ n^{s} }
	&=&
	\int^{\infty}_{0}
	\!\! dx
	\ \! x^{s-1}
	\sum_{n=1}^{\infty}
	e^{-nx}
\end{eqnarray}
よって、
\begin{eqnarray}
	\Gamma(s)
	\zeta(s)
	&=&
	\int^{\infty}_{0}
	\!\! dx
	\ \!
	\dfrac{x^{s-1}}{e^{x} - 1}
	\label{eqn:1302120248nerana}
\end{eqnarray}
の表式が得られた。




ところで、次の交代級数に関する恒等式
\begin{eqnarray}
	1
	-
	\dfrac{1}{2^{s}}
	+
	\dfrac{1}{3^{s}}
	-
	\cdots
	+
	\dfrac{(-1)^{n+1}}{n^{s}}
	+
	\cdots
	&=&
	\left\{
	1
	+
	\dfrac{1}{2^{s}}
	+
	\dfrac{1}{3^{s}}
	+
	\cdots
	+
	\dfrac{1}{n^{s}}
	+
	\cdots
	\right\}
	-
	2
	\left\{
	\dfrac{1}{2^{s}}
	+
	\dfrac{1}{4^{s}}
	+
	\cdots
	+
	\dfrac{1}{(2n)^{s}}
	+
	\cdots
	\right\}
	\nonumber \\
	\sum_{n=1}^{\infty}
	\dfrac{(-1)^{n+1}}{n^{s}}
	&=&
	\zeta(s)
	-
	2
	\sum_{n=1}^{\infty}
	\dfrac{1}{(2n)^{s}}
	\nonumber \\ &=&
	\zeta(s)
	-
	\dfrac{2}{2^{s}}
	\zeta(s)
	\nonumber \\ &=&
	\zeta(s)
	\left(
	1
	-
	\dfrac{1}{2^{s-1}}
	\right)
\end{eqnarray}
と式(\ref{eqn:1302120225hayonena})を用いて、次の右辺と左辺をそれぞれ独立に計算すると、
\begin{eqnarray}
	\sum_{n=1}^{\infty}
	\Gamma(s)
	\dfrac{(-1)^{n+1}}{n^{s}}
	&=&
	\sum_{n=1}^{\infty}
	\left\{
	\int^{\infty}_{0}
	\!\! ndx
	\ \! (nx)^{s-1} e^{-nx}
	\right\}
	\dfrac{(-1)^{n+1}}{n^{s}}
	\nonumber \\
	\Gamma(s)
	\sum_{n=1}^{\infty}
	\dfrac{(-1)^{n+1}}{n^{s}}
	&=&
	\int^{\infty}_{0}
	\!\! dx
	\ \! x^{s-1}
	\sum_{n=1}^{\infty}
	(-1)^{n+1}
	e^{-nx}
	\nonumber \\
	\Gamma(s)
	\zeta(s)
	\left(
	1
	-
	\dfrac{1}{2^{s-1}}
	\right)
	&=&
	\int^{\infty}_{0}
	\!\! dx
	\ \! x^{s-1}
	\frac{1}{e^{x} + 1}
\end{eqnarray}
が新たに得られる。

${}$

式(\ref{eqn:1302120225hayonena})より、
\begin{eqnarray}
	\dfrac{1}{n^{s}}
	&=&
	\dfrac{1}{\Gamma(s)}
	\int^{\infty}_{0}
	dt \ \! t^{s-1} e^{-nt}
\end{eqnarray}
である。このことを用いて、
\begin{eqnarray}
	\pi^{-s/2}
	\zeta(s)
	&=&
	\sum_{n=1}^{\infty}
	\dfrac{1}{(\pi n )^{s/2}}
	\nonumber \\ &=&
	\sum_{n=1}^{\infty}
	\dfrac{1}{\Gamma(s/2)}
	\int^{\infty}_{0}
	dt \ \! t^{s/2 \ \! -1} e^{-(\pi n^{2})t}
	\nonumber \\ &=&
	\dfrac{1}{\Gamma(s/2)}
	\int^{\infty}_{0}
	dt \ \! t^{s/2 \ \! -1}
	\sum_{n=1}^{\infty}
	e^{- \pi n^{2} t}
\end{eqnarray}

ここでヤコビのテータ関数の特殊な場合として$\psi(x)$を
\begin{eqnarray}
	\psi(x)
	&=&
	\sum_{n=1}^{\infty}
	e^{- \pi n^{2} x}
\end{eqnarray}
で定義する。
Poissonの和の公式より
$\psi(x)$は次を満たす。
\begin{eqnarray}
	2 \psi ( 1/x )
	+
	1
	&=&
	\sqrt{x}
	\Big( 2 \psi(x) + 1 \Big)
\end{eqnarray}




\section{まとめ}

このメモでは、Riemannのゼータ関数$\zeta(s)$について、特に負の整数における値について考察を行った。

${\rm Re}(s) > 1$における定義 $\zeta(s) = \sum_{n=1}^{\infty} n^{-s}$ を形式的に$s=-1$に適用すると、
\begin{eqnarray}
	\zeta(-1)
	&=&
	1+2+3+\cdots
\end{eqnarray}
となり、明らかに発散する級数が現れる。

しかし、関数等式(\ref{eqn:1302112125})を用いた解析接続により、$\zeta(-1)$は有限値 $-1/12$ を持つことが示された。

この値は、ベルヌーイ数$B_n$とゼータ関数の負の整数における値の関係式からも導出できる。

${}$

一般に、任意の負の整数$-m$($m \ge 0$)に対して、$\zeta(-m)$の値は次のようにベルヌーイ数を用いて表される。
\begin{eqnarray}
	\zeta(-m)
	&=&
	-
	\dfrac{B_{m+1}}{m+1}
\end{eqnarray}
本文中で計算されたベルヌーイ数の値、$B_1 = -1/2$、$B_2 = 1/6$、および$n \ge 3$の奇数$n$に対する$B_n = 0$を用いると、以下のように具体的な値が確認できる。

\begin{itemize}
	\item $m=0$ ($s=0$) の場合:
	      \begin{eqnarray}
		      \zeta(0)
		      &=&
		      -
		      \dfrac{B_{1}}{1}
		      =
		      - \left( - \frac{1}{2} \right)
		      =
		      \frac{1}{2}
	      \end{eqnarray}

	\item $m=1$ ($s=-1$) の場合:
	      \begin{eqnarray}
		      \zeta(-1)
		      &=&
		      -
		      \dfrac{B_{2}}{2}
		      =
		      - \dfrac{1/6}{2}
		      =
		      - \frac{1}{12}
	      \end{eqnarray}
	      これは、関数等式から導いた値と一致し、$1+2+3+\cdots = -1/12$ という有名な結果に対応する。

	\item $m=2k$ ($k \ge 1$, 負の偶数 $s=-2k$) の場合:
	      \begin{eqnarray}
		      \zeta(-2k)
		      &=&
		      -
		      \dfrac{B_{2k+1}}{2k+1}
		      =
		      0
	      \end{eqnarray}
	      これは、$B_n=0$ ($n \ge 3$, $n$は奇数) であることから導かれ、ゼータ関数の自明な零点 (trivial zeros) と呼ばれるものである。
\end{itemize}

このように、解析接続という概念とベルヌーイ数の性質が、一見発散するように見える級数に有限値を対応させる理論的背景を与えている。

発散級数に有限値を割り当てる試みは、解析接続以外にも、Borel総和法(ボレル総和法)をはじめとする様々な総和法において研究されている。

このメモで扱った $\zeta(-1)$ の値は、これらの総和法とは異なる文脈(解析接続)で得られるものであるが、発散級数を数学的に厳密に扱う枠組みの重要性を示唆している。

\ \\

また、このような発散級数の扱いは、物理学の分野でも深く関連している。

例えば、量子力学におけるWKB近似や摂動論において現れる漸近展開は、しばしば発散級数となる。

これらの級数は、最適に打ち切るなど、適切に解釈または処理することで、物理的に意味のある近似値や洞察を与える。

特に、$\zeta(-1)=-1/12$ という値は、場の量子論におけるカシミール効果の計算などで、発散を正則化する手法(ゼータ関数正規化)として実際に用いられており、数学的な概念が物理的な現実と結びつく興味深い例となっている。
\end{document}
\documentclass[uplatex,a4j,12pt,dvipdfmx]{jsarticle}
\usepackage{amsmath,amsthm,amssymb,bm,color,enumitem,mathrsfs,url,epic,eepic,ascmac,ulem,here,ascmac}
\usepackage[letterpaper,top=2cm,bottom=2cm,left=3cm,right=3cm,marginparwidth=1.75cm]{geometry}
\usepackage[english]{babel}
\usepackage[dvipdfm]{graphicx}
\usepackage[hypertex]{hyperref}
\title{ルンゲ=クッタ法}

\author{Masaru Okada}

\date{\today}

\begin{document}

\maketitle

\section{オイラー法}

まず、常微分方程式 (ODE) を数値的に解くための
最も単純な方法を紹介する。
これは(1次の)ルンゲ=クッタ法とも呼ばれる。
対象とする方程式は、以下のように仮定する。
\begin{eqnarray}
	\dfrac{dy}{dx} &=& f(x,y)
\end{eqnarray}
ここで $f(x,y)$ は既知の関数である。
記述を簡単にするため、
$y_{i}=y(x_{i})$
および
$x_{i+1} = x_{i} + h$
とおく。
初期条件 ($i=0$) は $(x_{0},y_{0})$ で与えられるものとする。
$x$ のメッシュ幅 $h$ を $| h | \ll 1$ となるように選べば、
$y_{i+1}$ は $h$ についてテイラー展開できる。
\begin{eqnarray}
	y_{i+1} &=&
	y(x_{i} + h)
	\nonumber \\[2mm] &=&
	y(x_{i}) \ \
	+
	\dfrac{dy}{dx} \Big|_{x=x_{i}} \hspace{-4mm} h \ \
	+
	O[h^{2}]
	\nonumber \\[2mm] &=&
	y_{i} \ \
	+
	f(x_{i},y_{i}) h
	+
	O[h^{2}]
\end{eqnarray}
したがって、次式が得られる。
\begin{eqnarray}
	f(x_{i},y_{i}) &=& \dfrac{y_{i+1} - y_{i}}{h}
\end{eqnarray}
これは $h$ の1次のオーダーまで正しい。
解は以下の漸化式によって与えられる。
\begin{eqnarray}
	\left\{
	\begin{array}{lcl}
		x_{i+1}    & = & x_{i} + h                                             \\[-1mm]
		           &   & \hspace{35mm} ( x_{0} {\rm \ \ \! is \ \ \! given.} ) \\[-1mm]
		y(x_{i+1}) & = & y(x_{i}) + f[x_{i},y(x_{i})]
	\end{array}
	\right.
\end{eqnarray}

\section{2次のルンゲ=クッタ法(ホイン法)}

同様にして、
$y_{i+1}$ を $h$ について展開する。
2次のオーダーまで書き下すと、以下のようになる。
\begin{eqnarray}
	y_{i+1} &=&
	y(x_{i} + h)
	\nonumber \\[2mm] &=&
	y(x_{i}) \ \
	+
	y'(x_{i}) h
	+
	\dfrac{1}{2} y''(x_{i}) h^{2}
	+
	\dfrac{1}{6} \dfrac{d^{3}y}{dx^{3}} \Big|_{x=x_{i}} \hspace{-4mm} h^{3} \ \
	+
	O[h^{4}]
	O[h^{3}]
	\\[4mm]
	\Delta y
	&=&
	y_{i+1} - y_{i}
	\nonumber \\[2mm] &=&
	y_{1} - y_{0}
	\nonumber \\[2mm] &=&
	y'(x_{0}) h
	+
	\dfrac{1}{2} y''(x_{0}) h^{2}
	\label{eqn:dokuritu1}
\end{eqnarray}
先ほどは $y'(x_{0})$ の値を用いたが、
今回はそれに加えて $y'(x_{1})$ という別の値も選んでみる。
\begin{eqnarray}
	\Delta y
	&=&
	h [ \alpha y'(x_{0}) + \beta y'(x_{1}) ]
\end{eqnarray}
ただし $\alpha$ と $\beta$ は未定の定数である。
ここで $\beta y'(x_{1})$ も同様に展開できる。
\begin{eqnarray}
	\beta y'(x_{1})
	&=&
	\beta y'(x_{0}+h)
	\nonumber \\[2mm] &=&
	\beta [ y'(x_{0}) + h y''(x_{0}) +O(h^{2}) ]
\end{eqnarray}
よって、
\begin{eqnarray}
	\Delta y
	&=&
	( \alpha + \beta ) y'(x_{0}) h + \beta y''(x_{0}) h^{2}
	\label{eqn:dokuritu2}
\end{eqnarray}
式 (\ref{eqn:dokuritu1}) と
式 (\ref{eqn:dokuritu2}) を比較すれば、
パラメータは $\alpha=\beta=\dfrac{1}{2}$ と決定される。
最終的に、2次のオーダーでの漸化式は以下のようになる。
\begin{eqnarray}
	\left\{
	\begin{array}{rcl}
		k_{1n}
		 & = &
		h f(x_{n},y_{n})
		\\[2mm]
		k_{2n}
		 & = &
		h f(x_{n}+h,y_{n}+k_{1n})
		\\[2mm]
		y_{n+1}
		 & = &
		y_{n} + \dfrac{1}{2} (k_{1n} + k_{2n})
	\end{array}
	\right.
\end{eqnarray}
これは「\textbf{ホイン (Heun) 法}」と呼ばれ、オイラー法よりも精度の高い解を与える。
\section{4次のルンゲ=クッタ法(RK4)}

同様の手順で、
$h$ の4次のオーダーまで考慮すると、
以下の漸化式が得られる。
\begin{eqnarray}
	\left\{
	\begin{array}{rcl}
		k_{1n}
		 & = &
		h f(x_{n},y_{n})
		\\[2mm]
		k_{2n}
		 & = &
		h f(x_{n}+\dfrac{h}{2},y_{n}+\dfrac{k_{1}}{2})
		\\[2mm]
		k_{3n}
		 & = &
		h f(x_{n}+\dfrac{h}{2},y_{n}+\dfrac{k_{2}}{2})
		\\[2mm]
		k_{4n}
		 & = &
		h f(x_{n}+h,y_{n}+k_{3n})
		\\[2mm]
		y_{n+1}
		 & = &
		y_{n} + \dfrac{1}{6} (k_{1n} + 2 k_{2n} + 2 k_{3n} + k_{4n})
	\end{array}
	\right.
\end{eqnarray}
RK4 は、常微分方程式の数値解法として、しばしば最も合理的な手法とされる。
これ以上高次の(例えば5次以上の)ルンゲ=クッタ計算を用いても、
同じ精度を得るために RK4 よりも多くの計算時間を要する場合があり、
計算コストに見合わないことがある。
\end{document}
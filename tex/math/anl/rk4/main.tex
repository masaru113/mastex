\documentclass[uplatex,a4j,12pt,dvipdfmx]{jsarticle}
\usepackage{amsmath,amsthm,amssymb,bm,color,enumitem,mathrsfs,url,epic,eepic,ascmac,ulem,here,ascmac}
\usepackage[letterpaper,top=2cm,bottom=2cm,left=3cm,right=3cm,marginparwidth=1.75cm]{geometry}
\usepackage[english]{babel}
\usepackage[dvipdfm]{graphicx}
\usepackage[hypertex]{hyperref}
\title{Runge-Kutta Method}

\author{Masaru Okada}

\date{\today}

\begin{document}

\maketitle

\section{Euler method}

To begin, let's introduce the simplest method for numerically solving
an ordinary differential equation (ODE).
This is also known as the (1st order) Runge-Kutta method.
The equation to be solved is assumed to be:
\begin{eqnarray}
	\dfrac{dy}{dx} &=& f(x,y)
\end{eqnarray}
Here, $f(x,y)$ is a given function.
To simplify the notation,
we set $y_{i}=y(x_{i})$
and
$x_{i+1} = x_{i} + h$.
The initial condition (at $i=0$) is given as $(x_{0},y_{0})$.
If the $x$-mesh step $h$ is chosen such that $| h | \ll 1$,
$y_{i+1}$ can be expanded by a Taylor series in $h$.
\begin{eqnarray}
	y_{i+1} &=&
	y(x_{i} + h)
	\nonumber \\[2mm] &=&
	y(x_{i}) \ \
	+
	\dfrac{dy}{dx} \Big|_{x=x_{i}} \hspace{-4mm} h \ \
	+
	O[h^{2}]
	\nonumber \\[2mm] &=&
	y_{i} \ \
	+
	f(x_{i},y_{i}) h
	+
	O[h^{2}]
\end{eqnarray}
Thus, we obtain the following expression:
\begin{eqnarray}
	f(x_{i},y_{i}) &=& \dfrac{y_{i+1} - y_{i}}{h}
\end{eqnarray}
This is correct to the first order in $h$.
The solution is given by the following recurrence relation:
\begin{eqnarray}
	\left\{
	\begin{array}{lcl}
		x_{i+1}    & = & x_{i} + h                                 \\[-1mm]
		           &   & \hspace{35mm} ( x_{0} \text{ is given.} ) \\[-1mm]
		y(x_{i+1}) & = & y(x_{i}) + f[x_{i},y(x_{i})]
	\end{array}
	\right.
\end{eqnarray}

\section{2nd order Runge-Kutta method (Heun method)}

In a similar manner,
$y_{i+1}$ is expanded with respect to $h$.
Writing it out to the second order, we have:
\begin{eqnarray}
	y_{i+1} &=&
	y(x_{i} + h)
	\nonumber \\[2mm] &=&
	y(x_{i}) \ \
	+
	y'(x_{i}) h
	+
	\dfrac{1}{2} y''(x_{i}) h^{2}
	+
	O[h^{3}]
	\\[4mm]
	\Delta y
	&=&
	y_{i+1} - y_{i}
	\nonumber \\[2mm] &=&
	y_{1} - y_{0}
	\nonumber \\[2mm] &=&
	y'(x_{0}) h
	+
	\dfrac{1}{2} y''(x_{0}) h^{2}
	\label{eqn:dokuritu1}
\end{eqnarray}
While the value $y'(x_{0})$ was used before,
this time let's try selecting another value, $y'(x_{1})$, in addition.
\begin{eqnarray}
	\Delta y
	&=&
	h [ \alpha y'(x_{0}) + \beta y'(x_{1}) ]
\end{eqnarray}
where $\alpha$ and $\beta$ are undetermined constants.
Here, $\beta y'(x_{1})$ can also be expanded:
\begin{eqnarray}
	\beta y'(x_{1})
	&=&
	\beta y'(x_{0}+h)
	\nonumber \\[2mm] &=&
	\beta [ y'(x_{0}) + h y''(x_{0}) +O(h^{2}) ]
\end{eqnarray}
Hence,
\begin{eqnarray}
	\Delta y
	&=&
	( \alpha + \beta ) y'(x_{0}) h + \beta y''(x_{0}) h^{2}
	\label{eqn:dokuritu2}
\end{eqnarray}
By comparing Eq. (\ref{eqn:dokuritu1}) and
Eq. (\ref{eqn:dokuritu2}),
the parameters are determined as $\alpha=\beta=\dfrac{1}{2}$.
Finally, the second-order recurrence relation is as follows:
\begin{eqnarray}
	\left\{
	\begin{array}{rcl}
		k_{1n}
		 & = &
		h f(x_{n},y_{n})
		\\[2mm]
		k_{2n}
		 & = &
		h f(x_{n}+h,y_{n}+k_{1n})
		\\[2mm]
		y_{n+1}
		 & = &
		y_{n} + \dfrac{1}{2} (k_{1n} + k_{2n})
	\end{array}
	\right.
\end{eqnarray}
This is called the '\textbf{Heun method}' and provides a more accurate solution than the Euler method.
\section{4th order Runge-Kutta method (RK4)}

Following the same procedure,
and considering terms up to the fourth order in $h$,
the following recurrence relations are obtained:
\begin{eqnarray}
	\left\{
	\begin{array}{rcl}
		k_{1n}
		 & = &
		h f(x_{n},y_{n})
		\\[2mm]
		k_{2n}
		 & = &
		h f(x_{n}+\dfrac{h}{2},y_{n}+\dfrac{k_{1}}{2})
		\\[2mm]
		k_{3n}
		 & = &
		h f(x_{n}+\dfrac{h}{2},y_{n}+\dfrac{k_{2}}{2})
		\\[2mm]
		k_{4n}
		 & = &
		h f(x_{n}+h,y_{n}+k_{3n})
		\\[2mm]
		y_{n+1}
		 & = &
		y_{n} + \dfrac{1}{6} (k_{1n} + 2 k_{2n} + 2 k_{3n} + k_{4n})
	\end{array}
	\right.
\end{eqnarray}
RK4 is often considered the most reasonable method for solving ODEs numerically.
This is because using even higher-order (e.g., 5th order or more) Runge-Kutta calculations
may require more computation time than RK4 to achieve the same precision,
sometimes making the extra computational cost unjustifiable.
\end{document}
\documentclass[uplatex,a4j,12pt,dvipdfmx]{jsarticle}
\usepackage{amsmath,amsthm,amssymb,bm,color,enumitem,mathrsfs,url,epic,eepic,ascmac,ulem,here,ascmac}
\usepackage[letterpaper,top=2cm,bottom=2cm,left=3cm,right=3cm,marginparwidth=1.75cm]{geometry}
\usepackage[english]{babel}
\usepackage[dvipdfm]{graphicx}
\usepackage[hypertex]{hyperref}
\title{
Tensor Algebra
}
\author{Masaru Okada}

\date{\today}

\begin{document}

\maketitle

\tableofcontents

\ \\

\section{The Dual Vector Space of Linear Maps}

\subsection{The Dual Space of a Vector Space}

Let $V$ be a finite-dimensional vector space.

A linear map from $V$ to $\mathbb{R}$ is called a linear functional on $V$.

For linear functionals $\varphi, \psi : V \to \mathbb{R}$,
a sum and a scalar product can be defined.
That is, for a real number $\alpha$ and
$\bf{x} \in V$,
\[
	\begin{array}{rcl}
		(\varphi + \psi) (\bf{x}) & = & \varphi (\bf{x}) + \psi (\bf{x})
		\\
		(\alpha \varphi) (\bf{x}) & = & \alpha \varphi (\bf{x})
	\end{array}
\]

The sum and scalar product possess linearity.
For $\bf{x}, \bf{y} \in V$,

For example, regarding the sum:
\[
	\begin{array}{rcl}
		(\varphi + \psi) (\bf{x} + \bf{y}) & = & \varphi (\bf{x} + \bf{y}) + \psi (\bf{x} + \bf{y})
		\\
		                                   & = &
		\varphi (\bf{x} + \bf{y}) + \psi (\bf{x} + \bf{y})
		\\
		                                   & = &
		\varphi (\bf{x}) + \varphi (\bf{y})
		+
		\psi (\bf{x}) + \psi (\bf{y})
		\\
		                                   & = &
		\varphi (\bf{x}) + \psi (\bf{x})
		+ \psi (\bf{y}) + \varphi (\bf{y})
		\\
		                                   & = &
		(\varphi + \psi) (\bf{x}) + (\varphi + \psi) (\bf{y})
	\end{array}
\]

Similarly, for the scalar product:
\[
	\begin{array}{rcl}
		(\alpha \varphi) (\bf{x} + \bf{y}) & = & \alpha ( \varphi (\bf{x} + \bf{y}) )
		\\
		                                   & = &
		\alpha \varphi (\bf{x}) + \alpha \varphi (\bf{y})
	\end{array}
\]

The set of all linear functionals on $V$ also forms a vector space.

That this satisfies the vector space axioms can be confirmed from, for example,
\[
	\begin{array}{rcl}
		\alpha (\varphi + \psi) (\bf{x} ) & = & ( \alpha \varphi + \alpha \psi) (\bf{x} )
		\\
		(\alpha + \beta) \varphi (\bf{x}) & = & (\alpha \varphi + \beta \varphi) (\bf{x})
	\end{array}
\]
and so on.

The vector space formed by all linear functionals on $V$ is called the dual vector space and is denoted by $V^{*}$.

\subsection{The Dual Space of the Dual Space of a Vector Space}

An element $\varphi$ of $V^{*}$ provides the mapping $\varphi : x \to \varphi(x)$.

From this perspective, $x$ appears as a variable, and $\varphi$ appears as a linear function.

Let us further consider its dual space.
In this case, $x$ can be seen as fixed, while $\varphi$ can be viewed as a variable that moves freely over $V^{*}$.
Written in symbols, we see it as follows:

$$
	x : \varphi \to \varphi(x)
$$

To clarify which is the variable, when we adopt the perspective of fixing one $x$ and letting $\varphi$ vary, we will write it as $\tilde{\tilde{x}}$.
That is,

$$
	\tilde{\tilde{x}}(\varphi) = \varphi(x)
$$

With this, the equalities that held in the linear space, for example,
\[
	\begin{array}{rcl}
		(\varphi + \psi) (\bf{x} + \bf{y})
		 & = &
		(\varphi + \psi) (\bf{x}) + (\varphi + \psi) (\bf{y})
		\\
		(\alpha \varphi) (\bf{x} + \bf{y})
		 & = &
		\alpha \varphi (\bf{x}) + \alpha \varphi (\bf{y})
	\end{array}
\]
These equalities can be viewed from a different perspective as, respectively,
\[
	\begin{array}{rcl}
		(\tilde{\tilde{x}} + \tilde{\tilde{y}} ) (\varphi + \psi)
		 & = &
		\tilde{\tilde{x}} (\varphi + \psi)  + \tilde{\tilde{y}} (\varphi + \psi)
		\\
		(\tilde{\tilde{x}} + \tilde{\tilde{y}} ) (\alpha \varphi)
		 & = &
		\alpha \tilde{\tilde{x}} (\varphi) + \alpha \tilde{\tilde{y}} (\varphi)
	\end{array}
\]

This $\tilde{\tilde{x}}$ is an element of the dual space of $V^{*}$, so we denote it as $\tilde{\tilde{x}} \in V^{**}$.


\subsection{Basis of a Vector Space and Dual Basis}

\subsubsection{Basis of a Vector Space}

$V$ is a finite-dimensional vector space, and let its dimension be $n$.
If we denote its basis as $\{ e_{1}, e_{2}, e_{3}, \cdots , e_{n} \}$,
then $\mathbf{x} \in V$ is expressed as
\[
	\begin{array}{rcl}
		V \ni \ \
		\mathbf{x}
		 & = &
		x^{1} e_{1} + x^{2} e_{2} + x^{3} e_{3} + \cdots + x^{n} e_{n}
		\\
		 & = &
		\displaystyle \sum_{k=1}^{n} x^{k} e_{k}
	\end{array}
\]

\subsubsection{Dual Basis}

Considering $V^{*}$, its elements were linear functionals $\varphi$.
We want to consider a basis for $V^{*}$, so, to make the notation look like a basis (using $e^{k}$ instead of $\varphi$),
$V^{*}$ is spanned by the linear functionals $\{ e^{1}, e^{2}, e^{3}, \cdots , e^{n} \}$.
Letting $\varphi(e_{1}) = a_{1}, \varphi(e_{2}) = a_{2}, \cdots ,\varphi(e_{n}) = a_{n}$,
$\varphi \in V^{*}$ is expressed in this notation as
\[
	\begin{array}{rcl}
		V^{*} \ni \ \ \varphi
		 & = &
		a_{1} e^{1} + a_{2} e^{2} + a_{3} e^{3} + \cdots + a_{n} e^{n}
		\\
		 & = &
		\displaystyle \sum_{k=1}^{n} a_{k} e^{k}
	\end{array}
\]

With respect to the basis
$\{ e_{1}, e_{2}, e_{3}, \cdots , e_{n} \}$
of $V$,
the basis of $V^{*}$,
namely the set of linear functionals
$\{ e^{1}, e^{2}, e^{3}, \cdots , e^{n} \}$,
is called the dual basis.

\subsubsection{Dual Basis of the Dual Space of the Dual Space $V^{**}$}

Using the definition of the dual basis
\[
	\begin{array}{rcl}
		V^{*} \ni \ \ \varphi
		 & = &
		a_{1} e^{1} + a_{2} e^{2} + a_{3} e^{3} + \cdots + a_{n} e^{n}
		\\
		 & = &
		\displaystyle \sum_{k=1}^{n} a_{k} e^{k}
	\end{array}
\]
the correspondence with the vector space basis is as follows:
\[
	\begin{array}{rcl}
		e^{i}(\mathbf{x})
		 & = &
		e^{i} (x^{1} e_{1} + x^{2} e_{2} + x^{3} e_{3} + \cdots + x^{n} e_{n})
	\end{array}
\]

Here, if we set the coefficient of the $j$-th component to 1 and all other coefficients to 0,
\[
	\begin{array}{rcl}
		e^{i} (0 e_{1} + 0 e_{2}  + \cdots + 1 e_{j} + \cdots + 0 e_{n})
	\end{array}
\]
From this, the following relationship between the bases is obtained:
\[
	e^{i}(e_{j})
	=
	\left\{
	\begin{array}{l}
		1, \ \ (i=j) \\
		0, \ \ (i \neq j)
	\end{array}
	\right.
\]

\paragraph{One-to-one correspondence between $V$ and $V^{**}$}

${}$

For $x,y \in V$ and $\tilde{\tilde{x}}, \tilde{\tilde{y}} \in V^{**}$,
$x \neq y \Rightarrow \tilde{\tilde{x}} \neq \tilde{\tilde{y}}$ holds.

This is because,
if we express
$$
	x = \sum_{i=1}^{n} x^{i} e_{i} , \ \
	y = \sum_{i=1}^{n} y^{i} e_{i}
$$
then if $x \neq y$, there exists some $j$ for which
$x^{j} \neq y^{j}$ holds.

Taking
$\{ e^{1}, e^{2}, e^{3}, \cdots , e^{n} \}$
as the dual basis,
$$
	\tilde{\tilde{x}}(e^{j})
	\ = \
	e^{j}(x)
	\ = \
	x^{j}
	, \ \
	\tilde{\tilde{y}}(e^{j})
	\ = \
	e^{j}(y)
	\ = \
	y^{j}
$$
Therefore,
$\tilde{\tilde{x}}$ and $\tilde{\tilde{y}}$ differ in their value at $e^{j}$.
Thus,
$x \neq y \Rightarrow \tilde{\tilde{x}} \neq \tilde{\tilde{y}}$
can be stated.

\paragraph{Dual Basis of $V^{**}$}

${}$

The relationship between the dual basis of $V^{*}$ and the basis of $V$ is
\[
	e^{i}(e_{j})
	=
	\left\{
	\begin{array}{l}
		1, \ \ (i=j) \\
		0, \ \ (i \neq j)
	\end{array}
	\right.
\]
Expressing this using elements of $V^{**}$,
\[
	\tilde{e}_{i}(e^{j})
	=
	\left\{
	\begin{array}{l}
		1, \ \ (i=j) \\
		0, \ \ (i \neq j)
	\end{array}
	\right.
\]
which shows that
the dual basis of $V^{**}$
$\{ \tilde{e}_{1}, \tilde{e}_{2}, \tilde{e}_{3}, \cdots , \tilde{e}_{n} \}$
is
the dual basis of the dual basis of $V^{*}$
$\{ e^{1}, e^{2}, e^{3}, \cdots , e^{n} \}$.

\ \\

To summarize what we have covered so far in different terms:

\begin{itembox}[l]{One-to-one correspondence between $V$ and $V^{**}$}
	There is a one-to-one map $\Phi : V \to V^{**}$
	from $V$ to $V^{**}$,
	which is expressed as
	$$
		\Phi: V \ni x = \sum_{k=1}^{n} x^{i} e_{i} \mapsto \tilde{\tilde{x}} = \sum_{k=1}^{n} x^{i} \tilde{e}_{i} \in V^{**}
	$$
\end{itembox}



\section{Bilinear Space}

Up to this point, the dual of a vector space was a vector space, and its dual was also a vector space (formed by the set of linear maps), so the discussion was closed within vector spaces.
From here, we will consider extending vector spaces.

\ \\

Consider the Cartesian product set $V^{*} \times V^{*}$.

$$
	V^{*} \times V^{*}
	=
	\{
	\ (\varphi, \psi ) \ | \ \varphi \in V^{*}, \psi \in V^{*}
	\}
$$

The bilinearity of a two-variable function
$M(\varphi,\psi)$ defined on this $V^{*} \times V^{*}$ is defined as follows.


\begin{itembox}[l]{Bilinear Function}
	A two-variable function
	$M(\varphi,\psi)$
	defined on $V^{*} \times V^{*}$
	is called a bilinear function on $V^{*}$
	when it satisfies the following properties (bilinearity). ($\alpha, \beta \in \mathbb{R}$)
	\begin{enumerate}
		\item $M(\alpha \varphi + \beta \varphi', \psi) = \alpha M( \varphi, \psi) + \beta M( \varphi', \psi)$
		\item $M(\varphi, \alpha \psi + \beta \psi') = \alpha M( \varphi, \psi) + \beta M( \varphi, \psi')$
	\end{enumerate}
\end{itembox}

A bilinear function is a function that is linear in each variable.

\subsection{The Vector Space Formed by All Bilinear Functions}

\paragraph{Sum and Scalar Multiplication of Bilinear Functions}

${}$

For bilinear functions $M, N$ on $V^{*}$,
$M + N$ and $\alpha M$ are also bilinear functions
($\alpha \in \mathbb{R}$).

In other words, the set of all bilinear functions on $V^{*}$ forms a vector space.

\ \\

Let us denote the vector space formed by all bilinear functions on $V^{*}$ as $L_{2}(V^{*})$.
The subscript 2 signifies that it is of two variables.

Using this notation, the vector space $V^{**}$ formed by all linear functionals on $V^{*}$ can be
expressed as $L_{1}(V^{*})$.
And this vector space $L_{1}(V^{*})$ could be identified with $V$.
$$
	V = L_{1}(V^{*})
$$

\subsection{Tensor Product}

We introduce the tensor product as an operation that connects $L_{1}(V^{*})$ and $L_{2}(V^{*})$.

\begin{itembox}[l]{Tensor Product}
	Let $V \otimes V = L_{2}(V^{*})$.
	The vector space $V \otimes V$ is called the second-order tensor product of $V$.
\end{itembox}

In this notation,
$$V = L_{1}(V^{*})$$
$$V \otimes V = L_{2}(V^{*})$$
which is useful for defining higher-order tensor products.

\ \\

For $x,y \in V, \ \ \varphi, \psi \in V^{*}$,
if we define
$$
	(x \otimes y)(\varphi, \psi) = \varphi(x) \psi(y)
$$
this becomes a map from
$V^{*} \times V^{*}$
to
$\mathbb{R}$.
$$
	x \otimes y : V^{*} \times V^{*} \to \mathbb{R}
$$
This map is an element of the tensor product.
$$
	x \otimes y \in V \otimes V
$$

\section{Multilinear Functions and Tensor Spaces}

By repeating the tensor product notation, we can define multilinear functions and tensor spaces.

\begin{itembox}[l]{$k$-linear Function}
	A function
	$M(\varphi_{1}, \varphi_{2}, \cdots , \varphi_{k})$
	defined on the $k$-fold Cartesian product $V^{*} \times \cdots \times V^{*}$ of $V^{*}$
	is called a $k$-linear function on $V^{*}$
	if it satisfies the following property.
	$$
		M(\varphi_{1}, \varphi_{2}, \cdots , \alpha \varphi_{i} + \beta \psi_{i} , \cdots , \varphi_{k})
		=
		\alpha M(\varphi_{1}, \varphi_{2}, \cdots , \varphi_{i} , \cdots , \varphi_{k})
		+
		\beta M(\varphi_{1}, \varphi_{2}, \cdots , \psi_{i} , \cdots , \varphi_{k})
	$$
\end{itembox}

The set of all $k$-linear functions on $V^{*}$ also forms a vector space,
which we denote as
$L_{k}(V^{*})$.

To make it a vector space,
for $M, N \in L_{k}(V^{*}) , \ \ \alpha \in \mathbb{R}$,
we define addition and scalar multiplication as follows, respectively.

\begin{itembox}[l]{Addition and Scalar Multiplication of $k$-linear Functions}
	$$
		( M + N ) ( \varphi_{1} , \cdots, \varphi_{k})
		=
		M ( \varphi_{1} , \cdots, \varphi_{k})
		+
		N ( \varphi_{1} , \cdots, \varphi_{k})
	$$
	$$
		( \alpha  M ) ( \varphi_{1} , \cdots, \varphi_{k})
		=
		\alpha M ( \varphi_{1} , \cdots, \varphi_{k})
	$$
\end{itembox}

Varying the natural number $k$ gives a sequence of vector spaces formed by $k$-linear functions.

$$
	L_{1}(V^{*}) \ , \ \ L_{2}(V^{*}) \ , \ \ L_{3}(V^{*}) \ , \ \ \cdots \ , \ \ L_{k}(V^{*}) \ , \ \ \cdots
$$

Just as $L_{2}(V^{*})$ was denoted $V \otimes V$, the vector space formed by general $k$-linear functions
is called the $k$-th order tensor space and is defined as follows.

\begin{itembox}[l]{$k$-th Order Tensor Space}
	$$L_{k}(V^{*}) = V \otimes V \otimes \cdots \otimes V = \otimes^{k} V$$
\end{itembox}

Although we have only changed the notation, the sequence of vector spaces formed by $k$-linear functions is now expressed as follows:
$$
	\otimes^{1} V (= V) \ , \ \ \otimes^{2} V \ , \ \ \otimes^{3} V \ , \ \ \cdots \ , \ \ \otimes^{k} V \ , \ \ \cdots
$$

Similar to the bilinear case,
the tensor product of $k$ elements $x_{1}, x_{2} , \cdots , x_{k}$ of $V$ can be defined as the following map.

$$ \otimes^{k} V \ni x_{1} \otimes x_{2} \otimes \cdots \otimes x_{k}: V^* \times V^* \times \cdots \times V^* \to \mathbb{R}$$

$$
	x_{1} \otimes \cdots \otimes x_{k} ( \varphi_{1} , \cdots, \varphi_{k})
	=
	\varphi_{1} (x_{1}) \cdots \varphi_{k} (x_{k})
$$

A note on terminology:
One might imagine 'tensor product' as the product operator itself, but it should be noted that an element of the tensor product (i.e., a tensor) is 'a map to the real numbers'.

\section{Polynomial Algebra}

\subsection{Direct Sum and Product of Sets of Monomials}

Before introducing tensor algebra, let's consider the more familiar and intuitive algebra of polynomials.

First, consider the set of all $k$-th degree monomials $\mathbf{P}^{k} = \{ ax^{k} \ | \ a \in \mathbb{R} \}$.

This $\mathbf{P}^{k}$ is a 1-dimensional vector space isomorphic to $\mathbb{R}$.

\[
	\begin{array}{rcl}
		\mathbf{P}^{0} & \ni    & a_{0}       \\
		\mathbf{P}^{1} & \ni    & a_{1} x     \\
		\mathbf{P}^{2} & \ni    & a_{2} x^{2} \\
		\mathbf{P}^{3} & \ni    & a_{3} x^{3} \\
		               & \vdots &             \\
		\mathbf{P}^{k} & \ni    & a_{k} x^{k} \\
		               & \vdots &             \\
	\end{array}
\]

These sets are mutually disjoint, and the disjoint union of vector spaces is called the direct sum, expressed using $\oplus$:

\[
	\begin{array}{rcl}
		\mathbf{P}^{0}                                                                                 & \ni    & a_{0}                                                \\
		\mathbf{P}^{0} \oplus \mathbf{P}^{1}                                                           & \ni    & a_{0} + a_{1} x                                      \\
		\mathbf{P}^{0} \oplus \mathbf{P}^{1} \oplus \mathbf{P}^{2}                                     & \ni    & a_{0} + a_{1} x + a_{2} x^{2}                        \\
		\mathbf{P}^{0} \oplus \mathbf{P}^{1} \oplus \mathbf{P}^{2} \oplus \mathbf{P}^{3}               & \ni    & a_{0} + a_{1} x + a_{2} x^{2} + a_{3} x^{3}          \\
		                                                                                               & \vdots &                                                      \\
		\mathbf{P}^{0} \oplus \mathbf{P}^{1} \oplus \mathbf{P}^{2} \oplus \cdots \oplus \mathbf{P}^{k} & \ni    & a_{0} + a_{1} x + a_{2} x^{2} + \cdots + a_{k} x^{k} \\
		                                                                                               & \vdots &                                                      \\
	\end{array}
\]

The product of monomials defines a map
$$
	\times : \mathbf{P}^{k} \times \mathbf{P}^{l} \to \mathbf{P}^{k+l}
$$
and can be expressed as $ (a_{k} x^{k}) \times (a_{l} x^{l}) = a_{k} a_{l} x^{k+l} \in \mathbf{P}^{k+l}$.

The set of all polynomials $\mathbf{P}$ is
$$
	\mathbf{P}
	=
	\mathbf{P}^{0} \oplus \mathbf{P}^{1} \oplus \mathbf{P}^{2} \oplus \cdots \oplus \mathbf{P}^{k} \oplus \cdots
$$

We know from experience that the product and sum of polynomials can be defined within $\mathbf{P}$ prepared in this way.

${}$

Let's introduce a convenient symbol for the direct sum.
$$
	\mathbf{P}
	=
	\mathbf{P}^{0} \oplus \mathbf{P}^{1} \oplus \mathbf{P}^{2} \oplus \cdots \oplus \mathbf{P}^{k} \oplus \cdots
	=
	\displaystyle \bigoplus_{k=0}^{\infty} \mathbf{P}^{k}
$$

\subsection{Algebra}

$\mathbf{P}$, prepared as above, has the following algebraic properties.

\begin{itembox}[l]{Algebra}
	\begin{enumerate}
		\item $\mathbf{P}$ is a vector space over $\mathbb{R}$.
		\item A multiplication $\times$ with the following properties is defined on $\mathbf{P}$.
		      \begin{enumerate}
			      \item For $p,q,r \in \mathbf{P}$, the associative law holds: $p \times (q \times r) = (p \times q) \times r$
			      \item For $p,q,r \in \mathbf{P}$, the distributive law holds: $(p + q) \times r = p \times r + q \times r \ , \ \ p \times (q + r) = p \times q + p \times r$
			      \item For $a \in \mathbb{R} \ , \ \ p,q \in \mathbf{P}$, $a (p \times q) = a p \times q = p \times a q$
			      \item For $1 \in \mathbb{R} \ , \ \ p \in \mathbf{P}$, $1p=p$
		      \end{enumerate}
	\end{enumerate}
\end{itembox}

A set with such properties is generally called an algebra, and $\mathbf{P}$ in particular is called a polynomial algebra.


\section{Tensor Algebra}

In the case of the tensor product, similar to polynomial algebra,
an index law such as
$\xi \in \otimes^{k} V$
and
$\eta \in \otimes^{l} V$
implies
$\xi \otimes \eta \in \otimes^{k+l} V$
holds.

By a construction similar to polynomial algebra, an algebraic structure is obtained for the tensor product as well, which is called a tensor algebra.

\begin{itembox}[l]{Tensor Algebra}
	For a vector space $V$,
	\[
		\begin{array}{rcl}
			T(V) & = & \mathbb{R} \oplus V \oplus ( \otimes^{2} V ) \oplus ( \otimes^{3} V ) \oplus \cdots \oplus ( \otimes^{k} V ) \oplus \cdots \\
			     & = & \displaystyle \bigoplus_{k=0}^{\infty} \otimes^{k} V
		\end{array}
	\]

	The space $T(V)$ defined by this satisfies the definition of an algebra and is called a tensor algebra.
\end{itembox}

The algebraic properties it satisfies are as follows.

\begin{itembox}[l]{Properties of Tensor Algebra}
	\begin{enumerate}
		\item $T(V)$ is a vector space over $\mathbb{R}$.
		\item A multiplication $\otimes$ with the following properties is defined on $T(V)$.
		      \begin{enumerate}
			      \item For $\xi,\eta,\zeta \in T(V)$, the associative law holds: $\xi \otimes (\eta \otimes \zeta) = (\xi \otimes \eta) \otimes \zeta$
			      \item For $\xi,\eta,\zeta \in T(V)$, the distributive law holds: $(\xi + \eta) \otimes \zeta = \xi \otimes \zeta + \eta \otimes \zeta \ , \ \ \xi \otimes (\eta + \zeta) = \xi \otimes \eta + \xi \otimes \zeta$
			      \item For $a \in \mathbb{R} \ , \ \ \xi,\eta \in T(V)$, $a (\xi \otimes \eta) = a \xi \otimes \eta = \xi \otimes a \eta$
			      \item For $1 \in \mathbb{R} \ , \ \ \xi \in T(V)$, $1\xi=\xi$
		      \end{enumerate}
	\end{enumerate}
\end{itembox}

\ \\

Starting from a vector space, via the dual vector space and bilinear space to the tensor space, we have finally, in these notes, arrived at the tensor algebra.

The elements of the tensor algebra are tensors, and from here, general relativity unfolds.

By introducing an ideal to the tensor algebra, the exterior algebra (Grassmann algebra) is obtained.

Exterior algebra is used everywhere in physics.
For example, it is used in the Chern-Simons form.

The elements of the Grassmann algebra can be regarded as fermions, describing various physical models.

\end{document}
\documentclass[uplatex,a4j,12pt,dvipdfmx]{jsarticle}
\usepackage[english]{babel}
\usepackage[letterpaper,top=2cm,bottom=2cm,left=3cm,right=3cm,marginparwidth=1.75cm]{geometry}
\usepackage{amsmath, amssymb}
\usepackage{graphicx}
\usepackage[colorlinks=true, allcolors=blue]{hyperref}
\usepackage{fancybox}
\usepackage{tikz-cd}
\title{
リーマン幾何学の初歩
}

\author{
岡田 大 (Okada Masaru)
}

\begin{document}
\maketitle

一般相対論の復習のためのリーマン幾何学の初歩についてまとめたい。

\section{リーマン空間}

リーマン空間は距離空間で、なおかつ微小距離だけ異なる2点
$x^{i}, x^{i} + dx$
の間の距離
$ds$
が
\[
	ds^{2} = g_{ij}(x) dx^{i} dx^{j}
\]
で定義されるような空間である。

\subsection{例:2次元平面}

2次元平面はリーマン空間になる。

\[
	g_{ij}(x) =
	\left(
	\begin{array}{cc}
			1 & 0 \\
			0 & 1
		\end{array}
	\right)
\]
すなわち
\[
	ds^{2} = dx^{2} + dy^{2}
\]

\subsection{例:2次元球面}

2次元球面もリーマン空間である。

\[
	ds^{2} = a^{2} d \theta^{2} + a^{2} \sin^{2} \theta d \phi^{2}
\]
すなわち、
\[
	g_{ij}(x) =
	\left(
	\begin{array}{cc}
			a^{2} & 0                     \\
			0     & a^{2} \sin^{2} \theta
		\end{array}
	\right)
\]

\subsection{例:ミンコフスキー空間}

ミンコフスキー空間もリーマン空間である。

\[
	g_{ij}(x) = \eta_{ij} =
	\left(
	\begin{array}{cccc}
			-1 & 0 & 0 & 0 \\
			0  & 1 & 0 & 0 \\
			0  & 0 & 1 & 0 \\
			0  & 0 & 0 & 1 \\
		\end{array}
	\right)
\]

\[
	ds^{2} = \eta_{ij} dx^{i} dx^{j}
\]

\end{document}
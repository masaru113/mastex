\documentclass[uplatex,a4j,12pt,dvipdfmx]{jsarticle}
\usepackage[english]{babel}
\usepackage[letterpaper,top=2cm,bottom=2cm,left=3cm,right=3cm,marginparwidth=1.75cm]{geometry}
\usepackage{amsmath, amssymb}
\usepackage{graphicx}
\usepackage[colorlinks=true, allcolors=blue]{hyperref}
\usepackage{fancybox}
\usepackage{tikz-cd}
\title{
An Introduction to Topological Groups
}

\author{
mastex
}

\begin{document}
\maketitle


\begin{abstract}
	The definition and examples of topological groups
\end{abstract}

\section{Topological Groups}

We will define a topological group, which has both the structure of a group and a topological space.

A set $G$ is a topological group if:

\begin{enumerate}
	\item $G$ is a group.
	\item $G$ is a topological space.
	\item The maps $\mu: G \times G \to G$ and $\nu: G \to G$, defined as $\mu(x,y)=xy$ and $\nu(x) = x^{-1}$ respectively, are continuous.
\end{enumerate}

The last condition is a definition that uses the continuity of a topological space for the usual group operations of multiplication and inversion.
It's a statement that the group operations are continuous maps.


If the topology of $G$ is derived from a metric space, the last condition can be rephrased as:

\[
	\left\{
	\begin{array}{r}
		\displaystyle \lim_{ m \to \infty } x_{m} \\
		\displaystyle \lim_{ m \to \infty } y_{m}
	\end{array}
	\right.
	\ \Rightarrow \
	\left\{
	\begin{array}{r}
		\displaystyle \lim_{m \to \infty } x_{m} y_{m} = xy \\
		\displaystyle \lim_{m \to \infty } x_{m}^{-1} = x^{-1}
	\end{array}
	\right.
\]

If a metric is present, this condition means that multiplication and inversion are also preserved in the limit.





\subsection{Examples of Topological Groups}

\subsubsection{The Circle}

It is known that the circle $S^{n} (n=0,1)$ is a topological group, but the sphere $S^{n} (n \neq 3)$ is not.
(This was shown by E. Cartan.)

\begin{enumerate}
	\item $S^{0} = \{ -1, 1 \}$
	\item $S^{1} = \{ x \in \mathbb{C} \ | \ |x| = 1 \}$
	\item $S^{3} = \{ x \in \mathbb{H} \ | \ |x| = 1 \}$
\end{enumerate}

All of these are topological groups with respect to multiplication.


\paragraph{Verification that $S^{0}$ is a topological group with respect to multiplication}

{}

(Verification that it is a group)

$$1*1=1 \in S^{0}$$
$$1*(-1)=(-1) \in S^{0}$$
$$(-1)*1=(-1) \in S^{0}$$
$$(-1)*(-1)=1 \in S^{0}$$

It is closed under multiplication, and the inverse of each element is also in $S^{0}$.

	${}$

(Verification that it is a topological space)

The subspaces of $S^{0}$ are
$$\varnothing, \{ -1 \} , \{ 1 \} , \{ -1 , 1 \}$$
and we define all four to be open sets.

1. The total space $S^{0}$ and the empty set $\varnothing$ are open sets.

2. The intersection of any finite number of these open sets is an open set.
For example,
$$\varnothing \cap \{ -1 \} = \varnothing$$
$$\varnothing \cap \{ 1 \} = \varnothing$$
$$\varnothing \cap S^{0} = \varnothing$$
$$ \{ -1 \} \cap \{ 1 \} = \varnothing$$
$$ \{ -1 \} \cap S^{0} = \{ -1 \} $$
$$ \{ 1 \} \cap S^{0} = \{ 1 \} $$
$$ \{ -1 \} \cap \{ 1 \} \cap S^{0} = \varnothing $$

3. The union of any number of these open sets is also an open set.
For example,
$$\varnothing \cup \{ -1 \} = \{ -1 \}$$
$$\varnothing \cup \{ 1 \} = \{ 1 \}$$
$$\varnothing \cup S^{0} = S^{0}$$
$$ \{ -1 \} \cup \{ 1 \} = \{ -1 , 1 \}$$
$$ \{ -1 \} \cup S^{0} = S^{0} $$
$$ \{ 1 \} \cup S^{0} = S^{0} $$
$$ \{ -1 \} \cup \{ 1 \} \cup S^{0} = S^{0} $$

From the above, $S^{0}$ is an open set.

	${}$

\paragraph{Verification that $S^{1}$ is a topological group with respect to multiplication}

${}$

$S^{1}$ is a subset of $\mathbb{C}$, and $\mathbb{C}$ can be identified with $\mathbb{R}^{2}$.

The usual Euclidean topology is defined on $\mathbb{R}^{2}$.

Here, a subset $U$ of $S^{1}$ is an open set if there exists an open set $V$ of $\mathbb{R}^{2}$ such that $U=S^{1} \cap V$.

	${}$

\ \textbullet 1. The empty set and the total set are open sets.

$\varnothing = S^{1} \cap \varnothing$, and since $\varnothing$ is an open set in $\mathbb{R}^{2}$,
$\varnothing$ is an open set in $S^{1}$.

$S^{1} = S^{1} \cap \mathbb{R}^{2}$, and since $\mathbb{R}^{2}$ is an open set of $\mathbb{R}^{2}$ itself,
$S^{1}$ is an open set in $S^{1}$.

	${}$

\ \textbullet 2. The intersection of a finite number of open sets is an open set.

Consider the open sets of $S^{1}$,
$U_{1}, U_{2} , \dots , U_{n}$.

By definition, these can each be expressed as the intersection with open sets of $\mathbb{R}^{2}$,
$V_{1}, V_{2} , \dots , V_{n}$ ($U_{i} = S^{1} \cap V_{i}$).

\[
	\begin{array}{rcl}
		U_{1} \cap U_{2} \cdots \cap U_{n}
		 & = &
		(S^{1} \cap V_{1}) \cap (S^{1} \cap V_{2}) \cap \cdots (S^{1} \cap V_{n})
		\\
		 & = &
		S^{1} \cap ( V_{1} \cap  V_{2} \cap \cdots \cap V_{n})
	\end{array}
\]

Here, by the axioms of topology in $\mathbb{R}^{2}$,
$V_{1} \cap  V_{2} \cap \cdots \cap V_{n}$
is an open set in $\mathbb{R}^{2}$.

Therefore,
$U_{1} \cap U_{2} \cdots \cap U_{n}$
is an open set in $S^{1}$.


	${}$

\ \textbullet 3. The union of an arbitrary number of open sets is an open set.

Consider a family of open sets of $S^{1}$, $\{ U_{\alpha} \}$.

These can each be expressed as the intersection with open sets of $\mathbb{R}^{2}$,
$V_{\alpha}$ ($U_{\alpha}=S^{1} \cap V_{\alpha}$).

Therefore, the union of the family of open sets is
\[
	\bigcup_{\alpha} U_{\alpha}
	=
	\bigcup_{\alpha} (S^{1} \cap V_{\alpha})
	=
	S^{1} \cap \left( \bigcup_{\alpha} V_{\alpha} \right)
\]

Here, by the axioms of topology in $\mathbb{R}^{2}$,
the union of an arbitrary number of open sets
$\bigcup_{\alpha} V_{\alpha}$
is an open set in $\mathbb{R}^{2}$.

Therefore,
the union of an arbitrary number of open sets of $S^{1}$,
$\bigcup_{\alpha} U_{\alpha}$, is an open set in $S^{1}$.

	${}$

Through these three verifications, it has been confirmed that $S^{1}$ is a topological space.

	${}$

Furthermore, let $z_{1},z_{2} \in S^{1}$ be defined as

\[
	z_{i} = \cos \theta_{i} + i \sin \theta_{i} \ \ (i=1,2)
\]
Then $z_{1} * z_{2} \in S^{1}$, the identity element is 1, and the inverse of $z$ is $\cos \theta_{i} - i \sin \theta_{i} \in S^{1}$.

Since multiplication is commutative, $S^{1}$ is an abelian group.

\subsubsection{The Real Numbers}

$\mathbb{R}$ is a topological group with respect to addition.

\subsubsection{The Positive Real Numbers}

$\mathbb{R}^{+} = \{ x \in \mathbb{R} | x > 0 \}$
is a topological group with respect to multiplication.

$\mathbb{R}$ and $\mathbb{R}^{+}$ are isomorphic as topological groups: $\mathbb{R} \cong \mathbb{R}^{+}$

In fact,
$f: \mathbb{R} \ni x \to e^{x} \in \mathbb{R}^{+}$
and
$g: \mathbb{R}^{+} \ni x \to \log(x) \in \mathbb{R}$
are
continuous homomorphisms that satisfy
$gf=1_{\mathbb{R}}$ and
$fg=1_{\mathbb{R}^{+}}$.


\subsubsection{The Continuous Endomorphisms of the Topological Module of All Real Numbers}

When $\mathbb{R}$ is the topological module of all real numbers,
a continuous endomorphism
$f: \mathbb{R} \to \mathbb{R}$
can be expressed as $f(x) = ax$ for some $a \in \mathbb{R}$.

	${}$

(Verification) (Algebraic procedure)

${}$

Since $f$ is a homomorphism, $f(x+y)=f(x)+f(y)$.
If we set $f(1)=a$, then by induction for integers $n,m$,
we get $f(\frac{n}{m}) = \frac{n}{m} a$.

From this algebraic procedure, we know that for a rational number $r$, $f(r) = ar$.

	${}$

(Continuation of Verification) (Topological procedure)

${}$

Since any $x \in \mathbb{R}$ can be expressed using a rational number $r_{n}$ as
$\displaystyle \lim_{n \to \infty} r_{n} = x$,
by the continuity of $f$,
$\displaystyle f(x) = f( \lim_{n \to \infty} r_{n}) = \lim_{n \to \infty} a r_{n} = ax$.

	${}$

With the above, it has been verified that a continuous endomorphism
$f: \mathbb{R} \to \mathbb{R}$
of the topological module $\mathbb{R}$ of all real numbers can be expressed as $f(x) = ax$ for some $a \in \mathbb{R}$.
The resulting function is differentiable.

The assumption that the map $f$ is a homomorphism is a group-theoretic (algebraic) assumption,
and the assumption that it is continuous is a topological assumption.

By including an algebraic assumption and a topological assumption, we find that $f$ becomes differentiable, even though we did not include any analytical assumptions.

Where did this analytical information of differentiability come from?

\paragraph{Generalization}

${}$

The content of this example can be generalized as:

``If $G,G'$ are Lie groups and $f: G \to G'$ is a continuous homomorphism, then $f$ is differentiable."

\paragraph{Further Generalization (Hilbert's Fifth Problem)}

${}$

Hilbert's Fifth Problem:
``Is a topological group $G$ that is also a topological manifold a Lie group?"

This problem also derives an analytical property from algebraic and topological properties.

This problem was solved affirmatively in 1952 by
A.M. Gleason, D. Montgomery, L. Zippin, and H. Yamabe.


\end{document}
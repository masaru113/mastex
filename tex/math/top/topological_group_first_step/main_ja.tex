\documentclass[uplatex,a4j,12pt,dvipdfmx]{jsarticle}
\usepackage[english]{babel}
\usepackage[letterpaper,top=2cm,bottom=2cm,left=3cm,right=3cm,marginparwidth=1.75cm]{geometry}
\usepackage{amsmath, amssymb}
\usepackage{graphicx}
\usepackage[colorlinks=true, allcolors=blue]{hyperref}
\usepackage{fancybox}
\usepackage{tikz-cd}
\title{
位相群のことはじめ
}

\author{
mastex
}

\begin{document}
\maketitle


\begin{abstract}
	位相群の定義と例
\end{abstract}

\section{位相群}

群と位相空間の構造を合わせて持つ位相群を定義する。

集合 $G$ が位相群であるとは、

\begin{enumerate}
	\item $G$ は群である。
	\item $G$ は位相空間である。
	\item 写像 $\mu: G \times G \to G, \nu: G \to G$ をそれぞれ $\mu(x,y)=xy, \nu(x) = x^{-1}$ と定義するとき、$\mu,\nu$ は連続である。
\end{enumerate}

最後の条件は、通常の群の条件の積と逆元について、位相空間の連続性を用いた定義になっている。
積と逆元が連続写像で送っても保たれるというステートメントである。


もし $G$ の位相が距離空間から導かれている場合、最後の条件は

\[
	\left\{
	\begin{array}{r}
		\displaystyle \lim_{ m \to \infty } x_{m} \\
		\displaystyle \lim_{ m \to \infty } y_{m}
	\end{array}
	\right.
	\ \Rightarrow \
	\left\{
	\begin{array}{r}
		\displaystyle \lim_{m \to \infty } x_{m} y_{m} = xy \\
		\displaystyle \lim_{m \to \infty } x_{m}^{-1} = x^{-1}
	\end{array}
	\right.
\]

と言い換えられる。
距離が入っていれば積と逆元が極限でも保存されるという条件になる。





\subsection{位相群の例}

\subsubsection{円周}

円周 $S^{n} (n=0,1)$ は位相群になるが、球面 $S^{n} (n \neq 3)$ は位相群にならないことが知られている。
(E. Cartan によって示された。)

\begin{enumerate}
	\item $S^{0} = \{ -1, 1 \}$
	\item $S^{1} = \{ x \in \mathbb{C} \ | \ |x| = 1 \}$
	\item $S^{3} = \{ x \in \mathbb{H} \ | \ |x| = 1 \}$
\end{enumerate}

はいずれも積に関して位相群になる。


\paragraph{$S^{0}$ が積に関して位相群になることの確認}

${}$

(群であることの確認)

$$1*1=1 \in S^{0}$$
$$1*(-1)=(-1) \in S^{0}$$
$$(-1)*1=(-1) \in S^{0}$$
$$(-1)*(-1)=1 \in S^{0}$$

積に関して閉じていて、それぞれの逆元も $S^{0}$ に入っている。

${}$

(位相空間であることの確認)

$S^{0}$ の部分空間は
$$\varnothing, \{ -1 \} , \{ 1 \} , \{ -1 , 1 \}$$
の4つであり、いずれも開集合とする。

1. 全体 $S^{0}$ と空集合 $\varnothing$ は開集合である。

2. いずれの開集合の有限個の共通部分は開集合になる。
例えば、
$$\varnothing \cap \{ -1 \} = \varnothing$$
$$\varnothing \cap \{ 1 \} = \varnothing$$
$$\varnothing \cap S^{0} = \varnothing$$
$$ \{ -1 \} \cap \{ 1 \} = \varnothing$$
$$ \{ -1 \} \cap S^{0} = \{ -1 \} $$
$$ \{ 1 \} \cap S^{0} = \{ 1 \} $$
$$ \{ -1 \} \cap \{ 1 \} \cap S^{0} = \varnothing $$

3. いずれの開集合の共通部分もまた開集合になる。
例えば、
$$\varnothing \cup \{ -1 \} = \{ -1 \}$$
$$\varnothing \cup \{ 1 \} = \{ 1 \}$$
$$\varnothing \cup S^{0} = S^{0}$$
$$ \{ -1 \} \cup \{ 1 \} = \{ -1 , 1 \}$$
$$ \{ -1 \} \cup S^{0} = S^{0} $$
$$ \{ 1 \} \cup S^{0} = S^{0} $$
$$ \{ -1 \} \cup \{ 1 \} \cup S^{0} = S^{0} $$

以上から $S^{0}$ は開集合になる。

${}$

\paragraph{$S^{1}$ が積に関して位相群になることの確認}

${}$

$S^{1}$ は $\mathbb{C}$ の部分集合であり、$\mathbb{C}$ は $\mathbb{R}^{2}$ と同一視できる。

$\mathbb{R}^{2}$ には通常のユークリッド位相が定義されている。

ここで、$S^{1}$ の部分集合 $U$ が開集合であるのは、ある $\mathbb{R}^{2}$ の開集合 $V$ が存在して、 $U=S^{1} \cap V$ となるときである。

${}$

\ ・1. 空集合と全体集合は開集合である。

$\varnothing = S^{1} \cap \varnothing$ で、$\varnothing$ は $\mathbb{R}^{2}$ の開集合なので
$\varnothing$ は $S^{1}$ の開集合である。

$S^{1} = S^{1} \cap \mathbb{R}^{2}$ で、$\mathbb{R}^{2}$ は $\mathbb{R}^{2}$ 自身の開集合なので
$S^{1}$ は $S^{1}$ の開集合である。

${}$

\ ・2. 有限個の開集合の共通部分は開集合である。

$S^{1}$ の開集合
$U_{1}, U_{2} , \dots , U_{n}$
を考える。

定義から、これらはそれぞれ $\mathbb{R}^{2}$ の開集合
$V_{1}, V_{2} , \dots , V_{n}$
との共通部分で表せる($U_{i} = S^{1} \cap V_{i}$)。

\[
	\begin{array}{rcl}
		U_{1} \cap U_{2} \cdots \cap U_{n}
		 & = &
		(S^{1} \cap V_{1}) \cap (S^{1} \cap V_{2}) \cap \cdots (S^{1} \cap V_{n})
		\\
		 & = &
		S^{1} \cap ( V_{1} \cap  V_{2} \cap \cdots \cap V_{n})
	\end{array}
\]

ここで $\mathbb{R}^{2}$ の位相の公理より、
$V_{1} \cap  V_{2} \cap \cdots \cap V_{n}$
は $\mathbb{R}^{2}$ の開集合である。

よって、
$U_{1} \cap U_{2} \cdots \cap U_{n}$
は $S^{1}$ の開集合となる。


${}$

\ ・3. 任意の個数の開集合の和集合は開集合である。

$S^{1}$ の開集合の族 $\{ U_{\alpha} \}$ を考える。

これらはそれぞれ $\mathbb{R}^{2}$ の開集合 $V_{\alpha}$ との共通部分で表せる
($U_{\alpha}=S^{1} \cap V_{\alpha}$)
。

よって開集合の族の和集合は
\[
	\bigcup_{\alpha} U_{\alpha}
	=
	\bigcup_{\alpha} (S^{1} \cap V_{\alpha})
	=
	S^{1} \cap \left( \bigcup_{\alpha} V_{\alpha} \right)
\]

ここで $\mathbb{R}^{2}$ の位相の公理より、
任意の個数の開集合の和集合
$\bigcup_{\alpha} V_{\alpha}$
は $\mathbb{R}^{2}$ の開集合である。

よって、
$S^{1}$ の任意の個数の開集合の和集合
$\bigcup_{\alpha} U_{\alpha}$ は $S^{1}$ の開集合である。

${}$

以上の3つの確認により、$S^{1}$ は位相空間になることが確認できた。

${}$

さらに $z_{1},z_{2} \in S^{1}$ を

\[
	z_{i} = \cos \theta_{i} + i \sin \theta_{i} \ \ (i=1,2)
\]
で定義する。

すると $z_{1} * z_{2} \in S^{1}$ であり、単位元は 1 、
$z$ の逆元は $\cos \theta_{i} - i \sin \theta_{i} \in S^{1}$ である。

乗法が可換なので$S^{1}$はアーベル群になる。

\subsubsection{実数}

$\mathbb{R}$は和に関して位相群になる。

\subsubsection{正の実数}

$\mathbb{R}^{+} = \{ x \in \mathbb{R} | x > 0 \}$
は積に関して位相群になる。

$\mathbb{R}$ と $\mathbb{R}^{+}$ は位相群として同型である:$\mathbb{R} \cong \mathbb{R}^{+}$

実際、
$f: \mathbb{R} \ni x \to e^{x} \in \mathbb{R}^{+}$
と
$g: \mathbb{R}^{+} \ni x \to \log(x) \in \mathbb{R}$
は
$gf=1_{\mathbb{R}}$、
$fg=1_{\mathbb{R}^{+}}$
を満たす連続な準同型写像になる。


\subsubsection{位相加群としての実数全体の自己準同型写像}

$\mathbb{R}$ を実数全体のつくる位相加群としたとき、
連続な自己準同型写像
$f: \mathbb{R} \to \mathbb{R}$
はある $a \in \mathbb{R}$ を用いて、$f(x) = ax$ と表される。

${}$

(確認)(代数的手続き)

${}$

$f$ は準同型なので $f(x+y)=f(x)+f(y)$ であり、
$f(1)=a$ と置くと、
整数 $n,m$ に対して帰納法を用いると
$f(\frac{n}{m}) = \frac{n}{m} a$ となる。

以上の代数的な手続きから有理数 $r$ に対して $f(r) = ar$ となることが分かる。

${}$

(確認の続き)(位相的手続き)

${}$

任意の $x \in \mathbb{R}$ は有理数 $r_{n}$ を用いて
$\displaystyle \lim_{n \to \infty} r_{n} = x$
と表されるので、$f$ の連続性より、
$\displaystyle f(x) = f( \lim_{n \to \infty} r_{n}) = \lim_{n \to \infty} a r_{n} = ax$

${}$

以上で実数全体のつくる位相加群 $\mathbb{R}$ の自己準同型写像
$f: \mathbb{R} \to \mathbb{R}$
はある $a \in \mathbb{R}$ を用いて、$f(x) = ax$ と表されることが確認できた。
得られた関数は微分可能になっている。

写像 $f$ が準同型であることは群論的(代数的)仮定であり、
連続であることは位相的仮定である。

代数的仮定と位相的仮定を入れると、解析的な仮定を入れていないのにもかかわらず、$f$ は微分可能になる。

一体どこから微分可能という解析的な情報が生み出されたのだろうか。

\paragraph{一般化}

${}$

この例の内容は

「$G,G'$ をLie群とし、 $f: G \to G'$ を連続な準同型写像とすると、$f$ は微分可能である」

と一般化される。

\paragraph{さらなる一般化(Hilbertの第5の問題)}

${}$

Hilbertの第5の問題
「位相多様体である位相群 $G$ はLie群か?」。

この問題も同様に代数的な性質と位相的な性質から解析的性質を導き出す問題である。

この問題は
A.M. Gleason, D. Montgomery, L. Zippin, H. Yamabe
らにより1952年に肯定的に解決された。


\end{document}
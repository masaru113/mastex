\documentclass{article}
\usepackage[english]{babel}
\usepackage[letterpaper,top=2cm,bottom=2cm,left=3cm,right=3cm,marginparwidth=1.75cm]{geometry}
\usepackage{amsmath, amssymb}
\usepackage{graphicx}
\usepackage[colorlinks=true, allcolors=blue]{hyperref}
\title{
Sierpinski Topological Space
}

\author{
Masaru Okada
}

\begin{document}
\maketitle

I was thinking about spaces that are compact but not closed, and I came across this example known as a Sierpinski topological space. It's such an interesting space, I wanted to share this knowledge.

\begin{abstract}
	In topology, open sets are a fundamental concept for studying the properties of a space. However, open sets can sometimes be abstract and difficult to handle. In this paper, we will see how open sets in a topological space correspond to continuous maps to the Sierpinski space $\mathbf{S}_ {2}$, which is the simplest non-trivial topological space consisting of just two points $\{0, 1\}$ and three open sets $\{\emptyset, \{0\}, \{0, 1\}\}$.

	The Sierpinski space has the remarkable property that the open sets of any topological space $X$ correspond one-to-one with the continuous maps into this space $\mathbf{S}_ {2}$. This correspondence makes it possible to translate the complex concept of open sets into the more manageable concept of continuous maps.
\end{abstract}

\section{What is a Topological Space?}

The term "topological space" is a bit confusing.

The first time I encountered the term "topological space" was as a student, in the context of physics.
This is also called a phase space,
which is a space where generalized coordinates $x$ and their time derivatives $\frac{dx}{dt} = \dot{x}$ from analytical mechanics are chosen as the coordinate axes.
For example, the Hamiltonian for a common harmonic oscillator with mass $m$ and constant $k$
$$
	H(x,\dot{x}) = \frac{1}{2}kx^{2} + \frac{\dot{x}^{2}}{2m}
$$
The trajectory of the possible states in the phase space is an ellipse.
In short, the phase space mentioned here is just a special name given to a Euclidean space in mathematics.

On the other hand, after entering the workforce and studying mathematics, I came across the term "topological space."
This is a very abstract concept not limited to Euclidean spaces, and it seems to be a collection of sets called open sets.
This concept of space is something I couldn't have imagined when I was doing physics. For someone familiar with physics, it might be more accurate to think of it as a special kind of logical structure (topo"logical" space) rather than a space.
It seems that people who study mathematics tend to use the term "space" rather liberally, even for these special logical structures.


\subsection{Definition of a Topological Space}

One way to define a topological space is by using open sets.
(There are other equivalent ways to define it, such as using closed sets or the closure operator.)

To give a really rough definition of a topology, for subsets $U,V$ of a set $X$:

\begin{enumerate}
	\item The intersection of two subsets is also a subset ($U \cap V$)
	\item The union of subsets is also a subset ($U \cup V$)
	\item The whole set and the empty set are also considered subsets ($X$, $\varnothing$)
\end{enumerate}

A collection of subsets that satisfies these three conditions is called a topology, and the pair with the original set is called a topological space.

An open set is intuitively an "open" set that does not contain its boundary. In the special case of Euclidean space, which is a specific type of topological space, defining open sets using open intervals is a good, easy-to-handle example.

${}$

To proceed with the discussion, let's provide a slightly more formal definition.

Let $P(X)$ denote the power set of a set $X$, which is the set of all subsets of $X$.
Thus, $P(X)$ is a set whose elements are (sub)sets (a set of sets).

A topology on $X$ is a subset $\mathcal{O}$ of $P(X)$ that satisfies the following three conditions for subsets $U,V$ of $X$:
\begin{enumerate}
	\item If $U,V \in {\mathcal O}$, then $U \cap V \in {\mathcal O}$
	\item For a set $I$, and for each element $i \in I$, if an open set $U_{i} \in {\mathcal O}$ is given,
	      their union is also an open set. That is, $\bigcup_{i} U_{i} \in {\mathcal O}$
	\item The whole set $X \subseteq X$ is an open set ($X \in {\mathcal O}$), and the empty set is also an open set ($\varnothing \in {\mathcal O}$)
\end{enumerate}

A collection of subsets of $X$ that satisfies these three conditions, denoted by $\mathcal{O}$, is called a topology, and the pair $(X, {\mathcal O})$ is called a topological space.

\subsection{Definitions of Common Terms in Topological Spaces}

Topology is full of so many terms that it can be overwhelming for beginners. I'll define just a few of the terms that will be used in the following discussion.

\subsubsection{Closed Set}

A closed set is the complement of an open set.
The complement of the empty set is the whole set, and the complement of the whole set is the empty set, so the whole set and the empty set are also closed sets.

This means the whole set and the empty set are both open and closed sets.


\subsubsection{Metric Space}

If we introduce the structure of a distance into a topological space, it's called a metric space. It's a slightly more concrete space.

Conversely, when we talk about a general topological space, it's assumed that the structure of a distance is not included.
Since it doesn't have a distance structure, a topological space itself doesn't have a geometric image.

${}$

For example, if we take the set $X$ to be the set of all real numbers $\mathbb{R}$ and introduce the (standard, everyday) distance structure $d(x,y) = |x-y|$, it becomes a metric space.

\subsubsection{Continuous}

A continuous map between two topological spaces $(X, {\mathcal O}_{X}), (Y, {\mathcal O}_{Y})$ is a function
$f: X \to Y$ such that for every $U \in {\mathcal O}_{Y}$, its preimage
$f^{-1}(U)$ belongs to ${\mathcal O}_{X}$.

This concept of continuity often differs from the concept of continuity we are familiar with in Euclidean spaces.

In a metric space, which is a topological space with a distance concept, continuity is what we would intuitively think of as "continuous." However, in a general topological space, continuity has the abstract meaning of "preserving the properties of the topological space."

A map $f: X \to Y$ that is continuous and whose inverse $f^{-1}: Y \to X$ is also continuous is called a homeomorphism.
A homeomorphism is a morphism in the category of topological spaces, $\mathbf{Top}$.

Indeed, there is an identity morphism ${\rm id}_{X}: X \to X$, and for $f: X \to Y, \ g: Y \to Z$, we have the composition $g f : X \to Z$.


\subsubsection{Strength of a Topology}

For two topological spaces $(X, {\mathcal O}_{X}), (Y, {\mathcal O}_{Y})$, if
${\mathcal O}_{X} \subset {\mathcal O}_{Y}$
holds, we say that
${\mathcal O}_{X}$ is weaker than ${\mathcal O}_{Y}$,
or ${\mathcal O}_{Y}$ is stronger than ${\mathcal O}_{X}$.

Instead of "stronger/weaker," the terms "finer/coarser" are also sometimes used.


\subsubsection{Discrete Topology}

For any set $X$, its power set $P(X)$ satisfies the conditions for a topology on $X$.

This topology is called the discrete topology.

The discrete topology is the finest, or strongest, topology.


\subsubsection{Indiscrete Topology}

For any set $X$, the set containing only the empty set and the set itself, $\{ \varnothing, X \}$, also satisfies the conditions for a topology on $X$.

This topology is called the indiscrete topology.

The indiscrete topology is the coarsest, or weakest, topology.

It is also called the trivial topology.

In mathematics, the term "trivial" often implies "uninteresting to investigate."

${}$

For a beginner, it's not trivial whether the indiscrete topology satisfies the conditions of a topology, so let's check it.

\ 1. Let's confirm that the intersection of subsets is also a subset.

$\varnothing \cap X = \varnothing \in \{ \varnothing, X \}$
So this is OK.

\ 2. Let's confirm that the union of subsets is also a subset.

$\varnothing \cup X = X \in \{ \varnothing, X \}$
So this is also OK.

\ 3. Let's confirm that the whole set and the empty set are included.

$\varnothing \in \{ \varnothing, X \} , \ X \in \{ \varnothing, X \}$
So this is also OK.

Therefore, we have confirmed that the set consisting of the empty set and the set itself, $\{ \varnothing, X \}$, also satisfies the conditions for a topology on $X$.



\subsubsection{Covering}

For an open set $U$, a collection $(U_{i})_{i \in I}$ such that $\bigcup_{i} U_{i} = U$ is called a covering of $U$.


\subsubsection{Basis}

A basis is a subset $S$ of the power set $P(X)$ of a set $X$
(this $S$ is also a collection of subsets)
such that
there exists the weakest topology ${\mathcal O}_{S}$
that satisfies $S \subset {\mathcal O}_{S}$.

This ${\mathcal O}_{S}$ is
called the weakest topology containing $S$.

Also, we say that $S$ generates ${\mathcal O}_{S}$.

\subsubsection{Generating a Topology}

For a basis $S$ to generate a topology $\mathcal{O}_S$, it means that every element (open set) of $\mathcal{O}_S$ can be constructed using the elements of $S$.
Specifically, any open set $O$ in $\mathcal{O}_S$ can be represented as the union of some elements $B_i$ from $S$.

$$O = \bigcup_{i \in I} B_i, \quad B_i \in S$$

Because of this property, the basis $S$ serves as a fundamental building block for "patching together" (covering) the topology $\mathcal{O}_S$.

In other words, the entire topology $\mathcal{O}_S$ can be defined just by taking unions of elements from the basis $S$.
This is very convenient because it allows us to define a topology by specifying a smaller family of sets $S$, instead of having to list every single open set in the topology.

\subsubsection{Conditions for a Basis}

Not every family of sets $S$ can be a basis for a topology.
For a basis $S$ to generate a topology $\mathcal{O}_S$, it must satisfy the following two conditions.

\begin{enumerate}
	\item The set $X$ must be expressible as a union of elements from $S$.
	\item The intersection of any two elements $B_1, B_2$ of $S$ must be expressible as a union of some elements of $S$.
\end{enumerate}

Formally:

\begin{enumerate}
	\item $X = \bigcup_{B \in S} B$ : This is a condition to satisfy the definition of a topology ($X \in \mathcal{O}_S$).
	\item $B_1 \cap B_2 = \bigcup_{B_j \in S} B_j$ : This is a condition to satisfy the definition of a topology (the finite intersection of open sets is an open set).
\end{enumerate}

If a family of sets $S$ satisfies these two conditions, a topology $\mathcal{O}_S$ can be constructed from it, and it will be \textbf{unique}.

\subsubsection{Compact}

The term "compact" is completely different from its everyday use.
It is a generalization of the properties of bounded and closed sets in $n$-dimensional Euclidean space $\mathbb{R}^{n}$.
It is one of the properties of a topological space.

There are two equivalent types of compactness:
\begin{enumerate}
	\item Bolzano-Weierstrass property
	\item Heine-Borel property
\end{enumerate}


\subsubsection{Compactness via Bolzano-Weierstrass}

A directed set is a set with an order defined on it, such that for any two elements in the set, there is always an element that is greater than or equal to both.
For example, if we take two elements, say 3 and 7, from the set of natural numbers $\mathbb{N}$, we can say $3 \leq 7$. For any two elements, not just 3 and 7, a magnitude relation including equality can be defined. Such a set is called a directed set, a set with an order defined.

A net is a set of elements from a directed set.
The definition of compactness by Bolzano-Weierstrass is stated as follows.

For any net $(x_{\lambda})_{\lambda \in \Lambda}$ on $X$,
there exists a subnet $(x_{\gamma})_{\gamma \in \Gamma}$ of $(x_{\lambda})_{\lambda \in \Lambda}$ that converges to some $x \in X$.

${}$

It becomes easier to imagine if we restrict $X \subseteq \mathbb{R}^{n}$.

A space $X$ is said to be compact if every sequence of points on $X$ has a convergent subsequence.

${}$

When we are studying abstract concepts like topological spaces, it's often difficult to think at such a high level of abstraction.
We can make it easier to imagine by restricting to $X \subseteq \mathbb{R}^{n}$,
but we must be mindful that this is a very dangerous specialization.


\subsubsection{Compactness via Heine-Borel}

A space $X$ is said to be compact if for any open cover $S$ of $X$, there exists a finite subcover $T$ of $S$ that also covers $X$.

In other words, a set is said to be Heine-Borel compact if it can be covered by a finite number of open sets.

\subsubsection{Hausdorff}

Felix Hausdorff is the name of one of the founders of topology, and "Hausdorff" also refers to an axiom named after him, as well as the property of a space that satisfies it.

A neighborhood of an element $p \in X$ in a topological space $X$ is a subset that contains "an open set that contains $p$." (Therefore, a neighborhood itself doesn't have to be an open set. A neighborhood that is specifically an open set is called an open neighborhood.)

A topological space $X$ is said to be Hausdorff, or to have the Hausdorff property, if for any two distinct elements $p,q$, there always exist an open neighborhood $U$ of $p$ and an open neighborhood $V$ of $q$ such that $U \cap V = \varnothing$.
A topological space with the Hausdorff property is called a Hausdorff space, or simply Hausdorff.

It is said that a topological space that is both compact and Hausdorff is very easy to work with, and is simply called "compact Hausdorff."

${}$

For example, for a continuous map $f: X \to Y$, if $X$ is compact and Hausdorff, then:

\ ・ $Y$ is Hausdorff.

\ ・ $f$ is a closed map.

\ ・ ${\rm ker} f $ is a closed set.



\section{Examples of Topological Spaces}

\subsection{Example of a Bounded and Closed Set that is not Compact}

The topological space formed by the set of all integers $\mathbb{Z}$ with the discrete metric

\begin{align*}
	d(x, x) & = 0                             \\
	d(x, y) & = 1 \ \ \ ({\rm if} \ x \neq y)
\end{align*}

is a space where every subset is bounded and closed, but it is not compact.


\subsection{Example of a Compact Set that is not Bounded and Closed: The Sierpinski Topological Space}

Consider the following topological space $(X, \mathcal{O})$.
$$X = \{0, 1\}$$
$$\mathcal{O} = \{\varnothing, \{0\}, X\}$$
In this space, $\{0\}$ is a compact set in $X$, but it is not a closed set.

\subsubsection{Verification that it is a topological space}

Let's check the three conditions for being a topological space.
\begin{enumerate}
	\item $\varnothing \in \mathcal{O}$ and $X \in \mathcal{O}$
	\item The union of any open sets in $\mathcal{O}$ is an element of $\mathcal{O}$.
	\item The intersection of a finite number of open sets in $\mathcal{O}$ is an element of $\mathcal{O}$.
\end{enumerate}

\subsubsection{Condition 1 (Containing the whole set and the empty set)}
By definition, $\varnothing$ and $X$ are elements of $\mathcal{O}$.

\subsubsection{Condition 2 (The union is also a subset)}
Consider the unions of elements in $\mathcal{O}$.
\begin{itemize}
	\item $\varnothing \cup \{0\} = \{0\} \in \mathcal{O}$
	\item $\varnothing \cup X = X \in \mathcal{O}$
	\item $\{0\} \cup X = X \in \mathcal{O}$
	\item $\varnothing \cup \{0\} \cup X = X \in \mathcal{O}$
\end{itemize}
Thus, Condition 2 is satisfied.

\subsubsection{Condition 3 (The intersection is also a subset)}
Consider the intersections of elements in $\mathcal{O}$.
\begin{itemize}
	\item $\varnothing \cap \{0\} = \varnothing \in \mathcal{O}$
	\item $\varnothing \cap X = \varnothing \in \mathcal{O}$
	\item $\{0\} \cap X = \{0\} \in \mathcal{O}$
	\item $\varnothing \cap \{0\} \cap X = \varnothing \in \mathcal{O}$
\end{itemize}
Thus, Condition 3 is also satisfied.

From the above, $(X, \mathcal{O})$ is a topological space.

\subsubsection{Verification that \{0\} is compact}
The definition of compact is that any open cover has a finite subcover.
\begin{itemize}
	\item $ \{ 0\} \subseteq \varnothing \cup \{ 0 \}$
	\item $ \{ 0\} \subseteq \varnothing \cup X$
	\item $ \{ 0\} \subseteq \{ 0 \} \cup X$
	\item $ \{ 0\} \subseteq \varnothing \cup \{ 0 \} \cup X$
\end{itemize}

Thus, $\{0\}$ is a compact set in $X$.


\subsubsection{Proof that \{0\} is not a closed set}

The definition of a closed set is that its complement is an open set.
The complement of $\{0\}$ is $X \setminus \{0\} = \{1\}$, and this complement $\{1\}$ is not an element of the topology $\mathcal{O} = \{\varnothing, \{0\}, X\}$.
Therefore, $\{0\}$ is not a closed set.

${}$

From the above, for the topological space
$$X = \{0, 1\} , \ \ \mathcal{O} = \{\varnothing, \{0\}, X\}$$
the set $\{0\}$ is a compact set in $X$, but it is not a closed set.


\subsection{The Sierpinski Topological Space}

This two-point topological space
$$X = \{0, 1\} , \ \ \mathcal{O}_{0} = \{\varnothing, \{0\}, X\}$$
is called the Sierpinski topological space.

For example, the following is also a Sierpinski topological space. (That is, they are isomorphic.)
$$X = \{0, 1\} , \ \ \mathcal{O}_{1} = \{\varnothing, \{1\}, X\}$$

% The following two are also Sierpinski topological spaces.
% $$X = \{0, 1\} , \ \ \mathcal{O}_{2} = \{ \{0, 1\}, \mathcal{O}_{0} \}$$
% $$X = \{0, 1\} , \ \ \mathcal{O}_{3} = \{ \{0, 1\}, \mathcal{O}_{1} \}$$


\section{The Sierpinski Topology and Category Theory}

For a topological space $(X, \mathcal{O})$, the topology $\mathcal{O}$ has a pre-order structure $(\mathcal{O}, \subseteq )$ using the subset relation $\subseteq$.

In general, a pre-order can be viewed as a category with the sets as objects and the order relations as morphisms, so $(\mathcal{O}, \subseteq )$ forms a category.

The Hasse diagram of the Sierpinski topology is simple, forming the following short sequence.
$$
	\emptyset \xrightarrow{\subseteq} \{0\} \xrightarrow{\subseteq} \{0, 1\}
$$

\subsection{Correspondence between Open Sets and Continuous Maps}

In this topological space, a continuous map for any open set $U \subseteq \mathcal{O}$ is the characteristic function $\chi_{U}: X \to \mathbf{S}_ {2}$, defined as follows.
Here, $\mathbf{S}_ {2}$ is the space formed by the two-point set $\{0,1\}$ with the Sierpinski topology.

\begin{align*}
	\chi_{U}(x) =
	\begin{cases}
		1 & (x \in U)    \\
		0 & (x \notin U)
	\end{cases}
\end{align*}

In this topological space, any continuous map $f: X \to \mathbf{S}_ {2}$ can be expressed as $\chi_{U}$ for $U = f^{-1}(1)$.

This shows that there is a one-to-one correspondence between any open set $U \subseteq \mathcal{O}$ and a continuous map $X \to \mathbf{S}_ {2}$.

This is a universal property of the Sierpinski topological space, and it means that in the category of topological spaces $\mathbf{Top}$, $\mathbf{S}_ {2}$ plays the role of "testing" open sets.

In the language of category theory, this is expressed as
$$
	\mathrm{Hom}_{\mathbf{Top}}(X, \mathbf{S}_ {2}) \cong \mathcal{O}
$$
where $\mathrm{Hom}_{\mathbf{Top}}(X, \mathbf{S}_ {2})$ is the set of continuous maps from the topological space $X$ to $\mathbf{S}_ {2}$, which shows that it is isomorphic to the topology $\mathcal{O}$.


\subsection{\texorpdfstring{A more careful look: $\mathrm{Hom}_{\mathbf{Top}}(X, \mathbf{S}_ {2}) \cong \mathcal{O}$}{A more careful look: Hom(X, S2) is isomorphic to O}}

Let $(X, \mathcal{O})$ be the Sierpinski topological space, where $X = \{0, 1\}$ and $\mathcal{O} = \{\emptyset, \{0\}, X\}$.
Also, $\mathbf{S}_ {2}$ is the space formed by the two-point set $\{0, 1\}$ with the Sierpinski topology, and it is the same space as $(X, \mathcal{O})$.

Let's confirm the definitions and show that there is a one-to-one correspondence (bijection) between the set of continuous maps $\mathrm{Hom}_{\mathbf{Top}}(X, \mathbf{S}_ {2})$ and the set of open sets $\mathcal{O}$.

\subsubsection*{Definition of the map $\Phi: \mathrm{Hom}_{\mathbf{Top}}(X, \mathbf{S}_ {2}) \to \mathcal{O}$}

For any continuous map $f: X \to \mathbf{S}_ {2}$, we define
$$ \Phi(f) = f^{-1}(1) $$
\subsubsection*{Verification that $\Phi$ is well-defined}

By the definition of a continuous map, for $f: X \to \mathbf{S}_ {2}$ to be continuous, the preimage of the open set $\{1\} \in \mathcal{O}$ must be an open set of $X$, that is, an element of $\mathcal{O}$.
Therefore, $\Phi(f) \in \mathcal{O}$, and $\Phi$ is well-defined.

\subsubsection*{Proof that $\Phi$ is a bijection}

\begin{enumerate}
	\item \textbf{Injective}:
	      Assume that $f, g \in \mathrm{Hom}_{\mathbf{Top}}(X, \mathbf{S}_ {2})$ and $\Phi(f) = \Phi(g)$.
	      This means that $f^{-1}(1) = g^{-1}(1)$.
	      In this case, both maps take the value 1 on the set
	      $$ f^{-1}(1) = g^{-1}(1) $$
	      Also, both maps take the value 0 on the set
	      $$ X \setminus f^{-1}(1) = X \setminus g^{-1}(1) $$
	      Therefore, for all $x \in X$, $f(x) = g(x)$, and $f = g$.
	      Hence, $\Phi$ is injective.

	\item \textbf{Surjective}:
	      Take any open set $U \in \mathcal{O}$.
	      For this $U$, define the characteristic function $\chi_U: X \to \mathbf{S}_ {2}$ as follows:
	      $$ \chi_U(x) =
		      \begin{cases}
			      1 & (x \in U)    \\
			      0 & (x \notin U)
		      \end{cases}
	      $$
	      We will show that this $\chi_U$ is continuous.
	      The open sets of $\mathbf{S}_ {2}$ are $\emptyset, \{1\}, \{0, 1\}$.
	      \begin{itemize}
		      \item $\chi_U^{-1}(\emptyset) = \emptyset \in \mathcal{O}$
		      \item $\chi_U^{-1}(\{1\}) = U \in \mathcal{O}$
		      \item $\chi_U^{-1}(\{0, 1\}) = X \in \mathcal{O}$
	      \end{itemize}
	      Since the preimage of every open set is an open set of $X$, $\chi_U$ is a continuous map.
	      Furthermore, $\Phi(\chi_U) = \chi_U^{-1}(1) = U$.
	      Therefore, for any $U \in \mathcal{O}$, there exists a continuous map $\chi_U$ that satisfies $\Phi(\chi_U) = U$, so $\Phi$ is surjective.
\end{enumerate}

Since it is both injective and surjective, $\Phi$ is a bijection.
Therefore, as sets, $\mathrm{Hom}_{\mathbf{Top}}(X, \mathbf{S}_ {2})$ and $\mathcal{O}$ correspond one-to-one.
$$
	\mathrm{Hom}_{\mathbf{Top}}(X, \mathbf{S}_ {2}) \cong \mathcal{O}
$$
This isomorphism shows that the Sierpinski space $\mathbf{S}_ {2}$ plays the role of a "classifying space" in the category of topological spaces $\mathbf{Top}$, encoding information about open sets.

${}$

Open sets of a topological space are usually treated as a collection of subsets.
However, this isomorphism transforms the static concept of an open set (a subset) into the dynamic concept of a continuous map (a morphism).

This means we can analyze problems concerning open sets using methods from category theory.

\end{document}
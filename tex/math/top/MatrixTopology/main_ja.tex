\documentclass[uplatex,a4j,12pt,dvipdfmx]{jsarticle}
\usepackage[english]{babel}
\usepackage[letterpaper,top=2cm,bottom=2cm,left=3cm,right=3cm,marginparwidth=1.75cm]{geometry}
\usepackage{amsmath, amssymb}
\usepackage{graphicx}
\usepackage[colorlinks=true, allcolors=blue]{hyperref}
\usepackage{fancybox}
\usepackage{tikz-cd}
\title{
行列の位相
}

\author{
mastex
}

\begin{document}
\maketitle


\begin{abstract}
	(概要作成中。あとで記載)
\end{abstract}

\section{行列の集合}

位相空間はある特定の条件を満たす開集合のあつまりのことを指すが、行列の集合もまた位相空間となることがある。
開集合という抽象的な概念が行列にも導入できる。

\paragraph{$M_{n}(K)$ : $K$ 上の $n$次正方行列の集合}

${}$

$M_{n}(K)$ を全ての成分が体 $K$ の元である $n \times n$ 行列がなす集合とする。

この集合は、行列の加法とスカラー倍についてベクトル空間の構造を持つ。

この集合には行列式がゼロとなる特異行列も含まれる。

\paragraph{$GL_{n}(K)$ : 一般線形群}

${}$

$GL_{n}(K)$ を全ての成分が体 $K$ の元である $n$ 次正則行列がなす集合とする。

$M_{n}(K)$ と異なり、$GL_{n}(K)$ には行列式が0となるような行列(特異行列)は含まれない。

$GL_{n}(K)$ は行列の乗法で群になる。

\paragraph{$SL_{n}(K)$ : 特殊線形群}

${}$

$SL_{n}(K)$ を全ての成分が体 $K$ の元である $n$ 次正方行列のうち、行列式が1に等しい行列がなす集合とする。

$SL_{n}(K)$ は $GL_{n}(K)$ よりもさらに条件が厳しくなった集合になる。

$SL_{n}(K)$ も行列の積に関して群をなす。

\paragraph{$O(n)$ : 直交群}

${}$

$n$ 次正方行列 $A$ で、なおかつ $A^{-1} = A^{T}$ を満たす行列の集合。

行列式は1または-1になる。

原点を中心にした回転や鏡映反転を対応する変換のなす集合を表す。

\paragraph{$SO(n)$ : 特殊直交群}

${}$

$O(n)$ のうち、行列式が1になる行列の集合。

回転群とも呼ばれる。
$SO(2)$ は2次元平面上の回転群であり、
$SO(3)$ は3次元空間上の回転群になる。

\paragraph{$U(n)$ : ユニタリー群}

${}$

$n$ 次複素正方行列 $A$ で、$A^{-1} = A^{*}$ を満たすもの。

複素空間における回転や鏡映に対応する変換の集合を表す。

\paragraph{$SU(n)$ : 特殊ユニタリー群}

${}$

$U(n)$ のうち行列式が1に等しい行列の集合。

SU(n)は、物理学、特に素粒子物理学の標準模型において、粒子の内部的な対称性を記述するのに不可欠な役割を果たす。

${}$

$\textbf{SU(2)とスピン}$

量子力学では粒子の状態は複素ベクトル空間のベクトルとして記述される。

このベクトル空間で行われる変換(例えば回転)は、ユニタリー行列によって表される。

$SU(2)$ は、量子力学的な自由度であるスピンを記述するために用いられる。

電子や陽子のような基本粒子は、古典的な意味での回転とは異なる量子力学的な自転のような性質を有しており、これをスピンと呼ぶ。

$SU(2)$ の行列は、2 $\times$ 2のユニタリ行列であり、行列式は1である。

これらの行列は、電子のスピン状態を記述する2次元の複素ベクトル空間(スピノル空間)の変換を表す。

${}$

例えば、電子のスピンは\textbf{上向き(spin-up)と下向き(spin-down)}の二つの状態をとる。

これらの状態は2次元ベクトル
$
	\begin{pmatrix} 1 \\
		0\end{pmatrix}
$
と
$
	\begin{pmatrix} 0 \\
		1\end{pmatrix}
$

で表される。

$SU(2)$ 群の要素は、これら2つの状態間の任意の変換を表現できる。
重ね合わせ状態への変化も記述できる。

$SU(2)$ は3次元空間での回転群 $SO(3)$ と関連している。

$SU(2)$ は $SO(3)$ の二重被覆群であり、
スピンのような半整数スピンを扱う際に自然に現れる数学的構造になっている。

${}$

\textbf{SU(3)とクォークの色荷(カラーチャージ)}

$SU(3)$ は強い相互作用を記述するために用いられる。

陽子や中性子といったハドロンを構成するクォークには、色荷(カラーチャージ)と呼ばれる自由度が存在する。

この色荷は、赤、緑、青の三種類であり、これらは古典的な色とは全く関係のない量子数である。

$SU(3)$ の行列は、3 $\times$3 のユニタリー行列で行列式が1のものである。

これらの行列は、3つの色荷の状態を記述する3次元複素ベクトル空間の変換を表す。

強い相互作用を媒介するグルーオンは、これらの色荷の状態を混ぜ合わせる役割を担っており、その変換は $SU(3)$ の対称性に従う。

$SU(n)$ 群は、粒子の状態(スピンや色荷)を数学的に記述する。


${}$

\textbf{SU(5)と大統一理論}

$SU(5)$ は素粒子物理における大統一理論の中で現れる。

標準模型は物理学における力を以下のようにそれぞれ異なる対称群で表現している。
\begin{itemize}
	\item 強い力:$SU(3)$
	\item 弱い力:$SU(2)$
	\item 電磁気力:$U(1)$
\end{itemize}

これらの3つの力が高エネルギー領域で単一の力に統一されるという考えに基づいた考え方が大統一理論である。

$SU(3) \times SU(2) \times U(1)$
をより大きな単一の群の部分群として含むようなモデルが提案された。

その最も自然な(単純でエレガントな)候補の一つが $SU(5)$ だった。

$SU(5)$ の理論から予言される現象の一つに陽子崩壊がある。

陽子は安定な粒子だと考えられているが、$SU(5)$ からは極めて長いが有限の寿命で崩壊する可能性があることが示唆される。
多くの実験が陽子崩壊を観測しようと試みているが、今のところ観測にかかっていない。

陽子崩壊の他にも実験データとの不整合があり、$SU(5)$ の理論はそのままでは正しくなさそうであると考えられるようになってきた。



\section{行列の集合も位相空間になる}

(作成中。)

\end{document}
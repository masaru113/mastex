\documentclass[uplatex,a4j,12pt,dvipdfmx]{jsarticle}
\usepackage[english]{babel}
\usepackage[letterpaper,top=2cm,bottom=2cm,left=3cm,right=3cm,marginparwidth=1.75cm]{geometry}
\usepackage{amsmath, amssymb}
\usepackage{graphicx}
\usepackage[colorlinks=true, allcolors=blue]{hyperref}
\usepackage{fancybox}
\usepackage{tikz-cd}
\title{
行列のなす群の位相
}

\author{
岡田 大 (Okada Masaru)
}

\begin{document}
\maketitle


\begin{abstract}
	行列のなす群の位相についてまとめ。
\end{abstract}

\section{行列の集合}

位相空間はある特定の条件を満たす開集合のあつまりのことを指すが、行列の集合もまた位相空間となることがある。
開集合という抽象的な概念が行列にも導入できる。

\subsection{$M_{n}(K)$ : $K$ 上の $n$次正方行列の集合}

$M_{n}(K)$ を全ての成分が体 $K$ の元である $n \times n$ 行列がなす集合とする。

この集合は、行列の加法とスカラー倍についてベクトル空間の構造を持つ。

この集合には行列式がゼロとなる特異行列も含まれる。

\subsection{$GL_{n}(K)$ : 一般線形群}

$GL_{n}(K)$ を全ての成分が体 $K$ の元である $n$ 次正則行列がなす集合とする。

$M_{n}(K)$ と異なり、$GL_{n}(K)$ には行列式が0となるような行列(特異行列)は含まれない。

$GL_{n}(K)$ は行列の乗法で群になる。

\subsection{$SL_{n}(K)$ : 特殊線形群}

$SL_{n}(K)$ を全ての成分が体 $K$ の元である $n$ 次正方行列のうち、行列式が1に等しい行列がなす集合とする。

$SL_{n}(K)$ は $GL_{n}(K)$ よりもさらに条件が厳しくなった集合になる。

$SL_{n}(K)$ も行列の積に関して群をなす。

\subsection{$O(n)$ : 直交群}

$n$ 次正方行列 $A$ で、なおかつ $A^{-1} = A^{T}$ を満たす行列の集合。

行列式は1または-1になる。

原点を中心にした回転や鏡映反転を対応する変換のなす集合を表す。

\subsection{$SO(n)$ : 特殊直交群}

$O(n)$ のうち、行列式が1になる行列の集合。

回転群とも呼ばれる。
$SO(2)$ は2次元平面上の回転群であり、
$SO(3)$ は3次元空間上の回転群になる。

\subsection{$U(n)$ : ユニタリー群}

$n$ 次複素正方行列 $A$ で、$A^{-1} = A^{*}$ を満たすもの。

複素空間における回転や鏡映に対応する変換の集合を表す。

\subsection{$SU(n)$ : 特殊ユニタリー群}

$U(n)$ のうち行列式が1に等しい行列の集合。

SU(n)は、物理学、特に素粒子物理学の標準模型において、粒子の内部的な対称性を記述するのに不可欠な役割を果たす。

\subsubsection{SU(2)とスピン}

量子力学では粒子の状態は複素ベクトル空間のベクトルとして記述される。

このベクトル空間で行われる変換(例えば回転)は、ユニタリー行列によって表される。

$SU(2)$ は、量子力学的な自由度であるスピンを記述するために用いられる。

電子や陽子のような基本粒子は、古典的な意味での回転とは異なる量子力学的な自転のような性質を有しており、これをスピンと呼ぶ。

$SU(2)$ の行列は、2 $\times$ 2のユニタリ行列であり、行列式は1である。

これらの行列は、電子のスピン状態を記述する2次元の複素ベクトル空間(スピノル空間)の変換を表す。

${}$

例えば、電子のスピンは上向き(spin-up)と下向き(spin-down)の二つの状態をとる。

これらの状態は2次元ベクトル
$
	\begin{pmatrix} 1 \\
		0\end{pmatrix}
$
と
$
	\begin{pmatrix} 0 \\
		1\end{pmatrix}
$

で表される。

$SU(2)$ 群の要素は、これら2つの状態間の任意の変換を表現できる。
重ね合わせ状態への変化も記述できる。

$SU(2)$ は3次元空間での回転群 $SO(3)$ と関連している。

$SU(2)$ は $SO(3)$ の二重被覆群であり、
スピンのような半整数スピンを扱う際に自然に現れる数学的構造になっている。

${}$

\subsubsection{SU(3)とクォークの色荷(カラーチャージ)}

$SU(3)$ は強い相互作用を記述するために用いられる。

陽子や中性子といったハドロンを構成するクォークには、色荷(カラーチャージ)と呼ばれる自由度が存在する。

この色荷は、赤、緑、青の三種類であり、これらは古典的な色とは全く関係のない量子数である。

$SU(3)$ の行列は、3 $\times$3 のユニタリー行列で行列式が1のものである。

これらの行列は、3つの色荷の状態を記述する3次元複素ベクトル空間の変換を表す。

強い相互作用を媒介するグルーオンは、これらの色荷の状態を混ぜ合わせる役割を担っており、その変換は $SU(3)$ の対称性に従う。

$SU(n)$ 群は、粒子の状態(スピンや色荷)を数学的に記述する。

\subsubsection{SU(5)と大統一理論}

$SU(5)$ は素粒子物理における大統一理論の中で現れる。

標準模型は物理学における力を以下のようにそれぞれ異なる対称群で表現している。
\begin{itemize}
	\item 強い力:$SU(3)$
	\item 弱い力:$SU(2)$
	\item 電磁気力:$U(1)$
\end{itemize}

これらの3つの力が高エネルギー領域で単一の力に統一されるという考えに基づいた考え方が大統一理論である。

$SU(3) \times SU(2) \times U(1)$
をより大きな単一の群の部分群として含むようなモデルが提案された。

その最も自然な(単純でエレガントな)候補の一つが $SU(5)$ だった。

${}$

$SU(3) \times SU(2) \times U(1)$
を部分群として含む、より大きな
$SU(5)$ の対称性を理解するために、
5次元の複素ベクトル空間 $\mathbb{C}^{5}$ を2つの部分空間に分割する。
$$ \mathbb{C}^{5} = \mathbb{C}^{2} \oplus \mathbb{C}^{3} $$
この分割に対応して、$SU(5)$ の元の行列 $M$ は次のようなブロック対角行列で表される。
$$
	M =
	\left(
	\begin{array}{cc}
			M_{2 \times 2} & 0              \\
			0              & M_{3 \times 3}
		\end{array}
	\right)
$$
ここで $M_{2 \times 2}$ は$SU(2)$ の要素の2 $\times$ 2 行列、
$M_{3 \times 3}$ は$SU(3)$ の要素の3 $\times$ 3 行列を表す。

この分解は、$SU(2) \times SU(3)$ が 行列式が1である $SU(5)$ の部分群であることの理由になる。

リー群のより深い構造を理解するためには、リー群に付随するリー環を考える必要がある。

リー群 $SU(N)$ に付随するリー環 $su(N)$ は、
トレースがゼロの $N \times N$ の反エルミート行列の集合になっている。

リー群の制約条件については $SU(3) \times SU(2)$ で満たされた一方で、
リー群に付随するリー環の要素の行列のrank(対角対称性の数)を考えると、
$\text{rank}(su(N))=N-1$
なので、

$\text{rank}(su(5))=4$

$\text{rank}(su(3))=2$

$\text{rank}(su(2))=1$

であり、rankが1不足する。
この不足している1つの対角生成子が $U(1)$ の生成子に対応する。

次元(生成子の総数)についてみてみると、
$\text{dim}(su(N))=N^{2} -1$
であることから、

$\text{dim}(su(5))=24$

$\text{dim}(su(3))=8$

$\text{dim}(su(2))=3$

$SU(5)$ の部分群である $SU(2) \times SU(3)$ のリー環の次元は11しかなく、自由度が13不足する。

不足した13のうち1つは$U(1)$の生成子、残りの自由度12についてはXボゾンやYボゾンといった、例えば陽子崩壊現象を媒介する可能性のある粒子に対応する。


${}$

$SU(5)$ の理論から予言される現象の一つに陽子崩壊がある。

陽子は安定な粒子だと考えられているが、$SU(5)$ からは極めて長いが有限の寿命で崩壊する可能性があることが示唆される。
多くの実験が陽子崩壊を観測しようと試みているが、今のところ観測にかかっていない。

陽子崩壊の他にも実験データとの不整合があり、$SU(5)$ の理論はそのままでは正しくなさそうであると考えられるようになってきた。





\section{行列の集合も位相空間になる}


行列の集合も位相空間になる。
行列の集合の要素間に「近さ」や「連続性」を定義することができる。

\subsection{位相の導入}

行列の集合 $M_n(\mathbb{R})$ を位相空間にする自明な方法としては、単にユークリッド空間と同一視すればよい。

$n \times n$ 行列 $A$ は、その $n^2$ 個の成分を並べることで、$n^2$ 次元実ベクトル空間 $\mathbb{R}^{n^2}$ の点と見なすことができる。
$$ A = (a_{ij}) \quad \leftrightarrow \quad (a_{11}, a_{12}, \dots, a_{nn}) \in \mathbb{R}^{n^2} $$
この同一視によって、$M_n(\mathbb{R})$ は自然に $\mathbb{R}^{n^2}$ の位相を継承する。

これにより、行列の集合にも開集合や閉集合の概念が導入できる。


同様に、複素行列の集合 $M_n(\mathbb{C})$ は、$2n^2$ 次元実ベクトル空間 $\mathbb{R}^{2n^2}$ と同一視でき、この空間の位相を継承する。

\subsection{連続写像の例}

行列の集合に位相が定義されると、行列を引数に取る関数が連続であるかどうかを議論できる。

\begin{enumerate}
	\item \textbf{行列式}:$\det : M_n(\mathbb{K}) \to \mathbb{K}$ は、行列の各成分の多項式であるため、連続写像。
	\item \textbf{行列の乗法}:$M_n(\mathbb{K}) \times M_n(\mathbb{K}) \to M_n(\mathbb{K})$ は、行列の各成分の多項式であり、連続写像。
	\item \textbf{逆行列に写す写像}:$A \mapsto A^{-1}$ は、$\det(A) \neq 0$ の行列の集合 $GL_n(\mathbb{K})$ 上で連続。
\end{enumerate}


\section{位相群}


\subsection{一般線形群 $GL_{n}(K)$ は位相群になる。}



このことを示すには

\begin{enumerate}
	\item $GL_{n}(K)$ は位相空間である。
	\item $GL_{n}(K)$ は群である。
	\item 写像 $\mu: GL_{n}(K) \times GL_{n}(K) \to GL_{n}(K), \nu: GL_{n}(K) \to GL_{n}(K)$ をそれぞれ $\mu(x,y)=xy, \nu(x) = x^{-1}$ と定義するとき、$\mu,\nu$ は連続である。
\end{enumerate}
ということを示せればよい。


\subsubsection{一般線形群 $GL_{n}(K)$ は $M_{n}(K)$ の開集合になる。}

$M_{n}(K)$ をその成分を並べたユークリッド空間 $\mathbb{R}^{n^{2}}$ と同一視する。

これにより $M_{n}(K)$ は自然な位相空間になる。

$GL_{n} (K)$ は $M_{n} (K)$ の部分集合であるが、定義は
$$
	GL_{n}(K) = \{ A \in M_{n}(K) \ | \ \text{det} A \neq 0 \}
$$
であり、行列式がゼロでないという条件が入っている。
行列式は行列の各成分に関する連続関数であり、連続関数のゼロでない点の集合は開集合になる。

したがって、$GL_{n}(K)$ は位相空間 $M_{n}(K)$ の開集合になっていて、それ自身が位相空間になる。


${}$


同じことであるが、もっとシンプルには、開集合 $K - \{0\}$ を連続な逆像 $\det^{-1} (K - \{0\})$ で送った先が開集合 $GL_{n}(K)$ である。

\subsubsection{一般線形群 $GL_{n}(K)$ の演算が連続である。}

一般線形群 $GL_{n}(K)$ は積に関して群を成す。

2つの行列 $A,B \in GL_{n}(K)$ の積 $C=AB$ の各成分は$A,B$の成分の多項式で表される。

多項式は連続関数なので、積の演算は連続。

${}$

行列 $A$ の逆行列 $A^{-1}$ の各成分は余因子を det$A \neq 0$ で割ったもので表される。

余因子は多項式なので逆元を取る演算も連続である。

${}$

以上から一般線形群 $GL_{n}(K)$ は位相群になる。

\subsection{$SL_{n}(K),O(n),SO(n),U(n),SU(n)$ も位相群である。}

$GL_{n}(K)$ の部分群であるから$SL_{n}(K),O(n),SO(n),U(n),SU(n)$ も位相群である。



\subsection{$O(n),SO(n),U(n),SU(n)$ はコンパクトである。}

\subsubsection{$K^{n}$ はコンパクトではない}

まず、そもそもの例として、$K^{n}$ 自体は有界ではないのでコンパクトではない。


\subsubsection{$K^{n}$ の単位球面 $S_{K}^{n-1}$ はコンパクトである。}

$K^{n}$ の単位球面 $S_{K}^{n-1}$ はコンパクトである。

確認すると、
$f(x) = |x|$
と置くと、$f$ は連続である。
{1} $\in \mathbb{R}$ は閉集合であるから、
$S_{K}^{n-1} = f^{-1}(1)$ は $K$ の閉集合になり、なおかつ有界なので、
$S_{K}^{n-1}$ はコンパクト。


\subsubsection{$O(n),U(n)$ はコンパクトである。}

$O(n),U(n)$ をまとめて $G(n,K)$ と書き、$SO(n),SU(n)$ は $SG(n,k)$ と書くことにする。

$f: M_{n}(K) \to M_{n}(K)$ の写像 $f(A) = A A^{*}$ は連続である。

$G(n,k) = f^{-1}(1_{G(n,K)})$ であるから、$G(n,k)$ は1点 $1_{G(n,K)}$ ($G(n,K)$の単位行列)の逆像として閉集合になる。

さらに $|A| \leq n$ で有界である。

従って$G(n,K)$ 、すなわち、 $O(n),U(n)$ はコンパクトである。


\subsubsection{$SO(n),SU(n)$ はコンパクトである。}

写像 det : $M_{n}(K) \to K$ は連続。

1点、$M_{n}(K)$ の単位行列 $1_{M_{n}(K)}$ は閉集合であり、
$SL_{n}(K) = \text{det}^{1}(1_{M_{n}(K)})$ は閉集合。

そして $SG(n,k) = G(n,k) \cap SL_{n}(K)$ であるが、2つの閉集合の共通部分は閉集合。

$SG(n,k)$ は有界集合 $G(n,K)$ の部分集合であるので有界。

以上から$SG(n,k)$、すなわち、$SO(n),SU(n)$はコンパクト。

${}$

注意点として、$GL_{n}(K), SL_{n}(K)$ はコンパクトでない。


\subsection{$O(n),U(n),SO(n),SU(n),GL_{n}(K),SL_{n}(K)$ は可算開基を持つ局所コンパクトハウスドルフ群である。}

\subsubsection{ハウスドルフ}

ユークリッド空間はハウスドルフであり、その部分空間の$M_{n}(K)$ もまたハウスドルフである。

$O(n),U(n),SO(n),SU(n),GL_{n}(K),SL_{n}(K)$ は $M_{n}(K)$ の部分空間なのでこれらすべてハウスドルフである。

\subsubsection{可算開基}

$\mathbb{R}$ は可算開基を持つので、その直積集合の$M_{n}(K)$ もまた可算開基を持つ。

可算開基を持つ位相空間の部分空間もまた可算開基を持つことから、
$O(n),U(n),SO(n),SU(n),GL_{n}(K),SL_{n}(K)$ は可算開基を持つ。

\subsubsection{局所コンパクト}

コンパクトならば局所コンパクトなので、
$O(n),U(n),SO(n),SU(n)$ は局所コンパクト。

$GL_{n}(K), SL_{n}(K)$ はコンパクトでない。

一方で、局所コンパクトの直積集合もまた局所コンパクトなので、
ユークリッド空間の直積集合である $M_{n}(K)$ は局所コンパクトである。

一般論として、局所コンパクトの開部分空間または閉部分空間は局所コンパクトである。

これより、局所コンパクトの開集合は局所コンパクトなので、$GL_{n}(K)$ は局所コンパクトである。

$SL_{n}(K)$ は $GL_{n}(K)$ の閉部分空間なので局所コンパクトである。

${}$

以上から
$O(n),U(n),SO(n),SU(n),GL_{n}(K),SL_{n}(K)$
は可算開基を持つ局所コンパクトハウスドルフ群である。

特に、
$O(n),U(n),SO(n),SU(n)$
はコンパクトハウスドルフである。

\end{document}
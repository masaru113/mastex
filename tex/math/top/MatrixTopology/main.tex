\documentclass[uplatex,a4j,12pt,dvipdfmx]{jsarticle}
\usepackage[english]{babel}
\usepackage[letterpaper,top=2cm,bottom=2cm,left=3cm,right=3cm,marginparwidth=1.75cm]{geometry}
\usepackage{amsmath, amssymb}
\usepackage{graphicx}
\usepackage[colorlinks=true, allcolors=blue]{hyperref}
\usepackage{fancybox}
\usepackage{tikz-cd}
\title{
\textbf{The Topology of Matrix Groups}
}

\author{
Masaru Okada
}

\begin{document}
\maketitle


\begin{abstract}
	A summary of the topology of matrix groups. In a digression, I'll also supplement with a note on unitary groups and quantum mechanics.
\end{abstract}

\section{Sets of Matrices}

A topological space refers to a collection of open sets that satisfy certain conditions, and a set of matrices can also form a topological space.
The abstract concept of an open set can also be introduced to matrices.

\subsection{\textbf{$M_{n}(K)$: The set of $n \times n$ square matrices over $K$}}

Let $M_{n}(K)$ be the set of $n \times n$ matrices whose components are elements of a field $K$.

This set has a vector space structure with respect to matrix addition and scalar multiplication.

This set also includes singular matrices, whose determinant is zero.

\subsection{\textbf{$GL_{n}(K)$: General Linear Group}}

Let $GL_{n}(K)$ be the set of $n \times n$ invertible matrices whose components are elements of a field $K$.

Unlike $M_{n}(K)$, $GL_{n}(K)$ does not include matrices whose determinant is 0 (singular matrices).

$GL_{n}(K)$ forms a group under matrix multiplication.

\subsection{\textbf{$SL_{n}(K)$: Special Linear Group}}

Let $SL_{n}(K)$ be the set of $n \times n$ square matrices whose components are elements of a field $K$ and whose determinant is equal to 1.

$SL_{n}(K)$ is a set with even stricter conditions than $GL_{n}(K)$.

$SL_{n}(K)$ also forms a group with respect to matrix multiplication.

\subsection{\textbf{$O(n)$: Orthogonal Group}}

The set of $n \times n$ square matrices $A$ that satisfy $A^{-1} = A^{T}$.

The determinant is 1 or -1.

It represents a set of transformations corresponding to rotations and reflections centered at the origin.

\subsection{\textbf{$SO(n)$: Special Orthogonal Group}}

The set of matrices in $O(n)$ whose determinant is 1.

It is also called the rotation group.
$SO(2)$ is the rotation group in the 2-dimensional plane, and
$SO(3)$ is the rotation group in 3-dimensional space.

\subsection{\textbf{$U(n)$: Unitary Group}}

The set of $n \times n$ complex square matrices $A$ that satisfy $A^{-1} = A^{*}$.

It represents a set of transformations corresponding to rotations and reflections in a complex space.

\subsection{\textbf{$SU(n)$: Special Unitary Group}}

The set of matrices in $U(n)$ whose determinant is equal to 1.

$SU(n)$ plays an essential role in describing the internal symmetry of particles in physics, especially in the Standard Model of particle physics.

\subsubsection{\textbf{SU(2) and Spin}}

In quantum mechanics, the state of a particle is described as a vector in a complex vector space.

Transformations (for example, rotations) that occur in this vector space are represented by unitary matrices.

$SU(2)$ is used to describe spin, a quantum mechanical degree of freedom.

Elementary particles like electrons and protons possess a quantum mechanical property similar to self-rotation, which is different from rotation in the classical sense, and this is called spin.

The matrices of $SU(2)$ are 2 $\times$ 2 unitary matrices with a determinant of 1.

These matrices represent transformations of a 2-dimensional complex vector space (spinor space) that describes the spin state of an electron.

\ \ 

For example, an electron's spin can take two states: spin-up and spin-down.

These states are represented by 2-dimensional vectors
$
	\begin{pmatrix} 1 \\
		0\end{pmatrix}
$
and
$
	\begin{pmatrix} 0 \\
		1\end{pmatrix}
$
.

Elements of the $SU(2)$ group can express any transformation between these two states.
They can also describe changes to a superposition state.

$SU(2)$ is related to the rotation group $SO(3)$ in 3-dimensional space.

$SU(2)$ is a double cover of $SO(3)$, and it is a mathematical structure that naturally appears when dealing with half-integer spins like spin.

\ \ 

\subsubsection{\textbf{SU(3) and Quark Color Charge}}

$SU(3)$ is used to describe the strong interaction.

Quarks, which make up hadrons like protons and neutrons, have a degree of freedom called color charge.

This color charge has three types: red, green, and blue, which are quantum numbers completely unrelated to classical colors.

The matrices of $SU(3)$ are 3 $\times$ 3 unitary matrices with a determinant of 1.

These matrices represent transformations of a 3-dimensional complex vector space that describes the three color charge states.

Gluons, which mediate the strong interaction, play a role in mixing these color charge states, and their transformations follow the $SU(3)$ symmetry.

$SU(n)$ groups mathematically describe the states of particles (spin and color charge).

\subsubsection{\textbf{SU(5) and Grand Unified Theories}}

$SU(5)$ appears in Grand Unified Theories in particle physics.

The Standard Model represents the forces in physics with the following different symmetry groups:
\begin{itemize}
	\item Strong force: $SU(3)$
	\item Weak force: $SU(2)$
	\item Electromagnetic force: $U(1)$
\end{itemize}

Grand Unified Theory is a concept based on the idea that these three forces are unified into a single force in high-energy regimes.

Models were proposed that contain $SU(3) \times SU(2) \times U(1)$ as a subgroup of a larger, single group.

One of the most natural (simple and elegant) candidates for this was $SU(5)$.

\ \ 

To understand the deeper structure of the $SU(5)$ group that contains $SU(3) \times SU(2) \times U(1)$ as a subgroup, we can split the 5-dimensional complex vector space $\mathbb{C}^{5}$ into two subspaces.
$$ \mathbb{C}^{5} = \mathbb{C}^{2} \oplus \mathbb{C}^{3} $$
Corresponding to this split, a matrix $M$ of an element of $SU(5)$ is represented by the following block-diagonal matrix.
$$
	M =
	\left(
	\begin{array}{cc}
			M_{2 \times 2} & 0 \\
			0 & M_{3 \times 3}
		\end{array}
	\right)
$$
Here, $M_{2 \times 2}$ represents a 2 $\times$ 2 matrix of an element of $SU(2)$, and
$M_{3 \times 3}$ represents a 3 $\times$ 3 matrix of an element of $SU(3)$.

This decomposition is the reason why $SU(2) \times SU(3)$ is a subgroup of $SU(5)$ with a determinant of 1.

To understand the deeper structure of a Lie group, we need to consider the Lie algebra associated with it.

The Lie algebra $su(N)$ associated with the Lie group $SU(N)$ is the set of $N \times N$ anti-Hermitian matrices with zero trace.

While the Lie group constraint was satisfied with $SU(3) \times SU(2)$,
considering the rank (number of diagonal symmetries) of the matrices of the elements of the Lie algebra, we find that
$\text{rank}(su(N))=N-1$
so,

$\text{rank}(su(5))=4$

$\text{rank}(su(3))=2$

$\text{rank}(su(2))=1$

and the rank is short by 1.
This missing one diagonal generator corresponds to the generator of $U(1)$.

Looking at the dimension (the total number of generators),
$\text{dim}(su(N))=N^{2} -1$
we have,

$\text{dim}(su(5))=24$

$\text{dim}(su(3))=8$

$\text{dim}(su(2))=3$

The dimension of the Lie algebra of the subgroup $SU(2) \times SU(3)$ of $SU(5)$ is only 11, which is short of 13 degrees of freedom.

Of the missing 13, one corresponds to the generator of $U(1)$, and the remaining 12 degrees of freedom correspond to particles like X bosons and Y bosons, which could mediate phenomena such as proton decay.

\ \ 

One of the phenomena predicted by the $SU(5)$ theory is proton decay.

The proton is thought to be a stable particle, but $SU(5)$ suggests that it might decay with an extremely long but finite lifetime.
Many experiments have tried to observe proton decay, but so far it has not been detected.

There are also other inconsistencies with experimental data besides proton decay, and it has come to be thought that the $SU(5)$ theory as it is might not be correct.

\section{A Set of Matrices can also be a Topological Space}


A set of matrices can also be a topological space.
We can define "closeness" and "continuity" between elements of a set of matrices.

\subsection{\textbf{Introducing Topology}}

A trivial way to make the set of matrices $M_n(\mathbb{R})$ a topological space is simply to identify it with a Euclidean space.

An $n \times n$ matrix $A$ can be regarded as a point in the $n^2$-dimensional real vector space $\mathbb{R}^{n^2}$ by arranging its $n^2$ components.
$$ A = (a_{ij}) \quad \leftrightarrow \quad (a_{11}, a_{12}, \dots, a_{nn}) \in \mathbb{R}^{n^2} $$
Through this identification, $M_n(\mathbb{R})$ naturally inherits the topology of $\mathbb{R}^{n^2}$.

This allows for the introduction of concepts like open and closed sets to a set of matrices.


Similarly, the set of complex matrices $M_n(\mathbb{C})$ can be identified with the $2n^2$-dimensional real vector space $\mathbb{R}^{2n^2}$ and inherits the topology of this space.

\subsection{\textbf{Examples of Continuous Maps}}

When a topology is defined on a set of matrices, we can discuss whether a function that takes a matrix as an argument is continuous.

\begin{enumerate}
	\item \textbf{Determinant}: $\det : M_n(\mathbb{K}) \to \mathbb{K}$ is a continuous map, as it is a polynomial of the components of the matrix.
	\item \textbf{Matrix Multiplication}: $M_n(\mathbb{K}) \times M_n(\mathbb{K}) \to M_n(\mathbb{K})$ is a continuous map, as its components are polynomials of the components of the matrices.
	\item \textbf{Inverse Map}: The map $A \mapsto A^{-1}$ is continuous on the set of matrices with $\det(A) \neq 0$, which is $GL_n(\mathbb{K})$.
\end{enumerate}


\section{Topological Groups}


\subsection{\textbf{The General Linear Group $GL_{n}(K)$ is a topological group.}}



To show this, we need to demonstrate that:

\begin{enumerate}
	\item $GL_{n}(K)$ is a topological space.
	\item $GL_{n}(K)$ is a group.
	\item The maps $\mu: GL_{n}(K) \times GL_{n}(K) \to GL_{n}(K)$ and $\nu: GL_{n}(K) \to GL_{n}(K)$, defined as $\mu(x,y)=xy$ and $\nu(x) = x^{-1}$ respectively, are continuous.
\end{enumerate}
If we can show these, it will be sufficient.


\subsubsection{\textbf{The General Linear Group $GL_{n}(K)$ is an open set of $M_{n}(K)$.}}

We identify $M_{n}(K)$ with the Euclidean space $\mathbb{R}^{n^{2}}$ by arranging its components.

This makes $M_{n}(K)$ a natural topological space.

$GL_{n} (K)$ is a subset of $M_{n} (K)$, but its definition is
$$
	GL_{n}(K) = \{ A \in M_{n}(K) \ | \ \text{det} A \neq 0 \}
$$
and it includes the condition that the determinant is not zero.
The determinant is a continuous function of the matrix components, and the set of points where a continuous function is not zero is an open set.

Therefore, $GL_{n}(K)$ is an open set of the topological space $M_{n}(K)$, and it becomes a topological space itself.


\ \ 


In other words, more simply, the image of the open set $K - \{0\}$ under the continuous inverse map $\det^{-1} (K - \{0\})$ is the open set $GL_{n}(K)$.

\subsubsection{\textbf{The operations of the General Linear Group $GL_{n}(K)$ are continuous.}}

The general linear group $GL_{n}(K)$ forms a group under multiplication.

Each component of the product $C=AB$ of two matrices $A,B \in GL_{n}(K)$ is expressed as a polynomial of the components of $A,B$.

Since a polynomial is a continuous function, the multiplication operation is continuous.

\ \ 

Each component of the inverse matrix $A^{-1}$ of a matrix $A$ is expressed by dividing a cofactor by det$A \neq 0$.

Since a cofactor is a polynomial, the operation of taking the inverse is also continuous.

\ \ 

From the above, the general linear group $GL_{n}(K)$ is a topological group.

\subsection{\textbf{$SL_{n}(K),O(n),SO(n),U(n),SU(n)$ are also topological groups.}}

Since they are subgroups of $GL_{n}(K)$, $SL_{n}(K),O(n),SO(n),U(n),SU(n)$ are also topological groups.



\subsection{\textbf{$O(n),SO(n),U(n),SU(n)$ are compact.}}

\subsubsection{\textbf{$K^{n}$ is not compact.}}

First, as an example, $K^{n}$ itself is not bounded, so it is not compact.


\subsubsection{\textbf{The unit sphere $S_{K}^{n-1}$ of $K^{n}$ is compact.}}

The unit sphere $S_{K}^{n-1}$ of $K^{n}$ is compact.

To confirm, if we set
$f(x) = |x|$
then $f$ is continuous.
Since $\{1\} \in \mathbb{R}$ is a closed set,
$S_{K}^{n-1} = f^{-1}(1)$ is a closed set of $K$ and is also bounded, so
$S_{K}^{n-1}$ is compact.


\subsubsection{\textbf{$O(n),U(n)$ are compact.}}

Let's write $O(n),U(n)$ together as $G(n,K)$ and $SO(n),SU(n)$ as $SG(n,k)$.

The map $f: M_{n}(K) \to M_{n}(K)$ where $f(A) = A A^{*}$ is continuous.

Since $G(n,k) = f^{-1}(1_{G(n,K)})$, $G(n,k)$ becomes a closed set as the inverse image of a single point $1_{G(n,K)}$ (the identity matrix of $G(n,K)$).

Furthermore, $|A| \leq n$ so it is bounded.

Therefore, $G(n,K)$, that is, $O(n),U(n)$, are compact.


\subsubsection{\textbf{$SO(n),SU(n)$ are compact.}}

The map det : $M_{n}(K) \to K$ is continuous.

The single point, the identity matrix $1_{M_{n}(K)}$ of $M_{n}(K)$, is a closed set, and
$SL_{n}(K) = \text{det}^{1}(1_{M_{n}(K)})$ is a closed set.

And $SG(n,k) = G(n,k) \cap SL_{n}(K)$, but the intersection of two closed sets is a closed set.

$SG(n,k)$ is a subset of the bounded set $G(n,K)$, so it is bounded.

From the above, $SG(n,k)$, that is, $SO(n),SU(n)$, are compact.

\ \ 

Note that $GL_{n}(K), SL_{n}(K)$ are not compact.


\subsection{\textbf{$O(n),U(n),SO(n),SU(n),GL_{n}(K),SL_{n}(K)$ are locally compact Hausdorff groups with a countable basis.}}

\subsubsection{\textbf{Hausdorff}}

Euclidean space is Hausdorff, and its subspace $M_{n}(K)$ is also Hausdorff.

$O(n),U(n),SO(n),SU(n),GL_{n}(K),SL_{n}(K)$ are subspaces of $M_{n}(K)$, so they are all Hausdorff.

\subsubsection{\textbf{Countable Basis}}

Since $\mathbb{R}$ has a countable basis, its Cartesian product $M_{n}(K)$ also has a countable basis.

Since a subspace of a topological space with a countable basis also has a countable basis,
$O(n),U(n),SO(n),SU(n),GL_{n}(K),SL_{n}(K)$ have a countable basis.

\subsubsection{\textbf{Locally Compact}}

If a space is compact, it is locally compact, so
$O(n),U(n),SO(n),SU(n)$ are locally compact.

$GL_{n}(K), SL_{n}(K)$ are not compact.

On the other hand, the Cartesian product of locally compact spaces is also locally compact, so
the Cartesian product of Euclidean spaces, $M_{n}(K)$, is locally compact.

As a general principle, an open or closed subspace of a locally compact space is locally compact.

From this, an open set of a locally compact space is locally compact, so $GL_{n}(K)$ is locally compact.

$SL_{n}(K)$ is a closed subspace of $GL_{n}(K)$, so it is locally compact.

\ \ 

From the above,
$O(n),U(n),SO(n),SU(n),GL_{n}(K),SL_{n}(K)$
are locally compact Hausdorff groups with a countable basis.

In particular,
$O(n),U(n),SO(n),SU(n)$
are compact Hausdorff.

\end{document}
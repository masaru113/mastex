\documentclass[uplatex,a4j,12pt,dvipdfmx]{jsarticle}
\usepackage[english]{babel}
\usepackage[letterpaper,top=2cm,bottom=2cm,left=3cm,right=3cm,marginparwidth=1.75cm]{geometry}
\usepackage{amsmath, amssymb}
\usepackage{graphicx}
\usepackage[colorlinks=true, allcolors=blue]{hyperref}
\usepackage{fancybox}
\usepackage{tikz-cd}
\title{
Topology of Matrices
}

\author{
mastex
}

\begin{document}
\maketitle


\begin{abstract}
	(Abstract in progress. To be added later.)
\end{abstract}

\section{Sets of Matrices}

While a topological space refers to a collection of open sets that satisfy certain conditions, a set of matrices can also form a topological space.
The abstract concept of an open set can be introduced to matrices as well.

\paragraph{$M_{n}(K)$ : The set of $n \times n$ matrices over $K$}

{}

Let $M_{n}(K)$ be the set of $n \times n$ matrices whose components are all elements of the field $K$.

This set has the structure of a vector space with respect to matrix addition and scalar multiplication.

This set also includes singular matrices, whose determinant is zero.

\paragraph{$GL_{n}(K)$ : The general linear group}

{}

Let $GL_{n}(K)$ be the set of $n \times n$ invertible matrices whose components are all elements of the field $K$.

Unlike $M_{n}(K)$, $GL_{n}(K)$ does not include matrices with a determinant of 0 (singular matrices).

$GL_{n}(K)$ forms a group under matrix multiplication.

\paragraph{$SL_{n}(K)$ : The special linear group}

{}

Let $SL_{n}(K)$ be the set of $n \times n$ matrices whose components are all elements of the field $K$, and whose determinant is equal to 1.

$SL_{n}(K)$ is a more strictly conditioned set than $GL_{n}(K)$.

$SL_{n}(K)$ also forms a group under matrix multiplication.

\paragraph{$O(n)$ : The orthogonal group}

{}

The set of $n \times n$ matrices $A$ that satisfy $A^{-1} = A^{T}$.

Their determinant is 1 or -1.

This set represents transformations corresponding to rotations and reflections centered at the origin.

\paragraph{$SO(n)$ : The special orthogonal group}

{}

The set of matrices in $O(n)$ whose determinant is 1.

Also known as the rotation group.
$SO(2)$ is the rotation group on a 2-dimensional plane, and
$SO(3)$ is the rotation group in 3-dimensional space.

\paragraph{$U(n)$ : The unitary group}

{}

The set of $n \times n$ complex matrices $A$ that satisfy $A^{-1} = A^{*}$.

This represents the set of transformations corresponding to rotations and reflections in complex space.

\paragraph{$SU(n)$ : The special unitary group}

{}

The set of matrices in $U(n)$ whose determinant is equal to 1.

$SU(n)$ plays an essential role in physics, particularly in the Standard Model of particle physics, where it is used to describe the internal symmetries of particles.

{}

\textbf{SU(2) and Spin}

In quantum mechanics, the state of a particle is described as a vector in a complex vector space.

Transformations (for example, rotations) in this vector space are represented by unitary matrices.

$SU(2)$ is used to describe spin, a quantum mechanical degree of freedom.

Elementary particles like electrons and protons have a property similar to quantum mechanical self-rotation, which is different from classical rotation; this property is called spin.

The matrices of $SU(2)$ are 2 $\times$ 2 unitary matrices with a determinant of 1.

These matrices represent transformations in the 2-dimensional complex vector space (spinor space) that describes the spin states of an electron.

{}

For example, an electron's spin can take one of two states: \textbf{spin-up and spin-down}.

These states are represented by the 2-dimensional vectors
$
	\begin{pmatrix} 1 \\
		0\end{pmatrix}
$
and
$
	\begin{pmatrix} 0 \\
		1\end{pmatrix}
$.

The elements of the $SU(2)$ group can represent any transformation between these two states.
They can also describe changes to superposition states.

$SU(2)$ is related to the rotation group in 3-dimensional space, $SO(3)$.

$SU(2)$ is the double cover of $SO(3)$ and is a mathematical structure that naturally appears when dealing with half-integer spin, like spin.

{}

\textbf{SU(3) and Quark Color Charge}

$SU(3)$ is used to describe the strong interaction.

Quarks, which constitute hadrons like protons and neutrons, have a degree of freedom called color charge.

This color charge comes in three types: red, green, and blue, which are quantum numbers completely unrelated to classical colors.

The matrices of $SU(3)$ are 3 $\times$ 3 unitary matrices with a determinant of 1.

These matrices represent transformations in the 3-dimensional complex vector space that describes the three color charge states.

Gluons, which mediate the strong interaction, play the role of mixing these color charge states, and their transformations follow the $SU(3)$ symmetry.

$SU(n)$ groups mathematically describe the states of particles (spin or color charge).

{}

\textbf{SU(5) and Grand Unification Theory}

$SU(5)$ appears in grand unification theories in particle physics.

The Standard Model expresses the fundamental forces of physics using different symmetry groups:
\begin{itemize}
	\item Strong force: $SU(3)$
	\item Weak force: $SU(2)$
	\item Electromagnetic force: $U(1)$
\end{itemize}

Grand unification theory is based on the idea that these three forces unify into a single force at high energies.

Models that include $SU(3) \times SU(2) \times U(1)$ as a subgroup of a larger, single group have been proposed.

One of the most natural (simple and elegant) candidates for this was $SU(5)$.

One of the phenomena predicted by the $SU(5)$ theory is proton decay.

Protons are considered stable particles, but $SU(5)$ suggests that they may decay with an extremely long but finite lifespan.
Many experiments have attempted to observe proton decay, but none have been successful so far.

There are also other inconsistencies with experimental data, so the $SU(5)$ theory is now considered to be incorrect in its simple form.


\section{Sets of matrices can also be topological spaces}

(In progress.)

\end{document}
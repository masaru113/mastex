\documentclass[uplatex,a4j,12pt,dvipdfmx]{jsarticle}
\usepackage{amsmath,amsthm,amssymb,bm,color,enumitem,mathrsfs,url,epic,eepic,ascmac,ulem,here,ascmac}
\usepackage[letterpaper,top=2cm,bottom=2cm,left=3cm,right=3cm,marginparwidth=1.75cm]{geometry}
\usepackage[english]{babel}
\usepackage[dvipdfm]{graphicx}
\usepackage[hypertex]{hyperref}

\title{\textbf{Fundamentals of Statistical Estimation}}
\author{Masaru Okada}
\date{\today}

\begin{document}

\maketitle

\section{Core Concepts of Statistical Inference: Population and Sample}

In estimative statistics, the distinction between the \textbf{population} under analysis and the \textbf{sample} drawn from it is essential. The population is defined as the entire set of subjects or items to which the analysis results are intended to apply. It is generally set as 'what we wish to know, but cannot know entirely.' A sample, on the other hand, often refers to a subset extracted from the population, or a subset of its characteristic values (such as height or weight). Expressing the set of characteristic values as $\{ x_{1}, x_{2}, ..., x_{n} \} \in \mathbb{R}^{n}$ enables quantitative analysis and inference. When the population size is large and a complete survey is difficult, a \textbf{sample survey} is conducted, wherein the population's characteristics are inferred by analyzing a sample. The estimation and hypothesis testing based on this approach constitute \textbf{statistical inference}.

\subsection*{Characterizing the Population: Parameters and Statistics}

Beyond the population's distribution itself, we often aim to determine the value of a \textbf{constant (parameter)} that characterizes that population. Examples of parameters include the \textbf{population mean}, \textbf{population variance}, \textbf{population standard deviation}, and \textbf{population proportion}. Conversely, a 'function' calculated from the sample is called a \textbf{statistic}. Examples of statistics include the \textbf{sample mean}, \textbf{sample variance}, \textbf{sample standard deviation}, and \textbf{sample proportion}. The goal of inferential statistics is to estimate the value of a population parameter based on the value of a statistic calculated from observed data.



\section{Types of Statistical Studies and Experimental Design}

The role of statistics extends beyond merely estimating means and variances; it also provides \textbf{methods (sampling)} for \textbf{correctly extracting} a sample from a population. Research aimed at understanding population characteristics falls into two categories: \textbf{experimental studies}, which involve \textbf{intervention} on the research subjects, and \textbf{observational studies}, where \textbf{intervention is not possible}.

\subsubsection*{Experimental Design and Fisher's Three Principles}

A study where researchers can control the setting of conditions is termed an experiment, encompassing clinical trials for new drug development and agricultural field trials. In experimental research, patients are typically divided into an \textbf{experimental group} receiving the new drug and a \textbf{control group} receiving a comparator (e.g., a standard drug or placebo), with the difference in outcomes measured. To ensure the validity of this design, \textbf{Fisher's Three Principles} are crucial.

\begin{itemize}
	\item \textbf{Randomization}: The process of assigning the test drug and the control drug to subjects randomly (random assignment). This equalizes the influence of unpredictable bias and enables a fair comparison when other conditions besides the treatment cannot be made uniform. The experimental condition imposed on an experimental unit (subject) is called a '\textbf{treatment}'.
	\item \textbf{Replication}: Since data inherently exhibits variability, repeating the experiment is necessary to determine whether the treatment's effect is discernible beyond that variability. Replication allows for an accurate estimation of the magnitude of variability. In clinical trials involving human subjects, this corresponds to obtaining data from a large number of participants.
	\item \textbf{Local Control (Blocking)}: Experimentation is conducted by dividing the experimental units into several \textbf{blocks} with uniform conditions to minimize variability arising from factors other than the treatment. For instance, it is desirable to set up blocks based on age or gender, ensuring uniformity within blocks while maximizing differences between them.
\end{itemize}

\subsubsection*{Limitations of Observational Studies}

Observational studies become necessary when ethical or practical constraints (such as studying the effects of smoking) prevent experimentation. In observational studies, subjects often choose their own 'treatment', meaning \textbf{randomization}, one of Fisher's Three Principles, \textbf{is not performed}. This characteristic can introduce a bias in treatment selection due to subject characteristics (a \textbf{confounding factor}), necessitating caution in interpreting treatment effects.



\section{Universal Theories and Concepts}
Population, Sample, Parameter, Statistic, Statistical Inference, Census, Sample Survey, Experimental Study, Observational Study, Simple Random Sampling, Fisher's Three Principles, Randomization, Replication, Local Control, Treatment, Experimental Group, Control Group, Blocking, Systematic Sampling, Stratified Random Sampling



\subsection{Comprehension Check Quiz}

\begin{enumerate}
	\item What is the term for the entire set of subjects under analysis, generally large in size and whose true characteristic values are difficult to ascertain?
	\item What do you call a fluctuating value, calculated from a partial group of data taken from a population, used to infer the population's characteristics?
	\item What is the general term for the series of methods used to estimate the unknown characteristic values of a population or to judge the truth of a hypothesis using values calculated from observed data (the sample)?
	\item What is the term for a constant that characterizes a population, whose true value is fixed but generally unknown, used to distinguish it from a value obtained from a sample?
	\item What is the research method that involves deliberately applying specific conditions or manipulations to research subjects to evaluate their effect?
	\item What is the research method that involves collecting and analyzing data from naturally occurring events or existing groups without intervention by the researcher?
	\item In experimental research, what design principle involves leaving the assignment of conditions to subjects to chance to equalize the influence of unexpected external bias and accurately measure the treatment effect?
	\item What principle is necessary to gauge the extent of the variability seen in experimental results and to statistically determine if there is an effect attributable to the treatment that exceeds that variability?
	\item What experimental design principle involves dividing the experiment into small groups (strata) with homogeneous conditions to minimize variability caused by factors other than the treatment?
	\item What term refers to the specific experimental condition or manipulation applied to an experimental unit (such as a subject) that is the object of comparison or effect measurement?
	\item What is the primary group where a new intervention or drug is administered, and the effect is to be measured by the researchers?
	\item What is the group that receives a standard existing treatment or a placebo (dummy drug) for comparison with the effect of a new intervention or drug?
	\item What is the survey method that involves studying only a part of the population and using the findings to infer the characteristics of the whole, rather than surveying the entire population?
	\item In the principle of local control, what is the term for the small, homogeneous subgroup of experimental units?
	\item What problem of bias, caused by factors other than the intervention, arises when assignment to the experimental and control groups depends on the subjects' choices or other non-random factors, potentially leading to a misinterpretation of the treatment effect?
\end{enumerate}

\subsubsection*{Answer Key}
1. Population, 2. Statistic, 3. Statistical Inference, 4. Parameter, 5. Experimental Study, 6. Observational Study, 7. Randomization, 8. Replication, 9. Local Control, 10. Treatment, 11. Experimental Group, 12. Control Group, 13. Sample Survey, 14. Block, 15. Confounding Factor



\begin{thebibliography}{9}
	\bibitem{Vickers}
	What is a p-value anyway $?$ - Vickers, Andrew J.
	\bibitem{toukei_saikyou}
	Statistics is the most powerful discipline (Practical Edition) - Kei Nishiuci
\end{thebibliography}

\end{document}
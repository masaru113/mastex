\documentclass[uplatex,a4j,12pt,dvipdfmx]{jsarticle}
\usepackage{amsmath,amsthm,amssymb,bm,color,enumitem,mathrsfs,url,epic,eepic,ascmac,ulem,here,ascmac}
\usepackage[letterpaper,top=2cm,bottom=2cm,left=3cm,right=3cm,marginparwidth=1.75cm]{geometry}
\usepackage[english]{babel}
\usepackage[dvipdfm]{graphicx}
\usepackage[hypertex]{hyperref}

\title{\textbf{統計的推定の基本事項}}
\author{岡田 大 (Okada Masaru)}
\date{\today}

\begin{document}

\maketitle

\section{統計的推測の基本概念:母集団と標本}

推定統計学において、分析の対象となる\textbf{母集団}と、そこから抽出された\textbf{標本}の区別は不可欠である。母集団とは、分析の結果を適用したい対象となる全体の集合であり、一般に「知りたいが、知り得ないもの」として設定される。一方、標本は母集団から抽出された部分集合、またはその特性値(身長、体重など)の部分集合を指すことが多い。特性値の集合として$\{ x_{1}, x_{2}, ..., x_{n} \} \in \mathbb{R}^{n}$のように表現することで、定量的な分析・推測が可能となる。母集団のサイズが大きく全数調査が困難な場合、標本を分析することで母集団の特性を知ろうとする\textbf{標本調査}が行われる。これに基づく推定や検定が\textbf{統計的推測}である。

\subsection*{母集団の特徴付け:母数と統計量}

母集団そのものの分布に加え、その母集団を特徴づける\textbf{定数(母数)}の値を知りたいというケースもある。母数の例としては、\textbf{母平均}、\textbf{母分散}、\textbf{母標準偏差}、\textbf{母比率}などがある。これに対し、標本から計算される「関数」を\textbf{統計量}と呼ぶ。統計量の例には、\textbf{標本平均}、\textbf{標本分散}、\textbf{標本標準偏差}、\textbf{標本比率}などがある。推測統計では、観測値から計算された統計量の値に基づいて母数の値を推し量ることを目的とする。



\section{統計的研究の種類と実験デザイン}

統計学の役割は、単に平均や分散を推測するだけでなく、母集団から標本を\textbf{正しく取り出すための方法(サンプリング)}を提供する点にもある。母集団の特性を知るための研究には、研究対象に\textbf{介入}を行う\textbf{実験研究}と、\textbf{介入できない}\textbf{観察研究}の2種類がある。

\subsubsection*{実験研究のデザインとフィッシャーの3原則}

条件の設定が研究者自らの手でできる研究を実験と呼び、新薬開発のための臨床試験や農事試験などがこれに含まれる。実験研究では、新薬を投与する\textbf{実験群}と対照薬を投与する\textbf{対照群}に患者を分けてその差を測定する。このデザインの有効性を担保するために、\textbf{フィッシャーの3原則}が重要となる。

\begin{itemize}
	\item \textbf{無作為化(ランダム化)}:試験薬と対照薬の割り付けをランダムに決定すること(無作為割り付け)。予期できない偏りによる影響を均一化し、処理以外の条件を均一にできない場合に公平な比較を可能にする。実験単位(被験者)に課される実験条件を「\textbf{処理}」という。
	\item \textbf{繰り返し}:データにはばらつきがあるため、処理の効果がそのばらつきを超えて見られるかを判断するため、実験を繰り返す必要がある。これにより、ばらつきの大きさを正確に見積もることができる。人間を対象とした臨床試験では、多くの被験者に対するデータ取得がこれに相当する。
	\item \textbf{局所管理(ブロック化)}:処理以外のばらつきを極力小さくするため、実験の条件が均一ないくつかの\textbf{ブロック}に分けて実験を行うこと。例えば、年代や性別をブロックとして設定し、ブロック内では均一に、ブロック間では違いが大きくなるように設定することが望ましい。
\end{itemize}

\subsubsection*{観察研究の限界}

喫煙の影響など、倫理的・現実的な問題で実験ができない場合に観察研究に頼ることになる。観察研究では、被験者が自ら処理を選択するため、フィッシャーの3原則の\textbf{無作為化がなされない}。この特性により、被験者の特性による処理選択の偏り(\textbf{交絡因子})が生じる可能性があり、処置効果の解釈には注意が必要である。



\section{普遍的な理論・コンセプト}
母集団, 標本, 母数, 統計量, 統計的推測, 全数調査, 標本調査, 実験研究, 観察研究, 単純無作為抽出法, フィッシャーの3原則, 無作為化, 繰り返し, 局所管理, 処理, 実験群, 対照群, ブロック化, 系統抽出法, 層化無作為抽出法



\subsection{理解度確認クイズ}

\begin{enumerate}
	\item 分析対象の全体像を示す集合であり、一般にサイズが大きく、特性値の真の値を知ることが困難な対象を何と呼びますか?
	\item 母集団の特性を知る目的で、そこから取り出された部分的なデータ群から計算される、変動する値のことを何と呼びますか?
	\item 観測されたデータ(標本)から計算される値を用いて、母集団の未知の特性値について推し量ったり、仮説の真偽を判断したりする一連の手法を総称して何と呼びますか?
	\item 母集団を特徴づける、真の値は一つに定まっているが一般に未知である定数のことを、標本から得られる値と区別して何と呼びますか?
	\item 研究対象に対して、意図的に特定の条件や操作を加え、その効果を評価する研究手法を何と呼びますか?
	\item 研究者が介入を行うことなく、自然発生的な事象や既存の集団のデータを収集・分析する研究手法を何と呼びますか?
	\item 実験研究において、予期しない外的要因による偏りの影響を均一化し、処理効果を正確に測定するために、被験者への条件割り付けを偶然性に委ねる設計上の原則は何ですか?
	\item 実験の結果に見られるばらつきの大きさを把握し、そのばらつきを超えて処理に起因する効果があるかを統計的に判断するために必要な原則は何ですか?
	\item 処理以外の要因によるばらつきを最小限に抑えるため、均質な条件を持つ小集団(層)に分けて実験を行う実験デザインの原則は何ですか?
	\item 実験単位(被験者など)に施される、比較や効果測定の対象となる特定の実験条件や操作を指す用語は何ですか?
	\item 新しい介入や薬が施される、研究者が効果を測定したい主要なグループを何と呼びますか?
	\item 新しい介入や薬の効果と比較するため、既存の標準治療やプラセボ(偽薬)が施されるグループを何と呼びますか?
	\item 母集団全てを調査するのではなく、その一部を取り出して調査し、全体を推測する調査方法を何と呼びますか?
	\item 局所管理の原則において、均質な実験単位をまとめた小集団のことを何と呼びますか?
	\item 実験群と対照群への割り付けが被験者の意思などに依存することで生じる、処理効果の誤った解釈につながる、介入とは別の要因による偏りの問題を何と呼びますか?
\end{enumerate}

\subsubsection*{解答一覧}
1. 母集団, 2. 統計量, 3. 統計的推測, 4. 母数, 5. 実験研究, 6. 観察研究, 7. 無作為化(ランダム化), 8. 繰り返し, 9. 局所管理, 10. 処理, 11. 実験群, 12. 対照群, 13. 標本調査, 14. ブロック, 15. 交絡因子



\begin{thebibliography}{9}
	\bibitem{Vickers}
	What is a p-value anyway $?$ - Vickers, Andrew J.
	\bibitem{toukei_saikyou}
	Statistics is the most powerful discipline (Practical Edition) - Kei Nishiuci
\end{thebibliography}

\end{document}
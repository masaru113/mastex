\documentclass[uplatex,a4j,12pt,dvipdfmx]{jsarticle}
\usepackage{amsmath,amsthm,amssymb,bm,color,enumitem,mathrsfs,url,epic,eepic,ascmac,ulem,here,ascmac}
\usepackage[letterpaper,top=2cm,bottom=2cm,left=3cm,right=3cm,marginparwidth=1.75cm]{geometry}
\usepackage[english]{babel}
\usepackage[dvipdfm]{graphicx}
\usepackage[hypertex]{hyperref}
\title{Reading Notes: High-Frequency Trading}
\author{Masaru Okada}
\date{\today}
\begin{document}
\maketitle
\tableofcontents

\section{The Rise of AI in the Modern Stock Market}

\subsection{Deepening Market Dominance by HFT and Execution Algorithms}
The trading environment of stock markets has been dramatically transformed in recent years along two axes: \textbf{automation} and \textbf{high-speed operation}, with \textbf{Artificial Intelligence (AI)} at its core. According to a document released by the Financial Services Agency in 2024, a staggering \textbf{90 percent} of all transactions on the Tokyo Stock Exchange are dominated by \textbf{High-Frequency Trading} (HFT) (p. 3). This fact clearly suggests a complete shift in market leadership from traditional human trading, based on experience and intuition, to \textbf{automated, algorithm-driven computer trading}. Consequently, when an individual investor's buy order is executed, the counterparty (i.e., the seller) is presumed in most cases to be an HFT entity operating on an extremely short timescale, underscoring the \textbf{market asymmetry} (p. 3).

The impact of HFT firms is immense, yet they are not exclusively competing against humans (p. 4). A substantial portion of market turnover, alongside HFT activities, is constituted by \textbf{Execution Algorithms} (Exec-Algos). These are computer programs that automatically execute \textbf{large-volume trade instructions} set by institutional or professional investors while minimizing their impact on the market price (p. 4). Furthermore, most of these execution algorithms are \textbf{equipped with AI} that learns the trading environment and market patterns to find the optimal execution method (p. 4). Naturally, HFT firms also leverage sophisticated \textbf{AI technology} to enhance their trading performance to the extreme. The result is that the modern stock market is no longer merely a contest for capital, but a \textbf{'battlefield of AIs,'} where \textbf{two high-performance AI systems compete} for better trading prices (p. 4-5).

\subsection{The Changing Role of Humans and the 'Enslavement' of Individual Investors}
While AI has become the primary market force, the market is not entirely entrusted to fully autonomous AI systems. The ones issuing \textbf{high-level, specific instructions} to the execution algorithms—such as 'which stock, when, and how much to buy'—and designing the fundamental strategies for HFT are still \textbf{professional investors}, who are human (p. 5). This scenario doesn't depict a direct confrontation between humans and AI, but rather a \textbf{hybrid structure of human-AI collaboration}. Professional investors entrust their intent to AI as a sophisticated \textbf{'gladiator'} to fight other AIs, overseeing the contest from the vantage point of the \textbf{Colosseum's spectator seats} and occasionally providing strategic guidance (p. 5).

The author likens the market landscape, dominated by AI, to humans watching a fierce battle between savage AI beasts from the stands (p. 5). However, many individual investors continue to participate in this highly automated, millisecond-driven environment with the \textbf{judgment speed and capabilities of a flesh-and-blood human}, often without grasping the technological and informational asymmetry (p. 5). This constitutes a stark warning about the current, \textbf{extremely disadvantageous and dangerous situation}: a state likened to a \textbf{slave who has mistakenly wandered onto the Colosseum's battlefield} and is forced to confront the powerful AI beasts (p. 5).

\subsection{Automation of Illicit Trading and the Severe Delay in Regulation}
AI functions as a powerful tool to provide liquidity and assist in the fair execution of trades, but its high processing power and anonymity mean it is also exploited for \textbf{illicit trading} (p. 5). A particularly serious ethical and legal issue arises when AI autonomously performs illegal trades, such as \textbf{spoofing}, leading to \textbf{market manipulation}, even if a human did not explicitly instruct the AI to do so (p. 6). In such cases, the question of \textbf{where legal responsibility lies}—\textbf{who should be held legally accountable, and on what grounds}—remains unresolved by current legal and regulatory frameworks (p. 6). The pace of establishing laws and regulations is \textbf{critically lagging} behind the rate at which these new AI-driven problems emerge, representing a pressing challenge in modern financial governance (p. 6).

Yet, those attempting to counter the threat of illicit trading and maintain market fairness must also rely on AI technology. In the field of \textbf{detection and crackdown} on illegal trading by financial authorities and exchanges, the work of detecting and identifying suspicious trading patterns and abnormal transactions is performed not by humans, but by \textbf{sophisticated surveillance AI} (p. 6). Therefore, the modern struggle between financial market malpractices and their enforcement is also fully integrated into the framework of a \textbf{battle between AIs}, competing on technical capability and information processing speed. This dynamic, similar to that in cybersecurity, is expected to grow increasingly complex (p. 6).

\section{Universal Theories and Concepts}
High-Frequency Trading, Execution Algorithm, Artificial Intelligence, Market Manipulation, Spoofing, Legal Responsibility, Market Asymmetry, Colosseum Slave, Surveillance AI, Algorithmic Trading

\subsection{Comprehension Check Quiz}
\begin{enumerate}
	\item According to 2024 FSA data, what is the trading method that accounts for the majority of all transactions on the TSE, involving a large volume of trades in an extremely short time?
	\item What is the collective term for the trading programs used by computers to automatically and efficiently execute trade instructions set by professional investors, while minimizing impact on market price?
	\item What is the author's metaphor for the state of technological competition waged between HFT firms and execution algorithms in pursuit of favorable trading prices in the modern stock market?
	\item What is the term for the automated trading entities, often the counterparties to individual investors' buy or sell orders, that hold a dominant position in the market?
	\item What is the name of the place the author used to exemplify the changing role of AI in the stock market, comparing it to humans observing a battle of beasts from the spectator seats?
	\item What is the social status the author used to liken the situation of many individual investors in an AI-dominated market, forced to fight powerful beasts in the Colosseum?
	\item What is the core technology that has dramatically improved the performance of HFT and execution algorithms in the modern stock market?
	\item What is the act of improperly manipulating market prices—where the location of legal responsibility becomes an issue if the AI performs the act unintentionally—for which legal frameworks are reportedly lagging?
	\item What is the act of placing large orders without the actual intention to trade to mislead other participants, often to improperly manipulate market prices?
	\item What is the legal concept referring to the location of the final entity or person who should be held accountable if an AI commits an illegal act?
	\item What is the technology that is also performing the task of detecting and identifying illegal trades in the stock market, opposing the AIs that commit the malpractices?
	\item What are the two main forms of automated trading that shape the modern stock market environment: High-Frequency Trading and what other form?
	\item What is the term for the specific instructions, such as which security and what volume to trade, that professional investors provide to the AI-driven trading systems?
	\item What is the term for the social norms and regulations that are reportedly lagging behind the problems caused by AI-driven illicit trading?
	\item What is one of the most important goals pursued by the automated trading systems that account for the majority of transactions in the modern stock market?
\end{enumerate}

\subsubsection*{Answer Key}
1. High-Frequency Trading, 2. Execution Algorithm, 3. Battlefield of AIs, 4. High-Frequency Trading Firms, 5. Colosseum, 6. Slave, 7. Artificial Intelligence, 8. Market Manipulation, 9. Spoofing, 10. Legal Responsibility, 11. Surveillance AI, 12. Execution Algorithm, 13. Trade Instructions, 14. Regulatory Frameworks, 15. Trading at a Favorable Price

\section{The Overwhelming Speed of HFT and Asymmetry with Humans}

\subsection{The Reality of Imperceptible Trading and Technological Advantage}
The speed of \textbf{High-Frequency Trading} (HFT) in modern financial markets vastly exceeds human perception, \textbf{going beyond physical limits} (p. 16). For example, the \textbf{order response speed} on the Tokyo Stock Exchange reaches an astonishing \textbf{200 microseconds}. In contrast, the minimum time required for a human to transmit external light from the optic nerve to the brain and recognize it as information is 0.1 seconds, or \textbf{100,000 microseconds} (p. 16). During this \textbf{perceptual latency} of 0.1 seconds, HFT has the capacity to complete \textbf{500 trades}. This signifies an \textbf{asymmetry in information and time}: while a human is making a decision based on past information, the AI has already completed 500 trades (p. 16).

Furthermore, this asymmetry is evident in information display. Even a high-performance PC display requires about \textbf{5,000 microseconds} to update its content. Within this time, HFT can place and cancel \textbf{25 orders}. This means that market changes caused by HFT are unfolding in a domain that is \textbf{fundamentally invisible}—\textbf{not even displayed on the human's screen} during that interval (p. 17). Therefore, it's crucial to understand the fact that an individual investor's claim of 'seeing suspicious order placements and cancellations by HFT' is \textbf{physically impossible from a temporal perspective}; the reality of HFT proceeds in a high-speed, invisible realm (p. 17).

\subsection{HFT Revenue Structure and the Veil of Secrecy}
The activities of HFT firms, particularly their \textbf{trading strategies and revenue structures}, are generally shrouded in a \textbf{veil of secrecy} to maintain a competitive advantage, and their true nature is rarely disclosed to the outside world (p. 17). However, the business model of listed companies, such as the US HFT firm \textbf{Virtu Financial}, offers a glimpse. In the document released to investors during its IPO, the company stated that it \textbf{'only lost money on a single day in 1,238 days'}—a testament to its \textbf{high win rate and stable revenue-generating capability} that is almost unthinkable in other financial businesses (p. 17). This stability suggests that their strategy is specialized in reliably capturing extremely short-term price differences and arbitrage opportunities, making it less susceptible to \textbf{market volatility}.

\section{The Battle of AIs: The Duel with Execution Algorithms}

\subsection{The Role of Execution Algorithms and Acceleration of Evolution through AI}
Besides HFT, computer-driven automated trading, particularly \textbf{Execution Algorithms} (Exec-Algos), is used extensively on stock exchanges (p. 19). An execution algorithm is a program designed to execute the \textbf{large volume of trade orders} conceived by professional investors \textbf{gradually over time}, making the trades \textbf{as inconspicuous as possible to other market participants} (specifically to prevent HFT from detecting their intent and trading at unfavorable prices), thus minimizing \textbf{adverse market impact} (p. 19).

Asset management companies tend to prioritize banks and brokerage firms that possess \textbf{high-performance execution algorithms with high levels of secrecy}, meaning they are less likely to become the \textbf{'prey'} of HFT firms. To meet this market demand, banks and securities firms are engaged in fierce competition to \textbf{improve the performance} of their execution algorithms. At the core of this effort is the essential \textbf{utilization of AI technology} to learn minute market patterns and determine the optimal timing and size of orders (p. 19). Meanwhile, HFT firms, aiming to decode the 'tactics' of execution algorithms and extract profits from them, are also intensifying their \textbf{countermeasures against algorithmic trading} through \textbf{detailed data analysis by AI} (p. 20). This interaction creates a \textbf{loop of strategic and technological evolution}, giving rise to a \textbf{fierce strategic battle between AIs} (p. 20).

\subsection{Market-Making Strategies and the Competition for Ultra-Short-Term Prediction}
Among the battles of AIs in the modern market, a \textbf{fierce technological competition} is particularly central: the struggle between HFT firms' \textbf{market-making strategies} and execution algorithms (p. 21). The market-making strategy aims for \textbf{stable profit} from the price spread by simultaneously presenting both a slightly lower \textbf{limit buy order} and a slightly higher \textbf{limit sell order} for a specific stock (p. 21). The key to this strategy is waiting for other market participants' orders to \textbf{hit their quotes}, making \textbf{speed} (minimizing latency) of utmost importance (p. 21).

However, speed alone is insufficient. In situations where the price \textbf{fluctuates sharply}, such as a sudden stock plunge, market makers face the \textbf{risk of holding inventory purchased at a high price}. To mitigate this risk and \textbf{cut losses quickly}, \textbf{accurate prediction of short-term price changes} is essential (p. 21). To maximize this predictive capability, HFT firms use AI to analyze the vast data of \textbf{all orders gathered at the exchange (the order book)}, allowing the AI to predict which combinations and changes in orders are likely to lead to a price rise or fall (p. 22). While this AI-driven \textbf{pattern recognition and ultra-short-term prediction} is utilized for HFT execution, it also contains an inherent \textbf{vulnerability} specific to AI: a susceptibility to being \textbf{deceived} by \textbf{fake orders} placed without the intention to trade, such as \textbf{spoofing} (p. 22).

\section{AI's Strengths and the Merits and Demerits of HFT}

\subsection{Clear Differentiation of Expertise between AI and Humans}
AI far surpasses humans in its ability to \textbf{instantly analyze the immense order book situation}, predict \textbf{ultra-short-term price fluctuations} from subtle information that the human brain cannot process, and handle the \textbf{execution of stock trades over very short periods} (p. 23). AI's strengths lie in \textbf{data-driven and patternable tasks}. However, current AI systems are extremely poor at \textbf{measuring a company's intrinsic value}, considering macro-economic trends, management quality, and constructing a \textbf{long-term business 'story' that integrates qualitative information} (p. 23). This is a result of AI's evolution diverging from the human \textbf{reasoning process involving thought and emotion}, indicating a ' \textbf{differentiation of capabilities}' where \textbf{AI and humans excel in clearly different domains} (p. 23). Professional investors entrust trades to execution algorithms precisely because they value and leverage this \textbf{ultra-short-term execution capability} of AI (p. 24).

\subsection{The 'Live and Let Live' Dilemma of HFT and Liquidity Provision}
While HFT firms appear as \textbf{'adversaries'} to execution algorithms, seeking to profit from them, their existence is \textbf{indispensable for the smooth functioning} of the market (p. 24). If HFT firms were to disappear, execution algorithms would \textbf{lose their counterparty}, and asset management companies, desiring large trades, would face the problem of \textbf{being unable to trade the volume they want} (p. 24). HFT plays a crucial role in \textbf{constantly supplying liquidity} to the market, ensuring that investors can trade whenever they want (p. 24). Thus, HFT firms are in an \textbf{adversarial yet coexistent relationship}, best described as \textbf{'live and let live'}—necessary for the market, but unwelcome to be exploited by (p. 24).

This complex co-existence is likened to the \textbf{relationship between a used bookstore and its customers} (p. 25). The used bookstore (HFT) profits by buying low and selling high (market-making strategy), basing prices not on the \textbf{content of the book or a company's value}, but on \textbf{market data}, such as the prices of other used bookstores (stock prices) (p. 25). Yet, to customers (other investors), the bookstore is a \textbf{beneficial presence}, providing the \textbf{convenience} (liquidity) of being able to acquire the books they want at a low cost, or having their finished books bought back, at any time (p. 25). The criticism of HFT for trading without considering a company's value is similar to criticizing a used bookstore for pricing books without considering their 'content,' and the author argues that its \textbf{functional value of efficiently supplying liquidity} should not be ignored (p. 25).

\subsection{Directional Strategies and Market Disruption: The Extremity of Information Warfare}
In addition to market-making, HFT employs a \textbf{Directional Strategy} (p. 26). This strategy involves \textbf{predicting the short-term rise or fall of a stock price} and building a position in that direction (p. 26). AI not only analyzes the order book but also instantly parses the \textbf{large volume of articles released online} the moment a \textbf{major news item} like an earnings announcement is published (sometimes even a \textbf{human-free trading cycle} exists, where AI writes the news, and another AI reads it to execute trades), immediately translating the analysis into trades (p. 26). A specific past example involved AI executing a \textbf{yen-selling strategy} during a Bank of Japan monetary policy change announcement, accessing the BOJ website \textbf{at an extremely high frequency}, leading the traffic to be \textbf{misinterpreted as a cyber-attack} (p. 26-27).

This directional strategy can cause \textbf{temporary market disruption}, with concerns that it often involves \textbf{trend-following} (going with the prevailing price movement), which \textbf{accelerates price fluctuations} (p. 27-28). While the market-making strategy contributes to liquidity provision and price stabilization, the directional strategy, through its \textbf{high-speed actions and collective nature}, has the potential to \textbf{destabilize the market} (p. 28).

\subsection{Regulation of HFT and the Debate on Ethical Issues}
The discussion of whether HFT and AI are \textbf{'good or evil'} is one that \textbf{inevitably arises with any new technology}, similar to the historical debate over the safety of automobiles versus horse-drawn carriages (p. 28). The critical point is not the \textbf{existence} of HFT or AI itself, but to discuss and establish \textbf{what rules are most appropriate for all market participants} and how to utilize the technology (p. 28).

Furthermore, HFT presents a major \textbf{dilemma} regarding \textbf{market stability}. Although HFT supplies liquidity during normal times, it is known to \textbf{prioritize its own profits and withdraw from the market when volatility is high and the risk of loss increases} (p. 30). The fact that HFT ceases its function when \textbf{liquidity is most needed—during times of market turmoil}—is a factor that increases \textbf{systemic risk}. This phenomenon was one of the causes of the \textbf{Flash Crash} in the US on May 6, 2010 (p. 30-31), making the debate over the \textbf{social merits and demerits} of HFT a crucial and ongoing topic (p. 30-31).

\section{Universal Theories and Concepts}
High-Frequency Trading, Execution Algorithm, Artificial Intelligence, Market Manipulation, Spoofing, Legal Responsibility, Market Asymmetry, Colosseum Slave, Surveillance AI, Algorithmic Trading, Arbitrage, Market Impact, Colocation Service, FPGA, Liquidity Provision, Negative Fees, Information Fairness, Price Discovery Function, IEX, Flash Crash

\subsection{Comprehension Check Quiz}
\begin{enumerate}
	\item According to 2024 FSA data, what is the trading method that accounts for the majority of all transactions on the TSE, involving a large volume of trades in an extremely short time?
	\item What is the collective term for the trading programs used by computers to automatically and efficiently execute trade instructions set by professional investors, while minimizing impact on market price?
	\item What is the author's metaphor for the state of technological competition waged between HFT firms and execution algorithms in pursuit of favorable trading prices in the modern stock market?
	\item What is the term for the automated trading entities, often the counterparties to individual investors' buy or sell orders, that hold a dominant position in the market?
	\item What is the name of the place the author used to exemplify the changing role of AI in the stock market, comparing it to humans observing a battle of beasts from the spectator seats?
	\item What is the social status the author used to liken the situation of many individual investors in an AI-dominated market, forced to fight powerful beasts in the Colosseum?
	\item What is the core technology that has dramatically improved the performance of HFT and execution algorithms in the modern stock market?
	\item What is the act of improperly manipulating market prices—where the location of legal responsibility becomes an issue if the AI performs the act unintentionally—for which legal frameworks are reportedly lagging?
	\item What is the act of placing large orders without the actual intention to trade to mislead other participants, often to improperly manipulate market prices?
	\item What is the legal concept referring to the location of the final entity or person who should be held accountable if an AI commits an illegal act?
	\item What is the technology that is also performing the task of detecting and identifying illegal trades in the stock market, opposing the AIs that commit the malpractices?
	\item What are the two main forms of automated trading that shape the modern stock market environment: High-Frequency Trading and what other form?
	\item What is the term for the specific instructions, such as which security and what volume to trade, that professional investors provide to the AI-driven trading systems?
	\item What is the term for the social norms and regulations that are reportedly lagging behind the problems caused by AI-driven illicit trading?
	\item What is one of the most important goals pursued by the automated trading systems that account for the majority of transactions in the modern stock market?
\end{enumerate}

\subsubsection*{Answer Key}
1. High-Frequency Trading, 2. Execution Algorithm, 3. Battlefield of AIs, 4. High-Frequency Trading Firms, 5. Colosseum, 6. Slave, 7. Artificial Intelligence, 8. Market Manipulation, 9. Spoofing, 10. Legal Responsibility, 11. Surveillance AI, 12. Execution Algorithm, 13. Trade Instructions, 14. Regulatory Frameworks, 15. Trading at a Favorable Price

\section{The Genesis of HFT: The Digitization of Human Strategy}

\subsection{The History of Arbitrage and Proprietary Trading Desks}
While High-Frequency Trading (HFT) is a product of modern technology, many of its strategies have their roots in trading methods \textbf{once executed by human hands} (p. 34). In addition to the market-making and directional strategies already discussed, a key HFT strategy is \textbf{Arbitrage} (p. 34). Arbitrage was actively conducted by the \textbf{proprietary trading desks} of securities firms long before HFT was born, dating back to the early days of stock trading (p. 34).

These arbitrage and market-making strategies executed by securities firms not only pursued self-interest but also played a role in \textbf{maintaining the functionality of the stock market} and providing an attractive trading environment for their clients (p. 34). Essentially, HFT can be described as the \textbf{digitization and sophistication} of the market functions and profit-seeking activities previously undertaken by these proprietary trading desks, using computers and algorithms (p. 35). Subsequently, many \textbf{newly entered HFT specialist firms}, focused exclusively on increasing their own capital, emerged (p. 35).

\subsection{The Reduction in Trading Costs Driven by HFT}
The high-speed transition of proprietary trading desks at securities firms, coupled with the entry of specialized HFT firms, has, from a societal perspective, led to \textbf{cost reductions} and a significant \textbf{lowering of the overall trading costs paid by general investors} (p. 35). The overall trading costs here include not only the \textbf{transaction fees} paid directly to the brokerage firm but also factors like \textbf{market impact} (the price fluctuation caused by one's own trade) (p. 35). Thus, the emergence of HFT has a positive side: it enhances market efficiency, and \textbf{general investors also benefit} (p. 35). However, the barrier to entry for HFT—requiring the introduction of expensive equipment and highly skilled engineers—is high, and many smaller brokerage firms were \textbf{forced to close down} because they could not adapt (p. 36). Today, human-driven market-making and arbitrage have been almost entirely replaced by HFT in Japan (p. 37).

\section{The Extremity of Speed Competition: The Sub-Millisecond World}

\subsection{Pursuit of Competitive Advantage through Colocation and Dedicated Circuitry}
In HFT, \textbf{speed} (\textbf{minimizing latency}) is critically important. Particularly in strategies like arbitrage, where \textbf{only the firm with the least delay captures all the profit}, even a slight delay directly translates into a difference in \textbf{revenue} and can be fatal (p. 37, 39). The speed here focuses on \textbf{how quickly an order reaches the exchange} after being placed (p. 37). HFT firms utilize \textbf{colocation services}, allowing them to place their computers next to the exchange's trading system in the same data center, \textbf{minimizing the physical distance} to the extreme (p. 39). Furthermore, the cables connecting each firm's computer to the exchange system are adjusted to be \textbf{exactly the same length to ensure fairness} (p. 39).

However, latency occurs not only due to physical distance but also during \textbf{computational processing within the computer} (p. 40). The entire process of the CPU reading a program from memory, performing calculations, and outputting the result takes time that is unacceptable in the world of HFT. To minimize this computational latency, HFT firms introduce \textbf{hardware technology} called \textbf{FPGA} (Field-Programmable Gate Array), where the necessary calculations for trading are \textbf{pre-built as dedicated electrical circuits} (p. 40-41). Since FPGAs specialize in \textbf{performing the same calculations with extremely low latency}, they are the optimal solution for the \textbf{extremity of the speed competition} that is HFT. Ideally, the pursuit is for a \textbf{processing speed at the electrical signal level}, where the data received from the exchange simply flows through the circuit to output the data to be sent back to the exchange (p. 40, 41).

\section{The Merits and Demerits of HFT and the Definition of 'Unfair'}

\subsection{The Liquidity Provision Dilemma: A Reality of Scarcity When Most Needed}
One of the merits of HFT is its \textbf{provision of liquidity} to the market, but we must also consider its \textbf{'demerits'} (p. 42). Since HFT trades to \textbf{maximize its own profit}, it \textbf{immediately stops participating in trades when there is a risk of loss} (p. 43). The market-making strategy profits during times of \textbf{stable price oscillation}, but the risk of loss dramatically increases during \textbf{sharp price movements}, such as sudden surges or plunges, due to increased inventory risk (p. 43).

Therefore, when prices fluctuate rapidly and \textbf{volatility remains high}, the market-making strategy decides to \textbf{withdraw and observe} (p. 43). This poses a significant problem for market stability: HFT \textbf{disappears from the market and stops supplying liquidity} precisely when \textbf{liquidity is most needed} due to increased investor trading needs and market turmoil (p. 43). To prevent this \textbf{lack of liquidity at critical junctures}, some exchanges have implemented \textbf{preferential systems} where HFT firms are provided with a \textbf{negative fee} (rebate) in exchange for a contractually obligated commitment to maintain continuous limit orders (p. 44).

\subsection{Is HFT 'Unfair'? Information Fairness and Market Efficiency}
The definition of \textbf{'unfair'} in stock trading focuses on \textbf{information inequality in the market} caused by using \textbf{information obtainable only through a special position} (p. 47). In other words, while market participants are considered to be fairly competing on personal abilities like \textbf{prediction skills and study effort}, the presence of \textbf{unfairness in the information obtained itself} is viewed as problematic (p. 47).

Indeed, some exchanges, such as \textbf{IEX}, have implemented designs that intentionally make HFT difficult (for example, slightly delaying order processing), but IEX has not significantly expanded its trading share (p. 49). This is due to the concern that if 'unfair' trading were eliminated, market functionality would decline, leading to a failure to provide the market's essential \textbf{price discovery function} (the formation of a fair price) and \textbf{liquidity provision} (p. 49). Currently, the assessment is that HFT is \textbf{not significantly disrupting the price discovery function}; rather, its \textbf{functional value of supplying liquidity} is appreciated. Thus, the \textbf{'current conclusion'} is that HFT is \textbf{not 'unfair' enough to be excluded} (p. 49-50). However, this evaluation carries a latent risk: it could change at any time if \textbf{more cunning and unfair 'gimmicks'} are invented in the future (p. 50).

\begin{thebibliography}{9}
	\bibitem{mizuta2025} Takanobu Mizuta, \textbf{High-Frequency Trading: The Market Impact of AI on the Stock Market}, Seikaisha Shinsho, 2025.
\end{thebibliography}
\end{document}
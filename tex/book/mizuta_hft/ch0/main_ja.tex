\documentclass[uplatex,a4j,12pt,dvipdfmx]{jsarticle}
\usepackage{amsmath,amsthm,amssymb,bm,color,enumitem,mathrsfs,url,epic,eepic,ascmac,ulem,here,ascmac}
\usepackage[letterpaper,top=2cm,bottom=2cm,left=3cm,right=3cm,marginparwidth=1.75cm]{geometry}
\usepackage[english]{babel}
\usepackage[dvipdfm]{graphicx}
\usepackage[hypertex]{hyperref}
\title{読書メモ:高速取引(はじめに)}
\author{Masaru Okada}
\date{\today}
\begin{document}
\maketitle
\tableofcontents

\section{はじめに}
現代の株式市場における取引のあり方は、劇的に変化している。金融庁が2024年に公開した資料によれば、東京証券取引所における全取引のうち、実に\textbf{9割が高速取引}によって占められているという (p. 3)。この事実から、個人投資家が取引を約定させた際、その相手方の売り手(または買い手)は、ほとんどの場合\textbf{高速取引業者}であると推測される (p. 3)。

\section{現代市場の構造とAIの役割}
\subsection*{AI主導の市場環境}
現在の株式市場では、高速取引に対峙しているのは人間ではなく、\textbf{AI}である (p. 4)。取引の多くは、高速取引業者によるものに加え、プロの投資家が決定した売買指示をコンピューターが自動で実行する\textbf{執行アルゴリズム}によって占められている (p. 4)。これらの執行アルゴリズムのほとんどにはAIが搭載されており、高速取引業者もまたAIによってその能力を強化している (p. 4)。結果として、市場では両者のAIがより有利な価格で取引を成立させようと\textbf{熾烈な競争を繰り広げている} (p. 4, p. 5)。

\subsection*{人間とAIの関係}
この状況は、株式市場が\textbf{AI同士の戦場}になっていることを意味する (p. 5)。しかし、この戦いの設計者、すなわち高速取引の基本戦略を構築し、執行アルゴリズムに対してどの銘柄をどれくらい売買するかといった指示を出しているのは、あくまで\textbf{プロの投資家である人間}である (p. 5)。このため、現代の市場は「人間 対 AI」の構図ではなく、プロの投資家がAIに取引を担わせ、他のAIと戦わせる様子を、あたかも\textbf{コロッセオの観客席から猛獣であるAI同士の戦いを眺めている}かのように俯瞰している状況と表現されている (p. 5)。

\subsection*{個人投資家の立ち位置}
一方で、多くの個人投資家は、このようなAI同士の高度な戦いに、\textbf{生身の人間のまま参加}しているのが実情である (p. 5)。これは、誤って\textbf{コロッセオの戦場に紛れ込んでしまった奴隷が、猛獣であるAIと戦う羽目になってしまっている}現実に例えられている (p. 5)。

\section{AIと不正取引・法の課題}
AIは、適法かつ公正な取引に利用される一方で、\textbf{不正取引}にも悪用されている (p. 5)。具体的には、人間が意図しないにも関わらず、AIによる\textbf{見せ玉}などの不正な取引が相場操縦を引き起こす事例がある (p. 6)。このような場合、誰に責任があるのか、という点で\textbf{法律の整備が追いついていない}という深刻な課題が存在する (p. 6)。しかし、その不正取引の発見と取り締まりもまた\textbf{AIが行っている}ため、この分野においても「不正取引と取り締まり」という名の\textbf{AI同士の戦い}が繰り広げられている (p. 6)。

\section{普遍的な理論・コンセプト}
高速取引, 執行アルゴリズム, AI(人工知能), 相場操縦, 見せ玉, コロッセオの例え

\section{理解度確認クイズ}
\subsection{理解度確認クイズ}
\begin{enumerate}
	\item 東京証券取引所における全取引のうち、2024年公開の金融庁資料で約9割を占めるとされる取引手法は何ですか。
	\item プロの投資家が売買の指示を出し、コンピューターが自動で実行する取引アルゴリズムの名称を、資料で言及されているとおりに記述しなさい。
	\item 現代の株式市場において、高速取引に対峙している主体は人間ではなく何であると述べられていますか。
	\item 著者は、プロの投資家とAIの関係を、人間とAIが直接戦っているのではなく、どのような競技場になぞらえて説明していますか。
	\item 著者は、多くの個人投資家がAI主導の市場に参加している状況を、コロッセオにおいて誰と誰が戦う状況に例えて批判的に表現していますか。
	\item 市場における適法な利用だけでなく、不正取引にも悪用されている技術は何ですか。
	\item 人間が意図しないのにAIが実行し、相場操縦につながる可能性が指摘されている不正な取引の一例を、資料で挙げられている用語で記述しなさい。
	\item AIによる不正な取引で相場操縦が発生した場合、法律の整備が追いついていないとされる、責任の所在に関する問題は何ですか。
	\item 不正な取引が市場で起きている一方で、その取り締まりにおいて不正取引の発見を担っている技術は何ですか。
	\item 現代の株式市場におけるAI同士の戦いにおいて、高速取引の基本的な戦略を設計し、指示を与えているのは誰ですか。
	\item 高速取引の取引相手として、個人投資家が最も遭遇しやすいのはどのような業者ですか。
	\item 執行アルゴリズムの能力強化に不可欠な、現在ほとんどの場合に搭載されている技術は何ですか。
	\item AIの取引能力向上により、市場における価格競争でより有利な価格での取引を目指しているのはどの主体ですか。
	\item 現代の市場をAI同士が戦っている場であると表現する一方で、その戦いを上から眺め、時々指示を出しているのはどのような立場の人物ですか。
	\item 不正取引とその取り締まりの間で繰り広げられている、AI同士の戦いの対象は何ですか。
\end{enumerate}

\subsubsection*{解答一覧}
1. 高速取引, 2. 執行アルゴリズム, 3. AI(人工知能), 4. コロッセオ, 5. 奴隷と猛獣, 6. AI, 7. 見せ玉, 8. 誰に責任があるのか, 9. AI, 10. 人間(プロの投資家), 11. 高速取引業者, 12. AI, 13. 両AI(高速取引業者と執行アルゴリズム), 14. プロの投資家, 15. 不正取引

\begin{thebibliography}{9}
	\bibitem{mizuta2025} 水田孝信, \textbf{高速取引 株式市場にAIがもたらすマーケット・インパクト}, 星海社新書, 2025年.
\end{thebibliography}

\end{document}
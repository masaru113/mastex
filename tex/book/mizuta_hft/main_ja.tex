\documentclass[uplatex,a4j,12pt,dvipdfmx]{jsarticle}
\usepackage{amsmath,amsthm,amssymb,bm,color,enumitem,mathrsfs,url,epic,eepic,ascmac,ulem,here,ascmac}
\usepackage[letterpaper,top=2cm,bottom=2cm,left=3cm,right=3cm,marginparwidth=1.75cm]{geometry}
\usepackage[english]{babel}
\usepackage[dvipdfm]{graphicx}
\usepackage[hypertex]{hyperref}
\title{読書ノート:高速取引}
\author{Masaru Okada}
\date{\today}
\begin{document}
\maketitle
\tableofcontents

\section{現代株式市場におけるAIの台頭}

\subsection{高速取引と執行アルゴリズムによる市場支配の深化}
近年、株式市場の取引環境は\textbf{自動化}と\textbf{高速化}という二つの軸で劇的に変化しており、その中核には\textbf{人工知能(AI)}の存在があります。金融庁が2024年に公開した資料によれば、東京証券取引所における全取引の実に\textbf{9割}という圧倒的な割合が\textbf{高速取引}(High-Frequency Trading, HFT)によって占められています (p. 3)。この事実は、市場の主役が従来の人間による経験や直感に基づいたトレーディングから、\textbf{アルゴリズムに基づくコンピューターによる自動売買}へと完全に移行したことを明確に示唆しています。そのため、個人投資家が市場で買い注文を約定させた際、その売買の相手方、すなわち売り手のほとんどは、極めて短い時間で取引を行う高速取引を担う主体であると推測され、\textbf{市場の非対称性}が浮き彫りになります (p. 3)。

高速取引業者が市場に与える影響は計り知れませんが、彼らが唯一対峙しているのは人間ではありません (p. 4)。市場の売買の大部分を構成しているのは、高速取引の活動に加え、機関投資家やプロの投資家が決定した\textbf{大量の売買指示}を、市場価格への影響を抑えつつコンピューターが自動で実行する\textbf{執行アルゴリズム}(Execution Algorithm)という取引形態です (p. 4)。そして、この執行アルゴリズムのほとんどには、取引環境や市場のパターンを学習し、最適な執行方法を見つけ出す\textbf{AIが搭載}されています (p. 4)。当然のことながら、高速取引業者側も、より洗練された\textbf{AI技術}を駆使してその取引性能を極限まで強化しています。結果として、現代の株式市場は、単なる資金の奪い合いではなく、より有利な価格での取引を求めて\textbf{二つの高性能なAIシステム同士が戦う「AI同士の戦場」}となっているのです (p. 4-5)。



\subsection{人間の役割の変化と個人投資家の「奴隷」化}
市場の主体がAIに移行したとはいえ、市場が完全に自律的なAIシステムに委ねられているわけではありません。高速取引の基本的な戦略設計や、執行アルゴリズムに対する「どの銘柄を、いつ、どれくらいの量で買いたいか」といった\textbf{上位レベルの具体的な指示}を出しているのは、あくまで\textbf{プロの投資家}という人間です (p. 5)。この状況は、人間がAIと直接対決する構図ではなく、むしろプロの投資家が自身の意図をAIという\textbf{高度な「闘士」}に託し、他のAIと戦わせる様子を、あたかも\textbf{コロッセオの観客席}から俯瞰し、時折戦略的な指示を与えるという、\textbf{人間とAIのハイブリッドな協働構造}を示しています (p. 5)。

筆者は、このAIが支配する市場の風景を、観客席から猛獣であるAI同士の激しい戦いを眺める人間に例えています (p. 5)。一方で、この高度に自動化され、ミリ秒単位で戦いが繰り広げられる市場環境に、多くの個人投資家は、その技術的・情報的な非対称性を理解しないまま、\textbf{生身の人間の判断速度と能力}で参加し続けています (p. 5)。これは、誤って\textbf{コロッセオの戦場に紛れ込んでしまった奴隷}が、強力な猛獣であるAIと戦わざるを得ないという、\textbf{極めて不利で危険な状況}に置かれているという、現状に対する厳しい警鐘となっています (p. 5)。



\subsection{不正取引の自動化と法規制の深刻な遅延}
AIは、市場に流動性を提供し、公正な取引の遂行を支援する強力なツールとして機能する一方で、その高い処理能力と匿名性ゆえに\textbf{不正取引}にも悪用されてしまっています (p. 5)。特に深刻な倫理的・法的な問題となっているのは、人間が意図的に不正を指示しなかったにもかかわらず、AIが自律的に\textbf{見せ玉}(Spoofing)などの不正な取引を行い、結果として\textbf{相場操縦}に至った場合です (p. 6)。このケースにおいて、\textbf{一体誰に、どのような根拠で法的責任を負わせるべきなのか}という\textbf{責任の所在}の問題が、現在の法律や規制の枠組みでは明確に解決されていません (p. 6)。法律や規制の整備の速度が、このAIが引き起こす新たな問題の出現速度に\textbf{決定的に追いついていない}のが、現代の金融ガバナンスにおける喫緊の課題となっています (p. 6)。

しかし、この不正取引の脅威に対抗し、市場の公正さを維持しようとする側もまた、AI技術に頼らざるを得ない状況です。金融当局や取引所が行う不正取引の\textbf{摘発と取り締まりの分野}においても、怪しい取引パターンや異常な売買を検出・特定する作業は、人間ではなく\textbf{高度な監視AI}が行っています (p. 6)。したがって、現代の金融市場における不正行為とその取り締まりを巡る攻防もまた、技術的な能力と情報処理速度を競い合う\textbf{AI同士の戦い}という構図の中に完全に収斂されていると言えるでしょう。このサイバーセキュリティにおける攻防と同様の構図は、今後も複雑化の一途を辿ると予想されます (p. 6)。

\section{普遍的な理論・コンセプト}
高速取引, 執行アルゴリズム, 人工知能, 相場操縦, 見せ玉, 法的責任, 市場の非対称性, コロッセオの奴隷, 監視AI, アルゴリズム取引

\subsection{理解度確認クイズ}
\begin{enumerate}
	\item 2024年の金融庁資料によると、東証の全取引の大部分を占めている、極めて短い時間で大量の売買を行う取引手法は何と呼ばれますか。
	\item プロの投資家が設定した売買の指示を、市場価格への影響を抑えながらコンピューターが自動的かつ効率的に実行するために用いられる取引プログラムの総称は何ですか。
	\item 現代の株式市場において、有利な取引価格を求めて高速取引業者と執行アルゴリズムの間で繰り広げられている、技術間の競争状態を指す筆者の比喩表現は何ですか。
	\item 買い注文や売り注文を約定させる際、個人投資家の相手方となることの多い、市場で優位な立場にある自動売買を行う主体のことを何と呼びますか。
	\item 株式市場におけるAIの役割の変化を、人間が観客席から猛獣の戦いを眺める光景に例えた場所の名前は何ですか。
	\item 筆者が、多くの個人投資家がAIに支配された市場で置かれている状況を、コロッセオの猛獣と戦う立場に例えた社会的地位は何ですか。
	\item 現代の株式市場において、高速取引や執行アルゴリズムの性能を劇的に向上させている中核技術は何ですか。
	\item 法律の整備が追いついていないとされる、AIが意図せず行ってしまった場合に、その責任の所在が問題となる、市場価格を不正に操作する行為は何ですか。
	\item 市場価格を不正に操作するために、売買する意思がないにもかかわらず大量の注文を出して他の参加者の誤解を誘う行為を何と呼びますか。
	\item AIによる不正行為が発覚した場合に、最終的にその行為の責任を負うべき対象や人の所在を指す法律上の概念は何ですか。
	\item 株式市場における不正な取引の発見と特定を担い、不正行為を行うAIと対立する役割を果たしている技術もまた何ですか。
	\item 現代の株式市場の取引環境を形作る主要な二つの自動取引形態とは、高速取引ともう一つは何ですか。
	\item プロの投資家がAIを用いた取引システムに対して提供する、どの銘柄をどれくらい売買するかといった具体的な指示を指す言葉は何ですか。
	\item AIによる不正取引が引き起こす問題に対して、現在追いついていないとされる、社会的な規範や規定を指す言葉は何ですか。
	\item 現代の株式市場において、取引の大部分を占める自動売買システムが目指している最も重要な目的の一つは何ですか。
\end{enumerate}

\subsubsection*{解答一覧}
1.高速取引, 2.執行アルゴリズム, 3.AI同士の戦場, 4.高速取引業者, 5.コロッセオ, 6.奴隷, 7.人工知能, 8.相場操縦, 9.見せ玉, 10.法的責任, 11.監視AI, 12.執行アルゴリズム, 13.売買指示, 14.法律の整備, 15.有利な価格での取引

\section{高速取引の圧倒的な速度と人間との非対称性}

\subsection{人間には知覚不可能な取引の現実と技術的優位性}
現代の金融市場における\textbf{高速取引}(HFT)の速度は、人間の知覚能力を\textbf{物理的な限界を超えて}はるかに凌駕しています (p. 16)。例えば、東京証券取引所における\textbf{注文応答速度}はわずか\textbf{200マイクロ秒}という驚異的な水準に達しています。これに対し、人間が外界の光を視神経から脳に伝達し、情報として認識するまでには最低でも0.1秒、すなわち\textbf{10万マイクロ秒}という時間を要します (p. 16)。この0.1秒という人間の\textbf{認識の遅延}の間に、高速取引は\textbf{500回もの取引}を完了させる能力を持っています。これは、人間が過去の情報を基に判断している間に、AIは既に500回も売買を完遂させているという、\textbf{情報と時間における非対称性}を意味します (p. 16)。

さらに、この非対称性は、情報の表示面でも顕著です。高性能なパソコンのディスプレイでさえ、表示内容の更新には約\textbf{5000マイクロ秒}が必要です。この時間、高速取引は\textbf{25回}もの注文を出し入れすることができます。すなわち、高速取引による市場の変化は、その間の人間の\textbf{ディスプレイにはそもそも表示すらされていない}という、\textbf{本質的に不可視な領域}で進行しているのです (p. 17)。したがって、個人投資家が「高速取引による怪しい注文の出し入れを見た」と主張することは、\textbf{時間的な観点から物理的にありえない}ことであり、高速取引の実態は人間には見えない、超高速の領域で進行しているという事実を理解することが重要です (p. 17)。



\subsection{高速取引の収益構造と秘密主義のベール}
高速取引業者の活動、特にその\textbf{取引戦略や収益構造}は、競争優位性を保つために一般に\textbf{秘密主義}のベールに包まれており、その実態が外部に明かされることは稀です (p. 17)。しかし、米国の高速取引業者である\textbf{Virtu Financial}のように上場している企業からは、その驚異的なビジネスモデルの一端が垣間見えます。同社が上場した際に投資家向けに公開した資料には、「\textbf{1238日のうち負けたのはわずか1日だけ}」という、他の金融ビジネスでは考えられないほどの\textbf{高い勝率と安定した収益能力}を示す記述があり、これは高速取引というビジネスの優位性を世界に知らしめることとなりました (p. 17)。この安定性は、彼らの戦略が\textbf{市場のボラティリティ(変動)}に左右されにくい、極めて短期的な価格差や裁定機会を確実に捉えることに特化していることを示唆しています。

\section{AI同士の戦い:執行アルゴリズムとの攻防}

\subsection{執行アルゴリズムの役割とAI技術による進化の加速}
高速取引以外にも、証券取引所ではコンピューターによる自動売買、特に\textbf{執行アルゴリズム}(Exec-Algo)がきわめて頻繁に使用されています (p. 19)。執行アルゴリズムは、プロの投資家が考えた\textbf{大量の売買注文}を、市場への\textbf{悪影響(マーケット・インパクト)を最小限}に抑えつつ(具体的には高速取引にその意図を察知され、不利な価格で取引されるのを防ぐため)、\textbf{他の取引参加者になるべく悟られないように、時間をかけて少しずつ執行}するためのプログラムです (p. 19)。

資産運用会社は、高速取引業者の\textbf{「餌食」になりにくい}、すなわち\textbf{秘匿性の高い高性能な執行アルゴリズム}を持つ銀行や証券会社を優先的に選ぶ傾向があります。この市場の要求に応えるため、銀行や証券会社は執行アルゴリズムの\textbf{性能向上}に激しい競争を繰り広げており、その中核には、市場の微細なパターンを学習し、最適な注文のタイミングや量を決定する\textbf{AI技術の活用}が不可欠となっています (p. 19)。一方で、執行アルゴリズムの「手口」を解読し、そこから利益を奪い取ろうとする高速取引業者側も、\textbf{AIによる詳細なデータ分析}を通じて\textbf{アルゴリズム取引への対策強化}を図っており (p. 20)、この相互作用が\textbf{技術と戦略の進化のループ}を生み出し、ここに\textbf{AI同士の激しい戦略的戦い}が生まれています。



\subsection{マーケットメイク戦略とAIによる超短期予測の競争}
現代市場のAI同士の戦いの中でも特に中心的な役割を担い、\textbf{熾烈な技術競争}が繰り広げられているのが、高速取引業者の\textbf{マーケットメイク戦略}と執行アルゴリズムとの攻防です (p. 21)。マーケットメイク戦略は、特定の銘柄について、少し安い\textbf{指値の買い注文}と少し高い\textbf{指値の売り注文}の両方を同時に提示し、市場の価格の往復運動(スプレッド)から\textbf{安定的な利益}を得ることを目的とします (p. 21)。この戦略の鍵は、他の市場参加者の注文が\textbf{自分たちの注文にぶつかる}のを待つことにあり、そのためには\textbf{速さ}(レイテンシーの最小化)が最も重要となります (p. 21)。

しかし、速さだけでは不十分です。価格が\textbf{大きく変動する局面}、例えば株価が急落するような場合、マーケットメーカーは\textbf{高値で掴んだ在庫を抱えるリスク}を負うことになります。このリスクを回避し、\textbf{いち早く損切り}を行うためには、\textbf{短期的な価格変動の正確な予測}が不可欠です (p. 21)。この予測能力を極限まで高めるため、高速取引業者は\textbf{取引所に集まった全ての注文の状況(注文板)}の膨大なデータをAIに学習させ、どのような注文の組み合わせや変化が価格の上昇や下落に繋がりやすいかを事前に分析させています (p. 22)。このAIによる\textbf{パターン認識と超短期予測}は高速取引の実行に活用されますが、同時に、取引する意図のない\textbf{フェイクの注文}、すなわち\textbf{見せ玉}(Spoofing)といった不正な操作に対して\textbf{騙されやすい}という、AI特有の\textbf{脆弱性}も内包しています (p. 22)。

\section{AIの得意分野と高速取引の功罪}

\subsection{AIと人間の得意分野の明確な分化}
AIは、\textbf{膨大な注文板の状況を瞬時に分析}し、人間の脳では処理しきれない微細な情報から\textbf{超短期的な価格変動を予測}する能力においては人間を遥かに超越し、\textbf{きわめて短期間での株式取引執行}を担うことができます (p. 23)。AIの得意分野は、\textbf{データ駆動型でパターン化できるタスク}にあります。しかし、AIは企業の\textbf{本質的価値を測定}したり、マクロ経済の動向、経営者の資質といった\textbf{定性的な情報を織り交ぜた長期的なビジネスの「ストーリー」を考える}ことは、現在の仕組みでは極めて苦手としています (p. 23)。これは、AIの仕組みが人間の\textbf{思考や感情を伴う推論プロセス}から離れて進化している結果であり、\textbf{AIと人間は得意とする分野が明確に異なってきている}という「\textbf{能力の分化}」が起こっていることを示しています (p. 23)。プロの投資家が執行アルゴリズムに取引を委ねるのは、このAIの\textbf{超短期執行能力}を最大限に評価し、活用しているためです (p. 24)。



\subsection{高速取引と流動性供給の「生かさず殺さず」のジレンマ}
執行アルゴリズムにとって、高速取引業者は自分たちの利益を奪い取る\textbf{「敵対者」}のように見えますが、その存在は市場の\textbf{円滑な機能維持}にとって不可欠です (p. 24)。高速取引業者がいなくなると、執行アルゴリズムは\textbf{取引の相手を失い}、大量の取引を望む運用会社は、\textbf{取引したい量を取引できなくなる}という問題に直面します。高速取引は、取引したいときにいつでも取引させてくれるという\textbf{流動性}(Liquidity)を市場に\textbf{常に供給}する重要な役割を担っています (p. 24)。このため、高速取引業者は「いなくなると困るけれども、ぼったくられたくない」という、まさに\textbf{「生かさず殺さず」}がちょうどよい、\textbf{敵対的かつ共存的な関係}にあると言えます (p. 24)。

この複雑な共存関係は、\textbf{古本屋とその客の関係}に例えられます (p. 25)。古本屋(高速取引)は、本の\textbf{中身や企業の価値}ではなく、他の古本屋の価格(株価)といった\textbf{市場のデータ}を参考に安く買い高く売る(マーケットメイク戦略)ことで利益を得ます (p. 25)。しかし、客(他の投資家)にとっては、読みたい本をいつでも低コストで入手できる、または読み終わった本をいつでも買い取ってくれるという\textbf{利便性}(流動性)を提供してくれる\textbf{ありがたい存在}なのです (p. 25)。高速取引が企業の価値を考慮せず取引することへの批判は、古本屋が本の「中身」を考えずに価格を決めることを批判するのと同様であり、その\textbf{効率的な流動性供給という機能的な価値}を無視すべきではないという主張です (p. 25)。



\subsection{ディレクショナル戦略と市場の混乱:情報戦の極限}
高速取引には、マーケットメイク戦略の他にも、市場の方向性に賭ける\textbf{ディレクショナル戦略}(Directional Strategy)が使われます (p. 26)。これは、\textbf{短期的な株価の上昇・下落を予想}し、その方向へポジションを構築する戦略です (p. 26)。AIは、注文板の分析だけでなく、\textbf{決算発表などの重大なニュース}が出た瞬間に、ネット上に\textbf{一気に公開される大量の記事}(時にはAIが記事を作成し、それをAIが読んで取引する\textbf{人間を介さない取引サイクル}も存在する)を即座に解析し、取引に結びつけます (p. 26)。過去には、日本銀行の金融政策の変更発表時に、AIが\textbf{円売り戦略}を実行するために、日銀のホームページに\textbf{極めて高頻度でアクセス}し、そのトラフィックが\textbf{サイバー攻撃と誤解された}という具体的な事例もあります (p. 26-27)。

このディレクショナル戦略は、市場に\textbf{一時的な混乱}をもたらすことがあり、特に\textbf{順張り}(トレンド追随)といった\textbf{価格変動に拍車をかける方向の取引が多い}のではないかという懸念が示されています (p. 27-28)。マーケットメイク戦略が流動性の供給と価格の安定化に寄与する側面を持つ一方で、ディレクショナル戦略は、その\textbf{高速な行動と集団性}により、\textbf{市場を荒らす原因}となる可能性があります (p. 28)。



\subsection{高速取引のルール整備と倫理的課題の議論}
高速取引やAIを\textbf{「正義か悪か」}という二元論で捉える議論は、\textbf{新しい技術が生まれた際に必ず起きる}議論であり、過去に自動車が登場した際に馬車との安全性を巡って議論された状況と類似しています (p. 28)。重要なのは、高速取引やAIそのものの\textbf{存在是非}ではなく、その技術をどのように活用し、\textbf{どのようなルールが市場参加者全体にとって最も適切なのか}を議論し、整備することです (p. 28)。

さらに、高速取引には\textbf{市場の安定性}に関する大きな\textbf{ジレンマ}が存在します。高速取引は通常時に流動性を供給しているにもかかわらず、\textbf{市場が荒れて価格変動リスクが高まると、損をする可能性が高まるため、自己の利益を優先して取引をやめて市場から撤退する}ことが知られています (p. 30)。流動性が\textbf{最も必要とされる「市場が荒れている時」}に、高速取引がその機能を停止するという事実は、\textbf{システミック・リスク}を高める要因となりえます。この現象は、2010年5月6日の米国での\textbf{フラッシュクラッシュ}(Flash Crash)の一因とも指摘されており、高速取引の\textbf{社会的功罪}を巡る議論は、今後も継続されるべき重要なテーマです (p. 30-31)。

\section{普遍的な理論・コンセプト}
高速取引, 執行アルゴリズム, 人工知能, 相場操縦, 見せ玉, 法的責任, 市場の非対称性, コロッセオの奴隷, 監視AI, アルゴリズム取引, 裁定取引, マーケット・インパクト, コロケーションサービス, FPGA, 流動性の供給, 負の手数料, 情報の公平性, 価格発見機能, IEX, フラッシュクラッシュ

\subsection{理解度確認クイズ}
\begin{enumerate}
	\item 2024年の金融庁資料によると、東証の全取引の大部分を占めている、極めて短い時間で大量の売買を行う取引手法は何と呼ばれますか。
	\item プロの投資家が設定した売買の指示を、市場価格への影響を抑えながらコンピューターが自動的かつ効率的に実行するために用いられる取引プログラムの総称は何ですか。
	\item 現代の株式市場において、有利な取引価格を求めて高速取引業者と執行アルゴリズムの間で繰り広げられている、技術間の競争状態を指す筆者の比喩表現は何ですか。
	\item 買い注文や売り注文を約定させる際、個人投資家の相手方となることの多い、市場で優位な立場にある自動売買を行う主体のことを何と呼びますか。
	\item 株式市場におけるAIの役割の変化を、人間が観客席から猛獣の戦いを眺める光景に例えた場所の名前は何ですか。
	\item 筆者が、多くの個人投資家がAIに支配された市場で置かれている状況を、コロッセオの猛獣と戦う立場に例えた社会的地位は何ですか。
	\item 現代の株式市場において、高速取引や執行アルゴリズムの性能を劇的に向上させている中核技術は何ですか。
	\item 法律の整備が追いついていないとされる、AIが意図せず行ってしまった場合に、その責任の所在が問題となる、市場価格を不正に操作する行為は何ですか。
	\item 市場価格を不正に操作するために、売買する意思がないにもかかわらず大量の注文を出して他の参加者の誤解を誘う行為を何と呼びますか。
	\item AIによる不正行為が発覚した場合に、最終的にその行為の責任を負うべき対象や人の所在を指す法律上の概念は何ですか。
	\item 株式市場における不正な取引の発見と特定を担い、不正行為を行うAIと対立する役割を果たしている技術もまた何ですか。
	\item 現代の株式市場の取引環境を形作る主要な二つの自動取引形態とは、高速取引ともう一つは何ですか。
	\item プロの投資家がAIを用いた取引システムに対して提供する、どの銘柄をどれくらい売買するかといった具体的な指示を指す言葉は何ですか。
	\item AIによる不正取引が引き起こす問題に対して、現在追いついていないとされる、社会的な規範や規定を指す言葉は何ですか。
	\item 現代の株式市場において、取引の大部分を占める自動売買システムが目指している最も重要な目的の一つは何ですか。
\end{enumerate}

\subsubsection*{解答一覧}
1.高速取引, 2.執行アルゴリズム, 3.AI同士の戦場, 4.高速取引業者, 5.コロッセオ, 6.奴隷, 7.人工知能, 8.相場操縦, 9.見せ玉, 10.法的責任, 11.監視AI, 12.執行アルゴリズム, 13.売買指示, 14.法律の整備, 15.有利な価格での取引

\section{高速取引の源流:人間の戦略の電子化}

\subsection{裁定取引と自己売買部門の歴史}
高速取引(High-Frequency Trading, HFT)は現代の技術の産物であると同時に、その戦略の多くは\textbf{かつて人間の手によって行われていた}取引手法にルーツを持ちます (p. 34)。高速取引の主要な戦略には、既に触れたマーケットメイク戦略とディレクショナル戦略の他に、\textbf{裁定取引}(Arbitrage)があります (p. 34)。裁定取引は、高速取引が誕生する遙か以前、株式取引が始まった初期の時代から、特に証券会社の\textbf{自己売買部門}によって活発に行われてきました (p. 34)。

これらの証券会社が行う裁定取引やマーケットメイク戦略は、単に自己の利益を追求するだけでなく、\textbf{株式市場の機能を維持}し、取引を依頼する顧客に対して魅力的な市場環境を提供するという側面も担っていました (p. 34)。つまり、高速取引は、こうした証券会社の自己売買部門が担ってきた市場機能と利益追求を、コンピューターとアルゴリズムによって\textbf{電子化・高度化したもの}だと言うことができます (p. 35)。その後、自己資金を増やすことに特化した\textbf{新規参入の高速取引専門業者}が数多く現れることになります (p. 35)。



\subsection{高速取引がもたらした取引コストの低下}
証券会社の自己売買部門の高速取引化、そして専門業者の参入は、社会全体で見ると\textbf{コストの削減}に繋がり、\textbf{一般投資家が支払う総合的な取引コストを大きく低下}させました (p. 35)。ここでいう総合的な取引コストには、証券会社に直接支払う\textbf{売買手数料}だけでなく、自身の売買によって価格が変動してしまう\textbf{マーケット・インパクト}なども含まれます (p. 35)。したがって、高速取引の出現は、市場の効率性を高め、\textbf{一般投資家も恩恵を受けている}側面があると言えます (p. 35)。しかしその一方で、高額な装置の導入や高度な技術者が必要となる高速取引への参入障壁は高く、小規模な証券会社の中には、対応できずに\textbf{廃業に追い込まれた}ところも少なくありませんでした (p. 36)。現在、日本では人の手によるマーケットメイクや裁定取引はほぼ高速取引に置き換えられています (p. 37)。

\section{高速化競争の極限:ミリ秒以下の世界}

\subsection{コロケーションと専用回路の導入による競争優位性の追求}
高速取引において「速さ」(\textbf{レイテンシーの最小化})は決定的に重要であり、特に裁定取引のように\textbf{最も遅延が少ない業者だけが利益を総取り}できる戦略においては、わずかな遅延がそのまま\textbf{収益の差}となり、命取りになります (p. 37, 39)。ここで言う速さとは、注文を出してから\textbf{いかに早く取引所に到達するか}という点に集約されます (p. 37)。高速取引業者は、取引所の取引システムと同じデータセンターの隣にコンピューターを設置できる\textbf{コロケーションサービス}を利用し、\textbf{物理的な距離を極限まで縮めて}います (p. 39)。さらに、各業者のコンピューターと取引所システムを繋ぐケーブルの長さも、\textbf{公平を期すために全く同じ長さ}になるように調整されています (p. 39)。

しかし、遅延は物理的な距離だけでなく、\textbf{コンピューター内部の演算処理}でも発生します (p. 40)。CPUがメモリーからプログラムを読み出し、演算を行い、結果を出力する一連のプロセスには、高速取引の世界では許容できない時間がかかります。この演算遅延を最小化するため、高速取引業者は、注文に必要な計算を\textbf{専用の電気回路}としてあらかじめ作り込んでしまう\textbf{FPGA}(Field-Programmable Gate Array)という\textbf{ハードウェア技術}を導入しています (p. 40-41)。FPGAは、\textbf{同じ演算を極めて低遅延で行うこと}に特化しているため、高速取引という\textbf{速度競争の極限}においては最適なソリューションとなっています。理想的には、取引所から受け取ったデータが電気回路に流れるだけで取引所に送るデータが出力される、という\textbf{電気信号レベルでの処理速度}の追求が行われています (p. 40, 41)。

\section{高速取引の功罪と「ずるい」の定義}

\subsection{流動性供給のジレンマ:必要な時に供給されない現実}
高速取引の功績の一つは市場への\textbf{流動性の供給}にありますが、一方で\textbf{功罪の「罪」}についても目を向ける必要があります (p. 42)。高速取引は\textbf{自己の利益の最大化}のために取引を行っているため、\textbf{損をする恐れがある場合には即座に取引参加を停止します} (p. 43)。マーケットメイク戦略は価格の\textbf{安定した往来時}に利益が出る戦略であり、株価が大きく変動する\textbf{急騰や急落局面}では、在庫リスクが高まり損をする可能性が急増します (p. 43)。

このため、価格が急変動し、その後も\textbf{価格変動が激しい局面}では、マーケットメイク戦略は\textbf{撤退して様子を見る}という判断を下します (p. 43)。これは、投資家の売買ニーズが高まり\textbf{最も流動性が必要とされる市場の混乱時}に、高速取引が\textbf{市場から姿を消し、流動性を供給しなくなる}という、市場の安定性にとって非常に大きな問題を引き起こします (p. 43)。このような\textbf{肝心な局面での流動性の不足}を回避するため、一部の取引所では、高速取引業者と契約を結び、常に指値注文を入れるという義務を負わせる代わりに\textbf{負の手数料}(リベート)を与える\textbf{優遇制度}を設けている例もあります (p. 44)。



\subsection{高速取引は「ずるい」のか?情報公平性と市場の効率性}
株式取引における\textbf{「ずるい」}の定義は、\textbf{特別な立場だからこそ得られる情報}を使うことによって、\textbf{市場における情報に不公平があること}に焦点が当てられます (p. 47)。つまり、市場参加者は、\textbf{予測能力や勉強量}といった個人的な能力を競うことは公平とされますが、\textbf{得られる情報そのものに不公平性}が介在することは問題視されるのです (p. 47)。

実際に、高速取引を意図的に困難にする設計(例えば、注文をわずかに遅延させるなど)を導入した\textbf{IEX}のような取引所も存在しますが、IEXは取引シェアを大きく伸ばせていません (p. 49)。これは、「ずるい」取引が存在すると参加者が減少し、市場が提供すべき\textbf{価格発見機能}(適正な価格の形成)と\textbf{流動性の供給}ができなくなるという、市場機能の低下への懸念があるためです (p. 49)。しかし、現在のところ、高速取引は\textbf{価格発見機能を大きく邪魔しているわけではない}という評価があり、むしろ\textbf{流動性を供給しているという機能的価値}が評価されているため、\textbf{締め出すべきほど「ずるい」わけではない}というのが「\textbf{今のところの結論}」とされています (p. 49-50)。ただし、この評価は、将来的に\textbf{より巧妙で不公平な「ずるい手法」}が発明されれば、いつでも変わり得るという潜在的なリスクを抱えています (p. 50)。

\begin{thebibliography}{9}
	\bibitem{mizuta2025} 水田孝信, \textbf{高速取引 株式市場にAIがもたらすマーケット・インパクト}, 星海社新書, 2025年.
\end{thebibliography}
\end{document}
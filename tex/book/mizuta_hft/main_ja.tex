\documentclass[uplatex,a4j,12pt,dvipdfmx]{jsarticle}
\usepackage{amsmath,amsthm,amssymb,bm,color,enumitem,mathrsfs,url,epic,eepic,ascmac,ulem,here,ascmac}
\usepackage[letterpaper,top=2cm,bottom=2cm,left=3cm,right=3cm,marginparwidth=1.75cm]{geometry}
\usepackage[english]{babel}
\usepackage[dvipdfm]{graphicx}
\usepackage[hypertex]{hyperref}
\title{読書メモ:高速取引(はじめに)}
\author{Masaru Okada}
\date{\today}
\begin{document}
\maketitle
\tableofcontents

\section{はじめに}
現代の株式市場における取引のあり方は、劇的に変化している。金融庁が2024年に公開した資料によれば、東京証券取引所における全取引のうち、実に\textbf{9割が高速取引}によって占められているという (p. 3)。この事実から、個人投資家が取引を約定させた際、その相手方の売り手(または買い手)は、ほとんどの場合\textbf{高速取引業者}であると推測される (p. 3)。

\section{現代市場の構造とAIの役割}
\subsection{AI主導の市場環境}
現在の株式市場では、高速取引に対峙しているのは人間ではなく、\textbf{AI}である (p. 4)。取引の多くは、高速取引業者によるものに加え、プロの投資家が決定した売買指示をコンピューターが自動で実行する\textbf{執行アルゴリズム}によって占められている (p. 4)。これらの執行アルゴリズムのほとんどにはAIが搭載されており、高速取引業者もまたAIによってその能力を強化している (p. 4)。結果として、市場では両者のAIがより有利な価格で取引を成立させようと\textbf{熾烈な競争を繰り広げている} (p. 4, p. 5)。

\subsection{人間とAIの関係}
この状況は、株式市場が\textbf{AI同士の戦場}になっていることを意味する (p. 5)。しかし、この戦いの設計者、すなわち高速取引の基本戦略を構築し、執行アルゴリズムに対してどの銘柄をどれくらい売買するかといった指示を出しているのは、あくまで\textbf{プロの投資家である人間}である (p. 5)。このため、現代の市場は「人間 対 AI」の構図ではなく、プロの投資家がAIに取引を担わせ、他のAIと戦わせる様子を、あたかも\textbf{コロッセオの観客席から猛獣であるAI同士の戦いを眺めている}かのように俯瞰している状況と表現されている (p. 5)。

\subsection{個人投資家の立ち位置}
一方で、多くの個人投資家は、このようなAI同士の高度な戦いに、\textbf{生身の人間のまま参加}しているのが実情である (p. 5)。これは、誤って\textbf{コロッセオの戦場に紛れ込んでしまった奴隷が、猛獣であるAIと戦う羽目になってしまっている}現実に例えられている (p. 5)。

\section{AIと不正取引・法の課題}
AIは、適法かつ公正な取引に利用される一方で、\textbf{不正取引}にも悪用されている (p. 5)。具体的には、人間が意図しないにも関わらず、AIによる\textbf{見せ玉}などの不正な取引が相場操縦を引き起こす事例がある (p. 6)。このような場合、誰に責任があるのか、という点で\textbf{法律の整備が追いついていない}という深刻な課題が存在する (p. 6)。しかし、その不正取引の発見と取り締まりもまた\textbf{AIが行っている}ため、この分野においても「不正取引と取り締まり」という名の\textbf{AI同士の戦い}が繰り広げられている (p. 6)。

\section{普遍的な理論・コンセプト}
高速取引, 執行アルゴリズム, AI(人工知能), 相場操縦, 見せ玉, コロッセオの例え

\section{理解度確認クイズ}
\subsection{理解度確認クイズ}
\begin{enumerate}
	\item 東京証券取引所における全取引のうち、2024年公開の金融庁資料で約9割を占めるとされる取引手法は何ですか。
	\item プロの投資家が売買の指示を出し、コンピューターが自動で実行する取引アルゴリズムの名称を、資料で言及されているとおりに記述しなさい。
	\item 現代の株式市場において、高速取引に対峙している主体は人間ではなく何であると述べられていますか。
	\item 著者は、プロの投資家とAIの関係を、人間とAIが直接戦っているのではなく、どのような競技場になぞらえて説明していますか。
	\item 著者は、多くの個人投資家がAI主導の市場に参加している状況を、コロッセオにおいて誰と誰が戦う状況に例えて批判的に表現していますか。
	\item 市場における適法な利用だけでなく、不正取引にも悪用されている技術は何ですか。
	\item 人間が意図しないのにAIが実行し、相場操縦につながる可能性が指摘されている不正な取引の一例を、資料で挙げられている用語で記述しなさい。
	\item AIによる不正な取引で相場操縦が発生した場合、法律の整備が追いついていないとされる、責任の所在に関する問題は何ですか。
	\item 不正な取引が市場で起きている一方で、その取り締まりにおいて不正取引の発見を担っている技術は何ですか。
	\item 現代の株式市場におけるAI同士の戦いにおいて、高速取引の基本的な戦略を設計し、指示を与えているのは誰ですか。
	\item 高速取引の取引相手として、個人投資家が最も遭遇しやすいのはどのような業者ですか。
	\item 執行アルゴリズムの能力強化に不可欠な、現在ほとんどの場合に搭載されている技術は何ですか。
	\item AIの取引能力向上により、市場における価格競争でより有利な価格での取引を目指しているのはどの主体ですか。
	\item 現代の市場をAI同士が戦っている場であると表現する一方で、その戦いを上から眺め、時々指示を出しているのはどのような立場の人物ですか。
	\item 不正取引とその取り締まりの間で繰り広げられている、AI同士の戦いの対象は何ですか。
\end{enumerate}

\subsubsection*{解答一覧}
1. 高速取引, 2. 執行アルゴリズム, 3. AI(人工知能), 4. コロッセオ, 5. 奴隷と猛獣, 6. AI, 7. 見せ玉, 8. 誰に責任があるのか, 9. AI, 10. 人間(プロの投資家), 11. 高速取引業者, 12. AI, 13. 両AI(高速取引業者と執行アルゴリズム), 14. プロの投資家, 15. 不正取引

\section{高速取引と執行アルゴリズムの戦い}

\subsection{高速取引の驚異的な速度}
現代の金融市場における\textbf{高速取引}の速度は、人間の認識能力を遥かに凌駕している。例えば、東証の注文応答速度は\textbf{200マイクロ秒}であり、これは人間が光を視覚で捉えて脳で認識するまでにかかる時間(0.1秒)の500分の1である (p. 16)。人間が0.1秒前を見ている間に、高速取引は\textbf{500回}もの取引を完了できる (p. 16)。高性能なパソコンのディスプレイでも、表示更新に約5000マイクロ秒かかるため、高速取引が25回注文を出す間に生じた価格変動は、そもそも人には見えていないことになる (p. 17)。したがって、個人投資家が高速取引の怪しい注文の出し入れを見たという証言は、技術的にありえない (p. 17)。米国の高速取引業者である\textbf{Virtu Financial}は上場企業として情報公開しており、過去1238日のうち\textbf{負けたのは1日だけ}という驚異的な実績を公開し、大きな注目を集めた (p. 17)。



\subsection{執行アルゴリズムの役割とAIの活用}
高速取引以外にも、証券取引所ではコンピューターによる自動売買が頻繁に行われており、その一つが\textbf{執行アルゴリズム}である (p. 19)。これは、人間が考案した大規模な取引を、他の市場参加者、特に高速取引業者に\textbf{バレないように少しずつ}市場に放出・執行するためのプログラムである (p. 19)。大量の買い注文を一度に出すと、高速取引に察知され、より高い値段で売り付けられる\textbf{(ぼったくられる)}可能性があるため、資産運用会社は、性能の良い執行アルゴリズムを持つ銀行や証券会社を取引相手として選ぶ (p. 19)。このため、銀行や証券会社はAIを用いて執行アルゴリズムの性能向上に努め、一方で高速取引業者もAIを活用して、執行アルゴリズムの取引パターンを分析し、利益を奪い取るための対策を強化している (p. 19-20)。



\subsection{AI同士の熾烈な戦い:マーケットメイク戦略対執行アルゴリズム}
現代の株式・為替市場では、特に高速取引業者の\textbf{マーケットメイク戦略}と執行アルゴリズムとの戦いが熾烈である (p. 21)。マーケットメイク戦略とは、特定の株式価格を挟む形で、買い注文と売り注文を両方出すことで、価格が行ったり来たりする間のわずかな差益を狙う戦略である (p. 21)。この戦略において重要な要素は、他の業者よりも速く注文を入れる\textbf{速度}と、\textbf{短期的な価格変動を予測する能力}である (p. 21)。

高速取引業者は、取引所に集まった全ての注文の状況(\textbf{注文板})のデータをAIに学習させ、価格の上がりやすさ、下がりやすさを分析し、その予測を取引に利用している (p. 22)。このAI分析の弱点として、取引するつもりのない大きな注文である\textbf{見せ玉}に騙されやすい点が挙げられる (p. 22)。また、単純な買い注文パターンをAIが検知すると、マーケットメイク戦略側は注文価格を上げて高く売りつけようとする (p. 22)。

これに対抗するため、執行アルゴリズム側も、手の内がバレないように注文板の状況をAIに分析させ、その結果に従って少しずつ注文を出し、高速取引業者から利益を奪われないようにしている (p. 22-23)。このように、注文板の状況分析と短期予測においては、\textbf{AI同士がしのぎを削っている}状態にある (p. 23)。



\subsection{AIと人間の得意分野}
AIはきわめて短期間での株式取引執行においては、人間を遥かに超える能力を発揮するが (p. 23)、企業の本質的価値を測定したり、長期的なストーリーや未来の展望を描くことは苦手である (p. 23)。AIの仕組みが人間の脳の仕組みから離れるにつれて、AIと人間の得意分野は異なってきている (p. 23)。



\subsection{高速取引業者の二面性:「生かさず殺さず」の関係}
執行アルゴリズムを用いるプロの投資家にとって、取引相手となる高速取引業者は、利益を奪い合う敵であるように見える (p. 24)。しかし、高速取引業者がいなくなると、執行アルゴリズムは取引したい量を取引できなくなり、市場の\textbf{流動性}が失われてしまう (p. 24)。高速取引は取引したいときに取引させてくれる\textbf{流動性の供給者}としての側面も持っている (p. 24)。したがって、プロの投資家にとって高速取引は、「いなくなると困るけれども、ぼったくられたくない」という、まさに\textbf{「生かさず殺さず」}がちょうどよい関係性にある (p. 24)。

この関係は、\textbf{古本屋とその客}の関係に例えられる (p. 25)。古本屋が本の中身を考慮せず、他の古本屋の価格を参考に安く買い、少し高く売ることで利益を得る様子は、高速取引のマーケットメイク戦略に酷似している (p. 25)。古本屋が本の価値を考えずとも、読みたい客に本を提供し、読み終わった客から本を買い取るありがたい存在であるのと同様に、高速取引も企業の価値を考えずに取引していても、企業を売り買いしたい投資家にとっては\textbf{ありがたい存在}なのである (p. 25)。



\subsection{ディレクショナル戦略と市場への影響}
高速取引には、短期的な株価の上昇・下落を予想し、その方向にかける\textbf{ディレクショナル戦略}も使われる (p. 26)。これは、注文板の状況や、決算発表などのニュースが出た瞬間にAIで分析・予想することで実行される (p. 26)。特にニュースの分野では、AIが記事を書いて、AIが記事を読んで、AIが取引をするという、人間が介在しない取引が発生している (p. 26)。

ディレクショナル戦略は市場を混乱させる可能性があり、例えば日本銀行の金融政策の変更発表時に、コンピューターがPDFファイルを自動で解析し、その変更内容から円売りを行う戦略が流行し、市場が乱高下した事例がある (p. 26-27)。また、ディレクショナル戦略には、価格変動に拍車をかける\textbf{順張り}の取引が多いと指摘されており、市場を荒らす要因となり得る (p. 28)。



\subsection{技術の是非とルール整備}
高速取引やAIは正義か悪かという論争はしばしば起きるが、これは自動車が生まれた時代の馬車との安全性の議論と同様、新しい技術が生まれた際に必ず起きる議論である (p. 28)。重要なのは、高速取引や生成AIそのものの是非ではなく、自動車を走らせるためのルール整備が行われたように、\textbf{どのようなルールが適切なのかを議論する}ことである (p. 28-29)。

マーケットメイク戦略は流動性を供給し、価格の安定化に寄与するが (p. 28)、実は\textbf{市場が荒れてくると}、高速取引業者は\textbf{取引をやめていなくなる}ことが知られている (p. 30)。高速取引が最も儲けやすいのは毎日同じことが繰り返されている時であり、損をするくらいなら取引しない方がマシだと判断するため、流動性が最も必要とされる時に供給を停止してしまう (p. 30-31)。2010年5月6日に米国で起きた\textbf{フラッシュクラッシュ}の一因は高速取引業者だった可能性も指摘されている (p. 31)。

\section{普遍的な理論・コンセプト}
高速取引, 執行アルゴリズム, マイクロ秒, 流動性, マーケットメイク戦略, 注文板, 見せ玉, ディレクショナル戦略, 順張り, フラッシュクラッシュ, AIと人間の得意分野の分化, 技術の是非とルール整備

\section{理解度確認クイズ}
\subsection{理解度確認クイズ}
\begin{enumerate}
	\item 株式取引において、人間が考えた大量の注文を、他の市場参加者に察知されないように細かく分けて執行する自動売買プログラムを何と呼びますか。
	\item 東京証券取引所の注文応答速度に見られるように、高速取引がその優位性を確保している極めて短い時間の単位を、1秒の百万分の一で示す単位は何ですか。
	\item 米国の高速取引業者が投資家向け資料で「1238日のうち負けたのは1日だけ」と公表し、その驚異的な収益性が注目された企業名を答えなさい。
	\item ある銘柄の買い注文と売り注文の両方を同時に提示し、価格の小さな変動を利用して差益を得る、高速取引業者の主要な戦略の一つは何と呼ばれますか。
	\item 大量の注文を一度に市場に出すことで、高速取引業者に察知されて不利な価格での取引を強いられるリスクを、この文章ではどのような行為と表現していますか。
	\item 高速取引業者が短期的な価格変動を予測するために最も頻繁にAIに分析させている、取引所に集まった全ての注文の状況を示す情報を何と呼びますか。
	\item 市場の参加者が、取引するつもりのない非常に大きな注文を一時的に提示し、注文板の分析に頼るAIを欺こうとする行為を指す用語は何ですか。
	\item 株式や為替市場において、取引したいときにいつでも取引できる状態、すなわち取引のしやすさを示す経済学的な概念は何ですか。
	\item 高速取引業者から見て、市場が荒れていない平常時に最も利益を上げやすい状態を、この文章ではどのように表現していますか。
	\item 短期的な株価の上昇または下落の方向性を予測し、その方向に賭けることで利益を狙う、マーケットメイクとは異なる高速取引の戦略は何ですか。
	\item AIが注文板の状況分析や短期予測に優れる一方で、人間が得意とする、企業の将来性や収益力など、企業そのものの価値を評価することを何と表現していますか。
	\item 価格が変動している方向にさらに追随して取引を行うことで、価格変動に拍車をかけるような取引手法を何と呼びますか。
	\item 高速取引と執行アルゴリズムの関係性が、お互いに利益を奪い合うが、一方がいなくなると困るという、絶妙なバランスを保っている状態を表現する慣用句は何ですか。
	\item 2010年5月6日に米国で発生し、高速取引が一因と指摘されることもある、短時間で株価が暴落した後、急速に回復した現象は何と呼ばれていますか。
	\item 高速取引が企業価値を考えずに株価だけを見て取引しているにもかかわらず、他の投資家にとってありがたい存在である理由を、この文章では何に例えていますか。
\end{enumerate}

\subsubsection*{解答一覧}
1. 執行アルゴリズム, 2. マイクロ秒, 3. Virtu Financial, 4. マーケットメイク戦略, 5. ぼったくられる, 6. 注文板, 7. 見せ玉, 8. 流動性, 9. 毎日同じことが繰り返されているとき, 10. ディレクショナル戦略, 11. 企業の本質的価値, 12. 順張り, 13. 生かさず殺さず, 14. フラッシュクラッシュ, 15. 古本屋とその客の関係

\begin{thebibliography}{9}
	\bibitem{mizuta2025} 水田孝信, \textbf{高速取引 株式市場にAIがもたらすマーケット・インパクト}, 星海社新書, 2025年.
\end{thebibliography}

\end{document}
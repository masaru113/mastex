\documentclass[uplatex,a4j,12pt,dvipdfmx]{jsarticle}
\usepackage{amsmath,amsthm,amssymb,bm,color,enumitem,mathrsfs,url,epic,eepic,ascmac,ulem,here,ascmac}
\usepackage[letterpaper,top=2cm,bottom=2cm,left=3cm,right=3cm,marginparwidth=1.75cm]{geometry}
\usepackage[english]{babel}
\usepackage[dvipdfm]{graphicx}
\usepackage[hypertex]{hyperref}

\title{洗脳護身術(読書メモ)}
\author{岡田 大 (Okada Masaru)}
\date{\today}

\begin{document}

\maketitle


\section{教育と洗脳の境界}

教育の本来の目的は、第三者の利益ではなく、\textbf{本人の利益}のためにあるべきであり、この観点からは教育は洗脳とは異なる(p. 11)。しかし、国家レベルを超えた視点で見ると、その国の教育が本当に本人の利益になっているか、あるいは第三者の利益のための「\textbf{洗脳}」になっていないか、注意深く見極める必要がある。例えば、現時点での北朝鮮の教育などが、世界的な視点からは洗脳にあたる可能性があるとされる(p. 11)。

\section{「正しいこと」がもたらす危険性}

問題の本質は、特定国の主義(例:イラクのフセイン政権やアメリカ民主主義)の正誤ではなく、\textbf{国民の圧倒的多数が特定の「正しいこと」を強く信じ込む}状況にある(p. 12)。北朝鮮の専制君主軍事国家主義よりもアメリカ民主主義が客観的に正しいとしても、アメリカ人のほぼ全員がその民主主義を強く信じ込んでいることが問題視される(p. 12)。その「正しいこと」が正しければ正しいほど、そしてそれを強く信じる人口の割合が高ければ高いほど、\textbf{「誤ったこと」を徹底的に排除}しようとする衝動が強まるからだ(p. 13)。結果として、戦争のような絶対に避けるべき行為にさえ言い訳が付けられ、正当化されて堂々と行われてしまう(p. 13)。過去の広島・長崎への原爆投下も、「誤った帝国日本の暴走を止める」という大義名分の下に正当化された可能性がある(p. 13)。

\subsection*{プリンシプルの解釈権の集中}

「正しい」思想や主義、宗教であっても、国民の圧倒的大多数が信じることが危険なのは、その\textbf{プリンシプル(基本原則)}を具体的な事象に適用する際の\textbf{解釈権}が、一つの政治権力に集中してしまうためである(p. 14)。かつてキリスト教における聖書の解釈権がローマ法王に独占されていたのと同様に、現代のアメリカではアメリカ民主主義の解釈権を大統領が独占している構図となっている(p. 14)。イラク戦争は、大統領が聖書の解釈権までも持っているかのように感じさせた一例である(p. 14)。本来、平和的な宗教であるはずのイスラム教やキリスト教がテロや戦争を引き起こすのは、十字軍の時代から、プリンシプルの解釈権が特定の政治権力に集中しすぎた結果である(p. 14)。

\section{戦争を避けるための多様性}

戦争を起こす、あるいは参戦の判断をするのは、軍隊を持つ単位、すなわち国家である。したがって、国家が参戦の判断をしなければ戦争は起きない(p. 18)。参戦の判断を避けるためには、\textbf{国民のほとんどが賛成する状況を避ける}、すなわち、国民の全員が同じ判断基準を持たず、\textbf{多様な価値を信じる}ことが解決策となる(p. 18)。例えば、アメリカの人口の10\%がイスラム教徒であったならば、イラク戦争は起きなかっただろうと推測される(実際は0.3\%)。アメリカのキリスト教徒の比率(85\%)は、一つの宗教に固まりすぎている状態と指摘される(p. 18)。

\section{洗脳技術と自己脱洗脳}

筆者の社会的責任として、\textbf{日本人全員に自己脱洗脳の技術を伝える}ことで、国民全員が単一の価値に縛られないようにし、結果として戦争を避けることに繋げたいという意図が示されている(p. 18)。

\subsection*{国会議員の世襲とブランド志向}

日本の国会議員の約4分の1が世襲議員であり、一般家庭の子息と比べて、世襲議員候補が選挙で実に\textbf{1万倍も有利}な状況にあるという分析がなされている(p. 22)。この状況は、人口比率でいえばアメリカのイスラム教徒(0.3\%未満)が米国議会の4分の1を占めるのに等しい、極めて不均衡な状態である(p. 22)。この選挙における有利さは、国民の\textbf{被洗脳的ブランド志向}の表れであると論じられている(p. 22)。

\subsection*{「あの世の論理」と「この世の論理」}

完全に煩悩から自由となった「あの世の論理」の人は、この世の社会的判断を超越しているため、善悪という言葉自体に意味を持たない(p. 24)。自己脱洗脳によって「この世の論理」から自由になったとしても、それと\textbf{善人であるかどうかは関係ない}という皮肉な見方が示される(p. 24)。そもそも、善とされている事柄自体が\textbf{社会的判断}であり、その時々の権力者側から与えられたものである以上、ある事柄を善と思うこと自体が洗脳されている結果であると考えるべきである(p. 24)。

\subsection*{洗脳の定義と脱洗脳技術}

\textbf{洗脳}とは、「\textbf{第三者の利益のため}、認知レベルを含む脳内情報処理に、何らかの介入的な操作を加えることで、その人の思考、行動、感情を思うままに制御する」ことである(p. 28)。カルトの信者に「騙されていた」という情報を与えるだけでは脱洗脳は不可能であり、操作された情報で書き込まれた脳を、\textbf{別の情報で書き戻す技術}が必要となる(p. 25)。本書が提供するのは、まさにこの\textbf{自己脱洗脳の技法}である(p. 25)。また、「地球の平和」や「宇宙の未来」といった曖昧で小綺麗な教義を掲げるカルトほど危険であるという指摘がなされ、著名人の活動についても疑問が呈されている(p. 29)。

% \section{普遍的な理論・コンセプト}
% 洗脳, 自己脱洗脳, プリンシプル, 解釈権の集中, 多様な価値, 被洗脳的ブランド志向, あの世の論理, この世の論理

\subsection{理解度確認クイズ}
\begin{enumerate}
	\item 脳内の情報処理に介入的な操作を加え、本人の利益ではなく、\textbf{第三者の利益}のために思考・行動・感情を制御する行為を何と呼びますか。
	\item 圧倒的多数の国民が特定の主義や思想を強く信じることで、その思想を具体的な事象に適用する際の\textbf{判断基準の決定権}が単一の政治権力に集中する現象を、本メモでは何と表現していますか。
	\item 「正しいこと」が社会の主流となったときに、その「正しいこと」と異なる見解や行動を社会から\textbf{徹底的に排除}しようとする危険な衝動は何によって引き起こされますか。
	\item ある思想や宗教の\textbf{根本的な原則}を指し、これを具体的な状況に当てはめる際の解釈権が政治権力に独占されることが危険だと論じられている概念は何ですか。
	\item 聖書などにおける解釈の権威をローマ法王が独占していた状況に例えられ、現代のアメリカ大統領が持つとされる、アメリカ民主主義の解釈を独占する権限を指す言葉は何ですか。
	\item 国民の大多数が単一の判断基準に縛られず、様々な考え方や信条を持つことで、国家が戦争参戦のような極端な判断を下すことを防ぐ効果が期待される態度は何ですか。
	\item カルトの信者に対して、騙されていたという事実を伝えるだけでは不十分であり、操作された脳内情報を\textbf{別の情報で上書き}するために必要とされる技術は何ですか。
	\item 煩悩から完全に自由になり、この世の社会的判断(善悪)を超越した状態にいる人の思考様式を、本メモでは「\_\_\_\_の論理」と表現しています。空欄を埋めてください。
	\item 善悪の判断など、特定の事柄を「正しい」と思うこと自体が、その時々の権力者側から与えられたものであり、操作された結果であるという考え方における、判断の根拠となる概念は何ですか。
	\item 日本の国会議員選挙において、一般家庭出身者と比較して世襲候補が極端に有利になる状況を生み出している、国民が持つとされる心理的傾向は何ですか。
	\item 「地球の平和」や「宇宙の未来」など、聞く人に心地よく響くが\textbf{曖昧で具体性に欠ける教義}を掲げる集団が危険であるとされるのはなぜですか。
	\item 宗教が本来の平和的な性質から離れて、テロや戦争を引き起こす原因として、十字軍の時代から指摘されている、ある権力の\textbf{過度な集中}は何ですか。
	\item 教育の本質は、誰の利益のためにあるべきであると、洗脳との対比において述べられていますか。
	\item 完全に「この世の論理」から自由になることと、社会的に「善人」であることの間には、本メモによるとどのような関係性がありますか。
	\item 洗脳を定義する上で、その行為の目的が「\textbf{本人の利益ではない}」ことを明確にするために挙げられている、利益を受ける対象は何ですか。
\end{enumerate}

\subsubsection*{解答一覧}
1. 洗脳, 2. 解釈権の集中, 3. 「正しいこと」を強く信じること, 4. プリンシプル, 5. 解釈権, 6. 多様な価値, 7. 自己脱洗脳の技法, 8. あの世, 9. 社会的判断, 10. 被洗脳的ブランド志向, 11. 曖昧で小綺麗な教義はカルトがよく掲げるため, 12. プリンシプルの具象への適応解釈権の集中, 13. 本人の利益, 14. 関係ない, 15. 第三者の利益

\begin{thebibliography}{99}
	\bibitem{toma} 苫米地英人, 『洗脳護身術』, 三才ブックス, 2003年10月.
\end{thebibliography}

\end{document}